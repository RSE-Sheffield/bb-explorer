\setcounter{page}{243}
\chapter{ROMANS}
\section{MANSFIELD-PARK. \large{( Troisième extrait. Voy. p. 101 de ce vol. )}}
Le départ de Henri, deux jours après le retour de Sir Thomas, détruit les espérances dont Marie s'étoit follement bercée. Julia se console par la certitude que Marie ne sera pas plus heureuse qu'elle, et la cessation de cette cause de rivalité rapproche les deux sœurs. La présence de Sir Thomas change beaucoup la manière de vivre de sa famille. Il n'accorde pas assez aux goûts et aux habitudes de la jeunesse. Les relations de voisinage sont presqu'entièrement supprimées. Cette privation est vivement sentie par les demoiselles Bertram. Sir Thomas s'aperçoit bientôt que Rushworth est un homme extrêmement borné, et que Marie n'a aucune considération pour lui. Il juge de son devoir d'interroger sa fille, déterminé qu'il est à rompre le mariage, si cet entretien confirme ses craintes. Marie hésite un moment, mais elle se décide à persister.\setcounter{page}{244} L'idée que Crawford pourroit se croire la cause de cette rupture, humilie son orgueil, et un desir impatient d'indépendance l'emporte sur tout autre sentiment. Elle dissimule si bien les vrais motifs de sa détermination, que son père se persuade, qu'à tout prendre, ce mariage convient à sa fille.
Marie quitte sans regret la maison paternelle, et Julia l'accompagne aux eaux de Brighton.
Fanny acquiert de l'importance et de l'intérêt dans la maison par l'absence de ses cousines. Flora lui fait beaucoup de prévenances et s'insinue dans son amitié, en dépit de son jugement, qui condamne souvent les opinions de Flora. Henri revient passer quelques jours à Mansfield. Edmond et Fanny se trouvent chez mistriss Grant au moment de son arrivée ).
Fanny se représentoit que Mr. Crawford seroit fort occupé du souvenir de ses cousines, mais il parut aussi gai qu'à l'ordinaire, et s'informa d'elles avec le ton de politesse générale qu'on a pour de simples connoissances. Cependant, lorsqu'Edmond et Mr. Grant furent engagés dans une conversation particulière, et tandis que Mistriss Grant faisoit le thé, il s'approcha de Flora, et avec un regard significatif qui déplut à Fanny,\setcounter{page}{245} il lui dit : "J'apprends que Rushworth et sa charmante épouse sont à Brighton : il faut convenir qu'il est heureux, cet homme là. - Et Julia est avec eux?'
"Oui, a répondi sa sœur; il y a environ quinze jours qu'ils sont partis.''
"J'imagine que Mr. Yates les suivra de près.''
"Oh! nous n'avons pas entendu parler de Mr. Yates. Je ne pense pas qu'il joue un grand rôle dans les lettres de ces dames. Qu'en dites-vous, mistriss Price? Croyez-vous que Julia aime à le rappeler au souvenir de son père?''
"Pauvre Rushworth!'' interrompit Henri. "Il a perdu l'occasion de dire ses quarante tirades.- Je doute que sa belle Marie soit jamais curieuse de les entendre. -Quel dommage qu'elle soit si mal associée! elle lui est beaucoup trop supérieure.'' Ensuite il s'adressa à Fanny avec une attention gracieuse et lui dit: "J'espère que Rushworth aura su apprécier comme il le devoit votre extrême bonté pour lui. Je ne puis vous dire combien j'admirois la complaisance avec laquelle vous l'aidiez à apprendre son rôle. Si vous n'avez pas réussi à lui communiquer un peu du superflu de votre esprit pour suppléer à ce que la nature lui a refusé, ce n'est pas manquer\setcounter{page}{246} que de bienveillance de votre part. "
Fanny rougit et ne répondit rien.
"Tout cela a passé comme un songe agréable," ajouta Henri. "Mais je me rappellerai toujours nos essais dramatiques avec un plaisir infini. Il y avoit tant d’intérêt ! nous étions si animés, si gais ! Chaque heure du jour amenoit quelque occupation, quelque espérance, avec ce léger mélange de difficulté, qui rend tout plus piquant. — Je n’ai jamais été plus heureux. "
"Jamais plus heureux ! " dit Fanny en elle-même ; "heureux ! en agissant contre votre conscience, en poursuivant un plaisir passager, aux dépens du bonheur d’autrui. Oh ! quelle perversité ! "
"Nous avons été chanceux, continua-t-il, en baissant la voix pour n’être pas entendu d’Edmond, et sans se douter de l’impression qu’il faisoit sur Fanny. "Une semaine de plus, et tous nos projets s’accomplissoient. J’aurois été bien loin de vouloir du mal à Sir Thomas, mais si nous avions disposé à Mansfield des vents d’équinoxe, nous aurions pu très-innocemment retarder un peu la course de son vaisseau, et prolonger d’autant le calme dont nous jouissions.
Henri s’arrêta alors comme pour attendre\setcounter{page}{247} une réponse. Fanny aurait voulu s'en dispenser, mais elle prit courage et lui dit, sans oser pourtant le regarder en face. "Quant à moi, monsieur, je pense qu'il était temps que mon oncle arrivât; et puisqu'il devait en avoir du chagrin, il est fort heureux que les choses ne fussent pas plus avancées. "
Après avoir exprimé son opinion d'une manière aussi tranchée, Fanny rougit de sa propre hardiesse. Henri en fut surpris; il la regarda d'un air de réflexion, et comme s'il eût reconnu qu'il avait parlé légèrement, il reprit avec gravité. " Je crois que vous dites vrai, miss Price. Tout cela était charmant, mais nous commencions à oublier un peu les leçons de la prudence. "
Alors Henri changea la conversation et prit un ton de galanterie, auquel Fanny donna peu d'encouragement.
Miss Crawford, que son esprit d'observations n'abandonnait jamais, n'avait point perdu de vue Edmond pendant son entretien avec le Dr. Grant.
" Ces messieurs, dit-elle, me paraissent engagés dans une discussion bien intéressante. "
" La plus intéressante possible, " répondit Crawford, " car il s'agit des moyens de s'enrichir. Le docteur donne d'excellentes instructions\setcounter{page}{248} à ce jeune homme sur la régie de son futur bénéfice; j'ai entendu qu'il en étoit déjà question à table. Le moment n'est pas éloigné où il devoit entrer en fonctions, et je suis charmé d'apprendre qu'il fait une bonne affaire. Un revenu de 700 liv. sterl.! Pour un cadet de famille, cela ne va pas mal; et puis fort peu de peine. J'imagine qu'il s'en tirera avec trois ou quatre sermons par année dans les grandes solennités."
"C'est une chose curieuse," dit Flora en riant, "de voir comme les gens qui ont leur fortune faite sont modérés dans l'estimation des besoins d'autrui. Je crois qu'il faudroit retrancher beaucoup de vos fantaisies, si vous étiez réduit à 700 livres de rente."
"Cela est vrai, mais tout est comparatif: les habitudes font une grande différence; et je dis que pour un second fils, même de Baronet, c'est une fortune très-honnête."
Flora dissimula l'intérêt qu'elle prenoit à cette question, sous un air d'indifférence, et bientôt la conversation devint générale.
"Mon cher Edmond," dit Henri, "comptez que je viendrai tout exprès entendre votre premier sermon. Je me fais un devoir d'encourager les talens naissans. Quand est-ce que vous nous mettrez à même d'en juger? Voilà miss Price qui se joindra à moi; j'en\setcounter{page}{249} suis sûr, pour applaudir à vos succès. Personne n'aura moins de distractions qu'elle ; et puis, nous prendrons des notes pour consigner les plus beaux passages dans nos tablettes. Ne manquez donc pas de m'avertir lorsque vous précherez.
"Je m'en garderai bien," dit Edmond. "Vous me déconcerteriez beaucoup, et qui pis est, je saurois que vous en avez l'intention, ce qui me seroit plus pénible de votre part que de tout autre."
Si cette manière de répondre à une moquerie ne désarme pas la malice, pensoit Fanny, il faut qu'elle soit bien invétérée.
Flora ne pardonnoit point à Edmond sa persévérance dans une carrière dont elle avoit cherché à le dégoûter par tous les moyens en son pouvoir. Elle lui avoit même laissé comprendre qu'elle ne se décideroit jamais à épouser un ecclésiastique, et son amour-propre blessé combattoit fortement le penchant qui l'entraînoit vers Edmond. Elle résolut de jouer l'indifférence, et de ne pas sacrifier les calculs de la raison à un sentiment romanesque.
Le lendemain, Henri annonça l'intention de passer une quinzaine de jours à Mansfield, et de faire venir ses chiens pour chasser. En cachetant un billet, par lequel il\setcounter{page}{250} donnoit cette commission à ses gens, il se retourna vers Flora, et voyant qu'elle étoit seule dans le salon, il lui dit: "J'ai aussi une autre espèce de chasse en vue. Devinez un peu à quoi je compte employer mes loisirs, car je suis trop paresseux pour courir tous les jours avec ma meute. "
"Mais en vérité je ne sais pas trop. Vous ne pourriez rien faire de mieux, mon cher frère, que de vous promener beaucoup avec moi, tantôt à pied, tantôt à cheval."
"Assurément, ma chère sœur, ce projet là m'est fort agréable; mais ce n'est pas ce que j'entends. Il faut un nouvel aliment à l'activité de mon esprit. Mes facultés s'engourdissent dans le repos. Un peu d'amour me réveilleroit, et j'ai jeté les yeux sur Fanny Price. "
"Bon! quelle folie! ne vous suffit-il pas d'avoir tourné la tête à ses deux cousines?"
"Non; celle-ci a beaucoup plus d'attrait pour moi. Vous n'avez point d'yeux vous autres femmes pour ces sortes de choses. Vous avez vu Fanny tous les jours, et vous ne vous êtes pas aperçu de l'étonnant développement de graces et de beauté qui s'est fait chez elle depuis six semaines. C'est inouï ce qu'elle a gagné. A peine l'avois-je regardée, moi; mais à présent c'est une charmante figure,\setcounter{page}{251} gare, pleine d'expression; une tournure élégante, les mouvemens les plus agréables; un ensemble délicieux. Elle a grandi de deux pouces au moins depuis le mois d'octobre. "C'est qu'alors vous la voyiez à côté de ses cousines, qui sont de grandes femmes; et puis elle se met avec plus de goût. Voilà, je gage, tout le secret de ce merveilleux changement. Moi, je l'ai toujours trouvée assez jolie. Ses yeux sont d'un brun un peu trop clair, mais elle a un sourire fin; elle gagne à être vue souvent. Vous pensiez à autre chose cette automne. A présent que vous n'avez personne qui mette en jeu votre imagination, vous vous avisez de découvrir dans Fanny mille perfections qui n'existent que dans votre tête."
Henri ne répondit à cette accusation qu'en souriant, mais un moment après, il dit à sa sœur: "Je ne sais point encore comment je m'y prendrai avec miss Fanny. Dites-moi un peu quel est son caractère. Hier, il m'a été impossible de la faire parler. Jamais je ne me suis donné autant de peine pour plaire avec si peu d'effet. J'en suis vraiment piqué. Ses regards semblent me dire: "je ne veux pas vous aimer.— Nous verrons si elle soutiendra la gageure."\setcounter{page}{252} "Que les hommes sont enfants!" dit Flora. "Il leur faut des jouets toute leur vie. Prenez garde cependant, que cela ne devienne pas sérieux. Un léger essai de coquetterie pourra lui faire du bien, à la bonne heure; mais je vous avertis que je ne veux pas que vous la rendiez malheureuse, car c'est la meilleure petite créature qui existe, et qui a tout plein de sensibilité." "Ce qui peut vous tranquilliser," répondit Henri, "c'est que je n'ai que quinze jours à lui donner. Vous voyez bien que cela ne suffit pas pour la faire mourir de chagrin, ou autrement, elle auroit un cœur si tendre que rien ne pourroit le sauver. Non, croyez-moi, je n'ai point de mauvais desseins. Je veux seulement qu'elle me regarde avec complaisance, qu'elle me sourie tout en rougissant, qu'elle ménage pour moi une place à ses côtés, et que je la voie tressaillir de joie, quand je m'approcherai d'elle. Je veux qu'elle sympathise à mes plaisirs et à mes peines, qu'elle s'efforce de me retenir à Mansfield, et qu'à mon départ, elle croie qu'il n'y a plus de bonheur pour elle. — Voilà simplement ce que je demande. "Admirable modération," dit Flora. "Allons, je vois que je puis l'abandonner à son sort sans scrupule."\setcounter{page}{253} Fanny avait heureusement au fond du cœur un préservatif plus efficace que les recommandations de miss Crawford. Sans cela, il n’est pas bien sûr que son jugement eût été à l’épreuve de la séduction d’un homme qui avait mille moyens de plaire, et beaucoup de discernement pour les employer à propos.
William Price, le frère aîné de Fanny, qui a été placé dans la marine, obtient un congé après sept ans de service pour venir voir ses parens. Il écrit à sa sœur en lui annonçant sa visite, et suit de près sa lettre. Fanny acquiert un nouveau mérite aux yeux de Henri par son amitié pour William. Celuici est un jeune homme d’un caractère énergique et loyal, qui inspire beaucoup d’estime et une sorte de respect à Henri. Il envie cette fraîcheur de sentiment, cette vigueur d’ame, que les habitudes et les maximes du monde affoiblissent ou détruisent. Sir Thomas voit avec plaisir l’impression que sa nièce paroît faire sur Mr. Crawford. Il imagine de donner un bal à la jeunesse du voisinage pour introduire Fanny dans la société et lui donner l’usage du monde, qu’elle n’a pu acquérir jusqu’alors. Quoique cette circonstance donne lieu à quelques développemens\setcounter{page}{254} qui ne sont pas étrangers à l'intérêt, nous supprimons les détails de la fête, qui ressemblent assez à ceux qu'on trouve dans tous les romans d'aujourd'hui. Le lendemain du bal, William et Henri partent ensemble pour Londres; ce dernier fait un mystère à sa sœur du but de cette course. Edmond va passer une semaine chez un de ses amis. Fanny éprouve un grand vide en son absence, mais aussi une espèce de soulagement, en voyant qu'Edmond n'est pas complètement sous le charme d'une femme qu'elle ne croit pas faite pour le rendre heureux. Flora, en revanche, commence à craindre qu'il ne lui échappe, et regrette de ne l'avoir pas assez ménagé.- Henri revient au bout de peu de jours).
Le lendemain de son retour au presbytère, Henri alla faire visite chez lady Bartram. Il avait dit à sa sœur qu'il n'y resterait que dix minutes, mais au bout d'une heure il n'étoit point encore revenu. Flora, qui l'attendoit dans le jardin, ne comprenoit rien à ce retard. Lorsqu'enfin elle le vit arriver, elle courut à sa rencontre et fut fort étonnée en apprenant qu'il avoit passé tout ce temps avec lady Bartram et Fanny.
Il avoit l'air absorbé, et au lieu de répondre\setcounter{page}{255} aux questions de Flora, il la prit par le bras et continua à marcher avec elle sans rien dire. Après quelques momens de silence, il s'écria: "jamais je n'avois trouvé Fanny si charmante! .... Seriez-vous bien surprise, ma sœur, si je vous disois que je suis décidé à l'épouser? " Flora eut en effet besoin que son frère lui répétât bien positivement que telle étoit sa résolution, pour y croire tout-à-fait. Mais cette idée, quoique nouvelle, lui fit une impression agréable. Sous plusieurs points de vue, elle pouvoit favoriser ses propres desseins. "Oui, ma sœur," ajouta Henri, "je vous avoue que je suis pris au piège que je voulois tendre à cette aimable Fanny: Elle a fait de moi un autre homme. Je n'ose encore espérer d'avoir fait de grands progrès dans son cœur; mais le mien est fixé pour la vie. " "Il faut convenir," dit Flora, "que cette petite fille est née sous une heureuse étoile. Miss Price ne devoit assurément pas s'attendre à un mariage aussi avantageux; mais cela n'empêche pas que votre choix ne soit fort bon, et je vous assure que je crois à votre bonheur presqu'autant que je le désire. Vous aurez là un petit ange de douceur, toute reconnaissance\setcounter{page}{256} et dévouement: cela vous va en ne peut pas mieux.— Quelle joie dans sa famille! Je me représente les exclamations de mistriss Norris.— Mais dites-moi donc tout; il me reste mille choses à savoir; cela me paroît si curieux.— Quand avez-vous commencé à y penser sérieusement?
Il aurait été difficile à Henri de répondre à cette question, quoiqu'il lui fût agréable de rechercher la trace de ses premières impressions, mais Flora ne lui en laissa pas le temps et l'interrompit au milieu de ses souvenirs en lui disant. "Je gage que je sais à présent pourquoi vous avez été à Londres. Vous vouliez consulter l'amiral."
"Non, non, ma sœur. Vous n'avez pas deviné. Mon oncle est la personne du monde que je consulterois le moins sur une pareille affaire. Il déteste le mariage et ne suppose pas qu'il puisse être autre chose qu'un calcul de fortune. Cependant je suis persuadé que s'il connoissoit Fanny, elle le convertiroit; car elle réalise précisément l'idéal de perfection dont il nie l'existence. Je ne lui ferai part de mon mariage que lorsque je serai engagé irrévocablement."
"Enfin, si ce n'est pas le secret de votre voyage à Londres, du moins Fanny y est pour\setcounter{page}{257} quelque chose. Ma curiosité ne va pas plus loin; mais vous n'avez pas d'idée combien tout cela m'intéresse. J'en aurai pour long-temps à m'étonner et à faire des questions. Qui auroit jamais cru que le séjour de Mansfield décideroit de votre sort!.... Mais en vérité plus j'y pense, et plus j'approuve votre choix. Vous êtes assez riche pour prendre une femme sans dot. La famille est honorable.— La nièce de Sir Thomas se présentera fort bien dans le monde comme l'épouse de Henri Crawford.— Mais, dites-moi où en sont les choses? est-elle préparée à ce haut degré de félicité?"
"Elle n'en sait encore rien. "
"Qu'attendez-vous donc? "
"Je saisirai la première occasion. Cela n'est pas si facile qu'avec ses cousines. Cependant je n'ai pas de doute sur le résultat. "
"Oh! je le crois bien. En supposant même qu'elle n'ait pas déjà de l'inclination pour vous, ce qui n'est guères probable, elle sera profondément touchée et reconnoissante d'une préférence aussi flatteuse. Cependant, je suis convaincue que s'il y a une femme au monde qui soit supérieure à des motifs d'ambition, c'est Fanny. Si elle étoit sûre de ne pas vous aimer, elle vous refuseroit;\setcounter{page}{258} mais comment ne vous aimeroit-elle pas?... "Henri ne demandoit pas mieux que de parler de Fanny. Sa sœur l'écoutoit avec plaisir, en sorte que l'entretien fut long et animé. La beauté, la grace, la bonté de Fanny étoient des sujets inépuisables. Il avoit observé mille traits de délicatesse, de modestie, de candeur; mais ce qui lui plaisoit sur-tout en elle, c'étoit la douceur de son caractère. Quel est l'homme, en effet, qui ne met pas la douceur au premier rang des attributs d'une femme? L'affection de Fanny pour son frère montroit qu'elle étoit capable d'une vive sensibilité. La possession d'un cœur si tendre devoit être d'un prix inestimable. Son esprit étoit prompt, son jugement sûr; ses manières et tout l'ensemble de sa personne offroient l'image d'une ame pure et prouvoient un goût délicat. Mais ce qui mettoit le sceau à tant de qualités, et ce dont Henri avoit trop de sens pour ne pas apprécier tout l'avantage, c'étoit la rectitude des principes de Fanny. Il avoit peu réfléchi sur des sujets sérieux. Ses notions de morale étoient vagues, cependant il sentoit obscurément, que l'élévation d'ame, le respect du devoir, la droiture du caractère, étoient chez Fanny le résultat de ces\setcounter{page}{259} principes religieux qu'il connoissoit à peine.
"J'aurois," disoit-il, "une confiance entière et illimitée pour une telle femme, et c'est à mes yeux la base la plus sûre du bonheur conjugal."
Flora partageoit l'opinion de son frère sur Fanny. "Je conviens," disoit-elle, "que je n'aurois pas deviné qu'elle vous inspireroit un pareil attachement, mais je ne doute point qu'elle ne le mérite. — Voyez comme tout cela a bien tourné. Vos intentions en commençant n'étoient assurément pas des meilleures."
"Je ne connoissois pas cette excellente créature: c'est là ma seule excuse; mais elle n'aura jamais lieu de se plaindre du hasard qui l'a placée en mon chemin. Je la rendrai parfaitement heureuse: elle n'a pas été gâtée jusqu'à présent, la pauvre petite! Je ne l'éloignerai pas de sa famille. Je louerai une habitation dans le voisinage de Mansfield, j'affermerai ma terre d'Everingham pour quelques années."
"Eh bien! c'est charmant," dit Flora, "nous vivrons tous réunis."
Cette exclamation ne lui eut pas plutôt échappé, qu'elle auroit voulu la reprendre, mais Henri n'y vit pas finesse. Il com-\setcounter{page}{260} prit seulement qu'elle parloit de son séjour au presbytère, et répondit en lui disant qu'il espéroit bien qu'elle partageroit son temps entre sa sœur et lui.
"Je suppose, dit Flora, que vous passerez les hivers à Londres et que vous y aurez une maison à vous. Je regarde comme une circonstance très heureuse pour votre entière conversion, que vous ne viviez plus sous l'influence immédiate de notre oncle: encore quelques années et vous étiez perdu sans ressources."
"Je connois les torts de mon oncle, mais il est bon: il m'a tenu lieu de père. Ne donnez pas à Fanny des préventions contre lui, car je désire beaucoup qu'il s'établisse entr'eux une relation douce."
Flora ne croyoit pas que cela fût possible, mais elle s'abstint de le dire, et ajouta seulement une réflexion sur le sort de sa tante défunte, en exprimant l'espérance que celui de Fanny seroit bien différent. Henri protesta avec chaleur, que son vœu le plus cher étoit de la rendre parfaitement heureuse, et revint avec complaisance sur les impressions qu'il avoit reçues dans cette dernière visite.
"J'aurois voulu que vous la vissiez ce matin,\setcounter{page}{261} chère Flora, si douce et si complaisante pour cette stupide lady Bartram, à qui elle consacre tous ses momens sans jamais avoir l'air d'y attacher aucun mérite, tant la reconnaissance et le dévouement lui sont naturels. Elle a une grace inimitable dans l'arrangement de ses cheveux. Ce matin, une boucle échappée tomboit sur son front, et vous n'avez pas d'idée du joli mouvement qu'elle faisoit de temps en temps pour la rejeter en arrière. Elle écrivoit un billet, et ne perdoit point cependant le fil de la conversation. Elle écoute avec esprit; elle rougit aisément, et cela donne beaucoup de variété à sa physionomie."
"Je suis ravie," dit Flora en souriant, "de vous voir si complétement amoureux. Mais que diront Marie et Julia?,
"Je me soucie peu de ce qu'elles pourront dire ou penser. Elles comprendront à présent, la différence qu'il y a entre les agrémens qui séduisent, et les qualités qui attachent. Je souhaite que cette découverte leur soit utile. Elles verront leur cousine traitée par tout le monde avec les égards qu'elle mérite, et je voudrois qu'elles pussent avoir honte de leur conduite passée vis-à-vis d'elle. Elles seront piquées au vif. Je\setcounter{page}{262} m'attends au ressentiment de mistriss Rushworth. Son orgueil sera blessé; mais je ne suis pas assez fat pour imaginer que son chagrin soit de longue durée.— Combien il me sera doux de voir ma Fanny honorée de tout ce qui l'entoure; de penser que c'est moi qui l'aurai tirée de cet état de dépendance, d'infériorité, d'isolement, pour la placer dans un rang où tout son mérite puisse être apprécié.
"C'est fort bien," répondit Flora, "mais permettez-moi de vous dire que vous chargez un peu le tableau de sa situation présente pour faire ressortir la brillante perspective qui l'attend. Son cousin Edmond est pour elle un ami véritable.
"Oui, cela peut être. Edmond se conduit bien avec elle. Sir Thomas aussi, à sa manière; c'est-à-dire, avec l'exigence et le ton sentencieux d'un oncle riche, qui est accoutumé à dominer. Mais Edmond et Sir Thomas que feroient-ils pour elle en comparaison de moi, qui la comblerai de tous les biens qui peuvent embellir l'existence d'une femme."
( Henri étoit allé à Londres solliciter de l'avancement pour William, mais il n'avoit pas attendu le résultat de ses démarches, et\setcounter{page}{263} c'est à Mansfield qu'il reçoit la lettre dans laquelle on lui annonce la promotion de William au grade de lieutenant. Il court porter cette nouvelle à Fanny, dont la joie, la reconnaissance et le trouble sont interprétés par lui comme des signes favorables à son amour. Il lui fait une déclaration très-animée, qui met Fanny dans une pénible alternative entre le sentiment de l'obligation que son frère vient de contracter, et le chagrin de devoir un tel service à des sentimens auxquels elle ne peut répondre. D'ailleurs, la conduite plus que légère de Henri avec ses cousines lui laisse une extrême défiance. Elle soupçonne qu'il tend un piège à sa vanité et ne sait quel ton elle doit prendre avec lui. Ses expressions douteuses, entrecoupées, ne détrompent point tout-à-fait Henri, mais au milieu de ses protestations, il entend la voix de son oncle, et sous-prétexte d'aller lui apprendre l'avancement de William, elle s'échappe et ne reparaît que lorsqu'elle s'est assurée que Crawford est parti. Celui-ci ne se tient point pour battu, et revient le lendemain faire sa demande en forme à Sir Thomas ).
Fanny qui aperçut Henri Crawford à quelques pas du château, s'enfuit précipitamment.\setcounter{page}{264} dans sa chambre. Au bout d'une demie heure, elle commençoit à espérer que sa visite ne la regardoit pas, lorsqu'elle entendit quelqu'un dans l'escalier; elle reprit toutes ses terreurs. C'étoit son oncle; mais avant de dire à Fanny le sujet qui l'amenoit, il remarqua avec surprise qu'il n'y avoit point de feu à la cheminée quoique le temps fût très-froid.
Fanny n'avoit aucune envie de se plaindre de sa tante Norris, mais malgré ses ménagemens, il fut évident pour sir Thomas que c'étoit une économie de sa belle-sœur.
"Je reconnois là," dit-il, "un système de votre tante, qui est bon en lui-même, mais dont elle a fait dans ce cas-ci une application peu judicieuse. Elle pense qu'il est nuisible pour les jeunes gens d'être élevés d'une manière trop délicate. Je me suis bien aperçu quelquefois qu'elle pousseroit cela un peu loin à votre égard; mais vous avez sûrement, ma chère enfant, un trop bon esprit pour en vouloir à votre tante. Vous comprendrez qu'il pouvoit lui paroître sage de vous préparer de bonne heure à la médiocrité de fortune qui devoit être votre lot. Soit que les événemens justifient ou non ce calcul, vous reconnoîtrez l'avantage des habitudes\setcounter{page}{265} d'économie qui rehaussent le prix de l'aisance et apprennent à s'en passer.
Ce petit préambule donna à Fanny le temps de se remettre.
"Voulez-vous que nous causions un moment ensemble," lui dit-il avec un léger sourire. "Vous ne savez peut-être pas que j'ai eu ce matin la visite de Mr. Crawford. N'êtes-vous point préparée, maintenant à ce que je viens vous dire?"
Sir Thomas vit Fanny si embarrassée à répondre qu'il n'insista pas. "Mr. Crawford, continua-t-il, est venu me déclarer ses sentimens pour vous et le vœu d'obtenir votre main. Cette offre et la manière dont il la présentée, sont également flatteuses. Après cette communication, que Fanny écouta en silence, il entra dans les détails de l'entretien qu'il avait eu avec Mr. Crawford, dont il avait été parfaitement content. Il était tellement persuadé du plaisir que Fanny prenait à l'entendre qu'il ne s'étonna point de ce qu'elle le laissoit parler sans l'interrompre. Enfin il se leva en lui disant: "À présent, ma chère nièce, j'ai rempli la première et la plus importante partie de ma mission. Je vous ai montré les choses telles que je les vois,"\setcounter{page}{266} c'est-à-dire, comme présentant des avantages à la fois folides et brillans. Il ne me reste qu'à vous engager à descendre avec moi dans ma chambre, où vous trouverez quelqu'un dont l'éloquence sera surement encore plus persuasive que la mienne."
L'impression de ces dernières paroles sur Fanny fut bien différente de celle que son oncle en attendoit, mais l'étonnement fut à son comble lorsqu'enfin forcée de s'expliquer tout-à-fait, elle manifesta une répugnance extrême à voir Mr. Crawford. "Dispensez-moi, je vous en conjure, de cette entrevue, s'écria-t-elle. Il doit connoître ma façon de penser, à moins qu'il n'ait voulu se faire illusion. Je lui ai dit hier sans déguisement, que sa demande ne m'étoit pas agréable, et qu'il m'étoit impossible de répondre à ses sentimens."
" Je crois que nous ne nous entendons pas," dit sir Thomas en se rasseyant, "car je ne peux donner à ce que vous me dites, aucune interprétation qui ait le sens commun. Je sais qu'il vous a parlé hier de ses intentions, et je conçois qu'une première déclaration à laquelle vous ne vous attendiez probablement pas, vous ait jetée dans un embarras très-naturel à votre âge. Ce qu'il m'a\setcounter{page}{267} m'a dit de votre modestie, à cette occasion, m'a fait grand plaisir, et je vous ai bien reconnue là; mais à présent qu'il s'adresse à vous par mon organe et avec mon approbation, je ne puis imaginer quel est le scrupule qui vous arrête."
"Vous êtes dans l'erreur, mon cher oncle.
— En vérité, Mr. Crawford vous a induit en erreur s'il vous a dit que ma réponse pût être prise comme un encouragement. Je ne me rappelle pas exactement quelles ont été mes expressions, mais elles ne pouvoient lui laisser aucun doute sur le fond de ma pensée. Il est vrai que je ne prévoyois pas sa démarche d'aujourd'hui, sans quoi je l'aurois prévenue. Je l'ai toujours vue si léger que j'ai mis peu d'importance à ses paroles.
"Enfin," dit sir Thomas, "quel peut être le résultat de tout cela? Votre intention n'est sûrement pas de refuser Mr. Crawford."
"Pardonnez-moi, monsieur, c'est bien ce que j'ai voulu dire."
"Refuser Mr. Crawford! par quelle raison? sous quel prétexte?"
"Parce que je sens que je ne puis l'aimer..."
"Cela est bien étrange! Il y a quelque chose là dedans qui dérange toutes mes idées. Il se présente pour vous un établissement le\setcounter{page}{268} plus avantageux à tous égards qu'il soit possible d'imaginer; figure, esprit, fortune, réputation, Mr. Crawford réunit tout ce qui peut captiver le goût et décider le jugement. Ses alentours sont agréables, vous êtes liée avec sa sœur. J'aurois cru sur-tout que le service qu'il a rendu à votre frère seroit une puissante recommandation auprès de vous: il est fort douteux que mon crédit eût obtenu le même succès pour William. "
" Je vous assure que j'en suis bien reconnoissante, " dit Fanny toujours plus embarrassée de la bizarrerie apparente de sa conduite.
" Il est impossible, " continua sir Thomas, " que vous n'ayez pas vu en dernier lieu combien Mr. Crawford étoit occupé de vous: Ses attentions étoient extrêmement marquées. Votre manière de les recevoir a été parfaitement convenable, je me plais à vous rendre cette justice, mais, assurément vous n'aviez pas l'air d'en être importunée. — Je suis tenté de penser, mon enfant, que vous ne connoissez pas bien vous-même l'état de votre cœur."
" Oh, ne le croyez pas, monsieur. Dans aucun moment je n'ai vu avec plaisir qu'il s'occupât de moi."\setcounter{page}{269} Sir Thomas regardoit sa nièce avec une surprise croissante. "Ceci passe ma compréhension," dit-il, "c'est une véritable énigme pour moi."
Il s'arrêta là en fixant sa nièce, comme pour pénétrer dans les replis de sa pensée. "Il n'y auroit qu'une supposition..., mais un instant de réflexion la rend tout-à-fait improbable."
Fanny trembloit d'avoir à soutenir un examen plus approfondi, et sans bien savoir ce qu'elle avoit à cacher, elle craignoit une découverte. Sir Thomas commandant à son impatience, reprit avec beaucoup de sang-froid." Indépendamment de l'intérêt que le choix de Mr. Crawford doit lui mériter, c'est un desir louable dans sa position que celui de se marier jeune. Il y a beaucoup d'avantage, à ce qu'un homme qui a une fortune honnête se marie lorsqu'il a atteint l'âge de vingt-cinq ans, et je vois avec peine que mon fils aîné ne paroisse point encore disposé à choisir une femme."
Ici le Baronet jeta un coup-d'œil scrutateur sur Fanny, puis il continua." Edmond est dans des circonstances différentes. Il doit penser au mariage, et si je ne me trompe, il a. déjà formé quelque vœu à cet. égard.\setcounter{page}{270} Ne le croyez-vous pas comme moi, ma chère ?"
"Oui, mon oncle," dit Fanny à demi voix, mais avec assez de calme pour écarter de l'esprit de Sir Thomas un soupçon qui commençoit à s'élever.
La difficulté d'expliquer ce refus n'en devenoit que plus grande, et rassuré sur ce point, il étoit moins disposé à pardonner ce qui lui sembloit un caprice inexcusable.
Il se leva et fit deux ou trois tours dans la chambre, puis il lui dit avec une expression qui commençoit à devenir sévère. "Avez-vous quelque raison d'avoir mauvaise opinion du caractère de Mr. Crawford ?"
"Non pas précisément," dit Fanny, mais elle n'osa pas ajouter ce qu'elle pensoit de ses principes, car elle ne pouvoit le faire sans rendre compte à son oncle de ses observations, et sans jeter en même temps du blâme sur ses cousines. Sir Thomas prit un air plus grave encore, et s'arrêtant vis-à-vis de Fanny, il lui dit : "je vois qu'une plus longue discussion seroit également inutile et pénible ; mais il me semble qu'il est de mon devoir de vous faire sentir ce que votre conduite dans cette occasion a de répréhensible. J'attendais tout autre chose de vous."\setcounter{page}{271} Jusqu'ici, vous m'aviez paru exempte de cette présomption, de cet esprit d'indépendance qui font des progrès effrayans parmi notre jeunesse, et qui, chez les femmes sur-tout, sont des défauts si choquans; mais je vois que vous croyez pouvoir vous passer de conseils et vous décider par vous-même dans l'affaire la plus importante de votre vie, sans aucun égard à l'opinion de ceux que vous deviez naturellement prendre pour guides et qui ont quelques titres à votre confiance.
"Vous ne faites entrer pour rien dans votre détermination l'avantage de votre famille, les intérêts de vos frères et sœurs. Mr. Crawford est un homme bien né, d'un esprit distingué, d'un caractère aimable; il a pour vous l'attachement le plus vif et le plus désintéressé; et parce que vous n'éprouvez pas précisément l'espèce de sentiment qu'une imagination exaltée vous représente comme nécessaire au bonheur, vous rejetez sans examen, sans même prendre le temps d'y réfléchir, l'offre d'un établissement supérieur à tous égards à ce que vous aviez lieu d'espérer, et tel qu'il ne s'en présentera probablement jamais d'autre."
"Si Mr. Crawford avoit demandé la main d'une de mes filles, je n'aurois point hésité\setcounter{page}{272} à la lui accorder, et j'aurois été fort surpris que l'une ou l'autre de vos cousines eût refusé une offre pareille; mais si elle l'avoit fait sans aucune attention, sans aucune différence à mes avis, j'aurois regardé cela comme un manque d'affection et de respect, qui m'auroit été fort sensible. Je sais que je ne dois pas exiger de vous tout ce que comporte le devoir filial, mais, Fanny, je doute que votre cœur vous acquitte tout-à-fait du reproche d'ingratitude.
Lorsque Sir Thomas en fut là de son chapitre d'accusations, et que le terrible mot d'ingratitude eut été prononcé, la pauvre Fanny pleuroit si amèrement, que la sévérité de son oncle en fut adoucie. "Vous pleurez, mon enfant. Je comprends votre chagrin, mais j'espère que vous sentez les motifs qui ont dicté ce que je viens de vous dire, et je souhaite de tout mon cœur que vous n'ayez pas le tourment des regrets, quand il sera trop tard pour réparer l'effet d'une résolution précipitée."
"Je sens," dit Fanny toujours en sanglottant, "combien je dois avoir de tort à vos yeux; mais il m'est impossible — tout-à-fait impossible. — Je ne pourrois faire son bonheur, et je serois moi-même malheureuse,"\setcounter{page}{273} Sir Thomas vit qu'il falloit laisser passer ce moment d'agitation un peu nerveuse, et sans croire que Crawford dût désespérer d'un changement favorable, il eut l'air d'être vaincu de l'inutilité de ses efforts.
"Allons, mon enfant," lui dit-il, "séchez vos larmes, et tâchez de vous tranquilliser. Je vois bien qu'il faut que je me charge de votre réponse à Mr. Crawford, quoique j'eusse fort désiré que vous la fissiez vous-même. — Nous l'avons tenu trop long-temps en suspens. Allez faire un tour dans le jardin pour effacer la trace de vos larmes, et si cela peut vous soulager, je vous promets que la chose restera ignorée de vos tantes: il est inutile d'en parler à qui que ce soit."
\setcounter{page}{101}
\chapter{ROMANS}
\section{MANSFIELD-PARK. \large{( Second extrait. Voy. p. 490 du vol. précéd. )}}
A la fin de l'été Tom Bartram amena à Mansfield-Park, une nouvelle connoissance, Mr. Yates, un de ces jeunes désœuvrés dont la seule recommandation est une fortune et des manières qui les placent parmi les gens du bon ton. Il faisoit partie d'une société dans laquelle on s'étoit fort occupé de jouer la comédie, mais la mort d'un parent avoit fait manquer la chose. Il en avoit encore la\setcounter{page}{102} tête pleine lorsqu'il arriva à Mansfield-Park, où ses récits firent naître la fantaisie de ce genre d'amusement. Tom et ses sœurs furent les premiers à en faire la proposition; Henri Crawford y applaudit vivement.
"Je me sens en verve," dit-il, "et je ne sais quel est le rôle que je n'accepterois pas avec plaisir; depuis le monarque au valet, depuis le héros tragique au farceur de trétaux; je suis prêt à tout essayer. Il me semble que rien ne nous manque ici de ce qui peut assurer le succès. Pour les rôles de femmes, du moins, je doute que dans toute l'Angleterre, on pût trouver une pareille réunion. Faisons quelques essais, ne fût-ce que des scènes détachées. — Nous n'avons pas besoin d'un théâtre. Il ne s'agit pas de faire effet, mais de nous amuser; qu'en dites-vous ?"
"Sans doute," dit Mr. Yates, "nous ferons les choses très-simplement. Deux ou trois décorations. Un barbouilleur d'enseigne peut vous faire cela. — Vous avez bien ici près un charpentier pour l'échafaudage, n'est-ce pas ?"
"Je crois," dit Marie, "qu'il faut nous contenter à moins encore. Si nous mettons des retards, les difficultés se multiplieront. J'adopte plutôt l'idée de Mr. Crawford ;\setcounter{page}{103} cherchons des pièces qui n'exigent point d'appareil."
" Je ne suis pas de cét avis là, " dit Edmond, qui commençoit tout de bon à prendre peur que la chose ne s'exécutât et qui en pressentoit tous les inconvéniens." Je n'aime rien à demi, si nous faisons tant que de jouer la comédie, ayons, je vous prie, tout ce qui doit rendre le spectacle complet, théâtre, parterre, loges, orchestre, etc. il faut faire parler de nous ou ne pas s'en mêler."
" Allons donc Edmond, " dit Julia, " ne faites pas le mauvais plaisant. Au vrai, quand vous vous serez bien moqué de nous, vous serez charmé d'être des nôtres, car personne n'aime mieux le théâtre que vous."
" J'en conviens, mais c'est la véritable comédie que j'aime, représentée par des acteurs qui ont fait de cette étude la principale affaire de leur vie, et non par des personnes, qui ont à combattre mille désavantages, et dont les qualités comme gens bien nés, sont autant de défauts comme acteurs."
Une petite pause dans la conversation suivit la remarque d'Edmond. On reprit bientôt le sujet avec vivacité, et cependant on se sépara sans avoir rien décidé. Édouard espéroit encore parvenir à écarter cette idée,\setcounter{page}{104} mais dès le soir même il fit un essai de son crédit qui lui en montra l'insuffisance.
Il tisonnoit le feu tout en méditant sur les moyens de faire échouer le projet de comédie. Lady Bartram étoit couchée sur son canapé et Fanny travailloit auprès d'elle.
Tom entra dans le salon, en s'écriant : "Je viens de vérifier une chose, c'est que la salle de billard est précisément ce qu'il nous faut pour un théâtre. Les dimensions en sont parfaites, et la chambre de mon père qui est à côté, nous sera d'un grand secours; il ne s'agit que d'établir la communication en enlevant les étagères des livres : c'est l'affaire de cinq minutes que cette opération."
"Tout cela est bon pour la plaisanterie," répondit Edmond.
"Parbleu! non. Je parle très-sérieusement. Qu'est-ce qui te surprend là dedans,?"
"Je pense qu'en général les comédies de société fourmillent d'inconvéniens, mais dans nos circonstances en particulier, je vois plusieurs raisons qui doivent nous interdire ce genre d'amusement. L'absence de mon père d'abord et les dangers auxquels il peut être exposé dans une longue traversée. Ensuite la position de Marie qui demande beaucoup de circonspection : tout cela mérite d'être pesé."\setcounter{page}{105} " Tu prends toujours les choses si sérieusement. A t’entendre, on diroit que nous allons ouvrir une salle de spectacle et inviter toute la province. Ce n’est point cela, nous voulons seulement exercer nos talens dramatiques dans un très-petit cercle. On peut s’en fier à nous, je pense, pour choisir une pièce convenable, et je ne vois pas qu’il y ait plus de mal à emprunter le langage de quelqu’un de nos bons auteurs qu’à discourir sur des lieux communs. Quant à l’absence de mon père, ce seroit, à mon avis, une raison de plus, car ma mère a besoin de distraction dans un moment où elle peut avoir de l’inquiétude, et celle là lui seroit fort agréable. " Cette réflexion dirigea naturellement les regards des deux frères vers lady Bartram, qui s’étoit endormie pendant la discussion, et ne paroissoit pas avoir grand besoin de donner le change à ses inquiétudes sur son mari. Edmond ne put s’empêcher de sourire. Tom éclata tout haut, en disant :
" Il faut avouer que je n’ai pas été heureux dans le choix du moment pour argumenter de de la sollicitude de ma mère. "
" Qu’est-ce donc ? " dit lady Bartram en s’éveillant, " vous faites bien du bruit, il me semble. Je ne dormois pas cependant. "
" Non, non, ma mère; nous n’avons garde de le croire."\setcounter{page}{106} Lady Bertram recommença à s'assoupir, et Tom continua en prenant toujours le ton le plus haut et plus positif jusqu'à ce qu'il eût réduit son frère au silence. Quand il fut parti, Fanny qui n'avoit pas perdu un mot de la conversation et qui pensoit exactement comme son cousin, chercha à le tranquilliser en lui persuadant que les obstacles naîtroient de la chose même, et qu'elle ne s'exécuteroit point. Edmond ne l'espéroit pas et fit encore une tentative auprès de ses sœurs; mais celles-ci eurent des argumens tout aussi péremptoires que Tom à opposer aux représentations de leur frère, et se montrèrent aussi peu disposées à céder à l'autorité de la raison. Lady Bertram n'écouta qu'à demi le pour et le contre, en sorte que le nombre des voix l'emporta auprès d'elle. Julia convenoit que pour Marie il pouvoit y avoir quelque chose à dire, mais quant à elle-même c'étoit fort différent. Marie, de son côté, se regardoit déjà comme ayant acquis un certain degré d'indépendance de plus que sa sœur. Mistriss Norris, qui vouloit ménager également ses deux neveux, hasarda quelques légères objections, mais se rangea bientôt à l'avis général: elle prévoyoit avec plaisir combien elle alloit être nécessaire à la petite troupe pour tous les arrangemens\setcounter{page}{107} matériels. Ce seroit un prétexte pour elle de s'établir pendant quelques jours chez sa sœur, ce qui lui feroit une économie à laquelle elle étoit fort sensible. Edmond combattoit encore quelque peu d'espérance lorsqu'Henri arriva du presbytère apportant le consentement de sa sœur à se joindre au projet de comédie. Elle professoit un grand désir de faire ce qui conviendroit aux autres en se chargeant d'un rôle subalterne. Marie regarda Edmond, comme pour lui dire: "Et bien, mon frère, qu'en pensez-vous à présent, ceci ne change-t-il point votre manière de voir?". Edmond, un peu embarrassé, s'attacha à louer la manière aimable dont miss Crawford se prétoit aux convenances des autres, en renonçant à toutes prétentions pour elle-même. La difficulté que Fanny avoit prévue, celle de trouver une pièce sur laquelle on tombât d'accord, arrêta cependant assez long-temps la marche des opérations. Le charpentier avoit déjà pris ses mesures, tout étoit ordonné pour l'arrangement du salon, tandis qu'on délibéroit encore sur ce choix. Il falloit trouver une pièce où les acteurs fussent en petit nombre et où tous les rôles\setcounter{page}{108} fussent agréables, en particulier ceux des femmes. On parcourut les divers théâtres et rien ne remplissoit toutes les conditions demandées. Enfin, la lassitude de l'indécision et la persévérance de Tom à dire plus de paroles et d'un ton plus haut que personne, fit prévaloir son choix. On se réunit pour la comédie, intitulée, Les vœux des amans. La distribution des rôles mit encore en évidence la jalousie réciproque des deux sœurs; et toute l'adresse de Henri ne put empêcher que Julia, supplantée par Marie dans un rôle auquel toutes deux prétendoient, ne crût à un projet concerté entr'eux pour l'exclure. Sa défiance une fois excitée et son orgueil blessé, elle ne voulut entendre à aucun accommodement, et déclara qu'elle renonçoit à jouer. Fanny observoit en silence et s'amusait assez de ce conflit de prétentions rivales plus ou moins bien dissimulées. Après que l'assemblée délibérante se fut séparée, elle eut la curiosité de lire la pièce, qui avoit été choisie. Elle avoit un sentiment fort délicat sur la décence qui doit être l'apanage des femmes, et ne pouvoit revenir de son étonnement en voyant quels étoient les rôles acceptés par ses cousines. Elle supposa qu'elles ne connoissoient pas bien ce à quoi elles s'engageaient, et\setcounter{page}{109} attendoit impatiemment Edmond pour le prévenir là dessus, mais Edmond fut absent toute la matinée.
Dans l'intervalle qui s'écoula jusqu'à son retour Rushworth arriva à Mansfield-Park. Il ne fut pas difficile à Marie de lui faire adopter le rôle qu'on lui avoit destiné, tout en ayant l'air de lui laisser un certain choix. Lorsqu'Edmond rentra, Rushworth courut à sa rencontre. "Tout va bien, lui dit-il," " nous avons trouvé ce qui nous falloit. Nous jouerons Les vœux des amans. C'est moi qui fais le comte Cassel. Je parois d'abord avec un habit bleu de ciel et un manteau de satin orange. Ensuite, dans le dernier acte, j'aurai un costume de chasse, mais je n'ai pas encore décidé la couleur. "
Fanny regardoit son cousin et comprit ce qui se passoit en lui, avant même qu'il s'écriât d'un air fort surpris, et en se tournant vers ses sœurs comme pour savoir s'il avoit bien entendu. "Les vœux des amans, dites-vous? "
"Oui," dit Mr. Yates. "Après toutes nos recherches nous avons trouvé que rien ne nous alloit mieux, et nous ne comprenons pas pourquoi nous y avons pensé si tard. Moi qui n'avois entendu parler d'autre chose à Ecclesford; je ne sais où j'avois l'esprit\setcounter{page}{110} Nous avons déjà attribué à chacun son rôle.
"Et ceux des femmes ?" dit Edmond en regardant Marie d'un air inquiet, Marie rougit et répondit : "Je fais le rôle d'Agathe ; c'est celui que lady Morgan devoit avoir à Eclesford ; et nous avons donné celui d'Amélie à miss Flora," ajouta-t-elle en levant les yeux vers Edmond avec plus d'assurance, et en souriant à demi.
"Je n'aurois pas cru," dit Edmond, "que ce genre de comédie trouvât des approbateurs parmi nous."
Rushworth, sans rien écouter, continua à parler de son rôle. "Je parois trois fois sur la scène, et je prends la parole quarante-deux fois. Ce n'est pas peu de chose que ce rôle là. J'aimenois mieux cependant un costume plus simple. J'aurai une singulière tournure avec mon manteau de satin orange."
Edmond ne lui répondit pas grand chose, mais un moment après, Tom étant appelé pour consulter avec le charpentier, fut suivi de Yates et de Rushworth. Il dit alors à sa sœur : "Je ne pouvois pas m'expliquer devant Mr. Yates sans faire la censure de ses amis d'Eclesford, mais à présent, ma chère Marie, permettez-moi de vous dire que cette pièce n'est point faite pour une société comme la\setcounter{page}{111} nôtre: Je ne vous demande que de lire le premier acte avec ma mère et ma tante que voilà, et je suis sûr que vous y renoncerez.
"Je conviens," dit Marie, "qu'il y a quelques petits retranchemens à faire dans le premier acte, mais rien n'est si facile, et alors tout le reste est parfaitement convenable. Vous voyez bien que miss Crawford n'a fait aucune objection."
"J'en suis fâché," répondit Edmond, "mais dans ce genre de choses, c'est à vous à donner le ton. Si vous vous montrez scrupuleusement délicate, votre exemple ramènera les autres."
Marie fut flattée de l'influence que son frère attribuoit à son opinion, mais elle repré senta la difficulté de revenir en arrière sur ce qui avoit été arrêté définitivement."C'est justement ce que j'allois vous dire," interrompit mistriss Norris."A présent que tout va être prêt et que l'on a déjà déboursé bien de l'argent, il seroit vraiment ridicule d'en rester là. Je ne connois pas la pièce, mais s'il y a, comme vous dites, des passages un peu scabreux, on les retranchera, et quant à Marie, puisque Rushworth doit en être, tout est dans l'ordre. Seulement, c'est bien dommage que mon neveu n'ait pas mieux su d'abord ce qu'il y avoit à faire\setcounter{page}{112} pour le charpentier, car il y a eu une bonne demi journée perdue. La toile ira à merveilles, et je crois que nous aurons au moins une douzaine d'anneaux à renvoyer au marchand; il n'est pas nécessaire de les mettre si près les uns des autres. 'Parlez-moi d'avoir quelqu'un qui entende l'économie. Les jeunes gens ne sentent point la conséquence des choses; ils jetteroient tout par les fenêtres, quand ils ont une fantaisie. "
Edmond vit bien qu'il étoit inutile de persister plus long-temps dans son opposition, et se consola en pensant qu'il avoit fait ce qui dépendoit de lui.
Le soir, lorsque la famille fut rassemblée, Henri et sa sœur arrivèrent, au grand contentement de la petite société. Miss Flora complimenta lady Bartram sur l'ennui qu'avoient dû lui donner toutes les discussions dramatiques, et en jetant un coup-d'œil sur Edmond, qui ne disoit rien, elle comprit qu'il n'approuvoit pas, mais elle n'en fit pas semblant. Ensuite s'approchant d'une petite table, autour de laquelle les jeunes gens étoient en conférence très-animée, toujours sur le même sujet, elle écouta leur conversation, puis tout-à-coup, comme si une idée nouvelle se présentoit à son esprit, elle les interrompit en disant: " A propos ?\setcounter{page}{113} 'Bus, vous ne m'avez point dit qui doit faire le rôle d'Anhalt. Il est pourtant assez intéressant pour moi de savoir quel est l'heureux mortel parmi vous à qui mon cœur et ma main sont destinés au dénouement. "
On se regarda un moment sans répondre, et il fallut avouer qu'on n'y avait pas encore pensé.
" Ils m'ont donné le choix," dit Rushworth, "entre Anhalt et le comte Cassel; et j'ai été assez embarrassé; mais miss Marie m'a conseillé le Comte, cependant je n'aime pas beaucoup à paraître dans un costume brillant. "
" Vous avez parfaitement bien choisi," répondit Flora en souriant de plaisir. "Anhalt est un assez plat rôle. "
" Le Comte parle quarante-deux fois. Ce n'est pas une bagatelle. Qu'en dites-vous? "
" Je ne m'étonne pas qu'il y ait eu peu d'empressement pour le rôle d'Anhalt. Cette Amélie dont vous m'avez chargée de faire le personnage est une petite péronnelle si décidée que cela effraie un amant. "
" J'aurois bravé ce danger-là," dit Tom. "Si j'avois pu être à-la-fois le fermier et Anhalt, mais ils paroissent en même temps sur la scène. Cependant je ne sais pas s'il\setcounter{page}{114} n'y auroit point moyen d'arranger encore cela."
"Votre frère devroit prendre ce rôle," dit Mr. Yates à l'oreille de Tom."Croyez-vous qu'on ne pourroit pas le gagner?,"
"Je ne l'essayerai surement pas," répondit Tom." Miss Crawford parla d'autre chose, et bientôt elle s'éloigna du groupe des acteurs et s'assit auprès de la table, en disant."Ils n'ont pas besoin de moi. Je les embarrasse parce qu'ils se croient obligés de me faire des complimens; mais vous, Mr. Edmond, qui êtes tout-à-fait en-dehors, vous nous donnerez un conseil désintéressé. C'est à vous que j'ai recours. Comment ferons - nous pour un Anhalt?"
"Mon avis," dit Edmond,"c'est que vous choisissiez une autre pièce."
Je m'y rangerois volontiers, quoique le caractère d'Amélie me paroisse assez piquant, s'il est bien rendu et bien associé; mais je suis déterminé à n'avoir point de volonté, et je doute qu'il fût facile de faire adopter votre conseil."
Edmond se tut, et miss Crawford continua."Si vous aviez pu vous déterminer pour un rôle quelconque, je suppose que ç'auroit\setcounter{page}{115} été pour celui d'Anhalt, car c'est un ecclésiastique, vous savez."
"Ce ne seroit pas cette circonstance qui me tenteroit ; je craindrois d'autant plus de m'en acquitter mal et de rendre ainsi le personnage ridicule. "
, Miss Crawford dissimula un petit mouvement d'humeur, et se tourna vers mistriss Norris pour lui faire la conversation.
"Fanny ! " s'écria Tom de l'autre extrémité du sallon, " nous aurons besoin ici de votre complaisance. "
Fanny, qui étoit fort accoutumée à cette espèce d'appel et que ses cousines employoient souvent pour soulager leur paresse, pour mille petits services, se leva immédiatement, croyant qu'on alloit lui demander quelque chose de ce genre.
"Oh ! ne bougez pas de votre place, " lui dit Tom, " ce n'est pas dans ce moment que nous avons besoin de vous ; mais nous vous mettons en réquisition pour notre troupe. Il faut absolument que vous fassiez le rôle de la fermière. "
"Moi ! " s'écria Fanny, que la seule idée de se mettre au spectacle fit frémir. " Pas pour tous les biens du monde ! "
"Bon ! quel enfantillage ! pourquoi cela\setcounter{page}{116} vous feroit-il peur? C'est presque un rôle muet. Pourvu que vous figuriez quelques momens sur la scène, c'est tout ce qu'il en faut. "
" Ah! par exemple, " dit Rushworth, " si vous avez peur d'avoir deux ou trois tirades à débiter, que seroit-ce, si vous aviez mon rôle à apprendre! moi qui parle quarante-deux fois. "
" Ce n'est pas la peine d'apprendre un rôle qui m'effraie, " dit Fanny, toujours plus déconcertée. " Mais je n'ai surement aucun talent pour la comédie. "
" C'est égal, " dit Tom, " apprenez seulement le rôle par cœur: nous vous formerons pour le reste. Vous n'avez que deux scènes, et comme je suis le fermier, je vous soufflerai, et je vous ferai cheminer parfaitement. "
" Non, non, je vous en conjure; n'en parlons plus. Vous n'avez pas d'idée combien j'en suis incapable. "
" Bah! bah! vous êtes beaucoup trop timide. Nous serons très-indulgents pour vous. On n'attend pas une perfection. Vous aurez une robe brune, un tablier blanc, une petite béguine bien avancée sur le front, nous vous peindrons quelques rides au visage, et\setcounter{page}{117} vous aurez une excellente mine de bonne petite vieille. "
Fanny, encouragée par un regard d'Edmond, persista dans son refus; cependant Tom insistait toujours. Alors mistriss Norris dit à l'oreille de Fanny, mais assez haut pour être entendue de tout le monde.
"Voilà bien des paroles pour peu de chose. Ne faites donc pas l'enfant. Il vous siérait bien en vérité de refuser d'obliger vos cousines, qui ont tant de bontés pour vous. Allons donc, faites de bonne grace ce qu'on vous demande, et que je n'en entende plus parler. "
"Ne la tourmentez pas ainsi, ma tante, ,, dit Edmond. "Vous voyez que cela lui fait du chagrin. Pourquoi ne se déterminerait-elle pas librement comme nous tous? Il n'y a pas de justice à exiger d'elle une chose qui lui est désagréable. "
"Je n'exigerai rien, à la bonne heure; mais je ne puis souffrir l'obstination, et je voudrais qu'on n'oubliât pas qui l'on est, et à qui l'on a à faire. "
Edmond était trop indigné pour répondre. Miss Crawford voyant que Fanny était prête à pleurer, prit un prétexte pour changer de place. Elle alla se mettre près d'elle, et lui dit à demi voix: "je ne sais sur quelle herbe\setcounter{page}{118} ils ont marché aujourd'hui. Ils sont tous insupportables. Croyez-moi, ne nous en embarrassons pas. Laissons-les dire et causons ensemble." Alors elle trouva moyen d'intéresser Fanny en lui parlant de son frère, dont Edmond lui avoit fait l'éloge, et peu à peu elle lui fit oublier l'impolitesse de Tom, et la brutalité de mistriss Norris. Elle se fit ainsi une très-bonne note auprès d'Edmond, à qui cette attention aimable n'échappa point.
Fanny n'aimoit pas miss Crawford, mais son cœur toujours accessible à la bienveillance, fut touché de son procédé. Tom interrompit leur conversation au bout de quelques momens en disant à miss Flora qu'il voyoit bien qu'il seroit impossible de faire le rôle d'Anhalt en addition à celui du fermier, mais qu'il connoissoit dans le voisinage quatre ou cinq jeunes gens très-présentables, qui demandoient instamment d'être reçus dans la société dramatique, et parmi lesquels, deux en particulier ne seroient point indignes de cette faveur. Miss Crawford regarda Edmond, espérant qu'il s'opposeroit à cette admission, mais il ne dit rien, et il fut convenu que dès le lendemain matin Tom iroit annoncer à Charles Madox que son offre étoit acceptée.\setcounter{page}{119} Fanny, soulagée momentanément par la politesse de Flora, retomba dans ses inquiétudes sur les persécutions du lendemain, quand elle fut retirée dans sa chambre. Elle ne se fioit pas assez à son propre jugement pour soutenir avec fermeté une résolution à laquelle la timidité avoit la plus grande part, et prévoyoit qu'il lui faudroit, pour refuser, une espèce de courage dont elle ne se sentoit pas capable. L'habitude de consulter Edmond lui avoit fait un besoin d'être approuvée par lui, et elle attendoit impatiemment de pouvoir lui en parler. Dès qu'elle fut levée, elle alla s'établir dans une chambre voisine de la sienne, qui servoit autrefois aux leçons, et qui étant abandonnée depuis que les demoiselles Bartram n'avoient plus de gouvernante, étoit devenue la propriété de Fanny. Mistriss Norris elle-même avoit acquiescé à cette prise de possession, en réservant la clause qu'on n'y feroit jamais de feu. Il y avoit une petite bibliothèque. Fanny y trouvoit des souvenirs doux, quoique mélangés, et une tranquillité rarement interrompue.
A peine cependant y eut-elle passé un quart d'heure, qu'elle entendit frapper dou\setcounter{page}{120} cement à la porte. C'étoit Edmond.
"Pourrois-je causer quelques momens avec vous, Fanny ?" lui dit-il.
"Oui, certainement."
"Je voudrois vous consulter. J'ai besoin d'un bon conseil, et j'ai beaucoup de confiance dans votre jugement."
"Rien ne peut me flatter davantage, mais je crains que vous n'ayez trop bonne opinion de moi."
"Je suis dans une grande perplexité. Le projet de comédie m'inquiète véritablement, et je vois que les choses vont de mal en pire, Le choix de la pièce est aussi mauvais que possible; et à présent ils ont encore inventé de s'associer un jeune homme que nous connoissons à peine. La familiarité qui s'introduit nécessairement dans les répétitions me paroit une chose très-fâcheuse pour de jeunes demoiselles, et je crois que nous devons faire tout ce qui est possible pour l'empècher. N'êtes-vous pas de mon avis ?"
"Sans doute, mais que faire ? Votre frère est si déterminé."
"J'y ai beaucoup pensé, et je ne vois qu'un seul moyen, c'est de prendre moi-même le rôle d'Anhalt."
Fanny fut désagréablement surprise et\setcounter{page}{121} ne put rien répondre. Edmond continua. "Je vous assure que cela me déplait beaucoup. Je sais qu'on me taxera d'inconséquence. Après tout ce que j'ai dit contre ce projet, il y a quelque chose de ridicule à m'y joindre; mais je vous en fais juge, ai-je quelque autre manière de prévenir un plus grand mal?"
"Il ne s'en présente point à moi," dit Fanny lentement et avec un peu d'hésitation, "cependant......"
"Je vois bien," interrompit Edmond, que vous n'êtes pas convaincue; mais réfléchissez y encore. Peut-être n'avez-vous pas pensé à tout ce qu'entraîne une semblable association. Ce jeune homme seroit reçu ici dans la plus grande intimité; à toute heure. Cependant c'est un étranger pour miss Crawford. Quant à moi, j'avoue que l'idée de ces répétitions m'est odieuse. Je parle de miss Crawford parce qu'elle y est sur-tout intéressée, devant faire le rôle d'Amélie. Elle mérite qu'on ménage ses sentiments parce qu'elle sait entrer dans ceux des autres. Ne trouvez-vous pas que j'ai raison Fanny? - Vous hésitez. "
"Je suis fâchée pour miss Flora, mais je le serois davantage de vous voir agir d'une\setcounter{page}{122} manière si opposée à votre première résolution, et à ce que vous savez être la façon de penser de mon oncle. Ce sera un sujet de triomphe pour eux tous. J'avoue que j'aurois de la peine à en prendre mon parti."
"Il faut savoir se mettre au-dessus des petites répugnances lorsqu'on a de fortes raisons d'agir. Si je ne me prête à rien, je serai mal placé pour m'opposer à quoique ce soit, mais au contraire cet acte de complaisance m'acquerra le droit d'avoir voix en chapitre, et je l'employerai à restreindre beaucoup le nombre des auditeurs. Il me semble, que cela vaut la peine d'entrer en considération."
"J'en conviens, mais. . . . . . . .
"Mais, il y a quelque chose en vous qui résiste à la persuasion; et cela ébranle beaucoup la mienne. Cependant comment se résoudre à laisser partir Tom? Il est si peu difficile sur le choix de ses amis. Ce Charles Madox n'est peut-être point un homme fait pour être admis dans notre société. J'aurois cru que vous partageriez tout-à-fait mon sentiment par rapport à miss Crawford. Elle fut si aimable pour vous hier au soir. Je lui en ai su un gré infini."
"Il est sûr que votre détermination lui fera\setcounter{page}{123} un très-grand plaisir , dit Fanny en s'efforçant d'accompagner ces mots d'une expression de consentement. Elle auroit voulu en dire davantage mais elle ne savoit feindre , dans aucune nuance , et Edmond étoit trop occupé de son idée pour s'apercevoir de ce qui manquoit à l'approbation de Fanny.
Je vais donc au Presbytère , continuat-il , d'abord après déjeûner, puisque vous trouvez que je ne puis pas faire autrement. De deux maux il fa?? choisir le moindre. Cette idée m'a tenu éveillé toute la nuit, mais j'avois besoin que vous m'approuvassiez. Je vous ai ennuyée bien long-temps de mon indécision. Vous allez reprendre votre lecture qui vaudra beaucoup mieux. Je vous envie ce joli petit établissement ; vous êtes bien tranquille ici , mais je crains que vous n'y ayez froid. N'allez pas vous enrhumer.
Fanny étoit bien loin de la disposition qui lui auroit permis de lire avec intérêt. Edmond venoit de bouleverser toutes ses idées. Elle voyoit dans chaque mot qu'il avoit prononcé l'influence de miss Crawford. Ce qui la concernoit elle-même ne l'occupoit guères. Tout lui étoit désagréable dans la perspective de cette comédie, et ce n'étoit pas son amour-propre qui pouvoit en souffrir le plus.\setcounter{page}{124} Ainsi que Fanny l'avoit prévu, la résolution d'Emond fut un sujet de triomphe pour son frère et pour Marie. Ils se conduisirent pourtant assez bien, et ne trahirent leur joie que par quelques sourires et des coups d'œil à la dérobée. Marie sur-tout, qui devinoit le secret de ce changement, rioit sous cape en voyant le grave Edmond entraîné, en dépit de tous ses raisonnemens, par l'influence d'un joli visage hors de la route qu'il s'étoit prescrite.
On feignit de sentir comme lui les inconvéniens qui auroient résultés de l'admission d'un étranger dans une réunion d'amis, et tout le monde fut de bonne humeur. Mistriss Norris offrit ses services pour le costume, Mr. Yates assura que le rôle d'Anhalt étoit susceptible d'un grand déploiement d'énergie, et Ruthworth se mit à compter combien de fois Edmond devoit prendre la parole.
Le plaisir que causa la nouvelle de son consentement ne fut pas moins vif au Presbytère qu'au Château. Miss Flora l'en récompensa par un redoublement d'amabilité pour lui ; et le désir d'entrer dans les convenances de chacun lui fit inventer d'engager mistriss Grant à se charger du rôle que Fanny avoit refusé. Celle-ci auroit été parfaitement\setcounter{page}{125} contente de cet arrangement si ce n'avait été pour Edmond une nouvelle occasion de louer avec chaleur l'obligeance et l'esprit conciliant de miss Crawford. Pendant toute cette journée, Fanny vit avec une espèce d'envie la gaieté générale, l'activité des préparatifs, l'importance que chacun acquéroit aux yeux de tous par le besoin réciproque de s'entendre. Elle seule restoit en dehors de cet intérêt. On n'avoit rien à lui dire. On ne s'aperecevoit pas de son absence. S'il y avoit eu de la sympathie entr'elle et Julia, elles se seroient rapprochées dans cette occasion où les circonstances devoient mettre quelqu'analogie dans leurs sentimens, mais il n'y en avoit aucun dans leurs caractères. Julia qui se croyoit trahie avoit le cœur profondément ulcéré. Elle éprouvoit des mouvemens de haine dont l'ame douce de Fanny n'étoit point susceptible.
Quelques jours se passèrent encore dans l'agitation des préparatifs, et Fanny s'aperçut bientôt que tout n'étoit pas plaisir, même pour ceux qui paroissoient avoir atteint le but de tous leurs vœux.
Malgré les représentations d'Edmond, son frère avoit fait venir de la ville un peintre de décoration. Cette fantaisie devenoit trèscouteuse\setcounter{page}{126} et Tom fut le premier à s'impatienter des retards qui en résultoient.
Chacun des acteurs trouvoit matière à critiquer chez ses confrères. Fanny recevoit toutes les confidences de ce genre. Au dire de Henri Mr. Yates chargeoit ridiculement. Celui-ci trouvoit qu'on avoit beaucoup exagéré le talent de Crawford. Tom débitoit trop vite et bredouilloit au point d'en être inintelligible. Mistriss Grant ne savoit pas garder son sérieux, et le pauvre Rushworth fatiguoit le souffleur plus que tous les autres ensemble : il est vrai que Marie n'avoit jamais le temps de répéter les scènes qu'elle avoit avec lui.
Dans le fait, Fanny fut peut-être la seule qui trouvât dans tout cela plus d'amusement qu'elle n'en avoit attendu. Henri étoit véritablement acteur ; Marie jouoit fort bien son rôle, et Fanny trouvoit même quelquefois qu'elle y mettoit trop d'expression.
On n'avoit encore répété que les deux premiers actes et c'étoit dans le troisième qu'Edmond et Flora avoient ensemble un dialogue fort intéressant dont l'amour et le mariage faisoient le principal sujet. Fanny l'avoit lu plusieurs fois, et le cœur lui battoit en pensant au moment où elle l'entendroit\setcounter{page}{127} sur le théâtre. Le jour où cette répétition devait avoir lieu, Fanny s'était retirée dans sa petite bibliothèque. Elle s'efforçoit d'y faire provision de philosophie et de générosité lorsqu'elle entendit frapper doucement à sa porte. Elle crut que c'était Edmond, mais elle fut surprise de voir Flora.
"Ne suis-je point indiscrète en venant interrompre votre solitude, ma chère miss Price," dit-elle en entrant. Fanny, un peu déconcertée tâcha cependant de recevoir de bonne grâce cette visite inattendue.
"Je viens ici pour vous demander un service d'amie. Auriez-vous la bonté de me faire réciter cette longue scène du troisième acte. Voilà mon livre que j'ai apporté, et vous serez toute aimable de vouloir bien m'entendre. J'avois compté répéter cette scène avec Edmond, ici, à nous deux seulement, mais je ne l'ai point trouvé sur mon chemin, et à présent, j'en suis bien aise, car en vérité j'ai besoin de m'aguerrir un peu à l'avance. Il y a des passages qui sont assez vifs. Connoissez-vous bien la pièce?", dit-elle en ouvrant le livre, "tenez, voyez ceci..... et cela encore...... D'abord je n'y avois pas fait grande attention, mais sur ma parole, c'est un peu fort. Je ne sais vraiment pas\setcounter{page}{128} comment je pourrai lui dire cela en face. En auriez-vous le courage? Au reste, vous êtes sa cousine, c’est tout différent. Il faut que vous vous mettiez là, vis-à-vis de moi. Je me représenterai que c’est Edmond. Cela me familiarisera avec la réalité. – Vous lui ressemblez un peu à votre cousin.
" Moi, vraiment! – Trouvez-vous? Allons je ferai de mon mieux pour le supplier.
" Prenons deux chaises et plaçons nous comme sur le théâtre. C’est ici, dit-on, que s’est faite l’éducation de la famille. Ah! les bonnes chaises d’école! Il me semble que je vois des petites filles qui tout en apprenant leur leçon frappent des talons contre les bâtons de traverse. La prudente institutrice ne croyoit pas qu’elles servissent jamais à figurer sur la scène. Je me représente que sir Thomas arrivât des Indes à l’heure qu’il est, il seroit plus surpris qu’édifié. Il n’y a pas un coin de la maison qui ne résonne de comédie. Yates tempête dans la salle à manger. Henri et Marie sont sur les planches. – Si ceux là ne savent pas leur affaire, il y aura du malheur. – Comme je passois vers la porte avec Rushworth, nous les avons justement trouvés au moment critique, c’é- toit l’endroit où l’on est convenu de ne pas s’embrasser.\setcounter{page}{129} s'embrasser. Le pauvre garçon faisoit une assez triste mine, et moi j'ai vite cherché un prétexte pour l'emmener. A présent, à notre affaire. "
,. Lorsqu'elles eurent commencé la scène, Edmond arriva. Il venoit demander à Fanny le même service que Flora. Cette rencontre parut agréable à tous trois et valut à Fanny des éloges très-animés sur sa complaisance. Cependant le plaisir qu'elle en ressentit fut de courte durée. Le moment de l'épreuve approchoit, et Fanny pensoit qu'elle alloit devenir pour eux un personnage bien insignifiant. Edmond pressa Flora de répéter avec lui. Elle fit quelques façons, mais au fond du cœur elle en mouroit d'envie. Fanny fut priée de faire ses observations, de donner ses conseils, et de critiquer sans ménagemens. Hélas ! elle en étoit si peu capable, que la tâche de souffleur étoit déjà trop forte pour elle. Sa distraction fut mise sur le compte de l'ennui et de la fatigue. On lui fit mille excuses de l'avoir dérangée. Elle répondit par des complimens, et fit bonne contenance jusqu'au bout ; mais il lui en avoit plus coûté qu'elle ne l'avoit cru d'avance ; et plus que personne, du moins elle l'espéroit, ne pouvoit jamais le soupçonner.\setcounter{page}{130} Le soir on illumina le théâtre, et une répétition dans les formes commença. Au milieu d'une déclaration d'amour faite avec beaucoup de naturel par Henri, et lorsqu'il pressoit tendrement la main de Marie contre son cœur, Julia entra avec une physionomie toute renversée, en disant : Voilà papa qui arrive.
Le sentiment d'être pris en faute fut malheureusement pour les enfans de sir Thomas celui qui se manifesta le premier. Ils se regardèrent les uns les autres sans pouvoir prononcer une parole. Henri cependant ne perdit point la tête. Il profita du trouble général pour retenir la main de Marie. Julia qui s'en aperçut, devint aussi rouge qu'elle avoit été pâle un moment auparavant, et sortit en disant : Pour moi du moins, je ne crains pas les regards de mon père. Ses frères la suivirent il n'y avoit pas à hésiter. Marie s'achemina après eux, sans faire attention à Rushworth qui lui avoit déjà demandé dix fois ce qu'il devoit faire. Une seule chose l'occupoit. Elle n'avoit plus de doute sur la passion de Henri; et le déplaisir de son père ne la touchoit que faiblement.
Fanny resta avec les Crawford et Mr. Yates.\setcounter{page}{131} Ses cousins n'avoient pas eu le temps de songer à elle; et son extrême timidité la retenoit. Toute sa sollicitude se portoit sur Edmond. L'idée qu'il avoit encouru le mécontentement de son père, et seroit confondu dans le blâme que les autres avoient mérité, lui faisoit une peine infinie. Elle plaignoit aussi son oncle en pensant au trouble que cet incident mêleroit à des momens qui auroient pu être si doux pour lui et pour sa famille. Tandis qu'elle étoit tremblante d'émotion, les Crawford faisoient leurs doléances sur cette interruption, aussi désagréable qu'inattendue. Mr Yates se flattoit encore que ce ne seroit qu'un retard, mais les Crawford, qui savoient mieux à qui ils avoient affaire, ne doutoient point de la clôture finale du théâtre. Ils se retirèrent sans prendre congé, et proposèrent à Mr. Yates de les accompagner au Presbytère. Celui-ci ne voyoit aucune raison de croire qu'il ne fut pas très-bien venu auprès de sir Thomas, et attendoit seulement pour se faire présenter à lui que la première vivacité des embrassemens fut passée. Fanny un peu remise de son trouble, et craignant qu'une plus longue absence ne fut prise en mauvaise part, se leva pour\setcounter{page}{132} aller rejoindre ses parens. Arrivée à la porte du sallon, elle eut besoin de courage pour l'ouvrir; mais en entrant, elle entendit sir Thomas prononcer son nom avec un accent de bonté qui dissipa ses craintes. "Où donc est ma petite Fanny? ,, disoit-il. "Pourquoi ne la vois-je point parmi vous?" Elle s'avança vers lui, et il l'embrassa tendrement en observant avec plaisir combien elle étoit développée et embellie. Le cœur de Fanny battoit de joie et de reconnoissance, jamais son oncle ne lui avoit témoigné autant d'affection. Il s'informa avec intérêt de ses parens et sur-tout de son frère William. Fanny trouva que son oncle étoit fort changé. Les fatigues d'un long voyage l'avoient vieilli plus encore que le cours des années. Elle éprouva une espèce de remords de n'avoir pas eu jusqu'alors autant d'attachement pour lui qu'elle l'auroit dû, et s'affligeoit en prévoyant qu'il ne trouveroit pas dans son intérieur le repos et les jouissances domestiques dont il étoit privé depuis si long-temps. On se rassembla autour du feu. Sir Thomas se félicita d'avoir trouvé toute sa famille réunie quoiqu'il ne fût point attendu. Rushworth avoit été accueilli par lui comme en faisant déjà partie. Sa figure n'avoit rien que\setcounter{page}{133} déprévenant, en sorte que le premier abord fît tout à fait en sa faveur. Lady Bartram qui ne voyait jamais rien au-delà de ce qu'on voulait lui faire voir, était parfaitement contente et tranquille. Son temps avait été bien employé; elle pouvait montrer à son mari un meuble de tapisserie qu'elle avait fait pendant son absence, et ne doutait point qu'il n'eût également lieu d'être satisfait du compte qu'il recevrait de tout ce qui s'était passé chez lui.
Mistriss Norris n'avait pas une sécurité aussi entière, et quoique sa pénétration n'allât pas bien loin, elle entrevoyait que sur quelques points, son beau-frère pourrait avoir du mécontentement; aussi s'applaudissait-elle beaucoup de la présence d'esprit avec laquelle elle avait fait disparaître le manteau de satin orange qu'elle essayait sur ses épaules, au moment de l'arrivée; mais elle ne pouvait se consoler de n'avoir pas été la première à répandre la nouvelle dans la maison, ni de ce que sir Thomas ne voulait pas dîner, malgré ses instances réitérées, ce qui lui aurait donné le plaisir de mettre la cuisine sans dessus dessous et de harceler les domestiques.
Lorsque l'échange des questions et des réponses\setcounter{page}{134} fut un peu ralenti, lady Barbara qui étoit en train de causer plus qu'elle n'eût fait depuis des années, répandit de nouveau la consternation parmi ses enfans en disant : "Vous ne devineriez pas, mon ami, à quoi toute cette jeunesse s'est amusée dernièrement ; à jouer la comédie. Vous n'avez pas d'idée du mouvement que cela a mis dans la maison. Au reste, je ne m'en suis pas beaucoup mêlée, mais j'ai pourtant dit mon mot par-ci par-là." "Réellement," répondit sir Thomas, "et quelle pièce ont-ils représentée ?" "Moi, je ne me souviens jamais du titre des ouvrages. Dites donc à votre père." "Oh cela n'est point pressé," interrompit Tom d'un air indifférent. "Ce n'est pas le moment d'ennuyer mon père de ces bagatelles. En deux mots voici ce que c'est ; l'automne a été si pluvieuse que nous avons eu l'idée de jouer quelques scènes entre nous pour égayer ma mère et faire passer le temps. Croiriez-vous que depuis le trois de ce mois il n'y a pas eu un jour où j'aie pu prendre mon fusil ; et même jusqu'alors nous n'avions pas fait grand chose, car nous avons voulu avant tout, vous ménager le plaisir de la chasse ; le gibier a été respecté,\setcounter{page}{135} et vous trouverez vos bois pour le moins aussi bien peuplés qu'ils l'étoient avant votre départ."
Tom réussit de cette manière à détourner la conversation, mais ce ne pouvoit être qu'un répit, car bientôt après sir Thomas se leva en disant qu'il se faisoit un vrai plaisir de retrouver sa chambre à coucher et toutes ses anciennes habitudes. A ces mots les alarmes se renouvelèrent. Sir Thomas sortit du salon sans que personne eût le courage de le suivre. Edmond fut le premier qui se mit en mouvement; et Tom se rappelant que Yates étoit resté seul sur le théâtre, courut le chercher, comptant un peu sur la volubilité de son éloquence pour soutenir le premier choc de la surprise. Il arriva tout à point pour voir la rencontre de son père et de son ami. Le premier avoit été attiré dans la salle du spectacle par la déclamation bruyante de Yates qui se démenoit en long et en large sur le théâtre, pour donner le change à sa mauvaise humeur.
Un spectateur indifférent auroit pu rire de la scène qui frappa les regards de Tom.\footnote{Une scène, telle que celle qui se présentoit.} La figure imposante de sir Thomas, produisit sur Yates la métamorphose la plus\setcounter{page}{136} subite et la plus comique. L'emphase du rôle factice fit place à la politesse un peu gauche d'un homme qui se sent embarrassé de sa position, et qui s'indigne de pouvoir être déconcerté par quoique ce soit.
Tom cependant étoit moins disposé à plaisanter qu'à l'ordinaire. Il s'avança et présenta Yates comme son ami particulier. Sir Thomas fut peu satisfait de voir ce nouveau personnage ajouté à la liste nombreuse des amis particuliers de son fils; mais il avoit trop d'usage du monde pour laisser voir son mécontentement. D'ailleurs, il ne vouloit pas que, ce jour, qui le rendoit à sa famille, fût marqué par des témoignages de désapprobation. Il se contint même assez pour écouter un moment le bavardage de Yates, qui avoit repris son assurance accoutumée, et faisoit remarquer à Sir Thomas le goût et la convenance avec lesquels l'arrangement du théâtre avoit été exécuté. Il ne s'en tint pas là, et accompagnant Sir Thomas au sallon, il reprit le sujet que celui-ci auroit voulu écarter, et dit précisément tout ce qu'il auroit fallu taire. Les regards de Sir Thomas, qui se portoient alternativement sur ses fils et ses filles, disoient assez ce qu'il pensoit. Fanny se tenoit à l'écart, mais elle n'interprêtait\setcounter{page}{137} terprétoit que trop bien ce langage. Il lui sembloit que les reproches de son oncle s'adressoient plus particulièrement à Edmond, comme à celui dans la sagesse duquel il avoit mis toute sa confiance. Yates ne s'apercevoit de rien. On avoit beau lui faire des signes, tousser, essayer de l'interrompre, il alloit toujours son train. Après avoir raconté l'origine et les progrès de ce plan de comédie, il ajouta: "Le fait est, monsieur, que nous étions sur le point de faire une répétition générale lorsque vous êtes arrivé. Le commencement alloit bien et promettoit beaucoup. La dispersion d'une partie des acteurs empêche que nous ne reprenions la chose pour aujourd'hui, mais demain, si vous nous faites l'honneur d'assister à notre représentation, nous serons tous animés d'un nouveau zèle. Cependant nous réclamons votre indulgence, car nous n'en sommes qu'à notre coup d'essai." "Vous me trouverez fort disposé à accorder mon indulgence, répondit Sir Thomas, avec un calme sérieux; mais c'est à condition qu'il ne sera plus question de comédie." Il ajouta d'un ton plus doux: "je reviens chez moi pour être heureux, et on ne l'est point sans indulgence."\setcounter{page}{138} Le lendemain, Edmond eut une conversation avec son père, qui rétablit entre eux la confiance la plus intime, mais Sir Thomas eut lieu de se convaincre toujours davantage des inconvénients graves qu'entraîne le genre d'amusement que ses enfants avoient adopté; sur-tout en égard à la position de Marie.
"Nous avons tous participé plus ou moins," dit Edmond, "au tort que vous nous reprochez. Fanny seule en est absolument exempte. Son jugement et sa conduite ont été toujours d'accord. Elle ne trompera point vos espérances, mon père, et vous trouverez en elle tout ce que vous pouvez désirer pour l'enfant de votre adoption."
Sir Thomas n'essaya pas de raisonner avec les autres sur un sujet dont il étoit plus impatient que personne d'effacer le souvenir, et fit mettre tout de suite la main à l'œuvre pour rétablir chaque chose à sa place dans la maison. Il se flatta que cette leçon indirecte leur suffiroit, et crut plus prudent de ne pas mettre leur sincérité à l'épreuve en cherchant à approfondir les choses.
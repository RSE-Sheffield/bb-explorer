\setcounter{page}{379}
\section{ROMANS \large{MANSFIELD-PARK.} \small{( Quatriéme extrait. Voy. p. 243 de ce vol. )}}
(FLORA continue à ménager adroitement Edmond, qu’elle aime au fond du cœur plus qu’elle ne s’en doute elle-même, et se flatte toujours qu’il lui fera le sacrifice d’une vocation qui ne peut s’accorder avec les plans de vie qu’elle a formés. Edmond, attiré par le charme de l’esprit et d’un naturel aimable, craint que ces heureux dons ne soient gâtés par l’effet d’une mauvaise éducation et de l’absence des principes religieux.
Il ne peut se déterminer à faire une démarche qui l’engageroit irrévocablement et cherche à s’appuyer du jugement de Fanny, à qui il fait des demi confidences, dont elle voudroit bien être dispensée. Enfin Crawford et sa sœur quittent Mansfield pour un temps sans avoir renoncé à leurs espérances et à leurs projets.
Sir Thomas regrette vivement pour sa nièce l’établissement qui lui étoit offert; il\setcounter{page}{380} imagine qu'un peu de séjour dans la maison paternelle pourra servir de correctif à cette disposition romanesque à laquelle il attribue son refus. Il pense que la privation des douceurs de la vie dont elle a joui chez lui pendant plusieurs années, lui fera mieux apprécier les avantages de la fortune, et lui donnera la prévoyance qui lui manque.
Fanny accepte avec joie et reconnaissance la proposition que lui fait son oncle de l'envoyer passer deux mois à Portsmouth chez ses parents; et son frère qui doit s'y embarquer l'accompagne).
Le mouvement du voyage, plaisir nouveau pour Fanny, et le bonheur d'être avec William, adoucirent les regrets du départ. Lorsqu'à la première poste, il fallut renvoyer l'équipage de sir Thomas, Fanny fit ses adieux au bon vieux cocher, et le chargea de ses souvenirs pour toute la famille sans trop d'attendrissement.
Pour William, il était d'une gaieté folle, il trouvait partout des sujets de plaisanterie, et l'on peut croire que la conversation entre le frère et la sœur ne manquait pas non plus d'objets d'intérêt. Celui sur lequel William revenait avec le plus de complaisance c'était son admiration pour la Sally, bâtiment sur lequel il devait faire sa première\setcounter{page}{381} expédition, comme sous-lieutenant. Il la voyoit déjà engagée dans quelque action glorieuse, à la suite de laquelle il obtenoit de l'avancement; il avoit sa part dans quelque bonne prise et l'employoit à acheter une jolie petite chaumière, où Fanny et lui finiroient doucement leurs jours ensemble. Il savoit qu'elle avoit refusé la main de Crawford; mais quoiqu'il déplorât son indifférence pour un homme dont sa reconnoissance lui exageroit le mérite, il étoit trop ami de la liberté pour ne pas approuver l'usage que sa sœur en avoit fait, et il s'étoit même abstenu de lui en parler.
Fanny, cependant, ne pouvoit pas se flatter que Crawford eût renoncé à ses espérances. Elle ne recevoit pas une lettre de Flora où il n'y eût quelques lignes de lui qui trahissoient sa confiance dans l'avenir. Cette correspondance, ainsi que Fanny l'avoit prévu, étoit souvent une occasion de peine, indépendamment de la facilité qu'elle donnoit à Henri pour renouveler ses protestations de constance; car Flora avoit un style charmant; elle exprimait son attachement pour Fanny de la manière la plus aimable, et Edmond trouvoit toujours moyen de se faire lire ces lettres; il falloit qu'elle vît le plaisir avec lequel il écoutoit, qu'elle entendit l'éloge\setcounter{page}{382} mille fois répété de son esprit, de sa sensibilité, de sa grâce. Fanny comprenoit bien que Flora ne faisoit pas tant de frais pour elle seule, et cette idée lui faisoit espérer que lorsqu'elle ne seroit plus à Mansfield la correspondance seroit moins vive.
Les jeunes voyageurs firent leur route sans accident et arrivèrent à Portsmouth à la fin du second jour. Fanny étoit fort émue; un mélange confus de souvenirs, d'espérances, de doutes, l'agitoient intérieurement. La voiture s'arrêta dans une rue étroite. Une servante qui attendoit à la porte de la maison où logeoit la famille Price, s'avança avec empressement, et au lieu d'aider les voyageurs à descendre, elle se hâta de dire à William que la Sally étoit sortie du port, et qu'un des officiers de l'équipage étoit venu le chercher. À peine eut-elle le temps de faire sa commission, qu'un jeune garçon de onze ans, accourut tout essoufflé et la poussant de côté pour se faire place, il s'écria. "Te voilà bien à propos. Il y a une heure que nous t'attendons. Sais-tu que la Sally a été lancée ce matin; on ne peut rien voir de plus beau. Mr. Campbell est venu ici à quatre heures te demander. Il va joindre la Sally ce soir, et il espéroit que tu serois arrivé pour aller avec lui; on croit qu'on aura les ordres\setcounter{page}{383} demain ou après demain. Pendant tout ce discours, William descendit de voiture et donna la main à Fanny. Elle embrassa son jeune frère, qui jusqu'à ce moment avait été trop occupé de son idée pour faire grande attention à elle. Il faut dire à sa justification, que c'était pour lui un objet du premier intérêt, car il devait accompagner William comme matelot. Après avoir traversé une allée étroite, Fanny se trouva dans les bras de sa mère, qui l'accueillit fort tendrement, et dans les traits de laquelle il lui fut doux de revoir l'image de sa tante Bartram.
Susanne et Betty ses deux sœurs se précipitèrent à sa rencontre. La première était une jolie personne de quatorze ans, grande et bien faite, l'autre un enfant de cinq ans, la cadette de la famille.
On la fit entrer dans une chambre si petite, qu'elle ne crut point d'abord que ce pût être là qu'on se réunissoit, mais elle s'applaudit de n'avoir pas manifesté son doute, lorsqu'elle vit qu'on était en effet dans le salon de compagnie. Mad. Price n'avait au reste pas le temps d'observer les impressions de sa fille. Après le premier moment donné à Fanny, elle retourna à la porte de la rue pour recevoir William, et lui répéter l'histoire de la Sally, qu'il avait déjà entendue\setcounter{page}{384} deux fois," ajouta-t-elle, "trois jours plus tôt que nous n'avions compté. Je suis dans un souci terrible pour la pacotille de ton frère Samuel: Jamais tout ne sera prêt à temps, car les ordres peuvent arriver demain. Cette idée me bouleverse. Et ne faut-il pas que tu te rendes à Spithead aussi? – Moi qui comptois que nous passerions notre veillée si tranquillement! – et puis tout arrive à-la-fois.,, William avec sa gaieté ordinaire, tâcha d'écarter l'idée de ce qui pouvait troubler le plaisir de la réunion, et ramena sa mère auprès de Fanny, en disant. "Venez, venez! vous n'avez presque pas vu cette chère petite. "
Ils rentrèrent ensemble, et Mad. Price, après avoir embrassé de nouveau sa fille et témoigné le plaisir de la revoir, commença à s'occuper des moyens de réparer les fatigues du voyage. "Pauvres enfants!" dit-elle. "Vous devez être bien las. Que voulez-vous prendre? Si l'on avait su le moment de votre arrivée, on aurait préparé quelque chose ; mais, comme je disois à Susanne, qui sait ce qu'ils aimeront le mieux, de manger un morceau ou de boire du thé? Enfin à présent, vous n'avez qu'à dire ; on fera comme on pourra. Nous sommes mal dans ce quar\setcounter{page}{385} tier-ci, parce que nous n'avons rien à portée... Ils assurèrent tous deux, qu'ils préféreraient beaucoup le thé à toute autre chose. " Eh bien ce sera plus tôt fait. Cours vite, Betty mon enfant, dis à Rebecca de mettre chauffer l'eau et de tout préparer. C'est dommage que la sonnette soit gâtée, mais Betty est une bonne petite messagère. "
Betty alla et revint lestement, toute glorieuse d'être traitée comme une grande fille, devant cette belle demoiselle qui étoit sa sœur.
" Quel pauvre feu nous avons là," continua Mad. Price. " Je gage que vous êtes transis de froid. A quoi Rebecca s'est-elle donc amusée? moi qui lui avois tant recommandé de mettre du charbon dans la grille! — Susanne, ma fille, tu aurois dû y faire attention. "
" Mais, maman, est-ce que je peux être partout?" dit Susanne."Vous savez bien qu'il a fallu que je transportasse mes effets dans l'autre chambre pour faire place à ma sœur. Rebecca n'a pas voulu me donner seulement un coup de main. "
La discussion fut interrompue par l'arrivée du cocher, qui venoit se faire payer; ensuite il y eut une dispute entre Samuel\setcounter{page}{386} et Rebecca sur la manière de transporter la malle de Fanny au second étage; enfin on entendit la voix de Mr. Price, qui se plaignoit, dans un langage fort énergique, de trouver le bagage des voyageurs dans l'entrée étroite et obscure de l'escalier, et demandoit en vain qu'on apportât de la lumière. Fanny se leva pour aller à sa rencontre, mais on ne voyoit pas clair, et elle se rassit tristement en attendant que son père eût le temps de penser à elle. Il fallut entendre pour la troisième fois toute l'histoire de la Sally, entremêlée de beaucoup d'exclamations, pour ne pas dire de juremens à la façon des marins. William attendoit impatiemment que son père eût fini de parler pour lui présenter Fanny, qu'il paroissoit avoir complétement oubliée. Il l'embrassa cordialement en disant. "La voilà, ma foi, grande, comme père et mère; dans quatre jours, il faudra déjà penser à la marier." Ensuite il recommença la conversation avec son fils et ne s'occupa pas plus de Fanny que si elle ne les eût jamais quittés.
Il se passa près d'un quart d'heure avant qu'il fût possible d'obtenir de la complaisance de Rebecca qu'elle apportât une chandelle; et comme l'espérance d'avoir du thé paroissoit encore assez éloignée, William se décida à monter dans\setcounter{page}{387} sa chambre et à faire tous ses préparatifs pour revenir ensuite boire le thé tout à son aise avec Fanny. Au moment où il sortoit, deux petits polissons de huit et neuf ans, à faces colorées, se précipitèrent dans la chambre, pour voir leur nouvelle sœur et annoncer que la Sally étoit en rade. Ils revenoient de l’école, avec des habits troués et des souliers couverts de boue. Tommy, le plus jeune de ces deux enfans, étoit né depuis le départ de Fanny; mais elle avoit vu Charles tout petit et se souvenoit d’avoir souvent aidé sa mère à le soigner; celui-ci avoit un degré d’intérêt de plus à ses yeux; elle auroit voulu le retenir quelques momens pour le bien regarder, mais Charles n’étoit pas d’humeur à se laisser caresser, et après le premier embrassement, il s’échappa pour courir et faire du tapage dans la maison; aussi, la pauvre Fanny, qui n’étoit pas accoutumée à un tel vacarme, en eut bientôt la tête rompue.
Elle avoit vu alors toute la famille, excepté deux frères, qui étoient pour l’âge entre’elle et Susanne. L’un étoit commis dans un bureau du gouvernement, et l’autre contre-maître sur un vaisseau de la Compagnie.
Au tintamarre que faisoient Charles et Tom, se joignit bientôt une confusion de\setcounter{page}{388} voix qui se répondoient d'un étage à l'autre. William se plaignoit de ne pas retrouver un de ses effets à la place où il l'avoit laissé; il accusoit Rebecca de ce désordre, et Rebecca rejettoit la faute sur Betty. Mad. Price avoit oublié une réparation essentielle à l'habit de William, et s'en excusoit sur le trouble occasionné par le prochain départ de Samuel. Tous parloient à-la-fois et crioient à tue-tête. Ce logement étoit si petit, et les planchers si minces, qu'on ne perdoit pas une parole de ce qui se disoit d'un bout de la maison à l'autre.
Fanny étoit restée avec son père, mais elle avoit tout le loisir de se livrer à ses réflexions, car il lisoit une gazette, et s'étoit emparé de la seule lumière qu'il y eût dans la chambre. Elle se retrouvoit dans cette maison paternelle, qu'elle avoit tant désiré de revoir, au milieu de tous ceux qu'elle aimoit sans les connoître, et elle n'éprouvoit point le bonheur qu'elle s'étoit promise de cette réunion. Sa présence ne sembloit pas y avoir apporté la joie dont elle s'étoit flattée. Il étoit bien naturel que les intérêts de William l'emportassent de beaucoup sur les siens dans le cœur de ses parens: elle en avoit été séparée depuis sa première enfance, et l'affection ne se nourrit que par les soins réciproques.\setcounter{page}{389} Elle se reprochoit de n'être pas raisonnable sur ce point; mais si du moins on lui avoit parlé de Mansfield! si l'on s'étoit seulement informé de ces chers amis qui avoient tant de droits à la reconnaissance de toute la famille! Peut-être se hâtoit-elle trop de juger. Dans ce moment, un seul objet absorboit toute la sollicitude de ses parens. Le tête-à-tête silencieux de Fanny et de son père étoit interrompu de temps en temps par les brusques entrées des enfans, qui se poursuivoient en renversant tout sur leur passage. Alors Mr. Price impatienté, quittoit sa lecture pour gronder et menacer, mais les petits drôles paroissoient accoutumés à ce langage et n'en tenoient compte. Lorsqu'ils furent las de courir et de sauter, ils vinrent s'asseoir, mais les espiégleries continuèrent sourdement, et la présence de leur père n'empêchoit pas qu'il n'y eût même par momens, des explosions bruyantes de gaieté ou de malice. Enfin l'arrivée du thé, dont Fanny commençoit à désespérer, vint faire une agréable diversion. Susanne, tout en se donnant du mouvement pour faire bouillir l'eau et préparer les beurrées, cherchoit à deviner si Fanny, accoutumée aux usages d'une maison opulente, ne trouvoit point ces petits\setcounter{page}{390} offices un peu trop bourgeois. Cependant elle étoit bien aise de montrer son adresse et son activité." Si je n'avois pas été moi-même à la cuisine," dit-elle, "je crois que nous aurions attendu le thé jusqu'à demain. Rebecca ne fait jamais rien qu'on ne la talonne. Ma pauvre sœur doit avoir pourtant bien besoin de prendre quelque chose."
Le calme se rétablit ainsi peu-à-peu dans la maison. Mad. Price et Betty vinrent s'asseoir auprès de la table à thé. William rentra avec son uniforme de lieutenant. Sa figure étoit plus noble et plus gracieuse encore sous ce nouveau costume. Fanny se leva toute émue; elle le contempla un moment en silence, puis jetant les bras autour de son cou avec une effusion de sentimens mêlés de plaisir et de peine, elle pleura sans pouvoir s'en empêcher, mais bientôt craignant de paroître triste, elle essuya ses larmes, et s'amusa à regarder en détail les différens ornemens de l'habit de son frère. La promesse qu'il fit de revenir tous les jours à terre, jusqu'à-ce que la Sally mît à la voile; et de conduire ses sœurs à Spithead avant son départ, ramena la gaieté générale.
Peu après, Mr. Campbell, chirurgien de l'équipage, et ami de William, vint le chercher. Graces à l'empressement de Susanne,\setcounter{page}{391} il y eut moyen d'avoir une chaise et une tasse de thé pour Mr. Campbell. La conversation de ces messieurs s'anima et devint bruyante. Le moment du départ arriva; William fit ses adieux; tout le monde se mit en mouvement, car les trois petits garçons voulurent accompagner leur frère, malgré les objections de Mad. Price, et leur père sortit en même temps pour aller rendre la gazette à un voisin.
Fanny se flattoit qu'un peu de repos alloit succéder à tant d'agitation, mais la table à thé ne put être débarrassée qu'après que Rebecca en eut été sollicitée à plusieurs reprises; et Mad. Price parcouroit la chambre en cherchant une manche de chemise que Betty découvrit enfin dans un tiroir de la cuisine, ensorte que rien ne paroissoit plus devoir troubler la tranquillité de la soirée. Mistriss Price eut alors le loisir de s'occuper de sa fille aînée, et de la questionner sur ses parens de Mansfield-Park. Un des premiers points sur lesquels sa curiosité s'exerça fut le gouvernement des domestiques. C'étoit le sujet éternel de ses doléances, et ce chapitre une fois entamé, il y en eut pour long-temps, car il la ramena sur ses griefs particuliers à l'égard de Rebecca. Susanne et Betty eurent\setcounter{page}{392} chacune leur mot à dire, en sorte que Fanny, tout en soupçonnant que les torts n’étoient pas entièrement du côté de la servante, s’étonnoit qu’on pût garder une personne qui n’avoit aucune des qualités d’un bon domestique.
En regardant Betty, Fanny se rappela une autre petite sœur qu’elle avoit laissée à-peu-près au même âge, et qui étoit morte depuis son départ. Ce souvenir lui étoit encore fort sensible, et elle auroit craint de rien dire qui pût réveiller l’idée de cette perte. Tandis qu’elle en étoit occupée, Betty lui montra de loin quelque chose qu’elle tenoit dans une main, en se cachant de Susanne avec l’autre, mais elle l’aperçut et s’élança vers la petite pour lui arracher ce qu’elle réclamoit comme lui appartenant. C’étoit un petit couteau d’argent. Betty se réfugia auprès de sa mère, qui la protégeoit toujours à tort et à droit contre Susanne. Celle-ci fut réduite à faire valoir ses droits, en disant que sa petite sœur Marie lui avoit fait présent de ce couteau à son lit de mort, et qu’il étoit bien dur qu’on ne lui permît pas de s’en servir, tandis que Betty trouvoit toujours moyen de le prendre, et qu’elle ne manqueroit pas de le gâter. Cette dispute\setcounter{page}{393} ainsi que le ton de Susanne, firent une impression doublement pénible sur Fanny; "mon Dieu! Susanne," dit Mad. Price, "que tu es ennuyeuse. N'aurez-vous jamais fini de vous chamailler pour ce malheureux couteau? Aussi pourquoi vas-tu le toucher, Betty? Tu sais bien que ta sœur gronde toujours. Je vois qu'il faudra que je le cache tout-à-fait, et alors vous ne l'aurez ni l'une ni l'autre. La pauvre Marie ne savoit guères quelle pomme de discorde elle jetoit entre vous deux, lorsqu'elle me le donna à garder deux heures avant d'expirer. — Pauvre chère ame! Elle aimoit tant ce petit couteau, qu'elle vouloit l'avoir sous son chevet pendant tout le temps de sa maladie. A peine pouvoit-on encore l'entendre, lorsqu'elle me dit: il faut donner mon couteau à ma sœur Susanne, pour qu'elle se souvienne toujours de moi. — Hélas, mon Dieu! ce fut sa dernière parole; mais je dis toujours quelle est bien heureuse d'avoir été retirée de ce monde. C'étoit sa marraine, fene Mad. l'amirale Maxwell, qui le lui avoit donné, ce couteau. — Pour toi, ma pauvre Betty, tu n'as pas le bonheur d'avoir une aussi bonne marraine. La tante Norris ne pense guères à toi."
Fanny n'avoit en effet apporté de sa part\setcounter{page}{394} à sa filleule que des complimens, et la recommandation d'être bien sage. Il est vrai qu'il avoit été question un moment de lui envoyer un vieux livre de prières, mais sur un plus mûr examen de la chose, ce mouvement de générosité s'étoit ralenti.
Fanny commençoit à avoir un besoin pressant de repos. La proposition que lui fit Susanne de la conduire dans sa chambre à coucher, lui fut très-agréable; et tandis que Betty pleuroit pour qu'on lui permît de veiller encore un peu, elle s'échappa avec Susanne. Au même moment ses trois frères rentrant a-la-fois, mirent tout en rumeur de nouveau en demandant du pain et du fromage. Mr. Price de son côté appeloit Rebecca pour avoir un verre de rum, mais Rebecca étoit partout ailleurs que là où elle devoit être.
Si Fanny avoit pu rendre un compte bien sincère de ses impressions dans la première lettre qu'elle écrivît à sa tante Bertram le lendemain de son arrivée à Portsmouth, sir Thomas n'auroit pas cru la cause de Henri tout-à-fait désespérée, et il se seroit applaudi de sa propre sagacité. Cependant une nuit de repos, une belle matinée, l'absence des trois petits garçons, et l'espérance de revoir bientôt William avoient donné à Fanny une\setcounter{page}{395} disposition d'esprit très-différente de celle de la veille.
Les jours suivants se passèrent tristement, et avant la fin de la semaine, William étoit parti sans qu'aucun des projets qu'ils avoient faits pour adoucir le chagrin de la séparation, eût pu se réaliser. Non-seulement William n'avoit point mené ses sœurs à Spithead, mais il n'avoit pu venir à terre que deux fois, et d'une manière si fugitive que Fanny n'avoit pas eu un moment pour causer avec lui. Elle avoit eu à décompter presque sur tous les points qui lui tenoient le plus au cœur, excepté l'attachement de ce bon frère dont la dernière pensée en quittant ses parens avoit été pour elle. Il revint même sur ses pas pour dire encore: "Je vous recommande notre chère Fanny, ma mère; elle est fort délicate, elle a besoin de plus de ménagemens que nous autres ; ayez en bien soin je vous en prie."
Fanny ne pouvoit pas se dissimuler que beaucoup de choses dans la maison de ses parens étoient fort différentes de ce qu'elle auroit désiré. Le désordre et le bruit y régnoient constamment. La subordination y étoit souvent méconnue. Rien ne s'y faisoit à propos et avec soin. Fanny savoit bien\setcounter{page}{396} que la profession de son père ne lui avoit pas permis d'acquérir les avantages d'un esprit cultivé, ni les manières d'un homme du monde, mais il étoit plus grossier, plus négligent sur les devoirs de père, moins instruit que ne le comportoient les circonstances où il avoit vécu. Il n'avoit d'autre conversation que celle qui avoit rapport à son métier, d'autre lecture que les papiers publics et l'almanach de la marine, d'autre plaisir que de fumer et de boire.
Elle sentoit qu'elle ne feroit aucun progrès dans son affection. Lorsqu'il lui adressait la parole, ce qui lui arrivoit rarement, c'étoit pour lui faire quelque plaisanterie de mauvais goût.
Fanny avoit éprouvé un mécompte bien plus sensible encore par rapport à sa mère, car elle s'étoit promise de grandes jouissances dans cette nouvelle relation, et chaque jour détruisoit quelqu'une de ses espérances. L'instinct maternel de Mad. Price s'étoit réveillé au premier moment, mais comme sa tendresse n'avoit pas d'autre source, elle s'étoit bientôt refroidie. Ses affections habituelles et ses nombreuses occupations avoient repris leurs cours et absorboient toutes ses facultés. D'ailleurs ses filles n'avoient jamais\setcounter{page}{397} en dans son cœur la même part que ses fils. Betty étoit la seule pour laquelle elle eut du foible. William flattoit son orgueil maternel, les autres occupoient toute sa sollicitude, et faisoient sa joie ou son tourment.
Ses journées se passoient dans un mouvement continuel, et pourtant elle restoit en arrière sur mille objets. Elle sentoit la nécessité d'épargner, mais elle étoit mauvaise économe parce qu'elle manquoit de régularité et de jugement. Sans cesse mécontente de ses domestiques, elle ne savoit point les diriger, et soit qu'elle les reprît ou qu'elle supportât leurs défauts, elle n'obtenoit d'eux ni égards ni respect. Si elle eût été appelée à mener la vie d'une dame de château, elle auroit représenté au moins aussi bien que lady Bertram avec laquelle elle avoit beaucoup de rapport, mais elle n'avoit point naturellement les goûts et la capacité qu'exigeoient la situation où un mariage imprudent l'avoit jetée, et Mad. Norris auroit été à sa place une mère de famille plus respectable.
Fanny désiroit fort se rendre utile à sa mère et contribuer pour sa part à l'avantage commun de la famille. Elle auroit été très-fâchée de laisser croire que son éducation\setcounter{page}{398} ne l'avoit pas rendue propre à tous les soins dont s'occupoient sa mère et sa sœur. Elle se mit à travailler fort assidûment pour le trousseau de Samuel. Elle se levoit matin et se couchoit tard, en sorte qu'avec son aide tout fut à-peu-près achevé; mais elle ne comprenoit pas comment on auroit fait sans elle.
Quoique Samuel fût assez indiscipliné et peu sociable, elle le regretta, car il avoit plusieurs bonnes qualités. Il étoit susceptible de gagner beaucoup par un traitement raisonnable. Susanne ne manquoit pas de sens, mais ses avertissemens n'étoient pas toujours donnés à propos, ni appuyés d'une autorité suffisante. Fanny avoit déjà obtenu davantage en employant avec tact le raisonnement et la persuasion. En revanche ses tentatives pour apprivoiser un peu Tom et Charles furent si parfaitement inutiles qu'elle perdit le courage de continuer cette tâche. Heureusement pour la paix intérieure, ils passoient la plus grande partie du jour à l'école, mais l'arrivée du dimanche lui causoit un véritable effroi. Elle n'avoit guères plus de succès avec la petite Betty. C'étoit un enfant gâté dans toute l'étendue du terme; elle ne haïssoit rien tant que son alphabet;\setcounter{page}{399} son séjour favori était celui de la cuisine, ce qui n'empêchoit pas qu'elle ne se plût à rapporter à sa mère tout ce qui s'y faisoit de mal.
Quant à Susanne, Fanny ne savoit pas bien encore qu'en penser. Son ton et ses procédés avec sa mère étoient en général très peu convenables, quoiqu'il fût vrai que la raison étoit souvent de son côté. Elle manquoit essentiellement de douceur, et Fanny dont le caractère et les habitudes étoient si opposés à ce défaut, ne pouvoit lui pardonner ses violences avec ses frères et sa petite sœur, tout en convenant que sa patience étoit mise à de terribles épreuves.
Fanny avoit eu raison de croire que miss Crawford mettroit moins de vivacité dans sa correspondance lorsqu'Edmond ne seroit plus à portée de lire ses lettres ; mais ce qu'elle n'avoit pas prévu, c'est que son changement de situation les lui feroit désirer. La privation de tout plaisir de société, les rapports qu'elle avoit si long-temps entretenus avec Flora, et qui étoient liés à ses plus chers intérêts, donnoient un prix tout nouveau à cette relation, dans l'espèce d'exil où elle vivoit. Elle eut donc un mouvement de joie très-vif en voyant arriver une lettre de Flora et se seroit\setcounter{page}{400} abonnées volontiers à en recevoir une semblable chaque semaine, quoiqu'elle ne lui apprît rien de fort intéressant.
Cependant Fanny trouva bientôt une véritable source de jouissances dans une connaissance plus approfondie de Susanne, et dans la conviction de lui être utile. En l'observant d'une manière suivie, elle vit qu'il y avait de grandes ressources dans la bonté de son cœur et dans la droiture de son jugement.
C'étoit une tâche bien nouvelle pour Fanny que d'avoir à diriger quelqu'un, elle qui avait une opinion si modeste d'elle-même qu'à peine se fioit-elle à ses propres lumières pour ce qui la regardoit, mais Susanne lui montroit de la confiance et un grand désir d'obtenir son approbation. Elle résolut de mettre ces heureuses dispositions à profit pour l'avertir avec amitié de ce qui lui paroissoit repréhensible dans sa conduite et dans ses manières. Susanne cherchoit la vérité de bonne foi, et Fanny avoit la satisfaction de voir souvent le bon effet de ses conseils. Lorsqu'elle considéroit l'éducation que sa sœur avoit reçue et l'ensemble des circonstances, elle admiroit qu'elle pût avoir conservé, au milieu de tant d'erreurs et de négligence, un esprit aussi sain et des intentions aussi droites.\setcounter{page}{401} Peu-à-peu il s'établit entre les deux sœurs une douce intimité et des communications plus étendues. Elles s'arrangèrent pour passer ensemble leurs matinées dans une chambre de l'étage supérieur, où elles étoient assez tranquilles. Susanne avoit de l'esprit naturel, Fanny regrettoit qu'il n'eût point été cultivé. Il n'y avoit pas de livres dans la maison, mais elle pouvoit disposer d'une petite somme que son oncle lui avoit donné en partant. Elle n'avoit encore fait usage de cet argent que pour acheter un joli couteau à Betty, afin qu'elle n'enviât plus celui de sa sœur. Elle se trouva donc assez riche pour s'abonner à un magasin littéraire et commencer avec Susanne un petit cours de lectures bien choisies.
Si Susanne devoit en retirer de l'avantage, il ne convenoit pas moins à Fanny d'avoir une distraction aux souvenirs et aux inquietudes qui auroient pu troubler son repos, car elle avoit reçu une lettre de sa tante, qui disoit Edmond prêt à partir pour Londres. Il avoit annoncé à Fanny qu'il lui écriroit lorsqu'il auroit quelque chose de bien intéressant à lui apprendre. Elle n'avoit que trop compris ce que cela signifioit, et l'heure de la poste n'arrivoit jamais sans lui donner une vive émotion.\setcounter{page}{402} Il y avait une semaine qu'Edmond devait être à Londres, et elle n'avait rien reçu de lui. Elle faisait plusieurs suppositions, qui tour-à-tour paroissient fort probables: ou son départ avait été retardé, ou il n'avait pas encore vu miss Crawford, ou bien peut-être, il étoit dans les transports d'un bonheur trop grand pour pouvoir écrire. (Tom Bartram fait une chute de cheval, dont les suites font craindre une maladie de langueur. Edmond se dévoue à le soigner et ajourne son projet de mariage. Il est d'ailleurs fort ébranlé dans ses résolutions par ce qu'il a vu de miss Crawford à Londres, où tout lui paroît avoir une influence fatale sur sa moralité. Crawford est attiré de nouveau par la coquetterie de Mad. Rushworth, qu'il revoit à Londres, et il l'enlève pendant une absence que fait son mari. Peu de jours après, Julia s'enfuit en Ecosse et se marie avec Mr. Yates. Sir Thomas, frappé à-la-fois de tous les coups les plus sensibles, se rend à Londres pour chercher la trace des fugitifs. Edmond l'y accompagne et va ensuite à Portsmouth, prendre Fanny pour la ramener auprès de lady Bartram. Susanne est invitée à la suivre, au grand contentement des deux sœurs. La tristesse d'Edmond pendant et après le voyage, paroît trop naturelle\setcounter{page}{403} elle à Fanny pour qu'elle en suppose d'autres raisons que celles dont elle a connoissance. Il a cependant un chagrin qu'elle ignore, et ce n'est qu'au bout de quelques jours, qu'il lui en fait la confidence. Il a revu Flora, dans le moment où profondément affligé de l'opprobre dont sa sœur vient de se couvrir, il cherche une ame qui sympathise avec la sienne. Il ne trouve, au contraire, que légéreté, sécheresse, calcul d'intérêt dans la manière dont elle considère cet évènement. Les mots d'extravagance, de sottise, de duperie, sont les seuls qu'elle emploie pour qualifier la conduite de son frère et de Marie. Il voit avec évidence que le respect humain est la seule règle de son opinion, qu'elle redoute les arrêts du monde, et non ceux de la conscience ; et que si la faute avoit pu rester cachée, elle l'auroit trouvée excusable. Edmond est révolté de ce langage. Elle s'en aperçoit, et tâche de raccommoder la chose avec son adresse ordinaire, mais le charme est rompu, et Edmond la quitte, bien déterminé à ne plus la revoir. Cependant il regrette son illusion, plus encore que la perte de ses espérances. Il déplore qu'une femme née avec des dispositions si heureuses, ait\setcounter{page}{404} été gâtée par l'exemple et les maximes d'un monde corrompu).
Quoique le cœur de Fanny fût toujours ouvert à une tendre sympathie pour les peines de ses amis, quoique celles d'Edmond en particulier la touchassent sensiblement, il étoit impossible qu'elle se refusât au sentiment du bonheur qui lui étoit rendu.
Elle se retrouvoit à Mansfield; elle étoit utile aux objets de son affection; elle en étoit aimée. Elle n'avoit plus de persécutions à craindre de la part de Henri; mais quand elle n'auroit pas eu tous ces sujets de satisfaction, une seule chose changeoit son existence: Edmond n'étoit plus subjugué par miss Crawford.
La douleur de Sir Thomas étoit aggravée par de tristes retours sur quelques erreurs dans sa conduite comme père, auxquelles il attribuoit une partie des malheurs de sa famille. Il s'accusoit de foiblesse à l'égard de Mad. Norris, à laquelle il avoit laissé exercer dans sa maison une influence dont il auroit dû prévoir les suites. Il s'étoit accoutumé à elle comme à un mal pour ainsi dire nécessaire; et il s'étoit beaucoup trompé, en croyant que sa propre sévérité, opposée à l'excessive indulgence de sa belle-sœur,\setcounter{page}{405} en corrigeroit le mauvais effet. Le résultat avoit été, que ses enfans, contraints en sa présence, ne lui laissoient pas connoître leur véritable caractère, et que la flatterie de leur tante avoit d'autant plus de prise sur eux.
Sir Thomas se reprochoit aussi d'avoir cédé à des considérations d'un ordre secondaire lorsqu'il avoit permis que sa fille épousât Rushworth, malgré la conviction qu'il avoit acquise de ce qui leur manquoit à tous deux pour former une union solide et honorable.
Il reconnoissoit que l'éducation de ses filles avoit été superficielle. On leur avoit enseigné la théorie de la religion, mais on n'avoit pas mis assez d'importance à la leur rendre utile et chère dans la pratique. Le temps seul pouvoit adoucir l'amertume de ces réflexions par rapport à Marie; mais Sir Thomas trouva dans ses autres enfans des compensations plus grandes qu'il ne l'avoit espéré. Julia, humiliée par le sentiment de sa faute, reconnut le besoin qu'elle avoit du pardon et de l'amitié de ses parens. Son mari étoit un homme fort léger, mais il mit beaucoup de prix à regagner l'estime de son beau-père, et y réussit en se conduisant d'après ses conseils.\setcounter{page}{406} Tom apprit à réfléchir pendant la maladie sérieuse à laquelle il échappa. Il devint meilleur, car la souffrance lui enseigna à sentir pour les autres, et il s'attacha à rendre son existence utile.
Marie parut long-temps inaccessible à tout sentiment de repentir. Elle vécut avec Henri tant qu'elle eut l'espérance de l'épouser; mais lorsqu'elle comprit qu'il falloit y renoncer, son humeur s'aigrit, sa passion changea en haine; la punition des coupables se trouva dans la relation même à laquelle ils avoient tout sacrifié; et ils se séparèrent volontairement.
Mad. Norris, dont l'aveugle partialité pour sa nièce sembloit augmenter à mesure que celle-ci se rendoit moins digne d'estime, intercéda pour obtenir son pardon et la réconcilier avec sa famille; mais Sir Thomas auroit cru donner une espèce de sanction au vice en recevant Marie dans la demeure paternelle après un égarement aussi coupable. Cependant il étoit disposé à faire pour elle tout ce qui pouvoit se concilier avec le respect pour la morale publique, et garantir sa fille de nouveaux dangers en lui donnant les moyens de revenir au bien. Mad. Norris offrit de la suivre dans une province éloignée;\setcounter{page}{407} et en adopta ce parti, quoiqu'il ne fut peut-être pas autant du goût de la nièce que de celui de la tante.
C'est une injustice, sans doute, que, dans nos mœurs, le complice d'un tel désordre ne porte point une part proportionnée du blâme public; mais Henri Crawford étoit de caractère à trouver dans ses propres réflexions un terrible châtiment de sa faute. Le sentiment d'avoir payé d'ingratitude l'hospitalité d'une famille respectable, d'y avoir porté le trouble et la douleur, et perdu ainsi l'espérance d'unir son sort à la personne qui seule pouvoit le rendre heureux, étoit pour lui un supplice cruel.
Après ce qui s'étoit passé, la relation de voisinage entre les familles Grant et Bartram devenoit pénible à soutenir. Les premiers firent une absence prolongée à dessein, et le docteur obtint un bénéfice à Westminster, qui l'éloigna pour toujours de Mansfield. Mad. Grant put alors offrir de nouveau un asyle à sa sœur. Flora en avoit assez de ses amis de Londres, de la vanité, de l'ambition. Le séjour de Mansfield lui avoit donné l'idée du bonheur domestique; et dans le nombre des hommes que tentoient sa jolie figure et ses vingt mille livres sterling,\setcounter{page}{408} aucun ne lui fit oublier Edmond. Celui-ci, à force de répéter à Fanny que jamais il ne trouveroit une autre Flora, vint à découvrir que celle à qui il adressoit les plaintes d'un cœur déçu dans ses espérances, lui conviendroit peut-être beaucoup mieux. Il commença à croire, et réussit aisément à persuader à Fanny, que cette vive amitié de sœur qu'elle lui portoit, seroit une très-bonne base d'amour conjugal.
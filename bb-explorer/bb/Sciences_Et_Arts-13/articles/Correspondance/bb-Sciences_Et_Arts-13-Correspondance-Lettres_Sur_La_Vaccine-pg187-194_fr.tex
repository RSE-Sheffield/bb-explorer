\setcounter{page}{187}
\chapter{CORRESPONDANCE.}
\subsection{LETTRE DU Dr. DE CARRO AUX RÉDACTEURS DE LA BIBLIOTHÈQUE BRITANNIQUE.}
Vienne 1er. Février 1809.
L'ACCUEIL favorable que vous avez fait à mes lettres sur la vaccine, m'engage à vous envoyer la copie de celle que je viens de recevoir du Dr. E. Jenner. Je vous la transcris telle qu'elle est, malgré les éloges trop flatteurs qu'il a bien voulu donner au zèle que j'ai mis à continuer ses recherches.
\setcounter{page}{188}

\subsection{LETTRE du Dʳ JENNER au Dʳ DE CARRO.\footnote{Nous en retranchons quelques détails inutiles. (R)}}
Berkeley 27 Nov. 1799.
"DEPUIS le commencement de mes recherches sur la vaccine, rien ne m'a fait autant de plaisir que votre lettre. Elle respire la candeur d'un vrai philosophe, et assure à son auteur toute mon estime. L'importance du sujet m'avait toujours fait espérer qu'il attirerait l'attention des Médecins du Continent, et je suis charmé de le voir à Vienne entre vos mains. Car j'avais à redouter qu'il n'eût pas tout le succès que j'en attendais, si les premiers observateurs qui s'en occuperaient, manquaient de logique ou d'impartialité."
"L'état des bras de votre malade ne me permet pas de douter que vous ne parveniez facilement, quand vous le voudrez, à reproduire la maladie par le pus séché sur la toile qui les recouvrait. Je vous en envoie cependant du nouveau que je viens de prendre, en date du 1er décembre, sur les bras de deux inoculés qui présentent la plus belle apparence. Pour le garantir autant que possible du contact de l'oxygène, je l'ai laissé sécher entre\setcounter{page}{189} deux verres. Vous pourrez le délayer avec un peu d'eau pour vous en servir."
"Je me propose de republier bientôt mes deux brochures avec un appendix. J'aurai soin de vous les envoyer par la première occasion."
"Si la vaccine est inconnue en Autriche, c'est probablement parce qu'on n'y emploie pas les mêmes domestiques à panser les chevaux et à traire les vaches. Il en est de même en Irlande et en Ecosse, où ce dernier soin n'est jamais confié à des hommes. Si mes conjectures sur l'origine de la maladie sont fondées, il est malheureux que nous ne puissions pas la communiquer directement du cheval à la vache. Mais on n'a pas mieux réussi, que je sache, en l'inoculant d'une vache à une autre vache, c'est-à-dire, en transportant sur la pointe d'une lancette le pus du pis d'une vache malade à celui d'une vache saine. Il y a donc probablement quelqu'agent inconnu, sans l'intervention duquel le virus du cheval n'a aucun effet."
"Après avoir lu ce que j'ai publié sur ce sujet, vous aurez probablement été surpris de voir qu'un grand nombre des personnes aux¹ quelles on a inoculé la vaccine dans l'Hôpital d'Inoculation pour la petite-vérole, à Londres, ont eu beaucoup de boutons, et que quelques-unes en ont été couvertes. Mais je présume\setcounter{page}{190} que votre surprise cessera lorsque vous saurez que les malades qui ont eu des boutons semblables à ceux de la petite-vérole, avoient été inoculés le premier jour avec du virus vaccin, et le sixieme avec du pus variolique; ensorte qu'il en est résulté une maladie mixte. Mais ce qu'il y a de très-singulier et de très-curieux, c'est qu'en passant d'une maladie à l'autre, le virus a successivement produit moins de boutons, et qu'actuellement il n'en produit plus du tout; les apparences qui résultent des inoculations qu'on fait avec ce virus transmis, sont toujours parfaitement semblables à celles qui résultent de l'insertion d'un virus fraîchement pris sur le pis même de la vache. Ce fait ne semble-t-il pas démontrer que la vaccine est la maladie primitive, dont la petite-vérole n'est qu'une variété, et que le virus variolique, plus foible que le virus vaccin, lui cède la place ou s'assimile à lui?"
"J'avois imaginé qu'il serait possible que cette différence vint de ce que la vache de Londres qui avoit fourni du virus au Dr. Woodville, étoit mieux choyée que celles sur lesquelles j'ai pris celui dont je me suis servi, et qui, comme c'est l'usage dans le Comté que j'habite, sont accoutumées à paître en liberté, et à vivre d'une maniere plus rapprochée de l'état de nature; pour m'en assurer,\setcounter{page}{191} j’ai fait venir du virus pris sur le pis d’une vache malade près de Londres. Mais quoique ce virus m’ait servi à inoculer successivement pendant plusieurs mois de suite, en le transmettant d’un individu à l’autre, au delà de 200 personnes, je n’ai vu de boutons sur aucune d’elles. Ce n’est pas que la maladie ait toujours été exempte de toute espece d’érüption. Quand l’aréole s’est fort étendue autour de l’incision, j’ai vu quelquefois le malade couvert d’une rougeur générale, et quelquefois plusieurs petits boutons durs et rougeâtres, quelques-uns même avec une apparence vésiculaire à leur sommet, se sont manifestés en différentes parties du corps. Mais cette apparence est fort rare, et j’imagine qu’elle dépend des mêmes causes que l’irritation locale de diverses autres substances acres, telles que les cantharides, la poix de Bourgogne, le tartre émétique et plusieurs autres qui produisent aussi, et, si je ne me trompe, plus fréquemment encore que le virus vaccin, quelque affection générale de la peau."
"L’inoculation de la vaccine se propage de plus en plus dans cette isle. Le nombre des inoculés monte actuellement à plus de 5000. Rien n’a diminué ma confiance dans ce moyen de détruire les effets de la petite-vérole. Au contraire, j’acquiers tous les jours des preuves\setcounter{page}{192} nouvelles et convaincantes de son efficacité."
"J'espere que vous me serez part aussie de vos succés, et suis etc."
E.J.

Je serais bien charmé de pouvoir vous envoyer de la matiere de Berkeley, mais les deux morceaux de verre, sont apposés l'un à l'autre, liés avec du fil, et cachetés avec tant de soin, que je me propose de ne les pas séparer jusqu'au moment où je m'en servirai \footnote{Nous n'en avons plus besoin, graces aux précédens envois, que le Dr. D. a bien voulu nous faire de Vienne, et aux fils qu'en a apportés le Dr. Peschier, la vaccine est actuellement bien acclimatée chez nous. Les inoculations que nous avons faites à la suite de celle dont nous parlions dans le dernier Journal, ont bien réussi. Nous en sommes à la cinquieme génération du virus vaccin, pris à Vienne, sur les bras du C. M. Un de nos inoculés nous a fourni beaucoup de pus, avec une grande rougeur autour de l'incision. Plusieurs ont eu des symptômes très-marqués d'infection générale; dans la plupart ils se sont manifestés beaucoup plus promptement que nous ne l'aurions imaginé. Nous suivrons à ces inoculations; nous ferons ensuite sur tous nos inoculés, l'expérience fondamentale de la petite-vérole; nous tiendrons note de ce que nous observons, et j'aurai soin de communiquer, au public, tous ces détails. Ils intéressent de trop près pour ne pas lui dire la vérité toute entiere. Jusqu'à présent, elle n'a rien que de très-satisfaisant.}.
\setcounter{page}{193}
Vous voyez donc, que le Dr. J. porte le même jugement que moi sur les résultats du Dr. Woodville. — Les Médecins Hanovriens sont de la même opinion.— Ce que dit le Dr. J. dans la lettre ci-dessus me paroît prouver sans replique, l'hybridité de la vaccine de l'hôpital de Woodville.
Puisque le Docteur a réussi à mettre cette matiere sur un verre, quoique cela m'eût paru difficile, je ferai une nouvelle tentative d'en recueillir de cette maniere, qui seroit sans doute préférable à celle du linge imprégné, non pour l'envoi mais pour l'inoculation.
Le desir que l'auteur de la lettre témoigne de voir ses recherches se continuer sur le Continent, m'est garant qu'il ne peut voir qu'avec plaisir que vous la publiez, dans votre Recueil, qui sera le dépôt le plus complet de tout ce qui regarde la vaccine.
J. DE CARRO.
P. S. Le Dr. Peschier vous communiquera l'extrait d'une lettre de MM. Balthorn et Strohmeyer, et vous lui communiquerez celle de Jenner. Vous êtes autorisés par les Médecins Hanovriens à publier la premiere lettre dont\setcounter{page}{194} j'ai envoyé copie au Dr. Peschier. La découverte de la vaccine, dans le Holstein, est intéressante \footnote{Extrait d'une lettre des Drs. Balthorne, et Strohmeyer au Dr. De Carro, en date du 12 Déc. 1799.
Cette lettre contient le récit de plusieurs inoculations faites à Hanovre, avec des fils reçus de Londres, et imprégnés du virus vaccin. Ces Mrs. ont fait usage de ces fils de deux manières, savoir: en les plaçant sur une incision faite avec la lancette, sur la peau, mise à nu par un vésicatoire. Ils préfèrent l'incision parce qu'elle produit des ulcères plus petits, et qui ne s'étendent pas autant pendant le cours de la maladie, au lieu que le vésicatoire produit souvent des ulcères corrosifs, et dont la guérison est lente et pénible. Ils soupçonnent que le grand écoulement que produit en certains cas le vésicatoire peut rendre la matière moins active en la délayant trop. Ils ont remarqué sur quelques-uns de leurs inoculés, une éruption de quelques boutons épars sur tout le corps, et semblables à ceux de la fausse petite-vérole. C'étaient des taches rouges, qui se changeaient en petites pustules, sensibles au toucher, et paroissant contenir à leur sommet un peu de sérosité. Ils n'ont eu occasion de réinoculer avec la vraie petite-vérole, que deux de leurs inoculés vaccins. Elle n'a produit sur eux aucun effet.
D'autres lettres du Dr. De Carro nous apprennent, 1º. que le Gouvernement d'Hanovre prend à ces expériences un tel intérêt, qu'un chirurgien, nommé Mr. Bock, va partir, ou est déjà parti aux frais de l'Electeur, pour aller prendre sur les lieux mêmes où règnent la vaccine et le javart (the grease) des informations précises sur la nature, l'origine et les effets de cette maladie; 2°. que le Dr. Nissen de Seegberg dans le Holstein écrit, que la vaccine n'y est point inconnue, non plus que ses effets antivariologiques, sur les personnes auxquelles les vaches qui en sont atteintes la communiquent. Il seroit intéressant de savoir, si dans ce pays, fameux par ses haras, on emploie les mêmes domestiques au pansement des chevaux et aux soins de la laiterie.}.

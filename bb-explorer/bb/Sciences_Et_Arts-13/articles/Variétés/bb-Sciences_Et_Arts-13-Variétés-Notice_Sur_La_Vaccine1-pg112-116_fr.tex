\setcounter{page}{112}
\chapter{Variétés}
\section{NOTE sur l'Inoculation de la VACCINE; par le Prof. ODIER.}
Je crois avoir enfin réussi à inoculer la Vaccine. Le 10 Nivose, le Dr. Coindet me conduisit auprès de deux enfans dont il avait déterminé les parens à faire encore l'essai des fils que nous avoit envoyés le Dr. De Carro de Vienne. L'un de ces enfans, François HIER, avoit treize mois. Ce fut par lui que nous commençâmes. Nous l'inoculames au bras droit avec un fil que le Dr. D. tenoit du Dr. Pearson et qui avoit été imprégné à Londres le 5 août. Au bras gauche, nous appliquâmes un fil tiré d'une des manches du Comte de M. en date du 14 octobre. Et pour éviter que le sparadrap ne fût en contact avec l'incision, circonstance qui me paroissoit la seule à laquelle on pût attribuer le manque de succès des premières inoculations que j'avois faites, nous couvrires le fil d'un petit linge avant d'appliquer le sparadrap. Le 13 Nivose nous levames l'appareil. La première incision étoit entièrement sèche, et avoit presque disparu. La seconde étoit encore béante, et il en suintoit un\setcounter{page}{113} peu de sérosité purulente, quoique sans rougeur et sans gonflement. Nous y appliquâmes un autre fil du Comte Le 16 il y avoit autour de cette incision un peu de rougeur et de gonflement. Le 17 l'enfant eut un peu de fièvre, de gringerie et de mal-aise. Le 19, on s'apperçut qu'il avoit mis une dent. La fièvre avoit presque cessé; mais la rougeur et le gonflement de l'incision étoient plus considérables et avoient près d'un centimètre et demi (un demi-pouce) de diamètre. Il en suintoit encore une sérosité purulente, sans qu'il y eût cependant aucune apparence vésiculaire. Le 21 la pustule étoit sèche et recouverte d'une croûte assez mince. Le 23, quoique la croûte fût plus épaisse, on pouvoit cependant encore humecter légèrement une lancette en la passant à plusieurs reprises sur la tumeur. Le 27 la dessiccation étoit complette. La croûte étoit brune, épaisse, dure, et d'une circonférence irrégulière, semblable à la description qu'en ont faite les Anglais.
Jaques Hier, frère du précédent, âgé de 6 ans, avoit eu à l'âge de 8 à 9 mois une maladie qu'on avoit prise pour la petite-vérole et dont il étoit légèrement marqué, mais qui sur ce que nous en dirent les parens, nous parut avoir été la petite-vérole volante, les boutons n'ayant duré que fort peu de temps. Le 21 nivose nous lui inoculâmes la vaccine au bras droit avec le même fil qui avoit servi à son frère, qui tenoit encore au linge dont on l'avoit recouvert, et qui ayant séjourné long-temps sur la plaie, étoit fortement imprégné d'un nouveau pus. Au bras gauche nous l'inoculâmes avec un autre fil de la manche du Comte de Vienne. Nous primes la même précaution que pour son frère, et nous appliquâmes sur le fil un petit linge qui recouvrant toute l'incision, la garantissoit du contact immédiat du\setcounter{page}{114} sparadrap. Le 23 l'incision du bras droit étoit béante, un peu rouge et en suppuration. Nous y applicquames un nouveau fil imprégné à Vienne au mois de mai dernier et tiré de la manche d'un des enfans De Carro. L'incision du côté gauche étoit un peu rouge, et il y avoit une tumeur sensible: nous fîmes à 3 centimètres au-dessus une nouvelle incision à ce bras, avec la lancette humectée sur le bras de son frère, et sans aucun appareil. Le 27 l'incision du bras droit étoit encore un peu béante, mais sans suppuration, sans rougeur et sans tumeur. Celle du bras gauche faite avec la lancette seule et sans fil étoit un peu rouge et élevée. Mais ces symptômes d'infection n'ont fait depuis aucun progrès. L'incision faite au même bras avec le fil, avoit augmenté en rougeur, en étendue et en élévation. Il en étoit suinté une sérosité purulente, mais on n'y avoit apperçu aucune apparence vésiculaire, et actuellement elle étoit recouverte d'une croûte sèche, jaunâtre, un peu farineuse et assez épaisse. La circonférence en étoit irrégulière, et l'étendue à-peu-près comme celle de son frère. Cette croûte a depuis bruni et durci avant de tomber. L'enfant n'a eu d'ailleurs aucune indisposition, aucun symptôme de fièvre ou de mal-aise, pas même de douleur ou de gonflement sous l'aisselle.
À-peu-près dans le même temps le 22 nivose. Le Dr. Peschier, nouvellement arrivé de Vienne, où il avoit été témoin oculaire des inoculations faites par le Dr. De Carro, inoculoit de son côté la vaccine à deux enfans, avec un fil envoyé de Londres par le Dr. Pearson,\setcounter{page}{115} et un autre tiré de la manche du Comte Les deux fils avoient été recouverts d'un sparadrap gommé, sans l'interposition d'aucun linge. Le Dr. a bien voulu me communiquer son Journal. En voici le résultat. De ces deux inoculations, l'une peut être considérée comme n'ayant pas réussi, puisque quoique l'incision recouverte avec le fil de Vienne, ait paru assez rouge dès le 3\textsuperscript{me} jour de l'inoculation, cette rougeur a promptement disparu, et n'a été ni accompagnée ni suivie d'aucune tumeur, croûte ou dureté, et qu'il n'y a eu d'ailleurs aucun mal-aise général, aucune douleur à l'aisselle, aucun symptôme de fièvre.
L'autre inoculation faite avec le fil de Vienne, a réussi presque au même degré que les miennes. L'incision a formé une tumeur rouge et dure qui s'est terminée au 10\textsuperscript{me} jour par une croûte, avec beaucoup de démangeaison; mais la tumeur m'a paru un peu moins considérable que celle de mes deux inoculés, et la croûte étoit aussi beaucoup moins épaisse. Il n'y a eu d'ailleurs aucun symptôme d'indisposition ni générale ni locale, aucune douleur à l'aisselle, etc.
Le fil de Londres a constamment paru tout-à-fait inerte.
Je ne sais si d'après ces résultats on peut affirmer que ces trois enfans ont eu la vaccine dans un degré suffisant pour les garantir sûrement de la petite-vérole. C'est ce que nous verrons au printemps prochain. Je regarde d'avance comme douteux le succès de l'inoculation variolique que nous nous proposons de faire\setcounter{page}{116} alors, parce qu'il n'y a eu ici ni aréole autour de l'incision, ni gonflement douloureux à l'aisselle, ni aucun indice bien marqué d'infection générale. Cependant, il paroît que les inoculateurs Anglais ont eu plusieurs cas semblables et qu'ils regardent dans ces cas là l'infection comme suffisante.
P. S. Le Dr. Peschier a fait depuis une autre inoculation de vaccine; dont le succès paroît moins équivoque. C'est aujourd'hui (11me. pluviosè) le 8me. jour; et non-seulement il y a un bouton suppurant, mais encore une rougeur bien marquée, et assez étendue; la fièvre s'est manifestée le 3me. jour: comme nous avons actuellement du virus frais; nous suivrons ses inoculations.

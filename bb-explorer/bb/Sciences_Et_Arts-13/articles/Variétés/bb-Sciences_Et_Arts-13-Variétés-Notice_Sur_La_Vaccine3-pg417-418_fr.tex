\setcounter{page}{417}
\chapter{Variétés}
\section{AVIS du Profr. ODIER sur la VACCINE.}
LA marche singulièrement rapide qu'avoit constamment eue à Genève la vaccine inoculée nous inspiroit quelques doutes sur la nature du virus dont nous nous étions servis. Nous attendions impatiemment le moment de soumettre nos inoculés à l'épreuve de la petite-vérole. Ce moment est arrivé. Trois d'entr'eux ont été inoculés avec du virus variolique. Il a eu complètement son effet sur l'un des trois. La petite-vérole, à la vérité, a été singulièrement bénigne, mais bien caractérisée tant par les progrès de l'inflammation locale que par la fiévre et l'éruption subséquente. L'un des deux autres n'a eu ni fiévre ni boutons; mais l'incision s'est enflammée comme à l'ordinaire, et au huitieme jour, elle a été entourée d'une efflorescence bien marquée. Le troisieme n'a eu que les apparences d'une irritation locale, semblable à celle qu'avoit ci-devant produite le virus vaccin, sans efflorescence, sans fiévre sensible, et sans aucune éruption.
\setcounter{page}{418}
Il paroît donc 1°. que le virus qui nous avoit été envoyé de Vienne par le Dr. De Carro, et qui avoit été pris sur les bras du Comte M., (Voy. la Bibl. Brit. Sc. et Arts, vol. XII. p. 308) avoit perdu en partie sa faculté de préserver de la petite-vérole. 2°. Qu'il avoit cependant conservé celle de produire une vaccine bâtarde, très-active, et susceptible de se transmettre par l'inoculation, d'un individu à l'autre, avec les mêmes caracteres de bâtardise ; c'est-à-dire, d'exciter successivement chez tous une inflammation très-précoce et très-étendue autour de l'incision, avec quelques symptômes d'affection générale, mais incapable de garantir complétement de la petite-vérole.
On se rappelle que le Comte M. avoit eu la petite-vérole dans son enfance. C'est probablement à cette circonstance qu'il faut attribuer la dégénération du virus vaccin sur son bras. Le Dr. Jenner avoit cru que la petite-vérole ne préserve pas de la vaccine. Le Dr. Pearson avoit affirmé le contraire. L'inoculation du Comte M. sembloit décider la question en faveur de l'avis du Dr. Jenner. Mais l'histoire de nos inoculés nous ramene à celui du Dr. Pearson, et nous prouve que pour inoculer la véritable vaccine, il ne faut pas prendre le virus sur le bras d'une personne qui ait eu la petite-vérole, parce que cette circonstance seule peut le faire dégénérer, et lui ôter en partie sa faculté préservatrice.

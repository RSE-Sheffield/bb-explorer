\setcounter{page}{315}
\chapter{Variétés}
\section{NOTE du Prof. Odier sur la Vaccine}
Nos Inoculations de Vaccine continuent avec succès. En inoculant de bras à bras, nous en sommes à la dixieme génération Genevoise, c'est-à-dire, que le virus pris sur les bras du Comte M. à Vienne, le 14 Oct. par le Dr. Decarro, a successivement passé par le corps de dix individus Genevois, sans que nous ayons aperçu aucune différence dans les résultats. La maladie qu'il a produite a toujours été très-bénigne; mais elle a eu ceci de particulier, c'est que sa marche a été très-rapide. L'incision a toujours présenté des signes non équivoques d'infection au bout de quelques heures. La fièvre et l'efflorescence, quand elles ont eu lieu, se sont toujours manifestées dès le lendemain, et au plus tard dès le troisième jour. Au bout de 72 heures, la maladie a été réduite à une simple croûte fort tenace sur l'incision. Sous cette croûte, nous avons trouvé, jusqu'au septieme ou huitieme\setcounter{page}{316} jour, du pus, plus ou moins liquide, qui nous a servi à inoculer d'autres sujets. Dans une de ces inoculations, la lancette plongée dans la croûte n'a été que salie et point humectée. Cependant, cette inoculation a aussi bien réussi que les autres. Nous n'avons point observé de boutons sur le reste du corps. Nous n'avons point encore soumis nos inoculés à l'épreuve de la petite-vérole, parce que nous voulons attendre que toutes les traces de la vaccine aient complètement disparu.

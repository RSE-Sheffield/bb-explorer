\setcounter{page}{183} PRINCIPES RAISONNÉS D'AGRICULTURE.
Traduit de l'allemand d'A. THAER, par
E. V. B. CRUD. Tom. Iᵉ. in-4°, 372 p.
Genève, 1811, chez J. J. Paschoud, Imprimeur-Libraire; et à Paris chez le même
rue Mazarine, n°. 82.
(Dix-neuvième extrait. Voy. p. 117.)
NOTRE globe paroît avoir intérieurement, un certain degré de chaleur. En effet; à une profondeur de dix pieds au-dessous de la surface du sol, la température se conserve, dans toutes les saisons, égale à sept degrés du thermomètre de Réaumur. Long-temps on a cru que cette chaleur étoit due à un feu central contenu dans l'intérieur de la terre, ou tout au moins à une grande chaleur, qui y étoit demeurée depuis le temps de la création, et qui avoit d'autant plus d'intensité qu'elle étoit plus rapprochée du centre. Mais cette opinion a été trouvée sans fondement, car à quelque profondeur qu'on ait pénétré dans les mines, on ne s'est aperçu d'aucune élévation de température. A douze cents pieds au-dessous de la\setcounter{page}{184} surface du sol, cette chaleur étoit toujours la même. Dans quelques mines de la Hongrie, cependant, on s'est aperçu d'une augmentation de chaleur; mais cet effet peut être attribué à des causes locales, telles que la température élevée de quelque source, ou la nature du terrain qu'elles parcourent. La chaleur qui résulte de ces espèces de fourneaux souterrains est un phénomène rare.
On distingue souvent des différences dans la température du sol, par le temps plus ou moins long que la glace et la neige emploient à s'y fondre, et par la disposition que certaines places ont à se geler plutôt que d'autres, sans que la position du sol puisse y contribuer.
Dernièrement, on a fait là-dessus, à l'aide du thermomètre, quelques observations plus précises; mais elles n'ont point encore été assez répétées, pour que je puisse présenter rien de plus positif que ce qui va suivre.
La température du sol dépend d'abord, visiblement, du degré d'humidité de ce sol. Le terrain humide est, en moyenne, plus froid; il s'essuie plus lentement; il gèle plus vite; et il lui faut plus de temps pour acquérir la chaleur nécessaire à la végétation.\setcounter{page}{185} C'est pour cela qu'on qualifie ce terrain d'humide et froid; le terrain sec, de chaud, et de très-sec, de brûlant. Il n'est pas douteux que cela ne provienne de ce que l'évaporation de l'eau consume une quantité de calorique libre, et l'enlève, par conséquent, au sol.
On trouve souvent aussi une différence de température dans des terrains d'une humidité égale. Un sol rempli de terreau, de fumier non épuisé, et de substances en putréfaction, a un degré de chaleur beaucoup plus élevé, et fait plus promptement fondre la neige dont il est recouvert. Dans ce cas, la chaleur provient des décompositions chimiques qui s'y opèrent, et dans lesquelles il se dégage presque toujours du calorique; ainsi, il est vrai, à la lettre, de dire que le fumier réchauffe le terrain. Il produit cet effet, en partie mécaniquement, en rendant le sol plus léger, plus meuble, et ainsi plus sec, et en partie chimiquement, en le décomposant.
L'on s'aperçoit que le terrain calcaire est plus chaud, parce qu'il accélère ses décompositions chimiques, et que, non-seulement il exerce l'action la plus forte sur le fumier et l'humus, mais que ceux-ci l'exercent également sur lui. Enfin, les sols ne sont pas tous également bons conducteurs de la cha\setcounter{page}{186} leur qu'ils reçoivent du dehors. Le sable l'est plus que l'argile, lorsque celle-ci n'est pas excessivement humide. Ainsi, un changement subit de température a plus d'influence sur les plantes qui végètent dans un terrain sablonneux, que sur celles qui croissent dans un terrain argileux, et par cette raison, les gelées de nuit; et sur-tout, les gelées blanches, sont plus nuisibles aux premières: on a souvent lieu de le remarquer sur celles des semailles auxquelles le froid est le plus contraire. Probablement aussi, il est telle couche inférieure du sol qui, mieux que telle autre, communique aux corps avec lesquels elle est en contact, la chaleur qui provient du dedans de la terre, et qui par cela même, fait que la chaleur pénètre moins avant, et se dissipe plus vite.
On détermine les degrés de température du sol par les expressions de brûlant, chaud, d'une chaleur moyenne, froid.
Des recherches plus précises, que l'on fera, à l'aide du thermomètre, sur-tout au printemps, à la cessation des gelées, donneront peut-être des résultats plus remarquables sur les différences, qui, sous ce rapport, existent entre les divers sols.
La valeur et les qualités du sol ne dépendent pas seulement de sa composition, et de ses propriétés intérieures, mais aussi de\setcounter{page}{187} PRINCIPES RAISONNÉS D'AGRICULTURE.
la disposition de sa surface, de sa situation, et de ce qui l'environne.
La disposition de la surface, sa position montueuse ou plate, horizontale ou inclinée, a une influence variée suivant les proportions diverses des terres élémentaires dont le sol est composé.
Le terrain sablonneux, mouvant et sec, est d'autant plus fertile qu'il est plus plat et plus bas, relativement à la contrée qui l'environne. Dans cette position, il conserve plus long-temps l'humidité, dont il a rarement une surabondance. En revanche, cette espèce de terrain perd d'autant plus de sa valeur qu'elle se trouve placée sur des hauteurs, sur les pentes des collines; parce que son humidité se perd plus promptement, soit par la pente soit par les vents. Les particules fertilisantes du sol s'évaporent aussi par cette dernière cause.
Une position élevée ou inclinée est au contraire avantageuse à un terrain argileux, ou qui repose sur une couche imperméable, parce que les eaux ont ainsi un écoulement facile, au moyen des tranchées et coulisses qu'on y pratique. Cependant une pente trop roide a divers inconvénients pour la culture.
On a beaucoup disputé sur la question de savoir si, sur une même base géométrique\setcounter{page}{188} \section{AGRICULTURE}
la surface d'un terrain montueux, nourris soit plus de plantes, que ne feroit un terrain horizontal. Les théoriciens soutiennent qu'il n'y a pas plus de place pour les plantes qu'il n'y en auroit sur la base : les praticiens soutiennent le contraire ; et ceux-ci paroissent avoir raison. Si la couche végétale a la même profondeur, il y a évidemment plus de surface sur un coteau que sur sa base ; et par conséquent les racines ont plus de place pour s'étendre ; en supposant le nombre des plantes le même. Les plantes des céréales ont d'ailleurs plus d'espace pour tirer de l'atmosphère les sucs qu'elles s'approprient ; ainsi, à qualité égale du sol, il paroît qu'un terrain en pente ne doit pas être estimé seulement selon l'étendue de sa base (laquelle seule doit être donnée dans les plans) mais aussi d'après l'étendue de sa surface \footnote{Il y a une autre raison encore que celles que l'auteur donnënt, pour que le produit d'une pente, soit (à nature de terre égale) supérieur au produit de sa base, c'est que toutes les plantes de prairies, considérant à part chaque touffe d'herbe, s'élèvent dans une, direction moyenne perpendiculaire au plan incliné de la pente ; et non perpendiculaire à l'horizon. Ce n'est que la tige qui monte en graine qui affecte la direction verticale. Or, ce sont les feuilles des graminées et non les tiges, qui font principalement la nourriture du bé-}.\setcounter{page}{189} L'élévation du sol au-dessus du niveau de la mer produit une grande différence dans le climat, et la température atmosphérique. Dans une même zône, la chaleur est toujours moins grande sur les montagnes que dans les plaines. Sous la zône torride même, les sommets des hautes montagnes sont toujours couverts de neige. Cependant les limites des glaces sont plus hautes dans les pays voisins de l'équateur, et s'abaissent en se rapprochant des pôles. La végétation diminue dans la même proportion que la chaleur. Sur les hauteurs, les arbres de plaine sont moins forts et moins élevés; plus haut il ne croit que des pins; puis seulement des plantes de montagne; et enfin toute végétation cesse.
On s'aperçoit d'une diminution dans le produit des céréales, sur les hauteurs, lors même que le terrain est fertile, et la qualité, surtout pour le pâturage. Ce qui est vrai des graminées, l'est encore plus pour les plantes qui rampent et s'épatent sur le sol, comme le sainfoin, les pimprenelles, les trèfles, les vesces, les pois, etc. On observe souvent des touffes épaisses d'herbe qui végétent au-dessus les unes des autres dans les fentes de vieilles murailles à-peu-près verticales. Il seroit impossible que la même quantité d'herbe, trouvât place sur la base d'un tel plan incliné: cela aide à faire comprendre comment il croît plus d'herbe sur une pente que sur sa base. (R)\setcounter{page}{190} turité y est toujours plus tardive. Les montagnes manquent rarement d'humidité, parce qu'il s'y fait un plus grand dépôt de celle que contient l'atmosphère; aussi les terrains chauds y ont-ils l'avantage. Un inconvénient grave des montagnes c'est la difficulté du transport des engrais et des récoltes; et enfin les eaux pluviales y forment des ravines, et emmènent la terre; ce qui fait que, même dans les pentes fertiles, il vaut mieux conserver des bois ou des pâturages, que de labourer le sol.
La direction de l'inclinaison du sol est une considération importante. Les terrains qui penchent vers le nord sont plus difficilement essuyés ou égouttés, et moins promptement réchauffés. Les substances des engrais demandent plus de temps pour se décomposer; la végétation y commence plus tard et finit plus tôt; les plantes manquent de chaleur et de lumière, donnent moins de fruits et ont moins de saveur; enfin elles y souffrent des vents froids et des gelées.
Les terrains qui penchent vers le sud, ont tous les avantages opposés à ces inconvénients; mais en revanche, ils sont sujets aux sécheresses, ainsi qu'aux bourasques de pluie et de grêle.
Les avantages et les désavantages de cette inclinaison du sol, dépendant aussi beaucoup\setcounter{page}{191} de sa composition et de ses propriétés. Le terrain argileux et froid gagne à être tourné à l'orient et au midi : il devient décidément stérile s'il penche au nord. C'est précisément le contraire pour les terres sablonneuses et graveleuses.
Indépendamment de la chaleur que donnent les rayons du soleil, sa lumière est nécessaire à la prospérité des plantes, et à la décomposition de certaines substances dans le sol. Tous les végétaux cherchent la lumière. Les plantes qui ont végété à l'ombre ont un tissu lâche, des pousses minces et foibles; elles ont l'air maladif et languissant, et n'ont point la saveur qu'elles devroient avoir. La couleur verte des feuilles dépend uniquement de la lumière, ainsi que des expériences directes l'ont prouvé. Dans les lieux ombragés, la végétation n'est donc qu'imparfaite, et l'herbe qui y croît est peu nourrissante.
Selon la nature de la composition du sol, il peut lui être avantageux d'être exposé à tous les vents, ou d'en être garanti d'un ou de plusieurs côtés. Le terrain sec, sablonneux et chaud, gagne à être abrité des vents, par les arbres ou les haies. Les terrains humides et froids, gagnent, en général, à être battus de tous les vents, parce qu'ils s'essuyent plus vite au printems.\setcounter{page}{192} Enfin, il faut prendre en considération les modifications de l'atmosphère et de la température que l'on désigne sous le nom de climat.
La température locale n'est pas seulement affectée par la direction plus ou moins verticale des rayons solaires, plusieurs causes paraissent également y avoir part, telles que les montagnes et les forêts qui coupent ou entourent le pays; la position du sol relativement à certains vents; le voisinage des sommets couverts de neige, l'élévation au-dessus du niveau de la mer; le voisinage de celle-ci ou des grands lacs; tout cela mérite d'être examiné, et indépendamment de la latitude, influe sur le climat d'un lieu déterminé.
L'eau que l'atmosphère tient en dissolution se précipite plus abondamment dans certaines contrées que dans d'autres. La quantité de pluie qui tombe annuellement ainsi que la longueur et la fréquence des ces pluies, intéressent essentiellement l'agriculteur. La fréquence des brouillards, et l'abondance des rosées méritent considération.
Le voisinage des grands réservoirs d'eau est favorable aux terrains secs, et défavorable aux terrains humides.
Les exhalaisons des marais affectent quelquefois\setcounter{page}{193} quefois d'une manière fâcheuse la végétation des céréales. Les bois de haute futaie et d'une grande étendue attirent l'humidité. Il y a des positions plus sujettes que d'autres à la fureur des orages de grêle.
Les couches inférieures de l'atmosphère contiennent des gaz qui ont une grande influence sur la végétation. On sait que le gaz acide carbonique et le gaz hydrogène carboné, sulfuré et phosphoré sont très-favorables à la végétation, et qu'ils contribuent réellement à l'amendement du sol; mais il y a probablement d'autres substances, surtout les substances animales, qui ne sont pas encore entièrement décomposées, ou dont les élémens se sont combinés d'une manière particulière. Les contrées très-peuplées, qui entretiennent une grande quantité de bétail, où l'on consume beaucoup de combustible, et où il s'opère de nombreuses décompositions, dont les produits se combinent avec l'atmosphère, se distinguent d'une manière frappante par leur fertilité : différentes observations semblent démontrer que cette grande fertilité est indépendante de la plus grande quantité de fumier que ces contrées produisent. Dans les grandes villes, et dans leurs environs, on peut à peine méconnoître cette influence de l'atmosphère. Vol. 18. N°. 5. Mai 1813. q\setcounter{page}{194} mosphère sur la fertilité des terrains, même les plus mauvais. Ce que nous avons dit précédemment des vapeurs qui s'élèvent des eaux stagnantes, prouve que l'air peut également contenir des substances nuisibles.
Des expériences directes ont également mis hors de doute l'influence fâcheuse de l'épine-vinette sur les céréales qui s'en trouvent voisines.
La valeur d'un terrain peut être beaucoup modifiée par la plus moins grande quantité des mauvaises herbes qui s'y trouvent. Cette quantité ne varie que du plus au moins, car il est très-rare qu'une terre en soit exempte.
On qualifie de mauvaise herbe, toute plante qui végète dans un lieu où elle n'a pas été semée ni plantée, et qui nuit aux végétaux cultivés sur le même terrain, en les privant de la place et des alimens qui leur étoient destinés. Il est ici question des mauvaises herbes qui ont rempli le sol de leurs semences et de leurs racines, à tel point, qu'on ne peut les détruire sans beaucoup de peine, et sans des sacrifices considérables.
Sous les rapports agronomiques, nous distinguons ces mauvaises herbes en trois genres. 1°. Celles qui se multiplient par les semences seulement. 2°. Celles qui ordinairement ne se propagent que par les drageons\setcounter{page}{195} (195 qui poussent de leurs racines. 3°. Celles qui se reproduisent des deux manières.
Les mauvaises herbes qui se reproduisent de semence, se divisent en deux espèces, savoir : les annuelles qui germent, produisent leur semence, la répandent, et périssent dans le même été ; et les bisannuelles, qui résistent à l'hiver ; et répandent leur semence à la seconde année de leur végétation. Les racines de ces deux espèces ne sont point vivaces, elles périssent avec la plante, lorsque la semence a atteint sa maturité.
Les semences des végétaux dont il est ici question ne germent que lorsqu'elles se trouvent très-rapprochées de la surface du sol : si elles sont à une certaine profondeur, ou renfermées dans des mottes de terre, elles demeurent saines, et conservent leur faculté de germer, au moment où les circonstances favorisent cette germination. La durée du temps pendant lequel les semences peuvent conserver cette faculté de germer paroît indéfinie. On a observé que dans les marais de l'Oder, si l'on dessèche des parties qui de temps immémorial sont en marais, et qu'on les mette en culture, il y pousse une quantité surprenante de moutarde des champs, dont sans doute les semences avoient été entraînées par les eaux, et déposées avec le\setcounter{page}{196} limon, dans des temps très-reculés. On voit souvent les mauvaises herbes pousser abondamment sur des terres tirées d'une profondeur de plusieurs pieds. On a trouvé, sous un bâtiment qui avoit subsisté deux cents ans, une terre noire, qui, transportée avec le platras dans un jardin, y produisit une quantité prodigieuse de marguerites dorées, plante qu'on n'avoit jamais vue dans ce lieu. De semblables faits ont souvent fait croire aux gens peu instruits que les plantes naissioient quelquefois sans semence ni germe, comme s'il pouvoit y avoir une exception à la règle de l'analogie, confirmée par toutes les bonnes observations, *omne vivum ex ovo*.
Le nombre des petites semences qui peuvent exister dans le sol, dépasse tout ce qu'on pourroit imaginer. Quand le terrain a été soigneusement pulvérisé, il pousse une quantité de mauvaises plantes qui le couvre. Le labour les tue; mais le terrain ramené à la surface produit un nombre tout aussi grand de mauvaises plantes. Cela peut se répéter six fois dans un été. On a essayé de répéter la jachère et les labours fréquens, trois ans de suite, sans pouvoir se débarrasser des marguerites dorées.
Les mauvaises herbes annuelles ne se montrent d'ordinaire que parmi les grains de printemps, parce quand leur levée s'est faite en\setcounter{page}{197} automne dans les blés d'hiver, le froid les tue. Les mauvaises herbes bisannuelles, au contraire, ne parviennent à leur développement complet, que parmi les céréales d'automne : elles poussent bien aussi parmi les céréales de printems, mais elles sont détruites par le labour, avant qu'elles soient parvenues à la floraison et à la maturité. Les plus nombreuses d'entre les mauvaises herbes annuelles sont du genre des moutardes et des raves sauvages, telles que la moutarde des champs ( sinapis arvensis ) qui ne réussit que dans les terrains forts, riches, et qui conservent leur humidité. Dans les terrains maigres, on peut semer sans danger de la graine de moutarde avec les autres graines. La moutarde lève, mais elle ne tarde pas à être étouffée. Dans les terrains riches et fort imprégnés d'humus, elle prend, au contraire, complétement le dessus sur les récoltes et les endommage beaucoup. Les cultivateurs soigneux arrachent cette plante épaisse et nourrissante avant que les céréales montent, et ils la donnent au bétail. Le radis sauvage ( raphanus raphanistrum ) végète sur les terrains moins forts et moins riches. Il réussit, lors même que la température de la saison est contraire, prenant ainsi le dessus sur les céréales, d'autant plus surement dans les mauvaises années. Sa fane\setcounter{page}{198} est plus âpre et plus dure au toucher que celle de le moutarde des champs : on en tire aussi parti pour le bétail. Plusieurs espèces de choux, le colza ou la navette, s'emparent quelquefois du terrain et l'infectent. Le chrysanthème des blés, ou la marguerite dorée, pousse avec une telle vigueur, il est si difficile à détruire, et se multiplie si promptement, qu'il peut rendre une terre tout-à-fait inutile. Elle ne commence à pousser que quand la terre est passablement réchauffée; mais sa végétation est si forte, qu'elle étouffe toute autre plante, et prend possession du champ entier. Si on arrache une plante en boutons, elle fleurit, et sa graine vient même à maturité. Si on jette en tas les plantes arrachées, elles ne pourrissent point, mais végétent et donnent leur graine : il faut les brûler ou les enterrer profondément. Les semences de cette plante pernicieuse passent sans s'altérer, au travers du corps des animaux, et les fumiers reportent les germes sur les champs. On prend les précautions les plus actives contre l'introduction de cette plante redoutable; et il y a des cantons où l'on impose une amende d'un gros, et même de deux, pour chaque marguerite dorée que l'on trouve dans les champs, à l'époque des visites ordonnées.\setcounter{page}{199} L'auteur désigne l'avena flatua ou sterilis (folle avoine) comme une des mauvaises herbes les plus nuisibles. Il indique les camomilles, les bleuets, les crêtes de coq, les pavots rouges, la nielle, comme des plantes dont la graine se conserve long-temps dans le sol, et dont par conséquent, il est difficile de le débarrasser. Il nomme le bromeseigle, le vesceron, l'arrête-bœuf, le petit lizeron, la prêle, le tussilage, et la ronce, comme des plantes souvent embarrassantes dans les champs. Enfin il parle du chiendent comme excluant toute récolte, tant que la terre en est infectée.
Les pierres disséminées dans le sol, lorsqu'elles ont un volume considérable, altèrent beaucoup sa valeur, parce qu'il faut souvent de grands travaux pour l'en débarrasser. Les pierres roulantes, que la charrue peut déplacer, nuisent aussi à la valeur du terrain, parce qu'elles rendent les labours difficiles, qu'elles usent les instrumens, et empêchent de faucher les cercales près de terre. L'auteur met en doute que le terrain puisse être déprécié par l'enlèvement des pierres, ainsi qu'on l'a prétendu. Cependant il pense que les pierres calcaires se décomposent lentement par l'action des engrais et des végétaux, et fournissent ainsi de la nourriture aux plantes.\setcounter{page}{200} Pour faire une description exacte d'un terrain donné, l'auteur propose d'y tracer des lignes parallèles, en numérotant leurs intervalles. On lève le plan géométrique du terrain ; on y trace les lignes et les numéros correspondans. On parcourt ensuite le terrain d'un numéro à l'autre, en faisant fouiller la terre, et en recueillant des échantillons du sol, qu'on renferme et étiquette avec soin. On indique sur le plan, tous les changemens de sol, en expliquant si ces changemens sont brusques ou gradués. Toutes les observations de détail sont enregistrées à part, avec des renvois qui se rapportent au plan. On trace sur celui-ci, les cours d'eau, les pentes, les hauteurs, les enfoncemens, on sépare par des nuances différentes les diverses qualités de terres, et on a ainsi sous les yeux un tableau exact du terrain dont on dispose. L'analyse chimique des divers sols dont on a recueilli les échantillons, complète la connaissance de la valeur d'une propriété, lorsqu'on a d'ailleurs égard pour l'apprécier, à toutes les données et à toutes les circonstances qui ont été détaillées ou indiquées.
\setcounter{page}{77} \section{AGRICULTURE.}
ISTRUZIONI PRATICHE, etc. Instructions pratiques sur la manière de bien faire et conserver le vin; par le Sénateur DANDOLO. Milan 1812.
IL y a trente ans que le célèbre chimiste, auteur de cet ouvrage, connoît les principes de la meilleure fabrication du vin; et depuis quatorze ans il en fait l'application avec un succès soutenu. Les Instructions pratiques, publiées par ordre du gouvernement d'Italie, ne sont que l'abrégé d'un plus grand ouvrage, donné par l'auteur il y a trois ans. Elles sont divisées en quatorze chapitres, lesquels traitent de la vigne, du raisin, de la vendange, des procédés du cuvage, des substances contenues dans la cuve et de leur action sur la formation du vin, de la fermentation vineuse, des substances nouvelles qui résultent de la fermentation, des signes qui indiquent le moment d'écouler, de l'écoulement de la cuve, du vin transporté dans les fustes, du transvasage, des maladies.\setcounter{page}{78} dies du vin, des remèdes et des soins du vin, des ustensiles et des emplacemens nécessaires pour bien faire et conserver les vins.
L’ouvrage, comme l’indique son titre, est purement pratique. Le savant auteur s’est interdit tous les raisonnemens de pure théorie : il ne s’aide de celle-ci que lorsqu’elle est indispensable à l’intelligence et à la meilleure application des procédés.
Dans le premier chapitre, on voit que le climat, le sol, l’exposition, la saison, la culture, influent sur la qualité des vignes. Le meilleur climat pour celles-ci, est celui où le raisin croît abondamment, et riche en substance sucrée. Il faut que le soleil frappe long-temps et fortement le raisin : si les vignes sont trop couvertes, il convient de les découvrir.
Le meilleur sol pour la qualité du raisin est maigre, volcanique, léger, peu propre aux céréales. Un fonds plus riche et bien amendé, donne plus de raisin, mais le vin y est d’une qualité moindre. Les terres argileuses et humides sont les moius favorables.
La meilleure exposition est celle où le soleil frappe le plus long-temps le raisin, car ce sont ses rayons, qui convertissent le suc\setcounter{page}{79} acide de celui-ci, en substance sucrée. La vigne doit être à l'abri des vents froids et des eaux.
A la sortie des raisins, un temps sec et doux est favorable. Le froid et les pluies, s'ils se prolongent à cette époque, altèrent ou détruisent le tissu que la nature destinoit à former le raisin.
A l'époque de la floraison, il est avantageux que le temps soit chaud, sec, serein, sans être trop calme.
Quand le raisin dé fleurit, il convient que le temps soit serein, que le vent règne, et qu'une pluie courte et abondante nettoie enfin les jeunes grappes. Selon la qualité des terres, il est utile que le retour des pluies soit plus fréquent ou plus rare. Lorsque la maturité du raisin approche, un temps sec et chaud promet le meilleur vin. On doit desirer pour la vendange un temps sec et frais.
Les vignes bien cultivées et très-peu fumées ne sont pas fort abondantes, mais donnent d'excellent raisin. Celles qui sont médiocrement fumées et bien travaillées donnent beaucoup de bons raisins. Celles qui sont bien travaillées et fortement fumées, donnent le plus, mais la qualité des raisins est moins bonne. Il importe d'écarter des\setcounter{page}{80} vignes, les eaux des pluies et des sources.
Pour planter de la vigne, il faut fumer avec du fumier bien consommé. Il faut choisir le plant reconnu le meilleur par les cultivateurs du pays; et on doit avoir égard à la faculté de tel ou tel plant de résister mieux qu'un autre à l'intempérie des saisons, comme aussi à l'avantage de mûrir tout à la fois dans une même vigne.
Il est nuisible de trop charger la vigne: elle ne doit l'être que dans la proportion de la richesse du fonds: avec cette précaution, les raisins réussiront mieux et seront meilleurs si l'année est favorable; et dans le cas où l'année seroit mauvaise, ils résisteront mieux.
La meilleure manière d'appuyer la vigne est d'y employer les échalas qui ne donnent point d'ombre et ne prennent point de substance à la terre: lorsqu'on appuie la vigne aux arbres, il convient de la charger peu et de l'éfeuiller beaucoup.
En liant la vigne, il faut toujours avoir égard à la convenance de laisser au raisin la lumière, la chaleur et l'air.
La taille de la vigne, l'éfeuillement et l'ébourgeonnement, demandent une main exercée. Il importe d'ôter ce qui épuise inutilement le cep, et ce qui empêche que le soleil ne frappe le raisin.\setcounter{page}{81} \section{Du raisin.}
Dans le raisin bien mûr, on trouve, outre de l'eau et un principe odorant particulier à chaque espèce, 1°, la substance colorante, laquelle adhère intérieurement à l'enveloppe du grain. Si l'on enlève légèrement cette enveloppe d'un grain de raisin rouge, on voit que la pulpe en est blanche comme d'un grain blanc. La substance colorante ne se répand que lorsqu'on rompt les cellules qui la contiennent, en écrasant l'enveloppe entre les doigts: c'est la substance qui colore le moût lorsqu'on le foule, et le vin lorsqu'il fermente.
2°. Le tartre, qui est contenu dans une substance un peu acidule appliquée à la matière colorante. Ce tartre reste ensuite en partie dissous dans le vin, et en partie déposé dans les lies, ou contre les parois des vases.
3°. Après cette couche, et en se rapprochant des semences ou pepins qui occupent le centre, on trouve une autre couche qui contient la substance sucrée, laquelle se cristallise et devient solide, lorsqu'on fait sécher très-lentement un grain. Nous verrons que cette substance sucrée est la seule qui se\setcounter{page}{82} convertisse toute entière en esprit-de-vin, et que c'est à elle seule aussi que le vin doit sa force et la faculté de se conserver.
4°. Immédiatement autour des semences et au centre du grain, est une substance liquide, un peu gélatineuse, douce au goût; mais qui ne se cristallise pas dans la dessication du grain. C'est dans ce liquide que se trouve principalement le levain qui, presqu'immédiatement après que le raisin a été foulé, agit sur le moût, et occasionne un mouvement de fermentation semblable à celui que le levain excite dans la pâte. C'est cette fermentation qui convertit la substance sucrée en esprit-de-vin.
5°. On trouve encore dans le grain de raisin mûr, une petite quantité d'un acide, que les chimistes ont appelé malique ( parce qu'il a le goût de pomme ), lequel donne au vin une saveur particulière.
6°. Si l'on exclut d'un grain de raisin la peau extérieure et les semences, on y trouve une aggrégation de vaisseaux, de conduits, et de vésicules, qui contiennent les divers sucs. Tous ces tissus solides ajoutent à l'action du levain ou ferment indiqué ci-dessus.
Les plus précieuses des substances qui composent le grain, sont le sucre et le principe odorant, parce que c'est à ces deux\setcounter{page}{83} les substances que sont dûs le goût exquis, la force du vin, et la qualité de se conserver.
Quoique chaque grain de raisin contienne ainsi tout ce qui peut exciter la fermentation vineuse et faire du vin, ce grain, s'il est abandonné à lui-même, ou en grappes isolées, se sèche ou se moisit, sans fermenter ni faire du vin. Il est donc évident que dans le grain de raisin, ces diverses substances sont disposées de manière à ne pouvoir agir les unes sur les autres, jusqu'au moment où on les déplace et les mêle toutes ensemble, en foulant le raisin.
Il est démontré qu'il y a aussi de la substance sucrée dans la partie adhérente aux semences, et dans la couche contigue à la substance colorante, quoique la véritable substance sucrée cristallisée ne réside que dans la couche qui sépare les deux autres.
Quand les circonstances sont favorables, le travail de la végétation tend continuellement à modifier les diverses substances du raisin vert ou approchant de sa maturité, et à l'amener à l'état de substance sucrée. Quand la maturité est complète, le raisin contient très-peu de principes acerbes et acides.
Lorsque l'année n'a pas été favorable, ou que la maturité a été prévenue par quelque autre cause, voici quels sont les caractères du raisin.\setcounter{page}{84} 1°. La couleur est d'autant moins intense que la maturité est moins avancée.
2°. Le tartre y est d'autant plus abondant, que le grain est moins mûr.
3°. Il en est de même de l'acide malique.
4°. La substance sucrée cristallisable n'est pas encore formée, ou ne l'est qu'imparfaitement.
5°. Dans la substance centrale, il y a plus de levain, et moins de douceur.
6°. La substance solide qui appartient aux tissus des différens vaisseaux et conduits, est en quantité d'autant plus grande que le raisin est moins mûr.
7°. Le principe odorant suit les degrés de la maturité, et augmente comme celle-ci.
Toutes les substances qu'on peut appeler nobles, sont donc en moindre quantité dans les raisins mal mûrs, et toutes ces nuisibles à la qualité du vin, y sont en plus grande proportion.
Le raisin vert est un composé de substances acides, de corps solides, et d'un levain. Il ne sauroit en résulter du vin, parce qu'il y manque la substance sucrée, la seule qui se convertisse en esprit-de-vin, et donne un caractère générique à toutes les liqueurs vineuses, en les préservant de la corruption.\setcounter{page}{85} Si l'on ajoute au jus des raisins verts, une substance sucrée, on obtient bientôt par la fermentation, une liqueur qui a les qualités du vin, sauf le parfum.
\section{De la vendange.}
Tout art qui a pour objet d'obtenir un produit quelconque, se compose d'une série d'opérations liées entr'elles, et le produit est d'autant meilleur que les opérations sont bien suivies. Dans l'art de faire le vin, une seule opération manquée, peut anéantir la valeur du produit.
La première opération seroit mal faite, par exemple, si l'on cueillit le raisin mal mûr, ou qu'on mêlât à la vendange des grappes entièrement vertes. Celles-ci, comme nous venons de le voir, porteroient dans le moût le germe de la corruption. Les grains moisis donnent un levain très-actif qui tend à faire dégénérer le suc des grains sains. Ils communiquent encore au vin, une odeur et un goût désagréables. Le peu de durée des vins faits avec des raisins qui ont souffert, suffiroit d'ailleurs pour engager l'agriculteur à séparer, à la vendange, les grappes mal mûres ou moisies.
Les grains secs font du mal au vin, de trois\setcounter{page}{86} manières. 1°. Ils sont un levain très-fort pour convertir le vin en vinaigre. 2°. Ils absorbent du vin au lieu d'en produire. 3°. Ils lui communiquent une certaine odeur qu'on appelle de sec ou de bois, et qu'on observe dans les vins faits avec des raisins où la grêle a marqué beaucoup de grains.
Il importe donc de séparer les grains secs, autant qu'il est possible, quand on fait la vendange. Voyons maintenant quel est le meilleur moment pour vendanger, et qu'elles sont les attentions nécessaires pour bien faire l'opération de la vendange.
\section{Du meilleur moment pour vendanger.}
Le meilleur moment pour la vendange est celui où l'agriculteur exercé juge que le raisin ne peut plus gagner, et risque au contraire de se détériorer.
\section{De la manière de vendanger.}
La veille du jour fixé pour la vendange, on désigne les vignes où l'opération doit commencer. On l'entreprend du côté où les raisins sont le plus exposés à être volés, ou commencent à souffrir.
Si la température est favorable, la meilleure heure est celle où le soleil a bien essuyé\setcounter{page}{87} les raisins. Si la température est mauvaise sans apparence d'amélioration, on est forcé de vendanger, même par le mauvais temps. Chaque ouvrier ou ouvrière a deux seaux, l'un pour le raisin sain, l'autre pour le rebut. L'ouvrier coupe aux ciseaux le pédicule de la grappe, en laissant celui-ci aussi long que possible du côté du cep. Il examine la grappe ; il coupe aux ciseaux les grains moisis ou verts, et les fait tomber dans un seau. Il jette à terre les grains secs. Ainsi débarrassée, de ce qui nuiroit au vin, la grappe est jetée dans l'autre seau. Toutes les grappes très-défectueuses se jettent dans le seau de rebut. Les raisins transportés dans un lieu de dépôt (capanna) y sont reçus par des femmes, qui examinent encore, grappe par grappe, les raisins de choix, pour enlever ce qui pourroit y rester de mauvais, et les jeter dans la cuve qui leur est destinée. Elles examinent aussi les raisins de rebut, et en ôtent les grains tout-à-fait verts ou secs.
Si l'on compare les vins faits de cette manière à ceux qu'on fait avec la négligence accoutumée, on trouvera que pour la qualité, la durée, et le prix, les premiers donnent à une famille diligente, un quart ou un tiers davantage. Ces vins d'ailleurs portent mieux l'eau, et rendent plus d'eau-de-vie.\setcounter{page}{88} De l'opération de fouler le vin et du remplissage de la cuve.
C'est en foulant et écrasant les raisins que l'on met en contact entr'elles les diverses substances contenues dans chaque grain. Sans cet écrasement, ces substances ne pourraient agir les unes sur autres, ni par conséquent, convertir le moût en vin.
Si l'écrasement complet du grain est indispensable, le mélange de la masse par l'opération de fouler n'est pas moins nécessaire, afin que la proportion de chaque substance soit la même dans toutes les parties de la cuve.
J'ai appris, par une série de faits, que la répétition de l'opération de fouler, est toujours plus ou moins nuisible au vin, et d'autant plus que ces opérations se font quand la fermentation est plus avancée, ou en d'autres termes, on fait d'autant plus de mal, que l'on fait évaporer plus de substances volatiles pendant la chaleur de la fermentation.
La répétition de l'action de fouler a d'ailleurs l'inconvénient de donner au vin un mauvais goût et une mauvaise odeur, en remêlant à la masse, les substances altérées qui sont à la superficie du moût.\setcounter{page}{89} C'est une grande erreur d'imaginer qu'en foulant à plusieurs reprises, on puisse réussir à écraser les grains qui ne l'ont pas été d'abord: l'expérience prouve que ces grains échappent continuellement au fouloir et demeurent entiers.
Il est fort important de remplir la cuve dans le moins de temps possible. Il faut au moût un temps déterminé pour se convertir en vin. Supposons que ce temps soit huit jours: si l'on a mis des raisins dans la cuve trois jours de suite, et qu'on écoule le vin tout à-la-fois, la liqueur sera composée d'un vin trop fermenté, c'est-à-dire, disposé à l'acescence, d'un vin qui n'aura pas assez fermenté et enfin d'une troisième partie qui sera encore du moût. Ce mélange sera disposé à s'altérer par la plus légère cause. Cette négligence occasionne de grandes pertes dans les vins, et s'oppose absolument à ce qu'ils soient d'une parfaite qualité.
Comme il importe de conserver au vin tout son esprit et son principe odorant, lesquels deux principes sont volatils, il faut couvrir avec soin, la cuve dans laquelle le moût fermente.
Quand le mouvement commence dans la cuve, plusieurs substances se disposent à monter à la superficie; elles y montent ensuite\setcounter{page}{90} réellement, et y forment ce qu'on appelle le chapeau ( il cappello). Si la cuve n'étoit pas couverte, ce chapeau seroit en contact avec l'air ambiant, lequel appauvrit le vin en emportant une partie de l'esprit et du fumet.
L'auteur donne ici, dans le plus grand détail, la manière d'écraser le grain. Il est difficile de faire comprendre tout-à-fait cette opération, sans les gravures des instrumens employés. Le but est de ne laisser, s'il est possible, échapper aucun grain à l'écrasement, avant de mettre le moût dans la cuve. Pour cela, deux hommes, après s'être lavé les pieds, se placent dans un grand vase de bois, dont le fond est un parallélogramme et un peu incliné du côté où l'on écoule à volonté, au travers d'un panier d'osier, lequel retient les grappes et les semences. Les raisins choisis sont déposés sous les pieds de l'homme placé le plus haut, puis glissent peu-à-peu sous les pieds de l'autre. Tous deux écrasent en trépignant, et terminent leur opération en lavant leurs pieds et les parois du vase avec le premier moût non coloré qui s'est écoulé. L'auteur insiste sur l'importance de cette opération de l'écrasement, comme ajoutant au vin une couleur et une qualité qui paient largement l'augmentation du travail.\setcounter{page}{91} Tout; est lié dans les arts, fondés sur des principes certains. Il ne suffit pas d'avoir bien vendangé, et bien écrasé le raisin, il faut encore bien fouler, c'est-à-dire, bien mêler la masse déposée dans la cuve, pour que la fermentation s'y établisse partout également. Trois quarts d'heure d'agitation du fouloir (espèce de bâton de perroquet) sont nécessaires pour bien mêler le moût d'une cuve, et y suffisent; c'est-à-dire, qu'on ne doit point y revenir, de peur d'occasionner la déperdition des substances volatiles qui font la qualité du vin.
Toutes les fois qu'on sent à quelque distance des cuves, une odeur de vin ou de vinaigre, on peut être assuré qu'il y a eu quelque faute dans les procédés. Lorsqu'ils ont été soigneusement exécutés, on sent à peine l'odeur d'esprit-de-vin dans la cuve elle-même.
Avant de commencer l'opération de l'écrasement du raisin, il faut en avoir assez en provision pour remplir une cuve entière. Sans cela, on seroit exposé à voir fermenter le moût dans la cuve, avant que celle-ci fût pleine. C'est en ajoutant ainsi du moût nouveau à une cuve qui commençoit à fermenter que l'on gâte beaucoup de vin. Il faut avoir\setcounter{page}{92} soin de ne jamais remplir tellement la cuve, que lorsque la fermentation fait monter le vin, il puisse dépasser les bords. A mesure que le moût fermente dans la cuve, il soulève un couvercle mobile, formé des substances solides du raisin, et qu'on nomme le chapeau. Celui-ci sert à retenir, autant qu'il est possible, l'esprit et le fumet de la liqueur. Lorsque le vin est fait, les parties qui composoient le chapeau tendent à se précipiter au fond de la cuve.
Un couvercle en bois, placé sur le chapeau et qui laisse un espace vide, de deux pouces entre ses bords et les parois de la cuve, remplit l'objet de conserver les principes volatiles précieux, en laissant échapper le gaz acide carbonique, à mesure qu'il se dégage.
Ce couvercle doit être suspendu à une corde fixée dans son centre, et passant dans une poulie. On le suspend de manière à ce qu'il porte sur le chapeau. Quand celui-ci commence à s'abaisser, parce que la fermentation diminue, il se détache du couvercle, lequel demeure suspendu à une hauteur fixe. Les avantages de ce couvercle sont : 1°. une fermentation très-régulière, 2°. une diminution notable dans l'acidité du moût, et l'impossibilité pour le chapeau de se sécher à l'extérieur.\setcounter{page}{93} térieur. 3°. Une évaporation presque nulle, au lieu qu'elle est très-sensible si la cuve est découverte. 4°. La propreté du moût, parce qu'il est à l'abri des guêpes, des abeilles, etc.
Dans un cellier où il y a en fermentation plusieurs cuves couvertes, on sent à peine l'odeur du vin. Les gaz qui s'échapperoient chargés du principe odorant, rencontrent le couvercle, et laissent en grande partie ce principe dans le vin, lorsqu'ils s'échappent par les bords.
Si le couvercle ne repose pas immédiatement sur le chapeau, une partie de ces avantages sont perdus.
Des substances libérées par l'écrasement des grains, et de leur manière d'agir dans la conversion du moût en vin.
Nous avons vu que les diverses substances contenues dans le raisin, et qui forment le moût, sont le principe sucré, le levain, le tartre, l'acide, la substance colorante, le principe odorant et les substances solides ; tout cela dissous et mélangé avec une grande quantité d'eau, qui est la base du moût et du vin.
Il est utile de savoir comment ces substances agissent les unes sur les autres pen-\setcounter{page}{94} dant que le moût se convertit en vin : on peut ainsi, en quelque sorte, prévoir la qualité du vin qu'on obtiendra d'un certain moût.
Celle de toutes les substances qui s'altère le plus pendant la formation du vin, est la substance sucrée. Elle se convertit toute entière, par la fermentation, en esprit-de-vin. Si elle abonde, les vins ont de la force : si elle est en petite quantité, les vins sont foibles et plats.
Le levain est ensuite la substance qui s'altère le plus : cette altération qu'il subit est proportionnée à la quantité de substance sucrée sur laquelle il agit pendant la fermentation, pour la transformer en esprit-de-vin.
Lors donc qu'il y a beaucoup de substance sucrée, le levain est tout, ou à-peu-près tout, employé à convertir le corps doux en esprit. Dans les vins foibles où il y avoit déficit de substance sucrée, il reste une partie de levain en nature; et comme le levain est une substance singulièrement altérable, et toujours prête à exciter la fermentation, les vins foibles sont beaucoup plus sujets à s'altérer et à tourner à l'aigre, que les vins forts.
Le tartre ne subit aucune altération par le procédé de la fermentation; il se mêle\setcounter{page}{95} en partie aux lies, et demeure en partie dissous dans le vin. Il donne à celui-ci plus ou moins le goût de sec. Les raisins les moins sucrés sont ceux qui contiennent le plus de tartre.
L'acide contenu dans le moût s'altère peu ou point par la fermentation, et il demeure tout entier dans le vin. Il faut se souvenir que moins les raisins sont mûrs, plus ils contiennent de cet acide. Les moûts peu sucrés donnent des vins acides. A mesure que le corps doux se change en esprit, l'acide devient plus sensible, à-peu-près comme si l'on enlevoit le sucre dissous dans la limonade, l'acide du citron deviendroit plus marqué. La substance colorante abonde dans la proportion de la maturité des raisins; et la couleur devient plus forte et plus fixe, à mesure que le principe colorant se dissout dans l'esprit-de-vin qui se forme.
Le principe odorant, ou le fumet, existe tout entier dans le raisin et dans le moût; à mesure que le vin se forme, il va perdant de sa densité. Les raisins de bonne qualité bien exposés, sont ceux qui ont le plus de principe odorant.
Les substances solides sont en plus grande proportion lorsque le raisin est mal mûr. Le seul effet de leur présence dans la cuve est\setcounter{page}{96} d'ajouter à l'efficace du levain et de la fermentation. Les vins communs sont meilleurs lorsque les substances solides sont restées dans le moût, que lorsqu'on les en a séparées avant la fermentation. Cependant, elles peuvent faire contracter au vin un mauvais goût, une mauvaise odeur, si les raisins n'ont pas été nettoyés de tous les grains pourris ou moisis, secs ou verts, ou s'il arrivoit qu'on foulât le moût de la cuve quand le chapeau seroit altéré à la superficie.
L'eau est en quantité plus grande qu'aucune autre substance dans le moût. Elle n'éprouve aucune altération pendant que celui-ci fermente. Elle tient divisées et dissoutes les substances solides qui étoient contenues dans le raisin; et la liquidité est nécessaire à toutes les opérations qui se passent dans la cuve.
Plus le raisin est mûr, moins il y a d'eau dans le moût, et vice versa.
Le calorique qui se trouvoit combiné avec toutes les substances, devient de plus en plus sensible quand la fermentation s'établit,
(La suite au Cahier prochain.)
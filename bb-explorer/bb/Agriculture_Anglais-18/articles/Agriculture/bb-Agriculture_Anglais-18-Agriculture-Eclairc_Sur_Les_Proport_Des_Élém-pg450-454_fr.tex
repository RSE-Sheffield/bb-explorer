\setcounter{page}{450}
\section{SUR LES NOMBRES par lesquels Mr. DAVY représente les élémens et leurs composés\footnote{Voyez dans le volume XLVI, Sci. et Arts de la Bibl. Brit., p. 38, un aperçu de la théorie de Dalton sur la composition chimique. Celle que Mr. Davy a embrassée a beaucoup d'analogie avec celle de Dalton, ou plutôt c'est la même théorie, avec quelques modifications.}}
PLUSIEURS physiciens, au nombre desquels se trouve Mr. Davy, adoptent maintenant l'opinion, que lorsque les substances chimiques se combinent pour former de nouveaux composés, elles se combinent toujours dans des proportions déterminées. En sorte que si deux corps s'unissent en proportions qui ne soient pas égales, et que l'un des corps soit en excès, cet excès est toujours dans un rapport qui peut s'exprimer\setcounter{page}{451} par quelque multiple simple de la plus petite proportion dans laquelle ce même corps puisse se combiner. Ainsi, par exemple, le sur-oxalate de potasse contient deux fois plus d'acide qu'il n'en faut pour saturer la potasse; le sous-carbonate, au contraire, contient deux fois plus d'alkali qu'il n'en faut pour saturer l'acide.
Maintenant, si l'on fait usage d'un nombre pour exprimer la plus petite quantité d'un corps quelconque qui puisse entrer en combinaison, toutes les autres quantités de ce corps qui se combineront seront des multiples de la première: les plus petites proportions dans lesquelles les corps non-décomposés peuvent entrer en combinaison étant connues, la constitution des composés qu'ils forment pourra être connue: et si nous exprimons par l'unité l'élément qui s'unit chimiquement aux autres corps dans la plus petite proportion connue, tous les autres éléments et leurs composés pourront être représentés par les rapports de leurs quantités avec l'unité.
De tous les éléments connus, le gaz hydrogène est celui qui se combine en plus petite quantité, car nul élément ne peut se combiner sous un poids plus petit que ce gaz: ce même gaz ne paroît pas entrer dans
Agricult. Vol. 18. No. 11. Nov. 1813 M m\setcounter{page}{452} aucun corps connu en moindre proportion que dans l'eau. L'eau est un composé de deux volumes de gaz hydrogène et d'un volume de gaz oxigène; et la pesanteur spécifique du gaz oxigène étant à celle du gaz hydrogène :: 15 : 1, le rapport de l'hydrogène dans l'eau est donc à celui de l'oxigène :: 2 : 15. Maintenant Mr. Davy admet que l'eau est un composé de deux proportions d'hydrogène et d'une d'oxigène, par conséquent, le nombre qui représente l'hydrogène étant 1, celui qui représentera l'oxigène sera 15 \footnote{Mr. Dalton admet que l'eau est un composé d'une proportion d'hydrogène et d'une d'oxigène; aussi en prenant cette proportion d'hydrogène pour l'unité, le nombre qui représentera l'oxigène sera 7,5 ou 7 si la pesanteur spécifique du gaz oxigène est à celle du gaz hydrogène :: 14 : 1. Les expériences de Mr. Davy lui font admettre, pour la pesanteur spécifique de ces gaz, le rapport de 15 : 1.}.
Voilà la base des calculs de Mr. Davy. S'agit-il maintenant de connaître le nombre qui représente une substance élémentaire, le phosphore, par exemple? On sait par expérience, que dix grains de phosphore, demandent, pour être convertis en acide phosphoreux, 7,7 grains, et que la quantité de phosphore étant la même, il faut deux fois plus d'oxigène pour le convertir en acide\setcounter{page}{453} phosphorique : 7,7 sera donc la plus petite proportion d'oxigène avec laquelle le phosphore puisse se combiner, et l'on dira : 10 : 7,7 :: 100 : 77, puis : 77 : 100 :: 15 (nombre qui représente l'oxigène) : x = 20, à-peu-près. 20 sera donc le nombre qui représentera le phosphore ; 20 + 15 = 35, l'acide phosphoreux ; 20 + 15 + 15 = 50 l'acide phosphorique. Veut-on connaître le nombre qui représente le carbone : les expériences sur l'oxide gazeux de carbone et sur le gaz acide carbonique nous démontrent que le dernier gaz est composé de 13 parties de carbone et de 34 d'oxigène, et que le gaz oxide est composé de 13 de carbone et de 17 d'oxigène. La plus petite proportion d'oxigène étant donc 17, nous dirons : 13 : 17 :: 100 : 130,7. 130,7 : 100 :: 15 (nombre qui représente l'oxigène) : x = 11,4, à-peu-près. 11,4 sera donc le nombre qui représentera le carbone ; 11,4 + 15 = 26,4 l'oxide de carbone ; et 11,4 + 15 + 15 = 41,4 le gaz acide carbonique. Donnons encore deux exemples, pris parmi les métaux. 100 parties de potassium s'unissent en poids à 20,1 d'oxigène pour former la po-\setcounter{page}{454} tasse pure, et à 57,8 pour former l'oxide jaune de potassium. On prendra 20,1, qui est la plus petite proportion, et l'on dira : 20,1 : 100 :: 15 : x = 74,99 ou 75, nombre qui exprime le potassium. On observera en outre, que 57,8 est bien près de 60 ou de 3 x 20, d'où l'on pourra conclure que la potasse est un composé d'une proportion d'oxigène avec une de potassium, et l'oxide jaune, de trois d'oxigène avec une de métal. La potasse sera donc représentée par 75 + 15 = 90 ; et l'oxide jaune par 75 + 45 + 15 = 120. Prenons un autre exemple, dans lequel la donnée est tirée du péroxide. Le péroxide de plomb contient 3 à 3,5 d'oxigène de plus que le minium. Le premier oxide de plomb, ou le massicot, sur 100 parties de plomb en contient 7,52 oxigène ; le second oxide, le minium, 11,5 environ, et l'oxide pur ou le péroxide, 15 : on peut fixer la plus petite proportion d'oxigène à 3,76 ; et nous dirons : 3,76 : 100 :: 15 : x = 398 ; nombre qui représente le plomb. Alors le massicot contiendra deux proportions d'oxigène, le minium trois, et le péroxide ou l'oxide pur quatre ; et ces oxid seront représentés respectivement par 398 métal, plus 30, 45, et 60 oxigène.
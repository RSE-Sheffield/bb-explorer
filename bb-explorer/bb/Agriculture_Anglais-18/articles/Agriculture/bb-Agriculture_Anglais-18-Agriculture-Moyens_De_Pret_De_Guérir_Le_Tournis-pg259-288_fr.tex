\setcounter{page}{259} fermant convenablement le crâne, après le trépan. Je recommande donc de préférence les deux manières d'opérer que l'expérience m'a appris être efficaces, et à la portée de tous les bergers. J'espère que ce que j'ai dit engagera ceux qui ont des intérêts de troupeaux, à s'occuper de prévenir cette maladie, et de guérir par l'opération les animaux attaqués. Il y a des fermiers et des bergers qui regardent avec indifférence la perte annuelle de quelques agneaux que le tournis leur enlève, parce qu'ils ne croient pas qu'il y ait de remède, mais à la longue, ces pertes accumulées deviennent un objet important.
JAMES HOGG.
\section{OBSERVATIONS, etc. Observations sur la culture et l'usage des pommes de terre.}
Rapport du Comité du Département d'Agriculture, nommé pour extraire ses informations des rapports des Comtés.
Des différentes sortes de pommes de terre.
DANS le comté de Lancastre, on cultive pour le bétail l'espèce dite bœuf noble (ox noble) et la grappe (cluster.) L'ancienne
* *\setcounter{page}{260} rouge d'hiver (old winter red) est particulièrement bonne au printemps, quand les autres sortes ont perdu leur saveur. Cette sorte n'éprouve jamais la maladie de la frisolée.
Dans l'arrondissement du nord du comté de York, l'ox-noble, la champion, et la surinam, sont cultivées, mais plus encore la rognon (kidney.)
Dans l'île de Man, les sortes les plus employées sont la rognon, qui est bonne; mais qui donne peu et ne se garde pas bien; la blanche et la pomme (white and apple) qui sont les meilleures pour le commencement de la saison, les yeux-roses (pink eyes) et les feuilles de cuivre (copper-plates), qui sont robustes et demandent peu de soin, enfin les noires qui se gardent jusqu'en août.
Dans le comté de Mid-Lothian on obtient quelques fois les pommes de terre de graines. La première année les tubercules ne sont pas plus gros que des noix: à la seconde ils ont leur grosseur. On obtient plusieurs variétés différentes de la même semence; et c'est de cette manière qu'on a des produits plus considérables. Les rognons sont les meilleures pour le goût; elles donnent beaucoup dans les bonnes\setcounter{page}{261} terres, mais presque rien dans les mauvaises. On évite la frisolée en changeant les tubercules d'un canton à l'autre pour planter. Dans West-Lothian la rayée-pourpre ( purple streaked ) donne les plus grands produits. C'est en plantant celles qui sont provenues de graine qu'on obtient le plus surement une sorte déterminée. A la seconde année qu'on les plante, elles sont dans leur perfection. Les variétés sont nombreuses et encore incertaines à la première année après le semis. La surinam blanche est préférée à la rouge. Dans East Lothian, les blackamoars et les killamancas sont maintenant les plus productives. On n'a jamais réussi en essayant de renouveler une sorte par les semis. Dans le même comté, la surinam rend trente pour cent de plus qu'aucune autre. Dans le comté de Selkirch, la rognon et la ronde blanche ( round-white ) sont estimées les meilleures et les plus farineuses ; ce sont aussi celles qui rendent le plus. Dans le comté de Roxburgh, on s'est bien trouvé de changer les tubercules pour planter, en les prenant dans un sol humide et froid, et en les transportant dans une terre légère et chaude, et le contraire. Dans le comté de Ayz on préfère la ronde rouge et la ronde blanche ( round red and\setcounter{page}{262} round white ) soit pour le goût, soit pour le produit.
En 1784. La rognon nez rouge ( red nose kidney ) a été mise de côté, parce qu'elle étoit toujours attaquée de la frisolée. La champion ne l'est jamais.
Mr. Turner de Suffolk a eu un champ planté moitié rognon nez rouge, et moitié de la blanche d'Aylesbury. La première a eu la frisolée, et la seconde a été très-vigoureuse.
Mr. Lord en Suffolk a planté la rognon rouge et la rognon blanche. Celles-ci furent toutes malades de la frisolée, et les plantes de la rouge n'en souffrirent pas du tout. Il planta aussi la rognon-nez-rouge, et l'œil-de-faisan ( pheasant eye) : les premières furent toutes attaquées de la frisolée, et pas une plante des dernières.
Mr. Buck de Suffolk a planté la hollandaise droite ( dutch upright ) et le nez-rose ( pink nose ) , la dernière frisola et non pas la première.
Mr. Pitt de Staffordshire, a essayé la champion hâtive, qui ne frisole jamais, mais ne rend pas beaucoup, la blanche d'Aylesbury qui est très-grosse et donne de grands produits ; la bœuf-noble qui donne beaucoup, et est très-grosse ; mais souvent creuse ; et enfin\setcounter{page}{263} celle de surinam, qui donne plus que toutes les autres, et est bonne pour le bétail.
Mr. Billingsley de Somerset a essayé la surinam, la bœuf-noble et la jambe-de-cheval (horse-leg). Il ne les a pas trouvées si nourrissantes que les autres sortes.
Mr. Woolward de Suffolk, estime que la trouveuse d'or, (gold-finder) qui est jaune en-dedans, est la meilleure sorte.
Mr. Young a comparé les sortes suivantes, même terrain, même culture.
\comment{table}
La grappe a rendu par acre . . 360 bushels
La rognon . . . . . . . . . . 144
La Tags dorée (golden tags) . . 207
Dans le comté d'Edimbourg la rouge tâchée (red-neb) est la plus hâtive de toutes : elle est mûre dans la dernière semaine de juillet.
Mr. Thomas Beevor a comparé les sortes suivantes, savoir les quatre premières dans un bord de jardin, après une rangée de pommiers qui avoient été arrachés un mois auparavant ; et les trois dernières sortes dans une terre de jardin.
\comment{table}
L'incomparable a rendu, à raison de, par acre . . 692 bushels
La dennes-hill . . . . . . . . . . . 668
La bayley . . . . . . . . . . . . 539
La manley blanche . . . . . . 670\setcounter{page}{264} Celle de Kent . . . . . . 1342
La champion . . . . . . . 708
La bœuf-noble . . . . . 1140
Mr. Whyn Baker en Irlande, a trouvé dans une expérience comparative les produits suivants :
La noire a rendu par acre . . . 111 barils
La quaker wise . . . . . . 108
La rouge Française . . . . . 88
La blanche Française . . . . . 85
La commune . . . . . . . 103
La pomme . . . . . . . 76
L'Anglaise blanche . . . . . 83
La munster blanche . . . . . 79
L'Espagnole . . . . . . . . 70
La crone . . . . . . . . 60
Mr. Hassal, de Narbeth, dit que les sortes de pommes de terre qu'il a trouvées les plus avantageuses pour la consommation, sont la rognon blanche et la pomme que l'on cultive avec grand succès dans les comtés de Wexford et Wicklow en Irlande : ces sortes produisent en grande abondance, sont fermes et farineuses, agréables au goût, et ne prennent point de saveur déplaisante à l'approche de l'été, comme plusieurs autres sortes.
Mr. Townley de Belfield indique les hâtives blanches et rouges, la rognon blanche\setcounter{page}{265} la platte, la vraie Espagnole, pour l'usage de l'été, comme il préfère pour l'usage de l'hiver la russet blanche et rouge, la Tag dorée, la Dum d'Irlande et la blanche unie d'hiver. Pour les terres noires, il indique la bleue Irlandaise et la vieille rouge Anglaise, qui sont robustes et ont une peau épaisse. La blanche de Lancastre devient très-grosse et donne beaucoup.
La royale, ou hâtive de Cumberland est très-grosse, donne abondamment, a un excellent goût, et mûrit assez-tôt pour qu'on puisse employer le terrain à une seconde récolte de pommes de terre, ou à des pois, ou à des turneps, ou à des choux. Ces circonstances en font une excellente acquisition; et il est probable que cette sorte remplacera peu-à-peu les autres. Ce qui donne à cette pomme de terre un avantage décidé, c'est qu'elle est prête à consommer dans un temps où les denrées sont le plus chères, c'est-à-dire, à la veille de la moisson.
Mr. Parkinson de Duncaster a comparé les produits des sortes suivantes:
\comment{table}
La noire a produit par acre . 1000 pec.
La blanche . . . . . . . 800
La rognon . . . . . . . 900
La champion . . . . . . . 1000\setcounter{page}{266} La bœuf-noble 1200 peç
La noire est plus pesante que la blanche, et fournit plus de fécule.
Observations.
Telles sont les autorités que le Département d'Agriculture a pu rassembler parmi les auteurs qui se sont occupés directement de cet objet. La recherche de la connaissance précise des différentes sortes qui portent un même nom, offre de grandes difficultés; mais dans l'état présent des connaissances, il n'y a pas de moyen d'y échapper. A l'avenir, il seroit fort utile que les agriculteurs pratiques s'occupassent de décrire la plante de chaque sorte, dans toutes ses parties, et d'indiquer toutes les circonstances qui peuvent promettre des distinctions exactes. Un autre objet qui mérite attention, et qui jusqu'ici n'a pas été suivi, c'est de déterminer par des expériences directes, quelles sont les sortes qui réussissent le mieux dans un terrain donné.
\section{Des diverses préparations pour la culture des pommes de terre.}
1 Gazon rompu.
2. Tourbe de montagne.\setcounter{page}{267} 3 Bois défrichés.
4 Jeunes plantations.
5 Avec turneps.
6 Dans les coins des champs.
7 Choix du sol.
8 Fumure.
9 Ecobuage.
10 Rotation.
\section{Sur le gazon rompu}
Dans le comté de Lancaster, où l'on entend merveilleusement la culture des pommes de terre, le gazon rompu passe pour la préparation la plus sûre d'une belle récolte.
Le colonel Mordaunt planta des pommes de terre sans fumier, de la sorte nommée grappe. Il montra sa récolte, en l'arrachant, à des agriculteurs intelligens, qui convinrent n'en avoir jamais vu de si abondante.
Mr. Bower, de Nottingham, rompt le gazon à la charrue, et plante de trois raies l'une.
A Warton, dans le Cheshire, on loue fréquemment aux journaliers de vieux prés-gazons pour y mettre des pommes de terre. Le prix de ferme est de deux shillings la perche, ce qui revient à seize livres sterling l'acre. On ne donne point de fumier.\setcounter{page}{268} En Suffolk, Mr. Kirby a recueilli quatre cents bushels par acre, sur un vieux pré sans fumier.
A Knutsford, en Cheshire, on a recueilli cinq cents bushels par acre sur un vieux pré rompu à la bêche.
\section{Observations.}
En Irlande, il est d'usage d'affermer de vieux prés aux pauvres journaliers, pour les rompre et y planter des pommes de terre. On y emploie la charrue et la bêche. Il ne paraît pas que les produits de la première méthode soient aussi forts que ceux de la seconde ; mais on manque, sur ce point, d'expériences comparatives.
\section{Culture des pommes de terre dans la tourbe.}
Dans le comté de Dumbarton on trouve beaucoup d'avantage à défricher les tourbes, et terrains tourbeux, par la culture des pommes de terre. On commence par couper la broussaille et ôter les grosses pierres, puis on plante en lazy beds\footnote{Nous avons rendu compte de cette méthode Irlandaise, dans nos premiers volumes. Elle consiste à diviser le gazon en bandes parallèles. Les bandes qui répondent aux nombres impairs reçoivent les tubercules, qu'on dépose sur le gazon; les bandes des nombres}.\setcounter{page}{269} Dans le West Lothian, on observe que dans les situations élevées, les produits sont plus considérables que dans les parties basses où la culture est meilleure. Dans l'East Lothian, on a observé que les récoltes sur les hauteurs sont plus fortes que celles dans la plaine, à la proportion de 60 pour 45.
Le baron de Rutherford, en Roxburghshire, a trouvé, par expérience, que les pommes de terre étoient la meilleure récolte pour opérer un défrichement.
Dans le comté de Dumbarton, Sir James Colquhoun a mis en culture un terrain tourbeux en plantant des pommes de terre en lazy beds. Il leur a fait succéder de l'avoine, avec laquelle il a semé de la houlque laineuse, qui prend très-bien dans la tourbe. Cela se fauche, parce que le marais est mouvant, et ne peut pas être pâturé.
Dans les montagnes d'Ecosse, les plus belles pommes de terre croissent dans la tourbe, et en lazy beds.
Dans le Perthshire, on trouve que les tourbes desséchées sont le meilleur sol pour les pommes de terre.
bres pairs, sont labourées à la bêche, et jetées, à trois reprises, sur les autres, à mesure que les pommes de terre poussent. Il n'y a ainsi que la moitié de la pièce qui soit ensemencée. (R)\setcounter{page}{270} Trois acres de terrains marécageux ( dit l'auteur de l'Irish Tour ) ont donné neuf cents bushels, sans fumier. Le même auteur dit, qu'à Charleville, les pommes de terre dans la tourbe échappent à la gelée, qui tue celles des autres terres sur les hauteurs.
Mr. Leslie en Irlande a brûlé et fumé des marais pour y cultiver des pommes de terre, qui lui ont rendu trois cent vingt bushels par acre, après quoi il a eu un pré excellent.
A Mecra, en Irlande, les plus fortes récoltes de pommes de terre sont dans les marais : ils donnent cinquante bushels de plus que les prés. C'est de beaucoup la meilleure manière de mettre en valeur les marais tourbeux; mais il faut un peu de fumier.
Mr. Irwin, en Irlande, a essayé de mettre en valeur une partie de marais sur une montagne. Il a très-bien réussi, et a eu les plus belles pommes de terre du pays.
Mr. Browne, en Irlande, a défriché vingt acres de terre tourbeuse de bruyère, qui ne rendoit rien, et qui aujourd'hui rend 15 shillings l'acre. Il marna à cent cinquante barrils par acre, avec de la marne blanche trouvée sous la tourbe, et laissa reposer un an. Il tua ainsi la bruyère. Il laboura deux fois, et prit deux récoltes de\setcounter{page}{271} \section{CULT. ET USAGE DES POMMES DE TERRE.}
pommes de terre sans fumier, la première, extraordinaire; la seconde, pas mauvaise; puis il mit du grain trois ans de suite. Lord Altamont, en Irlande, amenda un terrain tourbeux, de montagne, avec du gravier calcaire. Il lui en coûta 40 shillings l'acre. Il le laissa deux ans, puis le loua à 40 shillings aux indigènes pour planter des pommes de terre. Il y prit ensuite trois belles récoltes d'avoine, sema des graines de prés avec la dernière, puis le loua à 16 shillings. Il amenda avec du sable coquillier une autre grande pièce de tourbe, puis l'écobua, et y sema des turneps, qui furent fort beaux. Il fuma, et mit des pommes de terre, qui donnèrent prodigieusement. Il eut jusqu'à cent quarante-trois tubercules à une plante. Il prit trois récoltes d'avoine, puis mit en trèfle blanc, qui vaut 20 shillings l'acre. Sur une autre pièce, affermée cinq shillings l'acre, il fit mener du gravier calcaire, laissa reposer trois ans, puis afferma pour des pommes de terre à 3 et 10 liv. st. Il y mit après cela, trois ans de suite de l'avoine, puis en fit un pré, qui vaut 30 shillings l'acre. À Moniva, en Irlande, on plante les pommes de terre dans la tourbe, qu'on a précédemment desséchée et amendée par un peu de gravier calcaire ou de fumier.\setcounter{page}{272} Mr. French à Woodlawn, en Irlande, a fait beaucoup d'expériences variées et en grand, pour s'assurer de l'avantage qu'il y a à défricher les tourbes par la culture des pommes de terre, en desséchant et en fumant. Il a obtenu un plein succès: il a eu des récoltes de 12 liv. sterl. l'acre.
Mr. Bland, d'Irlande, a défriché beaucoup de terres tourbeuses, en chaufant, fumant, et plantant des pommes de terre: il a eu deux fortes récoltes consécutives; et semblables.
Mr. Shanley, en Irlande, a obtenu sur de la mauvaise. tourbe, de quatre pieds de profondeur, douze cents stones de pommes de terre par acre, après desséchement, amendement de gravier calcaire, et fumure. Il prit ensuite deux récoltes d'orge, et mit en pré, qui vaut 40 shillings de fermage.
A Swinton en Yorkshire sur une tourbe noire, affermée quatre shel. 6 d. l'acre, on a obtenu 120 à 158 bushels.
Mr. Start à Brownsea, sur une tourbe noire de quatre shel. ½ de fermage, a recueilli 600 bushels par acre.
Observation. Les autorités ci-dessus, qui tendent à éclaircir ce sujet important, sont satisfaisantes et donnent\setcounter{page}{273} donnent lieu de croire qu'on peut entreprendre la culture des pommes de terre sur la tourbe avec espérance de succès.
\section{Culture des pommes de terre dans les bois défrichés}
Mr. Abedy, d'Essex, membre honoraire du Département d'Agriculture, défricha un bois, fuma à vingt charretées par acre, et planta des pommes de terre : il en eut 565 bushels par acre. Les dépenses furent de Liv. st. 16, 13 s. 6 d.
Observation.
Cette expérience, qui est la seule, peut être utile à ceux qui arrachent des bois, car il est probable qu'aucune autre récolte n'est plus profitable pour ce cas. Il seroit utile de savoir si le fumier est nécessaire.
Dans le Shropshire, lord Clive se trouve très-bien de permettre aux journaliers ses voisins, de planter des pommes de terre dans ses jeunes plantations de bois, à l'année qui suit l'établissement de ce bois. Lorsqu'il s'agit d'une terre neuve, on n'y met point de fumier pendant les deux premières années : cette culture n'est continuée que Agric. Vol. 18. No. 7. Juillet 1813,\setcounter{page}{274} trois ans, et est extrêmement utile aux jeunes arbres.
Mr. Coke, de Holkham, en Norfolk, permet aux pauvres de mettre des pommes de terre dans ses jeunes plantations d’arbres, et trouve que cette culture leur est utile.
\section{Plantation des pommes de terre avec les turneps.}
Mr. Walker de Norfolk a inventé une singulière méthode de culture. Sur une jachère destinée aux turneps, il plante en juin, des pommes de terre à la charrue. Il sème ensuite des turneps, et il herse, de manière que la culture au hoyau, sert à-la-fois aux deux plantes. Il fait consommer sur place les turneps de bonne heure en automne; et en labourant pour le froment, il arrache ses pommes de terre. Il a une pleine récolte de turneps, et 120 bushels de pommes de terre par acre, valant 5 L. st.
Mr. Bell, de Dumfries, plante ses pommes de terre en lignes, espacées de quatre pieds et demi, et après la culture à la houe, à la fin de juin, sème des turneps entre deux.\setcounter{page}{275} Culture des pommes de terre dans les coins des champs.
Dans le comté de Leicester, les fermiers abandonnent aux journaliers les coins des champs, pour planter des pommes de terre, qu'ils cultivent et nettoient dans les heures du matin et du soir, avant et après le travail de la journée. Les bordures des champs, et les revers de fossés donnent d'assez belles récoltes, sans fumier. Cette pratique, qui ne coûte presque rien au fermier, mérite d'être encouragée, comme très avantageuse aux journaliers.
\section{Choix du sol.}
Dans le comté de Selkirk, les pommes de terre ne réussissent jamais que sur les terrains secs: si dans les champs il y a des places humides, on est sûr d'y voir les pommes de terre peu abondantes et de mauvaise qualité. Arthur Young nous apprend dans son Tour d'Irlande, que dans les terres riches voisines du Shannon, qui sont argileuses, les pommes de terre ne réussissent pas.
Dans le Mid-Lothian, sur un grand champ de diverses qualités de sols planté en entier de pommes de terre, la récolte a été bonne\setcounter{page}{276} dans les parties sèches, médiocre dans les parties humides, nulle dans les endroits mouilleux.
Mr. Townley, de Belfield, a planté quatre yeux du même tubercule, de la sorte nommée grappe, dans quatre terrains différens, savoir :
Nº. 1. Une forte terre végétale. Produit . . . . . . . . . . . . 34 livres
2. Une bonne terre légère . . 29
3. Une bonne terre graveleuse. 19
4. Un terrain sablonneux . . . 15
Observation.
Le fait que les pommes de terre viennent plus abondantes dans les terres sèches paroît bien contesté; et ce fait-là a beaucoup d'importance pour cette culture.
Fumure.
Dans le comté de Chester on fume à raison de vingt jusqu'à quarante voitures, ou tons par acre \footnote{Le ton pèse 1642 liv. de l'ancien poids de marc. (R)}; et le fumier se répand avant le dernier labour.
En Cornouailles, on fume les pommes de\setcounter{page}{277} terre avec des varecs, du sable de la mer, et du fumier.
En Devonshire, le mélange de la tourbe avec la chaux ou le fumier produit un excellent amendement pour les pommes de terre.
A Aveley, en Essex, on fume à dix charretées par acre de fumier frais.
En Lancashire, on emploie le fumier tout frais ; mais les engrais tirés de grandes villes font plus d'effet.
Dans le West Moreland dix voitures de fumier par acre. On loue ensuite le terrain à raison de dix shillings la perche carrée \footnote{La perche est de seize pieds et demi anglais.}, pour y planter des pommes de terre.
Dans l'arrondissement du nord, on fume à raison de dix à quinze charretées par acre.
Dans le comté d'Elgin, il est rare qu'on mette du fumier aux pommes de terre : les récoltes donnent 184 bushels par acre.
Dans le Mid-Lothian on répand quelquefois le fumier par dessus le champ ; et si l'on veut avoir des pommes de terre savoureuses, on met le fumier à la récolte qui précède, mais le produit est moins grand.
Dans le Perthshire on obtient de fortes récoltes en employant des fougères et des\setcounter{page}{278} feuilles d'arbres, au lieu de fumier: on fait
le même usage du genêt hâché, et répandu
sur les lazybeds.
Dans l'isle de Man, on emploie les varecs,
pour l'engrais des pommes de terre.
Dans le Berwickshire, quinze à vingt-cinq
voitures de fumier par acre dans des sillons
de trois pieds de large ouverts à la charrue
à double versoir.
En Dumfriesshire, on emploie la chaux
pour prévenir la pourriture des tubercules,
accident qui étoit très-fréquent avant cet
usage, et n'a plus eu lieu depuis.
Dans le Sutherland, quand les varecs sont
employés à fumer les pommes de terres,
elles sont ordinairement coriaces ou aqueuses.
\section{Expérience de plâtre sur les pommes de terre, par Mr. Weston de Leicester.}
Je plantai mes pommes de terre en avril.
Pour la moitié de ma plantation, je fis faire
l'opération suivante. A mesure que les tubercules étoient coupés, on les plongeoit
dans le gypse en poudre, ensorte que la
tranche demeuroit couverte du gypse. Je les
plantai en lignes espacées d'un pied, et à
mesure qu'on déposoit un quartier, on mettoit une pincée de gypse par dessus. Enfin\setcounter{page}{279} lorsqu'on eut recouvert de terre les quartiers, on mit encore un peu de gypse dessus.
L'autre portion du terrain fut plantée de la même manière, mais sans gypse. Pendant la végétation, on n'aperçut pas de différence sensible ; mais à la récolte, il y eut une différence d'un tiers à l'avantage des plantes qui avoient été gypsées, et les tubercules étoient plus gros.
\section{Expériences comparatives sur divers engrais pour les pommes de terre.}
\comment{table}
N°. 1. récolte
53 Yards \footnote{Vingt-sept pieds anglais cubes.} cubes de fumier de basse cour . . . . . . 400 bus.
160 Bushels de suie . . . . . . 360
160 Idem cendres de bois . . . . . 240
32 Yards cubes de fumier . . . . . 280
42 Dits . . . . . . . . . . 360
Sans fumier . . . . . . . . 180
\comment{table}
Expérience N°. 2.
32 Yards cubes fumier et 40 bushels cendres de bois . . . . . 400
Sans fumier . . . . . . . . 280\setcounter{page}{280} 160 Bushels de chaux éteinte 380
1 1/2 Ton de paillé d'orge 300
340 Bushels de potasse 380
32 Yards cubes de fumier 400
32 Dits et 160 liv. de sel ajouté en semant 400
32 Dits et 160 bushels de chaux 480
32 Dits et 480 gallons d'urine 520
Mr. Billingsley a prouvé que la chaux, la marne, la craie, les cendres lessivées, et les chiffons faisoient fort peu d'effet. Il a bien réussi en enterrant les vesces et le trèfle en végétation. Le fumier de cheval bien pourri, à la quantité de vingt charges (de trois yards cubes chacun) étoit l'engrais le plus efficace. Le fumier de cochon vient ensuite, quant à l'effet.
A Peckenham en Irlande, les vieux prés amendés avec du gravier calcaire, s'affermissent jusqu'à 5 L. st. l'acre pour y planter des pommes de terre.
Les expériences de Mr. Young, sur l'effet des différens engrais appliqués aux pommes de terre ont eu les résultats suivans.
Récolte par acre.
\comment{table}
No. 1 Point de fumier produit 120 par acre 140
No. 2 Matières fécales 10 charretées 600 640
No. 3 Dites 6 650 500\setcounter{page}{281} \comment{table}
1re. année. 2e. année.
4 Dites . . . . . . . . 500 . . . . 300
5 Des os . . . . . . . . 10 . . . . 650 . . . . 640
6 Dits . . . . . . . . 6 . . . . 640 . . . . 560
7 Dits . . . . . . . . 2 . . . . 560 . . . . 240
8 Fumier de cheval . . . . 60 tomber. 480 . . . . 300
9 Dits . . . . . . . . 30 . . . . 480 . . . . 160
10 Compost de basse cour. 60 . . . . 300 . . . . 240
11 Dits . . . . . . . . 120 . . . . 480 . . . . 300
12 Dits . . . . . . . . 30 . . . . 140 . . . . 140
Mr. Barber a comparé le fumier, les débris des murs de torchis, et la paille pourrie. Un acre amendé en fumier a donné six cent trente-sept bushels; le torchis, trois cent vingt, et la paille pourrie, deux cent cinquante-cinq.
En Ecosse, on se sert des varecs avec grand succès: on les met dans la raie après la charrue. On fait succéder l'orge aux pommes de terre, puis l'avoine, puis les pommes de terre encore; et toutes les récoltes sont bonnes.
En Inverness-shire, on emploie la fougère comme engrais sur les lazy beds; mais les récoltes sont foibles. On trouve dans ce pays-là que les cendres de tourbe sont le meilleur engrais pour ces racines.
Mr. Greenhill, de Hamshire, affirme que la chaux nuit aux pommes de terre: elle les rend galeuses, et même les corrode.\setcounter{page}{282} AGRICULTURE.
Mr. Towneley, de Belfield, a comparé
les effets de différens engrais, et les résultats
ont été comme suit:
*Livres pesant.*
N°. 1 Cendres de charbon. Produit 211 assez petites.
2 Fumier d'étable et cendres de
charbon mélées . . . . . . . 344 très-belles.
3 Fumier d'étable seul . . . . 315 de même.
4 Point de fumier . . . . . . 134 très-petites.
5 Mélange de fumier, chaux, et
matières fécales . . . . . . 204 médiocres.
6 Fumier d'étable et tourbe jaune 438 extrêm. belles.
7 Cendres de savoneurs . . . 383 très-belles.
8 Fumier d'étable et chaux . . 268 passables.
9 Chaux seule . . . . . . . 187 de même.
10 Cendres de charbon et chaux . 192 de même.
11 Fumier d'étable et cendres de
savonier . . . . . . . . 298 très-bonnes.
12 Suie, matières fécales et cendres
de charbon . . . . . . . 271 très-bonnes.
13 Sel et matières fécales . . . 200 de même.
14 Sciures de bois et cendres de
charbon . . . . . . . . 190 petites.
15 Fumier d'étable et sciures de bois 307 très-belles.
16 Fumier de volailles et cendres de
charbon . . . . . . . . 236 assez belles.
17 Fumier de volailles et sable . . 156 assez petites.
18 Sciure de bois et chaux . . . 197 de même.
19 Roseaux pourris, avec de la chaux 208 très-bonnes.
20 Tan et chaux . . . . . . . 76 misérables.
21 Tan et fumier d'étable . . . 144 un peu plus grâ.
22 Tan seul . . . . . . . . 35 misérables.
23 Fumier d'étable , et chaux répandus sur le terrain . . . . . 230 assez belles.
24 Du genêt haché avec de la chaux
répandue par dessus . . . . 256 très-belles.\setcounter{page}{283} Chaque espèce d'engrais occupoit un billon de cinq pieds de large. Comme le fumier d'étable mêlé à la tourbe avoit eu l'avantage, on sépara les deux ingrédiens pour leurs effets comme engrais; et les résultats de l'un et de l'autre furent exactement semblables.
\section{Observations}
Il est difficile de tirer des conclusions précises de cette application de divers engrais à des terrains tout-à-fait différens. Ce n'est pas toujours l'inexactitude qui produit les contradictions apparentes et les difficultés; elles résultent de circonstances cachées, qui échappent à celui qui fait l'expérience. Le mot load ( charge ) est si vague, qu'il ne faudroit jamais l'employer sans explication de poids ou de mesure. Il y a un vaste champ ouvert aux expériences sur le chapitre des engrais qui conviennent le mieux aux pommes de terre; et cet objet mérite toute l'attention de ceux qui peuvent s'en occuper.
\section{Culture des pommes de terre par l'écobuage.}
L'usage ordinaire en Cornouailles est de planter à la fin d'avril ou au commence-\setcounter{page}{284} ment de mai, sur un écobuage; et le produit est ordinairement de quatre à cinq cents bushels par acre.
Auprès Anns-grove en Irlande, on fait grand cas de l'écobuage comme préparation aux pommes de terre. On obtient ainsi des récoltes certaines et abondantes, sur-tout sur les terrains abandonnés. On plante en mars et avril.
Sur les montagnes auprès de Clovnell, on écobue pour planter des pommes de terre grappes, et les récoltes sont très-fortes.
Observations.
Ces autorités sont en petit nombre, mais elles ont rapport à un objet d'une extrême importance. On n'a pas tenu registre des expériences; mais l'écobuage s'est pratiqué fort souvent pour les pommes de terre, et on a obtenu ainsi de fortes récoltes sur des terrains abandonnés et peu fertiles. On auroit besoin d'expériences directes, afin de déterminer la profondeur la plus convenable pour opérer l'écroutement, le point auquel il convient de brûler, et la meilleure manière de planter les tubercules.\setcounter{page}{285} \section{Des rotations dans lesquelles on fait entrer les pommes de terre.}
A Aveley, en Essex, on sème en automne sur le chaume des blés, des vesces d'hiver ou du seigle, pour faire manger au printemps. On plante ensuite des pommes de terre sur un labour. On les recueille en novembre. On sème au printems des pois, qui se vendent verts au marché de Londres, puis on met des turneps.
En Lancashire, on fait succéder les turneps aux pommes de terre. On y recueille aussi plusieurs années de suite des pommes de terre sur le même terrain, et avec succès.
Mr. Eccleston de Lancashire eut une énorme récolte de turneps après une forte récolte de pommes de terre. Il y mit ensuite du froment. Le même agriculteur eut en 1793 sa meilleure récolte de froment d'hiver, dans un champ où il succédoit à des pommes de terre, et quoiqu'il eût été semé le 20 mars seulement.
A Fulham et à Edmonton, en Middlesex, on sème le blé après les pommes de terre. A Battle en Sussex, on plante les pommes de terre après le blé bien fumé.
En West-Moreland, on préfère les planter sur les chaumes d'avoine.\setcounter{page}{286} A Larington en Wilt-shire, on pratique l'assolement suivant, 1°. pommes de terre; 2°. blé; 3°. orge; 4°. trèfle.
Dans l'arrondissement de l'est du comté de Yorck, on pratique l'assolement suivant: 1°. pommes de terre; 2°. blé; 3°. avoine, orge ou pois.
Dans l'isle de Man, sur cent quatre-vingts acres de terres arables, il y en a vingt-quatre en pommes de terre. Dans la même isle on a l'assolement suivant: 1°. pommes de terre; 2°. orge; 3°. trèfle; 4°. blé; 5°. avoine, ou pois.
Dans les terrains secs de Mid-Lothian,
1 Pommes de terre bien fumées.
2 Blé.
3 Trèfle.
4 Avoine.
Dans le comté de Penfrew,
1 Avoine sur gazon rompu,
2 Pommes de terre.
3 Avoine.
4 Prés artificiels.
Dans le comté de Roxburgh, on trouve que la seconde récolte de pommes de terre sur le même terrain est meilleure que la première en quantité, qualité et grosseur des pommes de terre.
Dans le comté de Selkirk,\setcounter{page}{287} 1 Pommes de terre.
2 Orge.
3 Prés artificiels.
Dans la vallée de la Tweed, on sème l'orge après les pommes de terre.
Mr. Yeald de Herefordshire a l'assolement suivant sur des terres argileuses :
1 Pois après trèfle.
2 Blé.
3 Pommes de terre.
4 Blé.
5 Avoine et trèfle.
Auprès de Sunderland,
1 Avoine sur pré rompu.
2 Pommes de terre.
3 Blé.
4 Turneps.
5 Orge.
6 Pommes de terre.
7 Blé.
8 Prairies pour cinq ou six ans.
Le Dr. Wilkinson d'Enfield plante les pommes de terre après du trèfle, et recueille quatre cents bushels. Il en plante aussi jusqu'à quatre années de suite sur le même terrain, en fumant toujours, et en obtenant des récoltes également fortes.\setcounter{page}{288} Dans le comté de Fife , on fait suivre les turneps aux pommes de terre , et on les trouve plus beaux que si les pommes de terre n'eussent pas précédé.
Dans le comté de Ross , on plante les pommes de terre sur les pâturages rompus, puis on met des pois, puis de l'orge, et enfin un pré artificiel ; et tout cela réussit bien.
\section{Observations.}
Les assolemens dans lesquels les pommes de terre peuvent entrer sont si variés, qu'on ne sauroit rien inférer de positif, du petit nombre d'expériences enregistrées. Ce ne sont que des indications de ce qu'on peut faire.
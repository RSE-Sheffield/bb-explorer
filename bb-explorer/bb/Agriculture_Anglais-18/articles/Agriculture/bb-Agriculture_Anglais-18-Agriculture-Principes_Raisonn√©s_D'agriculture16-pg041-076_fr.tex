\setcounter{page}{41}
\section{PRINCIPES RAISONNÉS D'AGRICULTURE. Traduit de l'allemand d'A. THAER, par E. V. B. CRUD. Tom. Ier. in-4°, 372 p. Genève, 1811, chez J. J. Paschoud, Imprimeur-Libraire; et à Paris chez le même rue Mazarine, n°. 22. \large{(Seizième extrait. Voyez p. 371 du vol. précéd.}}
PARMI les concrétions que la chaux produit par son union avec les acides, une des plus intéressantes pour l'agriculture est le gypse, plâtre, ou sulfate de chaux qui résulte de la combinaison de cette terre avec l'acide sulfurique. Le gypse est insipide et blanc, lorsqu'il est pur. Bucholtz dit qu'il faut 401 ½ parties d'eau pour en dissoudre une de gypse. L'art ne peut l'obtenir en cristaux, mais seulement en petits grains cristallins.
Une dissolution de cette petite quantité de gypse dans l'eau, donne à celle-ci un certain goût difficile à décrire. Si l'on fait évaporer cette dissolution, le gypse se précipite.
Agriculture. Vol. 18. N°. 2. Févr. 1813. D\setcounter{page}{42} Les eaux qui contiennent de l'acide carbonique, dissolvent beaucoup plus de gypse que l'eau pure. Le contact de l'air, en lui faisant perdre la plus grande partie de cet acide carbonique, fait aussi déposer le gypse. Les eaux chargées de cette substance, sont impropres à plusieurs usages, mais très-fertilisantes dans les irrigations.
Selon Bucholtz, le gypse est composé de
Chaux ........ 33 pour cent.
Acide sulfurique .... 43
Eau de cristallisation ... 24
Il se peut qu'il existe des gypses de proportions différentes.
Le gypse, quoiqu'exposé à l'air, n'éprouve point d'efflorescence, c'est-à-dire, qu'il ne perd pas son eau de cristallisation. Soumis à l'action de la chaleur, il perd cette eau de cristallisation sans s'éclater. Ainsi calciné, il est tendre et s'écrase sous le doigt : on le nomme alors plâtre. S'il est réduit en poudre fine et mélangé avec de l'eau, il absorbe promptement ce liquide, se combine avec lui, et devient solide. Il se fait alors un dégagement de chaleur, mais moins sensible que pour la chaux. Si l'on y met plus d'eau que le gypse n'en demande pour se cristalliser, il reste quelques momens en bouillie, puis se durcit en masse. C'est sur cette pro-\setcounter{page}{43} priété que repose son utilité comme mortier, ou pour modeler.
Exposé à l'air, le gypse calciné absorbe peu-à-peu l'humidité de l'atmosphère, et se l'approprie comme eau de cristallisation. Son poids augmente, et il perd la faculté d'être employé comme mortier : si l'on veut lui rendre cette propriété, il faut le calciner une seconde fois.
Si l'on donne dans la calcination du gypse, un feu trop fort, on le brûle, et alors il ne peut plus servir, ni comme mortier, ni comme engrais. Pour décomposer le gypse par le feu, il faut le réunir à des matières combustibles, et porter la chaleur au degré d'ignition. L'oxygène de l'acide sulfurique se dégage alors; une partie du soufre est évaporée, et l'autre partie, unie à la chaux, forme du sulfure de chaux, (foie de soufre).
Il est vraisemblable qu'il se fait une décomposition du même genre, mais lente, lorsque le gypse se trouve réuni à des corps chargés de carbone, et qu'ils sont en putréfaction : la faculté que le gypse a d'amender les terres, tient peut-être en partie à cela. Lorsque les eaux séléniteuses, ou chargées de gypse, reçoivent des matières animales ou végétales en putréfaction; elles répandent une odeur de soufre.\setcounter{page}{44} Le gypse ne peut pas être décomposé par les alkalis, parce que l'acide sulfurique a plus d'affinité qu'eux avec la chaux; mais on opère cette décomposition du gypse par les carbonates alkalins, et par le moyen d'une double décomposition. Ces propriétés du gypse peuvent servir à expliquer ses qualités comme engrais.
On trouve le gypse par grandes masses, et souvent par montagnes entières. Le gypse pulvérulent se voit dans le voisinage des grandes masses. Le gypse compacte n'est pas très-dur, et il ne prend point de poli. On en trouve de diverses couleurs; mais le plus souvent il est blanc ou giis. L'albâtre appartient à ce genre : il est au gypse ce que le marbre est à la pierre à chaux; c'est-à-dire qu'il est à demi cristallisé, et qu'il est susceptible de prendre du poli. Quelquefois l'albâtre est coloré par les acides métalliques : exposé à l'air, il est susceptible de se déliter.
Le gypse cristallisé se trouve dans les lieux où il y a du gypse en pierre. On peut le fendre au couteau en lames minces, molles et transparentes.
On le trouve aussi en stalactites, formées par le dépôt des eaux carbonatées qui tenoient le gypse en dissolution. On trouve enfin le gypse dans la cendre de quelques végétaux;\setcounter{page}{45} mais on est en doute s'il ne s'y forme point pendant la combustion, et par la combinaison de l'acide sulfurique avec la chaux.
\section{De la Marne.}
Cette substance est connue depuis le temps des Romains, comme propre à l'amélioration des champs; mais l'opinion des agriculteurs n'est pas encore fixée sur son emploi. On a souvent attribué des effets nuisibles à cette substance, faute de la bien connaître, et pour avoir employé, au lieu de marne, une argile ferrugineuse, ou quelque espèce de terre qui ne convenoit pas au sol à amender.
La marne est une combinaison de carbonate de chaux avec l'argile. Ces deux substances s'y trouvent amalgamées si intimément, qu'on ne peut point distinguer à l'œil, ni au microscope, l'argile de la chaux. On ignore comment la nature fait cette préparation. Lorsqu'on a essayé, pour imiter la marne, de faire des mélanges de chaux et d'argile; on n'a réussi que très imparfaitement : en particulier, la propriété de se déliter à l'air et de se réduire en poussière, n'existe pas dans ces mélanges comme dans la marne naturelle.
L'argile et la chaux se combinent dans le marne en quantités très diverses : quelque\setcounter{page}{46} fois elles sont en doses égales : quelquefois c'est l'une ou l'autre qui prédomine. On appelle marne argileuse, celle qui contient une moitié et jusqu'à deux tiers d'argile. S'il y a plus des trois quarts d'argile, on appelle le mélange argile marneuse ou calcaire. Si c'est la chaux qui domine, et qu'il y en ait jusqu'aux deux tiers du tout, on l'appelle marne calcaire ; si la chaux dépasse les trois quarts, c'est une chaux argileuse.
Maintenant qu'on recherche la marne avec plus de soin, on en découvre presque partout en creusant la terre, et il y a peu de pays où l'on ne puisse extraire la marne avec profit.
On trouve surtout la marne en abondance dans le voisinage des montagnes secondaires ; elle y est disposée en bancs d'une grande étendue. Dans le pays plat, on y trouve le plus souvent cette substance par masses irrégulières. Certaines plantes donnent l'indication de la présence de la marne, à quelque profondeur au-dessous de la surface du sol. Le tussilage ou pas-d'âne, le tussilage des Alpes, la sauge glutineuse et la sauge des prés, végètent avec beaucoup de force sur les terrains qui contiennent de la marne ; et ces plantes peuvent servir de guides pour découvrir cette substance.\setcounter{page}{47} Lorsque le trèfle jaune abonde sur un terrain qui n'a pas été fumé, il est aussi une indice de la présence de la marne. On en trouve fréquemment sous les ronces. Lorsque dans les ravins ou dans les chemins creux, on voit l'argile mêlée de grains de chaux, il y a presque toujours de la marne à une plus grande profondeur. Ces couches de marne sont rarement homogènes dans toute leur épaisseur: il y a ordinairement moins de chaux dans la partie supérieure que dans celle qui est au-dessous.
L'argile et la chaux contenues dans la marne, y font un échange de leurs qualités réciproques. La ténacité de l'argile et son onctuosité sont modifiées par la chaux: la rudesse et l'âpreté de la chaux sont adoucies par l'argile. Le caractère de l'une ou de l'autre domine, selon que l'une ou l'autre est en plus grande quantité dans le mélange. Au reste, selon que l'argile qui est entrée dans la composition de la marne, est grasse ou maigre, la marne a plus ou moins le caractère argileux, quoique la dose d'argile y soit la même.
La marne varie beaucoup pour la couleur. Elle est blanche, jaune, jaunâtre, brune, grisâtre, violette, rougeâtre, grise, bleuâtre, noire, etc. Ces couleurs sont produites, soit par des oxides de fer et de manganèse, soit\setcounter{page}{48} par des matières combustibles, des bitumes, ou de l'humus. Les marnes qui sont colorées seulement par l'humus, sont ordinairement grises, bleuâtres, ou noires, et la combustion les rend blanches : celle qui est imprégnée de bitume, répand par la friction ou la combustion, une odeur qui lui est propre.
Au reste, la couleur de la marne n'est nullement un indice de sa qualité : tout au plus peut-elle servir à évaluer la quantité d'oxide métallique et de substances combustibles qu'elle contient. Des marnes de même couleur diffèrent souvent beaucoup dans les proportions de leurs élémens; et d'autres dont l'apparence extérieure est absolument différente, ont la même composition.
Les marnes varient cependant en consistance : quelquefois cette matière est molle et douce comme de la poussière, ou du moins se réduit en poudre sous le doigt; d'autrefois la marne est dure comme de la pierre : on nomme celle-ci concrète, et les autres terreuses. La contexture de la marne concrète est très-variable. Elle se montre quelquefois en couches superposées les unes aux autres, et qu'on sépare aisément au couteau : quelquefois elle se casse en morceaux irréguliers : on nomme celle-ci, marne pierreuse, par opposition à la première, qu'on\setcounter{page}{49} qualifie de schisteuse. Ces différences n'indiquent rien pour la composition de ces marnes : la marne concrète contient quelquefois plus d'argile que de chaux, et quelquefois c'est le contraire.
Lorsqu'on verse de l'eau sur la marne, cette eau pénètre dans les pores de cette substance, détruit la cohésion de ses parties, les sépare, et les réduit en poudre fine. C'est là une des qualités distinctives des marnes en général, et en vertu de laquelle ces substances améliorent le sol, en se mêlant avec la couche supérieure. Par l'immersion dans l'eau, l'air s'échappe de la marne et se développe en bulles. Toutes les terres qui se délayent dans l'eau, ne sont pas nécessairement de la marne; mais les terres qui ne se délayent pas, ne sont sûrement point des marnes. Elles ont également toutes la propriété de se déliter à l'air, et de s'y réduire en poudre : seulement les marnes pierreuses subissent cette altération avec plus de lenteur que les marnes terreuses. C'est par cette propriété que la marne est commode pour l'amélioration du sol. Il n'est point nécessaire de la pulvériser pour l'employer, puisque cette pulvérisation résulte toujours du contact de l'air. Pour les marnes concrètes, la gelée seule peut compléter la division des parties, et c'est par cette\setcounter{page}{50} raison qu'on préfère charier les marnes en automne : cette division est un effet de l'augmentation de volume que l'eau, absorbée par la marne, acquiert par la gelée.
Plus les proportions de la chaux et de l'argile se rapprochent du point d'égalité, et plus, en général, la pulvérisation de la marne est rapide : il y a cependant des anomalies qui tiennent à la nature de l'argile : si celle-ci est sèche, quoiqu'en dose égale à la chaux, la pulvérisation est plus lente que lorsque l'argile est plus grasse.
Les marnes schisteuses perdent plus promptement leur agrégation que les marnes pierreuses.
Si l'on traite les marnes avec les acides, il se produit une grande effervescence, pendant laquelle l'acide s'unit à la chaux. Lorsque toute la chaux a été ainsi absorbée, s'il y a une surabondance d'acide, il dissout aussi une portion d'alumine, et des oxydes métalliques,
Nous avons vu que l'argile seule est très-réfractaire, et que le carbonate de chaux seul l'est également ; mais ces deux substances réunies deviennent très-fusibles. La marne est donc une substance vitrifiable, et on l'emploie souvent comme fondant, pour obtenir le métal renfermé dans les gangues des mines de fer.\setcounter{page}{51} On rencontre souvent de la magnésie dans la marne, et c'est ordinairement dans celle dont les bons effets sont les plus marqués. Comme la magnésie s'y trouve dans un état de carbonate, elle fait également effervescence avec les acides ; ensorte que dans un examen superficiel des marnes, on confond souvent la magnésie et la chaux. On qualifie de marne magnésienne celle qui contient de cette substance. Il y a toujours du sable dans la marne. Lorsque la quantité du sable passe la moitié du tout, on l'appelle sable marneux. Le gypse se trouve aussi souvent dans la marne. Lorsqu'on la met sur des charbons, la présence du gypse se trahit par une odeur de soufre. On manque d'observations précises relativement à l'effet du gypse dans la marne.
\section{De la magnésie.}
Cette terre est moins répandue dans la nature, que les précédentes. On ne la trouve jamais pure, mais toujours combinée avec les acides, et mêlée d'autres terres. On la trouve dans l'eau de mer, combinée avec l'acide muriatique ; et dans les corps des animaux, combinée avec l'acide phosphorique. Les cendres de la plupart des végétaux\setcounter{page}{52} en contiennent; et quelquefois elle fait une partie considérable de la couche végétale ou de la marne.
Le carbonate de magnésie est insipide et inodore. Mouillé et mêlé avec de l'eau, il produit une matière peu liée, qui sèche promptement. Il se comporte comme le carbonate de chaux, quant à la propriété de retenir l'eau. Il est insoluble dans l'eau pure; mais il peut être dissous par l'eau imprégnée de gaz acide carbonique.
La magnésie pure n'est ni caustique ni alkaline. Elle ne produit point de chaleur, par son mélange avec l'eau. Ce mélange forme une bouillie, qui ne se durcit point en séchant; et si l'on y mêle du sable, elle ne forme pas du mortier. Elle paroît s'approprier l'eau, mais sans lui ôter de fluidité. Elle n'altère que fort peu les couleurs végétales. Les pierres qui contiennent de la magnésie, et qui sont savoneuses et grasses au toucher, sont les suivantes: la serpentine, qui est une pierre à grain fin d'un vert ou d'un gris foncé, et quelquefois avec de belles taches rouges. On la trouve souvent par couches, qui forment des montagnes entières. En Allemagne, la meilleure carrière de serpentine est à Zopplitz, en Saxe, où on la travaille en quantité à peine croyable. On\setcounter{page}{53} l'emploi sur le tour, où l'on en fait des tabatières, des boîtes, des vases, des chandeliers, des mortiers, que l'on polit ensuite avec un grès fin. Ses parties constituantes sont la magnésie, la silice et l'oxide de fer.
Le talc a une cassure lamelleuse, il est très-gras au toucher, on le trouve quelquefois sous la forme de terré, quelquefois sous celle de pierre. Le premier est composé de parties onctueuses, un peu luisantes, et pour l'ordinaire d'une couleur blanchâtre: ce dernier, au contraire, a de la consistance; il peut être divisé en lames minces, et a souvent le brillant de l'argent ou de l'or, ce qui fait qu'on le nomme argentin ou doré. Le talc est composé de quarante-quatre pour cent de magnésie, cinquante-six pour cent de silice et d'alumine.
La pierre ollaire est une variété de talc; elle a une couleur grisâtre ou vert foncé; on peut très-bien la travailler au tour pour en faire toutes sortes de vases. Elle se sépare en grandes masses, et on la trouve sur-tout en Suisse.
La pierre savoneuse est lisse, onctueuse, tendre, et elle tache ce qu'elle touche. Il y en a des variétés. La craie d'Espagne en est une. On peut s'en servir pour écrire sur le verre; et lorsqu'on a enlevé les traits par un\setcounter{page}{54} lavage, ils se montrent de nouveau quand la température est humide. On en trouve dans plusieurs parties de l'Allemagne.
L'asbeste est composé d'un tissu filamenteux. Si les fils sont parallèles et flexibles, on nomme aussi cette pierre amiante. Elle est ordinairement blanche verdâtre. C'est avec l'amiante qu'on prépare les mèches, le papier et la toile incombustibles.
L'écume de mer sert à faire des têtes de pipe. Autrefois on croyait que cette pierre étoit un produit de la mer: on la tire d'une mine en Natolie. Au sortir de la terre, c'est une pâte molle, qui ensuite se durcit à l'air. Les têtes de pipe faites de cette terre sont singulièrement légères, et peu fragiles. Selon Viegleb, elles est composée de parties égales de magnésie et de silice. On assure qu'il y a aussi de l'écume de mer en Espagne, en Hongrie, et dans l'Amérique septentrionale.
\section{Le fer}
Le fer se trouve souvent dans le sol, sous diverses formes. On l'y rencontre à divers degrés d'oxidation, de couleur blanche, verte, noire, et rouge. Mêlé ou combiné avec l'alumine, il donne à l'argile les diverses couleurs sous lesquelles elle se présente.\setcounter{page}{55} On ignore encore s'il a quelque influence sur la végétation. On a observé que le degré de son oxidation n'apportoit aucune différence dans la bonté du sol, ensorte que la couleur de celui-ci est indifférente, quant à l'influence que le fer a sur elle.
On trouve dans diverses glaises, l'oxide de fer en état de carbonate. Il ne paroît pas que, sous cette forme, il nuise à la végétation. Les forts acides font dégager le gaz acide carbonique avec effervescence, ensorte que ce phénomène n'est point un signe certain de la présence de la chaux ou de la marne.
On trouve aussi, mais moins fréquemment, le fer combiné avec l'acide sulfurique ou l'acide phosphorique: avec le premier, il produit la substance qu'on nomme vitriol (sulfate de fer.) Cette matière ne se trouve que dans des lieux où il y a des pyrites sulfureuses, dont la décomposition produit l'acide, qui se combine avec le fer: le plus souvent, c'est dans les marais tourbeux. Là où le vitriol se trouve en grande quantité, il est nuisible à la végétation, mais lorsqu'il est en plus petite proportion, et combiné avec des matières qui contiennent du carbone, il a une propriété fertilisante.
On trouve ordinairement le fer combiné\setcounter{page}{56} avec l'acide phosphorique, dans la matière qu'on appelle fer limoneux. Cette matière se délite et se mêle quelquefois avec la couche supérieure du sol. Les terrains qui reposent sur une couche de fer limoneux, sont toujours très-stériles. L'oxide de manganèse, qui entre quelquefois dans la couche supérieure du sol ne paroît avoir aucun effet sensible sur la végétation.
\section{De l'humus}
Nous avons parlé jusqu'ici des parties constituantes fixes, inépuisables et incombustibles du sol. Il reste à examiner une autre partie constituante de tous les terrains, et à laquelle tous doivent leur fécondité. On donne ordinairement à cette substance le nom de terreau. On a abusé de cette dénomination, lorsqu'on a désigné par là la couche de terre végétale, et non pas les substances particulières qui y entrent. La dénomination d'humus ne donne pas lieu à équivoque.
Si l'on en excepte l'eau, l'humus est la seule substance qui, dans le sol fournisse un aliment aux plantes. Il est le résidu de la putréfaction végétale et animale. C'est un corps noir, pulvérulent lorsqu'il est sec, mou,\setcounter{page}{57} mou et gras au toucher, lorsqu'il est humide. Quoiqu'il varie suivant la nature des corps qui l'ont produit, il a certaines propriétés inhérentes à sa nature. C'est un résultat de la force organique, une combinaison de carbone, d'hydrogène, d'azote et d'oxigène; combinaison qui ne sauroit être produite par les forces de la nature non organisée; car dans la nature morte, ces substances ne s'allient que deux à deux, et non pas toutes ensemble comme cela a lieu ici. A ces substances essentielles de l'humus, il s'en joint d'autres en plus petite quantité, savoir, du phosphore, du soufre, un peu de terre proprement dite, et quelquefois différens sels. L'humus qui est une production de la vie, est aussi indispensable à la vie individuelle, tout au moins pour les animaux et les végétaux les plus parfaits. C'est lui qui donne la nourriture aux corps organisés. Plus il y a de vie, plus grande est la production de l'humus. Chaque être organisé s'approprie durant sa vie, une quantité toujours croissante d'élémens naturels bruts; et en les travaillant en-dedans de lui produit enfin l'humus; ensorte que cette substance s'augmente d'autant plus que les hommes, les animaux, et les produits du sol, se multiplient dans une
Agriculture. Vol. 18. N°. 2, Févr. 1813. E\setcounter{page}{58} contrée. On n’a qu’à observer les progrès de la végétation sur les rochers nus, pour étudier l’histoire de l’humus dès le commencement du monde. Il s’y forme d’abord des mousses, dans la décomposition desquelles les plantes plus parfaites trouvent leur nourriture. Celles-ci, en se putréfiant, augmentent la masse de l’humus, et enfin la couche en devient assez considérable pour nourrir les arbres les plus vigoureux.
Voigt appelle l’humus un végétal en partie décomposé, mais pas entièrement désorganisé : c’est comme une vaste plante sans organisation, qui porte elle-même les autres plantes, et les nourrit. Le terreau végétal est composé de substances végétales, et il peut être de nouveau transformé en substances de même nature : souvent on le prépare soigneusement dans ce but.
L’humus a de l’analogie avec les corps dont il est produit, quant à la qualité de ses parties constituantes; mais ces parties y éprouvent un changement quant à leur quantité respective. Les substances élémentaires entrent dans une nouvelle combinaison, et il s’en évapore une partie. Suivant De Saussure, l’humus contient plus de carbone et d’azote, mais moins d’oxigène\setcounter{page}{59} que les végétaux dont il a été tiré. Les circonstances dans lesquelles l'humus se forme, ont probablement une grande influence sur les proportions de ses élémens, et sur les combinaisons de ses parties élémentaires. Ainsi, lorsqu'il s'est formé sous l'influence de l'air atmosphérique, il n'est pas précisément le même que lorsqu'il s'est formé dans un lieu clos et couvert. Ce fait est démontré, quoiqu'on n'ait pas encore des observations précises sur les différences et sur leurs causes.
Lors même que l'humus est déjà formé, il n'est point à l'abri d'altération et de destruction. Il est dans un état d'action et de réaction constante avec l'air atmosphérique. S'il est placé sous un récipient fermé avec du mercure, il attire fortement le gaz oxygène, lui donne du carbone et le change en gaz acide carbonique. Si le récipient est fermé avec de l'eau, il se fait un vide dans lequel l'eau s'introduit en absorbant le gaz acide carbonique : il se fait ainsi une consommation insensible d'humus. Il n'en est point de même du charbon de bois parfait : il faut donc que ce phénomène provienne de la combinaison particulière du carbone avec l'hydrogène et l'azote. C'est\setcounter{page}{60} vraisemblablement en produisant ainsi du gaz acide carbonique, que l'humus agit sur la végétation, soit directement soit par le moyen du sol, sur-tout lorsque la fane des plantes couvre fortement le sol, et par là empêche la trop forte évaporation de la colonne d'air enveloppée de gaz acide carbonique. De Saussure trouve que des plantes à moitié sèches, lorsqu'il les plaçoit sur de l'humus, ou sur une terre qui en étoit abondamment pourvue, se rétablissoient d'une manière évidemment plus prompte que lorsqu'elles étoient déposées sur un terrain maigre et humide. D'après les expériences faites sous le récipient, on peut calculer quelle énorme quantité de gaz acide carbonique doit se dégager d'un journal de terre riche en humus.
L'humus éprouve encore un autre changement que De Saussure nous a également appris à connoître d'une manière plus particulière. Il s'y forme une certaine matière qui est soluble dans l'eau et qu'on nomme matière extractive. On sépare cette substance en faisant bouillir à diverses reprises, avec de l'eau, l'humus qui a été exposé à l'air: en évaporant cette décoction on obtient pour résidu un extrait d'un brun noirâtre. Lors\setcounter{page}{61} que, par des coctions répétées, l'humus semble entièrement privé de cette matière soluble, et qu'on le met pendant quelque-temps en contact avec l'air, on peut de nouveau en obtenir de la matière extractive: si, au contraire, on conserve l'humus dans des vases fermés, il ne fournit plus de cette matière. Suivant De Saussure, l'humus ainsi privé de la matière extractive soluble, est moins fécond, et il contient proportionnément moins de carbone que celui qui n'a pas été soumis à la coction. De Saussure vit la matière extractive détrempée dans l'eau, passer immédiatement dans les racines des plantes. Il paroît donc que cette substance est, après l'acide carbonique, une des substances les plus propres à introduire les alimens et en particulier du carbone, dans les suçoirs des plantes. On n'obtient que peu de matière extractive de l'humus déjà ancien, par la pression simple, et à moins de lui faire subir une coction: on en obtient davantage de l'humus récent, ou mélangé d'engrais animaux. Cette matière extractive s'altère par l'exposition à l'air; il s'y forme une pellicule qui, lorsqu'on secoue le vase, se précipite en flocons, et est bientôt remplacée par une nouvelle. Ce précie\setcounter{page}{62} pité devenu insoluble dans l'eau, redevient soluble, lorsqu'on y joint un alkali. Une grande partie de l'humus que nous trouvons dans la nature paroît consister en cette substance qui a été ainsi séparée, et qui est devenue insoluble.
Les alkalis fixes dissolvent presqu'entièrement l'humus, et cette portion de matière extractive qui étoit devenue insoluble : pendant leur action, il se dégage de l'ammoniaque. Cette dissolution est décomposée par les acides, qui en précipitent une poudre inflammable, mais en petite quantité. L'alcohol ne dissoud pas l'humus : il en sépare seulement un peu de matière extractive et de résine.
L'humus n'est pas susceptible de putréfaction, proprement dite : il paroît, au contraire, être en opposition avec elle ; car la matière extractive peut entrer en fermentation putride, lorsqu'elle est séparée ; tandis qu'elle n'en est pas susceptible, aussi longtemps qu'elle est combinée avec les autres parties de l'humus. Cependant la végétation des plantes, et la formation de l'acide carbonique et de la matière extractive pendant l'exposition à l'air, consument entièrement l'humus, à la longue, si l'on ne remplace\setcounter{page}{63} par de nouveaux engrais les sucs absorbés par la végétation. S'il en étoit autrement, l'humus se seroit amassé à la surface de la terre, en beaucoup plus grande quantité qu'il ne s'y trouve en effet. Pour empêcher l'épuisement de l'humus, il suffit de rendre au sol, en engrais, une partie de ce que la végétation lui enlève, parce que celle-ci produit plus qu'elle n'absorbe. Si tout ce qui croît sur le sol, y étoit décomposé par la putréfaction, l'accumulation de l'humus devroit être très-considérable; et cela arrive, en effet, dans les forêts anciennes, et dans les plaines inhabitées dont la position est favorable à la végétation.
Selon les espèces de terrain auxquelles l'humus est incorporé, il se comporte d'une manière différente, et il produit des effets variés. L'argile, au moyen de sa ténacité, retient et protège contre l'influence de l'air atmosphérique, et par conséquent contre la décomposition, les particules d'humus qui y sont mélangées. C'est par cette raison, que pour se montrer fertile, il faut que cette terre soit mélangée de beaucoup d'humus, car les plantes ne peuvent pas étendre leurs racines dans tous les sens, avec autant de liberté, dans l'argile, que dans une autre\setcounter{page}{64} terre. Elle a donc besoin d'être beaucoup fumée, lorsqu'elle est mise en culture pour la première fois; mais si elle est imprégnée d'humus, elle demeure d'autant plus longtemps féconde sans avoir besoin de nouveaux engrais. Au reste, il paroît que l'humus se combine aussi avec l'argile intimément et chimiquement, de manière à perdre une partie de ses propriétés, sur-tout sa couleur noire. Nous avons analysé des argiles à-peu-près blanches, dans lesquelles on ne remarquoit aucune trace d'humus; soumis au feu d'ignition, ces argiles devenoient noires, et annonçoient par plusieurs indices qu'elles contenoient du carbone hydrogéné. A une plus grande chaleur, la couleur noire disparoissoit, et ces argiles perdoient considérablement en poids. Il arrive souvent que le sol arrosé dans les bas-fonds paroît tout-à-fait blanc, cependant sa grande fécondité fait supposer qu'il contient une forte proportion d'humus, ou des substances dont cet humus se compose. Dans les terrains ainsi arrosés, on trouve l'humus presque toujours intimément combiné avec l'argile: ce mélange est opéré par l'eau qui charrie l'humus, et le dépose sous forme de limon.\setcounter{page}{65} On ne peut attribuer au sable, sur l'hu-mus, qu'une action purement mécanique. L'absence de cohésion entre les parties du sable facilite à l'air atmosphérique la libre entrée dans toutes les parties de l'humus: il en résulte une plus prompte décomposition de celui-ci, parce que l'air et l'eau en séparent l'acide carbonique et la matière extractive.
Lorsque le sable est suffisamment mêlé d'humus, sans manquer d'humidité, le terrain est extraordinairement fertile; mais aussi sa fécondité est d'autant plus vite épuisée, parce que l'humus est promptement absorbé. On trouve dans les marais de l'Oder, des places où, sur le sable amassé par les eaux, il y avait encore une forte couche d'humus, il y a une douzaine d'années. Cet humus s'est épuisé, et on n'y voit plus aujourd'hui que le sable mouvant. Cependant, au printems, ce sable se couvre d'un beau gazon, ce qui ne peut être expliqué que par la quantité de gaz acide carbonique qui se développe dans ces marais. Le terrain mêlé d'une trop grande quantité d'humus est amélioré par une longue culture. Si l'on mêle du sable avec de l'humus spongieux accumulé sans addition de terres élémentaires, ce mélange est plus fertile que l'humus\setcounter{page}{66} seul; il n'absorbe point trop l'humidité, et les racines des plantes y trouvent plus de consistance. On voit ainsi quelquefois un amendement prodigieux, qui est produit par le sable. Celui-ci décompose aussi l'humus acide et la tourbe, ou plutôt il leur ôte leur humidité surabondante, et ensuite ces substances sont décomposées par l'atmosphère. L'humus qui a été long-temps soustrait à l'action de l'air, donne des résultats tout différens de ceux qu'on obtient de l'humus qui a été exposé aux influences de l'atmosphère, et cela, soit que le premier ait été couvert par la terre ou l'eau, ou bien qu'il ait occupé la partie la plus basse des couches épaisses de l'humus. Nous n'avons, au reste, que des probabilités sur les changemens qu'éprouve l'humus, lorsqu'il demeure pendant longtemps soustrait aux influences atmosphériques. On en trouve de cette espèce dans les bas fonds, et auprès des forêts. L'eau y a entraîné les végétaux, et même de l'humus déjà formé. Cet humus est presque toujours mêlé d'une terre semblable à celle des environs. Lorsqu'il est à une certaine profondeur, qui le met hors du contact de l'air, il se modifie par lui-même, et produit des matières d'une espèce\setcounter{page}{67} différente. Il est très-vraisemblable que la formation de l'acide carbonique, et de la matière extractive, ne peut avoir lieu sans le concours de l'air. Probablement, une partie de l'hydrogène et de l'oxygène est transformée en eau : une autre partie de l'hydrogène dissout du carbone, et s'évapore en gaz hydrogène carboné. Le carbone est enlevé à cet humus en plus petite quantité que ses autres éléments, et il lui arrive précisément l'opposé de ce qui a lieu pour l'humus en contact avec l'air libre.
Ainsi donc, plus long-temps l'humus demeure couvert et plus la proportion du carbone doit y augmenter. C'est une sorte de carbonisation lente. Aussi les couches profondes ont-elles une apparence plus charbonneuse : elles sont plus noires et plus compactes que les couches supérieures ; et dans la combustion, elles donnent plus de charbon. Mais si le carbone ne demeure soluble que dans sa combinaison avec l'hydrogène, cet humus doit être d'une décomposition difficile et moins efficace, jusqu'à-ce qu'un long contact avec l'air, ait de nouveau changé sa nature. L'expérience nous apprend, en effet, que, mêlé avec du fumier récent, et qui laisse échapper beaucoup d'ammoniaque, cet\setcounter{page}{68} humus devient plus promptement efficace; et souvent on n'aperçoit ses effets sur le sol, que lorsque celui-ci a été amendé avec du fumier.
La chaux accélère aussi la décomposition de l'humus; et comme souvent on trouve, sous ce terreau, une couche de chaux, produite par des coquillages, ce mélange peut se faire alors avec une grande facilité.
Lorsque l'humus demeure toujours dans l'humidité, sans être entièrement couvert d'eau, il s'y développe un acide qui est sensible à l'odorat, et qui rougit le papier bleu. Ce fait est connu depuis long-temps, et a fait donner le nom de prairies acides à celles qui présentent ce phénomène. L'auteur a examiné ces faits de plus près et recherché la nature de cet acide : il y a reconnu deux acides distincts, savoir : l'acide acétique et l'acide phosphorique : celui-ci adhère fortement à l'humus, et on ne peut l'en séparer, ni par le lavage, ni par la coction. Le liquide dans lequel l'humus est mis en ébullition, prend, à la vérité, une saveur acide; mais la plus grande partie de l'acide demeure attaché à l'humus. Ce que l'on a pu dissoudre consiste en une petite quantité de matière brune, cassante lorsqu'elle est sèche, et qui diffère beaucoup.\setcounter{page}{69} (69) de la matière extractive de l'humus ordinaire: elle n'a point la propriété de se précipiter de l'eau, lorsqu'on expose celle-ci à l'air. En revanche, cet humus acide contient une grande quantité de matière extractive insoluble, qui fait souvent la plus grande partie de son poids. Lorsqu'il est digéré dans une lessive alkaline, la lessive devient d'un brun foncé, et elle est épaissie par la dissolution de plusieurs substances. Si l'on verse un acide dans cette lessive, la matière extractive se précipite en flocons bruns; et (ce qui est remarquable) si l'on y joint un peu plus d'acide que cela n'est nécessaire pour neutraliser l'alkali, elle absorbe de nouveau l'acide acéteux et l'acide phosphorique, et redevient aussi acide qu'auparavant. Mais si l'on ne verse que la quantité d'acide nécessaire pour absorber l'alkali, la matière extractive cesse d'être acide. Cet humus acide contient de l'ammoniaque, lequel étoit combiné avec l'acide, et décèle sa présence par une odeur piquante, lorsqu'on traite la solution par les alkalis. Cet humus acide, loin d'être fertile, est nuisible à la végétation. Si l'acide est fort, et s'il a pénétré la totalité de l'humus, on ne voit plus végéter dans celui-ci que des joncs, des carex, des linaigrettes, etc. Les joncs surtout\setcounter{page}{70} indiquent avec certitude la présence de l'humus acide.
Lorsque le terreau est débarrassé de l'excès d'humidité qui favorisoit la reproduction des acides, on peut le changer en humus fertile: on y trouve alors un trésor de substances nutritives végétales, qu'on peut employer soit sur la place même, soit en le transportant sur les champs. L'alkali, les cendres, la chaux, la marne, débarrassent l'humus de son acide, et le rendent soluble. Si l'on n'a pas ces matières à sa disposition, l'écobuage tire de l'humus lui-même cet alkali et cette chaux qui produisent de si bons effets, et en même temps le feu détruit une grande partie de l'acide; aussi voit-on que cette opération a les meilleurs résultats sur les terrains de ce genre.
Cet humus acide est produit par les végétaux qui contiennent beaucoup de tannin, et en particulier par la bruyère, lors même qu'elle végète sur un terrain sec. Dans les lieux où cette famille de plantes a pris possession du terrain, on trouve souvent une terre noire dont la couleur est principalement due à l'humus. Cet humus est tout-à-fait insoluble: il ne favorise la végétation que de ces mêmes végétaux qui l'ont produit. La bruyère réussit difficilement dans\setcounter{page}{71} les lieux où cet humus n'existe pas; et là où elle est, elle souffre peu d'autres plantes. La marne, la chaux et l'ammoniaque, ou un écobuage complet, produisent de grands effets sur un tel terrain.
La feuille de certains arbres, sur-tout celle du chêne, produit un humus de cette nature, si elle n'est pas décomposée par un fumier très-chaud, ou par la chaux et les alkalis. Cet humus acide devient doux et fertile par une longue exposition à l'air.
Il y a aussi, à ce qu'il paroît, une différence sensible entre l'humus qui résulte d'une putréfaction complète, et celui dont les élémens n'ont été décomposés qu'en partie, parce qu'il leur manquoit quelques circonstances de la putréfaction, de la chaleur et de l'humidité, tandis que l'air y avoit accès. Le plus grand nombre des expériences ont été faites sur la première espèce d'humus, parce qu'on la recueille facilement dans les troncs des arbres pourris. On trouve aussi un humus semblable dans d'anciens marais desséchés, où il compose presque uniquement le sol, dans une épaisseur d'un à deux pieds. Il est remarquable que les céréales n'y réussissent pas, malgré l'abondance des sucs nutritifs. Cela est-il dû à\setcounter{page}{72} quelque propriété particulière de cet humus, ou à ce que le sol n'a pas assez de consistance ?
Il existe enfin des différences essentielles entre l'humus résultant de la putréfaction des végétaux, et celui qui provient en plus grande partie des animaux. Dans ce dernier cas, il contient plus d'azote, de soufre et de phosphore, ce qui est sensible par l'odeur qui se manifeste dans le combustible: cette odeur est semblable à celle qui se répand lorsqu'on brûle des corps animaux. Pour fixer la proportion des diverses parties intégrantes, qui entre dans les diverses humus, nous avons besoin de recherches et d'expériences pneumato-chimiques plus précises, que celles qu'on a faites jusqu'ici.
\section{De la tourbe}
La tourbe est aussi une espèce d'humus. Autrefois on lui attribuoit une origine minérale. On rencontre, à la vérité, des tourbes qui sont imprégnées de bitume : la plupart n'en contiennent pas du tout ; et d'ailleurs on sait aujourd'hui que le bitume est d'origine végétale. La tourbe n'est autre chose qu'une substance produite par l'accumulation\setcounter{page}{73} tion des débris des plantes plus ou moins décomposées. Elle se forme dans les lieux bas et humides, où il croît des mousses et des plantes herbacées, d'une décomposition difficile, et qui ainsi se réunissent au limon que les eaux y amènent. Les végétaux perdent de plus en plus leur tissu organique, et sont enfin réunis en une masse compacte et spongieuse. Si la putréfaction est assez avancée pour que le tissu organique soit détruit, la tourbe n'est plus qu'un humus acide; car celui-ci peut fort bien être employé comme combustible. Les plantes dont la tourbe est entièrement formée sont les carex ( *carices* ) la linaigrette ( *eriophorum* ) le lédon des marais ( *ledum palustre* ) et sur-tout la sphaigne des marais ( *sphagnum palustre* ).
Van Murum, naturaliste Hollandais, regarde la conferve des ruisseaux comme le principal élément de la tourbe: il pense qu'on peut se procurer la tourbe, et l'induire sur un terrain, en naturalisant cette plante dans quelque lieu humide.
La position du terrain, le degré d'humidité habituelle, la nature de la couche inférieure, et des plantes qui croissent sur le sol, influent beaucoup sur la formation
Agric. Vol. 18. N°. 2. Févr. 1813. F\setcounter{page}{74} de la tourbe, et produisent dans celle-ci de grandes différences. Là où tout favorisait une prompte décomposition, l'on trouve la tourbe en masse homogène, pesante et noire. Dans d'autres endroits où la décomposition ne s'opérait que lentement, la tourbe est légère et spongieuse; elle contient encore un grand nombre de filaments des plantes qui n'ont pas été totalement décomposées. La tourbe présente diverses irrégularités apparentes, et des différences qu'on aperçoit que dans une analyse exacte. Près de la surface, la tourbe est molle et filamenteuse: plus bas on pénètre, plus elle est compacte, solide et noire. La tourbe se forme peu-à-peu, et par couches successives. Lorsqu'une génération de plantes a péri, il s'en forme une nouvelle sur ses débris, et la masse augmente ainsi d'année en année. La décomposition est donc plus avancée dans les couches inférieures, qui sont les plus anciennes; et comme, à mesure que la décomposition avance, la masse se carbonise davantage, les couches inférieures doivent être plus noires et plus compactes.
Plus les fibres végétales sont décomposées, plus la tourbe se rapproche de l'humus. Elle doit pourtant son existence à d'autres causes,\setcounter{page}{75} et d'ailleurs l'humus formé par la putréfaction des végétaux, n'est pas exposée à une humidité continuelle, comme la tourbe. La terre du sol avec lequel l'humus est mêlé, agit continuellement sur lui, tandis qu'il n'y a point de terre dans la tourbe, proprement dite. Le plus souvent la tourbe et l'humus acide se ressemblent tellement par leurs propriétés essentielles, qu'on doit les confondre.
La tourbe, ainsi que l'humus acide, contient les acides acéteux et phosphorique, et enfin l'ammoniaque. Elle contient aussi beaucoup de matière extractive insoluble, qui peut être rendue soluble par la potasse et les cendres Quelquesfois on trouve, dans la tourbe, des pyrites, lesquelles y ont été apportées sans qu'on sache comment. Elle répand alors, dans la combustion, une forte odeur de soufre, et on voit aussi à sa surface un sel qui a une odeur d'encre, et qui n'est autre chose que du vitriol (sulfate de fer.)
De même que l'humus est un composé de carbone, d'hydrogène, d'azote et d'oxigène, la tourbe est formée de ces substances; mais la proportion du carbone y est plus forte, surtout lorsque la tourbe est vieille : sa qualité,\setcounter{page}{76} comme combustible, dépend de cette circonstance. Son exposition dans un lieu sec, et son mélange avec de la potasse ou de la chaux, peuvent lui faire subir une décomposition qui la délivre de tout acide, et la change en un humus fertile. Nous reviendrons sur ce sujet, en traitant de l’amendement des terres.
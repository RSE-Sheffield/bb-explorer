\setcounter{page}{371}
\section{ELEMENTS OF AGRICULTURAL CHEMISTRY, etc. Élémens de chimie-agricole en un cours de leçons pour le Département d'Agriculture ; par Sir HUMPHRY DAVY. Londres, 1813\footnote{La traduction de cet ouvrage ne tardera pas à paraître chez Paschoud, Libraire à Genève et à Paris.}. \large{(Second extrait. Voy. p. 329 de ce vol.)}}
LA seconde leçon traite des diverses propriétés de la matière qui influent sur la végétation, de la gravitation, de la cohésion, de l'attraction chimique, de la chaleur, de la lumière, de l'électricité, des élémens de la matière et de leurs combinaisons. L'auteur cite des expériences de Mr. Knight, desquelles il paroît résulter que la gravitation influe essentiellement sur la direction verticale des branches et des racines des végétaux. Il considère l'attraction de cohésion comme une force dont l'application influe sur les phénomènes de la végétation, et cause, en particulier, l'ascension des fluides dans les tubes capillaires, et l'absorption.\setcounter{page}{372} des sucs nourriciers, par les racines. Il regarde l'affinité chimique comme la cause probable de l'assimilation des principes de la sève aux organes des plantes, et comme travaillant sans cesse à modifier la forme, la consistance et les apparences des corps qui servent à la végétation. Il indique les propriétés générales de la chaleur, qui tend à dilater les substances, et il mentionne les exceptions. Il fait remarquer l'influence de la lumière sur la couleur des végétaux et la perfection de leurs sucs. Il touche aux propriétés générales de l'électricité, et aux diverses opinions des physiciens sur cet agent de la nature.
Le chevalier Davy présente ensuite l'énumération des corps simples ou élémentaires, c'est-à-dire ceux qui, dans l'état présent des connaissances chimiques, ne peuvent être décomposés. Ces corps, au nombre de quarante-sept, sont, deux gaz, qui peuvent entretenir la combustion, sept corps inflammables, et trente-huit métaux. Il indique les proportions dans lesquelles ces divers corps se combinent. Treize des trente-huit métaux sont encore imparfaitement connus. Rappelons ici que les alkalis et les terres ne sont plus au nombre des corps simples. Les expériences de cet illustre chimiste, et de ses émules sur le continent, ont prouvé\setcounter{page}{373} que chacune des terres connues, et qu'on croyoit élémentaires, est composée d'une substance métallique unie à l'oxigène.
Cette leçon étant plus propre à intéresser les chimistes que les agriculteurs, nous nous bornons à en indiquer ainsi la substance et nous passons à la troisième.
De l'organisation des plantes; des racines; du tronc et des branches; de leur structure; de l'épiderme; des feuilles; des fleurs; des semences; de la construction chimique des organes des plantes, et des substances qu'on y trouve; des substances mucilagineuses sucrées, extractives, résineuses, et huileuses, et des autres composans des plantes; de leur arrangement dans les végétaux; de leur composition, de leurs changemens, et de leurs usages.
La variété est un des caractères du règne végétal; mais il existe une analogie entre les formes et les fonctions des diverses classes de plantes, et c'est sur cette analogie que sont fondés les principes de la science. Les végétaux sont des constructions vivantes, distinguées des animaux, en ce qu'ils ne montrent aucun signe de perception ni de mouvemens volontaires, et que leurs or-\setcounter{page}{374} ganes n'ont pour objet que la nourriture de l'individu et la reproduction de l'espèce. Il y a à considérer, dans le règne végétal, la forme extérieure et la constitution intérieure.
On trouve dans toutes les plantes, en les examinant sous le rapport de leur construction extérieure, au moins quatre systèmes d'organes, ou parties analogues. Premièrement, les racines; secondement, le tronc et les branches; troisièmement, les feuilles; quatrièmement, les fleurs et les fruits.
La racine est essentielle à la plante; elle l'attache au sol; c'est l'organe de la nourriture, ou l'appareil par lequel la plante tire de la terre la substance qui l'alimente. Dans leurs divisions anatomiques, les racines sont très-semblables au tronc et aux branches. La racine est une continuation du tronc, laquelle se termine par des filamens déliés. Si l'on enterre les branches de certains arbres, et qu'on mette les racines en l'air, les racines produisent des bourgeons et des feuilles, et les branches poussent des fibres radicale et creuses. Woodward a fait cette expérience sur le saule, et elle a été répétée par plusieurs physiologistes.
Lorsqu'on coupe transversalement une branche ou une racine, on distingue trois\setcounter{page}{375} corps différens, l'écorce, le bois, et la moëlle; et chacun des trois est susceptible d'une division nouvelle.
L'écorce, lorsqu'elle est parfaitement formée, est couverte d'une fine peau, ou épiderme, qu'on peut aisément enlever. L'écorce est ordinairement composée d'un grand nombre de lames ou écailles, qui, dans les vieux arbres se détachent et se pourrissent. L'épiderme n'a point de vaisseaux: il ne sert qu'à défendre l'écorce contre les accidens extérieurs. Dans les arbres de haute futaie, et dans les plus forts arbrisseaux, l'épiderme a peu d'importance; mais dans les roseaux, les herbes, les joncs, et les plantes dont la tige est creuse, l'épiderme a beaucoup de force; et paroît au microscope comme un filet glacé, lequel est principalement composé de terre siliceuse. C'est le cas pour le froment, l'avoine, et différentes espèces d'équi-setum (prêle). L'épiderme du rattan contient une assez grande quantité de silex, pour donner des étincelle avec l'acier. Deux morceaux d'écorce frottés l'un contre l'autre en produisent également. J'eus connoissance de ce fait pour la première fois en 1798; il me conduisit à des expériences par lesquelles je m'assurai que la terre siliceuse existoit généralement dans l'épiderme des plantes à tige.\setcounter{page}{376} creuse. Cette enveloppe dure, donne de la consistance à l'écorce, et la protège contre les insectes : elle paroît jouer le même rôle pour ces végétaux foibles, que la coquille pour les insectes crustacées.
Immédiatement au-dessous de l'épiderme, on trouve le parenchyme, substance molle, composée de cellules remplies d'un fluide presque toujours verdâtre. Les cellules du parenchyme examinées au microscope, paroissent hexagones; Cette forme est en général celle des membranes cellulaires des végétaux, et paroît le résultat de la réaction générale des parties solides, comme dans la forme des alvéoles des ruches. Cet arrangement, qui avoit été attribué à l'adresse et à l'intelligence des abeilles, paroît le résultat des lois mécaniques dans la pression des cylindres composés d'une matière molle: le nid d'une abeille solitaire est toujours cylindrique.
La partie interne de l'écorce est composée de ce qu'on nomme les couches corticales, dont le nombre varie selon l'âge de l'arbre. Si l'on coupe l'écorce d'un arbre qui a plusieurs années, on distingue la production des diverses époques, quoiqu'il soit rarement facile de compter exactement le nombre des années par le nombre des couches.\setcounter{page}{377} Les couches corticales sont composées de parties fibreuses, qui paroissent entrelacées, et qui sont transversales, et longitudinales : les transversales sont membraneuses et poreuses, et les longitudinales sont en général composées de tubes.
Les fonctions des parties parenchymateuses et corticales sont d'une grande importance. Les tubes et les parties fibreuses paroissent être les organes qui reçoivent la sève. Les cellules semblent destinées à l'élaboration de cette substance, et à la maintenir exposée à l'action de l'atmosphère. Chaque printemps il se produit de la matière nouvelle immédiatement sur la surface intérieure de la couche corticale de l'année précédente.
Les expériences de Mr. Knight, et d'autres physiologistes, ont montré que la sève qui descend par l'écorce, après avoir été modifiée dans les feuilles, est la principale cause de la croissance de l'arbre. Si un arbre est blessé dans son écorce, la régénération de cette écorce s'opère principalement par la partie supérieure, et quand on a enlevé un morceau du bois, la formation du nouveau bois a lieu immédiatement au-dessous de l'écorce. Cependant il sembleroit, par les observations de Mr. Palisot de Beau-\setcounter{page}{378} vois, que la sève peut être transportée à l'écorce, de manière à exercer ses fonctions nutritives indépendamment d'un système général de circulation. Mr. Palisot a isolé, plusieurs portions d'écorce du reste de l'écorce, dans divers arbres, et il a trouvé que, dans le plus grand nombre des cas, l'écorce isolée croissoit comme l'autre. L'expérience a réussi particulièrement sur le tilleul, l'érable et le lilas. Dans les endroits où il avoit isolé des portions d'écorce, il a poussé des bourgeons.
Le bois des arbres est composé d'une partie externe et vivante, qu'on nomme aubier ou bois de sève, et d'une portion interne et morte, qu'on nomme le cœur. L'aubier est blanc, et rempli d'humidité. Dans les jeunes arbres, et dans les pousses de l'année, l'aubier va jusqu'à la moelle. Il contient le système vasculaire, par lequel la sève s'élève; et les vaisseaux dont il est rempli communiquent des racines les plus menues jusqu'aux feuilles.
Il y a dans l'aubier une substance membraneuse composée de cellules qui sont constamment remplies de sève. Mirbel a distingué dans l'aubier, quatre diverses formes de tubes, dont les fonctions paroissent différentes, savoir, les tubes simples, les tubes\setcounter{page}{379} poreux, les trachées, et les fausses trachées.
Les tubes simples paroissent contenir les résines et les huiles particulières à chaque plante.
Les tubes poreux contiennent également les fluides résineux et huileux: ils paroissent avoir pour fonction de les transporter dans la sève pour produire des dispositions nouvelles.
Les trachées contiennent une matière fluide, séreuse, transparente. Elles sont probablement destinées à séparer l'eau des sucs plus denses, lesquels se consolident ainsi, pour produire le nouveau bois.
Dans l'arrangement des fibres du bois, il y a deux apparences distinctes. On voit une suite de lames blanches et brillantes qui se dirigent du centre à la circonférence, et qui constituent le grain argenté du bois; (silver grain) et l'on observe des couches concentriques qui indiquent l'âge de l'arbre, et qu'on nomme le faux grain. (Spurious grain).
Le grain argenté est élastique, et susceptible de contraction; et Mr. Knight a supposé que le changement de volume qui résulte dans ces lames, des variations de température, étoit une des principales causes de l'ascension de la sève. Elle paroissent se\setcounter{page}{380} dilater le matin, et se contracter le soir; et l'ascension des liquides dans les tubes capillaires, dépend principalement de la chaleur, ainsi que nous l'avons vu dans la dernière leçon.
Le grain argenté est très sensible dans les arbres de haute futaye, mais les arbustes et les plantes annuelles ont aussi un système de fibres analogue. La nature est toujours en harmonie avec elle-même, et des effets semblables sont ordinairement produits par des organes du même genre.
La moelle occupe le centre de la tige. Sa texture est membraneuse. Elle est composée de cellules circulaires vers l'extrémité et hexagones vers le centre de cette substance.
Dans la première enfance de la plante, la moelle n'occupe qu'un petit espace. Peu-à-peu elle tient plus de place; son diamètre est considérable dans les jeunes arbres et les pousses d'un an. A mesure que l'arbre vieillit, la moelle pressée par les couches nouvelles de l'aubier, se resserre et elle disparoît enfin tout-à-fait.
Diverses opinions ont été adoptées relativement à l'usage de la moelle des plantes. Le Docteur Hales pensoit que la moelle étoit la principale cause du développement des autres parties de la plante; que sa position\setcounter{page}{381} centrale réunissoit sur cette substance l'action de tous les autres organes, et que les phénomènes de leur croissance résultoient de la réaction de la moelle sur ces mêmes organes.
Linné, dont l'imagination active, étoit sans cesse occupée à rechercher des analogies entre les systèmes de la végétation et de la vie animale, croyoit que la moelle remplissoit pour la plante les mêmes fonctions que le cerveau et les nerfs pour les animaux. Il la considéroit comme l'organe de l'irritabilité, et comme le siége de la vie.
Les dernières découvertes ont montré que ces deux opinions sont également erronées. Mr. Knight a enlevé la moelle de plusieurs jeunes arbres, et ils ont continué à vivre et à croître : la moelle n'est donc évidemment qu'un organe d'une importance secondaire. Elle est pénétrée d'humidité dans les jeunes pousses d'une végétation forte, et elle est peut-être, un réservoir de nourriture liquide, pour l'époque où celle-ci est le plus nécessaire. A mesure que le cœur du bois se forme, la moelle se trouve de plus en plus séparée de la partie vivante, qui est l'aubier ; ses fonctions cessent ; elle diminue ; elle meurt ; et enfin elle disparaît.
Les épines, les vrilles, et les autres parties analogues des plantes, offrent une or-\setcounter{page}{382} ganisation semblable à celle des branches, relativement à l'écorce et à l'aubier. Les dernières observations de Mr. Knight ont montré que la direction des vrilles, et la forme spirale qu'elles affectent, dépendent de l'action inégalement repartie de la lumière sur elles; et Mr. De Candolle assigne une cause semblable à la direction que certaines plantes prennent du côté du soleil: cet ingénieux physiologiste suppose que les fibres sont raccourcies par l'action chimique des rayons solaires, et qu'ainsi la plante se tourne vers le soleil.
Les feuilles, quoiqu'infiniment variées dans leurs formes, sont toujours semblables entr'elles dans leur organisation intérieure, et sont chargées des mêmes fonctions.
L'aubier s'étend depuis le bas de la tige, jusqu'aux extrémités des feuilles. Il conserve dans celles-ci son système vasculaire tout entier et sa faculté de vie. On distingue encore dans la feuille les tubes observés dans l'aubier et particulièrement les trachées.
La substance verte des feuilles peut être considérée comme une extension du parenchyme, lequel est également recouvert de l'épiderme. Ainsi l'organisation du tronc et des branches peut être suivie jusques dans les feuilles, dont toutefois la structure est plus délicate et plus parfaite.\setcounter{page}{383} Un des principaux usages de la feuille est d’exposer la sève à l’influence de l’air, de la chaleur, et de la lumière. Elles ont beaucoup de surface, leurs vaisseaux et leurs cellules sont extrêmement multipliés et déliés, et leur contexture est poreuse et transparente. Il s’évapore des feuilles, une grande partie de l’eau de la sève; cette dernière se combine avec de nouveaux principes, appropriés à l’entretien des organes; et ainsi préparée, elle passe, probablement, de l’extrémité des ramifications des tubes de l’aubier, dans l’extrémité des ramifications des tubes appartenans aux couches corticales, pour redescendre par l’écorce.
A la surface supérieure des feuilles, qui est exposée au soleil, l’épiderme est épais, mais transparent: il est composé d’une matière peu organisée, laquelle est principalement terreuse, ou d’une substance chimique homogène. Dans les graminées, cet épiderme est en partie, siliceux; dans le l’érable et l’aubépine, il est principalement composé d’une substance analogue à la cire. Cette disposition prévient l’évaporation, et ne la permet que par les tubes qui y sont destinés.
L’épiderme de la face inférieure des feuilles est, au contraire, une membrane exces-\setcounter{page}{384} sivement déliée, transparente et poreuse, par laquelle, probablement, l'eau, et tous les principes nécessaires à la végétation, sont absorbés de l'atmosphère.
Si l'on retourne la feuille de manière à présenter sa surface inférieure au soleil, ses fibres se tordent pour ramener la première position. Toutes les feuilles s'élèvent du côté du soleil pendant qu'il luit. Cet effet paraît dépendre en grande partie, de l'action mécanique et chimique de la lumière et de la chaleur. Bonnet fit des feuilles artificielles lesquelles se tournoient exactement de la même manière que les feuilles naturelles, lorsqu'on tenoit sous la surface inférieure une éponge mouillée, et au-dessus de la feuille, un fer chaud.
Ce que Linné a appelé le sommeil des feuilles, paroît résulter du défaut d'action de la lumière, de la chaleur et de l'excès de l'influence de l'humidité. Ce phénomène singulier et constant n'avoit jamais été observé avant d'avoir attiré l'attention du botaniste d'Upsal. Il examinoit un soir un lotus auquel il avoit remarqué quatre fleurs pendant le jour, et qui n'en avoit plus que deux. En y regardant de plus près, il vit que les deux autres étoient cachées par les feuilles, qui s'étoient repliées pour les envelopper. Cette\setcounter{page}{385} circonstance ne pouvoit être perdue pour un tel observateur: il prit une lanterne, et courut à son jardin; il y observa des phénomènes jusqu'alors inconnus. Toutes les feuilles des plantes qu'il examina étoient disposées d'une manière différente de ce qu'elles sont pendant le jour, et le plus grand nombre d'entr'elles étoient pliées ou fermées.
Il y a des cas où l'on peut produire artificiellement le sommeil des plantes. De Candolle a fait cette expérience sur la sensitive. Il l'enferma pendant le jour dans un endroit obscur, et les feuilles se refermèrent bientôt; mais lorsqu'il éclaira la chambre par plusieurs lampes, les feuilles s'étendirent aussitôt, tant elles sont sensibles aux effets de la lumière et de la chaleur rayonnante.
Dans la plupart des plantes, les feuilles périssent annuellement, et se reproduisent. Elles tombent ordinairement à la fin de l'été dans les climats très-chauds, lorsque la sécheresse du sol, et la grande évaporation empêchent qu'elles ne puissent être fournies de sève, ou bien dans le courant de l'automne, et aux premières gelées, comme cela arrive dans les climats du nord. Les feuilles ne conservent leurs fonctions qu'aussi long-temps qu'elles ont en elles une circu\setcounter{page}{386} lation. Dans le déclin de la feuille, sa couleur paroît dépendre de la nature des changemens chimiques qu'elle subit ; et comme ordinairement il s'y développe des acides, cette couleur est le plus souvent jaune ou rouge brun : cependant il y a de grandes variétés à cet égard : dans le chêne, la feuille sèche est d'un brun brillant, dans le bouleau, elle est orangée ; dans l'orme, elle est jaune ; dans la vigne, rouge ; dans le sycomore, d'un brun sombre ; dans le cornouiller, pourpre ; dans le chèvre-feuille, bleue.
La cause de la conservation des feuilles pendant l'hiver pour les arbres résineux, n'est pas exactement connue. Il paroît par les expériences de Hales, que là force de la sève est beaucoup moindre dans les plantes de cette espèce ; et elles conservent vraisemblablement quelque circulation pendant l'hiver. Leurs sucs sont moins aqueux que ceux des autres plantes, et probablement moins sujets à subir la congélation, enfin ils sont défendus contre l'action des élémens par des enveloppes plus fortes.
La production des autres parties de la plante a lieu dans le moment où elles font leurs fonctions avec le plus de vigueur. Si l'on ôte les feuilles d'un arbre au printems, cet\setcounter{page}{387} cet arbre meurt infailliblement, et lorsque des orages enlèvent aux arbres des forêts une grande partie de leurs feuilles, ceux-ci souffrent ensuite et languissent....
Les feuilles sont nécessaires à l'existence de l'individu; les fleurs le sont au maintien de l'espèce. Elles sont dans chaque plante la partie la plus délicate, et la plus belle dans sa structure: elles sont comme le chef-d'œuvre de la nature dans le règne végétal. La délicatesse et l'éclat de leurs couleurs, la variété de leurs formes, la finesse de leur organisation, le but final de chacune de leurs parties, tout dans les fleurs semble fait pour éveiller notre curiosité, pour exciter notre admiration.
Il faut observer dans chaque fleur, 1°. Le calice, ou partie membraneuse verte qui forme le support des feuilles de la fleur, ou pétales. Le calice est vasculaire, et a la même texture et organisation que les feuilles. Il défend, il soutient, il nourrit les parties plus parfaites. 2°. La corolle qui est composée ou d'une seule pièce ou de plusieurs: dans le premier cas, on l'appelle monopétale, dans le second polypétale. La corolle est ordinairement de couleur vive, et remplie d'une variété presque infinie de petits vaisseaux poreux; elle renferme et défend
Agricult. Vol. 18. N°. 10. Octob. 1813. Gg\setcounter{page}{388} des parties essentielles, et leur transmet les sucs de la sève. Les parties essentielles sont:
\section{3°. Les étamines et les pistils.}
La partie essentielle des étamines est les anthères ou sommet, qui sont ordinairement de forme arrondie, d'une texture très-vasculaire, et que la poussière fécondante recouvre.
Le pistil est cylindrique, et surmonté du style, dont le sommet est ordinairement rond et protubérant.
Si l'on examine le pistil au microscope, on découvre un amas de petits corps sphériques, qui paroissent être la base des semences futures.
Toute la classification de Linné est fondée, sur l'arrangement des étamines et des pistils. Leur nombre dans la même fleur, leur disposition ou leur division en diverses fleurs, sont les circonstances qui guidèrent le naturaliste Suédois, et qui servirent de fondement à un système admirablement calculé pour soulager la mémoire, et faciliter l'étude de la botanique; ce système, sans associer toujours les plantes qui ont le plus d'analogie entre'elles dans leurs caractères généraux, est si ingénieusement inventé, qu'il indique tous les rapports de leurs parties les plus essentielles.\setcounter{page}{389} Le pistil est l'organe qui contient les rudimens de la graine; mais cette graine ne se forme jamais avec son germe reproductif, sans l'influence de la poussière fécondante des anthères. Cette influence mystérieuse, nécessaire à la succession des diverses espèces de végétaux, est un des traits de cette grande et belle analogie établie dans la nature entre les divers ordres des êtres de la création.
Les anciens avoient observé que différens palmiers portoient différentes fleurs, et que ceux dont les fleurs avoient des pistils ne donnoient point de fruits, à moins qu'ils ne se trouvassent dans le voisinage immédiat des palmiers dont les fleurs avoient des étamines. Ce fait établi depuis long-temps frappa Malpighi; et il constata divers faits analogues dans d'autres végétaux. Grew fut le premier qui généralisa ces faits; et l'on trouve dans ses ouvrages une argumentation très-juste sur ce sujet. Linné donna une forme scientifique et distincte à ce qui avoit fait l'objet des observations générales de Grew: il eut la gloire d'établir ce qu'on a appelé le système sexuel, sur des observations et des expériences exactes.
La graine ou la semence, le dernier produit d'une végétation vigoureuse, est pro-\setcounter{page}{390} digieusement diversifié dans sa forme. Comme elle est d'une importance infinie dans les vues de la nature, elle est protégée plus qu'aucune autre partie de la plante, soit par une substance charnue, comme dans les fruits à pulpe, soit par de fortes membranes, comme dans les végétaux légumineux, soit par des coquilles dures, ou par un épais épiderme, comme dans les noix et les semences des graminées.
Dans toutes les semences, il faut distinguer, 1°. l'organe de la nourriture, 2°. la plantule, ou plante naissante, 3°. la radicule.
Dans la fève de jardin, l'organe de la nourriture est divisé en deux lobes appelés cotylédons. La plantule est le point blanc situé entre les deux parties supérieures des lobes, et la radicule est le petit cône recourbé, placé à leur base.
Dans le blé et dans plusieurs graminées, l'organe de la nourriture ne forme qu'une seule et même partie. Quelques graines ont plusieurs cotylédons, mais dans le plus grand nombre il y en a deux seulement. Lorsqu'on examine une graine, on n'y découvre ni les formes, ni les fonctions de la vie; elle semble être une matière inerte. Cependant si l'eau, la cha-\setcounter{page}{391} leur et l'air agissent sur cette graine, elle ne tarde pas à montrer les facultés d'organisation qu'elle recèle. Les cotylédons se dilatent, les membranes se rompent, la radicule descend dans le sol, et la plantule s'élève vers l'atmosphère. Peu-à-peu les organes de la nourriture, dans les plantes à deux cotylédons, deviennent vasculaires, et se montrent au-dessus du sol : ils sont convertis en feuilles qu'on nomme séminales. La nature a pourvu aux élémens de la germination partout. L'eau, l'air pur et la chaleur sont universellement en activité; et les moyens qui pourvoient à la conservation et à la multiplication de la vie, sont aussi simples qu'ils sont grands.
Il seroit incompatible avec l'objet de ces leçons d'entrer dans de trop grands détails sur la physiologie végétale : j'ai seulement essayé de donner sur ce sujet quelques idées générales pour aider l'agriculteur instruit à comprendre les fonctions des végétaux: ceux qui voudront en étudier l'anatomie comme une science distincte, trouveront d'abondans matériaux dans les ouvrages des auteurs que j'ai cités, ainsi que dans ceux de Linné, Desfontaines, De Candolle, De Saussure, Bonnet, et Smith.\setcounter{page}{392} L'histoire des particularités de structure dans les différentes classes de végétaux, appartient plutôt à la botanique qu'à l'agriculture. Ainsi que je l'ai dit d'abord, les organes des diverses plantes ont entre eux l'analogie la plus distincte, et sont gouvernés par les mêmes lois. Dans les graminées et dans le palmier, les couches corticales ont plus d'épaisseur relative que dans les autres plantes; mais leurs usages paraissent les mêmes que dans les arbres des forêts.
Dans les plantes à racines bulbeuses, la substance analogue à celle de l'aubier, forme la plus grande partie du végétal; mais elle paraît toujours contenir la sève ou les matériaux solides déposés par la sève.
Les feuilles légères et comparativement séches, du pin et du cèdre, remplissent les mêmes fonctions que les grandes feuilles pleines de suc du figuier ou du noyer.
Dans les plantes cryptogames, elles-mêmes, où l'on ne distingue aucune fleur, il est bien probable que la production de la graine s'effectue de la même manière que dans les plantes plus parfaites. Les mousses et les lichens, qui appartiennent à cette famille, et qui n'ont ni feuilles ni racines distinctes, sont pourvues de filamens qui font les mêmes fonctions; et enfin dans les cham\setcounter{page}{393} pignons, il existe un système pour l'absorption de la sève, et pour qu'elle jouisse de l'influence de l'air.
Nous avons vu, dans la dernière leçon, que les diverses plantes peuvent être décomposées en un petit nombre d'élémens. Leurs usages pour la nourriture de l'homme, ou dans les arts, dépendent de l'arrangement composé de ces mêmes élémens ; que nous pouvons retirer des parties organisées, ou des sucs qu'elles contiennent. L'examen de la nature de ces substances appartient essentiellement à la chimie agricole.
Les huiles s'obtiennent par l'expression de divers fruits ou semences ; les fluides résineux découlent de divers arbres ; la sève contient des matières sucrées ; les substances colorantes sont fournies par les feuilles et par les pétales. Divers procédés sont nécessaires pour séparer les unes des autres les substances dont la plante est composée : on y emploie la macération, l'infusion ou digestion dans l'eau ou dans l'esprit-de-vin ; mais pour entendre l'application de ces procédés, il faut étudier d'abord la nature chimique de ces substances.
(La suite au Cahier prochain.)
\section{Correction importante.}
Page 331, ligne 12, au lieu d'amender, lisez écobuer.\setcounter{page}{394} ELEMENTS OF AGRICULTURAL CHEMISTRY, etc. Élémens de chimie-agricole en un cours de leçons pour le Département d'Agriculture ; par Sir HUMPHRY DAVY. Londres, 1813\footnote{La traduction de cet ouvrage ne tardera pas à paraître chez Paschoud, Libraire à Genève et à Paris.}. (Second extrait. Voy. p. 329 de ce vol.)
LA seconde leçon traite des diverses propriétés de la matière qui influent sur la végétation, de la gravitation, de la cohésion, de l'attraction chimique, de la chaleur, de la lumière, de l'électricité, des élémens de la matière et de leurs combinaisons. L'auteur cite des expériences de Mr. Knight, desquelles il paroît résulter que la gravitation influe essentiellement sur la direction verticale des branches et des racines des végétaux. Il considère l'attraction de cohésion comme une force dont l'application influe sur les phénomènes de la végétation, et cause, en particulier, l'ascension des fluides dans les tubes capillaires, et l'absorption.
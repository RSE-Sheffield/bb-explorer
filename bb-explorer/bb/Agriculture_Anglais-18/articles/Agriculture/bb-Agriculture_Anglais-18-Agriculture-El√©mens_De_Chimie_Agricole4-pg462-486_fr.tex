\setcounter{page}{462} tres arbres. En choisissant les espèces, je crois que c'est le meilleur parti à tirer des tourbes molles. J'observe, cependant, que tout ce que j'ai dit ne s'applique point aux tourbes qui ont une consistance suffisante, et qui se couvrent naturellement d'herbe. Je crois qu'on peut les cultiver avec profit. Je n'ai rien dit qui ne fût fondé sur mon expérience et sur l'observation des faits.
\section{ELEMENTS OF AGRICULTURAL CHEMISTRY, etc.}
Élémens de chimie-agricole en un cours de leçons pour le Département d'Agriculture ; par Sir HUMPHRY DAVY. Londres, 1813. (Quatrième extrait. Voy. p. 411.)
L'EAU, et la substance animale et végétale en décomposition dans le sol, constituent la véritable nourriture des plantes. Les parties terreuses du sol sont utiles pour retenir l'eau, pour en fournir une juste proportion aux racines des végétaux ; et ces mêmes parties terreuses distribuent d'une manière égale les substances animales et végétales en décomposition, en modérant, par leur mé-\setcounter{page}{463} lange, la trop grande promptitude de cette décomposition. Les parties solubles se trouvent ainsi fournies dans de justes proportions. Outre cette manière d'agir des terres, laquelle on peut appeler mécanique, il y en a une que l'on peut qualifier de chimique. Les terres, et même les carbonates terreux, ont une certaine affinité avec les principes des substances végétales et animales. On le voit, par exemple, par l'alumine et les huiles. Si une solution acide d'alumine est mélangée à une solution de savon, (qui est composé d'huile et de potasse) l'huile et l'alumine se combinent, et forment une poudre blanche, qui se précipite. L'extrait de la matière végétale en décomposition, si on soumet cette matière à l'ébullition, avec de la craie ou de l'argile, cet extrait, dis-je, forme une combinaison, qui rend la substance végétale plus difficile à dissoudre. La silice, ou les sables siliceux, ont peu d'action de cette espèce; et les sols qui contiennent le plus d'alumine et de carbonate de chaux, sont ceux qui, par leurs qualités chimiques, conservent le plus long-temps les fumiers. On les appelle avec raison des terrains riches; car la nourriture des végétaux s'y conserve long-temps.\setcounter{page}{464} temps, à moins que les organes des plantes ne la leur enlèvent. Les sables siliceux, au contraire, sont dévorans; car les matières animales et végétales qu'ils contiennent, n'étant pas attirées par les parties constituantes du sol, sont, plus exposées à être décomposées par l'action de l'atmosphère, ou enlevées par les eaux.
Dans la plupart des terrains bruns, noirs, et riches, les terres paroissent être en combinaison, avec une matière extractive particulière, qui résulte de la décomposition des végétaux. L'eau attire lentement cette matière, laquelle devient ainsi une cause principale de la fertilité du sol.
La mesure de la fertilité des terrains doit nécessairement varier selon les climats et les productions: la quantité des pluies doit surtout y influer.
La capacité du sol d'absorber l'humidité doit être beaucoup plus grande dans les climats chauds et secs que dans ceux qui sont froids et humides. Les pentes doivent absorber l'eau plus promptement que les plaines ou les vallées; et la fertilité des terrains doit dépendre beaucoup de la nature de la couche sur laquelle repose le sol labourable.
Si le terrain labourable repose immédiatement sur un banc de rocher ou des pierres\setcounter{page}{465} rès, l'évaporation le sèche beaucoup plus promptement, que lorsqu'il repose sur la glaise ou la marne ; et la principale cause de la grande fertilité des terres en Irlande, où le climat est humide, c'est que le sol repose sur des bancs de roc.
Une couche inférieure de glaise est quelquefois fort utile à un terrain sablonneux, parce qu'elle retient l'eau suffisamment pour la fournir à la couche supérieure à mesure que celle-ci la perd par l'évaporation ou qu'elle est consommée par les plantes.
Une couche supérieure graveleuse ou sablonneuse corrige souvent l'imperfection d'un sol qui retient trop fortement l'humidité.
Dans les pays calcaires, où la surface du sol est une espèce de marne, on trouve souvent la pierre à chaux, à quelques pouces de profondeur; et ce voisinage du roc, qui occasionneroit une complète stérilité, si le sol supérieur n'absorboit pas l'humidité de l'air, ne nuit point à sa fécondité. Dans le Derbyshire, et dans la principauté de Galles, on distingue de loin, par la couleur de l'herbe dans la saison chaude, les parties sablonneuses, des parties calcaires. Les premières sont brûlées, quand celles-ci sont vertes.
Lorsqu'il s'agit de destiner les différentes parties d'une propriété territoriale à diverses\setcounter{page}{466} ses récoltes, il est évident, d'après ce que nous venons de dire, que pour le faire avec connoissance de cause, il faut être informé de toutes les circonstances de qualité, de composition, et de situation du sol supérieur, ainsi que la couche sur laquelle il repose.
Les méthodes de culture doivent aussi être différentes selon les terrains. Une pratique très-utile dans un cas, pourroit être nuisible dans un autre. Les labours profonds, très-avantageux dans les terrains qui ont de la profondeur, peuvent nuire dans les sols dont la couche végétale est peu épaisse, et qui reposent sur du sable, ou sur une glaise froide.
Dans les climats humides, où la quantité de pluie qui tombe annuellement varie de quarante à soixante pouces, comme dans le Lancashire, Cornouailles, et quelques parties de l'Islande, les sables siliceux sont beaucoup plus productifs que dans les pays sec. Le blé et les fèves n'y demandent point un terrain aussi cohérent que dans les situations sèches; et là où il pleut souvent, les plantes à racines bulbeuses, prospèrent dans un sol où, sur quinze parties, il y en a quatorze de sable.
La faculté épuisante des récoltes, est plus\setcounter{page}{467} ou moins grande selon qu'il pleut plus ou
moins souvent : là où les plantes ne peuvent
pas absorber assez d'eau, elles absorbent plus
d'engrais. En Islande, en Cornouailles, et
dans les montagnes occidentales de l'Ecosse,
les grains épuisent moins qu'ils ne font dans
les situations intérieures et sèches de l'Angleterre. L'avoine en particulier, épuise beaucoup plus dans les climats secs que dans les
climats humides.
Les terrains paroissent devoir leur origine
à la décomposition des rochers. Il arrive
souvent que le sol existe en nature sur le
roc duquel il a été formé. Il est aisé de se
faire une idée de la manière dont les terrains se forment, en examinant le granite
tendre ou granite de porcelaine : il est composé de quartz, de feldspath, et de mica.
Le quartz est presque de la silice pure ;
sous forme cristallisée. Le feldspath et le mica
sont des substances très-composées : l'un et
l'autre contiennent de la silice, de l'alumine,
et de l'oxide de fer. Dans le feldspath, il
y a ordinairement de la chaux et de la potasse : dans le mica, de la chaux et de la
magnésie.
- Lorsqu'un rocher granitique ainsi composé,
a été long-temps exposé aux influences de
l'air et de l'eau, celle-ci, et l'acide carbo-\setcounter{page}{468} nique de l'air agissent sur la chaux et la potasse qui se trouvent dans les parties constituantes du granite ; et l'oxide de fer, qui est ordinairement peu oxidé, tend à l'oxider davantage. Le feldspath et le mica se décomposent donc; mais le premier plus rapidement. Il étoit comme le ciment du granite : il forme une substance d'un grain fin et doux. Le mica, partiellement décomposé, s'y mêle sous forme de sable ; et le quartz reste sous forme pierreuse ou sablonneuse, en divers degrés de finesse. Dès qu'il s'est formé ainsi à la surface du roc granitique, une première couche qui peut recevoir des végétaux, les lichens, les mousses, et les autres plantes imparfaites dont les semences sont continuellement transportées par les vents, s'y établissent, et végètent. Leur décomposition fournit une certaine quantité de matière organique qui se mêle aux premiers matériaux terreux. Dans ce sol déjà un peu amélioré, des plantes plus parfaites peuvent végéter. Elles absorbent de l'eau de l'atmosphère une partie de leur nourriture, et fournissent, par leur mort, de nouvelles substances aux plantes qui leur succèdent. Cependant la décomposition du granite continue; et par cette suite de procédés naturels, le sol augmente et s'améliore\setcounter{page}{469} au point de pouvoir nourrir les plus grands arbres des forêts, ou payer les travaux des cultivateurs.
Lorsqu'une longue suite de générations de plantes vit et meurt sur le même lieu, sans que l'homme en transporte ailleurs les produits, ou que les animaux les consomment, la matière végétale s'accroit tellement qu'il en résulte une espèce de tourbe; et si, la situation est de nature à ce que l'eau y séjourne, les bonnes plantes ne peuvent plus y végéter.
Il y a beaucoup de tourbières qui paroissent s'être formés par la destruction des forêts. Si l'on abat les arbres des bords d'un bois, ceux qui étoient accoutumés à être à l'abri des vents, deviennent bientôt languissans, et meurent. Leurs branches et leurs feuilles se décomposent peu-à-peu, et forment une couche de matière végétale. Dans la plupart des grandes tourbières de l'Islande et de l'Écosse, les plus gros arbres que l'on trouve près des bords, portent les marques de la hache. Dans les parties intérieures de ces tourbières, on voit rarement des arbres entiers : c'est probablement parce qu'ils ont décliné peu-à-peu, et parce que la fermentation et la décomposition de la matière végétale, ont été plus rapides, vu son abondance.\setcounter{page}{470} Les lacs et les étangs se remplissent quelquefois par l’accumulation des débris de plantes aquatiques. Il en résulte une espèce de tourbe bâtarde, mais la fermentation qui la forme paroît être d’une nature différente. Il se dégage plus de matières gazeuses; et le voisinage des marais, dans lesquels il se décompose beaucoup de végétaux aquatiques, est ordinairement fiévreux et mal sain, tandis que le voisinage des tourbes parfaites, et formées dans des endroits originairement secs, est toujours salubre.
La partie terreuse de la tourbe est uniformément analogue à la nature de la couche sur laquelle cette tourbe repose: les plantes qui l’ont formée doivent avoir tiré du sol les terres qu’elles contenoient. Ainsi; dans le Wilshire, et le Berkshire, la couche sous la tourbe est crayeuse, la terre calcaire abonde dans les cendres de cette tourbe; et on y trouve très-peu d’alumine et de silique. Les cendres contiennent aussi beaucoup d’oxide de fer et de gypse: ils peuvent provenir des pyrites, qui abondent dans la craie.
Diverses expériences que j’ai faites, en brûlant des tourbes de districts granitiques et schisteux, m’ont donné des cendres principalement siliceuses et alumineuses; et une tourbe\setcounter{page}{471} tourbe du comté d'Antrim, dans un canton de basaltes, m'a presque fourni, par l'incinération, les parties constituantes des basaltes.
Des sols arides et ingrats, comme ceux qui résultent de la décomposition des granites, restent quelquefois pendant des siècles avant d'avoir une couche végétale peu épaisse. Les sols résultans de la décomposition des pierres à chaux, des craies, et des basaltes, se couvrent naturellement de graminées vivans, et fournissent pour la culture, toutes sortes de ressources.
Les bancs de rochers, qui sont l'origine des terres labourables, et ceux qui sont encore en nature dans l'intérieur du globe, sont disposés dans un certain ordre; et comme on voit des couches de nature très-différente, associées ensemble, et que les couches que l'on trouve en creusant, sont souvent utiles pour l'amélioration de la surface, il peut convenir de donner ici une idée de la manière dont les rochers et les diverses couches sont disposés dans l'intérieur de la terre.
Les géologues reconnoissent deux grandes divisions de rochers ou pierres: les primitifs et les secondaires.
Agricult. Vol. 18. No. 12. Déc. 1813. O q\setcounter{page}{472} Les rochers primitifs sont composés de matière cristallisée pure, et sans fragmens d'autres rochers.
Les rochers secondaires sont partiellement formés de matière cristallisée, et contiennent des fragmens d'une autre nature. Souvent aussi ils abondent en débris de végétaux et d'animaux marins : quelquefois ils contiennent des restes d'animaux terrestres.
Les rocs primitifs sont ordinairement disposés en grosses masses, en couches verticales, ou plus ou moins inclinées à l'horizon.
Les rocs secondaires sont ordinairement disposés en couches parallèles, ou presque parallèles à l'horizon.
Les rochers primitifs sont au nombre de huit.
1. Le granite, qui est composé de quartz, de feldspath, et de mica; et quand ces trois substances sont arrangées en couches parallèles dans le corps du rocher, on l'appelle gneiss.
2. Le schiste micacé, composé de quartz et de mica, disposés en couches, ordinairement curviligens.
3. La sienite, composée de la substance nommée hornblende, et de feldspath.
4. La serpentine, qui est composée de feld-\setcounter{page}{473} spath, et d'un corps nommé hornblende brillant. Leurs cristaux séparés sont souvent si petits, que la pierre a une apparence uniforme. La serpentine abonde en veines d'une substance nommée steatite, ou pierre savoneuse.
5. Le porphyre, qui est formé de cristaux de feldspath dans une pâte de même matière, mais ordinairement de couleur différente.
6. Le marbre granulé, qui est entièrement formé de cristaux de carbonate de chaux, et qui lorsqu'il est blanc, et que la texture est fine, est la matière employée par les statuaires.
7. Le schiste chlorite, qui est composé de chlorite, substance verte et grise, un peu analogue au mica et au feldspath.
8. Le rocher quartzeux, composé de quarz granulé, et quelquefois mélangé d'une petite quantité des élémens cristallisés qui appartiennent aux autres rochers.
\section{Les rocs secondaires sont plus nombreux que les primitifs}
mais douze variétés renferment toutes les pierres secondaires qu'on trouve dans les isles britanniques.
1. Le grauwacke, qui est formé de fragmens de quartz, ou de schiste chloriste dans\setcounter{page}{474} un ciment principalement composé de feldspath.
2. Grès siliceux, composé de fin sable quartzeux, lié par un ciment siliceux.
3. La pierre à chaux, qui consiste en carbonate de chaux, lequel a une texture plus compacte que le marbre granulé, et abonde souvent en débris marins.
4. Le schiste alumineux ou ardoise, qui est formé des substances décomposées de différens rochers, et cimentées par une petite quantité de matières ferrugineuses ou siliceuses : les ardoises contiennent souvent des impressions de végétaux.
5. Le grès calcaire, qui est du sable calcaire, cimenté par une matière calcaire.
6. La pierre de fer (iron stone), formée à-peu-près des mêmes matériaux que l'ardoise ou schiste alumineux, mais contenant beaucoup plus d'oxide de fer.
7. Les basaltes, qui consistent en feldspath et hornblende, avec des matériaux fournis par la décomposition des rocs primitifs. Les cristaux sont en général, si petits, que les basaltes ont une apparence homogène. Elles présentent ordinairement des colonnes à cinq ou six pans.
8. Le charbon de pierre.\setcounter{page}{475} 9. Le gypse ou sulfate de chaux, qui souvent contient du sable.
10. Le sel gemme.
11. La craie qui abonde d'ordinaire en débris d'animaux marins, et contient des couches horizontales de silex ou pierres à feu.
12. Les poudingues, qui consistent en petits cailloux fixés dans un ciment ferrugineux ou siliceux.
Il seroit inutile d'entrer dans plus de détails sur les parties constituantes des divers rochers. Il faudroit d'ailleurs, avoir les échantillons sous les yeux, pour apprendre à les distinguer: chose qui est facile avec un peu d'usage.
Les plus hautes montagnes des isles britanniques, et de tout le continent, sont formées de granite; et on a retrouvé les rochers granitiques dans les plus grandes profondeurs où l'homme ait pénétré. Le schiste micacée se voit quelquefois immédiatement sur le granite, et la serpentine ou le marbre, sur le schiste micacée; mais les rocs primitifs, sont groupés de diverses manières. Le plus souvent, le marbre et la serpentine, sont au-dessus; cependant le granite, qui paroît faire la base des couches pierreuses du globe, se trouve quelquefois placé par-dessus le schiste micacée,\setcounter{page}{476} Les rocs secondaires reposent toujours sur les primitifs. Le plus rapproché de ceux-ci, est ordinairement le grauwake. La pierre à chaux et le grès viennent ensuite. Le charbon de pierre se trouve ordinairement dans les ardoises ou les grès. Les basaltes sont le plus souvent au-dessus des grès et des pierres à chaux. Le sel gemme est presque toujours associé au grès rouge, et au gypse. Le charbon, les basaltes, les grès, les pierres à chaux, sont souvent disposés en lits alternatifs, d'une épaisseur et d'une étendue considérable. On a compté quatre-vingts couches alternatives de ces substances dans une profondeur de cinq cents yards (environ deux cent cinquante toises.) Les veines qui donnent des substances métalliques sont des fissures plus ou moins verticales, remplies d'une matière différente du rocher qui les recèle. Les substances qui constituent le minerai sont presque toujours cristallisées, et composées de spath calcaire, de spath fluor, de spath pesant, ou de quartz, soit ensemble, soit mélangés. Les substances métalliques sont en général, mélangées ou dispersées dans ces matières pierreuses. Les veines métalliques, logées dans le granite dur, donnent rarement beaucoup de métal utile;\setcounter{page}{477} mais l'on trouve souvent de l'étain, du cuivre et du plomb dans les gneiss et les granites tendres. Les veines métalliques de la serpentine ne donnent que du cuivre et du fer. Le chiste micacée, la siénite, et le marbre granulé, sont rarement métallifères. Le plomb, l'étain, le cuivre, le fer, et plusieurs autres métaux, se rencontrent dans les veines du schiste chlorite. Le grauwake est souvent métallifère, quand il existe en grandes masses. On y trouve de l'or, de l'argent, du fer, du plomb et de l'antimoine: quelquefois il contient des veines ou des masses de charbon de pierre, exemptes de bitume. La pierre à chaux est le rocher le plus métallifère parmi les secondaires: il donne souvent du plomb et du cuivre. On n'a jamais trouvé de veines métalliques dans les ardoises, ni dans les grès calcaires. Elles sont fort rares dans les basaltes et dans les grès siliceux. Lorsque les veines du rocher sont exposées à l'atmosphère, on peut souvent juger par leur apparence, si elles contiennent ou non des métaux. Si l'on trouve le spathfluor, il est extrêmement probable qu'elles contiennent aussi des substances métalliques. Une poudre brune à la surface, indique toujours la présence du fer, et souvent celle de l'étain.\setcounter{page}{478} Une poudre d'un jaune pâle annonce du plomb, et la couleur verte montre la présence du cuivre\footnote{}.
Je vais donner ici une description générale de la constitution géologique de l'Angleterre, de l'Écosse et de l'Irlande. La grande ligne de montagnes qui s'étend depuis Land's-End au travers de Dartmoor dans le Devonshire, est granitique. Les couches les plus élevées du Somersetshire sont du grauwacke et de la pierre à chaux. Les montagnes de Malverne sont composées de granite, de siénite et de porphyre. Les plus hautes montagnes du pays de Galles sont de chlorite, de schiste ou de grauwacke. On trouve du granite sur le mont Sorrel en Leicestershire. La grande ligne de montagnes qui s'étend dans le Cumberland et le Westmoreland, est de porphyre, de chlorite, de schiste et de grauwacke; mais on trouve le granite vers leurs frontières occidentales. Dans toute l'Écosse, les rochers les plus élevés sont de granite, de siénite et de schiste micacé. On ne trouve point de vrais rochers secondaires dans l'Angleterre, à l'ouest de Dartmoor, et point de basaltes au sud de la Severn. La craie s'étend de la partie occidentale de Dorset jusqu'à la côte orientale de Norfolk. Le charbon de pierre\setcounter{page}{479} abonde dans la partie qui sépare le Glamorgan du Derbyshire, ainsi que dans les couches secondaires du Yorkshire, de Durham, de Westmoreland, et Northumberland. On ne trouve la serpentine qu’en trois endroits, au cap Lézard, et dans les comtés d’Aberdeen et d’Ayr. Le marbre granulé noir et gris se trouve en Cornouailles; et des marbres primaires colorés, auprès de Plymouth. Les mêmes marbres abondent en Ecosse. Le marbre granulé blanc existe dans l’île de Sky, à Assynt, et sur le Lock Shin, en Sutherland. Les principaux bancs charbonneux de l’Ecosse sont en Dumbarton, Ayr, Tife, et sur la Brura. Des pierres à chaux et des grès calcaires se rencontrent sur-tout dans la partie basse au nord des hauteurs de Mendip.
En Irlande, il y a cinq grandes chaînes de montagnes primitives; celles de Moren dans le comté de Down, celles de Donegal, celles de Mayo et Galway, celles de Wicklow et celles de Kerry. Les rochers qui composent les quatre premières chaînes, sont principalement de granite, de siénite, de gneis, de schiste micacée et de porphyre. Les montagnes de Kerry sont principalement composées de quartz granulé et de chlorite. Près de Killarney, on trouve du marbre coloré.\setcounter{page}{480} et du marbre blanc sur la côte occidentale de Donegal. Les rochers secondaires les plus communs au midi de Dublin sont la pierre à chaux et le grès. La pierre à chaux, le grès, l'ardoise, la pierre de fer et le charbon de pierre, se trouvent dans les comtés de Sligo, de Roscommon et de Leitrim. Les montagnes secondaires de ces comtés s'élèvent à une hauteur considérable et leurs sommets sont formés de basaltes. La côte septentrionale d'Irlande est principalement basaltique. Les rochers qui la forment reposent sur de la pierre à chaux blanche, qui renferme des couches de silex et des fossiles ressemblables à ceux qu'on trouve dans la craie, mais elle est beaucoup plus dure que cette dernière substance. L'on voit dans quelques endroits de ces cantons, des colonnes basaltiques sur du grès et de l'ardoise; ces derniers sont disposés par lits alternatifs avec du charbon. Le charbon de pierre d'Irlande se trouve principalement dans le comté de Killkenny, associé avec la pierre à chaux et le grauwake. Il est évident, d'après ce que nous avons dit de la formation du sol par les rochers, qu'il doit y avoir autant de diversités dans les terrains qu'il y en a dans les espèces de pierres; il y en a même beaucoup plus. Indépendamment des changemens\setcounter{page}{481} produits par la culture et le travail de l'homme, les matériaux des diverses couches ont été mélangés et transportés d'un endroit à l'autre par les grands bouleversements de la terre, et par l'action constante des eaux. Ce serait un travail inutile que d'essayer de classer les terrains avec une exactitude scientifique. Les distinctions adoptées par les fermiers suffisent au but de l'agriculture, surtout si l'on applique les termes avec un certain degré de précision. Le mot sablonneux, par exemple, ne doit jamais être employé pour un sol qui ne contient pas au moins ⅖ de sable. Les sols sablonneux, qui font effervescence avec les acides, doivent être distingués par la dénomination de sablonneux calcaire, pour ne pas les confondre avec ceux qui sont siliceux. La qualification de sol glaiseux ne doit être appliquée qu'à celui qui contient au moins ⅗ de matière terreuse impalpable et qui ne fait pas une effervescence sensible avec les acides. Pour qu'un sol puisse être désigné sous le nom de tourbeux, il faut qu'il soit formé au moins pour la moitié de matières végétales.\footnote{Lorsque la partie terreuse d'un sol provient évidemment d'un roc décomposé, le mot qui désigne cette espèce de roc pent}\setcounter{page}{482} (182) AGRICULTURE.
avec convenance lui être appliqué; c'est ainsi qu'on appelle sol basaltique, une fine terre rouge, qu'on trouve immédiatement au-dessus des basaltes en décomposition, etc.
En général, les terrains composés de beaucoup de parties hétérogènes, sont ceux qu'on appelle terrains d'alluvion; c'est-à-dire, formés par le dépôt des rivières: ils sont, pour la plupart, extrêmement fertiles. J'ai examiné plusieurs sols d'alluvion très productifs, et je les ai trouvés fort différens les uns des autres. Celui des bords de la rivière Parret en Somersetshire, et dont j'ai déjà parlé, donnoit huit parties de matière terreuse, fort divisée, et une partie de sable siliceux. L'analyse de la matière terreuse produisit:
360 parties de carbonate de chaux
25 Alumine.
20 Silice.
8 Oxide de fer.
19 Matières animales, végétales, et salines.
Un sol riche dans le voisinage de la rivière d'Avon, dans la vallée d'Evesham en Worcestershire, m'a donné les ¹/³ d'un sable fin, et ²/³ de matière impalpable; celle-ci consistoit en :\setcounter{page}{483} 35 Alumine.
12 Silice.
14 Carbonate de chaux.
5 Oxide de fer.
8 Matières végétales, animales, et salines.
Un sol d'excellent pâturage de la vallée d'Avon près de Salisbury, m'a donné 1/11 de grossier sable siliceux, et la matière très-divisée consistoit en :
7 Alumine.
14 Silice.
69 Carbonate de chaux.
2 Oxide de fer.
14 Matières végétales, animales, et salines.
Un échantillon d'un bon sol de la vallée de Tiviot a donné 1/5 de fin sable siliceux et un sixième de matière impalpable, qui contenoit :
41 Alumine.
12 Silice.
4 Carbonate de chaux.
5 Oxide de fer.
8 Matières végétales, animales, et salines.
Dans tous ces cas, la fertilité paroît dépendre de l'extrême division, ainsi que du mélange des matières terreuses avec les substances animales et végétales : cela s'accorde\setcounter{page}{484} avec les principes que nous avons posés ci-devant.
Lorsqu'on examine un sol stérile avec le projet de l'améliorer, il faut porter son attention sur les substances qui nuisent à sa fécondité, et s'il est possible, le comparer avec un autre sol fertile du même canton et dans une situation semblable. Il arrivera souvent que la différence dans leurs compositions, indiquera la meilleure manière d'améliorer. Si lorsqu'on lave un échantillon d'un sol stérile, il se trouve contenir des sels ferrugineux ou des matières acides, il peut être amélioré par le mélange de la chaux.
Un échantillon d'un sol de bonne consistance apparente, me fut montré par sir Joseph Banks, comme remarquable par sa stérilité. En l'examinant, je vis qu'il renfermoit du sulfate de fer, et je proposai le remède facile, de répandre la chaux à la surface, opération qui convertit en un engrais le sol ferrugineux. Lorsqu'il y a excès de matière calcaire, on peut y remédier par une addition de sable ou de glaise. Les terrains trop sablonneux se corrigent par l'addition de la glaise, de la marne, ou de matières végétales.
- Un champ appartenant à Sir R. Vaughan\setcounter{page}{485} en Merionetshire, dont le sol étoit un sable léger, souffrit beaucoup de la sécheresse dans l'été de 1805. Je recommandai au propriétaire de répandre de la poussière de tourbe à la surface. Le bon effet en fut immédiatement sensible; et Sir Robert m'a informé que l'avantage avoit été permanent. Le défaut de matières animales et végétales, doit être corrigé par l'application du fumier. L'excès de la matière végétale, doit être corrigé par l'écobuage, ou par des terres rapportées. L'amélioration des terrains tourbeux, froids ou marécageux, doit être précédée de l'opération du dessèchement, car la stagnation de l'eau est nuisible à toutes les classes de plantes nutritives. Les tourbes noires et tendres une fois desséchées, sont souvent rendues fertiles par la simple application du sable ou de la glaise à leur surface. Lorsque les tourbes sont acides, ou contiennent des sels ferrugineux, il faut absolument l'application d'une matière calcaire pour les rendre fertiles. Si les tourbes renferment beaucoup de branches et de racines d'arbres, il faut les enlever ou les brûler. On doit en user de même lorsque la surface des tourbes est entièrement garnie de plantes en végétation. Dans ce dernier cas, les cendres fournissent\setcounter{page}{486} des matières terreuses propres à donner une bonne consistance à la tourbe.
Les meilleurs terrains naturels sont ceux dont les matériaux proviennent de couches différentes; lorsque le mélange opéré par l'air et l'eau, est devenu bien intime; et, dans l'amélioration d'un sol, un agriculteur ne saurait mieux faire que d'imiter le procédé de la nature.
Les matériaux nécessaires pour cet effet sont rarement séparés par de grandes distances. Le gros sable repose souvent immédiatement sur la craie; et l'on trouve communément le gravier sous la glaise : un grand avantage permanent paie les frais de ce travail : il faut ensuite moins d'engrais, et la fertilité du terrain est assurée indéfiniment.
(La suite à un prochain Cahier.)
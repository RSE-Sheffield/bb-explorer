\setcounter{page}{137} \section{ISTRUZIONI PRATICHE, etc. Instructions pratiques sur la manière de bien faire et conserver le vin; par le Sénateur DANDOLO. Milan 1812. \large{(Second extrait. Voyez p. 77.)}}

CHACUN sait, par la pratique, ce que c'est que la fermentation : c'est-à-dire, ce mouvement intestin qui agite, sépare, soulève et tourmente, en quelque sorte, les corps qui y sont soumis. Par exemple, un tas de fumier frais, ou la pâte à laquelle on a mêlé du levain.
Le raisin, avant que d'être écrasé, ne sauroit fermenter : nous en avons vu la raison. Il en est de même des poires, des pommes, des cerises, etc. tant que ces fruits sont entiers ; mais lorsque ces fruits doux ont été écrasés, toutes leurs parties se mélangent entr'elles, le levain qui auparavant ne pouvoit pas agir, exerce son action, et la fermentation s'en suit. Alors la liqueur douce\setcounter{page}{138} devient vineuse, ce qui démontre que le sucre s'est converti en substance spirituelle.
Il importe que l'agriculteur connoisse :
1°. Comment la fermentation commence, procède et se termine.
2°. Quelles sont les causes externes qui peuvent influer sur la fermentation.
3°. Ce qui se forme dans la cuve par ce mouvement de fermentation.
4°. Comment il faut régler la cuve pour obtenir le meilleur vin.
Du cours visible de la fermentation.
Peu de temps après que les raisins, bien écrasés, ont été mis dans la cuve, le mouvement de fermentation commence.
Des petites bulles s'élèvent à la superficie.
Les corps solides commencent à y monter aussi, et à former à la surface ce qu'on appelle le chapeau.
Il se manifeste de l'écume à sa surface.
Le chapeau continue à gonfler.
La cuve reste quelque temps dans une apparente tranquillité ; mais à ce repos succède bientôt un mouvement très-actif de fermentation, et le chapeau s'élève à sa plus grande hauteur.
Le gaz acide carbonique s'échappe par toutes les parties non couvertes. La chaleur\setcounter{page}{139} et le volume de la liqueur augmentent. L'odeur vineuse se répand avec d'autant plus de force qu'on prend moins de précautions pour empêcher qu'elle ne s'échappe. La fermentation commence à décliner. La liqueur qui auparavant étoit trouble, sans fumet, et d'un rouge pâle, devient claire, prend du parfum et une belle couleur. Le chapeau s'abaisse, la chaleur diminue, la fermentation tumultueuse cesse; et le moût est presqu'entièrement converti en vin.
\section{Des causes externes qui influent sur le mouvement de la fermentation.}
Les causes externes qui ont de l'influence sur la fermentation peuvent se réduire à trois. 1°. La saison, 2°. l'air extérieur, 3°. la grandeur des cuves.
Si la saison est chaude, la fermentation est vive et forte. Il est utile alors de donner accès à l'air frais, lequel modère le mouvement; et il faut veiller à ce que le vin ne bouillisse pas trop, de peur qu'il ne contracte une disposition à l'acidité. Une température douce, ou moyenne\setcounter{page}{140} de la cuve : elle permet d'observer la marche du procédé, et de la guider.
Si la température est un peu froide, la fermentation languit, et sa marche n'est pas si régulière. Il convient alors de fermer et réchauffer le lieu, pour obtenir la température de 10 à 12 degrés de Réaumur.
L'air extérieur influe sur la fermentation, en ce qu'il reçoit le gaz acide carbonique qui se dégage. Si ce gaz ne pouvait se dégager, la fermentation cesserait. Le courant d'air sur la cuve ne doit cependant pas être trop fort, pour ne pas absorber une portion considérable d'esprit-de-vin, de fumet, et du vin même qui se trouve dans le chapeau.
Ce chapeau n'est autre chose qu'une éponge; or, une éponge qu'on imbiberait de vin et qu'on exposerait à un courant d'air, se sécherait bientôt.
Si la cuve n'avait pas un couvercle, l'air emporterait une plus grande quantité de la liqueur. Si l'on essaie de laisser exposée à l'air, une certaine quantité de moût, et qu'on le pèse au bout d'un certain temps, on voit, qu'il s'en est évaporé une portion notable.
La grandeur des cuves a aussi une grande influence sur la durée de la fermentation.
Le degré de chaleur, et par conséquent la\setcounter{page}{141} force de la fermentation est toujours propor-tionnée à la grandeur de la cuve.
Dans les grandes cuves, la chaleur est plus forte et la fermentation s’accomplit plus promptement.
Dans les cuves moyennes, la chaleur l’est aussi, et la fermentation a une durée moyenne.
Dans les petites cuves, la chaleur est foi-ble, et la fermentation lente.
\section{Des substances et des modifications nouvelles qui se forment dans la cuve.}
Le gaz acide carbonique se forme et se dégage à mesure que le moût se change en vin. L’esprit-de-vin se développe et se fixe dans la liqueur. Les substances qui étoient combinées avec le moût, et que la fermen-tation a développées, sont, le calorique, la couleur fixe du vin, et le parfum.
\section{De la manière de gouverner la fermentation.}
Pour bien régler la fermentation, il faut empêcher que l’esprit-de-vin, le parfum, et le vin lui-même, ne se perdent. Pour cela il faut :
1°. Bien fouler la cuve une fois seulement.
2°. Après avoir rempli la cuve, et avoir laissé former le chapeau, le rendre égal par\setcounter{page}{142} dessus, avant de placer le couvercle.
3°. Empêcher que la cuve ne soit trop exposée au froid.
4°. Garantir la cuve des vents, qui emportent une partie du vin, et refroidissent la liqueur.
5°. S'il fait trop chaud, donner accès à l'air frais. Il y a toujours plus à craindre de la chaleur que d'une température moyenne.
\section{Des signes qui indiquent le moment convenable pour écouler la cuve.}
Nous avons vu de quelles substances est composée la matière renfermée dans la cuve, comment ces substances agissent, et quelle influence elles peuvent avoir sur la quantité et la qualité du vin. Nous avons vu quels sont les signes du développement, du progrès et de la fin du mouvement de fermentation dans la cuve; l'influence des causes externes sur cette fermentation; les nouveaux produits et les nouvelles modifications qui se forment dans la cuve; enfin, de quelle manière il convient de gouverner celle-ci, pour que la fermentation remplisse l'objet le mieux possible. Voyons maintenant quel est le moment qu'il faut saisir pour écouler le vin de la\setcounter{page}{143} cuve, pour le transporter dans les tonneaux. Le choix de ce moment est d'une grande importance pour la qualité du vin et sa conservation.
Si tous les raisins dont on fait du vin avoient mûri précisément dans les mêmes circonstances de climat, de sol, d'exposition, etc. on pourroit connoître avec certitude, le temps que le vin doit mettre à se former dans la cuve par la fermentation, et par conséquent le moment précis où il conviendroit d'écouler; mais l'expérience démontre que les raisins different entr'eux, et sur-tout relativement à la quantité de matière sucrée et de levain qu'ils contiennent. Par conséquent, à circonstances d'ailleurs égales, le cours de la fermentation dure plus ou moins, selon que le levain et la matière sucrée se trouvent plus ou moins abondans. Cependant, si l'on écoule trop tôt le vin de la cuve, il est d'une conservation extrêmement difficile; et si l'on écoule trop tard, il conserve une disposition à l'ascenscence.
Il faut donc réunir tous les signes et toutes les observations utiles pour saisir le meilleur moment d'écouler. Ces signes sont au nombre de six, savoir:
1°. L'affaissement du marc qui forme le chapeau.\setcounter{page}{145} 2°. La cessation d'ébullition.
3°. L'uniformité de la saveur du vin dans toutes les parties de la cuve.
4°. L'uniformité de la couleur.
5°. L'uniformité de la limpidité.
6°. Le refroidissement de la liqueur.
Le premier signe visible de l'approche de la cessation du mouvement, est l'abaissement du chapeau. Le volume de la liqueur diminue; quand le dégagement du gaz acide carbonique tire à sa fin. D'ailleurs, la pesanteur spécifique du vin étant moindre que celle du moût, le chapeau tend à s'y enfoncer. Il faut donc se tenir sur ses gardes, dès que l'on voit baisser le chapeau, parce que le moment d'écouler approche.
Ordinairement la fermentation régulière est terminée, lorsqu'en appuyant l'oreille au chapeau, on entend à des intervalles de quelques secondes, la petite détonation d'une bulle de gaz, qui part du fond ou du centre de la cuve. Cela indique que la substance sucrée et le levain sont au moment de ne plus agir l'une sur l'autre, parce que leurs quantités sont trop réduites par la fermentation.
Si l'on puise un peu de liqueur dans une tasse, à différentes époques de la fermentation, et à divers endroits de la cuve, l'on\setcounter{page}{146} remarque un goût différent à chaque fois que l'on puise, c'est-à-dire, à mesure que les diverses substances contenues dans la cuve se décomposent, et qu'il se forme de nouvelles combinaisons. Lorsque la saveur est devenue complètement vineuse, c'est le moment d'écouler. Un palais exercé ne s'y trompe pas.
L'uniformité de couleur montre également que les nouvelles combinaisons sont achevées. Auparavant, les teintes varient d'un endroit à l'autre de la cuve; et on s'en aperçoit si l'on puise doucement avec un verre en différentes places. Lors donc que la couleur du vin de la cuve devient uniforme, le moment d'écouler approche.
La limpidité de la liqueur résulte des mêmes causes qui égalisent le goût et la couleur. La décomposition de toutes les substances qui devoient y être soumises est achevée: sans cette décomposition, il subsiste toujours dans la liqueur un mouvement qui ne permet point aux corps qui la troublent de se déposer.
Enfin, à mesure que la substance sucrée se décompose en esprit-de-vin, la chaleur va en diminuant, et quand cette décomposition est achevée, le vin est à la température de l'atmosphère. C'est alors le moment d'écouler.\setcounter{page}{147} Il y a aux règles ci-dessus, diverses exceptions dépendantes des qualités du moût et des circonstances accidentelles.
Le moût est peu sucré, suffisamment sucré, ou très-sucré. Le premier, dans lequel le levain est en quantité surabondante, donne un vin léger, et qui passe aisément de la fermentation vineuse à la fermentation acide. Pour éviter ce danger, on ne doit pas attendre pour écouler le vin, le concours de tous les indices ci-dessus. Il suffit que le chapeau commence à s'abaisser, que les petites détonations du gaz soient moins fréquentes, et sur-tout que la liqueur ait pris un goût tout-à-fait vineux.
Dans la fabrication des vins légers, il convient d'écouler, tandis que le vin contient encore une portion de la matière sucrée. Alors, le levain, qui dans ces vins-là est toujours abondant, agit dans le tonneau, sur la substance sucrée, et non sur les autres substances. Le vin est ainsi préservé de l'accident auquel il est exposé lorsque la substance sucrée est totalement anéantie dans la cuve.
Le froid qui survient ensuite, ralentit le mouvement de la fermentation, le levain et la lie se déposent au fond des tonneaux; et en débarrassant le vin de ces deux en\setcounter{page}{148} nemis, l'agriculteur assure la conservation de son vin.
Pour les vins légers, la fermentation lente dans les tonneaux suppléée à ce qu'il manque en fermentation de la cuve.
Ce que nous avons dit de l'affaissement du chapeau s'applique aussi aux petites détonnations. Il ne faut pas attendre qu'elles soient devenues aussi rares, lorsqu'il s'agit de vins légers, que pour les vins forts.
Observons encore qu'un jour de trop dans la fermentation de la cuve peut développer les principes d'acidité qu'il est ensuite extrêmement difficile d'extirper, au lieu qu'un jour de moins qu'il ne faudrait dans le choix du moment pour écouler, laisse encore la ressource de la fermentation lente du tonneau, et des soins qu'on donne utilement au vin dans ce cas.
Quand le moût est bon, c'est-à-dire, que la proportion de la substance sucrée et du levain est juste, il n'y a aucun risque à courir en laissant achever la fermentation dans la cuve.
Si le principe sucré est plus abondant que le levain, il n'y a aucun inconvénient à laisser le vin dans la cuve vingt-quatre heures après que la fermentation est arrêtée et la liqueur refroidie. Il n'y a aucun danger\setcounter{page}{149} d'acesscence dans ces vins-là. Ce prolongement de séjour dans la cuve, a l'avantage de faire déposer tout ce qui peut l'être, en lie dans la cuve; mais les vins violents n'ont jamais, quand on les écoule, la même transparence que les vins légers.
Les circonstances accidentelles qui modifient l'application des règles pour le choix du moment pour écouler, sont 1°. les différentes grandeurs des cuves, 2°. les changemens brusques de température, 3°. le froid soutenu.
S'il s'agit d'une grande cuve, et qu'on attende que le vin soit froid, comme pour une petite, on risque de le voir tourner à l'aigre, avant d'être complètement refroidi.
Les grandes cuves achèvent plus promptement que les autres, la fermentation de la liqueur; en sorte que le vin est encore tiède, quand les autres indices de l'instant d'écouler se manifestent.
Souvent une température froide qui survient fait affaisser le chapeau, et ralentit le cours de la fermentation. Il ne faut pas s'en inquiéter; ce cours reprend dès que le temps s'adoucit. Le chapeau alors se relève, ou du moins il ne s'abaisse pas davantage. Cependant si le froid survient quand la fermentation approche de sa fin, elle cesse quelquefois\setcounter{page}{150} quelquefois brusquement. C'est alors la saveur de la liqueur qui doit diriger: cette saveur n'est pas complétement vineuse, et il convient de suspendre l'écoulement.
Le froid éclaircit aussi sensiblement la liqueur; mais ce n'est pas d'une manière égale et complète dans toutes les parties de la cuve; et d'ailleurs le goût du vin n'est pas tout-à-fait développé, ni égal par tout.
S'il revient tout-à-coup un temps chaud, la liqueur se trouble de nouveau, et il semble que la fermentation recommence. Alors le chapeau ne s'abaisse pas; et cependant la liqueur n'est ni froide ni claire. Il faut, dans ce cas, se décider sur le goût uniformément vineux dans toute la cuve, sans attendre la limpidité et le refroidissement complet.
On peut en dire autant des raisins blancs. La quantité de matière solide et féculente qu'ils contiennent, fait qu'ils sont lents à s'éclaircir, et que la saveur seule doit décider du moment de les écouler.
Si la température se soutient chaude, c'est-à-dire, au-dessus de douze degrés de Réaumur, qui est la température la plus favorable à une fermentation régulière, il convient d'écouler, lors même que les petites
Agriculture. Vol. 18. N°. 4. Avril 1813, M\setcounter{page}{151} détonnations du gaz sont encore fréquentes, que le vin n'est pas encore bien limpide, et que le chapeau n'est que peu affaissé; car le mouvement intestin de fermentation étant plus énergique, le dépôt des substances hétérogènes en est plus lent, et le volume de la liqueur, augmenté par le dégagement des gaz, soutient le chapeau plus élevé. Il seroit donc dangereux de voir disposer le vin à l'acescence si l'on n'écoulait aussitôt que la saveur est uniformément vineuse. S'il reste de la substance sucrée en nature, il faudra plus d'attention dans le gouvernement du vin en tonneaux; mais toujours vaudra-t-il mieux écouler trop tôt que trop tard si la température est chaude. Après ces indications et ces exceptions, un bon observateur doit juger les causes, et se déterminer sur le bon moment pour écouler une cuve. Si l'on desire plus de détails encore, on les trouvera dans les deux volumes de l'Œnologie \footnote{En deux volumes. Milan, 1812, chez Giovanni Silvestri} , duquel ouvrage celui-ci est tiré.
\section{De l'écoulement de la cuve}
Lorsqu'on écoule, la cuve contient du\setcounter{page}{152} vin clair, du vin trouble, et des substances pénétrées de vin. Ces trois choses doivent être soigneusement séparées.
Dans ce moment-là, le vin n'est pas encore parfait, c'est-à-dire, que ses parties ne sont pas encore bien combinées; qu'il a une force d'expansion, résultante du gaz acide carbonique qui cherche encore à s'échapper, sur-tout si on l'agite, et qu'enfin le vin étant encore plus chaud que l'air extérieur, perd aisément son esprit et son principe odorant.
Cependant la manière ordinaire d'écouler, semble inventée exprès pour favoriser l'évaporation et la déperdition des principes les plus précieux. Si l'on écoule sans précaution l'odeur d'esprit-de-vin se répand au loin. Or l'esprit et le fumet, une fois évaporés, le vin en perd d'autant sa qualité, et cela est sans remède. Le contact de l'air dans l'opération d'écouler nuit encore au vin par d'autres causes.
Si l'on écoule à l'air libre, il se mêle au vin, une quantité considérable d'air ambiant, avec les impuretés que cet air contient. Il s'échappe au contraire ( ainsi qu'on peut s'en apercevoir à l'écume) une certaine quantité d'air qui étoit dans le vin,\setcounter{page}{153} et qui emporte avec lui de l'esprit et du fumet. Cet esprit est si abondant qu'on s'énivrer par la seule odeur auprès d'une cuve que l'on écoule. Il se fait ainsi un échange des principes les plus précieux contre des principes altérans. Le gaz acide carbonique, qui ajoutoit au piquant et à la force du vin, s'échappe aussi en partie, et est lui-même, comme véhicule, une occasion de la perte de l'esprit et du fumet.
Il existe encore dans le vin devenu clair, et prêt à écouler, des substances solides qui se déposeront en lies par le repos, le froid, et l'effet d'une fermentation lente; mais le grand mouvement que le vin clair éprouve dans l'acte même d'écouler à l'air libre selon la méthode commune, accélère la séparation de ces parties solides, et trouble la liqueur. Ainsi l'on a quelquefois de la peine à reconnoître à l'apparence et au goût le vin récemment transporté dans le tonneau, en le comparant à ce qu'il étoit dans la cuve.
De la manière d'écouler le vin de la cuve, et de le transporter dans le tonneau.
Avant cette opération on doit avoir des tonneaux préparés en quantité convenable,\setcounter{page}{154} \section{ART DE FAIRE ET DE CONSERV. LES VINS.} (1539)
pour pouvoir les remplir complétement du vin que l'on va tirer de la cuve; sans cela il resteroit dans le tonneau un espace rempli d'air atmosphérique; et le contact de celui-ci est toujours dangereux pour le vin.
Lorsque les tonneaux sont bien lavés et préparés, on commence par enlever dourement de la superficie du chapeau, les grappes et matières solides qui s'y trouvent dans un état d'altération ou d'aigreur: cette opération se fait avec la main, ou avec une espèce de fourchette de bois qui y est destinée.
Ce qui importe ensuite, c'est de mettre tous ses soins à ce que le vin soit le moins possible en contact avec l'air extérieur dans le passage de la cuve au tonneau. Il seroit sans doute très-avantageux de pouvoir opérer cette transfusion par des tuyaux; mais la chose seroit d'une exécution difficile. Voici donc à quoi on doit se borner. Il faut avoir une brande couverte, que l'on place immédiatement au-dessous du robinet, que, pour cette opération, on a adapté à la cuve. Ce robinet doit être construit de manière que son bec puisse entrer dans la brande par une ouverture faite du couvercle de celle-ci dans ce but.\setcounter{page}{155} On ajoute au tuyau de l'entonnoir placé sur le tonneau à remplir, un autre tuyau de fer-blanc, qui descend verticalement jusques près du fond du tonneau. C'est avec des dispositions semblables, qu'on évite autant qu'il est possible, le contact de l'air dans le transport du vide, car on vide la brande dans l'entonnoir, par le même trou au couvercle qui a servi de passage au bec du robinet.
Il faut aussi que la partie de l'entonnoir dans laquelle le vin n'entre pas, demeure couverte. Le mieux est d'employer un tube aérifère. Mais faute de cet instrument, il faut avoir soin que le tube de fer-blanc qui termine l'entonnoir, soit d'un diamètre moindre que celui du trou du bondon, afin que l'air puisse s'échapper du tonneau, à mesure que le vin y entre.
Il faut employer plusieurs brandes pour accélérer l'opération. A mesure qu'un tonneau est rempli, on transporte à un autre, l'entonnoir avec l'appendice du tube de fer-blanc pour pénétrer jusqu'au fond du tonneau.
Lorsque le vin commence à sortir trouble de la cuve, on ne le mêle point avec le vin clair: on le met dans un tonneau qui a été préparé pour cela.\setcounter{page}{156} Dans la méthode ordinaire d'écouler et transporter le vin, il est battu, et exposé à l'air libre, de manière à perdre une grande partie de sa qualité; et lorsque la température se refroidit, il est en danger de s'altérer.
\section{Du vin trouble.}
C'est une attention essentielle à la réussite des vins, de séparer celui qui est trouble de celui qui est clair. Celui-ci ne contient que des substances bien dissoutes; l'autre en contient qui sont en partie dissoutes, et en partie divisées dans la liqueur.
C'est par le repos que les parties suspendues et divisées dans la liqueur, s'affaissent en lie; et comme ces parties non dissoutes mettent le vin en un danger continuel de s'altérer, en même temps qu'elles lui ôtent de son goût et de sa valeur; on ne doit pas risquer de gâter les vins clairs par le mélange du vin trouble.
La quantité plus ou moins grande du vin trouble dépend de la manière de faire le vin, de la température de l'année, de la qualité des raisins, de la durée de la fermentation, etc.
Si le trou de la cuve par lequel on écoule\setcounter{page}{157} le vin étoit trop petit, l'opération se prolongeroit trop, et le marc du chapeau pourroit tomber au fond. Il faut que l'écoulement soit rapide, et l'opération prompte.
Les raisins blancs contenant plus de matière solide, ne s'éclaircissent souvent pas bien dans la cuve, et la proportion du vin trouble y est toujours plus forte.
En général, plus les raisins étoient mûrs, et plus la fermentation a été régulière, plus aussi la quantité du vin trouble est petite : quelquefois on écoule jusqu'au fond de la cuve, avant de le trouver.
Dans tous les cas, il importe de fermer le robinet au moment où le vin qu'on soutire cesse d'être clair.
\section{Du marc, du sédiment, et du vin qu'on en tire.}
Quand on a soutiré le vin clair et le vin trouble, le marc qui étoit à la superficie, se trouve déposé au fond de la cuve, avec la lie qui s'est précipitée pendant la fermentation. Ces substances sont pénétrées de vin; et une partie de celui-ci s'écoule par le trou de la cuve.
On transporte ensuite le marc sur le pressoir.\setcounter{page}{158} soir, lequel, ainsi que la petite cuve destinée à recevoir le vin exprimé, doit avoir été préalablement lavée avec soin.
Il importe de ne point laisser trop longtemps le marc au fond de la cuve, parce qu'il s'échauffe promptement et qu'une partie tourne à l'aigre, s'il fait un peu chaud.
Le vin qui s'écoule du pressoir, doit être transporté sans retard dans les tonneaux; et et il faut remarquer que si ceux-ci sont très-alongés, le dépôt des lies se fait plus promptement.
C'est sur le vin qui est sorti du pressoir, que l'on peut observer sur-tout l'avantage qu'il y a, à suivre les indications données ci-dessus pour bien nettoyer les grappes, bien écraser les raisins, bien couvrir la cuve, et bien conduire la fermentation.
Si les raisins n'ont pas été nettoyés des grains secs, moisis ou verts, si tous les grains n'ont pas été écrasés, il en résulte un mauvais goût après l'action du pressoir, et du vin dont la conservation est plus difficile; au lieu que si les soins indiqués ci-dessus sont bien observés, le vin pressé est, au bout de quelques mois, fort difficile à distinguer du vin soutiré, et se vend à-peu-près au même prix. Or, comme le vin pressé monte\setcounter{page}{159} environ au quart de la quantité du vin soustiré, on voit que le résultat est important. Les mêmes soins qui donnent au vin une qualité supérieure, rendent aussi plus profitable l'opération de faire la buvande, ou piquette, au moyen de laquelle on tire plus de parti du marc. La proportion d'eau doit être le quart du vin soutiré, c'est-à-dire, autant qu'on auroit eu de vin pressé. Cette eau doit rester de vingt-quatre à quarante heures sur le marc, selon la température, c'est-à-dire, plus long-temps si elle est froide. On soutire ensuite une liqueur qui peut se garder plusieurs mois dans une bonne cave. Si l'on ajoute à cette buvande cinq pour cent de bon moût, l'on obtient un petit vin piquant, fort agréable, et qui peut se conserver. On auroit peine à se persuader quelle différence il y a entre le vin pressé d'une vendange bien faite, et celui que le pressoir donne après une vendange mal faite: la même différence existe entre le petit vin tel que nous recommandons de le faire, et la piquette ordinaire faite sur de mauvais marc. L'eau-de-vie qu'on retire du marc est d'une qualité toujours proportionnée à la quantité de vin fermenté qui s'y trouve contenue, et\setcounter{page}{160} non pas à la quantité des substances aigres ou non fermentées que le marc renferme. L'auteur a retiré d'un marc pressé, qui avoit donné un vin bien fabriqué, dix pour cent d'eau-de-vie au vingt-quatrième degré du pèse-liqueur de Beaumé, et sans aucun mauvais goût. Il a retiré aussi cinq pour cent d'eau-de-vie de même qualité, du même marc, sur lequel on avoit mis de l'eau avant de le presser, pour en faire de la buvande.
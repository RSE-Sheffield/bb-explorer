\setcounter{page}{224}
\section{PRINCIPES RAISONNÉS D'AGRICULTURE. Traduit de l'allemand d'A. THAER, par E. V. B. CRUD. Tom. Ier. in-4°, 372 p. Genève, 1811, chez J. J. Paschoud, Imprimeur-Libraire; et à Paris chez le même rue Mazarine, n°. 22. \large{(Vingtième extrait. Voy: p. 183.)}}
L'AUTEUR, après avoir traité de l'agronomie dans sa quatrième section, en vient à l'agriculture proprement dite. Il la définit, l'art de mettre le sol en état de produire les récoltes qu'on en exige, dans la perfection qu'on doit désirer. Il distingue l'agriculture chimique, qui consiste à amender les terres pour augmenter leur fécondité, et l'agriculture mécanique, qui consiste à travailler le terrain pour que les racines des plantes puissent y pénétrer facilement, et s'y approprier les substances analogues à leur nature. Nous allons le suivre dans le traitement de ces deux matières.
Des engrais ou de l'amendement des terres.
Les engrais agissent de deux manières sur le sol.
1°. En lui communiquant des sucs pro-\setcounter{page}{225} près à la nutrition des plantes. 2°. En exerçant sur les substances que le sol contient déjà, une action chimique qui les décompose et les combine ensuite de nouveau, de manière que leur introduction dans les suçoirs des plantes soit plus facile ; peut-être aussi, en rendant aux végétaux cette vigueur, et cette activité, à l'aide desquels ils s'approprient les sucs nutritifs.
Quelques espèces d'engrais paroissent ne produire que l'un de ces effets, ou du moins le produire plus particulièrement ; d'autres, au contraire, semblent opérer l'un et l'autre.
Pour l'ordinaire, nous exprimons l'action des engrais sur le sol, en disant qu'ils le fertilisent, et pour bien des gens, cette expression peut être suffisante ; cependant, nonseulement pour la théorie, mais aussi pour la pratique, il est d'une grande importance de distinguer de quelle manière chaque espèce d'engrais produit cet effet, et sous quelles circonstances, elle agit d'une manière plutôt que de l'autre. C'est seulement à l'aide de cette connoissance que nous parviendrons à nous expliquer divers faits contradictoires en apparence, et que nous pourrons faire un bon choix entre les procédés à suivre dans l'emploi de l'une ou de l'autre espèce d'engrais.
Ce n'est pas sans raison que les Anglais\setcounter{page}{226} ont comparé les engrais du premier genre aux alimens, et ceux du dernier, au sel, aux épices et aux boissons stimulantes. Toutes les substances organiques qui sont entrées en putréfaction ou en décomposition, contiennent les élémens nécessaires à la reproduction et au complet développement des végétaux que nous cultivons. Si, par le moyen des semences et des racines, nous mettons les germes d'une plante quelconque en contact avec ces substances, et que cette opération soit d'ailleurs faite de la manière convenable, il en résulte des végétaux du même genre que cette plante. Le terreau (ou humus) contient des alimens pour tous les végétaux; cependant, suivant toutes les apparences, il ne les renferme pas en quantités égales; c'est-à-dire, que ce terreau n'est pas toujours composé des mêmes substances élémentaires, et que suivant sa composition, il favorise plus la végétation d'une plante que celle d'une autre. C'est presque uniquement comme aliment que le terreau végétal paroît agir sur les plantes; il ne semble contribuer que peu au développement des parties que le sol contient déjà; de ces parties qui sont le résidu du terreau lui-même et qui sont devenues insolubles. Le terreau animal, au contraire, opère l'un et l'autre effet; non-seulement il\setcounter{page}{227} contient toutes les substances nécessaires à la nutrition des plantes, et même quelquesunes que le terreau végétal ne possède qu’en petite quantité, mais encore il favorise la décomposition de l’humus insoluble, et donne à l’activité des plantes une plus grande force.
Les engrais minéraux, s’ils ne contiennent aucune matière organique, opèrent uniquement, ou du moins essentiellement, par la faculté qu’ils ont de favoriser la décomposition.
Les divers corps organiques sont formés par la combinaison de trois ou quatre substances élémentaires, réunies par la force vitale dans des proportions déterminées; mais lorsque la force vitale cesse d’agir sur elles, ces combinaisons sont, du moins en partie, soustraites aux lois organiques des corps qu’elles constituoient. Les substances éléementaires se réunissent alors, tant en combinaisons simples, c’est-à-dire, deux à deux, d’après
les lois de l’affinité, qu’en combinaisons plus composées et d’un genre nouveau. Sans appartenir à la vie, ces dernières combinaisons lui doivent cependant l’existence, et elles lui servent à leur tour d’aliment. C’est principalement d’elles que se nourrissent les plantes, lesquelles deviennent elles-mêmes la nourriture des animaux.
Ces matières nouvellement formées, et\setcounter{page}{228} L'humus qui en provient, varient de nature suivant qu'elles doivent leur existence à des substances différentes, et suivant les circonstances qui ont opéré leur combinaison. Le procédé de leur transformation est ce que nous exprimons par décomposition, fermentation, putréfaction. Ce n'est pas ici le lieu de définir ces termes, cependant nous devons faire à leur sujet, les observations suivantes.
Les conditions de l'état que nous désignons par ces mots sont, outre l'absence de la vie, la chaleur, l'humidité, et une sorte de combinaison avec l'atmosphère. Selon certaines circonstances, ce procédé éprouve des modifications variées, il a une marche plus prompte, et il donne des résultats différens.
Les corps végétaux passent par les divers degrés de fermentation, et y demeurent plus ou moins long-temps avant d'arriver au dernier d'entr'eux, la putréfaction, et d'être entièrement décomposés, c'est-à-dire, réduits à l'état d'humus; état qu'on ne doit pas envisager comme permanent et inaltérable, mais seulement comme ayant de la durée.
Les corps animaux, au contraire, franchissent les premiers degrés de fermentation, ou du moins, ils passent si promptement au\setcounter{page}{229} travers, qu'à peine ces degrés y sont-ils perceptibles. Ces corps tombent immédiatement en putréfaction, et ils y entraînent les végétaux lorsqu'ils sont en contact avec eux. Cette putréfaction et le produit qui en résulte, éprouvent également des modifications variées suivant le degré de force de ces combinaisons, ou suivant l'intensité de l'action que la chaleur, l'humidité et l'air exercent sur elles.
A l'air libre sans humidité et sans augmentation de chaleur, la fermentation et la putréfaction ne peuvent pas être perceptibles; cependant, il en résulte une décomposition semblable à une lente combustion. Cette décomposition produit un résidu différent de celui de la putréfaction, et ordinairement, moins considérable, parce que la plus grande partie du carbone se combine avec l'oxygène, et s'évapore sous la forme d'acide carbonique.
La plus grande promptitude de la décomposition qui s'opère dans les corps animaux provient sans aucun doute, de ce que la nature de ces corps est plus compliquée; de ce qu'ils sont composés d'une variété infinie de substances, et parmi elles des nombreuses préparations végétales, qui servent à la nourriture des êtres animés. Le produit de cette putréfaction est différent, il a une ac\setcounter{page}{230} tion plus efficace sur les plantes, parce qu'il opère non-seulement comme aliment, mais encore comme stimulant; de là il suit qu'il est plus promptement et plus facilement consommé et épuisé. Aussi le fumier animal est-il plus actif, mais beaucoup moins durable. On diroit qu'il dépasse ce degré de décomposition dans lequel il peut fournir aux plantes la nourriture la plus abondante, et qu'il ne laisse après lui que ce résidu de la décomposition dont nous avons parlé précédemment.
Tous les corps animaux qui se putréfient sont convertis en engrais, et les engrais de ce genre sont les plus actifs de tous. Ces corps peuvent tous être employés à cet usage; mais le plus souvent nous y destinons les excrémens des animaux, parce que nous les avons en plus grande quantité, et que nous pouvons nous les procurer d'une manière moins coûteuse. Nous trouvons de l'avantage à allier ces excrémens avec des dépouilles de végétaux; par ce moyen, celles-ci sont disposées à une putréfaction plus rapide, et en changeant de forme, elles perdent moins, tandis que la fermentation d'ailleurs trop prompte des animaux est un peu retardée. On appelle ces engrais naturels, par opposition à d'autres que l'on nomme artificiels, non que le premier exige moins\setcounter{page}{231} d'art, mais parce qu'il est d'un usage plus général, et même le seul qui soit connu et employé par beaucoup d'agriculteurs.
Les excrémens des animaux renferment non-seulement le résidu des alimens et de la partie des filamens qui n'a pu être décomposée, mais aussi des molécules du corps des animaux, déposées dans les intestins, par conséquent de substances entièrement animalisées; de sorte que même chez les animaux qui ne se nourrissent que de végétaux, ces engrais participent plus de la nature animale que de la nature végétale; et il en est ainsi chez tous. Cependant la manière dont les animaux sont nourris, et leur état d'embonpoint, apportent à cela de grandes différences. Si l'on se borne à remplir l'estomac des bestiaux avec une nourriture qui contienne peu de sucs nutritifs, et une grande portion de fibres d'une décomposition difficile, telle que la paille sans herbe et sans grain; cette matière sort par le canal des intestins presque semblable à ce qu'elle était avant d'être donnée à ces animaux; et elle est d'autant moins animalisée, que le corps amaigri du bétail ne se dépouille plus que d'une très-petite quantité de ses parties. A la vérité cette petite quantité suffit pour donner à la paille, qui a passé au travers du corps animal, une tendance plus forte à la\setcounter{page}{232} putréfaction. Mais chez les animaux qui, au moyen de fourrages nourrissans pleins d'amidon, de gluten, d'albumine, de mucilage et de principes sucrés ont été mis dans un état d'embonpoint, et dont il se détache une plus grande quantité de molécules animales, (parce que ces molécules s'y reproduisent chaque jour) chez ces animaux, dis-je, les excrémens forment un fumier infiniment plus actif, qui contient une moins grande proportion de parties végétales et fibreuses. De là vient la différence frappante, qui existe entre le fumier produit par le bétail de toute espèce, qui est à l'engrais ; et celui qui provient de bêtes maigres et mal nourries. On peut allier au premier une quantité de litière proportionnellement beaucoup plus grande, sans empêcher et sans retarder cette fermentation uniforme, qui conduit à la putréfaction.
L'urine contient, outre une substance qui lui est propre, diverses autres substances très-actives, et divers phosphates, mais sur-tout, de l'ammoniaque. On a fait évaporer de l'urine, et les sels qu'on en a tiré, en petite quantité, ont été reconnus très-favorables à la végétation. Mais le Dr. Belcher dans ses Communications to the board of agriculture, a observé que les plantes peuvent facilement être\setcounter{page}{233} ètre trop stimulées, et même détruites par l'action de ces sels; il attribue ce dernier effet en partie à un petit insecte jaune qu'on rencontre souvent dans l'urine. Un grand nombre d'expériences semblent démontrer que les diverses substances contenues dans l'urine ne sont jamais plus efficaces que lorsqu'on les mélange avec les gros excréments, et qu'on les fait recueillir par des matières propres à cet usage; parce que ces substances contribuent beaucoup à la parfaite décomposition des unes et des autres, et à produire de nouvelles combinaisons.
Ainsi donc le fumier ordinaire est composé de ces deux espèces d'excréments et des substances végétales employées comme litière, c'est-à-dire, le plus souvent de paille; nous désignons communément cette composition sous le nom de fumier d'étable. Considérons d'abord cette espèce d'engrais.
- Le fumier d'étable a des qualités différentes, suivant l'espèce d'animaux qui l'a produit, alors même que ces animaux ont été nourris de la même manière. Quelques-unes de ces espèces de fumier seulement ont été décomposées, et analysées, d'une manière précise. Nous avons décomposé le fumier du bétail à cornes. Einhof l'a fait de même; cependant, il faudra des analyses encore plus Agricult. Vol. 28, No. 6. Juin 1813.\setcounter{page}{234} exactes, et faites sous l’appareil pneumatique, pour que nous puissions établir un parallèle positif entre les diverses espèces de fumier et leurs parties constituantes. Aussi, entre les divers phénomènes que présentent les fumiers d’étable, nous ne consignerons ici que ceux qui tombent sous les sens, et par lesquels ils se distinguent les uns des autres.
Lorsque le fumier de cheval est suffisamment humide, et qu’il est en contact avec un air modéré, il entre promptement en fermentation; il s’y développe alors une chaleur si forte qu’elle en chasse l’humidité, et avec elle, les substances volatiles; de sorte que si on ne l’arrose, il ne prend point la forme d’une bouillie épaisse, mais qu’au contraire, s’il est serré, il devient friable et pulvérulent, puis se consume au point de ne laisser que des cendres pour résidu; et si ses parties sont assez peu réunies pour que l’air puisse y pénétrer, il se décompose d’une manière inégale, se charbonne en partie comme de la tourbe, et prend beaucoup de moisi, ce qui diminue considérablement ses qualités pour l’amendement du sol, ainsi que l’expérience nous l’enseigne. Ces propriétés le caractérisent à un plus haut degré lorsqu’il provient de bêtes vigoureuses, et qui consomment beaucoup de grain; que\setcounter{page}{235} lorsqu'il a été produit par des bêtes, nourries seulement avec de l'herbe, du foin et de la paille; dans ce dernier cas, cependant, ces propriétés sont encore sensibles. Si ce fumier est transporté sur le sol avant que sa décomposition soit accomplie, il produit un effet très-prompt, et il active fortement la végétation des plantes; cet effet doit être attribué en partie à la chaleur qui s'y développe de nouveau, lorsqu'après avoir été enterré, il achève sa décomposition. Cette circonstance fait qu'il agit très-avantageusement sur les terrains humides, froids et glaiseux dont il corrige les défauts, tandis que le sol lui-même modère l'action excessive du fumier; en revanche il produit souvent de très-mauvais effets sur les terrains secs, chauds, sablonneux ou calcaires; il accélère et stimule trop la végétation des plantes dans les premiers périodes de leur développement, de manière que lorsque l'action du fumier cesse, la végétation devient faible, et languissante. Ses effets sont également peu durables, parce qu'il se consume lui-même dans la vivacité de sa fermentation, ensorte qu'il ne laisse qu'un chétif résidu. C'est dans les terrains humides et tenaces seulement, que cet inconvénient n'a pas lieu. Ce fumier produit des effets excellents.\setcounter{page}{236} dans les terrains qui contiennent une abondante quantité d'humus insoluble, parce qu'à l'aide de l'ammoniaque, il favorise d'une manière frappante, la décomposition de cet humus. S'il a achevé la fermentation qui s'opère avec dégagement de chaleur, il laisse à la vérité dans tous les sols auxquels on l'incorpore, un résidu très-favorable à la végétation et très-soluble; mais ce résidu ne forme qu'un fort petit volume.
Lorsqu'on veut l'employer seul, on le transporte sur des terrains glaiseux et humides, aussitôt qu'il a commencé sa première fermentation, ce qui ne tarde pas à avoir lieu, et on l'enterre. Il améliore même mécaniquement, et il ameublit le sol par sa fermentation continue et par sa chaleur; enfin, aumoyen des labours réitérés qui le combinent avec ce sol, il contribue essentiellement au succès des récoltes qu'on en retire. Mais si on veut l'employer sur des terrains chauds et légers, la manière la plus avantgeuse d'en tirer parti, c'est sans contredit, de le mêlanger avec des substances végétales qui aient conservé leurs sucs, ou avec de la terre et sur-tout, des gazons. Pour cet effet, on mêle le fumier avec ces diverses substances, ou bien on le met en couches successives pour en former des tas, en ayant soin de le préserver du trop libre accès de l'air, et\setcounter{page}{237} de lui donner de l'humidité lorsque la température est trop sèche. De cette manière, on obtient un compost très-actif, d'un effet durable, et très-avantageux aux terrains légers.
Le fumier d'étable, produit par le bétail à cornes, ne tarde pas également à entrer en fermentation, lorsqu'il est réuni et serré et qu'il n'a que son humidité propre; mais cette fermentation est moins accélérée, et dégage moins de calorique, ce qui fait que l'humidité de cette espèce de fumier s'en évapore moins, et qu'ordinairement, il n'est pas nécessaire de l'arroser. Il ne se réduit donc point en poussière; il prend plutôt la forme d'une bouillie qui a beaucoup de consistance. Aussi long-temps qu'il demeure réuni en tas, il ne se pulvérise point; et lorsque son humidité est complétement évaporée, il a l'apparence de la tourbe, et presque du charbon. Sa pesanteur spécifique est plus grande que celle de l'eau, tant lorsqu'il est récent, pourvu qu'il ne soit pas mêlé avec de la paille, que lorsqu'il est décomposé, et que les tuyaux de la paille, sont convertis en filamens. Il produit sur les terres un effet moins prompt, mais plus durable; il s'applique à des récoltes plus nombreuses et plus variées. Lorsque ce fumier n'a pas été extrêmement divisé, on le retrouve dans la terre sous une\setcounter{page}{238} forme tourbeuse, et en morceaux plus ou moins gros, deux ou trois ans après l'avoir employé. Quel que soit le degré de fermentation auquel il est parvenu lorsqu'on l'incorpore dans le sol, il ne paroît pas occasionner dans celui-ci une chaleur très-sensible. C'est par cette raison qu'il convient si fort, et pour ainsi dire, exclusivement, aux terrains chauds. On a coutume de dire qu'il rafraîchit ces derniers terrains : on devroit dire plutôt qu'il ne les échauffe pas. Sur des sols tenaces et glaiseux, il peut paroître inefficace lorsqu'il est enterré sous la couche végétale et qu'il n'est pas mis en contact avec l'atmosphère par des labours réitérés. Lorsqu'on l'enterre récent, les tuyaux de la paille lui conservent une sorte de communication avec l'atmosphère qui semble faciliter sa décomposition. La paille qui n'est pas brisée et qui a conservé ses tuyaux, produit aussi un effet avantageux sur ces terrains. Le fumier de bergerie se décompose promptement, lorsqu'il est compacte et qu'il conserve sa propre humidité; mais sa décomposition est difficile et lente, lorsqu'il n'est pas serré et que son humidité peut s'écouler. Dans le sol, il paroît toujours se dissiper promptement, parce qu'il produit son effet d'une manière rapide et avec force. Lorsqu'on a fumé abondamment, il donne\setcounter{page}{239} souvent trop de vigueur à la première récolte; aussi on ne doit l'employer sur les terres qu'en moins grand volume et en moins grands poids. Le plus souvent après deux récoltes, il cesse de produire son effet.
Il se dégage des excrémens, sur-tout de l'urine des moutons, beaucoup d'ammoniaque; cette circonstance fait que le fumier des bêtes à laine est d'un emploi très avantageux, particulièrement sur les terrains qui contiennent de l'humus insoluble.
Le fumier qu'on tire des bergeries est ordinairement de deux sortes. Celui de la couche supérieure est pailleux, sec et non décomposé; celui de la couche inférieure au contraire, consommé, humide et adhérent. Si l'on n'a pas eu auparavant soin de le remuer pour en faire une masse homogène, on commet une grande faute de le répandre sans distinction sur un même champ. Le fumier pailleux ne produit que de mauvais effets sur des hauteurs chaudes et sèches; mais il est d'autant plus avantageux dans les terrains humides, et un peu acide. Sur les terres de cette dernière espèce, on peut employer le fumier pailleux en grande abondance et sans inconvéniens: quant au fumier décomposé, il convient de le répandre en petite quantité sur toute espèce de terrain, parce que sans cela il fait verser les blés.\setcounter{page}{240} - Nous parlerons ailleurs du parc des bêtes à laine.
Les opinions sont partagées sur les qualités du fumier de cochon. Sa valeur dépend beaucoup de la manière dont ces animaux sont nourris. Il dépend aussi de la manière dont ils sont pourvus de litière, et du soin qu'on a de ne pas laisser perdre les urines: lorsque ce fumier est fait avec soin, il est très-actif.
On fait fort peu de fumier de volaille; mais en revanche il est d'une extrême activité. Il paroît par l'analyse que Vauquelin a faite de ce fumier, qu'il contient une substance particulière, composée en grande partie d'alumine. L'effet du fumier de volaille n'est jamais plus grand que lorsqu'on le répand en poudre sur la surface du sol, sans l'enterrer.
( La suite à un prochain Cahier. )
"LA notice que nous avons donnée dans notre dernier Numéro sur l'agriculture Bolognoise, est de Mr. DAVID BOURGEOIS , de Neuchatel."
\setcounter{page}{293} rapide pour toute autre production du fien, à l'étendue d'environ un acre et demi. J'ai préparé du terrain, en grande partie tourbeux, pour faire une plantation au mois de septembre prochain, de l'étendue de huit acres. Je n'ai pas de raison de craindre qu'il ne réussisse pas aussi bien que le premier; et s'il réussit en effet, ce sera une grande source de richesse pour ce pays-ci.
JOHN BEARD.
PRINCIPES RAISONNÉS D'AGRICULTURE. Traduit de l'allemand d'A. THAER, par E. V. B. CRUD. Tom. Ier. in-4°, 372 p. Genève, 1811, chez J. J. Paschoud, Imprimeur-Libraire; et à Paris chez le même rue Mazarine, n°. 22. (Vingt-unième extrait. Voy. p. 225.)
LES matières fécales font un engrais très-actif; et lorsqu'on sait les employer, elles font plus d'effet qu'aucun autre engrais. L'auteur estime que si les excrémens humains étoient recueillis soigneusement, et qu'on en tirât tout le parti dont ils sont susceptibles, on obtiendroit par la plus grande produc-\setcounter{page}{294} \section{SUR LA CULTURE DU FIORIN dans le comté de Lanark, en Écosse. Aux mines de fer de Shotts près de Whitburn (Farmer's Magazine). Février 1813.}
MR.
La plupart de vos lecteurs pourront s'intéresser à ce que je vais dire sur la culture du fiorin. La réputation de cette herbe excellente gagne rapidement parmi les agriculteurs; et l'utilité de cette culture en grand dans les tourbes est extrêmement intéressante; surtout lorsqu'il s'agit de cantons comme celui-ci, qui sont situés à huit cents pieds au-dessus de la mer, et où il est bien difficile de faire réussir le ray-grass et le trèfle.
Avant le mois de mai dernier, j'avois lu beaucoup de détails sur cette plante merveilleuse, et j'en avois beaucoup entendu parler; mais je regardois la chose comme peu digne d'une attention sérieuse; j'eus alors le bonheur de rencontrer le Dr. Richardson, le protecteur du fiorin, et qui est bien connu dans le monde agricole. Je lui entendis faire
Z 3\setcounter{page}{295} l'histoire de cette herbe, de manière à me convertir tout-à-fait. Le 12 de mai, je l'engageai à venir voir ma ferme, avec le général Sir James Stweart de Coltings, auquel on doit beaucoup pour son zèle à propager la culture du fiorin. Le 13 mai 1812 le docteur eut la complaisance de donner lui-même l'exemple à mes gens en commençant ma plantation de fiorin. J'avois alors un champ où il y avoit eu des pommes de terre l'année précédente, et où je comptois mettre de l'orge et des graines de pré. Je priai le docteur de choisir la portion de ce champ qu'il jugeroit convenable pour la mettre en fiorin. Il me répondit que la saison la plus avantageuse pour cette plantation étoit déjà passée; et qu'en conséquence il ne feroit l'essai que fort en petit, et seulement pour s'assurer si cette plante pourroit réussir en faisant la plantation aussi tard. Il prit une partie du champ qui étoit un peu tourbeuse. Je ne m'arrêterai pas à vous décrire la manière de planter les stolons ou marcottes; cela a souvent été dit par des gens qui le savoiement mieux que je ne le fais. Il suffit de dire que la plantation fut d'environ un acre anglais; et que deux ouvriers achevèrent cet espace avant la fin de mai. Je couvris ensuite ma plantation avec des\setcounter{page}{296} cendres du charbon de pierre de nos fonderies, et avec de la chaux; en marquant les espaces amendés par chaque engrais, pour juger de la différence de l'effet. La partie la plus belle a été celle que j'avois couverte de cendres; et le hasard a fait que c'étoit aussi la portion la plus tourbeuse de la pièce. J'ai coupé cette portion à la fin de novembre et au commencement de décembre. L'herbe fraîchement fauchée, et par un beau jour, pesoit à raison de vingt milliers et deux cents livres par acre anglais. Cette même quantité, réduite par une dessiccation suffisante, pour mettre le foin en meule, pesoit sept milliers par acre anglais. Il me paroît que j'ai trop séché mon fiorin; car il est plus nourrissant pour le bétail lorsqu'on ne le sèche pas trop, et il se garde également bien, pourvu que les meules ne soient pas trop fortes. J'ai trouvé que mes chevaux et mes bêtes à cornes aimoient beaucoup ce fourrage. Mes chevaux surtout le mangeoient avec avidité; ni eux, ni les bêtes à cornes, n'ont hésité à laisser de côté le meilleur ray-grass pour le fiorin. Je dois remarquer que mon ray-grass étoit de la première qualité, et acheté dans un canton plus chaud que celui de ma ferme. Il y a un mois que j'ai planté sur une pente trop
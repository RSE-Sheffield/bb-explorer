\setcounter{page}{411}
\section{ELEMENTS OF AGRICULTURAL CHEMISTRY, etc. Elémens de chimie-agricole en un cours de lecons pour le Départment d'Agriculture; par Sir Humphry Davy. Londres 1813 \large{(Troisiéme extrait. Voy. p. 371.)}}
En continuant la troisième leçon, l'auteur énumère et décrit les substances composées que fournissent les végétaux, savoir, les gommes ou les mucilages, l'amidon, le sucre, l'albumine, le gluten, la gomme élastique, l'extrait, le tannin, l'indigo, le principe narcotique, le principe amer, la cire, les résines, le camphre, les huiles fixes, les huiles volatiles, la fibre ligneuse, les acides, les alkalis, les terres, les oxides métalliques, les composés salins.
Après avoir indiqué les propriétés caractéristiques, et les applications connues de chacune de ces substances, l'auteur donne les procédés utiles à l'agriculteur pour la décomposition des substances végétales. Il rap-\setcounter{page}{412} pelle, d'après les belles expériences de Mr. Th. De Saussure et de MM. Gay-Lussac et Thénard, que les premiers élémens des substances végétales sont principalement l'hydrogène, le carbone et l'oxigène, en diverses proportions, et quelquefois combinés avec l'azote. Il fait connoître, sans les admettre tout-à-fait, les conclusions que les chimistes de Paris ont tirées de leurs expériences, et qu'ils appellent lois: savoir, 1°. qu'une substance végétale est toujours acide, quand l'oxigène s'y trouve en plus grande proportion relativement à l'hydrogène, que dans l'eau; 2°. qu'une substance végétale est toujours résineuse, huileuse, ou spiritueuse, lorsqu'elle contient l'oxigène en moindre proportion, relativement à l'hydrogène, qu'il n'est dans l'eau; 3°. qu'une substance végétale n'est ni acide ni résineuse, mais sucrée ou mucilagineuse, ou analogue à la fibre ligneuse ou à l'amidon, lorsque l'oxigène et l'hydrogène y sont dans la même proportion que dans l'eau.
L'auteur mentionne la belle découverte de Mr. Kirchoff de la conversion de l'amidon en sucre, par une simple ébullition de quarante heures dans l'acide sulfurique étendu d'eau, et il cite une expérience récente du Dr. Tuthill, qui a tiré une livre et un quart de sucre brun cristallisé, d'une\setcounter{page}{413} livre et demi d'amidon de pommes de terre: ce sucre tient le milieu entre celui de cannes et celui de raisins. L'auteur décrit les propriétés, et énumère les applications de ce produit de la fermentation vineuse qu'on nomme alcool, ou esprit-de-vin, puis de l'éther, qui est le produit de son mélange à l'acide sulfurique, et du procédé de la distillation. Il s'arrête sur les changemens qui résultent, dans les principes végétaux, de la séparation de l'oxygène et de l'hydrogène qui sont combinés dans l'eau, et des combinaisons nouvelles de ces substances élémentaires. La fabrication du pain est l'occasion d'une de ces métamorphoses de l'eau les plus intéressantes pour l'homme. Dans le pain de froment, plus d'un quart de l'eau employée, se fixe dans une nouvelle combinaison, et devient aliment solide. Le pain d'orge et le pain d'avoine décomposent et fixent encore une plus forte proportion de l'eau employée, mais la quantité plus grande de gluten que donne le blé, laisse l'avantage au pain de froment. Il a comparé, sous le rapport de la proportion du gluten, les blés de diverses contrées, et a trouvé que ceux des pays méridionaux en fournissaient le plus. Il donne les résultats de l'analyse des farines de seigle,\setcounter{page}{414} d'orge, de fèves, de lentilles, de l'analyse de diverses espèces de pommes de terre, et enfin un tableau très-instructif des quantités de matières solubles et nutritives contenues dans mille parties de trente-sept substances végétales différentes, employées principalement pour la nourriture de l'homme et des animaux. Nous pourrons revenir à ces objets très-curieux et très-intéressans dans leur applications usuelles, mais qui sont plus du ressort du chimiste que de l'agriculteur; et nous passons à la quatrième leçon, que nous donnerons à-peu-près en entier.
\section{Sur la nature des différens terrains; leurs parties constituantes; analyse des différens terrains; leur emploi. Des rochers et des couches que l'on trouve en creusant. De l'amélioration des différens terrains.}
La connaissance des différens terrains et des moyens de les améliorer est un objet de la première importance pour le fermier. La science de la chimie peut éclairer cette partie de l'agriculture mieux encore que toute autre.
Les terrains sont extrêmement variés dans leurs apparences et dans leurs qualités; cependant, comme nous l'avons dit dans la première leçon, cette diversité est due uniquement aux différentes proportions des mê-\setcounter{page}{415} mes élémens qui se combinent, ou à un mélange purement mécanique.
Nous avons déjà nommé les substances qui constituent les terrains. Ce sont de certains composés de silice, de chaux, d'alumine, de magnésie, d'oxide de fer, de manganèse, de matières animales et végétales en état de décomposition \footnote{Thaër a nommé humus ce que l'auteur désigne sous le nom de matières animales et végétales en décomposition.}(R), et d'une combinaison d'acides et d'alkalis.
Dans toutes les expériences chimiques relativement à l'agriculture, faites sur divers terrains, les parties constituantes obtenues ont été des composés; et c'est ainsi qu'elles agissent dans la nature. C'est donc dans cet état que je décrirai leurs propriétés caractéristiques.
1. La silice ou la terre du silex, dans son état de pureté et de cristallisation, est la substance connue sous le nom de cristal de roche. Dans l'état où le chimiste l'obtient, c'est une poudre blanche et impalpable. Elle n'est pas soluble dans les acides ordinaires, mais elle se dissout par la chaleur dans les lessives alkalines. Cette substance est incombustible, parce qu'elle est saturée d'oxygène. J'ai déjà prouvé que la silice étoit composée d'oxygène, et d'un corps\setcounter{page}{416} combustible que j'ai nommé silicium. D'après les expériences de Berzelius, il paroît que ces deux corps y sont contenus en quantité à-peu-près égales.
2. Les propriétés sensibles de la chaux sont bien connues. Elle est ordinairement unie à l'acide carbonique dans le sol. On les sépare facilement au moyen des acides ordinaires. On la trouve quelquefois combinée avec les acides phosphorique, et sulfurique. Ses propriétés chimiques, et les diverses manières dont elle agit; seront expliquées dans la leçon sur les engrais minéraux. La chaux est soluble dans le nitre et dans l'acide muriatique; lorsqu'elle est unie à l'acide sulfurique, elle forme une substance d'une solution difficile, qu'on appelle gypse. Le gypse se compose d'une proportion 40 de la substance métallique, que j'ai nommée calcium et d'une proportion 15 d'oxigène\footnote{Nous donnerons ensuite la clef de cette théorie des nombres, qui appartient au chimiste Dalton. (R)}.
3. L'alumine existe pure et cristallisée dans le saphir blanc: elle se trouve dans les autres pierres précieuses, unie à un peu d'oxide de fer et de silique. Dans l'état où les chimistes l'obtiennent; elle a l'apparence d'une poudre blanche soluble dans les acides et les liqueurs alkalines. D'après mon ex-\setcounter{page}{417} périence, il paroît que l'alumine est formée d'une proportion 33 d'alumine, et d'une proportion 15 d'oxigène.
4. La magnésie existe pure et cristallisée sous une forme semblable au talc; on la trouve dans l'Amérique septentrionale. Dans son état ordinaire, c'est la magnesia usta, ou magnésie calcinée des droguistes. Elle se trouve généralement dans les terrains mélangés d'acide carbonique. Elle est soluble dans tous les acides minéraux, mais non dans les lessives alkalines. On la distingue des autres terres qui se trouvent dans le sol, par la facilité avec laquelle elle se dissout dans les solutions de carbonates alkalins, saturées d'acide carbonique. Sa composition est, à ce qu'il paroît, 38 magnésium, et 15 oxigène.
5. Il y a deux acides de fer bien connus: le noir et le brun. L'oxide noir est la substance qui s'échappe en éclats sous le marteau du forgeron. On obtient l'oxide brun, en maintenant long-temps l'oxide noir exposé au contact de l'air, chauffé à rouge. Le premier paroît consister en une proportion 103 de fer, et deux proportions 30 d'oxigène; et le second est composé d'une proportion 103 de fer, et de 3 d'oxigène 45. Les oxides de fer se trouvent quelquefois dans le sol, combinés\setcounter{page}{418} avec l'acide carbonique. On les distingue facilement par leur propriété de donner une couleur noire aux solutions de noix de Galle, et une belle couleur bleue au précipité de prussiate de potasse et de fer.
6. L'oxide de maganesum est la substance communément appelée manganèse, et dont on se sert pour le blanchiment. Il paraît composé d'une proportion de manganesum 113, et de trois d'oxigène 45. Il se distingue des autres substances que l'on trouve dans le sol, par la propriété de décomposer l'acide muriatique, et de le convertir en chlorine.
7. Les matières animales et végétales se reconnaissent par leurs qualités sensibles, et ont la propriété de se décomposer par la chaleur. Leurs caractères peuvent être connus par celle-ci.
8. Les composés salins que l'on trouve dans le sol, sont: le sel commun, le sulfate de magnésie, le sulfate de fer, le nitrate de chaux, le nitrate de magnésie, le sulfate de potasse, les carbonates de potasse et de soude. Il est inutile de décrire leurs caractères en détail: nous avons déjà vu comment l'on s'assure de la présence de la plupart de ces composés salins.
La silice se trouve ordinairement combi\setcounter{page}{419} née dans le sol avec l’alumine et l’oxide de fer, ou bien avec l’alumine, la chaux, la magnésie, et l’oxide de fer; formant avec eux du gravier et du sable de divers degrés de finesse. Le carbonate de chaux est ordinairement en poudre impalpable, mais quelquefois en état de sable calcaire. La magnésie, si elle n’est pas combinée avec le gravier et le sable, est en état de poudre fine, et unie à l’acide carbonique. La partie du sol qui est en poudre impalpable, et qu’on appelle ordinairement glaise, ou lut (Clay or Loam) est composée de silice, d’alumine, de chaux, et de magnésie. Dans le fait, elle est ordinairement composée de même que le sable, mais plus divisée. Les matières végétales ou animales; (et les premières sont de beaucoup les plus communes dans le sol) se trouvent en différens degrés de décomposition. Quelquefois les fibres sont encore apparentes, et quelquefois elles sont absolument brisées et mélangées avec le sol.
Pour prendre une juste idée de la composition d’un sol, il faut se représenter différentes pierres ou rochers, décomposés ou réduits en poudre de divers degrés de finesse; quelques-unes de leurs parties solubles dissoutes dans l’eau, et cette eau adhérente à la masse; enfin le tout mélangé de débris\setcounter{page}{420} végétaux et de matières animales, en différens degrés de décomposition.
Il est convenable de décrire les différens procédés par lesquels on peut analyser toutes les variétés de sol. Je suis obligé d'entrer ici dans beaucoup de détails, et ils pourront paroître minutieux; mais l'agronome qui a des connoissances en chimie en sentira l'importance.
Les instrumens nécessaires à l'analyse des divers sols sont en petit nombre, et peu dispendieux. Il faut une balance qui puisse contenir un quart de livre de la terre qu'on veut examiner, et qui soit sensible jusqu'à un grain; un assortiment de poids, depuis un quart de livre (Troy) à un grain; un tamis suffisamment grossier pour laisser passer un grain de moutarde; une lampe d'Argand; quelques bouteilles de verre; des creusets de Hesse; des bassins de porcelaine ou de terre de pipe pour l'évaporation; un mortier et pilon de Wedgewood; quelques filtres faits avec une demi feuille de papier brouillard, pliée de manière à contenir une pinte de liquide, et graissée dans les bords; un couteau d'ivoire; et un appareil pour mesurer et contenir du fluide aériforme.
Les substances chimiques nommées réactifs qu'on emploie pour séparer les parties\setcounter{page}{421} constituantes du sol, ont déjà été mentionnées pour la plupart. Ce sont, l'acide muriatique, l'acide sulfurique, l'alkali volatil caustique dissous dans l'eau; une solution de prussiate de potasse et de fer; du succinate d'ammoniaque, une solution de potasse, et des solutions de carbonate d'ammoniaque, de muriate d'ammoniaque, de nitrate d'ammoniaque, et de carbonate de potasse.
Lorsqu'il s'agit de s'assurer de la nature du terrain d'un champ, il faut en prendre des échantillons à deux ou trois pouces de profondeur, et en examiner comparativement les propriétés. Il arrive quelquefois que dans les plaines tout le sol supérieur est de même nature, et alors une analyse suffit; mais dans les vallées et dans le voisinage des rivières, les différences sont très-grandes: souvent une partie d'un champ est calcaire, et l'autre est siliceuse. Il faut, dans ce cas-là, prendre et examiner à part chaque espèce.
Lorsqu'on ne peut examiner immédiatement les échantillons des terrains qu'on veut analyser, on peut les conserver sans leur faire éprouver aucun changement, en les renfermant dans des fioles, qu'on a soin de remplir tout-à-fait, et qu'on ferme avec des bouchons de verre.\setcounter{page}{422} Le poids le plus convenable de l'échantillon qu'on veut analyser, est de deux cents à quatre cents grains. Il faut le prendre par un temps sec, et l'exposer à l'air jusqu'à-ce qu'on n'aperçoive plus d'humidité au toucher.
On peut constater la gravité spécifique d'un sol, c'est-à-dire, son rapport avec la pesanteur de l'eau, en introduisant dans une phiole, qui peut contenir un poids connu d'eau, un mélange à volumes égaux d'eau et du terrain à analyser. Ce mélange à volumes égaux, peut aisément se faire, en versant d'abord de l'eau pure, jusqu'à moitié de la contenance de la phiole, puis en remplissant celle-ci avec la terre à examiner. La différence de la pesanteur de cette phiole, avec ce qu'elle auroit pesé étant remplie d'eau pure, donnera le résultat cherché, c'est-à-dire, que si la phiole pèse six cents grains, par exemple, au lieu de quatre cents qu'elle auroit pesé pleine d'eau, la pesanteur spécifique du sol sera 2. Il sera deux fois plus pesant que l'eau.
Il est important de connoître la gravité spécifique d'un sol, parce qu'elle indique la quantité de matières animales et végétales que le sol contient: ces matières se trouvent toujours en plus grande abondance dans les sols les plus légers.\setcounter{page}{423} Il faut examiner les autres propriétés d'un sol, avant d'entreprendre son analyse : ces qualités indiquent jusqu'à un certain point sa composition, et servent de guides dans l'expérience. Ainsi les terrains siliceux sont communément rudes au toucher, et raient le verre, si on le frotte dessus. Les terrains ferrugineux sont rougeâtres ou jaunâtres, et les sols calcaires sont doux au toucher.
1. Les terrains aussi secs qu'ils peuvent l'être par leur contact continuel avec l'air, n'en contiennent pas moins une très-grande quantité d'eau, qui adhère avec force aux terres et aux matières végétales, et ne peut en être chassée que par un haut degré de chaleur. C'est à quoi on doit procéder, avant de commencer l'analyse, en prenant garde toutefois de ne pas affecter sa composition. Il faut pour cela faire chauffer l'échantillon du sol pendant dix ou douze minutes sur une lampe d'Argand et dans un bassin de porcelaine à une température égale de 300° F. A défaut d'un baromètre, on peut s'assurer du degré de chaleur, en mettant un morceau de bois en contact avec le fond du bassin. Tant que la couleur du bois n'est point altérée, la chaleur n'est pas trop forte, mais lorsqu'il commence à se roussir, il faut arrêter l'opération. Il restera peut-être en-\setcounter{page}{424} çore une petite quantité d'eau dans l'échantillon du sol, mais cette épreuve donnera cependant des résultats comparatifs qui seront utiles. Si au contraire, on augmentoit la chaleur au-delà du point indiqué, les matières végétales ou animales soumises à son action, se décomposeroient, et l'expérience seroit manquée.
Il faut observer avec soin quelle est la diminution de poids opérée par le desséchement, et si sur quatre cents grains, il se réduit à cinquante, on peut en conclure que le sol est absorbant au plus haut point, qu'il retient l'eau, et qu'il contient beaucoup de matières animales ou végétales, ou une grande proportion d'alumine. Si le poids ne diminue qu'en raison de 10 à 20, on en tirera les conclusions contraires, et la silice prédominera dans le sol.
2. Les pierres, le gravier, et les filamens végétaux ne doivent point être extraits du sol avant que l'eau en ait été ôtée, car ces corps eux-mêmes sont très-absorbans et retiennent l'eau; par conséquent, ils influent beaucoup sur la fertilité de la terre. Cependant cette séparation devra se faire d'abord après l'opération du desséchement. On commence par piler en poudre fine l'échantillon qui doit être analysé; on passe ensuite au\setcounter{page}{425} tamis. Il est important de peser à part les fibres des végétaux qui y sont contenus: on doit encore s'assurer également de celui des pierres, et de leur nature. Si elles sont calcaires, elles font effervescence avec les acides. Si elles sont siliceuses, elles seront assez dures pour rayer le verre; et si ce sont des pierres alumineuses, elles seront douces au toucher, faciles à couper, et ne feront point effervescence avec les acides.
3. Presque tous les terrains, contiennent, outre les pierres, une plus grande ou plus petite quantité de sable de différens degrés de finesse. C'est une opération nécessaire, et qui doit se faire avant l'analyse, de séparer le sable grossier de celui qui est dans un état de plus grande division, ainsi que de la glaise, de la marne, des substances animales et végétales, et de la matière soluble dans l'eau. On peut faire cela d'une manière suffisamment exacte, en faisant bouillir l'échantillon du sol dans trois ou quatre fois son poids d'eau. Lorsqu'ensuite l'eau est refroidie, on agite le tout, et on laisse reposer. Le sable grossier se sépare alors ordinairement en une minute; et le plus fin en deux ou trois minutes; tandis que les parties les plus ténues des terres, ainsi que les substances animales et\setcounter{page}{426} végétales, demeurent beaucoup plus longtemps dans un état de suspension. En décantant avec précaution, le sable sera assez exactement séparé des autres substances. L'eau qui les tient suspendues sera versée sur un filtre ; la matière solide sera rassemblée, lavée, séchée, puis pesée. Le sable sera également pesé, et les poids respectifs seront notés. L'eau sera conservée, parce qu'elle contient les matières salines, et les substances animales et végétales solubles, s'il en existe dans le sol.
Par le procédé du lavage et de la filtration, le sol se trouve séparé en deux parties; et la plus importante des deux est ordinairement celle où la matière est le plus divisée. Il est rare que l'analyse exacte du sable soit nécessaire: on peut juger de sa nature, de la même manière que pour les pierres et le gravier. C'est toujours du sable siliceux, du sable calcaire, ou un mélange de tous deux. Si c'est uniquement du carbonate de chaux, il se dissout rapidement et avec effervescence dans l'acide muriatique; mais si le sable est composé en partie de cette substance, et en partie de matières siliceuses, on peut s'assurer des quantités respectives, en pesant le résidu après l'action de l'acide: il faut augmenter la dose de ce celui-ci\setcounter{page}{427} lui-ci, jusqu'à ce que l'effervescence ait cessé, et que la liqueur ait un goût acide. Le résidu est la partie siliceuse : il faut le laver, le sécher, l'exposer à une forte chaleur dans un creuset. La différence du poids de ce résidu avec le poids de tout le sable, indique la proportion du sable calcaire.
5. La matière très-divisée dans l'échantillon du sol, est ordinairement aussi très-composée. Elle contient souvent les quatre terres primitives, et des substances animales et végétales. Le difficile est de déterminer les proportions, avec une exactitude suffisante.
Il faut commencer par exposer à l'action de l'acide muriatique cette portion très-divisée. Il faut verser cet acide en poids double de la matière terreuse, sur cette matière, dans un bassin de porcelaine; mais il faut que l'acide soit étendu dans un volume d'eau qui soit double du sien. On remuera le mélange à plusieurs reprises, et on le laissera reposer une heure et demie, avant de l'examiner.
S'il existoit dans le sol du carbonate de chaux ou de magnésie, l'acide l'aura dissous; et celui-ci dissout aussi quelquefois un peu d'oxide de fer : très-rarement de l'alumine.
Agricult. Vol. 18. Nº. 11. Nov. 1813.\setcounter{page}{428} Le liquide sera passé au filtre; la matière solide rassemblée, lavée à l'eau de pluie, séchée à une chaleur modérée, puis pesée. La perte indiquera la quantité de matière solide enlevée. Le lavage sera ajouté à la solution, laquelle sera rendue acide, si elle ne l'est pas, en y ajoutant de nouvelles doses d'acide. On mêlera ensuite avec le tout, un peu de solution de prussiate de potasse et de fer. S'il se fait un précipité bleu, cela prouve la présence de l'oxide de fer. Il faut alors ajouter de la solution de prussiate de potasse, goutte à goutte, jusqu'à-ce que cet effet ne soit plus produit. Pour s'assurer de la quantité du précipité, on le rassemblera de la même manière sur le filtre; et on le fera chauffer à rouge. Ce sera de l'oxide de fer, peut-être mêlé d'un peu d'oxide de manganesum.
On verse une solution de carbonate de potasse neutralisé dans le liquide débarrassé de l'oxide de fer, jusqu'à-ce que toute effervescence ait cessé, et que l'odeur ainsi que le goût de la liqueur, indique un excès considérable de sel alkalin.
Le précipité qui s'affonce, est du carbonate de chaux; il faut le recueillir par filtration et le sécher à une chaleur moindre que le feu rouge.\setcounter{page}{429} On fait bouillir ensuite la liqueur pendant un quart d'heure. Si elle contient de la magnésie, celle-ci se précipitera, combinée avec l'acide carbonique, et on pourra s'assurer de sa quantité, de la même manière que pour le carbonate de chaux.
Si par quelque circonstance particulière, une faible portion d'alumine est dissoute par l'acide, elle se retrouvera avec le carbonate de chaux précipité. On peut l'en séparer en la faisant bouillir pendant quelques minutes dans une quantité d'eau de savon suffisante pour recouvrir la matière solide: l'eau de savon dissout l'alumine sans agir sur le carbonate de chaux.
Lorsque le sol, bien pulvérisé, est suffisamment calcaire pour produire une forte effervescence avec les acides, il y a un moyen très-simple et suffisamment exact dans tous les cas ordinaires, pour s'assurer de la quantité de carbonate de chaux qu'il renferme.
Le carbonate de chaux, dans ses divers états, contient une proportion déterminée d'acide carbonique, c'est-à-dire, près de quarante-trois pour cent. Ainsi, lorsqu'on a la quantité de ce fluide élastique, produit pendant la solution de la matière calcaire dans un acide, soit en poids, soit en mes-\setcounter{page}{430} sure, on connoît la quantité de carbonate de chaux.
Lorsqu'on veut procéder par réduction de poids, on pèse deux portions d'acide, dans une phiole, et une portion de la matière du sol dans une autre. On les mêle ensuite peu-à-peu, jusqu'à-ce que l'effervescence cesse tout-à-fait. La différence du poids avant et après l'expérience, indique la quantité d'acide carbonique qui s'est échappé: pour quatre grains et un quart d'acide carbonique, on peut compter qu'il y avoit dix grains de carbonate de chaux. La meilleure méthode pour rassembler l'acide carbonique, de manière à en connoître le volume, est d'employer un appareil pneumatochimique, dans lequel on puisse mesurer le volume de ce gaz par la quantité d'eau qu'il dissout.
6. Après qu'on a fait agir l'acide muriatique sur les parties calcaires du sol, il faut s'occuper de déterminer la quantité de matières animales et végétales insolubles qu'il contient. On peut y réussir avec une précision suffisante, en chauffant la masse fortement dans un creuset, sur un feu ordinaire, jusqu'à-ce qu'il n'y reste rien de noir. Il faut, pendant cette opération, mêler souvent la masse avec une baguette de métal.\setcounter{page}{431} pour exposer continuellement à l'air des parties nouvelles. La perte de poids éprouvée par l'opération, indique la quantité de substance destructible par le feu et l'air que le sol contient.
Il n'est point possible, sans des expériences délicates et très-difficiles, de s'assurer si cette substance est entièrement animale ou végétale, ou si c'est un mélange de toutes deux. Si l'odeur qui s'échappe pendant l'opération ressemble à celle des plumes brûlées, c'est une indication certaine d'une substance animale, ou analogue: une flamme bleue et abondante montre toujours une proportion considérable de matières végétales. Dans les cas où il faudroit que l'expérience fût promptement achevée, on peut s'aider de nitrate d'ammoniaque, qu'on jette peu-à-peu pendant l'ignition, pour accélérer la dissipation des matières animales et végétales, lesquelles se convertissent en fluides élastiques, tandis que le nitrate d'ammoniaque se dissipe également.
7. Les substances qui restent après la destruction des matières animales et végétales sont, ordinairement, des particules terreuses d'alumine, et de silice; et de l'oxide de fer et de manganesum.
Kk 3\setcounter{page}{432} Pour séparer ces substances les unes des autres, il faut faire bouillir la masse pendant deux ou trois heures dans de l'acide sulfurique étendu de quatre fois son poids d'eau. On règle la quantité de l'acide sur celle du résidu solide, en comptant cent vingt grains d'acide, pour deux cent vingt grains de ce résidu.
La substance qui reste après l'action de l'acide, peut être considérée comme siliceuse. Il faut la séparer, et s'assurer de son poids, en lavant et séchant de la manière indiquée ci-dessus.
L'alumine et les oxydes de fer et de manganèse, s'il y en a, sont dissous dans l'acide sulfurique: on peut les séparer par le succinate d'ammoniaque, lequel précipite l'oxyde de fer, et par de l'eau de savon, qui dissout l'alumine, et non l'oxyde de manganèse. Les poids des oxydes, après qu'on les a chauffés à un feu rouge, indiquent leurs quantités dans le sol.
Si une partie de la magnésie ou de la chaux, avoit échappé à la dissolution par l'acide muriatique, on la retrouveroit dans l'acide sulfurique. Cela arrive rarement.
La méthode d'analyse par l'acide sulfurique, est suffisamment exacte pour toutes les expériences usuelles; mais si l'on veut une\setcounter{page}{433} précision très-grande, il faut employer le carbonate de potasse, pour agir sur le résidu de l'incinération. Il faut mettre ce résidu dans un creuset d'argent ou de porcelaine, avec quatre fois son poids de carbonate de potasse sec, et chauffer à rouge pendant une demi-heure. On dissout la masse obtenue dans l'acide muriatique, et on laisse évaporer la solution jusqu'à-ce qu'elle soit presque solide. On ajoutera ensuite de l'eau distillée, et on aura une solution de l'oxide de fer et des terres (excepté de la silice) en divers muriates. La silice, après le procédé ordinaire du lavage, sera chauffée à rouge, et les autres substances pourront être séparées de la même manière que des solutions muriatiques et sulfuriques. C'est là le procédé ordinairement employé par les chimistes pour analyser les pierres.
8. Si l'on soupçonne dans le sol, des matières salines, ou des substances végétales, ou animales, ou solubles, on les trouvera dans l'eau du premier lavage qui a servi à séparer le sable. Il faut évaporer cette eau jusqu'à parfaite dessiccation, dans un bassin convenable, et en faisant chauffer à un degré de chaleur moindre que l'ébullition.
Si la matière solide obtenue, est d'une couleur brune et inflammable, on peut la\setcounter{page}{434} considérer comme l'étant en partie un extrait végétal. Si, lorsqu'elle est chauffée, elle prend l'odeur des plumes brûlées, elle est au moins partiellement animale ou albumineuse. Si cette substance est blanche, cristalline et non inflammable, on peut la considérer principalement comme une matière saline, et nous avons vu ci-devant comment on peut la reconnaître.
Si l'on a lieu de croire qu'il y ait dans le sol du sulfate ou du phosphate de chaux, il faut un procédé particulier pour s'en assurer. On chauffera un poids donné, cent grains, par exemple, de la matière du sol, pendant une demi heure dans un creuset, avec mélange d'un tiers de charbon en poudre. On fera bouillir ensuite ce mélange pendant un quart d'heure dans une demi pinte d'eau. Le liquide sera passé au filtre, et exposé quelques jours à l'air libre, dans un vaisseau ouvert. S'il existe une quantité notable de sulfate de chaux (gypse)\footnote{} il se formera un précipité blanc, que l'on séchera et pèsera.
Après ce procédé, on séparera le phosphate de chaux, s'il y en a, de la manière suivante. On mettra digérer l'échantillon de sol dans une quantité d'acide muriatique plus que suffisante pour saturer les terres solubles.\setcounter{page}{435} bles. Cette solution sera évaporée, et on verra de l'eau sur la matière solide. L'eau dissoudra les muriates formés avec les terres, et ne dissoudra point le phosphate de chaux. D'autres terres et oxydes métalliques peuvent se trouver accidentellement mêlés dans le sol, mais en quantité trop petite pour avoir aucune influence sur la fertilité. Les procédés ordinaires pour l'analyse dans ces cas-là, sont fort compliqués, et ne seroient pas d'une application usuelle.
10. Lorsque l'examen du sol est achevé, il faut arranger numériquement les produits et sommer les quantités. Si leur somme approche de la quantité totale du sol, l'analyse doit être regardée comme exacte. Il faut cependant observer, que quand le phosphate ou le sulfate de chaux se trouve dans le sol après les premiers procédés, il faut faire une correction, en déduisant leur poids, du poids total de carbonate de chaux sur lequel l'acide muriatique a agi. En arrangeant les produits, il faut suivre le même ordre que dans les procédés de l'opération. Ainsi, j'ai obtenu les produits suivans de quatre cents grains d'un bon sol sablonneux, et siliceux, pris dans une houblonière auprès de Tunbrige dans le comté de Kent.\setcounter{page}{436} \comment{table}
Eau d'absorption . . . . . . . . . 19
Petites pierrés et gravier, principalement siliceux . . . . . . . . . . 53
Fibres végétales non décomposées . . . . 11
Sable fin siliceux . . . . . . . . 212
-------------------                              295
Matière très-divisée, séparée par agitation et filtration, consistant, en carbonate de chaux . . . . . . . . 19
Carbonate de magnésie . . . . . . . . 3
Matière destructible par la chaleur et principalement végétale . . . . . 15
Silice . . . . . . . . . . . . . 21
Alumine . . . . . . . . . . . . 13
Oxyde de fer . . . . . . . . . . 5
Matière soluble, principalement de sel commun et d'extrait végétal. . . 3
Gypse . . . . . . . . . . . . . 2
-------------------                                81
Somme de tous les produits . . . . 379
Perte . . . . . . . . . . . . . 21
-------------------                                   400
Dans cette opération, la perte n'a pas été plus grande qu'elle ne l'est d'ordinaire, elle dépend de l'impossibilité de recueillir la quantité totale des différens précipités, et de la présence de plus d'humidité qu'on n'en at-\setcounter{page}{437} tribue à l'eau d'absorption, et qui est perdue dans le cours de l'opération; Quand celui qui fait les analyses s'est familiarisé avec l'emploi des divers instrumens, des propriétés des réactifs; et les rapports des qualités extérieures et chimiques des terrains, il sera rarement obligé de faire toutes les opérations que nous venons de décrire. Par exemple, si le sol à examiner ne contient pas une quantité notable de matière calcaire, on peut supprimer l'emploi de l'acide muriatique. En éprouvant un sol tourbeux, on devra sur-tout faire attention à l'action du feu et de l'air; et dans l'analyse d'un sol craieux, on pourra se dispenser de faire usage de l'acide sulfurique.
Ceux qui ne sont pas exercés aux opérations chimiques ne doivent pas espérer une grande précision dans leurs résultats. Ils trouveront beaucoup de difficultés; mais en travaillant à les vaincre, ils acquéront des connaissances pratiques très-précieuses. Il n'y a rien de si instructif dans les sciences expérimentales que la recherche des erreurs. Avant de faire des analyses, il faut être bien affermi dans la connaissance de la chimie générale; mais il n'y a peut-être pas de meilleure manière d'acquérir ces connaissances, que de se livrer à des recherches originales. En\setcounter{page}{438} faisant des expériences, on est continuellement obligé d'apprendre à connaître les propriétés des substances qu'on emploie, ou sur lesquelles on veut agir. Les idées théoriques acquièrent plus de prix lorsqu'elles sont liées aux opérations pratiques, et à des projets de découvertes.
Les plantes ne peuvent vivre que là où elles trouvent une nourriture suffisante; le sol est nécessaire à leur existence, soit en leur procurant cette nourriture, soit en leur donnant un appui qui leur permette d'obéir aux lois mécaniques qui fixent leurs racines en terre, et dirigent leurs branches vers le ciel. Comme le système des racines, des branches et des feuilles, varie dans les divers végétaux, ils réussissent plus ou moins bien dans les différens sols: ainsi les plantes à racines bulbeuses demandent un sol plus léger et plus friable que les plantes à racines fibreuses; et les végétaux qui n'ont que des racines fibreuses très-courtes, demandent un sol plus ferme que ceux qui ont des racines pivotantes, ou de grandes racines latérales.
Un bon terrain à turneps; pris à Holkham en Norfolk, me donna sur neuf parties, huit de sable siliceux. La matière pulvérulente consistoit en :\setcounter{page}{439} Carbonate de chaux . . . . . . . 63
Silice . . . . . . . . . . . . . . . . . . . . . 15
Alumine . . . . . . . . . . . . . . . . . . . 11
Oxide de fer . . . . . . . . . . . . . . . . . 3
Matières végétales et salines . . . . 5
Eau . . . . . . . . . . . . . . . . . . . . . . . 3
Je trouvai à Sheffield-place, en Sussex, un sol qui est remarquable pour les beaux chênes qu'il produit, et qui étoit composé de six parties de sable sur une de matière glaiseuse et fort divisée. En examinant un échantillon de cent parties de ce sol, je trouvai :
Silice . . . . . . . . . . . . . . . . . . . . . 54
Alumine . . . . . . . . . . . . . . . . . . . 28
Carbonate de chaux . . . . . . . . . . 3
Oxide de fer . . . . . . . . . . . . . . . . . 5
Matières végétales en décomposition . 4
Eau et perte . . . . . . . . . . . . . . . . 3
Un très-bon terrain à blé dans le voisinage de Drayton en Middlesex, a donné trois parties de sable siliceux sur cinq. La matière très-divisée consistoit en :
Carbonate de chaux . . . . . . . . . . 28
Silice . . . . . . . . . . . . . . . . . . . . . 32
Alumine . . . . . . . . . . . . . . . . . . . 29
Matières animales ou végétales et eau . 11
De ces terrains, le dernier était de beaucoup\setcounter{page}{440} le plus cohérent dans sa texture, et le premier celui qui l'étoit le moins. C'est toujours la matière extrêmement divisée qui donne la ténacité et la cohérence au sol; et si cette matière contient beaucoup d'alunine, la ténacité ou cohérence en est d'autant plus grande. Une petite quantité de matière très-divisée suffit pour rendre un sol propre à produire des turneps ou de l'orge, et j'ai vu une assez belle récolte de turneps dans un terrain qui contenoit onze parties de sable sur douze. Une beaucoup plus grande proportion de sable cause une stérilité absolue. Le sol de Bag-shothead, qui ne se couvre naturellement d'aucune plante, contient moins d'un vingtième de matière très-divisée. Quatre cents parties de ce sol, soumises à un feu rouge, donnèrent trois cent quatre-vingts parties de gros sable siliceux, neuf parties de fin sable siliceux, et onze parties de matière impalpable, qui étoit un mélange de glaise ferrugineuse et de carbonate de chaux. Les matières végétales ou animales, lorsqu'elles sont très-divisées, non-seulement donnent de la cohérence au sol, mais elles le rendent plus moelleux et plus pénétrable. Cependant, il ne faut pas que ces parties se trouvent en trop forte proportion; et un terrain qui se\setcounter{page}{441} roit en entier formé de matière impalpable, serait stérile.
L'alumine pure, ou la silice pure, les carbonates de chaux et de magnésie purs, ne peuvent entretenir des plantes vigoureuses.
Un sol n'est jamais fertile si de vingt parties, il en contient dix-neuf des matières constituantes ci-dessus mentionnées.
On demandera si les terres pures, n'ont qu'une influence chimique indirecte, en leur qualité d'agens mécaniques, ou bien si elles entrent dans les plantes comme aliment. C'est une question importante, et qui n'est pas difficile à résoudre.
Nous avons vu que les terres sont un composé de métaux et d'oxigène, et que ces métaux n'ont pas été décomposés. Il n'y a donc aucune raison de croire que les terres se convertissent en matières constituantes des plantes, savoir, le carbone, l'hydrogène et l'azote.
On a essayé de faire croître des plantes dans une quantité donnée de terre. Elles n'en ont consommé qu'une portion très-petite ; et cette portion se retrouve dans les cendres, ce qui prouve que ces terres ne s'étoient pas transformées en de nouveaux produits.
Les carbonates de chaux et de magnésie\setcounter{page}{442} peuvent être décomposés, si, par la fermentation de la matière végétale, il se forme un acide plus fort que l'acide carbonique; mais on ne peut pas supposer que la chaux et la magnésie elles-mêmes, puissent être converties en d'autres substance par aucune des opérations spontanées qui ont lieu dans le sol. Dans tous les cas, les cendres des plantes contiennent une partie des terres du sol dans lequel elles ont cru; mais la quantité de terre, comme je l'ai fait voir dans le tableau des cendres produites par les différentes plantes, ne s'élève jamais à plus d'un cinq pour cent du poids de la plante consommée. Il paraît donc que la véritable utilité des terres, dans la végétation, c'est de donner de la fermeté et de la consistance à l'organisation des plantes. Nous avons vu, par exemple, que le blé, l'orge, et plusieurs graminées à tige creuse, ont un épiderme siliceux, destiné à les fortifier, et à les défendre contre les insectes et les plantes parasites. Il y a beaucoup de terrains qu'on appelle froids; et quoique cette dénomination puisse paraître l'effet d'un préjugé, elle est fondée en effet. Il y a des sols que le soleil réchauffe plus que d'autres, à circonstances égales; et il y en a qui se refroidissent plus promptement. Cette circonstance\setcounter{page}{443} tance qui est de la plus grande importance en agriculture, n'a pas été considérée d'une manière exacte. En général, les glaises blanches et humides sont réchauffées difficilement, et se refroidissent avec promptitude. Les terrains crayeux se réchauffent lentement aussi; mais comme ils sont plus secs, ils conservent plus long-temps leur chaleur, car il s'en perd moins dans l'évaporation de leur humidité. Un sol noir qui contient beaucoup de matière végétale molle, se réchauffe beaucoup par le soleil et l'air. Les terrains colorés, ceux qui contiennent beaucoup de matière charbonneuse ou ferrugineuse, acquièrent dans des circonstances semblables, une beaucoup plus haute température par les rayons du soleil, que les terrains de couleur blanchâtre. Si l'on compare entr'eux sous ce rapport, des terrains bien secs, on trouve que ceux qui se réchauffent le plus promptement par les rayons solaires, se refroidissent aussi le plus promptement; mais je me suis assuré par des expériences, que le sol sec le plus noir (celui qui contient abondamment de la matière animale et végétale, substances qui facilitent le plus le refroidissement) s'il est chauffé au même degré qu'un sol blanchâtre et humide tout comme\footnote{Agric. Vol. 18. N°. 11. Nov. 1813.}\setcounter{page}{444} posé de matière terreuse, se refroidit plus lentement que ce dernier : bien entendu que le réchauffement du sol est dans les limites de la chaleur solaire.
J'ai trouvé qu'un terrain riche, et noir, lequel contenoit près d'un quart de son poids de matière végétale, étant exposé pendant une heure aux rayons du soleil, augmentoit de chaleur de 65° à 88°. F. ; tandis qu'un sol crayeux, soumis à la même épreuve n'augmentoit de chaleur que de 65 à 69°.
Mais, ayant transporté l'un et l'autre sol à l'ombre, où la température étoit 62°, je m'assurai que le sol noir perdit dans une demi-heure 15°, tandis que le sol crayeux n'en perdit que quatre.
Un sol brun et fertile, et une glaise stérile ayant été préalablement desséchés, furent chauffés artificiellement jusqu'à 88°. Il furent ensuite exposés à une température de 57°. Dans une demi-heure, le sol brun avoit perdu neuf degrés de chaleur; et la glaise seulement six. Une portion égale de la même glaise, mais humide, fut chauffée à 88°; et ensuite exposée à une température de 55°. En moins d'un quart-d'heure, la glaise descendit à cette température. Dans les expériences ci-dessus, les échantillons de sol étoient placés dans de petites auges d'étain.\setcounter{page}{445} de deux pouces en carré, et d'un demi pouce de profondeur. La température était déterminée par un thermomètre très-sensible.
Il est évident que la chaleur du sol, surtout au printems, doit être d'une grande importance, principalement pour les jeunes plantes. Lorsque les feuilles sont complétement développées, la terre est à l'ombre, et ainsi garantie de la trop grande ardeur des rayons du soleil. La température, de la surface, lorsqu'elle est nue, et exposée au soleil, fournit au moins un indice sur la fertilité d'un terrain.
L'humidité du sol influe sur la température; et la manière dont l'eau est distribuée dans le sol ou combinée avec les matériaux qui se composent, est très-importante relativement à la nourriture des végétaux. Si l'eau adhère trop fortement aux terres, elle ne sera pas absorbée par les racines des plantes; si elle est trop abondante, elle tend à nuire aux parties fibreuses, ou à les détruire.
L'eau paroît exister dans les terres, et dans les matières animales et végétales du sol, sous deux formes différentes, savoir, chimiquement combinée, ou seulement unie par attraction de cohésion.
Si l'on verse dans une solution d'alun\setcounter{page}{446} une solution d'ammoniaque, ou de potasse pure, l'alumine se précipite, combinée avec l'eau: la poudre séchée, en l'exposant à l'air, donne ensuite plus de la moitié de son poids d'eau, par la distillation. Cette eau étoit unie par attraction chimique. L'eau que le bois ou les fibres musculaires, ou la gomme, fournissent dans la distillation à feu rouge de 212° F., est également de l'eau qui étoit chimiquement combinée dans ces substances.
Lorsque la glaise séchée à la température de l'atmosphère, est ensuite mise en contact avec l'eau, elle absorbe promptement le liquide: cela est dû à l'attraction de cohésion. Les terrains en général, et les substances animales et végétales, que l'on a séchés à une chaleur au-dessous de celle de l'eau bouillante, augmentent en poids, par l'exposition à l'air; ce qui est dû à ce que ces substances absorbent l'eau existante en vapeur dans l'atmosphère. Cette observation a lieu en vertu de l'attraction de cohésion.
L'eau chimiquement combinée dans les élémens du sol, ne peut pas être absorbée par les racines des plantes, à moins qu'il ne s'agisse de la décomposition des substances animales ou végétales; mais l'eau qui adhère seulement aux parties du sol, est continuellement employée à la végétation. Dans le\setcounter{page}{447} fait, il y a peu de mélanges de terres dans le sol qui contiennent de l'eau combinée: l'eau est chassée des terres par la plupart des substances qui se combinent avec celles-ci. Ainsi, lorsqu'on expose à l'acide carbonique une combinaison de chaux et d'eau, l'acide carbonique prend la place de l'eau. Les composés d'alumine et de silice, et les autres composés des terres, ne s'unissent pas chimiquement à l'eau; et les terrains, ainsi que nous l'avons dit, sont formés, soit par des carbonates, soit par des combinaisons de terres pures, et d'oxydes métalliques.
Lorsqu'il existe des substances salines dans le sol, elles peuvent être unies à l'eau, soit chimiquement, soit mécaniquement; mais ils sont toujours en trop petite quantité pour influer essentiellement sur les rapports du sol avec l'eau.
La faculté du sol d'absorber l'eau par attraction de cohésion, dépend beaucoup de l'état de division de ses parties : plus elles sont divisées, et plus est grande leur faculté absorbante. Les diverses parties constituantes du sol paroissent douées de divers degrés d'énergie à cet égard : les substances végétales semblent absorber l'eau avec plus de force que les substances animales; les substances animales, plus que les composés d'alumine et de silice; et les composés d'alu-\setcounter{page}{448} mine et de silice, plus que les carbonates de chaux et de magnésie. Au reste, ces différences peuvent dépendre des divers états de division, et des différences entre les surfaces exposées.
La faculté d'un sol d'absorber l'eau de l'atmosphère influe beaucoup sur sa fertilité. Quand cette faculté est énergique, la plante est fournie d'humidité dans les temps de sécheresse; et l'effet de l'évaporation pendant la journée est combattu par l'absorption des vapeurs aqueuses de l'atmosphère, attirées à l'intérieur du sol, et plus encore pendant la nuit.
Les glaises tenaces, qui absorbent beaucoup d'eau sous la forme liquide, ne sont pas les terrains qui absorbent le plus d'humidité de l'atmosphère en temps sec. Ces glaises se durcissent, ne présentent à l'air que peu de surface; et les plantes qui y croissent sont ordinairement aussi vite brûlées que sur les sables.
Les sols qui entretiennent le mieux l'humidité des plantes par l'absorption des vapeurs aqueuses de l'atmosphère, sont ceux où il y a un juste mélange de sable, de glaise bien divisée, de carbonate de chaux, de matière animale et végétale, et qui sont assez légers, assez meubles pour être aisément perméables à l'air. A cet égard, le\setcounter{page}{449} carbonate de chaux, et les matières animales et végétales sont d'un grand avantage pour le sol: ils lui donnent la faculté absorbante sans lui donner en même temps de la ténacité. Le sable, qui ôte la ténacité ne donne pas cette faculté absorbante.
J'ai comparé la faculté d'absorber l'humidité atmosphérique, d'un grand nombre de divers terrains ; et j'ai toujours trouvé que cette faculté étoit la plus grande dans les terrains les plus fertiles. Cette faculté fournit donc un moyen de juger de la fertilité d'un sol.
Mille grains d'un terrain fameux pour sa fécondité, pris à Ormiston (East-Lothian) lequel contenoit plus de moitié de son poids de matière très-divisée, sur laquelle matière il y avoit onze parties de carbonate de chaux, et neuf parties de matière végétale, ayant été chauffés à 212° F., regagnèrent ensuite dans une heure dix-huit grains en poids, dans une température de 62° et un air saturé d'humidité.
Mille grains d'un sol très-fertile, des bords de la rivière Parret, en Sommerset, éprouvés de la même manière, regagnèrent seize grains.
Mille grains d'un sol de Mercea en Essex, qui s'afferme 45 shillings l'acre, regagnèrent treize grains.
Mille grains d'un sable fin d'Essex, qui\setcounter{page}{450} s'afferme 28 shillings l'acre, regagnèrent onze grains.
Mille grains d'un gros sable, affermé 15 shillings l'acre, regagnèrent huit grains.
Enfin mille grains du sable de la plaine de Bagshot heath, regagnèrent trois grains.
\section{SUR LES NOMBRES par lesquels Mr. DAVY représente les élémens et leurs composés\footnote{Voyez dans le volume XLVI, Sci. et Arts de la Bibl. Brit., p. 38, un aperçu de la théorie de Dalton sur la composition chimique. Celle que Mr. Davy a embrassée a beaucoup d'analogie avec celle de Dalton, ou plutôt c'est la même théorie, avec quelques modifications.}}
PLUSIEURS physiciens, au nombre desquels se trouve Mr. Davy, adoptent maintenant l'opinion, que lorsque les substances chimiques se combinent pour former de nouveaux composés, elles se combinent toujours dans des proportions déterminées. En sorte que si deux corps s'unissent en proportions qui ne soient pas égales, et que l'un des corps soit en excès, cet excès est toujours dans un rapport qui peut s'exprimer
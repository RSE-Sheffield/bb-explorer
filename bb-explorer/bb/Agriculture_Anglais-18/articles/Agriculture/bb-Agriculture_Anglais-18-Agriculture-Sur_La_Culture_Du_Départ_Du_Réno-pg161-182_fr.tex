\setcounter{page}{161}
\section{MÉMOIRE SUR LA CULTURE DU DÉPARTEMENT DU RÉNO. ( Article Communiqué ).}
Le Département du Réno est assurément un de ceux du Royaume d'Italie qui offrent la culture la plus soignée; les environs de son chef-lieu, la ville de Bologne, se distinguent d'une manière toute particulière. Le voyageur qui traverse cette contrée, y est sur-tout frappé de la beauté de ces plantations d'ormes, auxquels on a marié de riches ceps de vigne, qui étendent leurs pampres en festons d'un arbre à l'autre. Cependant si ce voyageur a des connoissances en agriculture, il ne tardera pas à regretter d'y voir si peu d'herbages destinés à la nourriture du bétail; il croira que celui-ci y est ou mal nourri ou en trop petite quantité; il regrettera qu'une partie de ces terres soumises exclusivement aux assolemens de chanvre et froment, ou de maïs et froment, ne soit pas employée à produire de la luzerne ou du trèfle; il blâmera un tel système d'écoAgricult. Vol. 18. N°. 5. Mai 1813. α\setcounter{page}{162} nomie rurale, et lui reprochera une disproportion apparente entre l'épuisement du sol, suite inévitable de l'assolement, et la quantité de fourrages destinée à procurer des engrains.
Il est certain qu'on n'entretient pas dans ce Département la quantité de bétail qui lui seroit nécessaire; on doit attribuer ce vice en grande partie à l'usage qu'ont beaucoup de propriétaires de ne point entrer en participation avec le cultivateur pour les bestiaux; et à ce que ceux-ci, appartiennent exclusivement au métayer, lequel en retire seul le bénéfice et doit, en conséquence, se pourvoir du supplément en fourrages nécessaire pour l'hivernage. L'étranger ne doit cependant pas se hâter de porter un jugement de désapprobation contre l'économie rurale Bolognaise; lorsqu'il en aura une connoissance exacte, il y trouvera peu de choses à corriger, et il accusera des défauts qu'il y remarque, bien plus les cultivateurs eux-mêmes, et leurs relâchement sur les bons anciens usages, qu'un système de culture qui a fleuri dans ce pays dès les temps les plus reculés. Le système de culture bolognais, lorsqu'il est bien suivi, soutient la fertilité du sol à l'aide seulement de quelques engrais qu'on se procure au-dehors pour le chanvre.\setcounter{page}{163} Ceci s'applique particulièrement aux environs de la ville, jusques à une lieue de distance : c'est surtout cet arrondissement qu'on trouve concentrée la bonne culture bolognaise.
Là les métairies sont en général moins grandes que celles qui sont les plus éloignées de la ville : elles ont de trois à dix hectares d'étendue; il n'y en a guères de plus grandes.
Un petit pré naturel fermé de haies, environne la maison de ferme. L'écurie forme à l'ordinaire un bâtiment séparé. Les champs sont divisés régulièrement en large bandes ou planches de 80 à 150 mètres de longueur, sur 30 à 38 mètres de largeur; celles-ci sont, autant que cela est possible, disposées du nord au sud.
Au couchant de chacune de ces planches, et à un mètre et demi du bord, se trouve une plantation d'ormes, éloignés de quatre mètres l'un de l'autre; au pied de chaque arbre sont un ou deux ceps de chaque côté, ensorte qu'un arbre porte communément deux à trois ceps.
La plate-bande, large d'un mètre et demi, est labourée soigneusement. Une bande de terre qu'on a soin de laisser enherbée, la soutient en-dehors, et sert d'épaulement à un\setcounter{page}{164} fossé de six décimètres de largeur, sur trois de profondeur, qui reçoit l'excédant d'humidité du champ, et le conduit dans l'un des canaux publics établis pour l'écoulement des eaux surabondantes.
La culture de la terre a lieu dans ce pays par des métayers, qui possèdent assez ordinairement, outre les outils et instrumens d'agriculture, le bétail nécessaire à leur exploitation. Une famille composée de trois hommes travailleurs, de deux ou trois femmes, et de quelques enfans, est jugée suffisante à l'exploitation de six hectares. Le propriétaire l'engage pour une année seulement, qui commence le premier novembre. Cependant cette convention subsiste tacitement jusqu'à-ce que une des parties voulant renoncer, avise l'autre avant la fin de mai, qu'elle le rompra au 31 octobre suivant.
Le cultivateur s'engage essentiellement: A cultiver soigneusement la terre qui lui est confiée; à en labourer à la bêche toutes les pièces qui doivent porter du chanvre, à payer le tiers ou la moitié du prix des engrais que l'on achètera; à les conduire à la ferme à ses frais; à entretenir les clôtures et les fossés d'écoulement en bon état; à faucher, dans les marais, la bâche ou les roseaux que le propriétaire lui assignera, pour\setcounter{page}{165} la litière de son bétail, et à les charier gratis à la métairie ; à remettre fidèlement au propriétaire la moitié des récoltes, excepté celle des foins et pailles qui, en totalité doivent être consommés dans la métairie ; à transporter à la ville, au lieu qui lui sera indiqué, la part des productions qui appartient au propriétaire ; enfin à payer à celui-ci, tant à titre de loyer que de redevance pour la jouissance du pré, un nombre déterminé d’œufs, de poulets, de poules et de chapons, et une somme en argent qui, y compris la valeur des volailles, s’élève à 12 francs environ par chacun des hectares de terre dont la métairie est composée.
L’assolement que les propriétaires Bolonais estiment le meilleur, est chanvre et froment ; il n’est pas rare de voir des possessions de six à huit hectares et plus, dont les terres, moitié en chanvre moitié en froment alternent, ainsi d’année en année depuis plusieurs génération. De cette règle, il faut seulement excepter des bandes d’environ quatre mètres de largeur, celles occupées par l’ombre des arbres, que, pour l’année réservée au chanvre, l’on sème en trèfles sur les fromens en végétation, ou, au printemps suivant en fenugrec ou autres herbes annuelles. Comme la culture du chanvre est\setcounter{page}{166} remarquablement bonne dans cette contrée, je dois entrer à ce sujet dans quelques détails.
Le froment ayant été semé sous raies, en billons d'un mètre de largeur environ, dès que la moisson est achevée on refend ces billons par un ou deux traits de charrue, suivant qu'on emploie à ce travail une charrue à deux versoirs, ou une charrue à un seul versoir immobile, qui toutes deux sont en usage dans le pays. On forme ainsi l'ados à la place qu'occupoit la rigole, et vice versa ; si le terrain est infesté de chiendent ou d'autres plantes vivaces, on refend de nouveau quinze jours après. Ces deux demi labours faits dans le mois de juillet, sont le moyen le plus efficace et le plus économique de nettoyer les champs de mauvaises herbes.
Avant la fin d'août, on régale le champ au moyen d'une forte herse, on fume à raison environ quinze chars, à six bœufs par chaque hectare; puis on recouvre le fumier en formant de nouveau des billons de deux traits de charrue.
Quelquefois au lieu de refendre d'abord après la moisson, on se borne à charier le fumier à la fin de juillet, en quantité de moitié moindre. L'on sème des fèves à la\setcounter{page}{167} volée, et l'on recouvre en même temps les engrais et les fèves par un léger labour, qui forme les billons à l'endroit qu'occupoit auparavant la raie. Les premières pluies d'août font lever les fèves ; et pour peu que la saison soit favorable à leur végétation, celle-ci atteignent en trois mois à-peu-près la hauteur d'un mètre, et peuvent être enfoncés à la bêche dès le commencement de novembre. C'est par les terres ensemencées en fèves qu'on commence le labour préparatoire pour celles que l'on veut ensemencer en chanvre, afin d'éviter que les gelées ne gâtent la tige et la feuille des fèves, et ne diminuent l'effet de celles-ci pour l'amendement du sol.
On sème aussi quelquefois des lupins pour les enterrer comme engrais, mais on les met en terre aussitôt après la moisson, c'est-à-dire, au commencement de juillet. Cette plante n'a peut-être pas autant que les fèves, la faculté de bonifier le sol, mais elle a sur celle-ci, l'avantage de résister au froid le plus rigoureux, et de pouvoir être enfouie même à la fin de l'hiver, lorsque les circonstances ont empêché de le faire en novembre ou décembre, époque la plus convenable pour cette opération.
\section{Le labour des terres destinés au chanvre,}\setcounter{page}{168} se, fait avec des bêches, d'environ 35 centimètres de longueur, dont le manche porte une pièce de fer plantée à angle droit et sur laquelle l'ouvrier pose le pied pour l'enfoncer plus profondément en terre; par ce moyen le labour atteint d'environ quarante centimètres de profondeur. Les cultivateurs, s'arrangent assez communément de manière à être en bon nombre pour ce genre d'ouvrage. Chaque ouvrier laboure le billon sur lequel il marche, retournant adroitement la terre qu'il soulève avec sa bêche; il a soin en même temps d'enfouir les plantes destinées à l'être, de repartir également l'engrais qu'il découvre, et de bien niveler son terrain. Aucune autre manière de préparer la terre ne peut approcher de la perfection de ce labour, que le cultivateur Bolognais exécute d'une manière vraiment admirable.
Il faut environ soixante journées d'ouvriers pour labourer de cette manière un hectare. Si l'on croit que la terre n'est pas assez amenée, on répand sur le sol, avant de bêcher, des engrais animaux, tels que des rapures de cornes et de sabots de bêtes, du poil de cochon, des rognures de peaux, ou des plumes de volailles; les uns et les autres à la quantité de six à huit cents kilogrammes par hectare. L'addition de cette sorte\setcounter{page}{169} d'engrais est très-favorable au chanvre et rend sa qualité beaucoup meilleure.
Dans les grandes exploitations, on ne peut pas labourer à la bêche toutes les terres à produire du chanvre; une partie en est labourée avec la charrue, à la profondeur de trois décimètres. Dans ce cas, six ou huit ouvriers sont constamment occupés à enlever de la raie, la terre que le versoir y a laissé retomber, et à jeter cette terre sur la bande que la charrue a renversée. D'autres fois, et particulièrement lorsque le champ n'a été que peu ou point engraissé, on donne un fort labour avec la charrue; on dispose vingt-quatre bons ouvriers à égales distances, moitié de chaque côté du champ. A mesure que la charrue passe, des femmes munies de paniers, répandent sur la bande de terre renversée, des plumes ou d'autres engrais animaux, à la quantité de deux mille kilogrammes par hectare. Tous les ouvriers avec leurs bêches, approfondissent de vingt-cinq centimètres environ la raie formée par la charrue, et recouvrent avec la terre qu'ils en ont sortie, l'engrais qui vient d'être répandu; au moyen de cela, l'engrais se trouve placé à moyenne profondeur dans un excellent labour. On accomplit ainsi en un jour, le labour d'environ cinquante ares de\setcounter{page}{170} terre; au reste ce labour doit absolument avoir lieu avant l'hiver, pour que la terre vierge, que l'on tire du fond de la raie, puisse être divisée et bonifiée, par la gelée et par les influences de l'atmosphère.
Les terres étant ainsi préparées, dès les premiers beaux jours de mars, on y sème le chanvre, que l'on répand à la volée, de manière que les grains de semence soient éloignés de cinq à six centimètres les uns des autres. On préfère généralement de semer plus épais, au risque de devoir éclaircir en sarclant. Avant d'ensemencer leurs champs, les bons agriculteurs les saupoudrent avec de la fiente de volailles, ou de la poudrette d'excrémens humains, ou des gâteaux de semences huileuses réduits en poudre, ou avec telles autres choses analogues; par ce moyen la semence du chanvre se développe plus promptement et prend une vigueur de végétation qui influe sur sa réussite.
On enterre la semence avec le fossoir plat, puis on régale le champ avec des rateaux à dents de fer que l'on promène en zig-zag, en marchant à reculons. Dès que le chanvre a quatre feuilles, on le sarcle soigneusement; et quelques jours après, on le sarcle de nouveau. Si dans la suite on aperçoit encore des mauvaises herbes, on le sarcle de\setcounter{page}{171} nouveau jusqu'à-ce que la plante empêche d'y mettre le pied. On tient dans ce pays les chanvres espacés à huit centimètres environ d'une plante à l'autre ; ils acquièrent de cette manière, la hauteur de trois mètres \footnote{Quelquefois 4 à 5 mètres.}. On dit dans le pays, lorsqu'on voit un champ dont les plantes sont plus rapprochées, que ce champ a déjà été grêlé, c'est-à-dire, que son trop d'épaisseur nuit à sa réussite comme une grêle lui auroit nuit.
Dans d'autres pays, on sème très-épais pour l'avoir plus fin. Comme alors les plantes ne peuvent se développer, la récolte n'en est jamais abondante.
Pour obtenir une belle qualité, il faut que les plantes soient convenablement espacées, comme je l'ai indiqué, et en faire la récolte après que les plantes mâles ont jeté leurs poussière d'étamines, et avant que la graine se forme. Dans les environs de Bologne, un hectare produit de onze à treize cents kilogrammes de chanvre de très-belle qualité.
Ainsi cultivé, le chanvre parvient à la fin de juillet à sa maturité ; et dès les premiers jours du mois d'août, on le coupe aussi près\setcounter{page}{172} de terre que cela est possible, au moyen d'une faucille coudée vers son manche; ainsi coudée cette faucille permet à l'ouvrier de tenir la lame toujours près du sol sans courir le risque de se blesser la main. Lorsque le chanvre est scié, on l'étend sur le sol; et lorsque ses feuilles sont desséchées, on secoue les tiges qui, assorties en longueur et réunies en petits faisceaux, sont ensuite liées en faisceaux d'environ huit décimètre de circonférence. Ces faisceaux sont ensuite étêtés, puis soumis au rouissage dans des marres garnies de pieux de chêne, auxquels sont fixés horizontalement de fortes traverses du même bois, au moyen desquels on retient le chanvre sous l'eau.
Cinq ou six jours après avoir mis le chanvre au rouissage, on le visite, et lorsqu'il est à son point, on le sort de l'eau après l'avoir lavé. Chaque petit faisceau, composé d'environ vingt tiges, se dresse sur le prévoisin de l'étang. Pour cet effet on élargit sa base en forme de cône. Si le vent ne l'abat pas, au bout de deux jours, il se trouve assez sec pour être mis à couvert. Avant de le teiller, on est obligé de concasser son bois, qui tombe presque tout par cette première opération; on le passe ensuite au battoir simple, puis un bon ouvrier le repasse au\setcounter{page}{173} battoir double, et lui donne ainsi sa dernière préparation. Les cultivateurs qui n'ont pas semé du chanvre pour graine parmi le maïs (méthode recommandable) laissent sur pied dans leurs chenevières, une quantité suffisante de plantes femelles destinées à fournir la semence pour l'année suivante. Le chanvre qui provient des plantes qui ont porté de la graine se vend à 50 ou 60 p.'.g. de moins que celui qui n'a pas porté graine.
Comme les chanvres de Bologne sont fort estimés, que leur qualité dépend en grande partie de leur bonne culture, qui est l'objet le plus marquant de l'agriculture de ce Département, j'ai cru devoir en donner un détail qui pourra être utile à quelques personnes. Le sol des chenevières, qui est un lut sablonneux et sans pierres, convient particulièrement à cette plante; le chanvre réussit bien dans les sables gras, qui par leur peu de consistance, ne conviennent guères à la culture du froment. En revanche, les terres argileuses ne lui conviennent pas. On a aussi observé que la semence recueillie dans ce Département, réussit bien ailleurs, tandis que celle que l'on importe de l'étranger n'y réussit pas. Je dois exhorter ceux qui, d'après ces indications, voudroient perfectionner la culture de leurs chanvres, à ne pas\setcounter{page}{174} les semer dans des champs exposés aux orages, parce que cette plante, dont l'accroissement est extrêmement rapide, redoute beaucoup le balottement des vents; ceux-ci, sans en casser les tiges, détériorent de beaucoup la qualité de la filasse: dans le commerce, l'on n'estime guères mieux un chanvre qui a été battu par le vent, qu'un chanvre grêlé.
Dès que les champs sont débarrassés du chanvre, on leur donne un léger labour.
La culture du froment n'a rien de remarquable; on le sème à la volée, à raison d'environ cent soixante-dix litres par hectare (un peu plus dans les terres maigres) sur un simple labour, et souvent sans avoir hersé; on recouvre la semence avec la charrue, à une oreille fixe, qui, en deux traits, forme le billon de la largeur d'un mètre. On traîne sur le champ une pièce de bois suivie de quelques branchages pour abaisser un peu les billons, dont l'arrangement s'achève par des femmes, à l'aide de houes larges et plates.
Dès le mois de mars, dans les bons champs, on scie la fane du blé avec la faucille. Cette opération a lieu jusques à trois fois sur le même champ, et on ne la cesse que lorsque l'épi commence à monter dans le tuyau.\setcounter{page}{175} c'est alors le moment le plus favorable pour nettoyer les fromens des mauvaises herbes que l'on arrache à la main, et qui, avec la fane du blé que l'on a coupée précédemment, et un peu de paille que l'on y mélange, suffit pour nourrir le bétail pendant environ deux mois et demi.
Dès la fin de mai, on n'entre plus dans les fromens pour les sarcler; on nourrit alors le bétail avec les herbes annuelles semées dans les bandes près des arbres; on laboure ces bandes; et si l'on a besoin d'herbages pour l'automne, on y sème tout de suite du maïs pour cet usage.
On moissonne à la faucille; et lorsque les fromens sont à couvert, on fauche le chaume, qui, se trouvant plus ou moins mélangé d'herbes, fournit aux bœufs une nourriture d'hiver passable. On ouvre ensuite au bétail en temps sec, ceux des champs dont les vignes sont assez fortes pour n'être pas endommagées par les bêtes. Ce qui manque à leur nourriture est complété à l'étable par les herbes, comme je viens de le dire. Lorsque celles-ci commencent à manquer, ce qui a lieu communément en août, on leur fait succéder la feuille des ormes. En aucun temps, les bœufs ne sont mieux nourris que lorsqu'ils mangent de ces feuilles. L'excel\setcounter{page}{176} lente qualité de ce fourrage les soutient si bien, que dans cette saison, ils suffisent au labour des terres qui doivent être ensemencées en froment, c'est-à-dire, à celui de la moitié de la possession, au charroi de la bûche, pour litière que l'on va chercher au loin, au transport de la vendange à la ville, puis aux semailles, en un mot, à un travail journalier très-pénible.
Environ la mi-octobre, on met les bestiaux à la nourriture d'hiver, qui consiste en paille mélangée de foin, et en chaume. Le bouvier garde en réserve un char ou deux de bon foin pour en donner en guise de provende aux vaches qui allaitent, et aux bœufs dans les temps des labours d'hiver et de printemps.
Ces foins s'achètent à quelques lieues de la ville, au prix d'environ 3 francs les cent kilogrammes sur pied, tous frais de fauchage, fanage et charroi, à la charge de l'acheteur.
Une métairie ou possession de huit hectares nourrit ainsi à l'ordinaire, deux bœufs, six vaches et quelques jeunes bêtes; quelques cultivateurs achètent de plus, pour l'été, une paire de bœufs, qu'ils revendent en automne; ceux qui sont le plus à proximité de la ville se procurent du marc de raisins, qu'ils conservent en tas bien serré, et donnent\setcounter{page}{177} \section{SUR LA CULTURE DU DÉPART. DU RENO. (177)}
nent en hiver comme provende à toutes leurs bêtes à cornes indistinctement ; c'est de cette manière que sont nourris en hiver tous les bœufs des charretiers de la ville. L'agriculture bolognoise présente plusieurs traits marquans qui lui sont particuliers, savoir, d'excellens labours profonds à la bêche, pour la culture du chanvre ; l'emploi judicieux du fumier, que l'on met en terre dès le mois d'août pour le chanvre qui doit y être semé au printemps suivant ; l'emploi comme engrais de quantité de substances animales et végétales, que dans d'autres pays on néglige de recueillir ; enfin une grande épargne dans la nourriture du bétail. Quant à ce dernier article, il faut le dire, cette épargne est poussée trop loin ; quelques propriétaires l'ont remarqué et y remédient par l'établissement des luzernières, ou en faisant succéder une plus grande quantité de trêfles à la culture du froment. A une plus grande distance de la ville, on trouve des fermes de vingt à trente-cinq hectares, dans lesquelles le chanvre n'occupe que la sixième ou la huitième partie des terres : le maïs, qui remplace le chanvre pour alterner avec le froment, s'y cultive d'ordinaire en trop grande quantité et\setcounter{page}{178} appauvrit la terre, dont le rapport se trouve en conséquence beaucoup moindre que celui des terres des environs de la ville. On entretient dans ces fermes moins de vaches et plus de bœufs ; sur une ferme de vingt hectares on trouve communément six bœufs, deux vaches, quatre à six jeunes bêtes et quelques veaux.
L’introduction en grand de la culture du trèfle, permettra aux agriculteurs d’avoir plus de bétail et de remplacer des bœufs par des vaches. Ce changement aura lieu sur-tout lorsque l’usage de la charrue belge, dont ce Département doit l’introduction à Mr. Crud de Genthod, deviendra plus général. Cette excellente charrue, comparée à plusieurs reprises à celles du pays, m’a constamment donné en résultat, une épargne de force de la moitié des animaux de trait, aussi l’ai-je adoptée pour mon exploitation de la plaine, où les terres tantôt légères et tantôt argileuses, sont toutes exemptes de pierres.
On commence à adopter le système de la persetta mezzadria, ce terme signifie que le propriétaire reçoit la moitié de tout ce qui se récolte sur sa ferme et participe par égale portion avec le cultivateur au bénéfice et à la rente du bétail dont il fournit la moitié. Cet arrangement, très-favorable à l’ang-\setcounter{page}{179} mentation du bétail et de ses produits et dont la conséquence naturelle est une rotation de récoltes moins appauvrissantes, a frappé bien des propriétaires, qui peu-à-peu adoptent ce mode d'exploitation.
Le système de culture à moitié fruit (mezzadria) présente bien des avantages sur les autres manières d'exploiter les terres, soit qu'on l'envisage sous le rapport de l'intérêt du propriétaire, sous celui de l'intérêt public, ou enfin sous celui de la moralité.
Le cultivateur, qui pour fruit de ses travaux, retire la moitié des produits de la métairie, lui consacre naturellement tous ses soins, sans que le propriétaire ait besoin de le surveiller d'une manière suivie; celui-ci n'est assujetti qu'à une inspection générale; s'il a des connoissances en agriculture, il fera donner plus de soins aux engrais, proportionnera son bétail aux besoins de son fonds, tant sous le rapport des engrais, que sous celui des travaux; il fera un bon choix des semences, et les fera préparer convenablement avant qu'elles soient semées; il dépensera tant pour ces objets qu'en bonifications quelque argent, dont il sera amplement remboursé par une augmentation de produits et par l'attachement toujours croissant des métayers, conséquence naturelle de l'amélioraP 2\setcounter{page}{180} tion de leur sort; il les verra faire de leur propre mouvement des travaux qu'il n'auroit pû exiger d'eux dans le principe.
De nombreux essais faits en différens temps et en divers lieux, ont prouvé que, sous un autre mode d'exploitation, la culture du chanvre, telle qu'on la fait dans le Département du Reno, ne conviendroit pas au propriétaire. Cette belle culture, qui produit annuellement environ dix millions de kilogrammes de chanvre, et fait entrer dans ce Département au-delà de six millions de livres d'Italie ou francs de France par an, ne pouvoit donc pas se soutenir sous un autre système d'économie. Au reste, le genre de culture dont je viens de parler est encore très-favorable à l'accroissement de la population, car ici les terres les plus peuplées sont toujours celles où l'on cultive le plus de chanvre.
Une vie laborieuse, des occupations suivies qui assurent au cultivateur Bolognois avec sa subsistance, tout ce qui est nécessaire à la vie, l'ordre établi dans chaque famille, qui souvent composée de parens au troisième degré, n'en est pas moins soumise à son chef, lequel seul à l'argent en maniement, et dirige l'ensemble de l'exploitation; toutes ces circonstances concourent au\setcounter{page}{181} \section{SUR LA CULTURE DU DÉPART. DU RENO.}
maintien de l'ordre en éloignant les vices; aussi voyage-t-on dans ce Département de nuit comme de jour en toute sécurité, à l'opposé du Milanais, où la culture des terres a lieu essentiellement par des journaliers.
Je pourrois alléguer d'autres avantages de cette manière d'exploiter les terres, mais il doit suffire d'en indiquer les principaux. Je suis tellement persuadé de leur réalité, que je desirerois voir ce genre d'économie se propager de plus en plus, et prendre la place de ces baux, dans lesquels l'intérêt du propriétaire se trouve sans cesse en lutte avec celui du fermier.\setcounter{page}{182} seule possession; souvent au contraire il est composé de 12, 15, même 20 métairies. Pour diriger un tel fonds, le propriétaire prend à ses gages un facteur, et quelquefois même un sousfacteur, chargés d'inspecter la culture, de percevoir la moitié des produits, et de faire les paiemens nécessaires. Ce facteur rend chaque mois un compte précis de sa recette et de sa dépense, ainsi que de l'entrée et sortie de denrée de toutes espèces. Son salaire va assez ordinairement au 5 pour cent du produit net des fonds qu'il a sous son inspection, celui du sousfacteur à 15 ou 20 francs par mois outre sa nourriture.
Les comptes mensuels du facteur sont remis à un teneur de livres, qui en fait le relevé, et à la fin de l'année; dresse le compte général du produit. Comme un teneur de livres peut suffire pour plusieurs propriétaires à-la-fois, le salaire qu'il retire de chacun d'eux ne leur est nullement onéreux. De cette manière le propriétaire jouit de la bonne culture de son fonds, sans en avoir aucun embarras; il n'a point à éprouver ces petites luttes, ces petites vexations, et il n'est point chargé de ces détails, qui dégoûtent tant de personnes de l'agriculture.
\setcounter{page}{289}
\section{ON PLANTING AND REARING, etc. De la plantation, et des moyens de faire réussir les haies d'épine. Par WILLIAM AITON. ( Farmer's Magazine ).}
ON fait les clôtures en Ecosse, avec des gazons, des pierres et des haies vives. Les gazons s'emploient de préférence dans les cantons tourbeux, où l'on ne trouve pas de pierres et où les épines ne végétéroient pas. Beaucoup de gens préfèrent les clôtures en pierres, là où on en a les matériaux ; mais il est plus convenable d'employer les haies vives d'aubépine partout où elles peuvent réussir. Ceux qui préfèrent les pierres observent qu'elles forment d'abord une barrière efficace, qu'elles n'ont pas besoin d'une fausse haie, qu'elles ne servent point de retraite aux oiseaux, qu'elles ne favorisent pas la végétation des mauvaises herbes, qu'elles n'ont pas, comme les haies vives, l'inconvénient d'empêcher le blé de sécher après qu'il est moissonné ; et qu'enfin lorsqu'elles
Agricult. Vol. 18. No. 8. Août 1813. Z\setcounter{page}{290} les se dégradent, on les reconstruit facilement. Mais elles coûtent trois ou quatre fois plus cher que les haies vives, ne durent pas autant, et ne donnent point au bétail un aussi bon abri. Lorsque les clôtures en pierres ne sont pas posées sur le roc, ou fondées très-profondément, les gelées et les dégels les renversent, sur-tout quand la direction de la clôture est de l'est à l'ouest \footnote{L'auteur n'explique pas quelle peut être la cause de ce fait. Voici ce qui nous paroît probable. Dans les fortes gelées, l'humidité dont la couche supérieure du sol est imprégnée, se convertissant en glace, éprouve la dilatation qui accompagne toujours ce changement de forme. La masse de la muraille sèche en est soulevée, et lorsque cette glace interposée dans la terre, redevient de l'eau, et que par conséquent la terre s'affaisse, cela doit arriver inégalement des deux côtés d'une muraille qui court est et ouest. La terre qui est du côté du nord est encore gelée, et par conséquent soulevée, tandis que de l'autre côté le sol est déjà affaissé par le dégel. La chute des pierres du haut du mur doit souvent résulter de ce penchement de la base.}.
L'aubépine semble nous avoir été donnée par la nature pour former des haies. Mais la réussite de ces haies demande du soin. La sévérité du climat n'est point un obstacle à la réussite de l'aubépine. Lorsque j'ai fait la reconnaissance agricole pour le Département d'agriculture des comtés d'Ayr, de Lanark et\setcounter{page}{291} du duché d’Hamilton, j’ai vu des haies d’épines de vingt pieds de haut dans les situations les plus élevées, comme, par exemple, près des sources de l’Aven, de la Clyde, et de la Tweed. La stérilité de la terre n’est point non plus un obstacle aussi grand qu’on l’imaginerait à la croissance de cette plante. J’ai vu des haies d’aubépine, qui faisaient une barrière complète dans des terrains extrêmement maigres, qui n’auraient pas pu se louer plus de dix shillings l’acre. Tandis que dans la même paroisse, des haies plantées sur des terrains qui valoient dix fois davantage avoient tout-à-fait manqué. La faculté de la terre de retenir l’humidité, et les soins donnés à la haie sont les deux circonstances qui en assurent la réussite.
( Ici l’auteur s’étend sur la manière dont on plante ordinairement les haies d’aubépine dans son canton, c’est-à-dire, en faisant un fossé sur le revers duquel on place les plants, en revêtant de gazon battu les deux pentes qui forment les faces supérieures du prisme. De cette manière, les eaux pluviales s’écoulent facilement des deux côtés, et manquent aux racines, ensorte que la plante languit et périt. Le but de se procurer d’abord une barrière suffisante par un fossé et un revers élevé, peut être également atteint, à moins\setcounter{page}{292} de frais, en garnissant les jeunes plants avec une haie de bois mort. Il recommande de planter les haies dans un fossé suffisamment profond, en ayant soin de jeter au fond de ce fossé la meilleure terre de la surface, et en garnissant les tiges à la superficie du sol, de la terre stérile tirée du fond, afin que les mauvaises herbes croissent avec moins d'abondance des deux côtés de la jeune haie. Il recommande encore d'émonder celle-ci dans les côtés et dans le haut, de manière à lui donner la forme d'un coin tranchant, ou celle de la crinière d'un cheval, coupée courte, et se tenant dans une direction verticale. De cette manière, le pied de la haie demeure garni. Receper les jeunes haies à la troisième ou quatrième année, est un moyen bien connu de renforcer les plantes par le bas; mais ce que l'auteur ne recommande pas, et qui est bien nécessaire à la pleine réussite des jeunes haies d'aubépine, c'est de les cultiver à la houe à deux pointes, à un pied au moins de chaque côté, une fois chaque printems, pendant les trois premières années).
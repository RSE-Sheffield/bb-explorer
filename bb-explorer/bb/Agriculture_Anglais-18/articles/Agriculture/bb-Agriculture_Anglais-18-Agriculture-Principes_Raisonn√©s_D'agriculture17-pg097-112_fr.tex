\setcounter{page}{97}
\section{PRINCIPES RAISONNÉS D'AGRICULTURE. Traduit de l'allemand d'A. THAER, par E. V. B. CRUD. Tom. Ier. in-4°, 372 p. Genève, 1811, chez J. J. Paschoud, Imprimeur-Libraire; et à Paris chez le même rue Mazarine, n°. 22. \large{(Dix-septième extrait. Voy. p. 41)}}
Des diverses espèces de terrains, leur valeur, leur emploi, et leurs propriétés, dans leurs rapports avec les proportions des parties constituantes du sol.
Le meilleur sol est le résultat des plus heureuses proportions du mélange des substances énumérées. L'infinie variété de ces proportions produit cette innombrable variété de terrains, qui ne sont séparés les uns des autres par aucune limite, mais seulement par des nuances imperceptibles.
Jusqu'ici on a principalement classé les terres par leur fécondité, et selon les produits dont elles étoient susceptibles. Cette classification n'ayant pas pour base la con-\setcounter{page}{98} noissance des parties constituantes de ces terres, ne pouvoit être que vicieuse. Lorsqu'on a essayé de classer les terrains d'après leurs parties constituantes, on n'a pas assez fait d'attention à leur fertilité. L'auteur a entrepris un travail destiné à réunir les deux classes d'observations. Il a analysé plusieurs centaines de terrains différens, en rassemblant, en même temps, les renseignemens les plus précis sur leur fécondité. Nous allons le faire parler lui-même.
Dans l'estimation des divers sols, que je vais faire d'après leurs parties constituantes, je suppose, avant tout, que ces terrains sont dans les mêmes circonstances de position, d'humidité, de profondeur, de couches inférieures, etc. et que d'ailleurs ils sont, à ces divers égards, exempts de défauts. Nous verrons ensuite quelle influence ces diverses circonstances exercent sur les divers sols.
La fertilité, la richesse du sol, dépend de l'humus et de ses proportions; mais cette matière, en tant qu'elle ne peut pas être décomposée, exerce aussi une influence physique sur le sol. Elle rend poreux le terrain argileux; elle facilite l'action de l'air sur le sol, sa pulvérisation, et son ameublissement. Elle donne au sable plus de consistance, et plus d'affinité avec l'eau; enfin cette ma-\setcounter{page}{99} tière produit ces effets d'une manière plus sensible lorsqu'elle est associée à l'argile ou au sable, que lorsqu'elle est isolée. L'humus rafraîchit, ou rend plus doux, le sol où la chaux domine; il lui donne plus de consistance, et empêche que l'humidité ne s'en évapore trop facilement.
Cette substance fécondante peut néanmoins exister en trop grande proportion dans le sol, et le rendre trop meuble et trop spongieux: il n'a plus alors la consistance nécessaire pour servir d'appui aux racines des plantes. Alors le terrain absorbe l'eau comme une éponge quand la température est humide; ensorte que les plantes qui y végètent éprouvent les accidens causés par le trop d'humidité: elles prennent des maladies et périssent. Dans les sécheresses, au contraire, ce même terrain laisse trop facilement évaporer l'eau, et devient pulvérulent à sa surface: les semences ne peuvent pas germer, ou périssent après avoir germé. A quelques pouces de profondeur, en revanche, le sol pénétré d'humus peut être tellement humide qu'on puisse en exprimer l'eau comme d'une éponge. Enfin un sol ainsi surchargé d'humus se contracte ou s'enfle à chaque changement de température, ce qui fait que les racines des plantes se\setcounter{page}{100} soulèvent et se détachent de manière à faire souffrir les plantes, et quelquefois à les faire périr. Aussi ces terrains-là conviennent-ils moins aux grains d'automne qu'à ceux de printemps, et moins à l'orge qu'à l'avoine. Ce sol a encore l'inconvénient de favoriser plus la végétation des mauvaises herbes que celle des céréales, de manière que celles-ci en sont quelquefois étouffées. Le sol qui contient le plus d'humus, lors même que c'est de l'humus doux, n'est donc pas le plus profitable, quoiqu'on puisse l'employer à amender d'autres terrains.
S'il est humide, c'est comme prairie qu'il est le plus avantageux. S'il n'est pas marécageux et qu'il soit ensemencé en vulpin des prés, en festuques et en pâturins, il forme les prairies les plus riches. S'il est sec on peut l'améliorer en y mêlant des terres plus maigres, ou mieux encore en l'écobuant, parce que le feu consume une partie de l'excès d'humus; mais il est dangereux ensuite que les grains n'y versent.
L'argile est de toutes les terres celle qui peut supporter la plus grande proportion d'humus, parce que les propriétés de celui-ci corrigent les défauts de celle-là. Le terrain le plus riche que nous ayons analysé, et qui avoit été pris dans les marais de l'O-\setcounter{page}{101} \section{PRINCIPES RAISONNÉS D'AGRICULTURE}
ʻder, contenoit 19 et 3/8 p. ‰ d’humus, 70 p. ‰ d’argile, un peu de sable fin et de la chaux en quantité à peine sensible. Ce terrain étoit placé trop bas pour qu’on pût bien apprécier sa fécondité. Les céréales d’automne n’y réussissoient pas ; et le succès de celles de printems n’y étoit pas assuré. Il avoit d’ailleurs assez de consistance, et un degré très-convenable d’adhésion avec l’eau. Onze et demi pour cent est la plus grande proportion d’humus que nous ayions trouvée dans la terre argileuse. Nous n’avons pas eu occasion d’analyser ces terrains inépuisables qui sont plutôt détériorés qu’améliorés par l’addition du fumier, et qui rapportent tous les ans des récoltes au point de maturité, sans s’affaiblir jamais, pourvu qu’ils reçoivent une culture suffisante. On assure qu’on en trouve de tels dans l’Ukraine, en Hongrie, dans les bas-fonds de la Thées, et dans des espaces d’une petite étendue, même en Allemagne. Quoique plusieurs des terrains que nous avons analysés, passassent pour être inépuisables, peu de temps après qu’on les eut enlevés à la mer ou qu’on eut rompu le gazon, qui les recouvroit pour les mettre en culture, il est démontré, qu’ils finissent par avoir besoin d’engrais, si on ne les remet pas en herbages ou pâturages, ou si,\setcounter{page}{102} en les défonçant à la bêche, on ne ramène pas à la surface, la terre non épuisée, des couches inférieures. Il n'y a plus qu'un petit nombre de contrées où l'on croie pouvoir se passer entièrement de fumier, et ce sont celles où le sol est consacré plutôt aux herbages et à l'entretien du bétail, qu'à la culture des grains.
Le sol argileux le plus riche que nous ayions analysé, et dont la fécondité a une grande réputation, étoit tiré de la rive droite de l'Elbe, à quelques milles de son embouchure. Il étoit composé, comme je l'ai dit, de 11 p. ‰ d'humus, 4 ½ p. ‰ de chaux, un peu de silice grossière; une assez grande proportion de silice fine, qu'on ne pouvoit séparer que par l'ébullition, et le reste argile. Il avoit beaucoup de cohésion; mais modérément humide, il n'étoit pas très-tenace. Il portoit les récoltes les plus riches de froment, de colza, d'orge d'automne, de fèves; mais tous les six ans, il étoit fortement amendé avec le fumier, et recevoit une jachère.
Nous avons trouvé l'humus mêlé à l'argile dans les terrains bas qui sont d'une extrême fertilité lorsque leur assolement est bon. Un terrain réputé le plus fertile, dans le pays de Brodjading, contenoit 8 ⅔ pour\setcounter{page}{103} cent d'humus, et 3 à 4 p. \textecr{c} de chaux: le reste étoit de l'argile presque pure. Un terrain du bailliage de Vollup, qui contenoit 6 \textecr{1/2} p. \textecr{c} d'humus, étoit aussi une excellente terre à froment, puisque la troisième récolte qu'il produisoit après avoir été fumé, étoit encore très-vigoureuse.
La couleur foncée du sol n'est pas toujours en rapport avec la quantité d'humus que ce sol contient. Quelquefois le sol est blanchâtre, comme je l'ai dit plus haut, et contient cependant une plus grande proportion d'humus qu'une autre qui a une couleur plus foncée; mais cette dernière couleur se développe, lorsqu'on fait subir à ce terrain l'incandescence\dag\ dans un creuset fermé.
On ne rencontre ces riches terrains argileux et glaiseux que dans des bas-fonds, sur lesquels les eaux ont déposé une couche plus ou moins épaisse de limon; ainsi qu'au bord des rivières dont le cours s'est étendu d'une manière douce et insensible, et s'est retiré de même, ou dans des vallées qui formoient des lacs avant que les eaux se fussent frayé une autre route. L'on range les terrains de cette espèce dans la première classe, et on les caractérise ordinairement sous la dénomination de riches terres à fro-\setcounter{page}{104} ment, parce que dans le système de culture avec assolement triennal et jachère, ils peuvent rapporter trois récoltes de froment avant d'être fumés de nouveau. Cependant les terrains compris dans cette classe ont des nuances dans leur fertilité et leur valeur. Je ne me permettrai pas de décider si l'on peut fixer celle-ci uniquement d'après la proportion d'humus que le sol contient, parce que des comparaisons de fertilité à des distances éloignées sont trop difficiles, et que d'ailleurs cette fertilité dépend beaucoup du climat. On ne peut également point encore décider si la plus ou moins grande proportion de chaux et de matières animales, qui vraisemblablement lui est souvent mêlée, influe sur la fécondité. D'après les résultats de nos recherches, je crois pouvoir admettre en principe que la terre végétale doit contenir au moins cinq pour cent d'humus pour être comprise dans les terres fertiles. Si l'humus est mêlé avec une forte proportion de sable, et avec peu d'argile, le sol manque de consistance. Il se pénètre aisément d'eau et se sèche de même; alors la proportion d'humus peut facilement être trop forte. Nous avons analysé un sol qui contenoit vingt-six pour cent d'humus, et\setcounter{page}{105} qui d'ailleurs étoit composé de parties à-peu-près égales d'argile et de sable, et nous l'avons trouvé déjà trop meuble et moins favorable à la culture des grains ; lorsqu'il fut débarrassé et qu'on eut rompu le gazon, les premières récoltes qu'il rapporta furent très-riches, mais la fécondité diminua bientôt, et quoique par d'abondans engrais on eût cherché à lui rendre ce qu'il avoit perdu, il ne put pas recouvrer son ancienne valeur.
En revanche, un autre terrain plus sablonneux, mais ne contenant que dix pour cent d'humus, nous a paru très-fertile, et propre à toutes sortes de céréales, excepté au froment. Cependant ce terrain demandoit beaucoup d'engrais. Si un tel terrain en est privé, il peut être facilement épuisé.
Si le sol contient vingt pour cent d'argile pure, ( c'est-à-dire, d'argile débarrassée de tout le sable qui peut lui être enlevé par le lavage) dix pour cent d'humus, et le reste de sable, ce terrain produit encore de belles orges. S'il contient sensiblement moins d'argile, ce n'est plus qu'une terre à seigle ; et encore faut-il que ce grain soit semé de bonne heure pour avoir la force de résister à l'hiver.
A quantité égale d'humus, on peut clas-\setcounter{page}{106} ser la fertilité du sol par sa consistance : celui qui en a le plus est le plus productif pour les céréales.
Nous avons supposé jusqu'ici que l'humus était doux, ou exempt d'acide. Quelquefois le sol n'est que très-faiblement acide, et sa fécondité en est alors peu altérée. A mesure que son acidité est plus forte, les produits en orge diminuent ; mais ceux de l'avoine se maintiennent. Le seigle s'y rouille, et est sujet à verser. Les céréales y donnent peu de farine ; et les herbes quoiqu'abondantes, y sont de mauvaise qualité. A mesure que l'acidité du sol augmente, sa valeur diminue, et il devient enfin terre de marais.
La couleur noire du sol est un indice qui n'induit en erreur que dans le petit nombre des cas où cette couleur est produite par un oxyde de fer ou de manganèse. Pour s'assurer de la chose on soumet une petite quantité de ce sol à l'incandescence, dans un creuset, pendant dix minutes, après l'avoir débarrassé de racines, de pierres, et l'avoir pesée avec soin, parfaitement sèche. On remue pendant l'opération avec un tube de verre ; et pour accélérer l'action totale de l'humus, on joint à la terre un peu de nitrate d'ammoniaque, lequel se volatilise complètement.\setcounter{page}{107} La diminution du poids indique la quantité d'humus que le sol contenoit. Sans doute le terrain, sur - tout celui qui est argileux, a perdu dans cette opération quelque peu d'eau qui avoit avec lui une adhésion telle qu'elle ne pouvoit être détruite que par l'incandescence ; mais cela ne peut faire qu'une différence insignifiante, et si la terre a été auparavant bien séchée, l'erreur ne peut s'élever au-dessus de demi pour cent. Si cependant, le sol contenoit beaucoup de chaux, la volatilisation de son acide carbonique et de son eau de cristallisation seroit d'une grande conséquence, et ainsi, il faudroit avant tout, séparer la chaux.
On découvre l'acidité de l'humus en plongeant une bande de papier teinte en bleu avec du tournesol dans une pâte liquide, faite avec la terre qu'on analyse et de l'eau. Si ce papier devient rouge, c'est un signe qu'il y a de l'acide. Du reste, l'humus acide se fait connaître déjà par l'odeur qu'il répand lorsqu'il est mis en ignition, odeur qui est semblable à celle de la tourbe qu'on brûle. Si dans sa combustion, l'humus donne une odeur de plume brûlée, c'est un indice qu'il a une origine animale, et par conséquent, qu'il est plus riche et qu'il peut mieux être décomposé.\setcounter{page}{108} Au moyen de l'appareil pneumatique et par la distillation à sec, on feroit sans doute une analyse bien plus précise de l'humus; mais une telle opération n'est pas du ressort de l'agriculteur. Cependant, Arthur Young l'a fréquemment exécutée, il a trouvé la quantité de gaz hydrogène obtenue, en rapport avec la fertilité du sol, de sorte qu'il a proposé cette opération comme un procédé propre à mesurer le degré de fertilité. Priestley s'est joint à lui, et l'a appuyé de ses observations.
L'argile augmente la fertilité du sol.
1°. Par son adhérence avec l'eau. Cette adhérence est telle, que même pendant une longue sécheresse, l'argile conserve l'humidité indispensable à la nourriture des plantes.
2°. En conservant l'humus. Non seulement elle l'enveloppe et le protège, mais encore elle se combine chimiquement avec lui.
3°. Par l'appui solide qu'elle donne aux racines des plantes, et même par la résistance qu'elle offre à leur trop grande extension, ce qui les oblige à pousser des racines chevelues, au moyen desquelles chaque plante cherche sa nourriture autour d'elle, et par conséquent l'enlève moins aux plantes voisines.
4°. En empêchant que l'air atmosphérique\setcounter{page}{109} que ne parvienne jusqu'aux racines des plantes, auxquelles il est toujours nuisible, et en leur communiquant la chaleur d'une manière plus modérée; en conservant aussi aux végétaux une température plus égale, malgré les changemens continuels qui s'opèrent dans celle de l'atmosphère. Lorsque le terrain argileux n'est pas trop humide, les effets du passage subit du chaud au froid, et le contraire, sont par conséquent moins nuisibles aux récoltes qui y croissent, qu'ils ne le sont dans les terrains sablonneux.
5°. En ce qu'elle attire l'oxigène, cette substance nécessaire à la formation de l'acide carbonique. Elle attire très-vraisemblablement aussi l'azote, et favorise ainsi l'action réciproque de ces substances entr'elles.
\section{En revanche, l'argile est nuisible :}
1°. Parce qu'en temps humide, elle conserve trop long-temps l'eau dont elle est pénétrée, qu'elle ne la laisse ni égoutter ni évaporer, et forme, avec ce liquide, une bouillie liée.
2°. Parce qu'à une température sèche, elle se durcit trop; qu'elle présente alors une trop grande résistance aux racines. Elle devient presque semblable à une masse de briques.\setcounter{page}{110} 3°. Parce que, soit pendant la sécheresse soit pendant la gelée, il s'y forme des crevasses par lesquelles les racines sont ou déchirées, ou mises en un contact immédiat avec l'air atmosphérique, ce qui peut leur être très-nuisible.
4°. Parce qu'elle attire fortement, et s'incorpore les sucs nourriciers des engrais, et qu'elle ne s'en sépare point avec la même facilité que la terre meuble. A la vérité, si elle en est une fois richement pourvue, et en quelque façon saturée, elle demeure d'autant plus long-temps féconde, mais si elle est une fois épuisée, les premiers engrais qu'on lui donne font beaucoup moins d'effet. Donc, pour que les récoltes se ressentent des premiers engrais, dans des terrains argileux, il faut qu'ils soient très-abondans.
5°. Parce qu'elle rend la culture beaucoup plus difficile ; qu'en temps humide, elle ne permet guères d'y entrer avec les charrues, les herses, et les chariots ; qu'elle s'attache comme une pâte, à ces instrumens, qu'elle empêche leur action, et ne peut être divisée qu'avec difficulté; parce qu'en temps sec, au contraire, elle se contracte, et se durcit tellement, que la charrue peut à peine ouvrir le terrain, en le faisant sauter en grosses mottes, lesquelles ne peuvent être\setcounter{page}{111} brisées, avant d'avoir reçu la pluie, ni par la herse ni par le rouleau ; en sorte qu'on est forcé de les casser à la main, avec des maillets, dont l'effet n'est pas même complet.
Le mélange de l'humus prévient une partie des effets nuisibles d'un excès d'argile. Une addition de chaux y porte aussi remède ; mais rien n'est plus efficace que le sable, et c'est aussi le moyen le plus ordinairement employé. La couche supérieure du sol contient presque toujours un peu de sable, sans lequel les champs pourraient à peine être attaqués par les instrumens aratoires. En conséquence l'estimation des divers terrains doit toujours avoir pour base la proportion dans laquelle l'argile et le sable s'y trouvent mêlés.
J'entends ici par sable, cette silice à grains grossiers qui se précipite au fond du vase, dans un lavage soigné. Mais si l'on met l'argile dans l'eau en ébullition, il s'en sépare encore beaucoup de silice à grains fins. Il paroît que la quantité de cette silice à grains fins constitue la différence entre l'argile grasse et l'argile maigre. Mais comme il s'agit ici de pouvoir reconnoître d'une manière facile, et qui soit d'un emploi universel, la proportion des parties constituantes du sol, pour en fixer la valeur, il ne faut faire aucune\setcounter{page}{112} attention à cette silice fine, qui ne peut être séparée par un simple lavage, et on doit regarder comme argile pure, celle qui a subi un lavage soigné à l'eau froide. Le plus souvent, on peut encore extraire de cette argile lavée, 15 p. % de silice fine par l'ébullition. Dans quelque terrain particulier seulement, elle s'est élevée plus haut, ainsi, par exemple, le sol d'une nouvelle alluvion dans l'isle de Nogal, près de Dantzig, contenoit une beaucoup plus grand quantité de silice fine. Il faudra des recherches plus étendues, pour pouvoir déterminer jusqu'à quel point l'argile qui contient une plus grande proportion de cette silice fine, a besoin d'une moindre addition de sable, pour atteindre la porosité nécessaire.
\section{A V I S.}
( Extrait des feuilles d'Hofwyl ).
On sait que la réussite d'une récolte de grains dépend, en grande partie, des semailles. La quantité de la semence ne doit être ni trop forte ni trop foible, mais proportionnée à la qualité et à l'état du terrain. Il importe que les grains de semence
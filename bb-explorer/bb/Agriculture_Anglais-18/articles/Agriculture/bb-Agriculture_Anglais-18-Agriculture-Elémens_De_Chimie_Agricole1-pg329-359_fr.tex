\setcounter{page}{329}
\section{ELEMENTS OF AGRICULTURAL CHEMISTRY, etc. Elémens de chimie-agricole en un cours de leçons pour le Département d'Agriculture; par Sir HUMPHRY DAVY. Londres, 1813.}
DANS un avertissement daté du 21 mars 1813, l'illustre chimiste rappelle que depuis dix ans il donne chaque année un cours de chimie agricole à l'usage du Département d'agriculture, en faisant usage des découvertes modernes, à mesure qu'elles ont eu lieu. Il observe que les découvertes en chimie se multiplient si rapidement, qu'au moment d'imprimer son ouvrage, il a été obligé d'y faire plusieurs changemens et additions.
Première leçon.
L'auteur observe d'abord, que la chimie agricole n'a pas encore reçu de forme systématique et régulière ; et il sollicite l'indulgence de ses auditeurs, à cause de la nouveauté et des difficultés du sujet.
Agric. Vol. 18. N°. 9. Sept. 1813. Cc\setcounter{page}{330} La chimie agricole, dit-il, a pour objet les changemens, et les dispositions de la matière, qui dépendent de la croissance et de la nourriture des plantes; la valeur comparative des végétaux comme nourriture, la constitution des terres, et la manière dont elles sont enrichies par les engrais, ou rendues fertiles par les divers procédés de la culture. Des recherches de cette nature ne peuvent qu’avoir beaucoup d’intérêt et d’importance, soit pour l’agriculteur théorique, soit pour le praticien. Elles fournissent au premier la plupart des principes fondamentaux, desquels dépend la théorie de l’art; elles dirigent le second dans ses travaux, et donnent la faculté de suivre un plan systématique d’améliorations. On ne sauroit presque faire aucune recherche en agriculture sans entrer dans le domaine de la chimie. Si un terrain est ingrat et qu’on projette de l’améliorer, la marche la plus sûre est de déterminer préalablement la cause de sa stérilité: celle-ci dépend de quelque déficit dans la constitution du sol, et l’analyse chimique découvre ce déficit. Certaines terres d’une consistance bonne en apparence, sont d’une stérilité extrême. L’observation et la pratique ne fournissent\setcounter{page}{331} aucun moyen d’en découvrir la cause. C’est évidemment le cas de l’application des procédés chimiques, puisque la stérilité est probablement due à la présence de quelques principes chimiques qu’on peut aisément découvrir et détruire. Existe-t-il, par exemple, des sels ferrugineux? on peut les décomposer par la chaux. Y a-t-il excès de sable siliceux? Il faut amender avec l’argile et la chaux. Y a-t-il défaut de terre calcaire? Le moyen d’y remédier est évident. Y a-t-il excès de matière végétale? Il faut amender et chauder. Y a-t-il défaut de cette même matière végétale? Il faut y suppléer par les engrais.
On est souvent en différend sur les diverses espèces de chaux à appliquer à l’amendement des terres. Il faudroit des années pour déterminer par voie expérimentale, leur valeur respective, et d’ailleurs les essais pourroient nuire aux récoltes; mais par l’usage des réactifs, on découvre en quelques minutes quelle est la nature d’une chaux donnée, et on détermine si elle est propre à telle ou telle terre, ou seulement à faire du mortier.
La tourbe, d’une certaine consistance et composition, est un excellent engrais; mais il y a des variétés de tourbes qui contiennent\setcounter{page}{332} nent beaucoup de matières ferrugineuses, et qui sont un poison pour la végétation. Rien n'est plus simple que l'opération chimique qui détermine la nature et les effets probables d'une tourbe quelconque. Il n'y a aucune question sur laquelle on ait plus différé d'opinion que celle de la convenance d'enterrer le fumier frais ou après le procédé de la fermentation. Cette question est encore un sujet de discussion ; mais si l'on a recours aux plus simples principes de la chimie, on ne sauroit à cet égard, conserver de doute. Dès que le fumier commence à se décomposer ; il perd ses parties volatiles, qui sont les plus précieuses et les plus efficaces. Lorsque le fumier a fermenté de manière à devenir une masse molle et cohérente, il a ordinairement perdu entre un tiers et une moitié de ses élémens les plus utiles. Evidemment, il convient de l'appliquer aussitôt que la fermentation commence, afin qu'il emploie toute son action sur la plante, et que sa faculté nutritive ne soit point perdue. Il seroit facile de citer beaucoup d'autres exemples semblables ; mais je crois en avoir dit assez pour prouver que la connexion de la chimie avec l'agriculture n'est point fondée sur une spéculation vague ; mais qu'elle offre des principes\setcounter{page}{333} pes qui doivent être entendus et suivis, et qui, dans leurs derniers résultats, ne peuvent qu'être extrêmement utiles à la communauté. Les phénomènes de la végétation sont une branche importante de la science de la nature organisée. L'existence des végétaux dépend en grande partie des lois de la matière inerte. Ils se nourrissent par les élémens dont ils sont entourés; ils assimilent cette nourriture à leur substance par des organes particuliers: c'est en examinant leur constitution physique et chimique, les substances et les forces qui agissent sur eux; enfin les modifications qu'ils subissent, que l'on fixe les principes de la chimie agricole.
Il est donc évident que l'on doit commencer cette étude par quelques recherches générales sur la nature et la composition des corps; et sur les lois des changemens qu'ils éprouvent. La surface de la terre, l'atmosphère et l'eau qui en tombe, doivent procurer tous les principes de la végétation; et ce n'est qu'en examinant la nature chimique de ces principes, que nous pouvons découvrir quelle est la nourriture des plantes, et de quelle manière cette nourriture est préparée et distribuée. La constitution des corps doit donc être le premier objet à considérer.\setcounter{page}{334} Les moyens d'analyse que les appareils chimiques et électriques, récemment découverts, nous ont donné, ont servi à prouver que toutes les variétés des substances matérielles peuvent se rapporter à un petit nombre de corps, que nous appellons élémentaires; parce que dans l'état actuel des connoissances chimiques, nous ne pouvons les décomposer. Les corps aujourd'hui reconnus incapables de décomposition ultérieure, sont au nombre de quarante-sept, dont trente-huit métaux, sept corps inflammables et deux gaz; qui, unis avec des métaux et des corps inflammables, forment les acides, les alkalis, les terres et d'autres composés analogues. Les éléments chimiques se combinent de diverses manières selon leurs facultés d'attraction. Dans leurs combinaisons les plus simples, ils produisent diverses substances cristallines qui se distinguent par la régularité de leurs formes. Dans des arrangemens plus compliqués, les éléments chimiques constituent les diverses substances végétales et animales, prennent le caractère plus relevé de l'organisation, et servent au maintien de la vie. L'influence de la chaleur, de la lumière et de l'électricité produit une suite non interrompue de changemens. La matière prend continuellement de\setcounter{page}{335} nouvelles formes. La destruction d'un ordre de corps organisés, tend à la conservation d'un autre ordre. La dissolution et la consolidation, le déclin et le renouvellement, sont liés; et tandis que les parties du système sont dans un état de fluctuation constante, l'ordre et l'harmonie de l'ensemble demeurent inaltérables.
Après avoir pris une idée générale de la nature des substances élémentaires, et des principes des modifications chimiques, nous considérerons la structure et la constitution des plantes. Il existe dans toutes un système de tubes ou vaisseaux, qui se terminent par une de leurs extrémités aux racines, et par l'autre aux feuilles. La sève, en montant, s'épaissit, et devient plus propre à déposer de la matière solide. Elle se modifie par la chaleur, la lumière et l'air, dans les feuilles; elle descend par l'écorce, et produit, dans sa marche, une nouvelle matière organisée: elle devient ainsi, en montant et en descendant, au printemps et en automne, la cause de la formation de nouvelles parties, et du développement plus complet des parties déjà formées.
Je tâcherai de réunir sous un même point de vue, en traitant cette matière, les observations des philosophes les plus éclairés,\setcounter{page}{336} qui ont étudié la physiologie des plantes, tels que Grew, Malpighi, Senebier, Darwin, et surtout Mr. Knight. Celui-ci est le dernier qui ait fait des recherches sur cet objet, et ses travaux ont contribué à porter beaucoup de lumière sur cette branche de l'économie de la nature. Depuis dix ans, la composition chimique des plantes a été examinée et soumise à des expériences, par plusieurs chimistes en Angleterre et ailleurs ; et les connaissances acquises par leurs travaux forment une partie intéressante de la chimie générale. Cette partie est trop étendue pour qu'on puisse la traiter avec détail ; mais il sera convenable de s'arrêter sur les points qui importent à la pratique.
Si l'on soumet à l'analyse les organes des plantes, on trouve qu'une variété presqu'infinie de formes dépend des divers arrangements et combinaisons d'un petit nombre d'éléments : il est rare qu'il y en ait plus de sept à huit ; et trois seulement constituent la plus grande partie de la matière organisée. Selon que ces éléments sont disposés, les produits de la végétation ont telles ou telles propriétés, soit comme aliment, soit pour les divers emplois auxquels on les destine.
La valeur et les usages de chaque produit\setcounter{page}{337} agricole, s'apprécient le plus correctement possible, lorsque la pratique s'aide des connaissances chimiques. Les composés vraiment nutritifs, pour les animaux, dans les plantes, sont en petit nombre, savoir, la fécule ou matière de l'empois, le sucre, le gluten, la gelée végétale, et l'extrait. De ces matières composées, le gluten est la plus nourrissante : c'est celle qui se rapproche le plus de la substance animale, et qui donne au froment sa supériorité sur les autres grains. Le sucre vient ensuite, pour la qualité nourrissante; puis la fécule; enfin la gélatine des végétaux, et la partie extractive. L'épreuve la plus simple de la faculté nutritive de divers produits végétaux, est la comparaison des quantités qu'ils contiennent de gluten, de sucre, de fécule, etc. : dans les temps de disette, une telle connaissance peut être très importante. Le sucre et la fécule ont beaucoup de rapports dans leur composition, et chacun des deux peut se convertir en l'autre, par des procédés chimiques. Lorsque nous examinerons ces substances, je ferai connoître les résultats de quelques expériences récentes qui fournissent des applications à l'économie de la végétation, et à certains procédés importans dans les manufactures.\setcounter{page}{338} Toutes les substances qu'on trouve dans les plantes sont produites par la sève; celle-ci provient de l'eau ou des fluides du sol; puis elle est altérée ou combinée par les principes que donne l'atmosphère. L'influence du sol, de l'eau et de l'air est donc l'objet à considérer. Le sol est toujours composé d'un mélange de plusieurs matières terreuses, de substances animales et végétales en décomposition, et de certaines substances salines. Les matières terreuses sont véritablement la base du sol: les autres substances, soit qu'elles se trouvent naturellement dans le sol, ou qu'elles y soient introduites avec intention, agissent de la même manière que les engrais. Quatre terres se trouvent généralement en abondance dans le sol, savoir, l'alumine, la silice, la chaux, et la magnésie. Ces terres, ainsi que je l'ai découvert, sont composées de métaux extrêmement inflammables, et d'oxygène: je ne crois pas qu'elles soient décomposées ou altérées dans la végétation. Le principal usage du sol, est de fournir un appui aux plantes, et de fixer leurs racines, afin qu'elles puissent pomper leur nourriture peu-à-peu des substances solubles et dissoutes qui se trouvent mêlées aux diverses terres. On ne sauroit douter que la fertilité d'un\setcounter{page}{339} sol ne dépend en partie, d'un certain mélange des terres, et presque tous les sols stériles sont susceptibles d'être améliorés par la modification des terres qui entrent dans leur composition. Je décrirai la méthode la plus simple, jusqu'ici connue, d'analyser les divers sols, et de reconnaître la constitution et les ingrédients chimiques qui influent sur la fertilité. Plusieurs difficultés qui accompagnoient ces recherches, ont été écartées par des découvertes récentes.
La nécessité de la présence de l'eau pour la végétation, et la force de croissance des plantes qui ont le bénéfice des arrosements dans les pays méridionaux, ont conduit à une opinion qui avoit déjà prévalu dans l'ancienne école : savoir, que l'eau étoit la substance qui pouvoit produire toutes les autres, et dans laquelle toutes celles-ci pouvoient se résoudre. Il paroît que les Grecs avoient emprunté cette opinion des Égyptiens, et les chimistes modernes l'avoient renouvelée. Van Helmont, en 1610, crut avoir prouvé par une expérience décisive, que tous les produits de la végétation pouvoient procéder de l'eau. En 1691, Woodward prouva que Van Helmont s'étoit trompé, mais le véritable usage de l'eau dans la végétation demeura ignoré jusqu'en 1785, que Caven-\setcounter{page}{340} dish découvrit que l'eau étoit composée de deux fluides élastiques ou gaz, savoir, l'hydrogène ou gaz inflammable, et l'oxigène ou air vital.
L'air avoit été regardé ainsi que l'eau comme un pur élément chez les anciens philosophes. Quelques chimistes dans le seizième et dix-septième siècle, formèrent des conjectures heureuses sur la véritable nature de l'air. Le chevalier Digby en 1660, croyoit que l'air contenoit une matière saline essentielle à la nourriture des plantes. Entre 1665 et 1680 Boyle, Hoocke et Mayow, constatèrent qu'une partie seulement de l'air, se consumoit dans la respiration des animaux et dans la combustion des corps, mais la véritable analyse de l'atmosphère est une découverte de la fin du siècle dernier, et qui est due à Scheele, Priestley et Lavoisier. Ces hommes célèbres ont démontré que les principaux élémens de l'air sont l'oxigène et l'azote, dont le premier est essentiel à la combustion, et à la vie des animaux. Ils trouvèrent aussi que l'air contenoit en petite quantité des vapeurs aqueuses, et du gaz acide carbonique. Enfin, Lavoisier montra que ce dernier gaz étoit lui-même un composé de carbone et d'oxigène.
Jethro Tull, dans son Traité d'agriculture,\setcounter{page}{341} publié en 1733, avança l'opinion que toute la nourriture des plantes étoit fournie par des molécules terreuses extrêmement divisées; que l'air et l'eau servoient de véhicules à ces particules terreuses, et que l'action des engrais étoit toute mécanique, c'est-à-dire, qu'ils n'agissoient qu'en modifiant la consistance du sol. Cet ingénieux auteur d'un nouveau système d'agriculture, ayant observé les excellens effets de la division et pulvérisation du sol, qui exposoient celui-ci au contact de l'air et de la rosée, porta les conséquences de ce principe trop loin. Duhá-mel adopta l'opinion de Tull, et établit dans un ouvrage imprimé en 1754, qu'en divisant beaucoup le sol, on pouvoit obtenir un nombre indéfini de récoltes successives sur le même champ. Il essaya aussi de prouver par des expériences directes, que toutes les plantes pouvoient réussir sans fumier. Cet homme célèbre vécut assez long-temps pour réformer son opinion sur ce point. Le résultat de ses dernières observations, et les plus soignées, fut la conviction qu'aucune substance ne servoit exclusivement à la nourriture des plantes. L'expérience de tous les agriculteurs étoit déjà décisive à cet égard; et on savoit que le fumier se consomme en-totalité par le procédé de la végétation. Le\setcounter{page}{342} fait que les terres s'épuisent par l'enlèvement des récoltes de grains, et qu'elles s'améliorent par la consommation sur place des récoltes par les bestiaux, est une preuve pratique de la vérité de ce principe. Des physiciens, en particulier Hassenfratz et De Saussure, ont prouvé, par des expériences satisfaisantes, que les substances animales et végétales déposées dans le sol, sont absorbées par les plantes, et deviennent partie de la matière organisée de celle-ci. Mais si ce n'est ni l'air, ni l'eau, ni la terre qui fournissent exclusivement à la nourriture des plantes, ils y contribuent chacun pour leur part. Le sol est le laboratoire dans lequel se prépare la nourriture des végétaux. Aucun engrais ne peut être absorbé par les racines sans la présence de l'eau, et celle-ci ou ses éléments se trouvent dans tous les produits de la végétation. La germination des semences a besoin de la présence de l'air ou du gaz oxigène, et lorsque le soleil luit, les feuilles décomposent le gaz acide carbonique de l'atmosphère, en absorbant le carbone, qui devient partie de leur matière organique, tandis que le gaz oxigène est dégagé. Ainsi, par le concours de divers agens, le procédé de la végétation, sert à maintenir l'ordre établi dans la nature.\setcounter{page}{343} Il est prouvé par différentes recherches, que depuis le temps où l'on a bien connu les parties constituantes de l'atmosphère, leurs proportions n'ont point souffert de changement, et cela est dû sans doute à la faculté qu'ont les plantes d'absorber, ou de décomposer les corps animaux ou végétaux en putréfaction, ainsi que les émanations gazeuses qui s'échappent de ces corps. Le gaz acide carbonique est formé par divers procédés de fermentation et de combustion, ainsi que par la respiration des animaux; et jusqu'à présent, on ne connoît aucune autre opération de la nature qui absorbe ce gaz, que la végétation. Les animaux produisent donc une substance qui paroît être nécessaire à la vie des végétaux, et les végétaux, à leur tour, exhalent une substance nécessaire à la vie des animaux. Ces deux classes d'êtres organisés sont ainsi liés par leurs fonctions vitales, et dépendantes l'une de l'autre pour leur existence. L'eau s'élève de l'océan, se distribue dans l'atmosphère, et retombe sur le sol, pour servir aux divers usages de la vie. Les différentes parties de l'atmosphère sont mêlées ensemble par les vents, ou par les changemens de température, et amenées successivement en contact avec la surface de la terre,\setcounter{page}{344} \section{AGRICULTURE.}
de manière à y exercer leur influence fertilisante. Les modifications du sol et l'application des engrais sont au pouvoir de l'homme, comme si la Providence eût voulu provoquer son travail et développer ses facultés.
La théorie de la manière d'opérer des engrais composés peut devenir très-claire à l'aide des principes chimiques; mais il y a encore beaucoup à découvrir sur la meilleure marche pour rendre solubles les substances animales et végétales, pour bien connoître les procédés de la décomposition, savoir retarder ou accélérer celle-ci à volonté, et produire les plus grands effets par les matériaux employés. Nous examinerons ces questions en traitant des engrais.
L'analyse démontre que les plantes sont principalement composées de carbone et de substances aériformes. Elles donnent, par la distillation, des composés volatils, dont les élémens sont de l'air pur, de l'air inflammable, une matière charboneuse, et de l'azote, c'est-à-dire, de cette substance élastique qui fait une grande partie de l'atmosphère et qui n'est pas propre à entretenir la combustion. Les végétaux tirent ces élémens, soit de l'air, par leurs feuilles, soit du sol, par leurs racines. Tous les engrais formés de substances organiques, contiennent\setcounter{page}{345} nent les principes de la matière végétale; et pendant la putréfaction, ces principes deviennent solubles à l'eau, ou prennent la forme aérienne: dans ces deux états, ils peuvent être assimilés aux organes des végétaux. Aucun principe ne fournit seul à la nourriture des plantes. Elles ne se nourrissent ni de carbone, ni d'hydrogène, ni d'azote, ni d'oxigène, mais de toutes ces substances diversement combinées.
Aussitôt que les substances organiques sont privées de la vitalité, elles commencent à subir une suite de changemens, qui se terminent par la destruction totale, c'est-à-dire, par la séparation et la dissipation de toutes leurs parties. Les matières animales sont les plus promptement détruites par l'influence de l'air, de la chaleur et de la lumière. Les substances végétales cèdent plus lentement, mais obéissent finalement aux mêmes lois. Le meilleur choix du moment pour l'application du fumier, de substances animales et végétales en décomposition, dépend de cette théorie. Je pourrai présenter quelques faits nouveaux et importans, qui sont fondés sur ces principes, et qui écartent, je crois, tous les doutes, quant à cette partie de la science.
Jusqu'ici, la chimie des engrais les plus
Agricull, Vol. 18. N°. 9. Sept. 1813. D d\setcounter{page}{346} simples, et qu'on emploie en petites doses, tels que le gypse, les alkalis, et plusieurs substances salines, a été très-obscure. On a supposé en général que ces engrais agissoient dans l'économie végétale de la même manière que les stimulans dans l'économie animale, et qu'ils rendoient les alimens ordinaires des plantes plus nourrissans. Il paroît cependant bien plus probable que ces engrais salins deviennent réellement partie des plantes, et fournissent à la fibre végétale cette matière qui est analogue à la substance osseuse des animaux. On sait qu'en Angleterre les effets du gypse sont extrêmement irréguliers, et qu'on n'a pas encore de données certaines sur son application. Il y a lieu de croire néanmoins que les recherches chimiques éclairciront ce point tout-à-fait. Les plantes dont la végétation paroît profiter le mieux de l'usage du gypse, donnent toujours cette substance à l'analyse. Le gypse se trouve dans le trèfle, et dans la plupart des plantes qui forment les prés artificiels: il existe en très-petite quantité dans le blé, l'orge et les turnips. Les cendres de tourbe qui se vendent souvent à un haut prix, sont en grande partie composées de gypse. J'ai examiné plusieurs des sols auxquels ces cendres ont été appliquées avec succès, et je n'y ai\setcounter{page}{347} pas trouvé une quantité sensible de gypse. En général les terres cultivées contiennent assez de gypse pour les plantes des prés artificiels; et dans ce cas, son application ne sauroit être avantageuse ; car les plantes ne demandent qu'une certaine quantité d'engrais; l'excès peut nuire et ne peut pas être utile.
La théorie de l'opération des substances alkalines comme engrais, est une des parties les plus clairement connues et les plus simples de la chimie agricole. On trouve les alkalis dans toutes les plantes, et ils peuvent être regardés comme un de leurs ingrédients essentiels; et par leur faculté de combinaison, les alkalis peuvent aussi servir à introduire dans la sève des végétaux, divers principes qui servent à les alimenter.
J'ai eu le bonheur de parvenir à décomposer les alkalis fixes, qui auparavant étoient regardés comme des substances élémentaires. Ils sont composés d'air pur et de substances métalliques très-inflammables; mais il n'y a pas lieu de penser que les alkalis se réduisent dans le procédé de la végétation à leurs élémens les plus simples.
Je traiterai au long le sujet important de la chaux comme engrais, et je pourrai offrir à cet égard des vues nouvelles.
Les Romains employoient la chaux éteinte\setcounter{page}{348} pour amender le sol dans lequel croissent les arbres; c'est Pline qui nous l'apprend. Dès les temps les plus anciens, les Bretons et les Gaulois ont employé la marne pour amender le terrain en la répandant à la surface; mais je ne crois pas que l'on connoisse l'époque à laquelle la chaux calcinée a été introduite généralement comme engrais pour les champs. C'est sans doute après l'avoir essayée dans les jardins qu'on l'aura employée dans l'agriculture; et il est naturel de penser qu'on aura appliqué la chaux calcinée, en remplacement de la marne, là où celle-ci ne se trouve pas.
Les anciens auteurs agronomiques n'avoient que des notions fausses sur la nature et les effets de la pierre calcaire et de la marne, et cela étoit la conséquence de l'imperfection de la chimie de leur temps. Les alchimistes considéroient la matière calcaire, comme une terre particulière, qui dans le feu se combinoit avec un acide inflammable. Evelyn, Hartlib, et même Lisle, dans leurs ouvrages d'agriculture, n'ont parlé de la substance calcaire calcinée que comme d'un engrais chaud qu'il falloit employer dans les terres froides. C'est au Dr. Black d'Edimbourg que nous devons les premières connoissances distinctes sur ce sujet. Ce professeur célèbre prouva par les expériences les\setcounter{page}{349} plus décisives, faites en 1755, que la pierre à chaux, et toutes ses modifications, comme les marbres, les craies et les marnes, étoient principalement composés d'une terre particulière et d'un acide aérien, que cet acide s'échappe dans la combustion, faisant perdre ainsi à la pierre quarante pour cent de son poids, et lui donnant le caractère de la causticité.
Ces faits importans expliquèrent les usages de la chaux, soit dans le mortier, soit comme engrais. Comme mortier la chaux employée caustique acquiert de la dureté et devient durable, en absorbant l'acide aérien ou, comme on l'a appelé depuis, l'acide carbonique, qu'elle attire de l'atmosphère, dans laquelle il existe toujours en certaine quantité: la chaux redevient ainsi, en quelque sorte, une pierre calcaire.
Les craies, les marnes calcaires, et les pierres à chaux pulvérisées agissent seulement comme formant un ingrédient utile dans le sol; car la substance calcaire paroît être essentielle à la fertilité d'un terrain: l'utilité de cette addition est donc proportionnée au déficit de la matière calcaire dans le sol.
Le premier effet de la chaux caustique est de décomposer les substances animales et végétales, et de les amener à un état qui\setcounter{page}{350} hâte leur transmutation en matières végétales. Cependant la chaux caustique est peu-à-peu neutralisée par l'acide carbonique, et convertie en une substance analogue à la craie; mais alors elle se mêle plus parfaitement avec les autres ingrédients du sol; elle se divise plus complètement; et est probablement plus utile au terrain que toute autre substance calcaire dans son état naturel.
C'est à Mr. Tennant que l'on doit le fait le plus remarquable concernant la chaux, qui ait été observé depuis plusieurs années. On avoit remarqué depuis long-temps que certaines chaux du nord de l'Angleterre, au lieu d'amender le sol, le rendoient stérile. Mr. Tennant en 1800, s'assura par un examen chimique que cette pierre calcaire contenoit de la magnésie, terre qu'il démontra être nuisible à la végétation, lorsqu'elle est dans son état caustique, et employée en quantité considérable. La chaux qui provient de cette même terre, est cependant appliquée avec succès sur les terres fertiles du Leicester-shire et du Derby-shire, mais en quantité modérée. Il paroît que lorsque la magnésie est unie au gaz acide carbonique, elle n'est pas nuisible à la végétation, et dans les terres riches en fumier, elle absorbe promptement ce gaz qui se dégage dans la décomposition putride.\setcounter{page}{351} Après avoir discuté la nature et la manière d'opérer des engrais, nous considérerons quelques-uns des procédés de l'agriculture qui peuvent être éclaircis par la chimie.
La théorie chimique des jachères est fort simple. La jachère ne produit aucune source de richesse pour le sol: elle tend seulement à accumuler les matières décomposées qui auroient été employées par la récolte s'il en avoit eu une; et il est presque impossible d'imaginer un cas où une terre cultivée eût pu demeurer un an en jachère, avec avantage pour le fermier. Le seul rapport sous lequel cette pratique soit utile, c'est la destruction des mauvaises herbes, ou le nettoiement de la terre.
Je discuterai en détail la théorie chimique de l'écobuage. Il est évident que cette opération doit détruire une certaine quantité de matières végétales, et qu'elle est principalement utile dans les cas où il y a excès de cette matière. Le brûlement rend aussi les glaises moins cohérentes, plus perméables à l'eau, et par conséquent plus propres à la végétation.
Les cas dans lesquels l'écobuage est évidemment nuisible sont ceux où le terrain, formé en grande partie, d'un sable siliceux, contient peu de matière animale et végétale.\setcounter{page}{352} Le brûlement décompose alors ce qui rendoit le sol productif.
Les avantages de l'irrigation, qui en dernier lieu, ont été beaucoup discutés, étoient bien connus des anciens, et lord Bacon la recommandoit il y a deux siècles à nos premiers Anglais. L'arrose-ment des prairies, disoit cet illustre auteur, n'agit pas seulement en fournissant de l'humidité aux plantes, mais en leur communiquant des principes nourrissans que l'eau tient en dissolution, et en défendant leurs racines contre les effets du froid.
On ne peut pas établir des principes généraux, et comparer le mérite des divers systèmes de culture et de rotations, adoptés dans les différens districts, sans connoître la nature chimique du sol, et ses circonstances physiques. Les sols roides et cohérens profitent le mieux des effets de la division, et de l'exposition à l'air : ces effets sont produits complétement dans le système de la culture au semoir et des sarc\underline{l}ages pendant la végétation ; mais dans certains districts, le travail et les dépenses peuvent bien ne pas être compensés par les avantages obtenus. Les climats humides sont les mieux appropriés aux prés artificiels, à l'avoine, et aux plantes à feuilles larges ; les terrains\setcounter{page}{353} aluminueux et roides sont en général, les plus propres aux récoltes de froment; et les terrains calcaires produisent de beaux sainfoins et de beaux trèfles.
L'agriculture n'a plus besoin que d'expériences dont toutes les circonstances soient enregistrées avec soin et d'une manière scientifique. L'agriculture avancera à proportion de ce que les méthodes deviendront exactes. Il faut, comme dans les recherches physiques, y tenir compte de toutes les circonstances: les résultats peuvent être affectés par six lignes d'eau de plus ou de moins qui tombent dans l'année, par quelques degrés de température, par une légère différence dans la nature du sol inférieur, et par l'inclinaison du terrain.
Les informations recueillies avec un esprit de recherche exacte, se lient mieux aux principes généraux de la science; et quelques résultats d'expériences vraiment philosophiques en chimie agricole, éclaireroient mieux les agriculteurs et leur seroient plus utiles qu'une grande accumulation d'expériences imparfaites, et conduites sans méthode. Il n'est pas rare d'entendre les gens qui sont prévenus en faveur de la pratique et de l'expérience, condamner généralement toute tentative pour perfectionner l'agriculture par\setcounter{page}{354} des recherches philosophiques et chimiques. Il n'est pas douteux qu'on ne trouve beaucoup de spéculations vagues dans les ouvrages de ceux qui abusent des expressions chimiques. On voit souvent employer en phrases bannales, les mots techniques d'oxigène, d'hydrogène, de carbone et d'azote, comme si la science était dans les mots et non dans les choses. C'est une raison pour étudier plus exactement la chimie agricole. Il est impossible de raisonner agriculture sans avoir recours à la chimie : on ne saurait faire un pas sans cette science ; et si l'on se contente d'énoncer des vues imparfaites, ce n'est pas parce qu'on les préfère à la science positive, mais parce que cela est plus facile. Pour le voyageur de nuit, le meilleur moyen d'éviter de s'égarer en suivant des lueurs trompeuses, c'est de porter un flambeau dans sa main. On a dit, et probablement avec raison, qu'un chimiste ferait de mauvaises affaires s'il voulait pratiquer l'agriculture. S'il n'était que chimiste, il n'y a pas de doute ; mais il est probable qu'il réussirait mieux que celui qui n'en saurait pas plus sur l'agriculture et qui n'aurait aucune notion de chimie. Mais la chimie n'est pas la seule connaissance nécessaire à un agriculteur : elle est seulement une partie importante de la base\setcounter{page}{355} De cette science ; et lorsqu'on en fait un usage convenable, elle doit produire de bons effets.
A mesure que la science avance, les principes deviennent moins compliqués, et par conséquent plus utiles : c'est alors que l'application s'en fait le plus avantageusement aux arts. Le simple laboureur ne sauroit être éclairé par les préceptes généraux de la philosophie; mais lorsque l'utilité d'une pratique qui lui est démontrée, il ne refusera pas d'en faire usage par la raison que cette pratique est fondée sur les principes de la philosophie. Le matelot se fie à la boussole, quoiqu'il ignore les découvertes de Gilbert sur le magnétisme, et les principes développés par le génie d'Epinus. Le teinturier emploie avec confiance dans le blanchiment des toiles, une liqueur dont les effets lui sont connus, quoiqu'il ignore jusqu'au nom de la substance dont ces effets dépendent. Le grand but des recherches chimiques relativement à l'agriculture, doit être sans doute, de perfectionner l'art de faire valoir la terre; mais les principes généraux de la science, et les connoissances pratiques y sont également nécessaires : les germes des découvertes utiles se trouvent souvent dans les spéculations de la théorie; et l'industrie n'est jamais plus\setcounter{page}{356} efficace que lorsqu'elle est secondée de la science.
Les principes des perfectionnemens doivent être propagés par les propriétaires du sol, par la première classe de la société, que son éducation rend capable de former des projets éclairés, et qui est assez riche pour les mettre à exécution; mais dans tous les cas, le bénéfice est réciproque; car ce qui est l'intérêt des fermiers est aussi celui des propriétaires. L'attention des domestiques et des ouvriers est plus soutenue, lorsque l'inspection du maître est éclairée. L'ignorance du propriétaire conduit presque nécessairement à la négligence des fermiers ou de l'économe.
C'est une erreur de croire qu'il faille employer beaucoup de temps, et acquérir une connoissance approfondie de la chimie générale, pour pouvoir faire les expériences nécessaires sur la nature des terrains et les propriétés des engrais. Rien de plus facile que de voir si une terre fait effervescence quand on change de couleur avec les acides, ou si elle brûle quand on la chauffe, ou ce qu'elle perd par l'incandescence: cependant ces simples indications peuvent être d'une grande importance pour un système de culture. Les dépenses nécessaires pour les re-\setcounter{page}{357} cherches chimiques d'un agriculteur, sont fort peu de chose. Un petit cabinet suffit à contenir tous les instrumens et les matériaux indispensables. Les expériences les plus intéressantes peuvent se faire avec un petit appareil portatif : quelques phioles, quelques acides, une lampe et un creuset : voilà tout le matériel nécessaire, ainsi que je le montrerai dans le cours des leçons.
Il arrive souvent, je le sais, que des expériences agricoles tentées d'après la théorie chimique la plus éclairée, manquent, ou donnent un résultat négatif. Cela doit arriver, vu la nature incertaine et capricieuse des causes dont l'action agit pour procurer le résultat ; et attendu l'impossibilité de calculer toutes les circonstances qui peuvent intervenir ; mais cela ne prouve nullement l'inutilité de telles expériences. Un seul résultat heureux qui peut améliorer d'une manière générale les méthodes de culture, vaut le travail de toute une vie ; et une expérience manquée, si les circonstances en sont bien observées doit établir quelques vérités ou tendre à écarter des préjugés.
A ne considérer la chimie agricole que comme une science naturelle, elle vaut assurément la peine d'être étudiée ; car que peut-il y avoir de plus intéressant que de\setcounter{page}{358} rechercher les formes, l'application et le but final des êtres organisés? D'examiner la marche et les variations de la matière inerte dans la suite des changemens qu'elle subit jusqu'au moment où elle atteint sa destination la plus relevée, savoir, de devenir utile à l'homme.
Diverses sciences sont suivies avec ardeur, et considérées comme des objets convenables d'étude pour les esprits éclairés, et cela uniquement à cause des plaisirs intellectuels dont cette étude est la source, à cause de l'extension qu'elle donne à nos vues, et de l'appréciation plus exacte des objets et des êtres avec lesquels nous sommes en rapport. Combien n'est pas plus digne de notre attention cette étude dans laquelle le plaisir résultant de l'amour de la vérité et de la science est aussi grand que dans toute autre branche de la philosophie, et se trouve lié avec des avantages beaucoup plus étendus: "Nihil est agricultura melius, nihil uberius, nihil homine libero dignius."
Les découvertes faites en agriculture ne sont pas seulement utiles au temps et au pays de ceux qui les font: elles influent sur les siècles à venir; elles sont un bienfait pour toute la race humaine; elles tendent à faciliter la subsistance des générations futures,\setcounter{page}{359} \section{DE LA CHARRUE BELGE. Par Ch. PICTET.}
J'AI donné dans le quatorzième volume d'agriculture (n°. de juin 1809) le dessin et la description détaillée de la charrue Belge, d'après l'excellent observateur et praticien Mr. SCHWERZ. Frappé des avantages que cet auteur attribuoit à la charrue qu'il décrivoit, je pensai à la faire venir de la Belgique ; mais comme j'avois éprouvé l'extrême difficulté qu'il y a à faire réussir, puis adopter par les domestiques , une charrue étrangère quelconque ; comme j'avois fait venir les charrues de Yorkshire , de Small , du Piémont, et de Guillaume, sans aucun succès, je résolus de me procurer un attelage complet, avec un laboureur, pris dans la
\setcounter{page}{359}
\section{DE LA CHARRUE BELGE. Par Ch. PICTET.}
J'AI donné dans le quatorzième volume d'agriculture (n°. de juin 1809) le dessin et la description détaillée de la charrue Belge, d'après l'excellent observateur et praticien Mr. SCHWERZ. Frappé des avantages que cet auteur attribuoit à la charrue qu'il décrivoit, je pensai à la faire venir de la Belgique ; mais comme j'avois éprouvé l'extrême difficulté qu'il y a à faire réussir, puis adopter par les domestiques , une charrue étrangère quelconque ; comme j'avois fait venir les charrues de Yorkshire , de Small , du Piémont, et de Guillaume, sans aucun succès, je résolus de me procurer un attelage complet, avec un laboureur, pris dans la\setcounter{page}{360} partie de la Belgique où l'on laboure avec le plus de perfection. Mr. Schwerz, qui voulut bien se charger de ma commission, ne réussit pas à l'exécuter, parce que la conscription mettoit obstacle à l'engagement d'un jeune homme au-dessous de vingt ans, et qu'il ne put déterminer un homme fait, à se dépayser. Je m'associai alors avec Mr. Crud pour faire venir une charrue. Il fut si content de l'instrument qu'il désira s'en charger seul et s'occupa immédiatement d'en faire faire de bonnes copies. Quelques-unes réussirent: le plus grand nombre furent fautives, et toutes étoient fort chères (1). Quelques essais que je fis, avec des charrues mal imitées, ne me satisfirent point. Cependant, au mois d'avril de cette année, j'eus occasion de voir l'ouvrage de la charrue Belge dans un grand domaine de terres argileuses appartenant à Mr. de Loys à Prevessin, où ces instrumens sont employés exclusivement.
\footnote{Je viens de voir, à Hofwyl, une de ces charrues mal imitées d'un bon modèle, et dont Mr. Fellenberg n'a pu faire aucun usage. Nos lecteurs ont vu néanmoins dans le Mémoire sur l'agriculture Bolognaise, que Mr. Crud a introduit, dans le voisinage de Bologne l'usage de la charrue Belge, et qu'il s'en trouve très-bien.}
Je\setcounter{page}{361} Je questionnai les laboureurs. Ils s'accordèrent à vanter cette charrue par comparaison avec la charrue ordinaire du pays. Le travail du labourage me parut très-bien fait; et Mr. Chappuis l'intendant de Prevessin, homme fort intelligent et exact, me dit qu'il trouvoit dans l'emploi de la charrue Belge, un avantage considérable relativement à celle du pays, quant à la quantité d'ouvrage qu'elle faisoit et à la qualité du travail; mais qu'elle étoit chère, d'un entretien assez couteux; par le renouvellement des lames du soc, et qu'il y avoit toujours un peu de hasard dans la réussite de sa construction.
Mr. Vernet, qui cultive un domaine de terres argileuses, avoit acheté, l'année derrière, une charrue venue de Belgique, et dont le propriétaire n'avoit pas su se servir. Il étoit fort content de cet instrument, et m'en avoit parlé plusieurs fois avec éloge. Le soc se trouvant usé, il envoya la charrue chez Charles Machet, le charron-maréchal de Lancy, connu par ses semoirs et ses cultivateurs. J'essayai cette charrue dans un trèfle à rompre, et je fus frappé de l'avantage qu'elle avoit sur toutes celles de la Belgique que j'avois essayées ou vu marcher. Mr. Vernet voulut bien la donner pour modèle au charron. Agric. Vol. 18. No. 9. Sept. 1813. E e.\setcounter{page}{362} ron, lequel consentit à faire de pareilles charrues pour quatre-vingt-seize francs, à condition qu'on lui assurât la vente de six de ces instrumens à la fois. J'ai surveillé la construction des deux premières de ces charrues, que j'ai prises pour mon compte. Elles sont très-exactement imitées du modèle, et je les ai employées à faire tous mes labours de semailles.
La perfection de la charrue est un objet d'une si haute importance que je crois utile de ne point perdre de temps pour fixer l'attention des agriculteurs sur cette charrue Belge. Telle qu'elle est maintenant exécutée par Charles Machet, je n'hésite point à la mettre au-dessus de toutes les charrues à moi connues\footnote{Je dis telle qu'elle est maintenant exécutée par Charles Machet ; car la moindre différence peut changer les résultats. Par exemple, elle a une singularité de construction frappante. Le prolongement inférieur du manche s'assemble dans le sep en inclinant à gauche de huit degrés de la verticale. Il en résulte une déviation à gauche pour l'âge ou la flèche, de manière que la verticale s'élevant de la pointe du soc passe à droite de la flèche. Apparemment l'expérience a appris que le centre de résistance se trouvoit ainsi mieux dans la ligne du tirage, et que par conséquent la charrue en étoit plus ferme dans la raie.}. Il faut quelques détails pour faire\setcounter{page}{363} comprendre en quoi consiste cette supériorité que je lui attribue, principalement sur notre charrue du pays. Pour les lecteurs qui n'ont pas sous la main le quatorzième volume d'Agriculture de ce recueil, où se trouve la gravure de cet instrument, d'après Mr. Schwerz, je rappelle que cette charrue est sans avant-train : un support mobile, qui glisse sur le terrain, en tient lieu. Le soc et l'oreille fixés à droite, sont en fer battu, et forment un ensemble contourné en hélice. Le soc est armé d'une aile d'acier ou lame, dont le tranchant se projette à droite de trois pouces et demi, en formant un angle de quarante degrés avec la face gauche du sep. Il résulte des dimensions du soc et de sa marche oblique, qu'il tranche horizontalement la terre dans une largeur d'un pied. Comme cette charrue jette toujours la terre à droite, on ne peut labourer qu'en à-dos. On attelle les animaux de trait à l'extrémité de l'age, à un têtard en fer, pourvu de plusieurs trous, afin de changer à volonté la direction du sep, pour certains cas particuliers.
Les charrues qui viennent de Belgique sont à un seul manche, derrière lequel il y a un petit mancheron ou poignée, pour la faci-\setcounter{page}{364} lité du conducteur. J'ai fait mettre deux manches à celles que nous avons imitées, parce que nos laboureurs sont accoutumés à cette forme, qui au fond, est assez indifférente, car la charrue a beaucoup de stabilité dans la raie; et comme l'a observé Mr. Schwerz, deux doigts suffisent à la diriger.
La marche de cette charrue a deux caractères qui lui sont propres : le premier c'est de prendre, sans inconvénient pour la qualité du travail, une tranche d'un pied à quatorze pouces de large; le second trait caractéristique de sa manière de travailler la terre; c'est qu'elle fait monter la tranche toute entière contre l'oreille, qui la renverse ensuite obliquement, et sans dessus dessous, sans la pousser sensiblement. Je vais tâcher de me faire comprendre.
Dans la marche de la charrue du pays, dont le soc est un coin étroit, lequel fait son chemin comme une taupe, la terre tranchée verticalement par le coître, est déchirée dans toute la portion qui sépare le soc de la raie ouverte; et comme le soulèvement graduellement opéré par le soc précède toujours le déchirement dont il s'agit, il en résulte une torsion spirale de la tranche, laquelle tourne comme sur une\setcounter{page}{365} charnière, et est ensuite poussée de l'oreille, qui l'applique de force contre la tranche précédemment retournée, nettoyant ainsi la raie à la largeur de seize à dix-sept pouces, qui est l'écartement ordinaire des oreilles dans leur partie postérieure et inférieure. Je suppose que la terre ait le degré d'adhérence nécessaire pour que les effets décrits soient bien sensibles : c'est le cas lorsqu'on rompt un gazon ou un trèfle en terre glaise humide.
Comparons maintenant avec ce procédé la manière dont la charrue belge opère le déplacement et le renversement de la terre. La lame du soc, qui est en acier et tranchante, agit comme un plan incliné, dont l'angle est de sept degrés seulement. Dans la partie antérieure du soc, cette ouverture de l'angle se prolonge jusqu'à sept pouces de la pointe, ensorte que le soc s'insinue facilement par dessous, et fait glisser sur lui, sans beaucoup d'effort, la tranche déjà détachée par le coutre. Supposons que la charrue rompant un gazon en terre humide, prenne un pied de large, six pouces de profond, et qu'elle ait cheminé de vingt-huit pouces, à côté d'une raie déjà ouverte. Suspendons un instant sa marche pour porter notre attention sur la masse de terre de\setcounter{page}{366} vingt-huit pouces de long, sur douze de large, et six de haut, qui est à la droite de la charrue. Cette masse peut se considérer comme deux parallélipipèdes égaux, de quatorze pouces de long. Appelons n°. 1, celui qui répond au soc, et n°. 2, celui qui répond à l'oreille de la charrue. Le n°. 1 n'est encore qu'à demi détaché par dessous : le tranchant de la lame du soc s'est arrêtée sur la diagonale de la face inférieure de ce parallélipipède. Son angle postérieur et inférieur de gauche est déjà soulevé de cinq pouces au-dessus du plan qui marque le fond de la raie : ce n°. 1 éprouve donc un premier degré de torsion à droite. Le n°. 2 est complétement détaché. Il a monté en glissant contre l'oreille, déclinant et se tordant à droite, jusqu'à-ce qu'enfin il se soit dressé presque verticalement sur son extrémité antérieure et sur-tout sur son angle de droite. Il est prêt à tomber quand il dépassera la verticale, et que son adhérence avec le n°. 1 sera rompue par son poids. Si la charrue chemine d'un pouce en avant, ce n°. 2 tombe obliquement à droite, le gazon dessous, dans une direction qui est assez exactement à quarante-cinq degrés de la ligne du tirage. On se représente que ce parallélipipède n'est pas régulier : il est écorné,\setcounter{page}{367} peut-être déchiré obliquement, et plus semblable à un rhomboïde. Il demeure appuyé, par son extrémité supérieure, sur le terrain déjà labouré, tandis que son autre extrémité repose sur le fond de la raie : il est donc incliné à l'horizon, sous un angle qui varie selon la profondeur du labour. Il forme un peu la voûte, parce qu'il conserve en partie la torsion que lui a imprimée le versoir. Il n'éprouve de celui-ci aucune pression, car la largeur totale de l'oreille, dans le bas, à compter de la gauche du sep, n'étant que de douze pouces, elle ne pousse point latéralement. Dans le haut, son évasement de treize pouces et demi suffit à déterminer la chute de la masse de terre, mais sans frottement fatigant pour le tirage, ou qui puisse durcir la terre. Le gazon est donc caché, sans être écrasé. Le guéret est comme soulevé et soufflé. Il semble avoir été fait par un labour à la bêche, dans lequel les masses seroient plus grosses qu'elles ne le sont d'ordinaire et arrondies par dessus comme des tuiles courbes\footnote{Je n'ai pas besoin de dire que lorsque la terre est meuble et pulvérulente, elle se brise irrégulièrement, et que les observations ci-dessus ne sont point applicables. Si, au contraire, il s'agit d'un vieux gazon fortement lié, les rhomboïdes ont souvent jusqu'à trois pieds de long.}.\setcounter{page}{368} Si la charrue du pays travaille sur une terre argileuse et humide, elle retourne une bande qui tient ensemble d'un bout de la raie à l'autre, et qui forme ce que nos laboureurs appellent une latte. Cette bande est comme pétrie et lissée de trois côtés, savoir, par la pression de bas en haut opérée par le soc, dans l'action du déchirement, et ensuite par le frottement dur de la partie postérieure de l'oreille. Or, cette même tranche étoit déjà lissée par le trait de charrue précédent, ainsi que nous le verrons tout-à-l'heure. Cette triple pression fait, en quelque sorte, du pisé de cette terre lattée; et le guéret ne dépasse pas sensiblement le niveau de la terre non labourée. Avec la charrue belge, au contraire, lors même qu'il s'agit d'un sol argileux et humide, la terre n'est jamais lattée. L'oreille n'exerce aucune pression sensible sur les tranches renversées. Sa partie postérieure les pousse très-légèrement, et seulement ce qu'il en faut pour décider le renversement des petites masses successives, ainsi que je viens de l'expliquer. Le guéret demeure de trois à quatre pouces plus élevé que la terre non labourée, et aucune partie du terrain remué n'a éprouvé une pression qui la convertisse en pisé. Il\setcounter{page}{369} résulte de ces deux manières d'opérer, que le guérêt formé par la charrue du pays, se durcit comme de la pierre, s'il succède un temps chaud et sec, au lieu que l'autre, dans les mêmes circonstances de température, est disposé à s'émietter plus facilement, et profite mieux des influences atmosphériques. J'ai dit que le soc de la charrue belge tranchoit une largeur de douze pouces. La partie inférieure du versoir de cette charrue, a précisément la même largeur d'un pied, à compter depuis le bord du sep du côté gauche. Cette partie inférieure du versoir, est de deux pouces au-dessus du niveau du talon du sep. Dans la charrue du pays, le soc n'a que deux pouces de large à sa partie antérieure, et quatre dans sa partie postérieure; en revanche, l'écartement de l'oreille est de seize à dix-sept pouces, et la partie inférieure de celle-ci est au niveau du talon du sep. Il en résulte qu'une grande partie de la force est employée dans le poussement latéral de la bande de terre, par l'oreille. La réaction de ce poussement latéral a lieu contre la terre non entamée, et occasionne un frottement d'autant plus grand, que le sep et l'oreille sont très longs, que celle-ci ne s'applique pas complétement contre le
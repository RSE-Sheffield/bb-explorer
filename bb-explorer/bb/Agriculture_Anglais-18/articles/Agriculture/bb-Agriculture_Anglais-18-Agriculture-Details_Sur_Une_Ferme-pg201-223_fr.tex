\setcounter{page}{201} \section{AGRICULTURE.}
SOME PARTICULARS, etc. Quelques détails sur la ferme exploitée par J. C. CURWEN, de Workington - Hall, en Cumberland, recueillis en septembre 1812, à l'assemblée de Workington, avec quelques observations sur les améliorations dont le système agricole de Mr. Curwen est susceptible, par Sir JOHN SINCLAIR.
L'ÉTENDUE totale de l'exploitation de Mr. Curwen monte à neuf cent soixante-treize acres, dont cinquante sont une bonne terre végétale reposant sur le gravier, et tout le reste une terre argileuse reposant sur une couche imperméable à l'eau. Cela paroît une beaucoup trop grande étendue pour les forces d'un seul individu, sur-tout en ayant égard à la qualité du terrain. Il faudroit au moins que l'habitation fût placée au centre, ou qu'il y eût deux établissemens dans des situations convenables, pour faciliter l'inspection et l'exploitation \footnote{Depuis l'assemblée de Workington, Mr. Curwen s'est décidé, à cause de la difficulté des transports des Agricult. Vol. 18. N°. 6. Juin 1813. R}.\setcounter{page}{202} La rotation adoptée est froment, et récolte verte, à moins qu'il ne s'agisse de terrains nouvellement défrichés, auquel cas l'avoine est aussi employée. On a remarqué que le blé ne réussit pas bien sur le trèfle qui a été coupé deux fois, à moins qu'on ne fume avec du fumier ou du compost, et qu'on ne sème le froment avant le milieu d'octobre. Voici l'état de la culture en 1812.
\comment{table}
Froment\footnote{Vingt-huit acres en froment rouge ont rendu, cette année, à raison de vingt-huit bushels par acre, à soixante livres pesant le bushel. Une autre sorte de blé a rendu trente-deux bushels. Mr. C. trouve que le blé rouge est le meilleur chez lui. (A)} . . . . . . . . . . . . 350 acres.
Avoine sur des défrichemens . . . 73
Turneps blancs . . . . . . . . . 60
Rutabaga . . . . . . . . . . . . 20
Colza . . . . . . . . . . . . . . 30
Pommes de terre . . . . . . . . . 50
Choux . . . . . . . . . . . . . . 10
Carottes . . . . . . . . . . . . . 10
Jachère . . . . . . . . . . . . . . 40
Récoltes labourées . . . . . . . . 643 acres.
récoltes, à affermer deux cent vingt acres à un fermier actif, lequel s'est engagé à alterner entre les récoltes céréales, et les récoltes vertes, ou la jachère. Il paie cinquante-cinq schelings l'acre, quoique en 1806, cette terre ne fût affermée que douze schelings. (A)\setcounter{page}{203} \section{DÉTAILS SUR UNE FERME}
Trèfle . . . . . . . . 250
Prés arrosés . . . . . . 40 } 330
Pâturages permanens . . . 40
Total 973 acres.
Un intelligent agriculteur Écossais a remarqué, sur l'exposé de l'occupation des terres, que dans un sol de cette nature, on ne pouvoit pas compter aussi bien sur les récoltes-jachères, pour le nettoyage du terrain, que sur la jachère morte. Il croit que la rotation suivante de dix ans seroit plus productive.
150 acres en jachère morte.
150 froment après jachère.
150 Trèfle.
150 Avoine.
150 Fèves, pommes de terre, turneps et colza.
150 Blé ou orge.
900
Si le fumier peut s'acheter aisément, il proposeroit une rotation de huit ans, savoir, 1°. jachère ; 2°. blé ; 3°. trèfle ; 4°. avoine ; 5°. fèves, pommes de terre, turneps ou colza ; 6°. orge ; 7°. fèves ; 8°. blé.
Les pâturages permanens sont la partie la moins productive de toute la ferme ; mais
R 2\setcounter{page}{204} on les conserve, soit par l'idée qu'il y a de la convenance, soit parce que c'est une plaine où la milice du canton s'exerce ordinairement. Mr. Curwen n'en est pas moins convaincu que c'est une grande faute de conserver en pâturage le terrain qui avoisine les bâtiments de ferme. Il calcule la dépense de charier une récolte verte qui est à un mille de la ferme, à raison de trois livres sterling par acre, et à six livres sterling si la distance est de deux milles. D'après ce calcul, Mr. Curwen a rompu vingt-cinq acres de prés qui se touchaient, et a mis en pré une même étendue, qui est fort éloignée. Il a résolu d'affermer toutes les parties qui sont trop loin de ses bâtiments.
Il y a une perte considérable à ne pouvoir faire manger aux moutons les turneps sur place; et c'est un grand désavantage comparatif des terres argileuses. Il trouve du profit à faire manger aux moutons ses jeunes trèfles, en ayant soin de ne les pas laisser brouter trop ras.
Les bestiaux entretenus sur la ferme sont:
34 vaches à lait.
6 génisses.
100 moutons de Southdown.
20 bœufs à l'engrais.
100 chevaux de travail.
30 bœufs de travail.\setcounter{page}{205} Comme les chevaux sont aussi employés au chariage du charbon de terre, il est impossible de faire exactement le compte des dépenses et du bénéfice de l'exploitation agricole. Mr. Curwen calcule que dans un sol aussi argileux et aussi fréquemment labouré que celui de sa ferme, une paire de chevaux ne peut suffire qu'aux labours de vingt-cinq acres. Il estime que trois bœufs répondent à deux chevaux. Cela suppose un grand nombre d'animaux de labour et une dépense bien considérable, laquelle dépend du système de culture de Mr. Curwen. Mr. Brown de Markle, en East-Lothian, a démontré qu'une paire de chevaux, convenablement nourris et employés, peut suffire à cinquante acres anglais, avec un assolement de six ans, savoir, dix acres en turneps ou en jachère, dix en blé ou orge, avec trêfle, dix en avoine après le trèfle, dix en fèves, et dix en blé. Si dans le système de culture de Mr. Curwen, il faut dix charrues de plus, et que l'on compte chaque charrue avec deux chevaux et un conducteur à cent liv. sterling par an, (ce qui est la supposition la plus modérée), voilà mille livres sterling de dépense de plus. Mr. Brown estime la dépense annuelle d'une charrue à cent trente-cinq livres sterling: à ce taux, dix charrues\setcounter{page}{206} coûteroient treize cent cinquante livres sterl. annuellement.
Tous les animaux sont nourris en vert à l'écurie, toute l'année, savoir, en été avec du trèfle, et en hiver avec des turneps et pommes de terre mêlées à de la paillée hâchée et cuites à la vapeur. Chaque bête a aussi en hiver, quatre livres de foin cuit à la vapeur, ou quatre livres de gâteau de colza pilé et mêlé avec la paillée hachée après qu'elle est cuite.
Des instrumens employés sont, la charrue de Small, les semoirs de Mac-Dougall, le râtissoir de Gregg, les charrettes ou tombe-reaux à un cheval, et beaucoup d'autres instrumens qui sont connus ailleurs. Il a une machine à battre très-énergique, et dont le vent est le moteur. Elle est de la construc-tion de Dun, de Coldstream. Lorsque le vent est suffisant, cette machine bat de quarante à cinquante bushels par hectare.
Les champs que l'on destine à des récoltes vertes se labourent de dix à quatorze pouces de profondeur, avec la charrue de Small, et quatre bœufs. Le labour se fait, s'il est possible, avant Noël; après quoi on fait les rigoles d'écoulement, et on a soin que le terrain soit parfaitement égoutté. Les récoltes vertes sont toujours semées en lignes, et sarclées deux fois.\setcounter{page}{207} Si la saison le permet, on arrache et charrie les turneps, que l'on dépose en provision à la ferme centrale, et on sème le terrain en blé.
On emploie, pour les travaux, des ouvriers, loués à la journée; ce qui, à tout prendre, n'est pas un système aussi avantageux que d'avoir des domestiques à gages; mais cela doit dépendre de l'usage du pays et des habitudes du peuple. Les laboureurs ont quinze shellings par semaine, les ouvriers ordinaires deux shellings par jour, et les femmes un shelling. Les enfans gagnent de six à huit deniers (douze à quatorze sous de France) pour les sarclages. Pendant la moisson, les femmes gagnent d'un shelling et demi à deux shellings par jour.
La dépense annuelle de la ferme monte à six mille livres sterling, soit six livres sterling par acre. Il peut y avoir mille livres sterling qui sont appliquées aux opérations du charbon de terre; mais comme on ne peut pas le savoir au juste, le compte exact de l'exploitation agricole ne peut pas se faire.
La quantité de fumier qu'on répand annuellement sur la ferme, s'élève à quatorze mille cinq cent vingt tombereaux à un cheval, chaque tombereau contenant six quintaux et trois quarts de fumier. On peut être\setcounter{page}{208} étonné que l’on charge si peu les chevaux, tandis qu’un fort cheval mène très-bien, dans une bonne route, de douze à vingt quintaux pesant; mais Mr. Curwen estime que huit quintaux de fumier est le plus qu’on doive donner à trainer à un cheval sur les terrains labourés: or, c’est sur les terres labourées que le fumier se mène pour pouvoir le placer convenablement (1)\footnote{C’est une erreur de croire qu’il convienne de mener les fumiers sur les terres labourées. Dans un assolement bien combiné on doit éviter absolument des charriages qui doublent les frais, et fatiguent excessivement les attelages dans le moment de l’année où cela a le plus d’inconvénient. On peut consulter sur ce sujet, le Tableau de l’assolement de douze ans, p. 39 et 40 chez Paschoud, libraire à Genève et à Paris.}. Sur la quantité totale des engrais, il faut compter quatre mille tombereaux de cendres, qui s’achètent à Workington, et qui servent à amender environ cent vingt acres annuellement. Les dix mille cinq cents tombereaux restans sont produits par la ferme, et par les chevaux employés aux charriages du charbon. Il faut encore y comprendre quelques débris de la mine de charbon. L’usage est de transporter le fumier dès qu’il est suffisamment putréfié, et d’en former des tas d’environ cent tombereaux chacun. On les recouvre de terre pour empêcher l’évapo\setcounter{page}{209} ration des gaz pendant la fermentation.
Les urines se déposent soigneusement dans des réservoirs couverts. On les en retire avec une pompe pour arroser les fumiers trois fois par semaine. De cette manière, le fumier est assez fait au bout de six semaines pour être conduit sur les champs. Au moyen de ce que les charriages se font dans la saison convenable, et que les dépôts sont placés dans les champs, il est facile ensuite de distribuer le fumier promptement dans les endroits où l'on veut l'appliquer. Aussitôt que la récolte d'un champ est faite, on estime le fumier qu'il lui faudra pour l'année suivante, et on commence à le charrier, jusqu'à ce que la quantité soit complète. On a ainsi continuellement sous les yeux, la marche de la totalité de l'exploitation.
La laiterie est une partie importante de la ferme de Mr. Curwen. Il paraît que ses vaches ont rendu chacune en moyenne, dans les quatre dernières années, trois mille sept cent trente-neuf quarts de lait, ce qui, à 2 pence le quart, fait 30 liv. st. 3 s. 2 den. L'entretien peut coûter 10 pence par jour soit 15 liv. 4 sols 2 den. par an. L'intérêt, les risques, etc. peuvent s'élever à 3 liv. st. par bête. Chacune rendroit donc par an, et en lait seulement, plus de 12 liv. sterl. sans le veau. Dans ce compte, on a égard aux\setcounter{page}{210} pertes inévitables. L'augmentation du lait a été considérable, par l'introduction de la race à petites cornes.
Mr. Curwen nourrit ses chevaux de travail avec du trèfle vert, en été, et des pommes de terre cuites à la vapeur, en hiver. C'est une grande économie relativement à l'étendue de terrain nécessaire pour nourrir un cheval. Dans le moment des grands travaux, chaque cheval a dix livres d'avoine par jour. On sait qu'en Écosse, dans le temps des forts ouvrages, on donne dix-huit livres d'avoine par cheval, outre seize livres de foin, et souvent des rutabagas au printems.
Les récoltes de trèfle sont d'une abondance qui excite l'admiration de ceux qui les voient. Mr. Curwen attribue cette abondance au système des labours profonds, et des récoltes vertes, qui maintiennent le sol net. Là où ces deux conditions ne sont pas réunies, le trèfle doit souvent manquer\footnote{Dans les commencemens Mr. Curwen a eu souvent des trèfles manqués. Sa seconde coupe dépend beaucoup de la sécheresse ou de l'humidité de la saison. En 1811, la seconde coupe donna deux cent dix quintaux de trèfle vert par acre. Cette année il donne environ la moitié seulement. On coupera pendant tout le mois d'octobre. (A)}. Les grands produits qu'il obtient doivent sans doute être attribués, en partie, à ce que la\setcounter{page}{211} La culture du trèfle est nouvelle dans ces terres. Les récoltes de turneps sont extrêmement considérables. Une récolte de cinquante acres présente cette année une parfaite singularité. Expérience faite, ces racines donnent au 24 septembre, quarante tons ou voitures faisant la charge de deux forts chevaux par acre. Les rutabaga donnent 28 voitures par acre. Mr. Curwen attribue cet immense produit à l'adoption de la méthode de Northumberland et de Lothian, de semer en lignes, et au soin qu'il a d'enterrer son fumier aussi profondément qu'il le peut. Les mêmes moyens lui procurent d'énormes récoltes de choux \footnote{Ce seroit un sujet bien intéressant d'expériences que de déterminer la profondeur à laquelle il convient d'enterrer le fumier pour que son effet soit le plus grand possible sur une récolte donnée, et en même temps assez durable pour que, tout balancé, il y eût un avantage décisif dans la méthode. Si on le répand par dessus le froment en hiver, et qu'on sème du trèfle ou de la luzerne, les pluies font pénétrer peu à peu les sucs de l'engrais à mesure que les racines de ces plantes descendent, et cependant il se fait probablement une perte considérable par le dégagement de gaz en assurant la réussite. L'usage reçu dans le Vurtemberg, et ailleurs, de fumer les pommes de terre par dessus le terrain, et en les plantant, pour ensuite accumuler l'engrais en butant, paroît donner des récoltes extrêmement considérables. On croit généralement qu'il y a de l'avantage}.\setcounter{page}{212} Il y a de grandes difficultés à cultiver les pommes de terre dans un sol si argileux, mais elles sont si nécessaires à l’exécution du plan de Mr. Curwen, pour la nourriture de son bétail, qu’il a fallu surmonter tous les obstacles, et s’assurer d’abondantes récoltes de cette racine. On les plante sur des billons ou planches de quatre pieds de large, et cela de la manière suivante : On commence par ouvrir un sillon avec une charrue à un cheval. On approfondit ensuite ce sillon avec une charrue à quatre bœufs, à double versoïr, et en labourant aussi profond qu’il est possible. On dépose le fumier dans cette tranchée, et on place les pommes de terre dessus, en quinconce, en formant deux lignes distantes d’un pied, et les tubercules à neuf pouces les unes des autres dans la ligne. On couvre ces tubercules par deux traits de charrue. L’intervalle qui sépare cette double rangée de plantes se travaille à la charrue, et on les bute très-haut. Autrefois on enlevoit, par un trait de charrue, la terre des deux à enterrer peu le fumier afin que la descente des sucs au-dessous de la couche de la végétation, par l’effet des pluies, soit moins prompte ; et voilà un habile agriculteur qui s’attache à enterrer son fumier à quatorze pouces pour s’assurer de belles récoltes de turneps et de choux. (R)\setcounter{page}{213} côtés de la double rangée, pour la remplacer par le trait suivant; mais l'intendant, Mr. Thomson, a soupçonné que par cette méthode on diminuoit la récolte, en rompant les petites racines qui sont destinées à porter des tubercules et à nourrir la plante; en conséquence, cette année, on n'a pas fait approcher autant la charrue des rangées, et la récolte est extrêmement abondante. Dix-huit acres ont rendu à-peu-près vingt-huit tones ou charretées par acre, ce qui, à trois pence le stone (14 livres) fait 25 livres sterling de produit brut.
Cette culture est couteuse; mais Mr. Curwen estime que c'est la seule manière d'obtenir d'abondantes récoltes de ces racines sur un sol argileux, et dans son climat. Il n'y a que la nécessité qui puisse faire cultiver une telle quantité de pommes de terre sur un tel sol. Le fumier nécessaire à cinquante acres de pommes de terre suffiroit à cent acres de turneps, et le fumier fourni par ces turneps seroit plus que double de celui que fournissent les pommes de terre \footnote{Voilà une proposition qui paroît étrange. Arrêtons-nous un moment à l'examiner. 1º. On ne voit pas pourquoi l'auteur établit qu'une terre argileuse est plus propre à la culture des turneps qu'à celle des pommes de terre; jusqu'ici on est convenu du contraire. La}.\setcounter{page}{214} Mr. Curwen calcule que la quantité de fumier ainsi absorbé ou produit est de moins par les pommes de terre que par les turneps; méthode de Northumberland, de cultiver les turneps en lignes, méthode reconnue avantageuse et devenue aujourd'hui très-générale en Ecosse et dans le nord de l'Angleterre, donne à la fois plus de facilité pour nettoyer les terres à peu de frais avec les ratissoirs et les houes à cheval, et pour les égouter dans l'automne, ce qui permet de charier la récolte des turneps, au lieu que dans les terres glaises, la chose étoit quelquefois impossible; mais toujours faut-il charier les turneps quand il s'agit de terres argileuses, au lieu de les consommer sur place, procédé infiniment plus économique, et praticable seulement sur les terres légères et sèches. Cette circonstance ôte aux turneps une grande partie de leurs avantages quand il s'agit des terres glaises, sur-tout lorsqu'au lieu de distribuer ces racines sur des prés à fumer ou sur des jachères, où on les fait manger sans autre main-d'œuvre, on les transporte jusqu'aux bâtimens de ferme, pour en faire des magasins; au risque des avaries par gelée ou pourriture, avaries plus fréquentes pour les turneps que pour les pommes de terre. Cependant il paroît que, par la méthode de Mr. Curwen, l'énormité du produit brut couvre tous les inconvéniens de cette culture; dans un tel terrain, mais il reste vrai que, pour un tel sol, les pommes de terre offrent moins de difficultés et d'inconvéniens. 2º. Est-il vrai de dire que le fumier nécessaire à cinquante acres de terre suffiroit à cent acres de turneps? Si, dans la méthode de culture de Mr. Curwen, on fume légèrement les turneps et fortement les pommes de terre, on n'en peut rien conclure contre celles-ci; car, dans les principes de Norfolk, par exemple, terre clas-\setcounter{page}{215} équivaut à tout l'engrais qu'il tire de la ville de Workington.
Pour les récoltes céréales, les sillons ou sique pour la culture des turneps, on les fume aussi fortement qu'aucune autre récolte. Les turneps ne peuvent point se cultiver sans fumier, au lieu qu'on a des récoltes passables et même belles, de pommes de terre, sans y mettre d'engrais, ainsi qu'une longue expérience l'a démontré au rédacteur. 3º. Est-il vrai de dire que le fumier fourni par les turneps de cent acres seroit plus que double que celui que rendent cinquante acres de pommes de terre? En réunissant les expériences les plus probantes, Thaer a trouvé que deux cents livres de pommes de terre équivaloient à cent livres de bon foin, pour la faculté nutritive; et Einhof a estimé qu'il falloit cinq cent vingt-cinq livres de raves ou turneps (non pas de rutabaga) pour répondre à cent livres de foin. Thaer a aussi conclu, en rapprochant les observations et les expériences les plus probantes, qu'un poids donné de foin et paille, mangé ou consommé en litière par les animaux (en supposant assez de paille pour recueillir les urines et les gros excrémens), rendoit un poids double en fumier humide sans l'être assez pour laisser dégouter l'eau. On doit donc supposer que deux cents livres de pommes de terre rendent en fumier autant que cinq cent vingt-cinq livres de turneps, c'est-à-dire, autant que cent livres de bon foin, en admettant que chaque quantité est associée à la paille nécessaire pour maintenir les animaux au sec, et que ceux qui mangent du foin boivent plus d'eau, que ceux qui mangent des pommes de terre, et que ceux qui se nourrissent de pommes de terre boivent plus d'eau que les animaux nourris aux turneps. Si maintenant nous comparons les récoltes de Mr. Curwen entr'elles, nous voyons\setcounter{page}{216} planches ont de neuf à douze pieds de large, et sont suffisamment élevés pour bien égouter les terres. Cette largeur est jugée la plus avantageuse pour semer au semoir.
Mr. Curwen met en valeur les terres neuves, par l'écobuage qui lui coûte 2 liv. sterl. par acre. L'amendement de la chaux et la main-d'œuvre pour étendre les cendres font monter les frais à 4 livres 15 shelings par acre.
La première récolte après l'écobuage est ordinairement du froment, puis des turneps si la distance permet de les charier, puis de l'avoine : il met un peu de fumier à ces deux récoltes.
Mr. Curwen recommande une jachère après les turneps pour le blé, puis trèfle.
Lorsqu'on sème de l'avoine, on remarque une différence prodigieuse entre ce qui a été labouré dans l'automne, et dans l'hiver ;
qu'un acre lui donne quarante charretées de turneps, de deux milliers pesant, équivalentes à 7 voitures de foin de même poids. Un acre de même terrain rend vingt-huit charretées de deux milliers pesant, en pommes de terre, équivalentes à quatorze voitures de foin de même poids. Donc, les deux acres en turneps ne rendent guère en substance nutritive et en fumiers, ce que rend un acre de pommes de terre. Reste à l'avantage de celles-ci, l'économie du terrain, du travail, de l'engrais, du chariage, la plus facile conservation, l'application possible à nourrir l'homme, sans parler de moindres chances qu'offre leur culture. (R)\setcounter{page}{217} \section{DÉTAILS SUR UNE FERME}
l'hiver ; l'avantage est pour ce qui a été le plus tôt labouré. Là où le terrain ne comporte pas la consommation sur place, des turneps, par les moutons, il est évident qu'à une certaine distance de la ferme, par exemple, au-delà d'un demi-mille, il ne peut pas être avantageux de cultiver des récoltes-jachères, et que dans de telles circonstances la jachère morte est la seule ressource \footnote{Pourquoi la jachère morte deviendroit-elle nécessaire dans les champs de terre glaise situés au-delà d'un mille des bâtimens de ferme, c'est-à-dire, à dix minutes de marche ? Une telle distance rend-elle la culture des pommes de terre onéreuse ? Un produit de vingt-huit tons, équivalent à quatorze voitures de vingt quintaux de foin par acre (trente-quatre voitures par hectare) peut-il présenter une perte, par la raison seulement que la distance pour charier est de plus de dix minutes ? La grande distance des pièces est toujours un inconvénient; mais cet inconvénient est encore plus grand dans le système de la jachère, parce qu'il est sans compensation. (R)}.
L'année prochaine, Mr. Curwen aura deux cents acres de plus à défricher pour froment. Il se propose d'employer la charrue à écrouter : il y gagnera beaucoup, soit pour la dépense soit pour le temps.
Quand le prix des grains est élevé, la produit de la ferme est très-considérable.
Agricult. Vol 18. N°. 6, Juin 1813.\setcounter{page}{218} paroit qu'il y a eu cette année 13,370 tas de blé de douze javelles chaque, et 3358 d'avoine. Les causes fondamentales de cet immense produit sont 1°. les labours profonds, qui maintiennent la terre exempte d'humidité. C'est une expérience de l'évêque de Landaff, consignée dans les Essais de Chimie, vol. III. p. 51 qui a donné à Mr. Curwen cette idée; 2°. l'abondance des engrais; 3°. le soin de les enterrer profondément. Par ce procédé, les gaz qui se dégagent du fumier s'appliquent peu à peu aux racines des plantes. Par ce système le froment et les récoltes vertes peuvent se succéder indéfiniment; en donnant des produits abondans, et les pommes de terre et les turneps peuvent être cultivés sur un sol argileux.
Il est probable cependant qu'une partie des trêfles seroit mieux employée si on les faisoit suivre par l'avoine. Au moins que la terre ne soit en très-bon état, et que le blé ne soit semé dans le moment le plus favorable, il sembleroit, d'après le calcul suivant, qu'il y auroit de l'avantage à préférer l'avoine.
Produit du blé.
\comment{table}
Vingt-huit bushels de froment, à 10 shellings....... Liv. st. 14
A déduire pour les semences quatre bushels....... 2
Reste....... Liv. st. 12\setcounter{page}{219} \section{Produit de l'avoine}
Cinquante bushels, à 4 shillings. Liv. st. 10.
A déduire pour les semences cinq bushels. Liv. st. 1.
Reste Liv. st. 9.
Le blé donne 3 livres sterling de plus que l'avoine ; mais dans un terrain en assez bon état pour produire le blé ci-dessus, on auroit pu recueillir soixante bushels d'avoine, qui auroient valu 12 livres sterl. Or, si l'on considère que le terrain seroit moins épuisé par l'avoine, que par le blé, il demeure probable qu'il seroit plus avantageux aux fermiers de préférer l'avoine ; mais on doit considérer la nécessité où se trouve l'Angleterre de produire du blé, en quantité suffisante pour se soustraire à la dépendance ruineuse et humiliante où elle se trouve sous le rapport des subsistances\footnote{Dans un pays qui nourrit un très-grand nombre de chevaux de luxe et où l'usage général est de donner beaucoup d'avoine aux chevaux de travail, le prix de l'avoine est plus élevé qu'il ne devroit l'être relativement à celui du froment. Il en résulte un encouragement continuel à faire entrer l'avoine dans les assolements, de manière qu'une partie des terres qui devroit être employée à nourrir l'homme, l'est à nourrir des chevaux. Cet abus de la culture de l'avoine est sur-tout sensible.}.\setcounter{page}{220} Mr. Curwen se propose d'augmenter l'année prochaine l'étendue de ses jachères. Il pense, que tout combiné, dans un terrain tel que le sien, il conviendroit peut-être de revenir à la jachère tous les six ans.
Il a essayé de semer du blé sur un guéret déjà ancien; et sa récolte a réussi. Il a eu le même succès avec l'avoine. Lorsque ses terres sont bien ameublies après les labours du printems; elles sont sujettes à se durcir
dans les Départemens de la France où l'assolement triennal est encore en usage. On est obligé d'avoir un plus grand nombre de chevaux pour labourer les terres destinées à nourrir les chevaux; et il y a une perte évidente pour la nation dans ce cercle vicieux. Les expériences faites depuis quelques années en Allemagne, en Suisse, et dans quelques parties de la France, démontrent qu'on peut se passer d'avoine pour les chevaux de travail, sans perdre sensiblement sur le travail qu'ils font. Pour suffire à la production des récoltes qui, dans ce système, doivent remplacer l'avoine, il faut établir des assolemens dans lesquels, le trèfle, la luzerne, et les pommes de terre soient combinées de façon à produire d'abondantes récoltes pour l'entretien des animaux en été et en hiver (voyez ce que nous avons dit ci-devant sur la consommation de la luzerne en vert.) Un tel système de culture botteroit la production de l'avoine aux cantons montueux et froids, ou aux défrichemens; aux-quels elle est souvent applicable avec avantage. Elle seroit consommée par les chevaux de cavalerie et les chevaux de luxe. Il en résulteroit un gain incalculable pour l'agriculture en général.(R)\setcounter{page}{221} \section{DÉTAILS SUR UNE FERME}
si fortement que les plantes ont de la peine à lever.
La ferme de Mr. Curwen réunit divers avantages : 1°. le sol, quoique d'une nature difficile à cultiver, est susceptible d'une culture profonde, circonstance très essentielle à une bonne agriculture. 2°. Le voisinage de la mer rend le climat plus doux et la maturité plus précoce que dans les provinces de l'intérieur. 3°. La ville de Workington dont il est voisin, et qui contient neuf mille habitans, lui fournit des ouvriers en grand nombre, et des engrais pour suffire annuellement à cent vingt acres. Les bons effets de cet accroissement d'engrais s'accumulent continuellement sur le domaine. 4°. Les débris des mines de charbon mélangés à la proportion d'un tiers avec de la chaux font un fort bon engrais. L'année dernière, Mr. Curwen a fumé vingt acres de blé, qui ont donné une fort bonne récolte. 5°. La demande de tous les objets produits par la ferme est toujours extrêmement active, circonstance fort encourageante pour une bonne culture. Enfin Mr. Curwen joint à tous ces avantages celui de n'avoir lui-même aucun préjugé, et de chercher continuellement au dehors ce qui peut être utilement applicable à son exploitation \footnote{Nous avons déjà eu l'occasion de présenter Mr:}.\setcounter{page}{222} Curwen à nos lecteurs comme un agriculteur extrêmement distingué, et comme un de ces hommes dont l'ardeur et la capacité doivent rendre de grands services à la science. Malheureusement pour l'instruction qu'on pourrait tirer de son exemple, son exploitation agricole se trouve compliquée d'une exploitation de mines de charbon, qui rend la comptabilité de son agriculture, sinon impossible, du moins fort douteuse dans ses résultats. Si les chariages du charbon ne remplissaient les vides que l'exploitation agricole laisserait dans le travail de ses cent chevaux, ceux-ci seraient ruineux, et s'il en avait un moindre nombre, il ne pourrait pas mener avec la même vigueur une exploitation d'une si prodigieuse étendue, en nourrissant tous ses bestiaux au vert et à l'écurie, et dans une nature de terres qui force tour-à-tour à l'inaction et à de grands travaux; car pour exploiter avantageusement les vastes fermes de terres argileuses, il faudrait toujours avoir un emploi accessoire et profitable des attelages, quand on ne peut ni charier, ni labourer, ni semer, ni recueillir, afin de les retrouver en grand nombre dans les moments toujours trop courts où ces diverses opérations peuvent se faire convenablement.
Une autre circonstance empêche que l'exemple de Mr. Curwen ne puisse être aussi utile qu'on le voudrait; c'est qu'il achète annuellement, où tire de ses mines de charbon de terre, de quoi fumer cent quarante acres, c'est-à-dire, à-peu-près la septième partie de ses terres. Il est probable qu'il fait une bonne spéculation en achetant ainsi des engrais, aux chariages desquels il emploie ses chevaux dans les moments où ils seraient oisifs; mais avec une si forte addition d'engrais, il n'est plus possible de juger du mérite de l'assolement. Une des plus importantes conditions d'un bon assolement, c'est, non-seulement de se suffire à lui-même quant à la production des engrais, mais de créer ceux-ci en quantité plus\setcounter{page}{223} \section{FERME.}
considérable, d’année en année, de manière que le produit de la ferme s’élève graduellement avec eux. Dans les terres qui ne sont pas décidément stériles, et qu’on doit excepter, ce fait peut servir de pierre de touche pour juger de la bonté d’un assolemement. L’amélioration a seulement une marche plus lente dans les terres médiocres que dans les bonnes terres. On ne sauroit trop répéter que la comptabilité doit toujours marcher de front avec les opérations agricoles, et leur servir de contrôle Le difficile n’est pas de produire de belles récoltes, mais de les produire à bon marché: c’est sur le produit net de la totalité d’une exploitation qu’il faudroit toujours fixer les yeux, pour juger de son mérite, en ayant soin de comparer ce produit net à l’état décroissant, stationnaire, ou croissant de la fertilité des terres. Revenons à dire que les abondantes récoltes de Mr. Curwen ne prouvent rien. Il faudroit savoir ce qu’il lui en coûte pour recueillir, année commune, cinq mille six cents quintaux de froment sur sa ferme; il faudroit savoir quelle partie des rentrées de son agriculture on doit imputer à ses mines de charbon, il faudroit savoir s’il ne paie point trop cher le fumier qu’il fait de plus en chariiant à l’étable des récoltes vertes à des distances telles que le transport seul revient à trois livres sterling par acre. Enfin quand on auroit démontré qu’abstraction faite des autres entreprises de Mr. Curwen, son agriculture est profitable telle qu’elle est, son exemple n’en seroit pas mieux applicable, puisqu’il est rare qu’on puisse tirer du dehors des engrais en quantité suffisante et à bon prix, de quoi fumer les deux cinquièmes des froments de chaque année. (R)
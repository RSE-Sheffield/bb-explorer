\setcounter{page}{315} ISTRUZIONI PRATICHE, etc. Instructions pratiques sur la manière de bien faire et conserver le vin ; par le Sénateur DANDOLO. Milan 1812.
(Troisième extrait. Voy. p. 137 de ce vol.)
En général les agriculteurs croient avoir terminé la fabrication du vin, lorsqu'une fois ils l'ont mis dans les tonneaux. Ils pensent que les soins ultérieurs sont de peu d'influence. Il n'en est point ainsi : des soins bien dirigés complètent la qualité du vin et en assurent la durée au lieu que la négligence ou des procédés mal entendus détériorent le vin qui étoit bon lorsqu'on l'a mis dans le tonneau. Il se gâte de cette manière annuellement une très-grande quantité de vin.
Afin de montrer à l'agriculteur de quelle manière le vin s'améliore ou se gâte dans les tonneaux, nous allons examiner.
1°. De quelles substances sont composés les vins qui ont passé de la cuve dans la fuste.
2°. A quels changemens utiles ou nuisibles ces substances peuvent conduire les vins.
3°. Quels sont les temps les plus conve\setcounter{page}{316} nables pour enlever par le transvasage les substances nuisibles.
Les vins qui ont complétement fermenté, ceux dont la fermentation n'a pas été complétement achevée, et ceux enfin qui ont peu fermenté, contiennent des quantités différentes de substances solides. Enfin le vin d'une même cuve contient aussi des proportions très-diverses de substances solides. Examinons d'abord les vins différens qui peuvent sortir d'une même cuve; et ensuite les vins de qualités différentes.
Le vin limpide qui a subi le cours régulier de la fermentation, puis qui se trouble par son passage dans le tonneau; est le même moût qui existoit dans la cuve, sauf trois changemens essentiels, ainsi que nous l'avons vu. 1°. Au lieu de la substance sucrée que le moût contenoit, la liqueur contient de l'esprit-de-vin. 2°. Une portion seulement du levain et des substances solides tartareuses et ce que le moût contenoit, est demeurée dissoute, et une autre partie est suspendue dans la liqueur : tout le reste s'est séparé, ou a changé de forme. 3°. Une partie de la substance colorante, et du parfum; s'est plus développée et mieux fixée dans la liqueur vineuse qu'elle ne l'étoit dans le moût.
L'eau n'a point subi d'altération, et elle\setcounter{page}{317} continue à tenir en dissolution les principes qui composent le vin. L'acide est passé intact dans le vin, et ressort plus ou moins sensible par la conversion de la substance sucrée qui le masquoit dans le moût. Le vin limpide, au moment où l'on le soutire de la cuve, est donc principalement composé d'esprit-de-vin, de tartre, d'acide, de substance colorante, du principe de l'odeur, et d'une portion de sucre ou de levain non encore décomposée. Le tout est dissous dans l'eau, laquelle est la base du vin. Le vin trouble qu'on retire de la même cuve contient la même quantité de substances dissoutes que le vin clair, mais le vin trouble contient une plus grande quantité de ces substances seulement suspendues : or, ces substances suspendues sont les ennemies du vin, et empêchent qu'il ne soit de durée. Les vins pressés, quoique tirés de la même cuve que ceux qui ont été écoulés, contiennent en plus grande quantité des substances ainsi suspendues et non dissoutes. Si les tonneaux contiennent un vin dont la fermentation n'ait pas été achevée, parce qu'il a été écoulé trop tôt, alors le levain et la substance sucrée s'y trouvent en plus grande abondance. Il en résulte que ce vin pressé doit avoir une plus grande disposition\setcounter{page}{318} à fermenter sourdement dans les tonneaux; et comme il est toujours plus ou moins trouble, il contient aussi plus de lie et de levain, qui sont les ennemis de sa durée.
Les vins pressés, dont la fermentation n'a pas été complète dans la cuve, ont encore, en plus grande quantité les substances altérantes qui empêchent leur durée.
Les vins blancs qui contiennent, plus que les rouges, des substances solides et parenchymateuses, et certains vins rouges qui sont dans le même cas, peuvent être considérés comme ceux dont la fermentation n'a pas été complète. Les vins pressés, et provenans de cette dernière qualité de raisins, sont les plus chargés de matières étrangères.
Enfin le vin doux, qui se fait avec des raisins en partie desséchés et épaissis, s'écoule toujours dans la cuve, encore trouble, et contient beaucoup de matière sucrée, non décomposée.
Des causes qui contribuent à opérer des changemens utiles ou fâcheux dans les vins qui ont passé de la cuve au tonneau.
Nous avons vu que la fermentation tumultueuse cesse dans la cuve, aussitôt que la substance sucrée est près d'être décomposée par l'action du levain. Nous avons vu que tous les vins contiennent plus ou moins\setcounter{page}{319} de matières hétérogènes. A l'action de ces substances, il faut ajouter l'influence des causes externes ou secondaires, qui peuvent occasionner des changemens. Ces causes sont, 1º. une fermentation sourde ; 2º. une température de quatre à huit degrés au-dessus de la congélation ; 3º. un repos parfait ; 4º. la fermeture absolue d'un tonneau parfaitement plein. Ces diverses causes peuvent agir simultanément ou séparément les unes des autres.
La fermentation sourde, soit qu'elle commence dans le tonneau, ou qu'elle ne soit qu'une prolongation de celle qui a commencé dans la cuve, tend sans cesse à convertir en esprit-de-vin toute la matière sucrée. Ici il faut prendre garde qu'indépendamment de l'amélioration de qualité qui résulte pour le vin, de cette fermentation sourde, elle a encore deux avantages ; 1º. elle altère plus ou moins cette portion du levain, qui agit sur la substance sucrée, et dans cet état d'altération, ce levain tombe au fond du tonneau ; 2º. cette fermentation sourde favorise la séparation de la substance tartareuse, et de la matière colorante, lesquels les tombent à la lie, ensorte que la qualité du vin en est améliorée.\setcounter{page}{320} Influence de la saison froide pour opérer un changement utile sur le vin dans le tonneau.
L'expérience démontre, comme la saine physique l'indique, que le froid aide à faire déposer aux lies, les matières qui altéreraient le vin dans le tonneau.
Nous voyons que les vins légers, si on les place dans des endroits très frais, s'épurent peu à peu, et se conservent un grand nombre d'armées. Cette température, la plus favorable au vin, est toujours au-dessus de la congélation, c'est à dire, de quatre à huit degrés de Réaumur. Le froid a donc pour effet de faciliter la séparation d'une quantité de levain et de lies nuisibles au vin : on peut aussi par le moyen du froid, faire recommencer la fermentation sourde.
Influence du repos parfait pour opérer un changement utile sur le vin dans le tonneau.
Le vin soutiré très clair de la cuve, ne l'est plus lorsqu'il est dans le tonneau. Le mouvement seul qu'il a éprouvé suffit à faire paroître des parties solides qui altèrent sa transparence.
Si l'air est humide lorsqu'on écoule la cuve, la transparence du vin en est aussi affectée.
Si\setcounter{page}{321} Si les corps qui troublent le vin se manifestent dans le vin qui était déjà limpide, il est facile de comprendre que la quantité de ces corps doit être plus considérable, si le vin est trouble dans la cuve ou provient de la presse.
Quoique tous les corps qui troublent le vin, soient plus pesans que lui, et doivent s'enfoncer, ils demeurent suspendus s'ils sont en petit nombre et fort divisés. Le repos seul, pendant un temps suffisant, peut achever de débarrasser la liqueur des substances solides étrangères qui lui nuisent. Et il faut remarquer que lorsque, par une circonstance quelconque, ces substances se trouvent séparées de la liqueur, elles agissent plus promptement pour altérer le vin, que lorsqu'elles y demeurent dissoutes.
Influence de la fermeture exacte et de la plénitude des tonneaux, pour produire un changement utile sur les vins.
Ou il s'agit du vin dont la fermentation est achevée dans la cuve, ou de celui dont elle n'est pas tout-à-fait complète, ou enfin de celui qui est loin d'avoir assez fermenté:
Les vins qui ont bien fermenté doivent être renfermés presqu'entièrement dans le tonneau, dès qu'on les y met, et au bout\setcounter{page}{322} de cinq à six jours cette fermeture doit être complète. Pendant les premiers jours on laisse un peu d'évent par le bondon pour faciliter le dégagement du gaz acide crayeux qui s'est développé dans le soutirement de la cuve, et dans le transport, ainsi que de l'air atmosphérique interposé par le mouvement éprouvé. Le soin de tenir les tonneaux toujours bien pleins tend au même résultat d'empêcher le contact de l'air extérieur avec la surface du vin. Si donc un tonneau est bien plein et bien bouché, l'esprit et le parfum s'y conservent entiers, ainsi que l'addition de quantité qui résulte de la fermentation lente.
Les vins dont la fermentation n'a pas été achevée dans la cuve, contiennent plus de substance sucrée, non décomposée, et de levain en nature. Celui-ci, dans les vins communs, est plus abondant, et ils ont, par cette raison, plus de disposition à fermenter dans le tonneau; mais si ce tonneau est bien plein et bien bouché, cette fermentation est retardée. Le gaz acide crayeux qui se forme dans le tonneau, demeure interposé ou dissous dans le vin, et s'échappe en petite quantité, lorsqu'on ouvre le bondon pour remplir la pièce.
Si donc les vins de cette qualité demeurent bien clos dans le tonneau pendant dix\setcounter{page}{323} jours depuis l'écoulement, en supposant les fustes pleines et bien fermées, la liqueur reste imparfaite, et si on les met en contact avec l'air atmosphérique, il est dangereux qu'ils ne tournent à l'aigre.
Il n'en est point ainsi lorsque l'on a écoulé du vin qui a peu fermenté. Le tonneau alors est plein d'une liqueur qui contient encore beaucoup de levain et de substance sucrée. Cela se fait pour les vins mousseux, desquels je ne traite point ici.
Il est donc toujours utile de tenir les tonneaux bien pleins et bien fermés, et cela non-seulement lorsqu'on écoule de la cuve, mais aussi lorsqu'on transvase, ainsi que nous le verrons plus tard.
Des changemens qui peuvent arriver au vin dans le tonneau et qui le détériorent.
Les causes principales de cette détérioration que le vin peut subir dans les tonneaux, sont au nombre de cinq.
1°. L'excès du levain et de la substance des lies.
2°. Le renouvellement de la fermentation lente,
3°. Une saison ou un local dont la température est alternativement tiède et froide.
4°. L'agitation, quelle qu'en soit la cause.
5°. Le contact de l'air extérieur, et le vide sur le vin dans les tonneaux.\setcounter{page}{324} \section{De l'excès du levain et de la substance des lies.}
Le levain tend constamment à favoriser la fermentation lente, ou la fermentation acide, ou bien il tend à se séparer du vin, et à le gâter. Cela posé, il est évident, 1°. que le vin provenant de raisins trop murs, dont le suc s'est épaissi, et qui contiennent un excès de substance sucrée comparativement au levain, ne risque rien, lors même qu'il passeroit plusieurs années dans le tonneau, 2°. que les vins provenant du moût dans lequel la substance sucrée est abondante relativement au levain, et qui d'ailleurs ont été bien fabriqués doivent se conserver très-bien dans les tonneaux. 3°. Il y a toujours à craindre pour les vins qui ne sont pas naturellement d'une bonne composition, c'est-à-dire, dans lesquels le levain est en excès; parce que lorsqu'il ne trouve plus de substance sucrée sur laquelle il puisse agir, il porte directement son action sur les autres principes constituans du vin, et dispose celui-ci à se gâter ou à tourner à l'aigre. Les vins chargés de trop de levain sont donc ceux qui se gâtent le plus aisément, et malheureusement c'est de ceux-là qu'il y a le plus. Si, à cette surabondance de levain se réunissent d'autres matières des lies, l'altération est encore plus rapide et plus\setcounter{page}{325} forte ; le vin est alors fort difficile à racommoder. Si même on réussit à empêcher qu'il ne tourne à l'aigre ou ne se gâte tout-à-fait, il perd toujours une partie de son fumet et de sa qualité.
\section{Du renouvellement de la fermentation lente considérée comme une cause de l'altération du vin dans les tonneaux.}
Le renouvellement de la fermentation dans le tonneau peut devenir utile si l'on a les soins convenables : elle est nuisible si on les néglige.
Si un vin a suffisamment fermenté dans la cuve, peu importe que la fermentation lente du tonneau se renouvelle plus tôt ou plus tard ; mais s'il a mal fermenté dans la cuve, il reprend jusqu'à trois et quatre fois un mouvement intestin dans le tonneau.
Ce mouvement ne seroit qu'utile s'il s'achevoit sans interruption ; mais il arrive souvent que le premier mouvement de fermentation paroît peu après que le vin a été écoulé de la cuve, puis s'arrête au premier froid. Au printemps, le mouvement se reproduit et le vin se trouble. Enfin vers l'automne, et quelquefois même, l'année suivante, cela arrive encore. Ces diverses fermentations sont d'autant moins sensibles que le tonneau est mieux bouché et plus plein,\setcounter{page}{326} et à chaque fois que ce mouvement se renouvelle, il se dépose de la lie. Si donc on ne sépare pas les matières déposées lorsque la saison se réchauffe, on s'expose à voir gâter le vin, parce que la liqueur devenue tiède dissout une partie de la lie qui s'étoit déposée. Il devient trouble, ou gras, ou faible, ou enfin il tourne à l'aigre, surtout s'il y a contact de l'air.
Il y a donc toujours du danger à exposer les vins communs à achever leur fermentation tumultueuse, dans les tonneaux, parce que tout renouvellement de fermentation peut devenir nuisible au vin, si l'agriculteur n'use pas des plus grands soins.
\section{De l'influence fâcheuse des changemens de température sur le vin dans les tonneaux.}
Nous avons vu que la température de quatre à huit degrés au-dessus de zéro affoiblit dans le vin la capacité de tenir en dissolution les substances solides, et diminue l'action fermentante, jusqu'à faire cesser tout à fait ce mouvement intestin; mais lorsque la température s'élève à quatorze ou quinze degrés au-dessus de la congélation; le vin dissout une seconde fois une partie des substances qui s'étoient séparées, et contracte un mauvais goût ou devient acide.
Si on faisoit geler le vin dans le tonneau;\setcounter{page}{327} il s'opérerait une séparation de l'eau pres-que pure convertie en glace ; et il se pré-cipiterait une trop grande quantité de ma-tières solides. La concentration par un tel moyen altérerait plus ou moins la nature du vin.
Ce que nous venons de dire suffit pour faire comprendre qu'un tonneau de vin qui est exposé à des grandes alternatives de froid et de chaud, souffre aussi des dissolutions et des précipitations successives par le re-nouvellement et la cessation du mouvement intestin de la fermentation, en sorte que ce vin s'affaiblit ou s'aigrit promptement, ainsi qu'on le voit dans les celliers ou mauvaises caves.
\section{De l'altération des vins par l'agitation des tonneaux.}
Toutes les fois qu'on remue un tonneau, il en résulte pour le vin le même effet que s'il s'était réchauffé, et eût ainsi dissout une seconde fois une partie des lies déposées. Le mouvement remêle les lies, et empêche qu'elles ne puissent s'enfoncer. Plus il fait chaud, et plus cette circonstance est fâcheuse. Il arrive quelquefois qu'une forte commotion de tonnerre\footnote{Il ne paroît pas que la commotion électrique considérée seulement comme secousse, puisse suffire à expliquer le phénomène de l'altération que les vins subissent par le} agite tellement le vin dans\setcounter{page}{328} les fustes, que les substances déjà séparées se remêlent, et le vin se gâte, quoique les tonneaux soient bien pleins et bien fermés, ou qu'il tourne à l'aigre si les tonneaux se trouvent mal fermés ou non pleins.
\section{Du contact de l'air, et de la non plénitude des tonneaux, comme causes de l'altération des vins.}
Le vin peut se gâter sans le contact de l'air; mais ce contact est une circonstance nécessaire pour le faire tourner à l'aigre. Il suffit de la portion d'air qui est renfermée dans un tonneau mal plein, pour donner au vin une disposition continuelle à l'acescence: cette disposition se manifeste par les fleurs, qui montent à la surface dans les fustes et dans les bouteilles mal fermées.
Si l'air ne produit pas l'acescence, c'est parce que la liqueur manque d'esprit-de-vin, ou est très-chargée de levain et de matière colorante. Alors le vin se trouble ou devient gras. Lorsqu'un tonneau reste longtemps en perce, on sent une grande différence dans la qualité de celui qu'on tire à la fin, en comparaison de celui qu'on a tiré dans le commencement.
tonnerre : le fluide subtil de l'électricité y joue sans doute son rôle, comme dans l'altération qu'on voit subir au lait en temps d'orage. (R)
(La suite à un prochain Cahier.)
\setcounter{page}{117}
\section{PRINCIPES RAISONNÉS D'AGRICULTURE. Traduit de l'allemand d'A. THAER, par E. V. B. CRUD. Tom. Ier. in-4°, 372 p. Genève, 1811, chez J. J. Paschoud, Imprimeur-Libraire; et à Paris chez le même rue Mazarine, n°. 22. \large{(Dix-huitiéme extrait. Voy. p. 112)}}
On appelle glaise les terres composées d’argile et de sable, aussi long-temps que la proportion de ce sable siliceux ne passe pas 60 p. %. Au-dessous de 40 p. % de sable c’est un terrain argileux, et les défauts de l’argile y augmentent à mesure que la proportion du sable diminue. Cependant s’il y a beaucoup de silice combiné avec son argile s’il contient une forte proportion d’humus, les défauts inhérens à l’argile sont moins sensibles.
On trouve rarement sur les hauteurs une terre qui contienne plus de trois pour cent d’humus. Les sols argileux peu riches en\setcounter{page}{118} humus sont des terres à froment de seconde qualité. Elles sont propres à l'avoine et aux fèves. Si la proportion du sable y est plus forte; elles deviennent propres à l'orge et aux pois.
Si le terrain argileux contient peu de sable, d'humus et de chaux, il est extrêmement difficile à travailler. Cependant, s'il est fortement fumé, et que la température favorise la culture et les récoltes, il a quelquefois la supériorité sur les autres terres.
La glaise qui contient de 40 à 50 p.° g de sable, et qui est suffisamment pourvue d'humus, est la bonne terre à froment. A mesure que la proportion du sable y augmente, elle devient terre à orge, et terre à seigle. Ce sol présente peu de casualités; il est d'un travail facile, et est moins affecté par les changemens relatifs à l'humidité et à la sécheresse. C'est une terre propre aux légumes, au trèfle, aux pommes de terre, au colza, au lin, au tabac; et s'il y a des années où la terre argileuse donne de plus fortes récoltes de froment, cette glaise (à 40 ou 50 p.° g de sable) doit pourtant lui être assimilée comme terre à blé, quant à la valeur du sol.
Lorsque le sable entre en trop grande pro\setcounter{page}{119} portion dans la constitution du terrain, il est nuisible.
1°. Parce qu'il ne se combine pas avec l'humus, qu'à peine il s'unit mécaniquement avec lui, et parce qu'il n'absorbe pas les sucs fertilisans de l'atmosphère.
2°. Parce qu'il ne retient pas l'humidité.
3°. Parce qu'il ne peut pas supporter les cultures fréquentes, et que cependant les mauvaises herbes qui s'y multiplient, exigeroient ces cultures. Le remuement répété du sol sablonneux amène en contact avec l'atmosphère l'humus contenu dans les interstices, et non combiné, ensorte que l'air se l'approprie et que les vents l'emportent.
4°. Parce que le sable étant un bon conducteur de calorique, rend les influences des gelées et des chaleurs très-sensibles aux plantes, lorsque les changemens de température sont subits.
Quand le sol contient au-delà de 75 p. % de sable, il ne peut plus donner que du seigle et du blé-noir, encore faut-il qu'il soit suffisamment amendé. La pomme de terre peut encore y être cultivée avec succès.
Le meilleur moyen d'en tirer parti, c'est de le mettre en herbages. Le trèfle blanc, le ray-grass, la pimprenelle, la festuque des moutons y réussissent,\setcounter{page}{120} Si le sol contient 90 p.% de sable, ou davantage, il appartient au plus mauvais terrain. Il lui faut un long repos, et des engrais pour porter une récolte de grains, après laquelle il est épuisé. On peut admettre en principe qu'un tel sol perd un pour cent de sa valeur à mesure qu'il y a un pour cent de plus dans la proportion du sable; jusqu'à-ce qu'enfin ce terrain tombe dans la cathégorie des sables mouvans, qui sont absolument sans valeur. Il y a, au reste, des sables qui sont mêlés de grains calcaires, et qui sont beaucoup plus favorables à la végétation que les sables purement siliceux; mais l'auteur manque d'observations directes sur ce point.
La présence de la chaux, quand celle-ci est mêlée intimément à l'argile, est favorable à la fécondité du sol,
1°. Parce qu'elle rend l'argile plus friable et plus meuble.
2°. Parce qu'elle empêche que l'eau n'adhère trop fortement à l'argile; tandis qu'au contraire elle fait que l'eau adhère plus fortement au sol.
3°. Parce qu'elle favorise la décomposition et l'action réciproque des sucs nourriciers contenus dans le sol, et qu'elle sépare les substances végétales ou animales\setcounter{page}{121} qui adhèrent trop fortement à l'argile. Il est encore douteux si elle transmet son acide carbonique à l'humus, ou peut-être aux plantes mêmes, si, en revanche, elle attire cet acide carbonique de l'atmosphère, et agit directement comme corps nutritif. On a plusieurs motifs de le croire. On y reviendra en traitant de la chaux comme engrais.
4°. Parce qu'elle empêche la formation des acides dans le sol, qui nuisent à la végétation, ou que, quand ces acides sont formés, elle les absorbe et les neutralise.
5°. Parce que les grains qu'elle produit ont l'enveloppe plus mince, et donnent par conséquent une plus grande proportion de farine.
6°. Parce qu'elle est singulièrement favorable à toutes les plantes qui appartiennent à la diadelphie, et qu'ainsi les trèfles et les plantes à gousses y réussissent bien.
L'excès de chaux peut nuire.
1°. Parce qu'elle ne conserve pas l'humidité, et se réduit en poussière dans les temps chauds.
2°. Parce qu'elle consume promptement le fumier et l'humus; qu'elle accélère souvent trop le passage de celui-ci dans les plantes.\setcounter{page}{122} tes, précipitant ainsi leur végétation, et ne leur réservant pas assez de sucs pour le dernier période de leur croissance. Voici les termes de Chaptal sur les effets de cet excès de chaux: "Les terres où la chaux prédomine sont poreuses, légères, très-perméables à l'eau, ne recevant pas de retraite sensible par le feu. L'air pénètre aisément, la terre calcaire, et peut y vivifier les germes à une certaine profondeur; mais l'eau qui s'y infiltre sans résistance, s'en échappe avec une égale facilité. Une terre de cette nature est alternativement inondée et desséchée; et la plante qui ne sauroit résister à toutes ces variations, languit et s'éteint dans un sol de cette nature, pour peu que les sécheresses et l'humidité se prolongent."
La proportion de chaux la plus avantageuse est celle d'une quantité égale avec l'argile pure. De tous les terrains produits par des mélanges artificiels, au nombre de cinquante-quatre, sur lesquels Tillet fit des expériences sur la végétation des grains, le plus avantageux lui parut être celui qui étoit composé de ⅓ argile de potier, ⅓ marne coquillère, et ⅓ de sable. L'expérience en grand confirme que cette proportion de Tillet est en effet la meilleure.\setcounter{page}{123} Si l'on mêle la chaux à la couche supérieure, en trop petite quantité pour qu'elle change la consistance du sol, elle ne laisse pas d'influer avantageusement sur sa fertilité, probablement par son action chimique sur l'humus et le fumier. D'après des expériences qui demandent pourtant à être répétées, une addition de dix pour cent de chaux à un terrain argileux élève sa valeur de cinq à dix pour cent, et d'autant plus que ce terrain contenoit plus d'humus.
La chaux est nuisible, lorsque sa proportion dans le sol excède celle de l'argile; et d'autant plus que cet excès augmente. Si la chaux est mélangée avec beaucoup de sable, elle forme un sol trop sec et trop chaud. On ne peut alors y espérer de bonnes récoltes que des plantes qui supportent très-bien la sécheresse, comme par exemple, le maïs. Le terrain crayeux, composé en plus grande partie de chaux, souffre non-seulement de la sécheresse, mais aussi de l'humidité, parce que ce terrain se convertit en boue.
Il n'y a que les terrains d'alluvion dans lesquels l'humus soit mécaniquement sensible; car pour cela, il faut que sa proportion soit au moins de cinq pour cent. Ra-\setcounter{page}{124} rement, on trouve jusqu'à cinq pour cent d'humus dans les terrains élevés, soit argileux, soit sablonneux. Les plus riches ne contiennent guères au-delà de trois pour cent d'humus doux et soluble. La quantité d'humus diminue dans la proportion des récoltes qu'on en retire, mais sous la déduction des engrais qu'on y applique. Il ne faut pas au reste, se représenter que la production de l'humus par les engrais, soit très-considérable. Un amendement de deux cents quintaux de fumier, laisse après sa décomposition, à peine trente quintaux d'humus sec, et cette quantité est repartie sur un journal dont la terre soumise à la culture, pèse environ douze mille quintaux : donc ce n'est qu'environ, un quart pour cent d'humus ajouté à la quantité déjà existante dans le sol. Il est d'autant plus essentiel que celui-ci contienne naturellement beaucoup d'humus, que moins il en contient, plus il est difficile de lui en donner.
Dans un bon terrain glaiseux, l'auteur a trouvé ordinairement deux pour cent d'humus, en prenant la terre à la fin de la rotation. Il prend cette proportion pour la quantité moyenne de la bonne terre végétale glaiseuse, 1 et demi pour cent, pour la quantité moyenne de l'humus\setcounter{page}{125} dans les glaises sablonneuses, et 1 p. % pour le sable. Dans la classification ordinaire des terrains la quantité d'humus est prise en considération. Le même terrain glaiseux est classé parmi les terres à avoines, s'il ne contient qu'un pour cent d'humus, et parmi les terres à froment de seconde qualité, s'il contient trois p'. % d'humus.
Le terrain peut être très-riche en humus acide, et cependant, n'être pas fertile. L'auteur a trouvé cinq pour cent d'humus dans un terrain sablonneux de la Poméranie, où le seigle ne produisoit néanmoins que quatre pour un. L'humus de ce sol étoit acide; ce qui est dû à la méthode usitée dans le pays, d'amender la terre avec du fumier composé de bruyères. On peut améliorer ce terrain par de la marne; mais toutes les fois qu'il est possible de dessécher le sol, le meilleur moyen de l'amender, est l'écobuage. L'action du feu chasse une partie de l'acide, et le surplus de celui-ci est neutralisé par la potasse contenue dans la cendre. Ce terrain peut ainsi devenir une riche terre à froment.
La bruyère, qui ne vit qu'en famille, et qui prépare elle-même les sucs qui lui servent d'alimens, chasse toutes les autres plantes.\setcounter{page}{126} tes de son voisinage. Si on détruit la bruyère, et que l'on corrige par la chaux, la marne ou les cendres, l'acidité de l'humus, qui le rend impropre à la nutrition des autres végétaux, on obtient souvent une terre très fertile.
L'auteur se promet des résultats utiles d'une suite d'observations sur les divers sols, et sur leur analyse, dont il est encore occupé, avec le Prof. Crome, et que celui-ci publiera. D'après l'expérience de Thaer lui-même, et les renseignemens sûrs qu'il a rassemblés, la valeur réelle des terrains peut être calculée exactement, sur les principes qu'il indique. Le tableau suivant rend compte des parties constituantes des terrains qu'il a analysés, et présente l'estimation de leur valeur, en prenant le nombre 100 pour le maximum de fertilité.\setcounter{page}{127} TABLEAU des divers mélanges de substances des différens sols, et de leur valeur intrinsèque.
\comment{table}
Dénomination systématique | Dénomination usuelle | Combien ils contiennent pour cent | Valeur
| | | d'argile | de sable | de chaux | d'humus |
1.Argile fortement imprégnée d'humus. | Riche terre à froment. | 74 | 10 | 4 ½ | 11 ½ | 100
2.Terre très-tenace et imprégnée d'humus. | de même. | 81 | 6 | 4 | 8 ½ | 98
3.De même. | de même. | 79 | 10 | 4 | 6 ¼ | 96
4.Riche terre marneuse. | de même. | 40 | 22 | 36 | 4 | 90
5.Terrain léger imprégné d'humus. | Terrain de prairies. | 14 | 49 | 10 | 27 | *
6.Terrain sablonneux imprégné d'humus. | Riche terre à orge. | 20 | 67 | 3 | 10 | 78
7.Riche terrain argileux. | Bonne terre à froment. | 58 | 36 | * | 4 | 77
8.Terrain marneux. | Terre à froment. | 56 | 30 | 12 | 2 | 75
9.Terrain argileux. | de même. | 60 | 38 | * | * | 2 | 70
10.Terrain glaiseux. | de même. | 48 | 50 | * | * | 2 | 65
11.Glaise. | de même. | 68 | 30 | * | * | 2 | 60
12.Terrain glaiseux. | Terre à orge 1re. cl. | 38 | 60 | * | * | 2 | 60
13.De même. | dite. . . . 2de. cl. | 33 | 65 | * | * | 2 | 50
14.Glaise sablonneuse. | dite. . . . dite. | 28 | 70 | * | * | 2 | 40
15.De même. | Terre à avoine. | 23 ½ | 75 | * | * | 1 ½ | 30
16.Sable argileux. | de même. | 18 ½ | 80 | * | * | 1 ½ | 20
17.De même. | Terre à seigle. | 14 | 85 | * | * | 1 | 15
18.Terrain sablonneux. | de même. | 9 | 90 | * | * | 1 | 10
19.De même. | Terre à seigle de six ans. | 4 | 95 | * | * | ¾ | 5
20.De même. | Terre à seigle de neuf ans. | 2 | 97 ½ | * | * | ½ | 2\setcounter{page}{128} En comparant souvent les terrains dont on a fait une analyse exacte, et en examinant avec soin leurs caractères extérieurs, on acquiert peu-à-peu la faculté de découvrir, à l’aide de ceux-ci, la nature de cette composition. La couleur noire, la légèreté du sol, son odeur de moisi, et les pousses blanches du lichen humosus, sont les signes de la présence de l’humus dans le terrain. L’argile se distingue à son onctuosité et sa ténacité; le sable à son âpreté au toucher. On le distingue aussi à la loupe, dans le terrain émietté : le même moyen sert à distinguer l’humus noir. L’effervescence avec les acides trahit la présence de la chaux; et la quantité de celle-ci est indiquée par la force de cette effervescence. Relativement à la consistance du sol, on qualifie de tenace et intraitable celui qui, lorsqu’il a un peu trop d’humidité, s’attache à la charrue et à la herse, comme une pâte glutineuse. Il ne peut être pénétré que par des instrumens pointus ou tranchans, et ses coupures présentent des faces lisses et luisantes. Lorsque ce terrain est sec, il est dur comme de la brique; et ses mottes, brisées à force de coups, présentent des fragmens informes et feuilletés. Si le soleil luit après la pluie, il se forme à la surface une croûte.\setcounter{page}{129} dure, et l'intérieur conserve son humidité. Les terrains de cette espèce contiennent plus de quatre-vingts pour cent d'argile.
Les terrains roides ou forts se divisent avec moins de peine lorsqu'ils sont secs. Ils se rompent en morceaux d'une cassure matte, et s'émiettent. La charrue et la bêche ne le pulvérisent guère, mais le réduisent en mottes, qui ne peuvent être brisées que par un fort hersage. Ces terrains-là contiennent plus de cinquante pour cent d'argile.
On appelle le sol léger ou meuble, lorsqu'étant humide, il forme, à la vérité, des mottes, mais que ces mottes se brisent et se pulvérisent sous une légère pression: c'est ce qui a lieu dans les terrains qui ne contiennent que vingt à quarante pour cent d'argile.
On dit que le sol est mouvant, lorsque ses particules sèches ont peu ou point de consistance, ou de cohésion, et qu'elles tombent en poudre sans former des mottes. Les sols qui contiennent plus de quatre-vingt-dix pour cent de sable, les terrains crayeux, et enfin, ceux qui contiennent beaucoup d'humus, et peu d'argile, appartiennent à la classe des terrains mouvans. Si le vent déplace et transporte facilement un tel terrain, on l'appelle alors pulvérulent.\setcounter{page}{130} C'est en examinant le sol quarante-huit heures après une pluie douce, qu'on peut le mieux juger de la ténacité, et de la cohésion de ses parties. Avec un peu d'habitude, on parvient très-bien à les distinguer, en y plantant un bâton, ou même avec la pression du pied.
Dans l'estimation de la valeur du sol, l'examen de sa profondeur, c'est-à-dire, de l'épaisseur de la couche de terre végétale qui est imprégnée d'humus, doit suivre immédiatement l'examen des parties constituantes de ce même sol. Quelquefois, cette profondeur n'est que de trois pouces; souvent, elle va à dix ou douze pouces; mais l'auteur envisage déjà comme profond un sol qui a plus de six pouces d'épaisseur; et il prend pour profondeur moyenne celle de six pouces, pour que le sol ne soit pas défectueux sous ce rapport.
Lors même que le terrain n'est pas remué dans toute sa profondeur, il sert à la nutrition des plantes qui pivotent profondément; et une grande épaisseur de couche végétale met à la disposition de tout bon laboureur, et pour toutes sortes de produits les couches inférieures : il suffit de fouiller tous les six ou sept ans plus profondément qu'on ne le fait par les labours ordinaires\setcounter{page}{131} à la charrue. Alors les racines, même celles des céréales, pénètrent plus avant et vont chercher à une plus grande profondeur la nourriture que, dans un terrain peu profond, elles ne peuvent trouver qu'en s'étendant latéralement. Ainsi elles peuvent se serrer davantage entr'elles sans que la sphère d'activité de chacune soit diminuée. Les récoltes céréales sont, par cette raison, toujours plus épaisses dans les terres profondes, à nature égale de sol. On a affirmé que les racines des blés ne pénétroient pas à plus de six pouces : l'auteur les a observées à douze pouces de profondeur, dans une terre fouillée jusque-là. Les racines du trèfle et de certaines plantes qui appartiennent aux récoltes-racines, comme aussi celles de la luzerne pénètrent beaucoup plus profond; et les terrains où la couche végétale est épaisse sont particulièrement propres à la culture alterne.
Un terrain défoncé a d'ailleurs l'avantage de souffrir moins de l'humidité et de la sécheresse. L'eau des pluies a plus d'espace pour descendre avant de trouver la couche que les instrumens de labour n'ont point entamée : auparavant elle mettoit la surface en bouillie. Or, dans ce terrain défoncé, il y a un réservoir plus considérable pour\setcounter{page}{132} L'humidité, laquelle remonte lorsqu'il en est besoin, vers la surface, et entretient la fraîcheur des racines. On observe aussi que, dans les terrains défoncés les céréales sont moins sujettes à verser, quoique les épis soient plus hauts.
Quand la profondeur de la couche végétale est telle qu'on peut à peine la comprendre en entier dans un défoncement, on peut conserver au sol sa fertilité sans y mettre d'engrais, mais en creusant successivement de place en place, pour répandre le nouveau sol à la surface de l'ancien. De telles terres ont une valeur presqu'incroyable.
L'auteur, en partant de six pouces comme profondeur moyenne de la terre végétale, pense que chaque pouce de plus en profondeur de celle-ci, augmente de huit pouces la valeur du terrain; ensorte qu'un sol où la couche végétale a douze pouces de profondeur vaut, toutes choses d'ailleurs égales, à-peu-près la moitié de plus que celui qui n'en a que six au-dessous de la profondeur que peuvent atteindre les labours ordinaires, l'augmentation de l'épaisseur de la couche végétale n'a plus autant d'avantages. Cependant, comme cette terre inférieure est encore utile aux plantes, l'auteur estime à 5 pouces l'augmentation en valeur qui résulte de\setcounter{page}{133} de chaque pouce de plus en profondeur de la couche végétale, au-delà d'un pied.
En revanche, il évalue à 8 p. % de moins en valeur pour le fonds, la diminution de chaque pouce d'épaisseur de la couche végétale depuis six pouces jusqu'à trois, qui est la moindre épaisseur présumée.
La nature de la couche qui est au-dessous de la terre végétale influe beaucoup sur la valeur du fonds. Si la couche végétale est sablonneuse, mais repose sur une terre imperméable aux eaux pluviales, le sol se maintient beaucoup plus frais que sa nature ne semble l'indiquer. Si l'on entame à la charrue la glaise sur laquelle repose la terre légère ou sablonneuse, ce mélange peut améliorer le fonds d'une manière durable, quoique cet effet ne devienne sensible qu'au bout d'un certain temps, lorsque l'influence des météores, des engrais, et du parfait mélange se fait sentir.
Mais, si l'argile inférieure est disposée de manière à former comme des réservoirs dont les eaux pluviales ne puissent pas s'écouler, il en résulte des places mouilleuses, même dans les champs de terre sablonneuse ou légère.
Si la couche inférieure est marneuse ou
Agricult. Vol. 18. N°. 4. Avril 1813. L\setcounter{page}{134} calcaire, tandis que la terre végétale est argileuse ; il résulte alors de l'approfondissement des labours, une amélioration sensible. Il en est de même si l'on ramène par des labours profonds, une partie de la couche sablonneuse qui, quelquefois se trouve dessous la terre argileuse des champs. Si une couche mince de terre végétale repose sur le sable, le terrain est fort exposé à souffrir des sécheresses. Si la couche inférieure de sable ou gravier, est très-mince, et repose elle-même sur l'argile; si le terrain manque de pente pour écouler les eaux ; celles-ci s'amassent comme dans un réservoir, dans la couche de sable, et rendent le champ mouilleux et stérile. Un terrain ne peut être amélioré que par des saignées. Plus le sable inférieur est profond, plus le terrain est sec. Dans les terrains dont la couche inférieure est pierreuse ; les plus fertiles sont ceux qui reposent sur des pierres à chaux. Ces pierres sont souvent délitées, pénétrables par l'eau et les racines. Peut-être celles-ci en tirent-elles une partie de l'acide carbonique qui nourrit les plantes. Les rochers calcaires et gypseux sont moins stériles que ceux d'un autre genre.\setcounter{page}{135} Les schistes argileux se délitent à l'air. On assure avoir rendu plus profonde, une couche mince de terre végétale, en entamant peu-à-peu à la charrue, le schiste argileux inférieur.
Le granit exclut toute végétation : le mélange qui pourroit se faire peu à-peu avec la couche végétale, de la couche granitique inférieure, n'améliore point le terrain.
Si la couche végétale repose sur des cailloux roulés, et qu'elle ait une épaisseur suffisante, la présence des cailloux roulés au-dessous de cette couche est une circonstance favorable, parce qu'elle permet l'écoulement des eaux surabondantes.
L'ocre, ou la mine de fer, qu'on trouve souvent au-dessous de la couche végétale, est extrêmement nuisible à la végétation; dès que les racines des plantes atteignent cette mine de fer, elles dépérissent.
La couche inférieure se distingue en perméable ou imperméable, relativement à l'humidité du sol. Celle-ci est le plus souvent due à l'imperméabilité de la couche inférieure: Quelquefois l'argile de celle-ci est alliée d'une assez grande quantité de sable pour que l'eau pût la pénétrer; mais si l'on laboure toujours à la même profondeur, les poids des animaux de labour, et la pression\setcounter{page}{136} de la charrue rendent la terre inférieure imperméable à l'eau.
La difficulté de dessécher certaines terres, tient à des causes très-différentes les unes des autres. Quelquesfois les eaux viennent, d'un terrain plus élevé, et n'ont pas d'écoulement ultérieur. Quelquesfois elles suintent par les couches sablonneuses et proviennent, d'un bassin, d'un lac, ou d'une rivière, qui sont dans une situation plus élevée. Quelquesfois les eaux pluviales se rassemblent sur une couche imperméable, et horizontale, qui forme comme des réservoirs de place en place, où la terre demeure en bouillie, après les pluies, et ne se dessèche que lentement et par l'évaporation.
L'humidité habituelle du sol le rend quelquefois propre aux prairies, mais elle lui ôte toujours la fertilité pour les grains, sur-tout pour les grains d'automne.
Pour bien juger de la disposition du sol à retenir l'humidité, il faut l'examiner quelques jours après une pluie douce. En le comprimant dans la main, on le trouve ou sec, ou seulement essuyé, ou frais, ou humide : cela dépend de l'affinité d'adhésion du sol avec l'eau; mais si l'eau en sort lorsqu'on l'exprime dans la main, il est aqueux\setcounter{page}{137} \section{ISTRUZIONI PRATICHE, etc. Instructions pratiques sur la manière de bien faire et conserver le vin; par le Sénateur DANDOLO. Milan 1812.}
(Second extrait. Voyez p. 77.)
CHACUN sait, par la pratique, ce que c'est que la fermentation : c'est-à-dire, ce mouvement intestin qui agite, sépare, soulève et tourmente, en quelque sorte, les corps qui y sont soumis. Par exemple, un tas de fumier frais, ou la pâte à laquelle on a mêlé du levain.
Le raisin, avant que d'être écrasé, ne sauroit fermenter : nous en avons vu la raison. Il en est de même des poires, des pommes, des cerises, etc. tant que ces fruits sont entiers ; mais lorsque ces fruits doux ont été écrasés, toutes leurs parties se mélangent entr'elles, le levain qui auparavant ne pouvoit pas agir, exerce son action, et la fermentation s'en suit. Alors la liqueur douce
\setcounter{page}{241} \section{A G R I C U L T U R E}
ON THE CAUSES, PREVENTION, SYMPTOMS, etc.
Des causes et des symptômes d'une maladie destructive parmi les troupeaux, et des moyens de la prévenir et de la guérir. Par JAMES HOGG, berger d'Ettrick.
( Farmer's Magazine ).
Mr.
Je vais vous rendre compte en aussi peu de mots que je le saurai, des faits et des expériences dont j'ai connoissance, concernant la maladie destructive qu'on nomme tournis, et qui attaque les agneaux, et quelquefois les antenois.
Quant aux causes de cette maladie, les bergers sont d'accord à croire qu'elle est due à un refroidissement de la région dorsale, pendant les vents et les pluies froides de l'hiver. Les agneaux contractent ainsi une espèce d'engourdissement qui se termine par un épanchement d'eau dans la tête. D'après des observations longues et exactes, je suis
Agric. Vol. 18. N°. 7. Juillet 1813. V.\setcounter{page}{242} convaincu que c'est là la cause de cette maladie \footnote{L'auteur de cette lettre, qui en effet a été berger pendant un grand nombre d'années, mérite d'être écouté sur l'objet très-intéressant qu'il discute, quoique sa théorie puisse faire sourire les gens de l'art; et que dans sa longue pratique, il n'ait pas seulement appris que le tænia nebuleux est un animal qui se nourrit de la substance même du cerveau, par des suçoirs qui, à ce qu'il paroît, traversent au besoin l'enveloppe du fluide dans lequel il nage. Nous avons déjà eu occasion de citer la pratique curative de ce berger, qu'on croyoit alors être l'inventeur de l'opération du wiring (perforation au fil de fer). Il paroît que cette opération, très-anciennement pratiquée dans les montagnes d'Écosse, et traitée avec mépris par les gens instruits qui en ont eu les premiers connaissance, mérite néanmoins toute l'attention des vétérinaires.}. Quant à l'opinion de certains bergers, que l'eau des pluies filtre au travers de la peau, pénètre dans le canal de la moëlle épinière, et de là dans le cerveau, où elle est forcée de s'arrêter, je suis fort disposé à la combattre. Comment supposer, en effet, que l'eau traverse la peau, la chair et les os pour pénétrer sous la colonne vertébrale? Que l'animal prenne en effet cette maladie par le froid et l'humidité, cela me semble prouvé, d'après les faits dont je parlerai tout-à-l'heure; mais je pense que cette eau qui se réunit dans le cerveau, n'est point\setcounter{page}{243} \section{MOYENS DE PR. ET DE GUÉRIR LE TOURNIS.}
l'eau pure de la pluie : je crois plutôt qu'elle est une sérosité procédant en effet de l'humidité rassemblée sur le dos, et modifiée; sans cela, comment expliquer cette qualité corrosive, qui consume et détruit tout ce qui se trouve en contact avec ce fluide, ainsi que sa tendance à corrompre les autres fluides du système animal.
J'en appelle au témoignage de tous les bergers attentifs, sur la vérité des observations suivantes, qui montrent quelles sont les causes de la maladie.
1°. Elle est toujours plus générale après un hiver où il a beaucoup plu et neigé.
2°. Elle est toujours plus destructive dans les fermes où les troupeaux ne sont pas à l'abri, et ont à souffrir des vents froids et des pluies violentes.
3°. Elle ne se développe que chez les agneaux à leur première année, probablement parce que leur laine s'ouvre sur le dos, et expose ainsi l'épine à l'humidité et au froid.
4°. Si l'on fixe sur le dos des agneaux une bande de drap ou de peau, de manière à garantir l'épine, ces individus ne prennent pas le tournis.
Ce dernier fait est sans doute le plus décisif; et s'il n'était pas parfaitement prouvé,\setcounter{page}{244} Je ne me mettrois point de le mettre en avant. L'expérience est facile à faire. Outre l'avantage de prévenir le tournis, cette bande d'étoffe de laine, fixée le long du dos, fait prospérer les agneaux, en leur donnant une espèce d'abri portatif pendant les plus mauvais temps. La dépense n'est presque rien. Une paire de vieilles couvertures, valant quatre ou cinq shillings, peuvent couvrir une cinquantaine d'agneaux; et si l'on a soin d'ôter ces bandes au retour de la belle saison, il y en a pour deux ou trois ans.
On objectera que l'expérience acquise par un individu sur un troupeau peu nombreux d'agneaux ainsi vêtus, est faite trop en petit pour qu'on puisse en tirer des conclusions rassurantes relativement à cette maladie en général. Je puis répondre que j'ai fait moi-même un assez grand nombre d'observations, et que, sur-tout, j'ai recueilli beaucoup de faits observés par d'autres, et sur lesquels je puis compter.
Je me souviens du temps où il n'y avoit pas plus de deux mille bêtes à laine de la race de Cheviot dans le comté de Selkirk, le plus beau pays de pâturage de toute l'Ecosse, quoique très-élevé et fort exposé aux orages pendant l'hiver. Aujourd'hui, tous les pâturages du Comté sont couverts de cette\setcounter{page}{245} \section{(245)}
race; et par approximation, j'estime qu'il y a soixante mille moutons de Cheviot dans le Comté. L'usage n'étoit pas de les introduire par croisemens, mais de les acheter là où on pouvoit les trouver sur les frontières de la province. En conséquence de cela, la plupart des troupeaux de la forêt d'Ettrick ont été formés par des agneaux de rebuts, qui sont ceux que les fermiers ne veulent pas garder sur leurs propres fermes. Il étoit difficile de faire passer l'hiver dans un pays montueux et froid à des agneaux de cette qualité. L'expédient de les habiller (bratting) est celui qui a le mieux réussi. On en voyoit fréquemment jusqu'à cent dans une ferme ainsi couverts, et c'étoient tous les plus foibles du troupeau. Je n'ai pas connu un seul exemple d'un agneau ou antenois ainsi habillé qui ait pris le tournis. Je n'ai jamais ouï citer non plus un seul exemple de la chose, pourvu que l'animal ne fût pas attaqué avant qu'on prît cette précaution. Il y a, dans la partie basse de Dumfries, beaucoup de petites fermes où les fermiers nourrissent au pâturage pendant l'hiver un certain nombre d'antenois. Ordinairement, ils ont la précaution de les enduire de gouadron; mais souvent ils remplacent le goua\setcounter{page}{246} dron par une bande d'étoffe de laine ou de peau fixée sur le dos. J'ai trouvé tous les témoignages des fermiers d'accord: ils affirment que les agneaux ainsi préservés, ne sont jamais attaqués du tournis. Je dois rappeler ici, qu'il y a une autre maladie dont les symptômes ont beaucoup de rapport à celle dont je parle. Elle se manifeste dans les pâturages de bois; ses progrès sont rapides et elle est à-peu-près incurable. C'est peut-être une espèce de vertigo semblable à celui qui attaque les rennes dans les forêts du nord du continent. Je n'ai jamais disséqué de moutons morts de cette maladie: on la dit contagieuse parmi les rennes. Le premier symptôme par lequel le tournis commençant peut être reconnu, est l'apparence de l'œil: il prend une couleur bleuâtre, et l'orbite semble devenir plus grand. Un berger expérimenté, qui choisit un lot d'agneaux, découvre par cet indice les sujets qui sont déjà sourdement affectés, quoiqu'ils ne tournent point encore. Quand la maladie est un peu plus avancée, l'animal a l'œil fixe et semble regarder à côté de l'objet qui l'inquiète. Il prend ensuite un air triste et se détache du troupeau. Enfin, lorsqu'il pâture, il tourne sans cesse du même côté. Une observation que j'ai faite aussi, c'est\setcounter{page}{247} que les animaux qui ont le tournis se rapprochent continuellement, et tout en tournant d'un bruit constant, s'il a lieu à leur portée, comme la chute d'un ruisseau, ou le bèlement des agneaux et des brebis.
Pendant la seconde époque de la maladie, qui est celle où l'animal tourne, on découvre sur la tête, en appuyant les deux pouces, le siège du mal, parce que l'os est tendre dans un espace grand comme un sou, dans le voisinage des sutures du crâne. Si l'on ne trouve pas là le siège du mal, il faut examiner les deux tempes tout auprès de l'endroit où sortent les cornes. Si l'on ne trouve rien, le mal sera probablement sous le milieu du front; et alors on ne peut s'en assurer que quand la maladie est fort avancée. C'est un cas plus rare : deux fois sur trois, le siège est dans la partie supérieure du crâne.
On ne doit pas essayer l'opération jusqu'à ce qu'on soit sûr de l'endroit où le mal existe; mais le plus souvent, on peut s'assurer du siège du dépôt avant que l'animal soit fort affoibli: du moment qu'on en est sûr, il faut essayer d'opérer, parce que cette maladie est toujours mortelle. Le progrès du mal est infaillible. La substance du cerveau se fond par le contact avec cette bulle d'eau.\setcounter{page}{248} qui semble se nourrir et s’accroître en détruisant cette substance même. Dans quelle place du cerveau que cette goutte d’eau s’établisse, et quoique dans le début on puisse à peine la discerner, elle ne cesse de croître aux dépens du cerveau lui-même, jusqu’à-ce que la substance de celui-ci soit réduite quelquefois à la moitié, et que l’animal périsse.
A l’ouverture de la tête, on trouve une vessie dont la grosseur est proportionnée à la durée de la maladie. Cette vessie est d’une texture très délicate, mince, transparente et à peine visible. Le fluide est comme de l’eau pure, mais quelquefois mélangé d’une substance blanchâtre. La vessie est ordinairement placée entre le cerveau et le crâne, mais toujours enfoncée dans la substance médullaire, et ne touchant au crâne que par un espace peu étendu. L’effet de ce contact est d’abord d’amollir, et ensuite de corroder l’os lui-même. La vessie augmente en volume à mesure que la maladie se prolonge, et le volume du cerveau diminue proportionnellement. Mais il y a des circonstances qui sont bien difficiles à expliquer.
Il semble naturel de supposer qu’à mesure que la vessie grossit et comprime le cerveau, celui-ci doit acquérir plus de con-\setcounter{page}{249} sistance : c'est tout le contraire : plus la maladie est avancée, et plus le cerveau est amolli et en quelque sorte aqueux. Le fluide est toujours renfermé dans une ou deux enveloppes, qui sont indépendantes du cerveau et du crâne, et n'y tiennent par aucun filament quelconque. Si l'on ouvre le crâne vis-à-vis de la vessie, on peut la retirer toute entière. Quel est donc le mécanisme par lequel cette vessie croît et se nourrit? Après cela, comment expliquer la diminution graduelle de la masse du cerveau. Si l'on suppose qu'il est rongé par des insectes, est-ce ceux-ci qui produisent cette vessie d'eau, ou la vessie qui produit les insectes? Je sais bien qu'on a découvert des vers dans la substance médullaire du cerveau des moutons. J'avoue que je n'ai jamais su y découvrir aucune apparence d'insectes. Il est vrai que je n'ai pas employé le microscope. Il est vrai encore, que j'ai souvent observé dans les cloches d'eau, des petits corps blanchâtres, qui y sont suspendus, et que j'ai cru être des animaux, par leur ressemblance avec le ver de la fourmi; mais je n'ai jamais observé de la vie dans ces petits corps.
Dans le nombre des diverses opérations par lesquelles on peut guérir la maladie du\setcounter{page}{250} tournis, je dois, d'après mon expérience préférer celle que je vais décrire, et qui probablement étonnera ceux d'entre vos lecteurs qui forment leur opinion sur la théorie et non sur la pratique. Plusieurs de vos médecins et professeurs d'Edimbourg ont déjà douté de la vérité des faits, et avec une inconcevable présomption, ils m'ont nié en face ce que j'affirmois pour en avoir été mille fois témoin. L'opération que je recommande comme efficace consiste à introduire un fil-de-fer pointu, dans les narines, et à percer ainsi le cerveau lui-même, pour atteindre et percer également la vessie pleine d'eau qui cause le mal. Voilà, monsieur, je le répète, le moyen le plus sûr de guérir cette maladie. Si ce moyen ne sauve pas l'animal, il le tue avec certitude. Cette manière est à la portée de tous les bergers; et il est, en quelque sorte, impossible qu'ils opèrent mal, s'ils font attention aux directions suivantes \footnote{Il n'est pas étonnant que ceux qui connoissent l'anatomie de la tête, et qui sayent qu'il n'existe aucune communication directe entre l'intérieur du nez et le cerveau, nient la possibilité d'une telle opération, avant d'en avoir été témoins. Cependant comme l'os éthmoide, toujours spongieux, est tendre encore dans les agneaux, et qu'on opère avec une broche d'acier forte}.\setcounter{page}{251} L'instrument employé est un fil de fer de la grosseur d'une aiguille à tricoter, telle qu'on l'emploie pour les bas les plus grossiers. Il faut le maintenir propre, bien pointu et huilé pour prévenir la rouille. La pointe de cet instrument ne doit avoir qu'une ligne et demi de longueur; car si l'instrument alloit peu-à-peu en pointe comme une aiguille, il dévieroit aisément dans le nez, et se fixeroit dans l'os qui est au-dessous du cerveau, tourmentant ainsi l'animal sans utilité. Si la pointe de l'instrument est émoussée, il arrive souvent qu'il ne perce pas la vessie, mais qu'il la pousse inutilement : il faut donc que la pointe soit à-la-fois très-courte et bien aiguisée. Le berger commence à tâter l'animal sur la tête, jusqu'à-ce qu'il trouve le siège du mal, qu'il reconnoît à la mollesse de l'os. Si le mal est au sommet de la tête, et au milieu, ce qui arrive deux fois sur trois, il faut tenir l'animal serré entre les genoux, mettre le pouce de la main gauche sur le siège du mal, puis avec la main et aiguë, cet os peut être percé de part en part dans cette singulière opération. Reste à expliquer ce que devient dans la boîte osseuse du crâne l'épanchement séreux, le tœnia lui-même, la membrane ou kyste qui enveloppoit la sérosité, et comment la portion de cerveau détruit se régénère. (R)\setcounter{page}{252} droite, faire pénétrer l'aiguille par celui des naseaux qui se trouve le mieux vis-à-vis de la vessie. Cette opération ne dure pas plus d'une seconde. A l'instant où on sent la pointe de l'aiguille sous le pouce qui appuie sur le crâne, il faut lâcher l'animal et le mettre sur ses jambes. S'il fait froid, il convient de lui faire passer la nuit dans la bergerie.
Si la vessie se trouve placée immédiatement sous la suture du milieu du crâne, il faut essayer d'opérer par les deux naseaux successivement; si l'on manque la vessie d'un côté, on est sûr de l'atteindre de l'autre. S'il est impossible de découvrir avec le tâtonnement des pouces, le siège de la maladie, la vessie est alors placée au centre du cerveau entre les deux ventricules. Il faut dans ce cas tenter l'opération ainsi que je viens de la décrire; mais la guérison est extrêmement douteuse parce qu'ordinairement il existe un grand nombre de petites vessies, et que si l'aiguille en perce quelques-unes, elle manque les autres.
C'est par cette opération très-simple que j'ai guéri des centaines de tournis. Je n'en ai point tenu registre, mais j'estime avoir réussi, ou vu réussir, trois fois sur quatre, dans toutes les opérations que j'ai faites, ou\setcounter{page}{253} qui sont venues à ma connoissance, quoique dans le nombre de ces dernières il y en ait eu beaucoup qui étoient mal faites, parce qu'on n'avoit pas pris les précautions que j'ai indiquées pour s'assurer du siège du mal; Quelquefois j'en ai opéré douze de suite; d'autres fois dix, sans en manquer un. D'autres fois il m'est arrivé, sur tout dans la saison froide, d'en tuer trois ou quatre successivement.
Sir George Mackensie a fait entendre dans son ouvrage sur les moutons, que j'étois l'inventeur de ce moyen de guérison: il n'en est point ainsi: cette pratique de percer la cloche d'eau avec un fil de fer (wiring) est en usage parmi les bergers d'Ecosse depuis des siècles, mais ils sont obligés de faire l'opération en se cachant de leurs maîtres, car ceux-ci sont persuadés, comme les professeurs d'Edimbourg, qu'en transperçant le cerveau d'un animal avec une broche de fer, on doit le tuer. Sir Georges, au reste, n'a pas bien saisi ce que j'ai dit là-dessus dans les Transactions de la Société des montagnes d'Ecosse: je ne prétendois pas avoir découvert l'art de guérir les moutons de cette manière; mais je me félicitois d'avoir appris à l'appliquer. Je disois, dans les Transactions, que lorsque j'étois aide berger, ce que j'ai\setcounter{page}{254} été un grand nombre d'années, je m'emparois de tous les agneaux malades du tournis, que je pouvois trouver, à qui que ce fût qu'ils appartinssent. Je leur faisois l'opération, et les laissois courir. Comme pendant le printems et l'été, j'habitois un pays couvert de troupeaux, il y avoit toujours un grand nombre de tournis, et il en a résulté que ma pratique a une prodigieuse étendue. Il s'est passé plusieurs années sans que j'eusse la preuve d'avoir manqué la cure dans aucun cas; mais il est vrai qu'il m'étoit souvent difficile de vérifier les conséquences de l'opération, parce que les troupeaux changeoient de pâturages, et que je perdois de vue les bêtes opérées. Ce qu'il y a de certain, c'est que j'ai guéri ainsi un très-grand nombre de jeunes bêtes qui appartenoient à divers propriétaires, parce que je n'étois point autorisé à tenter l'opération.
Voici les symptômes qui peuvent faire juger si l'on a opéré avec succès ou non. Si l'animal est très-foible immédiatement après, c'est ordinairement un bon signe. Si la foiblesse augmente et qu'il ne mange pas, ni le jour ni le lendemain de l'opération, il est probable qu'il périra et s'il n'est pas très-maigre, il faut le tuer pour le manger. La\setcounter{page}{255} viande des animaux qui ont le tournis est tout aussi bonne que celle des autres moutons : on dit même meilleure et plus tendre, ce qui vient probablement de ce que l'animal est toujours jeune.
Le premier symptôme du rétablissement de l'agneau après l'opération, c'est le bêlement : Quelqu'appesanti et foible qu'il paroisse, s'il bêle, on peut être sûr qu'il est guéri : il commence à reprendre le besoin de société, que le tournis lui avoit fait perdre depuis long-temps.
Lorsque j'ai ouvert la tête des moutons, morts de l'opération, j'ai trouvé la trace du fil de fer, et vérifié que les membranes que l'instrument avoit percée étoit très-enflammées. J'ai trouvé une fois une déchirure dans la pellicule qui enveloppe le cerveau, assez grande pour y passer le doigt. L'animal, à l'instant de la perforation étoit tombé comme s'il eût été assommé d'un coup de hache : je n'ai observé que trois fois la même chose dans toute ma pratique.
Il faut que l'opérateur fasse principalement attention à ceci. Si le siége du mal est contigu au tempes, c'est-à-dire, à la base des cornes, pour les races qui en ont, l'opération de perforer est inutile. Il m'a toujours paru que si l'on perce le cerveau dans un\setcounter{page}{256} endroit parfaitement sain, il y a beaucoup plus de danger pour l'animal que si l'on perce cet organe dans un endroit déjà affecté, et amolli. J'ai quelquefois essayé, et toujours sans succès, de donner au fil de fer une certaine courbure pour le faire parvenir sous les tempes. Le Dr. Andrew Duncan le jeune, d'Edimbourg, ayant entendu parler de mes succès dans la guérison de cette maladie, me fit présent d'un petit trocar d'argent avec lequel j'enlevois très-proprement le fluide contenu dans les vessies, et sans que l'animal parût en souffrir; mais la guérison n'étoit qu'apparente, et au bout de quelques temps le tournis se manifestoit de nouveau.
Voici à quoi j'attribue ma réussite avec le fil de fer, et le non-succès avec le trocar. Dans le premier cas, la partie inférieure du sac qui contient l'eau, est percée, et cette eau s'échappe; j'ignore, au reste comment et par où, car jamais je n'ai vu sortir une seule goutte d'eau par les nazeaux : il en sort seulement une ou deux gouttes de sang. La vessie étant percée par dessous, non-seulement l'eau qui y étoit contenue s'échappe, mais l'eau qui s'y reforme, ou s'y rassemble de nouveau, s'écoule également. Le cerveau reprend ainsi peu-à-peu la place qu'il\setcounter{page}{257} qu'il occupoit, et la consistance qu'il doit avoir. Dans l'autre manière de faire l'opération, en perforant au trocar, ou en trépanant, c'est la partie supérieure de la vessie qui est percée; et il est peut-être impossible de détruire la maladie par cette voie, car outre le fluide il y a probablement des animalcules qui restent dans la vessie, et dont la présence fait bientôt reparoitre les symptômes du tournis. J'ai vu, au reste, jusqu'à trois vessies pleines d'eau dans le crâne.
Je ne prétends point faire des observations scientifiques, mais seulement rapporter les faits. Je suis obligé d'avouer ici que l'opérateur le plus habile et le plus heureux que j'aie jamais vu, faisoit l'opération d'une manière toute différente de celle que je pratique et recommande. Au lieu d'un fil de fer comme une aiguille de bas, il n'employoit qu'une grosse et longue épingle, qu'il portoit toujours piquée à son bonnet. Comme moi il opéroit tous les agneaux tournis qu'il rencontroit, à qui que ce fût qu'ils appartinssent; mais il faisoit toujours l'opération par en haut, c'est-à-dire, qu'il enfonçoit son épingle dans la partie ramolie du crâne; ainsi qu'on le fait avec le trocar. Comme ceci ne met nullement en danger la vie de l'agneau.
Agricult. Vol. 18. N°. 7. Juillet 1813.\setcounter{page}{258} nimal, j'ai souvent essayé de faire de même, avant d'avoir recours à ma manière d'opérer; mais je ne me rappelle pas en avoir guéri un seul en employant l'épingle. Je me souviens d'en avoir causé avec ce berger, et de lui avoir oui-dire que dans ses propres troupeaux il n'en manquoit presque jamais; mais que dans d'autres fermes il en guérissoit très-peu. Il sembloit d'après cela, que selon les localités, la maladie a un caractère différent. Je suis maintenant convaincu que cet opérateur enfonçoit ordinairement son épingle assez profond pour percer la vessie de part en part, chose que je n'ai jamais eu le bon sens d'essayer, et qui, si l'on raisonne par analogie, doit être aussi efficace, moins difficile et moins dangereux que ma manière d'opérer. En un mot il me semble que pour assurer la guérison il est nécessaire de percer le fond de la vessie qui contient l'eau. Vous devez avoir pensé quelquefois, monsieur, que ce qu'il faudroit pouvoir faire, seroit d'enlever complétement le sac et ce qu'il contient. Il est probable que si cela étoit fait fort adroitement, cela réussiroit bien; mais les bergers n'ont ni assez d'adresse, ni assez de soin, pour mettre le cerveau à l'abri des influences de l'air, en res-\setcounter{page}{259} fermant convenablement le crâne, après le trépan. Je recommande donc de préférence les deux manières d'opérer que l'expérience m'a appris être efficaces, et à la portée de tous les bergers. J'espère que ce que j'ai dit engagera ceux qui ont des intérêts de troupeaux, à s'occuper de prévenir cette maladie, et de guérir par l'opération les animaux attaqués. Il y a des fermiers et des bergers qui regardent avec indifférence la perte annuelle de quelques agneaux que le tournis leur enlève, parce qu'ils ne croient pas qu'il y ait de remède, mais à la longue, ces pertes accumulées deviennent un objet important.
JAMES HOGG.
\section{OBSERVATIONS, etc. Observations sur la culture et l'usage des pommes de terre.}
Rapport du Comité du Département d'Agriculture, nommé pour extraire ses informations des rapports des Comtés.
Des différentes sortes de pommes de terre.
DANS le comté de Lancastre, on cultive pour le bétail l'espèce dite bœuf noble (ox noble) et la grappe (cluster.) L'ancienne
* *
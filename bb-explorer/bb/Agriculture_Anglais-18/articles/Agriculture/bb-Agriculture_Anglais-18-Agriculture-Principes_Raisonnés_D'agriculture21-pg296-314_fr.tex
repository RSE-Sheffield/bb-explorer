\setcounter{page}{296} rapide pour toute autre production du fien, à l'étendue d'environ un acre et demi. J'ai préparé du terrain, en grande partie tourbeux, pour faire une plantation au mois de septembre prochain, de l'étendue de huit acres. Je n'ai pas de raison de craindre qu'il ne réussisse pas aussi bien que le premier; et s'il réussit en effet, ce sera une grande source de richesse pour ce pays-ci.
JOHN BEARD.
PRINCIPES RAISONNÉS D'AGRICULTURE. Traduit de l'allemand d'A. THAER, par E. V. B. CRUD. Tom. Ier. in-4°, 372 p. Genève, 1811, chez J. J. Paschoud, Imprimeur-Libraire; et à Paris chez le même rue Mazarine, n°. 22. (Vingt-unième extrait. Voy. p. 225.)
LES matières fécales font un engrais très-actif; et lorsqu'on sait les employer, elles font plus d'effet qu'aucun autre engrais. L'auteur estime que si les excrémens humains étoient recueillis soigneusement, et qu'on en tirât tout le parti dont ils sont susceptibles, on obtiendroit par la plus grande produc-\setcounter{page}{297} tion de la terre, de quoi nourrir un million d'hommes de plus en Europe. Il y a malheureusement, un préjugé qui s'oppose à l'emploi de cet engrais.
L'effet des matières fécales est étonnant si on les charie sur les terres avant que leur fermentation soit achevée, et si on les répand avec soin. Il convient aussi de les employer en compost avec des gazons et de la chaux calcinée. Leur trop grande vigueur se trouve ainsi réduite à une mesure convenable. Ce fumier perd son odeur fétide, et se mêle à une terre féconde. Le meilleur moyen d'employer ce compost est de le répandre sur le sol sans l'enterrer. Il convient de remuer plusieurs fois le tas avant de s'en servir.
Il existe près de Paris, un établissement considérable, dans lequel on fabrique avec les matières fécales un engrais très-actif, qu'on nomme poudrette. On place ces matières sur une surface inclinée revêtue de plateaux de pierre. On les y met en tas, de manière qu'elles puissent entrer en fermentation, puis on les étend pour les dessécher. On les transporte ensuite sous des couverts, où elles se dessèchent complètement. On pulvérise cette matière, et elle se vend fort cher aux cultivateurs, sur-tout aux jardiniers.\setcounter{page}{298} Les habitants de la Belgique font beaucoup de cas de cette espèce d’engrais, et vont le chercher à de grandes distances, avec des charettes, et des bateaux, sans se laisser arrêter par le désagrément de l’odeur. Ils l’employent en compost, ou mélangé de beaucoup d’eau. Ce fumier est fort estimé à la Chine et au Japon, et c’est pour cette raison qu’on le nomme fumier du Japon.
Lors même que l’usage de recueillir le fumier des bêtes à cornes avec de la paille ne seroit pas le plus commode, on devroit le préférer parce que le mélange accélère la décomposition de la paille, et que celle-ci empêche l’évaporation des parties volatiles du fumier. Les tuyaux de la paille absorbent sur tout les parties liquides et l’urine.
Les procédés varient beaucoup pour recueillir le fumier sous les animaux. Il y a des gens qui le laissent long-temps dans l’étable en le recouvrant tous les jours de nouvelle paille, de manière que la couche s’élève, et qu’il faut des râteliers mobiles pour que le bétail ne soit pas trop haut. Cette méthode épargne du travail, et fait, à ce qu’on prétend, de meilleur fumier. Il commence à se décomposer à l’aide de son humidité naturelle, et comme il est peu exposé à\setcounter{page}{299} PRINCIPES RAISONNÉS D'AGRICULTURE. L'air atmosphérique, il perd peu par l'évaporation. Il absorbe même les vapeurs pesantes qui s'exhalent du bétail, et sont entraînées vers la terre. Ces faits sont hors de doute, et la crainte manifestée par plusieurs personnes, que les vapeurs qui s'élèvent du fumier ne nuisent au bétail, est sans aucun fondement. On ne remarque point de mauvaise odeur dans ces étables; l'air y demeure très-respirable, pourvu qu'on ne ferme pas toute entrée à celui du dehors. Le fumier qu'on s'est procuré de cette manière, surtout celui de la couche inférieure est d'excellente qualité, et lorsqu'on le déblaie, il a dépassé l'époque, où il s'évapore le plus par la fermentation : ses parties volatiles sont déjà réunies aux solides.
Mais cette méthode ne peut guère être mise en pratique lorsqu'on donne au bétail une nourriture abondante et composée de végétaux qui ont conservé leur suc, à moins qu'on n'emploie une énorme quantité de paille pour litière. Avec une telle nourriture, la masse d'excrémens que rend le bétail est si considérable que leur humidité ne peut pas être suffisamment absorbée par la litière, et qu'ainsi les bestiaux y enfoncent et sont toujours dans la fange.
Pour obtenir les avantages de cette longue\setcounter{page}{300} gue conservation du fumier dans les étables, et cependant éviter les inconvénients, la manière de construire les écuries que Schwerts indique dans le second volume de son Agriculture Belge, et qu'il explique par des planches est, sans aucun doute très-convenable. A côté de la place assignée au bétail et derrière celui-ci, est un espace au moins aussi large que celui qu'il occupe, mais un peu plus bas ; c'est là que le fumier est déposé à mesure qu'on le déblaie de dessous les bêtes ; et c'est également là que se réunissent les urines et l'humidité de l'étable. Lorsque le fumier a subi sa décomposition, on le conduit directement sur le champ auquel il est destiné.
Si dans la plupart des exploitations rurales on n'étoit pas arrêté par les frais que coûte cette place, du double plus grande qu'elle ne l'est ordinairement, cette méthode mériteroit décidément la préférence ; et devroit être adoptée universellement. Si les écuries sont assez larges pour qu'on puisse mettre le fumier en tas derrière le bétail pendant quinze jours ou trois semaines, c'est déjà un grand bien, parce qu'alors le moment où le fumier subit la plus grande évaporation se trouve déjà passé.
Il faut donc laisser le fumier dans l'éta\setcounter{page}{301} ble aussi long-temps que cela est possible, parce qu'il gagne d'autant plus en qualité. Mais cela doit toujours être subordonné à la convenance d'entretenir le bétail dans un état de propreté. Si on laissoit les bêtes dans la fange, on perdroit bien plus par les maladies qu'on leur attireroit qu'on ne gagneroit par l'augmentation de valeur du fumier. Les écuries humides occasionnent des maladies très-fâcheuses, que l'expérience nous apprend pouvoir devenir mortelles. Il n'est également pas douteux que si les vaches sont couchées mal proprement leur lait ne devienne mauvais.
Si on laisse le fumier sous le bétail, il faut avoir soin qu'il ne s'accumule pas une plus grande épaisseur sous les pieds de derrière que sous ceux de devant; parce qu'alors les animaux se trouveroient dans une position gênante. Cela peut avoir lieu d'autant plus facilement qu'on répand souvent de la paille pour couvrir le fumier qui tombe précisément dans cet endroit. Ce n'est donc que lorsque le bétail reçoit une nourriture sèche et pailleuse qu'il est possible de laisser la totalité du fumier, à moins, cependant que l'étable ne soit planchéiée avec des plateaux qui laissent au-dessous d'eux un espace vide, et entre lesquels l'humidité\setcounter{page}{302} s'écoule. Cette dernière méthode est suivie dans quelques contrées où le fumier a peu de prix.
Mais la méthode la plus usitée consiste à transporter d'abord les engrais d'étable dans des places à fumier, où on les laisse pendant un temps plus ou moins long, en tas de diverses grandeurs avant de les transporter sur le sol.
Ces places à fumier sont arrangées de diverses manières. Quelquefois elles ont un grand enfoncement à-peu-près comme une fosse. Cette forme est essentiellement vicieuse; non seulement parce que l'humidité qui s'amasse dans ces fosses empêche la fermentation et la décomposition du fumier; mais encore parce que celui-ci est privé du contact de l'air atmosphérique. Outre cela elle rend difficile le transport du fumier, qui est ordinairement plein d'eau lorsqu'on le charge, et qui laisse ainsi tomber goutte à goutte sur le chemin, ses parties les plus fertilisantes. Les inconvénients de ces fosses pour le fumier des bêtes à corne, sont si généralement connus qu'on n'en voit presque plus, excepté dans les lieux où l'on manque d'espace pour étendre et amonceler les engrais.
D'autres personnes, au contraire, con\setcounter{page}{303} vaincues des désavantages d'une position trop humide, placent leurs fumiers sur une surface plate ou même sur une élévation; mais alors ils perdent trop leur humidité, et ils sont privés de leurs parties les plus actives. Ce qui paroît le plus avantageux, c'est de creuser légèrement, d'incliner un peu d'un seul côté, et d'avoir une ouverture par laquelle les eaux surabondantes se jettent dans un réservoir destiné à les conserver. La place des fumiers doit être entourée d'une bordure élevée qui empêche qu'il ne s'y mêle aucune eau étrangere: de cette manière l'humidité n'y sera point trop grande, lors même qu'on y réuniroit toutes les urines des étables. Le fumier absorbe l'humidité naturelle des excrémens, et les eaux de pluie qui tombent immédiatement dessus: le superflu s'évapore. Le meilleur moyen de profiter les urines, est de les incorporer au fumier pailleux. On a proposé d'établir des couverts sur les fumiers pour les garantir des eaux de pluie et de l'action du soleil; mais un tel couvert est coûteux, dure moins qu'il ne feroit ailleurs, à cause des vapeurs chaudes qui s'élèvent du fumier, et le chariage des engrais en est presque nécessairement embarrassé si l'on emploie plusieurs attelages à-la-fois.\setcounter{page}{304} Si l'on ne veut charier les fumiers que lorsqu'ils sont très-pourris, il faut que la place à fumier ait plusieurs compartimens, qui se remplissent et se vident à leur tour : sans cela on est forcé de transporter du fumier frais avec le fumier consommé, d'employer beaucoup de temps à déblayer auparavant le fumier frais.
Lorsqu'on a beaucoup de place et beaucoup de différens terrains, il y a de l'avantage à séparer les divers fumiers. On met alors le fumier de cheval dans une fosse étroite et profonde, en ayant soin de l'arroser pour modérer sa chaleur. Le mélange du fumier de cochon avec celui-là est très-convenable. En général il convient plutôt de mêler ensemble tous les fumiers, parce que la trop rapide fermentation du fumier de cheval et de mouton, est retardée par la fermentation plus lente de celui des bêtes à cornes.
Dans beaucoup d'exploitations rurales, on laisse le fumier tout l'hiver sous les montons, en ajoutant toujours de la nouvelle paille. Il y auroit de grands inconvéniens à enlever le fumier de mouton pendant l'hiver, lors même que les bêtes sortiroient pendant le jour. Lorsque le fumier a été accumulé et qu'on le remue, il s'en exhale des vapeurs ammoniacales très-piquantes qui pourroient nuire aux bêtes. La\setcounter{page}{305} Des expériences répétées ont prouvé que le fumier acquiert plus de force, et ne diminue pas autant de volume, lorsqu'on le prive du contact de l'air, pendant la plus forte fermentation. Le couvrir de terre, entraîne trop de main-d'œuvre; mais la couche pailleuse supérieure sert de couvert aux couches inférieures, et ne fermente pas sensiblement, tandis que la masse fermente. Les gaz sont ainsi arrêtés, et comme ils sont plus pesans que l'air atmosphérique, il est probable qu'ils se combinent de nouveau avec le fumier. On ne remarque pas d'odeur sensible au-dessus d'un fumier disposé de cette manière: ce n'est que quand on le remue que l'odeur se manifeste. Ceci prouve que, dans ce dernier cas, il se dégage beaucoup d'acide carbonique, d'azote et d'hydrogène, mais que lorsque le fumier reste en repos, et, en partie à l'abri du contact de l'atmosphère, ces substances, au lieu de s'évaporer sous la forme de gaz, entrent plutôt dans de nouvelles combinaisons.
Il est très-essentiel d'étendre le fumier d'une manière égale, et sur une surface qui ne soit pas trop grande. Si on le jette sur le tas en petits monceaux, on perd les avantages de cette espèce de couverture; entre les monceaux, il se forme des vides où la moisis-\setcounter{page}{306} sure ne tarde pas à paroître ; et l'on sait que celle-ci nuit beaucoup à la qualité du fumier. Il est évident que le fumier ainsi disposé par lits, gagne à être un peu comprimé ; c'est pourquoi il convient de l'entourer d'une balustrade, afin que le bétail qu'on fait sortir de l'étable se promène par dessus. Je sais que quelques auteurs ont envisagé comme nuisible la compression du fumier, mais je ne puis pas avoir cette opinion, car dans un lieu où chaque jour, plusieurs chariots passoient par dessus le tas, j'ai obtenu un fumier de la meilleure qualité, et parfaitement décomposé.
Lorsqu'une partie du tas se trouve élevée de cinq à six pieds, qu'on veut le consommer d'une manière uniforme, et que, par conséquent, on commence un autre tas, il est très-convenable de couvrir le premier avec un lit de terre ou de gazon. Ainsi couvert, le fumier subit une putréfaction égale sans qu'il s'en évapore une partie sensible. Les vapeurs qui s'en élèvent sont absorbées par la terre, et lorsqu'on charrie le fumier, on met au fond de la fosse, les gazons qui étoient au-dessus, et qui ne sont pas encore décomposés ; de cette manière, ces gazons sont transformés en une espèce d'engrais très-riche.
Pour empêcher qu'il ne se perde aucune\setcounter{page}{307} partie des urines et des engrais liquides, par l'infiltration en terre, on a conseillé de faire battre l'aire de la place à fumier, ou de la faire paver avec de petits cailloux; de la garnir de pierraille, ou même, de la revêtir de mortier ou de ciment, afin qu'elle soit imperméable. Si le sol est naturellement argileux, ces précautions sont superflues; s'il est sablonneux, elles peuvent être utiles lorsqu'on établit une place à fumier; mais si la place a déjà été employée à cet usage, on peut se dispenser de ce soin; même sur du sable; parce que quand celui-ci a une fois été suffisamment imprégné, et comme saturé du suc de fumier, il ne paraît pas pouvoir en absorber davantage. J'ai trouvé le sol d'une place dans ce cas-là, imprégné d'eau de fumier, et tout noir; mais au-dessous de cette couche, il y avait du sable blanc pur parfaitement séparé : il ne me paraît donc pas que le fumier pût s'y infiltrer plus avant. Lorsqu'on a vidé la place à fumier, et qu'on veut y commencer un nouveau tas, il convient de mettre sur l'aire une couche de toutes sortes de substances végétales d'une décomposition difficile, de feuilles d'arbres, d'herbes sèches, de tiges de plantes, de terreau de bois ou de gazons. En un mot tout\setcounter{page}{308} ce qui est propre à absorber le liquide du fumier, et qui, après sa putréfaction, peut être employé comme engrais. En Suisse, où l'on donne un très-grand soin à toutes les manipulations, le fumier fait avec une litière de paille, est mis en tas régulier, tandis qu'on emploie séparément les urines après les avoir recueillies à leur sortie des écuries et renfermées dans une purinière. De la partie la plus pailleuse du fumier, on forme les bords du tas, et pour cet effet on la ploie en deux avec une fourche, de manière que le fumier proprement dit, y soit renfermé, et mis hors du contact de l'atmosphère. Ces tas sont élevés perpendiculairement de cinq à six pieds, et disposés avec le plus grand soin. Ils ont alors l'apparence d'une grande ruche de paille, parce qu'en dehors on ne voit que cette paille retroussée qui est arrangée avec une uniformité parfaite. En temps de sécheresse, on arrose ces tas avec du purin ou avec de l'eau, afin de leur conserver l'humidité nécessaire à la fermentation. Le fumier qui y est renfermé, quoiqu'on lui ait enlevé une partie des urines, devient alors excellent, homogène, et gras. Etat qui, comme nous l'avons dit, est désigné sous le nom de beurre-noir. De cette manière on a la possibilité d'employer le fumier au\setcounter{page}{309} degré de décomposition qu'on préfère, parce que les tas demeurent séparés les uns des autres. Ce sujet vaut assurément la peine qu'on fasse des expériences comparatives pour déterminer les avantages ou les inconvénients de l'une ou l'autre méthode.
Les opinions sont extrêmement partagées sur le temps où il convient de charier le fumier dans les champs, et l'état dans lequel il faut qu'il soit alors. Le plus grand nombre des agronomes s'en sont tenus au principe qu'il ne faut charier les fumiers que lorsqu'ils sont entièrement consommés, et que le tissu de la paille dont ils sont composés a perdu son agrégation, sans être entièrement détruit; lorsqu'ils se laissent pénétrer d'une manière uniforme ou qu'ils sont en consistance de graisse. Le fumier atteint cet état dans un espace de temps plus ou moins long, suivant qu'il conserve un degré d'humidité plus convenable, et que la température est plus ou moins élevée; en été huit ou dix semaines suffisent; en hiver, il en faut vingt et au-delà. Le fumier qui dans cet état a entièrement perdu sa chaleur de fermentation, ne donne des vapeurs que dans les premiers moments de son déplacement, d'abord avec une odeur fétide de pourriture, ensuite avec une odeur musquée; il a une couleur jaunâtre qui ne tarde pas à devenir brune\setcounter{page}{310} quand ce fumier est exposé à l'air. Lorsqu'on le répand sur le terrain, il prend l'apparence d'une tourbe charboneuse; mais il absorbe promptement l'humidité et se devise; alors il peut être mêlé d'une manière uniforme avec la couche de terrain qui est en labour. D'autres donnent la préférence au fumier long et non décomposé, et cherchent à disposer les choses de manière qu'ils puissent le transporter directement de l'étable au champ. Si ce fumier a déjà subi dans l'étable sa principale fermentation, sa couche inférieure, est dans l'état qu'elle eût atteint, si elle eût été déposée dans la place à fumier; et en hiver, sa fermentation est plus accélérée dans l'écurie, parce que la température y est plus chaude. Quelquefois aussi l'on transporte dans les champs le fumier tout frais et pailleux, et on l'enterre aussi bien que cela est possible; dans quelques cas, on a cru en avoir éprouvé des effets plus sensibles que du fumier consommé. Pour les terrains froids et tenaces; ce dernier procédé doit sans aucun doute; être mis en pratique, lorsque les circonstances de l'exploitation n'y mettent pas d'obstacle; mais alors il faut avoir un soin particulier de faire entrer le fumier dans le sillon de\setcounter{page}{311} manière qu'il soit bien couvert par la terre.
Le fumier a assez de force pour commencer là sa fermentation, pour s'échauffer, pour communiquer sa chaleur au sol, pour y introduire de l'air par les ouvertures que la paille pratique, et par ce moyen, aussi bien que par le développement des gaz, il maintient le terrain léger et bien imprégné. L'ammoniaque qu'il produit, agit avec force sur l'humus insoluble que ces terrains contiennent. Il en résulte divers effets, comme de mettre en action les parties nutritives que le sol contient encore, tandis que le fumier consommé n'opère cet effet que très-imparfaitement. En revanche sur les terrains secs, on retire peu ou point d'avantages du fumier non-consommé. Lorsqu'il est enterré peu de temps avant la semaille et qu'il n'a pas le temps de se décomposer, si la sécheresse survient, les plantes en souffrent beaucoup plus. Si au contraire il tombe beaucoup de pluie, les plantes prennent une croissance plus rapide, mais elle finissent par jaunir, une partie d'entr'elles périssent ou demeurent foibles : elles sont sujettes à la rouille et ne donnent que des grains imparfaits.
Lorsque le fumier s'est desséché sur le sol et qu'on l'enterre, il ne se divise pas de quelques années, il ne se mêle pas avec la terre\setcounter{page}{312} végétale, et c'est seulement assez long-temps après qu'il est transformé en terreau fertilisant, parce qu'il ne peut plus entrer en fermentation, mais seulement se diviser: c'est ce qui a donné lieu à cette maxime, que le fumier qui n'opère pas sur la première récolte n'a point d'effet sur la seconde. Il est donc très-important de charier et d'enterrer le fumier dans un état qui soit en rapport avec les besoins du sol.
Il est dangereux de remuer le fumier lorsqu'il est au plus haut degré de la fermentation. Suivant toutes les apparences, une partie des substances les plus actives qu'il contient s'évapore, lorsqu'à cette époque il est mis en contact avec l'air. Mais si on le déplace avant ce moment-là, ou après que la fermentation est affoiblie, il ne paroît pas qu'il y perde rien, ou du moins il regagne d'une autre manière.
Il y a des avantages évidens à étendre en hiver le fumier récent et pailleux sur le sol, et à l'y laisser jusqu'aux labours du printems; pourvu que l'eau n'en emmène pas les sucs hors du champ, mais que plutôt elle les entraîne dans la terre: cette manière de couvrir le sol pendant l'hiver le rend beaucoup plus meuble et remarquablement fertile. On ramasse quelquefois la paille lavée et non pourrie du fumier\setcounter{page}{313} ainsi répandu, et on s'en sert de nouveau pour litière. J'ai trop souvent éprouvé les bons effets du fumier mis sur des pois et des vesces, et qu'on y avoit laissé pendant leur végétation, pour n'être pas convaincu des bons effets de cette méthode sur un terrain chaud, meuble et d'une force moyenne: lorsqu'on avoit semé tard, sur-tout, elle m'a toujours procuré une belle récolte de ces deux espèces de grains. Mais ce qui paroît plus remarquable et difficile à expliquer, c'est qu'aux récoltes suivantes, le terrain qui avoit été traité de cette manière avoit la supériorité sur ceux dans lesquels on avoit enterré une plus grande quantité de fumier consommé.
En 1808 je semai de la navette de printemps avec du trèfle sur un terrain maigre, que je couvris ensuite de fumier récent et pailleux. Dans l'automne 1809, je fis rompre le trèfle et il fut ensemencé en seigle qui, au printems 1810 se distingua très-avantageusement de celui du champ voisin qui avoit été fumé en été, sur la jachère.
D'après un grand nombre d'expériences comparatives, il paroît presque hors de doute que le fumier qui a dépassé le point de la plus forte fermentation, lorsqu'il est répandu sur le sol, même pendant la saison la plus chaude et durant la sécheresse, non seule-\setcounter{page}{314} ment ne perd rien de sa qualité, mais même gagne encore. Cela semble d'abord incroyable, parce qu'on pense que le fumier doit nécessairement perdre par l'évaporation, et cette supposition paroît tellement vraisemblable, que partout on conseille de se hâter d'enterrer le fumier dès qu'il est répandu; mais probablement l'évaporation du fumier consommé n'est point aussi considérable que cela semble devoir être. Il s'opère bien quelque décomposition, lorsqu'il est dans un état d'humidité, parce qu'alors il absorbe de l'oxigène, et qu'il s'y développe de l'acide carbonique: mais il est vraisemblable que cet acide carbonique est entraîné par l'eau dans le sol, et que même il contribue à l'amender. Pendant la sécheresse, aucune décomposition n'y a lieu. Si l'on examine un champ en jachère, à la surface duquel le fumier est demeuré ainsi répandu pendant quelques semaines, on y verra une abondante quantité de jeunes plantes d'une couleur vive, même dans les places qui n'étoient pas en contact avec le fumier; ce qui prouve que la faculté améliorante de celuici sé répand autour de lui, même avant qu'il soit recouvert par la terre et absorbé par elle.
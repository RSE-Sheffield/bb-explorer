\setcounter{page}{112}
\section{A V I S. \large{( Extrait des feuilles d'Hofwyl ).}}
On sait que la réussite d'une récolte de grains dépend, en grande partie, des semailles. La quantité de la semence ne doit être ni trop forte ni trop foible, mais proportionnée à la qualité et à l'état du terrain. Il importe que les grains de semence\setcounter{page}{113} soient placés à une profondeur uniforme, et en rapport avec la qualité de la terre; que leur distance entr'eux soit aussi égale qu'il est possible. Cela ne peut s'obtenir que d'une manière imparfaite lorsqu'on sème à la volée. Une grande partie des grains reste à la surface du champ, et est mangée par les oiseaux ou détruite par la gelée et la sécheresse. Les grains qui se trouvent enterrés trop profond sont perdus, ou ne lèvent que long-temps après les autres. Il en résulte des places vides, qui sont occupées par les mauvaises herbes.
Partout où l'agriculture est parvenue à un certain point de perfection, on a donc remplacé l'usage de semer à la main, par celui du semoir, qui prévient les objections ci-dessus énumérées. On y a réussi, du moins jusqu'à un certain point, par divers semoirs connus en Angleterre, en France, en Allemagne, et sur-tout en Suisse. On a épargné, au moyen de cet instrument, et selon les terrains, depuis un tiers jusqu'à deux tiers de la quantité de semence nécessaire pour semer à la volée: on a obtenu des récoltes plus assurées, plus abondantes en grains et en pailles; et les champs en ont été maintenus plus exempts de mauvaises herbes.
Il y a douze ans que l'on fait usage du semoir à Hofwyl; mais les instrumens employés\setcounter{page}{114} jusqu'ici ont été compliqués, ou trop fragiles, ou trop chers pour la plupart des cultivateurs. Enfin, aucun de ces instruments n'avoit l'avantage de pouvoir semer également toutes les espèces de graines. On fabriquoit donc, à Hofwyl, des semoirs de diverses constructions: les uns servoient pour les graines très menues, d'autres s'employoient à semer des fèves, et des pois, d'autres enfin à semer les graines céréales; mais ces derniers ne semoient pas l'avoine d'une manière pleinement satisfaisante. L'un de ces instruments revenoit à onze cents francs de Suisse; un autre à huit cents francs, un autre à six cents francs, un autre enfin à quatre cents francs. Il restoit toujours quelque chose à désirer: aucun de ces semoirs ne pouvoit semer à volonté, les graines de toutes les grosseurs, à la profondeur convenable, en lignes, à des distances déterminées à choix, ou en imitant la semaille à la main, laissant à la volonté de l'agriculteur la détermination de la quantité de graine à répandre sur une étendue donnée; enfin, aucun de ces instruments ne présentoit l'avantage de pouvoir semer le trèfle dans les intervalles qui séparoient les lignes des céréales, et de manière à ne laisser aucun vide, après la récolte de celles-ci.
Le problème à résoudre nous paroissait néanmoins d'une si haute importance pour\setcounter{page}{115} L'agriculture en général, et pour la Suisse en particulier, que nous n'avons voulu abandonner nos recherches et nos essais sur cet objet, pour nous occuper de la machine à battre, qu'après avoir complétement atteint le but, ainsi que nous l'avions fait précédemment pour la houe-à-cheval. Depuis l'automne dernière, nous avons enfin réussi à construire un semoir, sur un principe tout-à-fait différent de ceux qui ont été connus jusqu'ici. Il est simple; il remplit très-bien les divers objets indiqués ci-dessus; il est durable, et peu cher. Avec cet instrument, deux hommes et un cheval peuvent ensemencer jusqu'à quatre cent mille pieds de Berne de surface, soit dix poses dans la journée\footnote{La pose de Berne est à-peu-près l'arpent de Paris, ancienne mesure. (R)}.
Le but se trouvant ainsi complétement atteint, nous n'essayerons plus aucun changement ou perfectionnement de ce semoir; et comme il importe, soit à la parfaite fabrication de l'instrument, soit à son bas prix, de pouvoir, dès le début, l'exécuter en fabrique, en divisant le travail entre les ouvriers, nous desirerions fabriquer en même temps quelques centaines de semoirs semblables. Cela réduiroit le prix tellement, que chaque petit cultivateur, qui sème de cinq à dix poses, trouveroit son compte à faire les frais de cet instrument.
En conséquence, nous invitons les agriculteurs qui desireroient acquérir un semoir perfectionné, à venir prendre connoissance de cet instrument à Hofwyl, ou à le faire examiner et essayer par des connoisseurs.\setcounter{page}{116} et s'ils en sont contens, comme nous le pensons, à s'inscrire, ou se faire inscrire au bureau. Il n'y a jusqu'ici qu'une vingtaine de ces instrumens commandés. Si la commande pouvoit s'élever à un millier d'instrumens, il en résulteroit une telle économie sur leur fabrication, que nous oserions promettre, que sur une exploitation de quinze poses de semature, le semoir seroit payé dès la première année, par l'économie de la semence, et l'augmentation des produits. L'instrument est d'ailleurs si solide, qu'il passera aux enfans et petits-enfans, avec le même avantage.
Dans trois mois, à compter de la publication de cet avis, nous ferons connoître aux souscrivans à quel prix sera fixé le semoir, en conséquence du nombre de ceux qui auront souscrit. Après cette communication, les souscrivans seront encore libres de se retirer. S'ils persistent, ils devront nous faire connoître lequel des trois semoirs suivans, ils desirent avoir:
1º. Celui qui sert à semer le blé, le seigle, l'orge, l'avoine, les pois, les vesces, le chanvre, le maïs, les betteraves, et les carottes.
2º. Celui qui, outre les grains ci-dessus, sème le trèfle, les raves, les pavots, le colza, etc. et peut, à volonté, répandre les semences comme à la volée ou ou en lignes plus ou moins rapprochées.
3º. Celui auquel est adapté, en outre, un mensurateur ou machine au moyen de laquelle, on peut, dans tous les momens de la semaille d'une journée, ou d'une pièce, savoir précisément l'étendue de terrain déjà ensemencée, et par conséquent, combien de grain on a semé sur une surface déterminée.
Les instrumens seront livrés selon la date de l'inscription.
Signé, le chef des Instituts d'Hofwyl.
EMANUEL DE FELLENBERG.
Hofwyl le 15 mars 1813.
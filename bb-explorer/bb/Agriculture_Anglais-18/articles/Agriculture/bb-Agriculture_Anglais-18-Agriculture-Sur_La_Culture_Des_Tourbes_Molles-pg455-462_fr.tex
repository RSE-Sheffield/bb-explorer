\setcounter{page}{455}
\section{A ACCOUNT OF SEVERAL ATTEMPTS, etc. Exposé de quelques essais infructueux, entrepris dans le but de cultiver des tourbes molles. ( Farmer's Magazine ).}
Il y a actuellement seize ans que j'entrepris de destiner à la culture, une tourbière molle de six à douze pieds de profond, sur laquelle rien ne végétoit qu'un peu de bruyère et d'herbe marécageuse. Ce qui m'y engagea, ce fut le voisinage où ce morceau tourbeux se trouvoit de ma maison, et en même temps une publication, faite dans le temps, des succès de Mr. Smith de Swinridgemuir. J'essayai donc de suivre les indications de Mr. Smith, espérant ajouter à mon revenu, et convertir en effet, cette tourbe en un sol fertile. Je fus très-mortifié de trouver que ma première récolte, qui étoit des pommes de terre en lazy-beds, étoit presque nulle ; mais une fois engagé, il falloit poursuivre, ou me soumettre à la Agricul. Vol. 18. Nº 12. Déc. 1813. Nn\setcounter{page}{456} raillerie de mes voisins, qui regardaient mon entreprise comme absurde.
L'année suivante, j'essayai de fumer avec du fumier frais, pour remettre encore des pommes de terre. J'en mis aussi dans une portion nouvelle que j'avois préparée à la manière de Mr. Smith. Je les soignai bien pendant l'été; mais je ne préparai pas une plus grande étendue de terre, parce que je voulois voir le succès de cette seconde année. Il fut presque nul. Je commençai à avoir de grands doutes sur la convenance de persévérer dans le même plan. Cependant, l'hiver suivant, je profitai d'une longue gelée, pour faire charier, d'une hauteur voisine, une certaine quantité de gravier, que je fis étendre à un pouce d'épaisseur sur ma tourbière, que j'avois préalablement nivelée et bien chaudée, dans une portion nouvelle. J'y plantai au printemps des pommes de terre en lazy-beds. Je ne recueillis presque rien. La même année, je semai l'avoine sur la portion précédente : la récolte fut très-misérable. L'année suivante, je semai le tout en avoine avec des graines de prés. Sur la première portion, l'avoine manqua, et le foin fut nul. Sur la portion gravelée, à laquelle j'avois ajouté encore de la terre par dessus, l'avoine donna quatre quarters par\setcounter{page}{457} acre, et il y eut une petite récolte de foin? Je me décourageai, et j'abandonnai la plus mauvaise portion: quant à l'autre, qui touche à un champ, je l'ai soumise à un assolement régulier, quoiqu'elle paie à peine les frais annuels. A l'égard des dépenses primitives d'amélioration, je les considère comme décidément perdues.
Depuis ce temps-là, j'ai essayé plusieurs fois et par divers procédés, de mettre en valeur des tourbes molles; mais j'ai toujours trouvé que c'étoit une opération ruineuse. J'ai observé aussi, que tous ceux qui avoient entrepris de faire porter du grain à des tourbes molles, avoient fini par abandonner l'entreprise.
Voici à quelle occasion j'écris ceci. Je lis dans votre Journal, Vol. XII; p. 152, la lettre d'un de vos correspondans, qui dit alors "qu'on réussit à dessécher la surface des tourbes molles, on peut leur faire rendre comme aux meilleurs terrains, des grains et des plantes de prés." Il cite ensuite les cultures qui réussissent sur les tourbes. Tout cela fait fort bien sur le papier. Il ajoute "il est inutile de raisonner avec les ignorans, les paresseux, et les gens prévenus: un exemple de succès fait plus d'effet que tous les raisonnemens." Jusqu'ici\setcounter{page}{458} ques là, c'est fort bien; mais il ajoute: "dix à quinze quarters d'avoine d'un acre de tourbe, sont une récolte qui fait ouvrir les yeux." Alors je redoublai d'attention, comme de surprise. Mais ce fut bien autre chose quand je vis qu'il citoit, pour de pareilles récoltes, précisément les mêmes tourbes molles que je sais, de science certaine, n'avoir jamais produit jusqu'à deux quarters par acre, savoir, les tourbes de MM. Mackenzie et Miller, près de Glasgow. Il est très-vrai que l'un et l'autre ont fait de grandes expériences, très-bien conduites, pour rendre les tourbes productives; mais si votre correspondant avoit vu les produits pendant leur végétation et sur-tout à la récolte, il ne citeroit pas ces opérations pour en encourager de semblables. Il paroit que Mr. Aiton s'est trop fié aux rapports d'autrui. Il seroit utile de publier un exposé bien exact des frais, des produits, et de toutes les circonstances de cette amélioration des tourbières.
Je suis à portée de rendre un compte semblable sur les opérations et les résultats, dans les tourbes de Garnkirk, parce que j'en ai été témoin. Cette plaine tourbeuse est divisée en deux parties presqu'égales, par la route de Cumberland à Glasgow.\setcounter{page}{459} gow. Mr. Mackenzie commença d'un côté de la route, et fit un grand fossé autour de la portion qu'il entreprit. D'autres fossés divisèrent le terrain en bandes de dix-huit pieds de large. La surface fut ensuite nivelée. On charia la glaise d'un champ voisin, en grande quantité, puis on la transporta sur la tourbe, avec des brouettes à bras. On amena aussi du fumier de Glasgow, et on la transporta de même dans des brouettes à bras, sur des planches. Au printems, on sema de l'avoine, et au lieu de la herser, on l'enterra en sarclant à la main. L'avoine leva et épia bien; mais elle valut à peine les frais de moisson, excepté dans un petit espace, qui depuis plusieurs années avait été couvert par les eaux toutes les fois qu'il pleuvoit abondamment. L'année suivante, on planta des pommes de terre sur une grande partie de ce champ, après un labour à la bêche; elles furent bien cultivées pendant l'été; mais les tubercules ne dépassèrent pas la grosseur d'une prune. L'année d'après, on suivit la même manière pour la partie au-delà de la route, et on y sema des pois, d'après Mr. Aiton, qui assure qu'ils donnent une belle récolte dans ce genre de terrain. Je suppose que Mr. Aiton, ou celui qui lui a communiqué cette expérience, avoit vu la\setcounter{page}{460} récolte dans le moment le plus favorable de la croissance ; car je puis assurer qu'au moment de la moisson le produit se trouva trop faible pour payer les frais.
Je pourrois encore, monsieur, vous donner des détails sur des expériences très-bien faites, et à grands frais, par Jeffrey et Miller, et celles de Sir Charles Edmondston sur un terrain tourbeux, dirigées par James Davidson ; mais tous leurs résultats ont été semblables aux miens ; et en conséquence le terrain a été abandonné à son premier état, et nourrit aujourd'hui de la bruyère et des canards sauvages. Dans le douzième volume du Farmer's Magazine, Mr. Aiton dit, en parlant des fermiers: "si on leur dit soit qu'on peut faire croître du grain dans une tourbe molle, on ne feroit qu'exciter leur mépris ; ils assureroient qu'on ne peut rien faire de pareil sur les tourbes de leurs fermes." Dans un autre endroit, il dit encore : "on a trouvé que le lin réussissoit très-bien dans les terrains tourbeux" ; et pour confirmer cette assertion il ajoute : "le plus beau lin qu'on ait vu dans cette partie du pays, avoit été produit par un terrain tourbeux, chez Mr. John James de Springkell, dans le comté de Dumfries." Je ne mets nullement en doute la véracité de Mr. Aiton ;\setcounter{page}{461} mais j'ai tout lieu de croire qu'on lui avoit fait des rapports peu corrects. S'il avoit été modéré dans ses citations de récoltes abondantes, il auroit probablement persuadé quelques jeunes fermiers d'adopter sa théorie favorite; mais dire à un agriculteur qui a quelque expérience, qu'on peut faire de superbes récoltes de froment, de fèves, et de lin, dans une tourbe molle, c'est lui donner l'opinion que l'auteur n'entend rien au sujet qu'il traite. Cette supposition n'excitera pas son mépris; mais il dira probablement que de telles récoltes ne croîtroient pas dans les tourbes molles de sa ferme... Je m'arrête, parce que je ne pense pas que les écrits de Mr. Aiton puissent induire en erreur beaucoup d'agriculteurs pratiques: c'est de la théorie pure, comme ce qu'il a publié sur les laiteries.
Quoique je ne conseille pas d'appliquer au grain les tourbes molles, je suis très-éloigné de les considérer comme inutiles: elles sont, au contraire, un trésor inépuisable comme engrais, si on les fait entrer dans les composts. J'en ai tiré un grand parti de cette manière. Elles sont aussi susceptibles de recevoir des plantations. Les mêmes tourbes de Garnkirk, qui ne peuvent pas porter des grains, nourrissent des pins et d'autres\setcounter{page}{462} tres arbres. En choisissant les espèces, je crois que c'est le meilleur parti à tirer des tourbes molles. J'observe, cependant, que tout ce que j'ai dit ne s'applique point aux tourbes qui ont une consistance suffisante, et qui se couvrent naturellement d'herbe. Je crois qu'on peut les cultiver avec profit. Je n'ai rien dit qui ne fût fondé sur mon expérience et sur l'observation des faits.
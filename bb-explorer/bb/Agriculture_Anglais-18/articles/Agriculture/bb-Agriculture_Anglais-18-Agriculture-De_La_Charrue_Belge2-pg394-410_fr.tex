\setcounter{page}{394}
\section{DE LA CHARRUE BELGE. Par Ch. PICTET. SECONDE PARTIE}
J'AI dit, dans le Numéro de septembre, en quoi me paroissoit consister la supériorité de la charrue belge bien construite, sur notre charrue du pays. J'ai suivi la marche de deux de ces charrues, attelées alternativement de quatre bœufs, de deux bœufs et un cheval, de quatre chevaux, et de trois chevaux, pour les labours de semailles de soixante-poses (. de 25600 pieds.) soit dans des trèfles et luzernes à rompre, en terre argileuse, ou légère, soit pour rompre des prés-gazons, soit pour labourer en divers terrains après les pommes de terre, ou en rompant des chaumes. J'ai mesuré jour par jour, le travail de vingt-six jointes ou reprises, en tenant compte du temps employé, et du nombre des animaux de trait. Deux laboureurs de mon voisinage, qui, depuis long-temps, labouroient mes terres, à raison de 12 francs\setcounter{page}{395} coupe \footnote{20480 pieds de surface, soit les quatre cinquièmes de la pose.}, de semature, se sont engagés à faire, pour mon compte, avec la charrue belge, des reprises de trois heures, pour le prix d'une piastre. Comme ordinairement leur jointe est de quatre heures, ils ont presque toujours fait pour leur propre compte, une troisième jointe de deux ou trois heures, quand ils en faisoient deux pour moi. Ainsi, ils atteloient entre quatre et cinq heures du matin, travailloient trois heures, faisoient manger leurs bœufs pendant deux heures, sur le lieu même, le foin qu'ils avoient apporté, puis faisoient une seconde reprise, jusqu'à une heure après midi. Le soir, ils labouroient pour leur propre compte. De cette manière, je n'ai eu que les deux tiers de la puissance de travail des bœufs qui labouroient pour moi. La seconde jointe, qui se fait dans le gros du jour, est la moins bonne des trois. Quand je n'étois pas présent, les repos, aux deux bouts du champ, étoient probablement un peu plus prolongés; enfin, il est arrivé plusieurs fois que le travail, qui devoit être fait avec quatre bœufs, l'a été avec deux bœufs et un che-\setcounter{page}{396} val, chose que j'ai tolérée, parce que l'ouvrage pressoit, et que les charrues étoient de requête.
J'ai labouré environ trente poses, c'est-à-dire, la moitié du tout, avec mes chevaux de voiture. Ce sont de vieux chevaux normands, accoutumés aux chariages, mais non à la charrue; il a donc fallu les ménager. J'en ai toujours attelé au moins trois, ordinairement quatre: leur reprise presque toujours de deux heures, n'a jamais passé deux heures vingt-cinq minutes. J'ai tenu compte de leur travail, jour par jour.
En mettant ensemble tout le travail fait par les bœufs, et dont j'ai mesuré le temps et l'espace, dans les trèfles, les luzernes et les prés-gazons à rompre, je trouve 5106 pieds de surface labourés par heure, soit 30636 pieds dans les deux reprises de trois heures chacune.
Le travail des chevaux employés dans les mêmes champs à rompre les trèfles, les luzernes et les prés-gazons a été de 7716 pieds de surface par heure, soit 30864 pieds dans deux reprises de deux heures chacune.
Dans les terres argileuses très-compactes, les bœufs labourant après qu'on avoit enlevé les pommes de terre au fossoir, ont\setcounter{page}{397} fait 5214 pieds de surface par heure, soit 31084 dans deux reprises de trois heures.
Le travail des chevaux, dans les terres légères, après l'arrachement des pommes de terre a été de 10500 pieds par heure, soit de 42,000 pieds dans les deux reprises de deux heures.
Tout le travail ci-dessus, quelle que fût la nature du sol, a été fait à une profondeur de six pouces et demi à sept pouces et demi, soit de sept pouces en moyenne. Cette profondeur peut se régler plus exactement et plus facilement avec la charrue belge, qu'avec une autre, parce que le support qui sert d'avant-train, une fois fixé, la profondeur du labour est très-uniforme.
Il est facile de comparer maintenant les frais de labour avec la charrue du pays, et ceux de la charrue belge.
La moyenne du travail des bœufs dans les terres argileuses, et pour rompre les luzernes, les trèfles et les prés-gazons, a été de 30960 pieds dans les deux reprises. Cette étendue me coûte 10 francs 40 centimes. Il m'en coûte donc 6 francs 88 centimes pour le labour d'une coupe de semature ( soit 20480 pieds ) au lieu de 12 francs que je payois pour le travail de la charrue du pays;\setcounter{page}{398} or, si l'on observe que ce prix de 12 francs est le même pour les terres les plus légères et du travail le plus facile, tandis qu'ici il ne s'agit que des champs les plus difficiles à labourer; si l'on fait attention aux circonstances que j'ai indiquées ci-dessus, comme des causes de temps perdu, ou d'affaiblissement des animaux qui travailloient, on se convaincra que l'économie d'argent est au moins de cinquante pour cent dans l'emploi de cette charrue, comparativement à celle du pays. Cependant l'économie d'argent n'est peut-être pas le plus grand avantage de cet instrument : l'économie de temps a des conséquences plus importantes. Pendant la plus grande partie du mois d'août, la marche de la charrue du pays a été suspendue par la sécheresse. Il étoit impossible de rompre les trèfles, même dans les terres moyennes, à plus forte raison dans les terres argileuses. La charrue belge n'a pas été arrêtée un seul jour; et quoique sa flèche soit mince, et sa construction très-légère, elle a résisté sans le moindre dommage au labourage des terres les plus compactes, reliées par la sécheresse, ainsi que par les racines de trèfle ou de luzerne. Les agriculteurs qui ont l'habitude d'une\setcounter{page}{399} grande exploitation, sur-tout dans les sols argileux, savent de quelle importance il est de pouvoir expédier les labours dans le temps où la terre est, ce qu'on appelle de bonne prise, temps toujours trop court, sur-tout quand la charrue est d'une construction vicieuse. Dans les momens où le travail presse, il seroit facile, avec deux chevaux de relais seulement, de labourer un demi hectare par jour, des terres les plus compactes; et dans les terres légères, quatre chevaux labouroient, dans six heures de travail, 63,000 pieds de surface. Avec deux attelages, et en relevant les laboureurs, le même instrument laboureroit le double de cet espace dans une journée.
Immédiatement après moisson, ou pendant la moisson même, la promptitude des labourages importe souvent essentiellement à la réussite des récoltes dérobées; parce que, dans les grandes chaleurs, l'humidité de la terre s'évapore rapidement, et que le bénéfice des pluies, s'il en survient à cette époque, est d'une application plus profitable et plus étendue lorsqu'on a une charrue qui expédie le travail, et qui d'ailleurs a l'avantage de le faire aussi léger qu'on le veut. Pour semer des raves, du millet, de la sper-\setcounter{page}{400} gule, des vesces, des pois, du blé sarrasin, du maïs fourrage, etc. On peut ne labourer qu'à quatre ou cinq pouces de profondeur, et alors deux chevaux ou deux vaches suffisent à cette charrue dans les terres légères.
Si cet instrument se prête aux labours les plus légers, et même à l'opération de l'écroutement, il n'est pas moins précieux pour défoncer les terres à un pied de profondeur : pour cela on peut, ou avoir deux charrues belges qui se succèdent dans la même raie, ou faire le travail avec la même. Le tétard ou rateau en fer auquel on attelle les animaux, et qui se tourne à volonté d'un côté ou de l'autre de la flèche, permet de diriger la seconde charrue sur la trace même du cheval de droite qui marche dans la raie ouverte par la première, et l'on obtient ainsi à peu de frais un résultat semblable à celui du défoncement à la bêche, qui est une opération coûteuse et lente \footnote{J'ai commencé le défoncement d'un champ qui a donné du froment, et qui est destiné à porter au printems une récolte sarclée. La profondeur du labour en deux coups de charrue dans la même raie, est de dix à douze pouces, et l'ouvrage est parfait à tous égards. Je l'ai fait, jusqu'ici avec la même charrue, c'est-à-}.\setcounter{page}{401} J'ai fait une expérience pour mesurer la force du trait employée à la charrue belge, et à la charrue du pays, dans les mêmes circonstances. Je me suis servi pour cela, d'une romaine, composée de deux longues branches, réunies par une troisième très-courte, avec une articulation à charnière à un de ses bouts, et une soudure à l'autre. Vis-à-vis de l'angle soudé, est un crochet, auquel on attelle les animaux de trait. L'autre bout de la longue branche inférieure s'accroche à la charrue. L'effort du tirage, qui agit par un levier très-court, tend à soulever la longue branche supérieure, laquelle est bridée dans un cadre en fer, qui lui laisse l'ébat-tement nécessaire. A cheval sur cette branche, sont deux pesons mobiles à volonté, sur une division qui indique la résistance surmontée par la force, ou en équilibre avec elle. Cet équilibre existe lorsque, pendant la marche de la charrue, la branche mobile
dire, que les chevaux font un tour (aller et venir) à six pouces de profondeur; on remue ensuite le crochet de la chaîne d'attelage de deux ou trois trous à gauche, puis on fait un tour (aller et venir) dans les mêmes raies. Il retombe un peu de la terre tirée du fond; mais la plus grande partie reste à la surface où le versoir la jette.\setcounter{page}{402} se maintient au milieu du cadre dans lequel elle peut s'ébattre, ou que du moins elle frappe également en haut et en bas de ce même cadre.
Je fis l'épreuve dans une terre de force moyenne qui n'avoit pas été labourée depuis un an, et où il y avoit quelques racines de luzerne de cet âge. La profondeur fut fixée exactement à six pouces et demi, pour les deux charrues. La belge tranchoit douze pouces de large, et celle du pays sept pouces et demi.
L'effort nécessaire pour faire cheminer la première étoit représenté par cinq cent six livres de marc, et l'effort nécessaire pour la seconde répondoit à huit cent quarante-trois livres. On peut remarquer que la largeur relative des tranches donne, en outre, à la charrue belge, l'avantage dans le rapport de huit à cinq, sans compter que le fond de la raie est uniformément tranché.
Pour obtenir toute la précision que comporte ce genre d'expérience, le dynamomètre de Reynier seroit préférable; car, avec la balance dont il s'agit, il m'a paru difficile de déterminer, à vingt livres près, quelle est la force employée.
Nous avons vu que sur la largeur de la tranche,\setcounter{page}{403} tranche, la belge a un avantage de $\frac{1}{3}$. Pour connoître l'emploi de force nécessaire au tirage de la charrue du pays, tranchant une bande de douze pouces de large, il faut donc dire, si sept pouces et demi demandent une force de huit cent quarante-trois livres, quelle force exigera une largeur de douze pouces? Réponse: mille trois cent quarante-huit livres. Si l'on remarquoit que la charrue du pays ne peut pas prendre une bande d'un pied de large, je répondrois qu'en dernière analyse, c'est la même chose, parce qu'elle emploie plus de temps pour labourer le même espace, et cela dans la proportion que je viens d'indiquer.
Je dois faire remarquer encore que dans un grand nombre d'observations que j'ai faites, en comptant les traits de charrue sur le guerêt, et en mesurant l'espace qu'ils occupoient, j'ai toujours trouvé la largeur moyenne des traits au-dessus de treize pouces. Les douze à dix-huit lignes que le soc ne tranche pas, sont déchirées, sans qu'il en résulte d'inconvénient sensible, car je n'ai jamais vu qu'il restât le moindre prisme non labouré. C'est donc au moins $\frac{1}{3}$ de gagné en étendue; et on voit par cette expérience comparative du tirage (laquelle cependant,\setcounter{page}{404} je ne donne que comme un à-peu-près que je ne puis bien ne pas craindre d'exagérer en annonçant une économie de moitié dans l'emploi de la charrue belge.
J'ai eu aujourd'hui même (10 octobre) la preuve que quand les laboureurs travailleront à prix fait avec la charrue belge, ils expédieront beaucoup plus l'ouvrage qu'ils ne l'ont fait en travaillant pour moi, à tant par heure. J'ai prêté une charrue belge à un de mes laboureurs du village pour labourer une terre légère qui a donné du froment cette année. Le champ contenant 24530 pieds, a été labouré avec quatre bœufs, dans quatre heures et demie: c'est plus du double de ce que les mêmes bêtes font dans le même temps, avec la charrue du pays.
Pour faire l'histoire complète de la charrue belge, il faut parler de ses inconvéniens. Le premier de tous, et qui lui est commun avec toutes les charrues à un versoir fixe, c'est de ne pouvoir labourer à plat, et d'être très-incommode à manier dans les pentes. L'inconvénient de ne pouvoir labourer à plat est moins grand qu'il ne le paroît au premier coup-d'œil. Les lecteurs peuvent consulter notre quatorzième volume d'agriculture, pour voir comment les Belges ti\setcounter{page}{405} rent parti de cette charrue dans tous les cas pour faire précisément le labour le plus convenable.
Les terrains extrêmement pierreux ne conviennent point à cette charrue, vu la forme aplatie de son soc. Cependant quelques cailloux roulés, même dans la glaise, ne sont point un obstacle absolu à sa marche : elle n'en est pas sensiblement plus dérangée qu'une autre charrue.
En général, elle chemine mieux quand la terre est sèche, que quand elle est humide. Comme son oreille est un peu courte, relativement à son degré d'écartement, elle est sujette à se charger de terre, si celle-ci est au point d'humidité qui la rend adhérente aux instrumens. Alors la charrue chemine mal, et fait tirer beaucoup les animaux. Cela arrive plus aisément quand l'instrument est neuf, et avant que l'oreille soit polie par le frottement.
La distance de la pointe du soc à la flèche n'étant que de treize pouces et demi, la flèche se trouve à six pouces seulement de la surface du sol, lorsqu'elle travaille. S'il y a beaucoup d'herbe, si cette herbe sur-tout est mouillée, comme cela arrive toujours à la reprise du matin, elle s'attache, et s'accu-\setcounter{page}{406} mule en haut du coutre, et entre le coutre et la gorge de la charrue. Ce paquet d'herbe mouillée se charge de terre, et augmente au point de faire l'effet d'un coin entre la flèche et le soc, et de forcer la charrue à sortir de terre. On remédie à cela en faisant marcher un ouvrier à côté de la charrue, pour débarrasser avec un bâton, l'herbe qui commence à s'attacher, et avant que la terre s'y accumule; mais cette précaution n'est pas toujours efficace, et l'on est quelquefois obligé de s'arrêter, pour enlever l'herbe, et la terre, sur-tout s'il y a des racines de luzerne, de trèfle, de chardons, ou d'arrête-bœuf, qui s'insinuent derrière le coutre. Cette dernière circonstance m'a fait imaginer de remplacer le coutre ordinaire par un autre plus large, dont le tranchant antérieur fût le même, mais qui ne laissait point d'intervalle derrière lui, et suivît exactement la courbe de l'oreille et du soc, en s'implantant par une saillie dans celui-ci, entre la pointe et la lame tranchante qui se projette encore en avant. L'essai ayant réussi avec une des charrues, j'ai armé l'autre de même; et je me suis assuré avec la balance que l'effort du tirage n'en étoit point augmenté. Ce changement me semble avoir plusieurs avantages.\setcounter{page}{407} tages. 1º. On n'est pas obligé d'affaiblir la flèche par un trou pour le coutre. 2º. Le tout en est plus solide, parce que cette pièce fixe, qui tient à la flèche par des écrous, et aussi à la pointe du soc, sert à lier fortement entr'elles les parties de l'instrument qui supportent le plus grand effort. 3º. Les réparations seront moins fréquentes, et le déplacement du coutre par ignorance ou négligence du laboureur, ne pourra pas nuire à la marche de la charrue, comme il arrive quelquefois. 4º. Enfin, on évite l'accumulation de l'herbe et des racines entre le coutre et le soc.
C'est un inconvénient assurément, que d'employer une charrue qui soit nouvelle dans le pays où l'on cultive, et cela non-seulement parce que les domestiques et les ouvriers commencent toujours par se déclarer contre elle, mais encore parce que les charrons et maréchaux répugnent à entretenir ou raccommoder les instrumens qu'ils n'ont pas faits. Les avantages de la charrue belge valent bien la peine qu'on se donnera pour surmonter ces petites difficultés. La lame d'acier adaptée au soc par des clous rivés, est la partie qui supporte le plus grand frottement et qui s'use le plus vite. Cette lame,\setcounter{page}{408} lorsqu'elle est émoussée et arrondie sur le devant, et dans sa partie postérieure, doit être changée, et il n'y a aucun maréchal qui ne puisse en mettre une neuve. Il importe qu'elle soit en entier d'acier, et que cet acier soit de bonne qualité et bonne trempe : cela fait une grande différence pour la durée de cette lame. Je tiens de Mr. Crud, qui a introduit l'usage de cette charrue dans les environs de Bologne, que les maréchaux du pays rechargent d'acier cette bande, sans la détacher, et en mettant le soc au feu, pour la souder avec celui-ci, opération qui se renouvelle sans difficulté, toutes les fois que le soc en a besoin, comme on feroit pour une autre charrue. L'opération de forger le soc, et le versoir, demande beaucoup d'attention et d'adresse, à cause de la rigoureuse précision que ce travail exige, et de la singulière conformation du soc, composé de quatre pièces distinctes. C'est là ce qui renchérit cette charrue; mais il est probable que lorsqu'elle sera livrée à l'imitation et à la concurrence des ouvriers adroits, elle baissera sensiblement de prix; et comme elle fait l'économie d'un avant-train, elle ne se trouvera peut-être, finalement, pas plus chère que l'autre, en combinant les frais d'achat et d'entretien. Si même la charrue belge étoit\setcounter{page}{409} un peu plus coûteuse, ce seroit une bagatelle en comparaison du bénéfice qui résulte de son usage.
Pour ne pas compliquer, je n'ai point parlé de l'alonge-versoir, décrit dans notre XIV\textsuperscript{e} volume déjà cité. Les Belges savent tirer un parti admirable de cette pièce additionnelle, pour certains travaux. Quand nous serons familiarisés avec cette charrue, il sera facile de nous approprier les avantages que l'emploi de l'alonge-versoir offre dans certains cas particuliers.
Avec des animaux bien dressés, il est facile de se passer d'aide dans la conduite de la charrue belge; mais si le laboureur se sert des guides pour mener deux chevaux de front, on trouvera peut-être de l'avantage à reprendre le manche tel que les belges l'emploient, c'est-à-dire, avec une petite corne, ou mancheron derrière, pour maintenir les guides qui passent par dessus, et glissent librement. Le laboureur peut avoir un fouët de poste à longue chasse, pendu au poignet de la main droite, et ne gênant aucun de ses mouvemens. Lorsque, de la main droite il tire les guides, ou se sert de son fouët, il tient de la main gauche le manche de la charrue, et la dirige. Deux bœufs, ou deux vaches de forte taille, pour-\setcounter{page}{410} ront également traîner cette charrue, avec un conducteur sans aide. Les Belges y réussissent : pourquoi n’y réussirions-nous pas? Il ne faut rien inférer de défavorable au tirage de cette charrue, de ce que, pour mes labours de semaille, j’y ai attelé au moins trois chevaux. Dans mon assolement, j’ai toujours à rompre des luzernes ou des trèfles, qui exigent un attelage plus fort; et quant aux terres légères qui avoient donné cette année des pommes de terre, c’est pour ménager des chevaux qui ne sont pas de travail, qu’on en a toujours attelé trois ou quatre.
Je ferai une dernière observation sur la charrue belge, c’est que tous les laboureurs qui l’ont maniée s’y attachent, et se dégoûtent de celle du pays. Maintenant que mes labours de semaille sont terminés, les paysans mes voisins, frappés de la promptitude et de la perfection du travail, me demandent, de tous côtés, mes charrues à emprunter. Ceux qui connoissent la ténacité des préventions et des habitudes chez les ouvriers de terre, pourront juger par ce seul fait, que la charrue belge est, en effet, un excellent instrument.
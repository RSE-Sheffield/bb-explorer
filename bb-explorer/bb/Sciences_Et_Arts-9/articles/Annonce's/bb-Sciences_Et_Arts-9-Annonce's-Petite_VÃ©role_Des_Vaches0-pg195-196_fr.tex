\setcounter{page}{195}
\chapter{Annonce's}
\section{AN INQUIRY INTO THE CAUSES AND EFFECTS OF THE VARIOLE VACCINE, &c. By EDWARD JENNER, M. D. F. R. S., &c. London 1798. \large{Recherches fur les causes & les effets de la Petite-vérole des vaches, par Edouard JENNER, Dr. Méd. Membre de la Société Royale, &c. Londres 1798.}}
Cet ouvrage contient l'exposé d'une découverte fort finguliere, & qui peut jeter un grand jour fur la théorie des maladies fufceptibles d'être inoculées ; en même temps qu'elle peut devenir fort utile pour en préferver fans danger l'efpece humaine. Nous en donnerons dans un prochain numéro un Extrait détaillé. Nous nous empreffons en attendant de faire connoître en peu de mots à nos Lecteurs, en quoi la découverte confifte, & nous les invitons à fufpendre leur jugement fur les faits que nous annonçons, jufqu'à-ce qu'ils en aient lu le développement & la preuve.
On fait que les chevaux fatigués font fujets à une maladie qu'on appelle le javart ( the greafe ). C'est une tumeur inflammatoire qui leur vient au bas de la jambe, & qui fuppure comme un gros furoncle. Dans les pays de pâturages, où les laiteries occupent tout le monde, hommes, femmes & enfans, il arrive fouvent que les mêmes hommes qui foignent les chevaux malades traient auffi les vaches, & leur communiquent la maladie, qui fe manifeste par des puftules irrégulieres fur le pis de la vache. Ces puftules font d'un bleu pâle, un peu livide, & entourées d'un cercle érysipélateux. L'animal devient bientôt malade, & fon lait diminue beaucoup. Mais c'est l'affaire de quelques jours. Cependant la maladie fe communique pour l'ordinaire aux hommes & aux femmes qui traient les vaches dans cer\setcounter{page}{196} état. Elle ressemble alors parfaitement et par sa marche, et par l'apparence de la pustule qui vient dans l'endroit infecté, et par les symptômes fébriles qui l'accompagnent, à une Petite-vérole inoculée dans laquelle il n'y aurait aucune éruption; et c'est cette ressemblance qui lui a fait donner dans le pays le nom de Petite-vérole des vaches (the Cow-pox).
Or, le Dr. Edward Jenner de Berkeley, dans le Comté de Gloucester, a démontré par une multitude d'observations et d'expériences très-bien faites que tous ceux qui ont eu cette maladie, pourvu qu'ils l'aient prise, non du cheval malade, ce qui arrive quelquefois, mais d'une vache infectée, sont toujours à l'abri de la Petite-vérole ordinaire, et incapables de la prendre, soit par contagion, soit par inoculation.
Et comme la Petite-vérole des vaches n'est jamais accompagnée d'aucun danger; comme elle n'est suivie d'aucune éruption après la fièvre, et que d'un autre côté elle n'est jamais contagieuse que par inoculation, c'est-à-dire par le contact immédiat du pus sur la peau mise à nu, et dépouillée de l'épiderme, il est évident qu'il y aurait un grand avantage à employer ce préservatif contre la Petite-vérole ordinaire, plutôt que l'inoculation de la Petite-vérole même. L'on n'aurait plus à craindre ni les dangers d'une éruption trop abondante, ni l'inconvénient très-grave de répandre la contagion.
Ces considérations ont déterminé le Dr. Jenner et quelques autres Praticiens de l'Angleterre à inoculer de préférence la Petite-vérole des vaches pour préserver de la Petite-vérole ordinaire; et ces inoculations très-nombreuses ont eu le plus grand succès, comme on le verra dans l'ouvrage que nous annonçons, dans lequel on trouvera d'ailleurs des détails de théorie et de pratique fort intéressants et fort inattendus. (O)

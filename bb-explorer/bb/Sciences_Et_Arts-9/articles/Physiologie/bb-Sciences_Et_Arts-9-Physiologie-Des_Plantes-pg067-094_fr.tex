\setcounter{page}{67}
\chapter{PHYSIOLOGIE.}
\section{A TRANSLATION OF Dr. BELL’S THESIS De Physiologia plantarum. Dissertation sur la physiologie des plantes, publiée en latin à Edimbourg par le Dr. GEORGE BELL, traduite par le Dr. JAMES CURRIE, à la réquisition de la Société Littéraire & Philosophique de Manchester, & insérée dans le 2d. volume des Mémoires de cette Société.}
QUOIQUE l’étude des plantes ait été assez cultivée dans ce siecle, les Naturalistes se sont plus occupés de leurs formes extérieures, de leur nomenclature & de leur distribution par classes ; ordres ; genres & espèces que de leur structure interne ; & du jeu de leurs organes. Cette dernière étude est cependant d’une toute autre importance que la première ; soit pour l’agriculture, soit pour la médecine, soit pour la plupart des arts qui ont pour objet des substances végétales. Elle a les plus grands rapports avec celle de l’économie animale, à laquelle on a donné par excellence le nom de Physiologie, ou étude de la nature. Il n’est presque aucun problème important dans l’une de ces deux sciences qui n’ait dans l’autre un problème analogue, dont la solution jette\setcounter{page}{68} nécessairement du jour sur celle du premier.
Aussi de nos jours de grands Philosophes ont-ils fait de la physiologie des plantes l'objet particulier de leurs recherches. Indépendamment d'un grand nombre d'excellents mémoires sur cet important sujet qu'on trouve dans les Transactions philosophiques, dans les mémoires de l'Académie des Sciences, et dans d'autres Recueils scientifiques, on peut consulter les ouvrages de Du Hamel, du Dr. Hales, du Dr. Hill, sans oublier ceux du savant Hedwig, si connu par ses nombreuses recherches sur la physiologie des plantes, et ceux de nos célèbres compatriotes, Charles Bonnet, qui le premier, dans ses Considérations sur les êtres organisés, a fait sentir la liaison qui existe entre les principales fonctions des animaux et celles des végétaux, et Jean Senebier, auteur de la physiologie végétale, dans l'Encyclopédie méthodique, des mémoires physico-chimiques sur l'influence de la lumière solaire &c.
À côté de ces grands noms, on aime à voir la Société Littéraire et Philosophique de Manchester, placer celui d'un jeune homme de 22 ans, le Dr. George Bell qui en 1777 avait fait de la physiologie des plantes le sujet de sa dissertation inaugurale pour obtenir le grade de Dr. en médecine à Edimbourg, et qui y avait acquis une si grande réputation de science sur la physique végétale que le célèbre Lord Kames\setcounter{page}{69} demandant un jour au profeffeur Cullen quelques informations fur ce fujet, dont il avoit befoin pour un ouvrage important auquel il travailloit alors, & qu'il a publié depuis ( The gentleman Farmer ) le favant Profeffeur lui recommanda le jeune Bell, alors âgé de 20 ans, comme l'homme le plus capable de lui donner tous les renfeignemens dont il auroit befoin. C'eft à cette circonftance que notre jeune auteur a dû la haute eftime, l'amitié & la protection, dont le Lord Kames n'a ceffé de l'honorer. Après avoir pris fon grade, le Dr. Bell étoit venu s'établir d'abord à Berwick, & enfuite à Manchefter, où il pratiquoit avec diftinction depuis plufieurs années, lorfqu'une mort prématurée l'enleva à la fcience & aux lettres. La Société, dont il n'étoit pas un des membres les moins actifs, témoigna fes regrets, en inférant dans le 2d volume de fes mémoires fon Eloge Hiftorique, & une traduction de fa differtation fur la phyfiologie des plantes; à laquelle fon ami le Dr. James Currie, de Liverpool, auteur de l'un & de l'autre, a ajouté de favantes notes. C'eft de cette differtation dont nous allons préfenter un Extrait à nos lecteurs.
Quand on coupe tranfverfalement un arbre, on y diftingue trois parties, l'écorce, le bois & la moëlle.
1°. Il y a deux parties à confidérer dans l'écorce, l'épiderme, & l'écorce proprement dite.\setcounter{page}{70} L'épiderme est composé de plusieurs couches ; aisément séparables l'une de l'autre, & dont les fibres sont circulaires \footnote{Cela est vrai des arbres dont l'épiderme s'enlève par bandes circulaires, tels que les cerisiers, pruniers &c, mais cela n'est pas prouvé pour les autres plantes. (R)}. Il recouvre toutes les parties de la plante, & même les semences qui sont dans le centre des fruits, ce qui prouve que sa formation n'est point due, comme l'a cru le Dr. Hill, à l'endurcissement de la vraie écorce par l'action de l'air. L'écorce proprement dite est entièrement composée d'un tissu cellulaire ou parenchyme dont les cellules communiquent les unes avec les autres, & qui sont traversées par deux fortes d'organes, les vaisseaux propres de la plante, & les fibres longitudinales.
2°. Le bois situé au-dessous de l'écorce, est comme celle-ci, composé de cellules, mais beaucoup plus denses & plus rapprochées les unes des autres. Elles sont aussi traversées non-seulement par les vaisseaux propres de la plante, & par des fibres longitudinales, mais encore par de grands tubes ou canaux dont l'enveloppe est contournée en spirale & qui s'étendent d'un bout de l'arbre jusqu'à l'autre. On les appelle Trachées, ou vaisseaux aériens. Entre le bois & la moelle on trouve une substance verte, à laquelle le Dr. Hill, qui est le premier qui l'ait\setcounter{page}{71} bien décrite a donné le nom de couronne. Il prétend qu'elle contient toutes les parties de la plante en embryon.
3°. Au centre de l'arbre est la moëlle, d'autant plus abondante que l'arbre est plus jeune; car à mesure qu'il grandit, grossit & approche de sa maturité, elle se sèche & diminue, tellement que dans les vieux arbres, elle est entièrement oblitérée \footnote{Cela n'est pas vrai de tous les arbres. Le fureau, par exemple quoique vieux, conserve toujours une assez grande quantité de moëlle. (R)}. Sa structure est entièrement cellulaire, & le Dr. Hill assure qu'elle est uniformément la même dans toutes les plantes.
Telles sont les parties solides. Quant aux fluides, il y en a de deux fortes; l'un qui est exactement de la même nature dans toutes les plantes, qui diffère très-peu de l'eau commune \footnote{On a cru que la sève de certains arbres, comme l'érable contenait un principe sucré, que celles d'autres arbres contenait un acide &c. le Dr. Bell s'est assuré par plusieurs expériences qu'il n'y a rien de semblable. C'est pourquoi il suppose que lorsqu'on a cru trouver dans la sève de pareils principes, c'est qu'il s'y était mêlé accidentellement du suc propre de la plante.}, & que l'on obtient abondamment de plusieurs plantes, telles que la vigne ou le hêtre, par une incision faite au printemps, s'appelle la sève ou le suc commun; l'autre qui varie suivant les différentes plantes dans lesquelles il se trouve\setcounter{page}{72} & duquel dépendent leurs qualités sensibles & leurs propriétés s'appellent le suc propre. Ces deux sucs ne se mêlent jamais dans l'arbre, & le dernier ne se trouve que dans les vaisseaux propres.
On n'a point encore déterminé si la sève se transmet d'une partie de la plante à une autre par des vaisseaux ou par la substance cellulaire seulement. On a essayé de décider cette question par des injections. Elles ont pénétré à une certaine hauteur dans la plante, mais toujours avec de telles irrégularités qu'on n'a rien pu en conclure en faveur de l'une ou de l'autre opinion. Chacune a eu ses partisans. Notre auteur penche pour la première, 1°. à cause de l'analogie des vaisseaux propres & des vaisseaux aériens qui contiennent certainement les uns le suc propre des plantes, & les autres l'air qu'elles renferment. Pourquoi la sève n'aurait-elle pas comme eux, les vaisseaux déférents. 2°. Parce qu'on n'a point d'exemple de sécrétion, fonction dont les plantes jouissent certainement aussi bien que les animaux, sans l'intervention des vaisseaux particuliers; 3°. parce que le Dr. Hales ayant inventé un instrument au moyen duquel il mesurait exactement le diamètre d'une branche de vigne, a 1/100 de pouce près, trouva que sa grosseur était exactement la même, soit que sa branche fut remplie de sève, où qu'elle en fut totalement dépourvue.
\setcounter{page}{73} Les botanistes ont fait un grand nombre d'expériences pour déterminer le cours de la sève, & sa direction. Monte-t-elle des racines dans le tronc, & du tronc dans les branches, ou descend-elle des branches dans le tronc, & du tronc dans les racines ? On a généralement adopté la derniere opinion parce que quand on fait des incisions à un arbre au travers de l'écorce & du bois jusqu'à la moëlle dans la saison où la sève est abondante, c'est-à-dire au printemps, on en voit toujours couler une plus grande quantité des bords supérieurs de la playe que des bords inférieurs, ce qu'on explique par la grande quantité d'humidité que l'arbre absorbe par ses branches.
Notre auteur réfute cette opinion par deux expériences qui le portent à croire que la sève monte & ne descend pas. 1. Après avoir fait plusieurs incisions à différentes hauteurs sur la tige de diverses plantes, il a plongé leurs racines dans une décoction de bois de Brésil. La liqueur colorante est montée des racines dans le tronc à la hauteur de l'incision & gueres au-delà ; ce qui n'a pas empêché qu'on ne l'ait toujours vue couler du bord supérieur, & jamais du bord inférieur. 2. Après avoir fait plusieurs incisions profondes sur une branche de vigne, & avoir vu couler la sève en grande abondance par le bord supérieur de la playe, il a renversé cette branche, de manière que\setcounter{page}{74} le bord supérieur fut placé par ce renversement au-dessus du bord inférieur, & alors la seve a constamment coulé à l'incision du bord le plus voisin de la racine qui étoit devenu le bord supérieur. Ce n'est donc que par l'effet de sa pesanteur & non en conséquence de son éloignement des racines que la seve coule d'un bord de l'incision plutôt que de l'autre \footnote{Pourquoi donc le suc propre découle-t-il des deux bords de l'incision ? (R)}.
D'autres observations tendent encore à prouver qu'au printemps la seve monte & ne descend pas. 1. Le Dr. Hope a fait sur la fin de l'hiver & avant l'apparence de la seve plusieurs incisions sur un hêtre à différentes hauteurs. Quand la seve a commencé à se manifester, elle a constamment coulé par le bord supérieur de l'incision, & jamais par le bord inférieur ; mais toujours elle a commencé à couler d'abord à la premiere incision , celle qui étoit le plus près de terre, ensuite à la seconde, puis à la troisieme, & ainsi de suite en remontant constamment dans le tronc des racines jusqu'aux branches. 2. Si au commencement de la saison pendant laquelle la vigne pleure , c'est-à-dire donne une grande quantité de seve, mais avant que cette seve se soit manifestée dans le tronc ou dans les branches , on fait une incision profonde sur ses racines, il en découle toujours une grande quantité de seve ; 3 l'abondance\setcounter{page}{75} de la seve est toujours proportionnée à l'humidité du terrain. La seve monte donc des racines dans les branches, & elle ne descend pas de celles-ci dans le tronc.
Le suc propre des plantes paroît suivre une direction contraire. Car notre auteur ayant fait une incision profonde sur un pin, trouva que quoiqu'au premier moment le suc coulât en grande abondance des deux bords de la playe, au bout de quelques instans il ne coula plus que du bord supérieur, c'est-à-dire, du bord le plus éloigné des racines, & cette direction ne changea point, comme on l'avoit obfervé à l'égard de la seve, lorsque la branche sur laquelle on avoit fait l'incision fut placée horizontalement ou même renversée.
Dans l'une & l'autre de ces deux positions, le suc coula toujours comme dans la position verticale du bord de l'incision le plus éloigné des racines. Ce n'est donc pas le poids du suc, mais son éloignement des racines, qui détermine son écoulement par l'un des bords de l'incision plutôt que par l'autre.
Il n'est pas aussi aisé de déterminer la direction de l'air dans les vaisseaux aériens. On trouve ces vaisseaux dans le bois, dans les feuilles, dans les pétales ; mais on n'en trouve ni dans l'écorce, ni dans les plantes herbacées\footnote{Il seroit bien extraordinaire que les plantes herbacées qui ont tant de rapport avec les arbres fussent privées d'un organe qui paroît si général & si essentiel à la végétation. Aussi a-t-on trouvé dans ces plantes, comme dans les autres, un grand nombre de vaisseaux aériens ou de trachées. Ils forment la plus grande partie, ou la totalité des filets ligneux qui les parcourent. (R).}. Ils\setcounter{page}{76} font composés de plusieurs petites filaments contournés en spirale, de manière à former un canal dans le centre. On leur a donné le nom de vaisseaux aériens parce qu'ils ne contiennent jamais aucun liquide\footnote{On verra plus bas que c'est une erreur & que ces vaisseaux sont au contraire souvent remplis de liquides. (R)} ; & que comme le bois dans lequel on les trouve en grande abondance, renferme beaucoup d'air, sans aucun autre organe propre à le contenir, on présume que c'est dans ces vaisseaux qu'il est contenu. On les regarde donc comme analogues aux poumons dans les animaux, comme les principaux organes de la respiration dans les plantes.
Mais comment se fait cette respiration? Comment l'air pénètre-t-il dans les plantes? Quelques auteurs ont imaginé qu'il y entre par les racines sous une forme non élastique, & qu'en les traversant, il recouvre graduellement son élasticité. Mais 1°. on trouve un grand nombre de vaisseaux aériens dans les racines mêmes, où les sucs n'ont point encore circulé, & où par conséquent il ne peut pas encore s'être développé d'air; 2°. la situation des racines, enterrées comme elles le sont, & souvent à une\setcounter{page}{77} grande profondeur, ne permet gueres de supposer qu'elles absorbent un fluide aussi peu à leur portée que l'air.
D'autres ont cru que l'air pénètre dans les plantes par les feuilles. Mais fi delà, il descend dans le tronc & dans les racines, ne gênera-t-il pas le mouvement de la seve ascendante ? Cette objection nous paroît foible, d'autant plus qu'elle milite également contre toutes les théories, & en particulier contre celle qu'adopte notre auteur, d'après le Dr. Hill.
Cet ingénieux botaniste, confidérant que l'épiderme des plantes est évidemment vasculaire, que dans les arbres & les arbrisseaux, qui font les seules plantes dans lesquelles on découvre des vaisseaux aériens, les vaisseaux de l'épiderme ont un orifice externe, tandis qu'ils n'en ont point dans les plantes herbacées, & qu'enfin si l'on place une branche d'arbre fous le récipient de la machine pneumatique, on n'en voit sortir l'air que par le bois, suppose que les vaisseaux de l'épiderme font les véritables organes de la respiration dans les plantes\footnote{Il est difficile d'imaginer qu'une écorce entièrement morte comme celle qui recouvre les grands arbres contienne les organes de la respiration des plantes. Puisque les plantes herbacées ont aussi leurs trachées, elles devroient avoir aussi des orifices externes à l'épiderme. Le fait est qu'on ne les voit pas mieux dans les arbres que dans les autres plantes.},\setcounter{page}{78} ceux par lesquels, lorsque la douce température du printemps commence à dilater leurs orifices contractés par le froid de l'hiver, l'air pénètre dans les vaisseaux aériens, & par ceux-ci jusqu'aux racines, qu'il ranime & vivifie, de la même manière qu'il produit cet effet sur les muscles des animaux; peut-être même assiste-t-il par son élasticité le mouvement d'ascension de la seve; ces mouvemens intérieurs des fluides dans la plante développent une nouvelle quantité d'air qu'il circule avec eux, & s'échappe enfin par les feuilles.
A l'appui de cette théorie, le Dr. Bell rapporte une expérience qui prouve en même temps, le grand besoin d'air qu'ont les plantes ainsi que les animaux. Il vernit plusieurs jeunes arbres au milieu de l'hiver & les enveloppa d'une toile cirée; en ne laissant que le sommet des branches exposé à l'air. Ils demeurèrent tous dans cette situation jusqu'à l'été. Le résultat de cette expérience fut que tous les arbres qui avoient été exactement garantis de l'air périrent. Les autres languirent & n'eurent que peu de feuilles. C'est sans doute par cette raison que les arbres qui sont furchargés de mousse à leur surface ne poussent que peu de feuilles, de foibles rejetons, & point de fruits. Aussi les jardiniers ont-ils grand soin de dépouiller l'écorce de leurs arbres de cette mousse qui les étouffe & leur ôte le bénéfice de l'air extérieur\footnote{Il n'est pas décidé chez les agriculteurs si la mousse qui vient sur les arbres nuit réelleme nt á leur végétation ou bien si elle ne s'attache qu'a ceuz quil song déja vieux & languissans. Au moins est-il certain qu'on ne l'apprecoit que rarement sur les autres.}.
\setcounter{page}{79} On a supposé que l'air après avoir pénétré dans les plantes y demeure susceptible de dilatation & de contraction suivant les changements de température de l'atmosphere, & que c'est principalement à ces oscillations de l'air interne qu'est dû le mouvement ascendant & descendant des fluides. Notre auteur réfute cette opinion, en faisant voir, 1. que les vaisseaux aériens des racines sont trop profondément situés pour pouvoir participer aux variations de chaleur dans l'atmosphere; 2. que la seve monte & que le suc propre descend, quelle que soit la température de l'air extérieur; 3. que si les vaisseaux qui contiennent les sucs des plantes sont pourvus de valvules, ces sucs ne font pas susceptibles d'un mouvement rétrograde, & que s'ils n'en ont point, la compression qu'exerceront sur eux les vaisseaux aériens gênera encore plus leur mouvement qu'elle ne le favorisera.
Quelles sont donc les forces qui font mouvoir ces sucs? seroit-ce, comme on l'a cru généralement, l'attraction capillaire rendue permanente par l'évaporation qui se fait par les feuilles? mais l'attraction capillaire ne peut avoir lieu que lorsque la résistance à l'une des extrémités du tube est diminuée. Elle expliquerait\setcounter{page}{80} tout au plus le mouvement d'ascension de la sève; lorsque la chaleur de l'atmosphère est plus grande que celle de la terre. Elle ne faudrait expliquer le mouvement descendant du suc propre, ni celui de la sève dans la même direction, si comme cela est assez probable, il a quelquefois lieu. Comment concevoir d'ailleurs que les vaisseaux des plantes soyent d'un diamètre assez petit pour pouvoir élever leurs sucs par la seule force d'attraction capillaire depuis les racines d'un grand arbre jusqu'au sommet de ses branches? Comment expliquer dans cette supposition l'écoulement abondant de la sève par les incisions faites sur les tiges ou sur les branches de la vigne au printemps, tandis que celles qu'on fait en été ne produisent point cet effet? Les vaisseaux ne font-ils pas les mêmes? Enfin on a beau rompre des tuyaux capillaires remplis d'une liqueur; elle n'en sort point; au lieu que comme l'a démontré le Dr. Hales, il suffit de couper transversalement une tige de vigne pour obtenir un écoulement abondant de fluides.
Il faut donc avoir recours à une autre force motrice, & celle qu'adopte notre auteur est le principe vital, dont il suppose, avec tous les botanistes modernes, que les plantes sont douées, ainsi que les animaux. Il est un grand nombre de plantes qui se montrent sensibles à l'action des stimulants & dans lesquelles quand on les\setcounter{page}{81} touche on remarque des mouvemens extraordinaires. L'action seule de la lumiere suffit pour donner à leurs tiges une direction différente, ou pour redresser leurs feuilles, que l'obscurité couche au contraire les unes sur les autres, quelle que soit la température de l'air. C'est cette influence de l'obscurité qu'on a appellée avec assez de justesse le sommeil des plantes, & que tous les observateurs ont remarquée. Puis donc que les plantes sont évidemment douées d'un principe d'irritabilité, rien n'empêche qu'on n'explique par ce principe leurs mouvemens, & ceux de leurs fucs, de la même maniere que tout le monde s'accorde à lui attribuer ceux des animaux. L'auteur parle à cette occasion de diverses expériences qu'il avoit faites pour déterminer l'influence de différens gas, de la lumiere & de quelques solutions salines sur l'accroissement & les qualités des végétaux\footnote{Dans une dissertation inaugurale sur l'irritabilité des plantes, publiée à Edimbourg, en juin 1797 par notre compatriote le Dr. Jean Pefchier, on trouve plusieurs expériences faites dans le but de déterminer l'effet des acides, des alkalis, des corps odorans, tels que le musc, le camphre & l'æther, de l'oxigene, de la lumiere, de la chaleur & du froid, sur les contractions de la sensitive, de l'épinevinette, & d'autres plantes irritables. En général, les vapeurs acides & alkalines ont agi sur ces plantes comme stimulans, & les ont évidemment mises en contraction. Les corps odorans en particulier ont produit de la contraction. Les corps odorans ont eu peu d'effet sur elles. Le camphre a diminué leur irritabilité. L'auteur n'a pas observé que la présence de l'oxigène ou de la lumière eût aucune influence marquée sur leurs mouvements. La chaleur d'un fer rouge, le froid de la neige n'en ont guère eu davantage. (R)}. Il en supprime le détail, parce qu'il\setcounter{page}{82} se proposoit de les répéter sur un plan plus vaste. Mais leur résultat général fut que l'accroissement des plantes peut être augmenté par l'action de certains stimulans, dont l'effet ne permet pas de douter de l'existence du principe vital qui les anime.
Les plantes transpirent comme les animaux. Mais d'après les expériences du Dr. Hales (Statique des végétaux. vol. I. p. 49) & celles de Guettard (mémoires de l'Académie des Sciences 1748) l'humidité qu'elles exhalent ne paroît différer que très-peu de l'eau pure, si ce n'est qu'elle est plus susceptible de putréfaction. Sa quantité est d'autant plus grande que la surface de la plante est plus augmentée par un grand nombre de feuilles, que l'air extérieur est plus chaud & plus sec, que la plante est exposée à une plus vive lumière, & qu'elle est dans un plus grand état de vigueur & de santé.
D'un autre côté les plantes absorbent aussi de l'humidité; & ce sont encore les feuilles qui sont le principal organe de cette absorption, comme elles sont celui de la transpiration. Mais il paroît que c'est par leur surface supérieure\setcounter{page}{83} qu'elles exhalent l'humidité, & par leur surface inférieure qu'elles la pompent. Si de deux feuilles semblables, on enduit de vernis la surface supérieure de l'une & la surface inférieure de l'autre, la premiere se trouve au bout d'un temps donné n'avoir point diminué de poids, & la seconde beaucoup. De même si on laisse flotter sur l'eau des feuilles parfaitement semblables, les unes avec leur surface supérieure, les autres avec leur surface inférieure en contact avec l'eau, les premieres n'augmentent point de poids, & meurent en peu de jours; les autres deviennent bientôt plus pesantes, & conservent leur fraîcheur pendant plusieurs mois. Les organes de la transpiration ne sont donc pas les mêmes que ceux de l'absorption \footnote{Cette remarque ne peut gueres s'appliquer qu'aux feuilles des arbres dont la surface supérieure est réellement différente de l'inférieure. Dans les plantes herbacées les deux surfaces font presque toujours semblables, & par conséquent leurs fonctions ne peuvent pas être fort différentes. Pourquoi dans toutes les recherches de ce genre n'a-t-on presque considéré que les grands végétaux ? Les autres auroient peut-être été plus utiles pour la recherche de ces vérités qui sont encore si profondément cachées.}. Il est probable que ces organes ont une structure particuliere, & que leur action tient au principe de vie. Car les feuilles séchées n'absorbent point; & si l'on coupe un bout de racine d'une\setcounter{page}{84} oignon de jacinthe pouffant dans l'eau, l'absorption ne se fait plus, toute la racine périt & tombe, & de nouvelles racines ayant leur extrémité adaptée à la fonction d'absorption à laquelle elles sont destinées, sortent de l'oignon, pour remplacer celle qui est devenue inutile par la section de son extrémité.
Il est vraisemblable aussi qu'une grande partie de la matiere exhalée se réabsorbe dès qu'elle est parvenue à la surface. Elle servoit de véhicule aux matieres nuisibles dont l'expulsion étoit nécessaire à la vie de la plante; & si ces matieres n'avoient pas été délayées dans une grande quantité d'humidité, elles auroient pu par leur concentration nuire aux organes qu'elles traversent; mais quand ces matieres nuisibles sont arrivées à la surface & rejettées hors du corps de la plante, rien n'empêche que l'humidité dont elle a un besoin continuel ne soit de nouveau repompée, de la même maniere qu'on voit dans les animaux des urines trop acres pour que les reins eussent pu sans danger être exposés à nud à leur action concentrée, être délayées & enveloppées dans un grand volume de sérolité, qui lorsque ces urines sont, sous sa sauve-garde, arrivées à la vessie sans endommager les organes tendres & délicats qu'elles ont du traverser avant d'y être déposées, est de nouveau en grande partie repompée & rentre dans la masse des fluides.
\setcounter{page}{85} Puis donc que les plantes exhalent & absorbent en même temps de l'humidité, on ne peut jamais juger exactement de la quantité qu'elles en perdent par l'exhalation, ou gagnent par l'absorption. On ne peut connaître que l'excès de l'une sur l'autre. Qu'un tournesol par exemple perde dans l'espace d'un jour 20 onces de son poids, on ne pourra pas en conclure que la quantité d'humidité exhalée dans les 24 heures par cette plante est de 20 onces, mais qu'elle surpasse de 20 onces celle qui a été repompée. Qu'au lieu de perdre, il gagne en poids 10 onces; on n'en conclura autre chose si ce n'est que l'absorption a surpassé de 10 onces l'exhalation. Qu'il conserve exactement le même poids; on en conclura que l'exhalation est exactement compensée par l'absorption.
Les plantes ont le pouvoir de former leurs différentes parties; & ce pouvoir elles l'exercent par voie de sécrétion. Nous pouvons jusqu'à un certain point en deviner les agents, mais nous ignorons leur manière d'opérer. Il y a lieu de croire cependant que le principe vital n'est pas assez actif dans les plantes pour s'acquitter seul de cette fonction. Il a besoin d'être fécondé par la fermentation \footnote{La fermentation peut concourir à l’élaboration des sucs de la plante; mais on ne comprend pas comment elle pourrait servir à la formation des parties solides. Le germe préexiste dans le noyau avec son organisation propre; & cette organisation seule peut suffire pour y fixer celles des molécules nutritives qui font nécessaires à son développement. (R)}, par l'absorption, &\setcounter{page}{86} par la distribution particulière des vaisseaux. Lorsqu'il est question par exemple de former une partie dure, comme les noyaux des fruits, les vaisseaux font diverses circonvolutions, avant d'arriver à l'organe qui en dépose les matériaux, afin de ralentir le cours des sucs, & de leur donner le temps de fermenter & de s'épaissir. Le suc propre paroît ne se former qu'après que la sève est montée jusqu'aux feuilles, & pendant qu'elle descend des feuilles aux racines. Il en est de même du bois : car lorsqu'on fait une ligature autour du tronc, le bois se gonfle & s'épaissit beaucoup, mais seulement au-dessus de la ligature, & non pas au-dessous.
Quant aux éléments qui servent de nourriture aux plantes, l'auteur présume que ce sont principalement l'eau & la lumière\footnote{Le calorique doit aussi contribuer à leur développement. C'est un fait connu, il est vrai, que des plantes qui s'étiolent à l'obscurité, se fortifient & se colorent à l'air. Mais il est également vrai que le même individu renfermé dans une terre chaude s'y colore beaucoup plus fortement que lorsqu'il est exposé en plein air, & que les plantes des pays chauds, après s'être décolorées à un air froid, ont reverdi dans toutes leurs parties, lorsqu'elles ont été placées dans l'intérieur des couches. Or dans ces couches il y a plus de chaleur, mais il n'y a pas plus de lumière. (R)}. Il avoit\setcounter{page}{87} fait quelques expériences qui l'acheminoient à cette conclusion; mais il ne les a pas publiées. La seve & les autres sucs végétaux circulent-ils dans les plantes, comme le sang circule dans les animaux? Cette question a partagé les botanistes. L'expérience de la ligature dont nous venons de parler a fait croire aux uns que la seve monte le long du bois, & descend le long de l'écorce. D'un autre côté on a essayé d'écorcher un arbre, d'enlever une section entière du bois & de remettre l'écorce en place. Cette opération n'a point empêché l'arbre de croître. On en a conclu que la seve monte le long de l'écorce. Tout cela est encore fort obscur. La seve montant certainement des racines dans le tronc, & du tronc dans les branches, & le suc propre suivant une direction contraire, on ne peut se refuser à croire qu'il y a dans les plantes une forte de circulation. Mais les vaisseaux par lesquels elle se fait font-ils, comme dans les animaux, placés les uns à côté des autres, ou font-ils les uns dans le bois & les autres dans l'écorce, c'est ce qu'on n'a point encore éclairci. \footnote{Il n'est pas probable que la partie ligneuse, soit d'une grande utilité dans l'acte de la végétation, puisque plusieurs arbres, tels que les saules, dépouillés de cette partie ligneuse, n'en vivent pas moins.}
\setcounter{page}{88} Quoiqu'il en soit, on a tout lieu de croire que le principe de vie a la plus grande part à toutes les fonctions des plantes, à leur formation, à leur accroissement, à leurs sécrétions &c. Ce principe semble être le même que celui de l'organisation. Partout où l'on voit un arrangement particulier d'organes, on doit soupçonner de la vitalité, & réciproquement partout où il y a vie, il y a une organisation particulière. Il est vrai que la nature n'a pas prodigué ce principe avec la même libéralité à tous les êtres organisés. En thèse générale, les animaux en sont plus richement pourvus que les plantes; mais il y a des nuances infinies entr'eux, des gradations d'intelligence & de sensibilité, telles que les limites entre le règne animal & le règne végétal font fort éloignées d'être bien distinctes. Les coraux & plusieurs espèces de polypes adhérent aux rochers comme les plantes à la terre, & comme elles, ils meurent si on les détache du lieu où ils ont pris leur accroissement. D'un autre côté, il y a des plantes dont les mouvements ressemblent beaucoup à ceux des animaux. Les feuilles du Burrhum Chundalli\footnote{Cette plante originaire du Bengale y a été découverte par Mylady Monfon, & a été rangée par Linné fils dans le genre des sainfoins sous le nom d'Hedysarum gyrans. Elle a les feuilles ternées, & la foliole intermédiaire beaucoup plus grande que les deux autres. Cette foliole n'a aucun mouvement pendant le jour. Dans la nuit, elle se recourbe & vient s'appliquer sur les branches. Mais les folioles latérales sont toujours en mouvement. Elles se portent alternativement vers le haut & vers le bas; & ce mouvement continue depuis que la plante naît jusqu'au moment où elle cesse de fleurir. Elle est cultivée dans nos serres, où elle fleurit assez communément. Voyez sa description dans le Journal de Physique pour le mois de mai 1787. (R)} ont un mouvement continuel\setcounter{page}{89} & en apparence spontanée qui annonce un principe de vie fort énergique. Il en est de même de la sensitive & de la Dionœa muscipula qui montrent l'une & l'autre une admirable activité dans leurs feuilles à la plus légère impression, au point d'enfermer & d'étouffer par leurs contractions les mouches & les autres insectes qui viennent se poser sur elles. Certainement il y a des animaux qui semblent en vitalité fort au-dessous de ces plantes & de plusieurs autres; & "je ne fais" ajoute l'auteur, "si nous n'avons pas tort d'attribuer exclusivement au regne animal la faculté d'éprouver tour-à-tour le plaisir & la douleur. Il me semble que les fleurs dont brillent nos parterres auroient plus de beauté, les arbres qui décorent nos forêts plus de dignité, & l'étude de la botanique plus d'attraits, si nous consentions à envisager les plantes comme des êtres sensibles."
\setcounter{page}{90} On voit assez d'après la dissertation de Mr. Bell combien nos connoissances sur la physiologie des plantes sont faibles & incertaines. Nous savons que l'intérieur des végétaux est composé de plusieurs parties distinctes, telles que la moëlle, le bois, &c. qu'il contient deux espèces de vaisseaux faciles à distinguer, les vaisseaux propres, & les trachées, que les feuilles sont des organes propres à la respiration, & à l'inspiration, que l'air ainsi que la lumière contribuent à leur accroissement: mais nous ne savons encore rien de précis, sur la manière dont circulent les fluides, dont s'opèrent les sécrétions, &c. nous ignorons également l'usage des diverses parties qui composent le végétal, du bois, par exemple, de la moëlle, du parenchyme, &c. —Et si de ces objets qui ne sont que la base de la physiologie nous passons à d'autres qui en dépendent & qui sont plus relevées, si nous nous demandons par exemple en vertu duquel mécanisme les plantes exécutent ces mouvements nombreux qui leur sont propres, comment elles s'avancent vers la lumière, par quelle force leurs feuilles se retournent & leurs fleurs s'épanouissent ou se ferment à des heures déterminées, quelle est la raison pour laquelle leurs radicules se dirigent toujours vers la terre & leurs tiges vers le ciel, & en général quelles sont les causes de tant de phénomènes curieux que l'on observe.\setcounter{page}{91} dans presque tous les genres des végétaux, nous avouerons que nous n'avons pas même des conjectures à présenter pour expliquer le plus grand nombre de ces faits, & que l'étude de la physiologie est encore au berceau. Depuis quelques années on s'est accoutumé à considérer les trachées comme des vaisseaux qui au lieu d'être destinés à l'air seul, charioient encore d'autres fluides & particulièrement la seve. Hedwig ce grand observateur assure les avoir vu remplis de ce dernier fluide, & il les considère comme revêtus à l'intérieur d'une membrane fine & transparente qui sert à le contenir. Il va même plus loin, & il croit que la subsistance du bois, est toute entière formée par des trachées endurcies & qui ont perdu leur élasticité.
Par rapport au mouvement de la seve, voici de toutes les explications celle qui nous paroît la plus conforme à la marche de la nature, & qui est le plus à l'abri des objections, on la doit à notre illustre compatriote Defauffure & elle est extraite de la physiologie végétale de l'Encyclopédie Méthodique, où elle a été pour la première fois publiée.
"Supposez dit-il, un tube fléxible dans une situation verticale & rempli de seve ou de tout autre liquide depuis le bas jufqu'à la moitié de sa hauteur; fuppofez ce tube ouvert par ses deux extrêmités & plongé dans\setcounter{page}{92} l'eau par en bas, le fluide qu'il contient va s'écouler et se mêler à l'eau; mais s'il se forme un étranglement dans la partie inférieure, cet étranglement empêchera l'écoulement et le fluide demeurera renfermé dans le tube: qu'immédiatement au-dessus de cet étranglement, le tube se resserre encore, le fluide fera chassé vers le haut, et si de proche en proche le tube continue à se resserrer, le fluide fera progressivement chassé vers le haut du tube: Qu'après que le fluide a été ainsi chassé à une certaine hauteur l'orifice inférieur du tube se rouvre, tandis que la partie moyenne demeure resserrée, l'eau dans laquelle plonge le tube, sollicitée ou par la pression de l'air intérieur, ou par la succion capillaire, entrera dans le vide qui vient de se former et l'on conçoit distinctement comment la répétition des mêmes alternatives de contraction et de dilatation peut chasser continuellement dans le même sens un fluide renfermé dans un tube élastique."
"Si la contraction avoit commencé par le haut du tube, le fluide auroit été chassé vers la terre: si dans un fluide horizontal, la contraction commence à droite, le fluide marchera à gauche et réciproquement, &c."
"On voit en même temps comment par un simple changement dans l'ordre des contractions les mêmes vaisseaux peuvent faire mar\setcounter{page}{93} cher la fève dans une direction opposée à la première. Il est indispensable d'expliquer ce fait, puisque dans l'arbre qu'on plante à l'envers, la fève prend une direction contraire à celle qu'elle avoit d'abord : & il est impossible de l'expliquer dans l'hypothèse des valvules ou des soupapes."
"Je fais bien que ce mouvement alternatif n'a point été constaté par des observations immédiates ; mais si on l'avoit observé ce ne feroit pas une hypothèse, ce feroit un fait. — Il est évident que soit sa lenteur, soit la ténuité des vaisseaux dans lesquels on le suppose, peuvent le dérober à nos yeux, c'est ainsi qu'on ne peut point observer la pulsation des artères dans leurs dernières ramifications, quoique personne n'osât affirmer que ce mouvement cesse dès le terme où l'on cesse de l'apercevoir."
"La cause de ce mouvement seroit sans doute l'irritabilité. Il faut supposer que les vaisseaux dans lesquels se meuvent les sucs des plantes sont irrités par le passage de ces sucs, & que cette irritation excite dans ces vaisseaux une contraction progressive, & quel genre de mouvement paroit plus adapté aux plantes que celui qui suffît seul à expliquer toutes leurs fonctions vitales & qui possède la simplicité & l'uniformité qui constituent le caractère de cette classe d'êtres organisés."
\setcounter{page}{94}Il ne reste donc plus qu'à découvrir cette irritabilité, sur laquelle notre illustre Professeur a déjà fait des expériences assez satisfaisantes & alors le grand problème de la circulation de la sève sera enfin résolu.

\setcounter{page}{258}
\chapter{Médecine}
\section{AN INQUIRY INTO THE CAUSES AND EFFECTS OF THE VARIOLE VACCINE, &c. Recherches sur l'origine & les effets d'une maladie connue dans quelques Comtés de l'Angleterre, & particuliérement dans le Gloucestershire, sous le nom de Petite-vérole des vaches, par Edouard JENNER, Dr. Méd.; Membre de la Soc. Roy., &c. in-4°. 75 p. Londres 1798. \large{Premier Extrait.}}
UNE maladie cruelle & contagieuse, apportée de l'Orient par les Arabes, la Petite-vérole, ravage l'Europe depuis plusieurs siecles. Elle fait périr au moins la vingtieme partie de ses habitans \footnote{Depuis l'an 1661 jusqu'en 1772 il est mort à Londres 2,538,450 personnes, dont 193,432 de la Petite-vérole, c'est-à-dire, au moins une sur quatorze. Dans le même espace de temps, il en est mort à Genève 76,050 dont 3972 de la Petite-vérole, c'est-à-dire, au moins une sur vingt. Et en calculant d'après la totalité de nos Extraits mortuaires depuis 1581 jufqu'en 1772, je trouve 116,935 morts, dont 7292 de la Petite-vérole, c'est-à-dire, au moins un sur seize (Voyez le Journal de Médecine de septembre 1773 & Janvier 1776.) A Edimbourg, Mr. Alex. Monro rapporte que depuis 1744 jufqu'en 1763, il est mort 24,322 perfonnes, dont 2441 de la Petite-vérole, c'est-à-dire, plus d'une sur dix. (Lettre à la Faculté de Paris.) A la Haye, depuis 1755 jufqu'en 1769, il est mort 18,671 perfonnes, dont 1455 de la Petite-vérole, c'est-à-dire, plus d'une sur treize. (Bibl. des Sciences 1769.) Il seroit aisé d'accumuler les citations de ce genre. On trouveroit partout la mortalité de la Petite-vérole supérieure à celle qu'on annonce ici. (O)}. En Asie, où elle est connue\setcounter{page}{259} depuis un temps immémorial, elle ne feroit pas moins meurtriere, fi l'on n'avoit trouvé & adopté depuis long-temps dans ce pays un moyen de diminuer infiniment fes dangers par l'inoculation. L'Europe a connoiffance de ce moyen depuis près d'un fiecle; mais elle n'a sù en profiter qu'imparfaitement. En vain 80 ans d'expérience lui ont prouvé que lorfque les hommes ont une fois été atteints de la Petite-vérole, tant artificielle que spontanée, ils deviennent incapables de la prendre une seconde fois, & que d'un autre côté, la Petite-vérole inoculée eft toujours infiniment plus bénigne que la Petite-vérole naturelle; au point qu'il ne meurt \setcounter{page}{260} pas un inoculé sur 300\footnote{On a souvent porté beaucoup plus loin cette différence: C'est ainsi qu'en 1775 Sutton, me disoit avoir inoculé entre 36 & 37,000 personnes, & n'avoir perdu que quatre de ces inoculés, dont trois n'avoient pas voulu suivre ses instructions. On nous a encore assuré que dans l'Hôpital des Enfans trouvés à Londres, ( Foundling-Hospital ) on a inoculé jusqu'à présent plus de 4000 enfans, sans en perdre un seul. Mais je me défie beaucoup de l'exactitude de ces rapports. Il faut bien supposer un mois pour l'inoculation. Or, dans tous les pays, à Londres surtout, il meurt au moins une personne sur trente par année, c'est-à-dire, une sur trois cent soixante par mois. Dire donc, que dans le mois destiné à l'inoculation, il n'est pas mort un inoculé sur 4000, c'est dire, que l'inoculation a décuplé leur probabilité de vie, ce qui n'est gueres vraisemblable. Cherchera-t-on à justifier l'inoculation, des morts accidentelles qui peuvent survenir pendant la durée de ce mois en les attribuant à des causes étrangères. C'est bien ce qu'on a souvent fait; & c'est ce qui explique peut-être les exagérations des inoculateurs. Mais c'est se jeter dans un océan sans bornes, c'est s'ôter tout moyen d'estimer avec vérité la chance favorable à l'inoculation. Car rien n'empêche qu'on n'use du même artifice en faveur de la Petite-vérole naturelle; & alors on ne saura plus à quoi s'en tenir.(O)}, tandis qu'il en meurt au moins un sur dix de la Petite-vérole naturelle. L'Europe ne méconnoît point cet avantage de l'inoculation; mais elle semble le dédaigner; & l'on peut dire que même en Angleterre,\setcounter{page}{261} où plus que partout ailleurs on a fait les plus grands efforts pour la rendre générale, il n'y a, comparativement à la grande masse des habitans, qu'un petit nombre d'individus qui ayent été soustraits par ce moyen à l'influence meurtriere de la petite-vérole naturelle ; que la plupart des peres & des meres négligent d'y avoir recours pour leurs enfans ; & que partout les gouvernemens, s'ils n'ont pas indirectement blâmé ou même absolument proscrit l'inoculation, ne l'ont au moins que foiblement encouragée.
Qu'est-ce qui a pu produire cette funeste insouciance? C'est peut-être 1°, que quoique la petite-vérole inoculée, soit pour l'ordinaire une maladie peu dangereuse, il est cependant vrai qu'on a vu quelquefois des inoculés avoir la petite-vérole confluente & en mourir, mort qui étant pour ainsi dire provoquée, a naturellement dû laisser beaucoup plus de regrets & répandre beaucoup plus d'effroi que celles qui ont été la suite d'une maladie spontanée. On sent que jamais une mere ne peut se consoler d'un malheur de ce genre, & la seule possibilité de perdre ses enfans d'une maniere aussi déchirante a dû fréquemment suffire pour imposer silence à la raison. 2°. Que quoique les foyers de contagion soyent infiniment moins multipliés par la petite-vérole inoculée que par la petite-vérole naturelle, puisque le nombre\setcounter{page}{262} des boutons y est beaucoup moins considérable ; on ne peut nier cependant que la petite-vérole inoculée ne soit jusqu'à un certain point contagieuse. Les gouvernemens ont pu craindre que les encouragemens qu'ils donneroient à cette pratique ne fussent un moyen de répandre la contagion & d'augmenter par-là les ravages de la petite-vérole naturelle, crainte d'autant plus plausible qu'elle a été suggérée plus d'une fois par des médecins très-célèbres, & en apparence accréditée par l'augmentation de mortalité de cette maladie à Londres, depuis que l'inoculation y est devenue en usage \footnote{Cette objection a surtout été fortement pressée par Mrs. Dehaën & Raft ; mais j'ai démontré son peu de fondement, en faisant voir que l'augmentation de mortalité de la petite-vérole à Londres depuis le commencement de ce siècle n'a point été proportionnée aux progrès de l'inoculation; qu'elle avoit commencé avant qu'on inoculât; qu'elle tenoit à une augmentation réelle de malignité dans la maladie, & non à un plus grand nombre de malades ; qu'on n'avoit rien observé de semblable à Genève, où l'on n'a cessé d'inoculer toutes les années, épidémiques ou non, depuis 1751, un très-grand nombre d'enfans, & qu'enfin la même augmentation avoit eu lieu à Londres, & dans la même proportion relativement à d'autres maladies épidémiques & contagieuses, telles que la rougeole, qui ne sont pas susceptibles d'inoculation. (Voyez mes Lettres à Mr. Dehaën sur la mortalité de la petite-vérole, Journal de Médecine de *1773, 1776 & 1777.). (O)}.
\setcounter{page}{263} Si donc on trouvoit un préservatif, une manière d'inoculation qui mit infailliblement les hommes à l'abri de la petite-vérole, sans leur faire courir les dangers d'une éruption trop abondante, & sans avoir pour le public l'inconvénient de répandre la contagion, on pourroit espérer que ce moyen, plus généralement adopté, détruiroit enfin les germes de cette terrible maladie, pour la faire disparoître de dessus la face de la terre.
L'ouvrage que nous annonçons indique un préservatif de ce genre, une manière d'inoculation qui garantit sûrement de la petite-vérole, sans jamais produire aucune éruption générale, & sans que la maladie qui en résulte soit susceptible de se répandre de proche en proche par la contagion, à moins que le virus ne soit fortuitement ou volontairement introduit entre l'épiderme & la peau.
Indépendamment de ce double avantage, cette découverte est singulière, & digne de l'attention des Philosophes ; parce que si elle se vérifie, elle peut conduire à quelques recherches utiles sur l'origine de la petite-vérole & des autres maladies contagieuses : parce qu'elle présente des phénomènes nouveaux sur la marche & la propagation de ces maladies ; parce qu'enfin elle seroit, au moins dans les pays de laiteries, d'une grande utilité pour prévenir une maladie des bestiaux qui sans être bien dangereuse,\setcounter{page}{264} ne laisse pas d'être fort à charge aux propriétaires de grandes fermes. C'est-ce qui nous fait espérer que nos lecteurs nous sauront gré de mettre sous leurs yeux la presque totalité de cet ouvrage qui n'est pas bien volumineux ; & dont nous prendrons d'ailleurs la liberté d'élaguer les détails superflus.
Voici d'abord comment notre auteur expose le fait dont il s'agit.
"Les chevaux sont fréquemment atteints d'une maladie que les maréchaux ferrans appelens le javart\footnote{Je ne suis point parfaitement assuré que la maladie des chevaux dont il est ici question, (the greafe ou fore-heels) soit bien ce qu'on appelle en Français le javart. Le Cit. Sylvestre, médecin vétérinaire de cette ville, qui y exerce son art depuis plus de 17 ans, m'assure que le javart est une maladie fort rare; qu'il ne l'a vue que sept à huit fois, & qu'il n'a jamais eu connaissance qu'elle se fût communiquée aux hommes ou aux bêtes. Mais je n'ai trouvé dans les auteurs vétérinaires, dans le Dictionnaire de Lafosse, par exemple, aucune autre maladie des chevaux qui répondît à ce qu'en dit notre auteur; & les circonstances dans lesquelles se trouvent à cet égard les grandes laiteries de l'Angleterre sont si rares partout ailleurs, qu'il n'est pas surprenant que la petite-vérole des vaches ne soit connue que là. (O)} ; (the greafe) c'est une tumeur inflammatoire qui leur survient au bas de la jambe, 'tumeur' dont sort un pus qui a des propriétés très-particulières, & qui semble\setcounter{page}{265} capable de produire dans le corps humain, au moyen de quelques modifications dont je parlerai bientôt, une maladie tellement semblable à la petite-vérole, qu'il me paroît probable que celle-ci peut avoir tiré de-là son origine."
"Dans ce pays abondant en pâturages (l'auteur demeure à Berkeley dans le Comté de Gloucester) on nourrit un grand nombre de vaches; & l'on occupe à les traire des hommes & des femmes indistinctement. Or, si après avoir pansé les ulcères d'un cheval atteint du javart, un homme se met à traire les vaches, sans avoir auparavant la précaution de se laver les mains, il arrive souvent que le pus de ces ulcères s'attachant à ses doigts, il communique aux vaches, une maladie que celles-ci transmettent à leur tour aux autres personnes qui les traient; & de cette manière la maladie se propage tellement que tout le troupeau en est atteint, ainsi que tous les domestiques attachés à la laiterie.
"Cette maladie porte dans le pays le nom de petite-vérole des vaches; (cow-pox.) Elle se manifeste sur le pis des vaches sous la forme de pustules irrégulières, qui dès leur première apparence sont d'un bleu pâle, ou plutôt un peu livides & entourées d'une inflammation érysiplateuse. Ces pustules dégénérent fréquemment en ulcères phagédéniques ou rongeans, qui à moins qu'on n'y remédie à temps, donnent\setcounter{page}{266} beaucoup de peine pour leur guérison\footnote{Les remedes usités avec succès dans le pays pour arrêter les progrès des ulcères, sont des remedes chimiques de nature à agir directement sur le virus; tels que les solutions de sulfates de zinc, de cuivre, &c. (A)} L'animal devient malade, & son lait diminue beaucoup. Des taches enflammées commencent alors à paroître sur les mains & les poignets des domestiques employés à traire les bêtes infectées. Ces taches viennent promptement à suppuration, ressemblant d'abord aux ampoules produites par une brûlure. Ordinairement elles se manifestent sur les jointures & à l'extrémité des doigts, sous une forme circulaire. Leurs bords sont plus relevés que le centre, & elles ont une couleur un peu bleuâtre. Le pus s'absorbe ensuite, & il se manifeste des tumeurs sous les aisselles. Alors tout le système s'en ressent. Le malade éprouve des frissons, une lassitude générale, des douleurs vagues dans les lombes & dans les membres, des maux de cœur & des vomissements. Il se plaint constamment de mal de tête, & il survient même quelquefois du délire. Ces symptômes continuent avec plus ou moins de violence depuis un jour jusqu'à trois ou quatre, & sont accompagnés d'ulcères sur différentes parties des mains, ulcères qui font très-douloureux, trèsincommodes,\setcounter{page}{267} lents à se cicatriser, souvent phagédéniques comme ceux dont ils ont tiré leur origine. Les lèvres, les narines, les paupières en font aussi quelquefois affectées ; mais cela n'arrive que lorsque le malade a l'imprudence d'y porter le pus de ses doigts en se grattant. Je n'ai jamais vu d'éruption en d'autres parties du corps suivre la fièvre, si ce n'est en un seul cas, où il sortit sur le bras quelques boutons très-petits, d'un rouge vif, & qui se flétrirent promptement sans suppurer ; ce qui me fait douter qu'ils eussent aucune connexion avec les symptômes antécédens."
"Telle est la marche de la maladie, qui se communique d'abord du cheval au pis de la vache, & de la vache aux domestiques occupés à la traire. Il est possible que toute espèce de pus absorbé par la peau produise des effets jusqu'à un certain point semblables ; mais une propriété particulière à cette maladie & qui mérite surtout notre attention, c'est que toutes les personnes qui en ont été atteintes, deviennent dès lors & pour toujours à l'abri de la petite-vérole, & incapables de la prendre soit par contagion, soit par inoculation. C'est surtout ce fait extraordinaire que je vais prouver par un grand nombre d'observations."
L'auteur en rapporte ici en détail 23, que nous allons transcrire sans rien omettre d'essentiel, mais en abrégeant celles qui ne font qu'une\setcounter{page}{268} répétition des mêmes faits, & qui ne méritent d'être rapportées séparément qu'en considération de quelque circonstance particulière.
l est bon de remarquer auparavant avec l'auteur, que l'on voit fréquemment des ulcères & des pustules se manifester spontanément sur le pis des vaches, dans toutes les saisons de l'année; mais surtout au printemps lorsqu'on les met à l'herbe. Elles y sont aussi plus sujettes lorsqu'on les laisse nourrir leur veau. Quelquefois, mais rarement, les domestiques employés à les traire dans cet état, en contractent des ulcères aux mains, & éprouvent même quelques symptômes d'indisposition en conséquence de l'absorption du pus. Mais ces sortes de pustules sont toujours d'une nature beaucoup plus bénigne que celles que produit la contagion qui constitue la véritable petite-vérole des vaches. Elles n'ont jamais cette apparence bleuâtre & livide qui est si manifeste dans celle-ci. Elles ne sont accompagnées d'aucun érysipèle, & n'ont aucune disposition à former un ulcère phagédénique ; mais elles se terminent promptement par une croûte, sans produire sur la vache aucune affection générale. En un mot, c'est une maladie très-différente de celle dont il est question dans cet ouvrage, & qui n'a point les mêmes effets sur le corps humain. C'est à quoi il faut bien faire attention pour ne pas croire trop légèrement à l'abri de la\setcounter{page}{269} petite-vérole les malades qui en font atte???.
Premiere Observation, Joseph Merret, a???lement sous-jardinier du Comte de Berkeley était domestique chez un fermier du voisinage en 1770 ; & aidait occasionnellement son maître à traire les vaches. Plusieurs chevaux de la ferme furent atteints du javart, & Merret les pansait. Les vaches furent dans peu affectées ; & bientôt après, il survint aux mains du domestique plusieurs ulcères qui furent suivis de tumeurs, & de roideur dans les aisselles. Il en fut tellement indisposé que pendant plusieurs jours il ne put vaquer à ses occupations ordinaires.
Avant que la maladie se manifestât parmi les vaches, aucun des domestiques n'en était attaqué, & l'on n'avait amené dans la ferme aucune vache nouvelle.
En 1795, c'est-à-dire 25 ans après, on fit à Berkeley une inoculation générale. Merret fut inoculé avec sa famille. On lui fit à plusieurs reprises & avec beaucoup de soin plusieurs incisions au bras, sans pouvoir lui donner la petite-vérole. Il n'y eut qu'une légère efflorescence d'un rouge pâle autour des incisions, avec une apparence érysipélateuse dans le centre. Cette efflorescence, dit l'auteur, qu'on voit souvent dans les personnes qu'on inocule après qu'elles ont eu la petite-vérole, & qui dans ces cas-là est toujours beaucoup plus prompte que celle\setcounter{page}{270} qui furvient en conséquence d'une inoculation qui réussit\footnote{Cette inflammation locale qui furvient autour des incisions faites avec du pus variolique aux personnes qui ont eu la petite-vérole, paroît avoir été mieux observée par notre auteur que par aucun autre. Quant à moi, je n'ai jamais inoculé personne qui eût bien certainement eu la petite-vérole, à ma connoissance; mais j'ai fréquemment inoculé sans succès & à plusieurs reprises des personnes qui ne favoient pas si elles l'avoient eue & qui vouloient se mettre à l'abri de toute crainte. Or il m'est arrivé aussi souvent de ne voir survenir dans ces cas là aucune inflammation, que de voir celle que décrit ici Mr. Jenner. Quand je l'ai vue, je l'ai confidérée comme un demi effet du virus & comme une raison de réinoculer, ce qui m'a quelquefois réussi pour produire enfin une petite-vérole complète. D'un autre côté, j'ai vu dans quelques cas une inflammation très-prompte, très-étendue, & même une fois accompagnée de beaucoup de fièvre & de convulsions, survenir le jour même de l'inoculation, se dissiper ensuite, & faire place cependant à la véritable inflammation locale de la petite-vérole, qui commence pour l'ordinaire au 5e. jour, & produit l'affection générale au 7e. ou 8e. J'ai vu dans lesquels cette première inflammation sans être bien étendue, avoit produit un bouton suppurant qui se terminoit par une croûte, après la chûte de laquelle commençoit la véritable inflammation. Enfin ce qui est encore plus singulier, j'ai vu deux cas dans lesquels au 8e. jour d'une inoculation qui n'avoit produit aucune apparence quelconque d'inflammation locale, & où la trace des incisions étoit à peine visible, il survient cependant une petite-vérole très-bénigne & très-bien caractérisée. Mais toutes ces anomalies n'empêchent pas que l'observation de Mr. Jenner ne puisse être généralement bien fondée. (O)}. Merret demeura avec les autres\setcounter{page}{271} inoculés pendant tout le temps que dura la petite-vérole. Un d'eux en eut abondamment.
Il est à remarquer que dans ce cas, ainsi que dans tous ceux qui suivent, on s'étoit parfaitement assuré que les individus dont il est question n'avoient point eu la petite-vérole ordinaire, avant de subir les tentatives qu'on fit pour la leur donner. Cette certitude feroit peut-être difficile à avoir dans une grande ville, ou dans un canton très-peuplé. Mais à Berkeley où la population est très-petite, & où l'on tient note avec beaucoup de soin de ceux qui ont eu la petite-vérole, il n'y avoit aucun risque d'inexactitude à cet égard.
2de. Observation: Il y avoit 27 ans que Sarah Portlock avoit eu la petite-vérole des vaches, lorsque son enfant prit la petite-vérole ordinaire. Elle le soigna pendant sa maladie, & se fit inoculer aux deux bras. Mais l'inoculation ne prit point. On n'en observa aucun autre effet que ceux qui avoient eu lieu dans le cas précédent.
3me. Observation. John Philips avoit eu la petite-vérole des vaches à l'âge de 9 ans. On lui inocula la petite-vérole ordinaire à l'âge de 62 ans. Le pus fut pris des bras d'un enfant\setcounter{page}{272} inoculé, immédiatement avant la fièvre éruptive, dans le moment de sa plus grande activité \footnote{C'est ce que j'ai constamment vu. Jamais aucune inoculation ne m'a mieux réussi & d'une manière plus régulière que lorsque j'ai pu inoculer avec un pus encore limpide & non délayé. Un pus épais manque souvent, même quand il est délayé. Cependant on vient d'imprimer dans un Journal nouveau, dont on doit naturellement attendre beaucoup d'instruction, que Mr. le Professeur Hufeland, d'Iena recommande pour l'inoculation, de prendre un pus bien cuit & bien formé, parce qu'une matière qui n'est pas mûre, prise" avant la bonne suppuration ne produit souvent que" la fausse petite-vérole, tandis qu'une matière bien" cuite en produit toujours une belle." (Bibliothèque Germanique, N°. 1. pag. 8 & 11.) De la part d'un praticien aussi célèbre que le Professeur Hufeland, cette assertion m'étonne, parce qu'elle est diamétralement contraire à ce que j'ai cru voir; mais en lisant le reste du Mémoire, il me semble qu'en cette occasion il a plutôt écrit d'après ce qu'il croyoit, que d'après ce qu'il avoit vu. (O)}. Il ne fut point délayé. On l'inféra sur le champ. Le malade ressentit bientôt aux incisions une douleur semblable à celle d'un aiguillon. Une légère efflorescence parut ensuite tout autour. Elle s'étendit au 4me. jour, & fut suivie d'un peu de douleur & d'enroidissement aux aisselles. Mais au 5me. jour, ces symptômes diminuèrent, & deux jours après, ils se dissipèrent entièrement, sans produire aucune indisposition générale.\setcounter{page}{273}
4me. Observation. Mary Barge fut inoculée en 1791. L'inoculation produisit autour des incisions, une efflorescence d'un rouge-pâle qui se dissipa dans quelques jours sans aucune indisposition. La maladie servit ensuite de garde à plusieurs enfans atteints de la petite-vérole; Mais elle ne la prit point. Elle avoit eu la petite-vérole des vaches, 31 ans auparavant.
5me. Observation. L'inoculation faite avec un pus très-actif, & avec beaucoup de soin, n'eut pas plus d'effet en 1778 sur une vieille. Dame qui avoit pris la petite-vérole des vaches dans son enfance, pour avoir manié les ustensiles dont se servoient des domestiques qui en étoient atteints. Elle avoit peu de temps après cet événement, vu & soigné une de ses parentes qui mourut de la petite-vérole, sans en avoir elle-même aucune apparence.
6me. Observation. La propriété qu'a la petite-vérole des vaches de préserver de la petite-vérole ordinaire est jusqu'à un certain point réciproque. Ceux qui ont eu celle-ci, ne sont pas susceptibles de prendre celle-là ou ne la prennent que très-légèrement. Ce fait est si bien connu parmi nos fermiers, qu'aussitôt que la maladie se manifeste dans le troupeau, on cherche à se procurer des domestiques qui ayent eu la petite-vérole: sans quoi tout le train ordinaire de la ferme pourroit être arrêté.
Au mois de mai 1796, la petite-vérole des\setcounter{page}{274} vaches se manifesta chez Mr. Baker, fermier demeurant dans le voisinage de Berkeley. La maladie étoit venue d'une vache infectée qu'on avoit achetée à la foire. Aucune des vaches de la ferme ayant du lait, & elles étoient au nombre de 30, n'échappa à la contagion. La famille consistoit en deux domestiques mâles & deux servantes qui, ainsi que le fermier lui-même, étoient occupés deux fois par jour à traire les vaches. Ils avoient tous eu la petite-vérole à l'exception de Sarah Wynne, une des servantes. Aussi le fermier & le plus jeune des valets échappèrent entièrement à la contagion. L'autre valet & l'autre servante n'eurent qu'un ulcère sur un de leurs doigts qui ne produisit aucune affection générale.
Sarah Wynne n'en fut pas quitte à si bon marché. Elle prit la maladie d'une manière si violente qu'elle fut obligée de se mettre au lit, & fut pendant plusieurs jours incapable de suivre à ses occupations ordinaires dans la ferme. Au mois de mars 1797, j'inoculai cette fille avec beaucoup de soin. Il parut, comme à l'ordinaire, un peu d'inflammation autour des incisions; mais dès le cinquieme jour cette inflammation disparut entièrement sans produire aucune affection générale.
7me. Observation. La petite-vérole ordinaire ne préserve cependant pas toujours aussi complètement ni aussi sûrement de la petite-vérole des vaches.\setcounter{page}{275} Chez Mr. Andrews, autre fermier du voisinage, une vache achetée à la foire dans l'été de 1796, infecta de même toutes les vaches de la ferme. La famille consistoit en six personnes, le fermier, sa femme, ses deux fils, un valet & une fervante. Tous s'aiderent à traire les vaches, excepté le fermier qui craignit pour lui-même les conséquences de la maladie. Tous, à l'exception du valet, & Will. Rodway, avoient eu la petite-vérole. Cependant aucun d'eux n'échappa entièrement à la contagion. Ils eurent tous des ulcères aux mains, & quelques symptômes d'indisposition générale précédés de douleurs & de tumeurs aux aisselles. Mais la maladie qu'éprouverent ceux qui avoient eu la petite-vérole, fut incomparablement plus bénigne que celle du valet qui ne l'avoit pas eue. Car il fut obligé de se mettre au lit pendant plusieurs jours, tandis que les autres purent facilement vaquer à leurs occupations ordinaires. - Au mois de février 1797 j'inoculai William Rodway. Toutes les incisions s'enflammerent au troisième jour; mais cette inflammation se dissipa bientôt; & à l'exception d'une légère rougeur érysipélateuse qui dura jusqu'au 8me. jour, & qui alors produisit pendant une demi-heure seulement, une sensation désagréable à l'aisselle droite, il n'y eut aucun symptôme d'affection générale.
8me. Observation. Quelque bénigne qu'ait\setcounter{page}{276} été la petite-vérole des vaches, dans le corps humain, à quelque distance qu'elle se soit manifestée, du moment où l'on est exposé à la contagion de la petite-vérole ordinaire, elle conserve toujours sa propriété d'en préserver complètement. — Elisabeth Wynne avoit eu la petite-vérole des vaches à l'âge de 19 ans; mais si légère que la maladie s'étoit bornée à un petit ulcère sur le petit doigt de la main gauche, & à peine y avoit-il eu quelques symptômes d'indisposition générale. Trente-huit ans après, je lui inoculai la petite-vérole ordinaire. Une légère efflorescence se manifesta bientôt autour des incisions, & la malade y éprouva une sensation douloureuse jusqu'au 3me. jour, que ces symptômes commencèrent à diminuer. Au cinquième jour, ils se dissipèrent entièrement, sans aucune apparence d'indisposition générale.
9me. Observation. Puisque la petite-vérole ordinaire ne se manifeste jamais qu'une fois dans la vie, & que la petite-vérole des vaches en garantit sûrement, il semble que celle-ci ne devroit non plus se manifester qu'une fois dans le même individu, & que quand on l'a eue, on devroit en être à l'abri. Cependant il est démontré qu'on peut l'avoir plusieurs fois. Il est vrai que pour l'ordinaire, elle est plus bénigne à la seconde qu'à la première, même dans les vaches; mais quelquefois elle est tout aussi forte. En voici un exemple, — William Smith de\setcounter{page}{277} Pytton étant en 1780 chez un fermier du voisinage, fut appelée à panser les ulcères d'un des chevaux de la ferme qui avoit pris le javart. Il porta l'infection aux vaches, & en fut ensuite lui-même atteint. Il eut plusieurs ulcères aux mains & les symptômes ordinaires d'affection générale décrits ci-dessus. En 1791 il se trouvoit chez un autre fermier, parmi les vaches duquel la maladie se manifesta, & il la prit pour la seconde fois aussi fortement que la première. Enfin il l'eut une troisième fois en 1794, sans aucune diminution de gravité dans les symptômes. En 1795, il fut inoculé deux fois, mais sans succès, & dès-lors il a impunément bravé la contagion de la petite-vérole.
10me. Observation. Le développement des maladies contagieuses ne se fait pas toujours dans un espace de temps uniforme. L'affection générale qui résulte de la petite-vérole des vaches commence pour l'ordinaire comme celle de la petite-vérole inoculée, sept à huit jours après le moment de l'infection. Mais il n'est pas rare de voir la petite-vérole inoculée se manifester que long-temps après l'infection du pus variolique \footnote{L'intervalle le plus long que j'aye vu s'écouler entre le moment de l'infection de la petite-vérole, & le commencement de l'inflammation, a été de dix-sept jours. C'étoit un enfant de deux ans. Croyant que l'inoculation manquoit, je le réinoculai; mais en découvrant son bras, j'apperçus un commencement de rougeur & d'élévation à l'endroit de l'incision. Je l'inoculai cependant pour la seconde fois. Mais les premieres incisions seules s'enflammerent. Les secondes manquèrent absolument, & la petite-vérole fit son cours d'une manière très-régulière & très-heureuse.}. Voici un exemple très-long.\setcounter{page}{278} remarquable mais peu précis du long intervalle qui s'écoule aussi quelquefois entre le moment de l'infection, & celui où les symptômes de la petite-vérole des vaches commencent à paroître.
Simon Nichols étant en 1782 domestique chez Mr. Bromedge à Berkeley, eut à panser les ulcères d'un cheval atteint du javart, & en même-temps à traire les vaches. Celles ci devinrent malades, mais ce ne fut que quelques semaines après le pansement des chevaux par Nichols que les pustules commencerent à se manifester sur le pis des vaches. Nichols quitta le service de Mr. Bromedge & entra chez un autre fermier. Il n'avoit point encore d'ulcères, mais bientôt ils se manifesterent, & il eut tous les symptômes de la petite-vérole des vaches dans un degré assez violent : il cacha la nature de sa maladie, continua à traire les vaches de son nouveau maître & à la leur communiqua. Quelques années après, la petite-vérole s'étant manifestée dans la maison dans laquelle il se trouvoit, je l'inoculai avec plusieurs autres personnes qui la prirent , &\setcounter{page}{279} avec lesquelles il demeura pendant toute sa durée. Mais ni l'inoculation, ni la communication avec les autres malades n'eurent aucun effet sur lui. Il eut une inflammation passagère aux incisions, & rien de plus.
11me. Observation. William Stinchcomb servoit avec Nichols chez Mr. Bromedge, lorsque les vaches furent infectées. La contagion l'atteignit aussi d'une manière grave. Il eut plusieurs ulcères rongeants à la main gauche, & une tumeur considérable à l'aisselle du même côté. Le côté droit fut aussi affecté, mais beaucoup moins. En 1792, Stinchcomb fut inoculé avec plusieurs autres personnes, & demeura avec elles pendant tout le temps de la petite-vérole, qui se manifesta chez elles avec plus d'abondance qu'à l'ordinaire. Quant à lui, il n'eut qu'une inflammation passagère aux incisions; mais il fut frappé de la ressemblance qu'il observa entre les symptômes qu'éprouvèrent ses compagnons, & ceux qu'il avoit éprouvés lui-même, lorsqu'il avoit eu la petite-vérole des vaches, & il en fit plusieurs fois la remarque.
12me. Observation. Tous les pauvres du village de Tortworth furent inoculés en 1795 par M. Henry Jenner, chirurgien de Berkeley. Huit d'entr'eux avoient eu la petite-vérole des vaches dans différentes périodes de leur vie. J'en avois vu moi-même Hefter Walkley affectée\setcounter{page}{280} grièvement atteinte en 1782. Mais ni elle, ni les sept autres personnes qui avoient eu cette maladie, parmi lesquelles il se trouvoit plusieurs femmes enceintes, ne prirent la petite-vérole, quoiqu'elles demeurassent constamment avec les autres inoculés.
13me. Observation. La petite-vérole des vaches tirant son origine du javart, il arrive quelquefois que les hommes qui pansent les chevaux atteints de celui-ci, contractent directement les mêmes symptômes que produit celle-là, quoique cela ne soit pas bien ordinaire. Quand cela a lieu, la maladie qui en résulte garantit aussi jusqu'à un certain point de la petite-vérole ordinaire\footnote{C'est peut-être par cette raison que tous les inoculateurs ont remarqué en Angleterre que quand on inocule des ferriers, (qui dans la campagne font presque tous l'office de maréchaux ferrants) l'inoculation manque souvent, ou ne leur communique qu'une petite-vérole anomale & imparfaite. (A)}. Mais pas aussi sûrement que quand elle se communique par les vaches. Sur trois malades auxquels la maladie avoit été communiquée directement par des chevaux, l'un devint par-là, incapable de prendre la petite-vérole ordinaire; un autre la prit, mais incomplète; le troisième enfin l'eut complète, mais très-bénigne. Voici ces trois observations.
Thomas Pearce avoit eu des ulcères aux\setcounter{page}{281} doigts, suivis d'une indisposition générale bien marquée & assez grave, après avoir pansé des chevaux atteints du javart. Six ans après, je lui inoculai à plusieurs reprises la petite-vérole, sans pouvoir produire en lui aucun autre effet qu'une inflammation locale, peu considérable, très-prompte & passagère autour des incisions.
14me. Observation. Mr. James Cole avoit été affecté de la même manière pour avoir pansé des chevaux atteints du javart. Je l'inoculai quelques années après avec du pus variolique. Il eut un peu de douleur à l'aisselle & une légère indisposition générale qui ne dura que trois ou quatre heures. Il lui survint ensuite quelques boutons au front, mais qui disparurent bien-tôt, sans suppurer.
15me. Observation. La jument de Mr. Abraham Riddiford ayant pris le javart, il la pansa lui-même, & eut, bientôt après, des ulcères aux mains, des tumeurs à chaque aisselle, & une indisposition générale assez grave. Un chirurgien du pays l'assura que cette maladie le mettoit pour toujours à l'abri de la petite-vérole. Il se trompoit. Car 20 ans après, le malade s'étant exposé à la contagion, il la prit, mais très-régulière & très-bénigne. Il est vrai que les boutons n'eurent pas l'apparence ordinaire, sans qu'on pût exprimer en quoi ils en différoient. D'autres praticiens qui virent le malade apperçurent comme moi cette différence,\setcounter{page}{282} mais quelques doutes qu'elle pût donner sur la réalité de la maladie, ils furent bientôt levés par l'inoculation de quelques personnes avec le pus contenu dans ces boutons. Elles eurent une petite-vérole régulière & dont les boutons furent exactement ce qu'ils devoient être.
16me. Observation. Sarah Nelines, laitiere chez un fermier du voisinage, prit la petite-vérole des vaches de la maniere ordinaire; mais elle reçut l'infection sur une écorchure qu'elle s'étoit faite à la main avec une épine. Il en résulta un gros bouton ulcére & accompagné de tous les fymptômes qui conftituent la maladie; mais ce bouton me parut si bien caractérisé que je le fis graver dans son état d'ulcération complète, pour donner une idée nette de la maladie. On voit sur la gravure deux autres boutons qu'elle eut en même temps au poignet, & un autre sur l'index, qui a été dessiné d'après un autre malade pour montrer leur apparence lorsqu'ils commencent à se manifester.\footnote{Cette gravure, ainsi que trois autres qui se trouvent dans le corps de l'ouvrage, portent le nom d'Edouard Pearce, dessinateur, & Wil. Skelton, graveur. Elles sont coloriées & d'une grande beauté. Il est impossible de rien voir de plus parfait tant pour le coloris que pour les formes. C'est exactement comme si l'on voyoit les malades eux-mêmes. Pourquoi n'a-t-on pas des gravures semblables pour toutes les maladies de la peau? On en auroit une idée bien plus parfaite que celle que peut donner la meilleure description en paroles. (R)}\setcounter{page}{283} 17me. Observation. La petite-vérole des vaches peut comme la petite-vérole ordinaire, s'inoculer artificiellement, & alors elle est communément plus bénigne que quand elle vient naturellement par le contact des vaches; mais elle a la même propriété & garantit aussi surement de la petite-vérole.
Un jeune garçon de huit ans & d'une bonne santé, fut inoculé le 14 mai 1796 avec le pus d'un des ulcères survenus aux mains de Sarah Nelmes. Les deux incisions pénétroient à peine jusqu'à la peau, & avoient environ un demi pouce de long. Au 7me. jour, il se plaignit d'une sensation de mal-aise à l'aisselle. Au 9me. il eut des frissons, du dégoût & un léger mal de tête. Il fut malade tout ce jour-là & eut beaucoup d'inquiétude la nuit suivante. Mais le lendemain il étoit parfaitement bien. L'apparence des incisions dans leur progrès vers la suppuration fut à peu-près la même que s'il avoit été inoculé avec du pus variolique ordinaire. Seulement la sérosité limpide contenue dans l'ampoule étoit d'une couleur un peu plus foncée, & l'efflorescence des incisions étoit d'un rouge un peu plus vif que dans l'inoculation de la petite-vérole ordinaire ( c'est le sujet de la 2de. gravure). Mais les incisions séchèrent & se terminèrent par une croûte exactement semblable à celle de la petite-vérole.
\setcounter{page}{284} Curieux de voir si une maladie aussi légère auroit garanti cet enfant de la petite-vérole, je l'inoculai le 1. juillet suivant avec du pus variolique frais & non délayé. Je fis sur ses deux bras plusieurs piqûres & incisions que je frottai bien avec ce pus; mais il n'en résulta que l'inflammation passagere qu'on voit ordinairement quand on inocule ceux qui ont eu la petite-vérole des vaches ou la petite-vérole ordinaire; & une seconde inoculation faite plusieurs mois après avec du pus variolique bien choisi n'eut pas plus d'effet.
Ici finissent les premieres observations de l'auteur. L'humidité du printemps de cette année ayant produit le javart dans plusieurs chevaux du voisinage, & la maladie s'étant communiquée aux vaches de plusieurs laiteries, il eut occasion de faire de nouvelles recherches qui feront le sujet d'un autre Extrait.
( La suite au Numéro prochain. )
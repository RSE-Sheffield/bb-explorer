\setcounter{page}{367}
\chapter{Médecine}
\section{AN INQUIRY INTO THE CAUSES AND EFFECTS OF THE VARIOLE VACCINE, &c. Recherches sur l'origine & les effets d'une maladie connue dans quelques Comtés de l'Angleterre, & particulièrement dans le Gloucestershire, sous le nom de Petite-vérole des vaches, par Edouard JENNER, Dr. Méd.; Membre de la Société Royale, &c. \large{Second Extrait.}}
Les premieres observations du Dr. JENNER avoient prouvé que la petite-vérole des vaches quoique susceptible elle-même de revenir plusieurs fois, est un préservatif sûr & infaillible\setcounter{page}{368} contre la petite-vérole ordinaire, pourvu qu'elle ait été communiquée d'un cheval atteint du javart à une vache, & de celle-ci aux personnes employées à la traire; que si elle se communique directement du cheval malade aux domestiques qui le pansent, ce qui arrive quelquefois, quoique rarement, elle ne les garantit point aussi sûrement de l'influence variolique; & qu'enfin cette maladie, susceptible comme la petite-vérole ordinaire d'être inoculée, produit à-peu-près les mêmes apparences aux incisions, & suit la même marche fébrile, à l'éruption près, dont elle est toujours exempte.
Ces intéressantes recherches furent suspendues en 1796, par la cessation de la maladie. Mais l'humidité du printemps dernier ayant rendu le javart plus fréquent parmi les chevaux, les nombreux troupeaux du Comté de Glocester ne tardèrent pas à être de nouveau infectés, & l'auteur saisit cette occasion de faire sur ce sujet d'autres expériences. Voici quel en fut le résultat.
"La maladie commença par une jument qui appartenoit au propriétaire d'une grande laiterie. Elle fut atteinte du javart à la fin de février 1798. Les domestiques qui la pansoient, Thomas Virgoe, William Wherret, & William Haynes, eurent en conséquence des ulcères aux mains, des engorgemens douloureux dans les glandes des bras & des aisselles, des frissons\setcounter{page}{369} alternans avec une chaleur brûlante, une lassitude générale, & des douleurs vagues dans les membres. Ces fymptômes de fiévre ne durent que 24 heures. Haynes & Virgoe qui avoient eu la petite-vérole inoculée, trouvent une grande raffemblance entre les deux maladies. Wherret n'avoit jamais eu la petite-vérole. Haynes fut journellement employé à traire les vaches, & leur communiqua l'infection. Elles commencerent à être malades dix jours après les premiers panfemens des chevaux. Mais comme on leur adminiftra fur-le-champ des remedes, elles furent promptement guéries."
"18ᵉ. Obfervation. Curieux de déterminer exactement les progrès & les effets de la maladie communiquée de cette maniere, je l'inculai à John Baker, enfant de cinq ans, le 16 mars 1798, avec du pus pris fur les ulceres de Thomas Virgoe. Il tomba malade le fixieme jour, eut les fymptômes produits par la petite-vérole des vaches, & fut guéri le huitieme jour. L'apparence du bouton qui furvint à l'endroit de l'incifion n'étoit pas tout-à-fait auffi femblable à celle qu'on obferve dans la petite-vérole inoculée que lorfqu'on inocule la petite-vérole des vaches. Je l'ai fait graver pour que le lecteur puiffe en juger."
( A l'afpect de la gravure, qui eft la feconde du livre, ce bouton nous paroît plus inégal,\setcounter{page}{370} plus blanchâtre & plus élevé que dans la petite-vérole inoculée, & l'érysipele autour de l'incision est moins bien circonscrit & plus rouge dans le centre que vers les bords.)
"Il auroit été intéressant de voir si la maladie communiquée de cette manière à cet enfant l'auroit garanti aussi sûrement de la petite-vérole, que si elle lui avoit été communiquée par l'intervention d'une vache; si le pus du cheval que nous avons vu être un préservatif peu efficace avoit acquis cette propriété dans son passage par le corps humain; s'il y avoit subi la même modification qu'il paroît subir dans le pis de la vache. Mais l'enfant prit bientôt après dans la maison de travail où il étoit élevé, une fièvre contagieuse qui empêcha de l'inoculer \footnote{C'est bien dommage. Si cette expérience avoit été faite, & si elle avoit réussi, on auroit un moyen facile de répéter les expériences de l'auteur; si du moins il est facile de trouver des gens qui consentent à les subir en inoculant le javart même qui probablement est une maladie de tous les pays. Mais s'il faut attendre qu'une vache en ait été infectée; si pour qu'elle le soit, il faut que la personne chargée de panser les ulcères du cheval soit en même temps employée à traire la vache, le concours de ces circonstances est trop rare & trop difficile pour multiplier les observations de ce genre en d'autres pays; d'autant plus que bien avertis, comme le font aujourd'hui les Fermiers, du danger qu'il y a pour leurs bêtes à les faire traire par les mêmes domestiques qui pansent leurs chevaux, ils ne feront plus affez fots pour les y exposer, ce qui fera peut-être bientôt disparoitre entièrement la petite-vérole des vaches. Il est vrai qu'en tout pays on pourroit inoculer directement le javart à la vache ; mais outre que la chose n'est pas aussi facile qu'on l'imagineroit bien, puisque comme on le verra bientôt, cette inoculation faite avec un pus déja épaissi n'a point réussi au Dr. Jenner, cela compliqueroit l'expérience au point d'en détourner peut-être les observateurs. Il y auroit donc un très-grand avantage à n'avoir besoin que du cheval pour produire la maladie fur les hommes, fans lui faire perdre sa propriété de garantir de la petite-vérole ordinaire. A la vérité, les personnes qui l'ont prise directement du cheval même, en pansant ses ulceres n'ont pas paru complétement à l'abri, ce qui pourroit faire croire que le corps humain n'est point propre à produire sur le virus la modification qu'il subit dans le corps de la vache, mais peut-être cette modification n'a-t-elle lieu qu'après la premiere action générale du virus. Pour bien faire l'expérience, il faudroit donc d'abord inoculer un homme, ou un enfant d'après la férosité contenue dans les ulceres d'un cheval atteint du javart ; inoculer ensuite un autre homme d'après le premier, & celui-ci derechef d'après le second. Il faudroit ensuite leur inoculer à l'un & à l'autre la petite-vérole ordinaire. Si, comme je le crois, ils ne la prenoient ni l'un ni l'autre, si cette expérience répétée avoit toujours le même succès, on seroit fondé à croire que le corps humain est tout aussi propre que celui des vaches à donner au virus du javart la modification nécessaire pour qu'il puisse garantir de la petite-vérole.}."
\setcounter{page}{371} 19ᵉ. Observation. Je voulus voir enfuite s'il y auroit quelque différence entre la maladie\setcounter{page}{372} communiquée directement de la vache par inoculation & celle qui avoit été communiquée à l'enfant que j'avois inoculé d'après Sarah Nelmes. (Voyez la 17e. Observation dans le premier Extrait.) Le 16 mars j'inoculai pour cet effet William Summers, enfant de cinq ans & demi, avec du pus pris directement sur le pis d'une vache infectée. Les symptômes furent à-peu-près les mêmes. Il tomba malade le sixieme jour, eut un vomissement, des frifons & un léger mal de tête. Mais ces symptômes se differerent le huitieme jour. La couleur de l'ampoulle fut un peu moins livide que dans le dix-septieme cas."
"J'inoculai ensuite la petite-vérole ordinaire à cet enfant, mais sans succès, quoiqu'une autre personne qui n'avoit jamais eu la petite-vérole des vaches, ayant été inoculée avec le même pus, de la même maniere, & dans le même instant, prit bien la petite-vérole, comme à l'ordinaire."
"20ᵉ. Observation. Je voulus voir après cela, si en passant successivement par le corps de plusieurs personnes, la maladie changeroit de nature & de propriété. William Pead, âgé de 8 ans, fut en conséquence inoculé le 28 mars avec du pus pris sur le bras de William Summers. Au sixieme jour, il se plaignit d'une douleur à l'aisselle. Au septieme jour il fut atteint des symptômes qu'on observe dans\setcounter{page}{373} l'inoculation de la petite-vérole ordinaire, & la ressemblance étoit si grande que je voulus examiner attentivement tout le corps pour voir s'il n'y auroit point d'éruption. Il n'y en avoit aucune apparence. Mais l'efflorescence autour de l'incision ressembloit si parfaitement à celle de la petite-vérole inoculée que je la fis graver au moment où le bouton commençoit à sécher, & à être entouré d'une aréole plus rouge à la circonférence qu'au centre. (C'est la troisieme gravure. Tous ceux qui l'ont vue, en ont été singulièrement frappés.)
"Cet enfant fut ensuite inoculé avec du pus variolique ordinaire, mais sans succès. Les incisions s'enflammerent le troisieme jour. Le malade y éprouva une grande démangeaison; mais le sixieme jour cette inflammation se dissipa, sans produire aucun effet général."
"21ᵉ. Observation. Plusieurs personnes de tout âge furent inoculées le 5 avril, du bras de William Pead. La plus grande partie d'entr'elles tombèrent malades le sixieme jour, & furent guéries le septieme; mais trois eurent une indisposition sécondaire, en conséquence d'un érysipèle fort étendu qui se manifesta autour de l'incision, & qui paroissoit provenir de ce que le bouton étoit plus douloureux & plus gros qu'à l'ordinaire, ayant quatre ou cinq lignes (8 à 11 millimétres) de diametre. Un de ces trois malades n'avoit que six\setcounter{page}{374} mois. L'érysipele fut promptement guéri par l'application de l'onguent mercuriel, application qu'on a recommandée en pareilles circonstances dans la petite-vérole."
"Hannah Excell, âgée de 7 ans, l'une des personnes dont je viens de parler fut inoculée par trois piqûres qui produisirent chacune un bouton tellement semblable à celui que produit l'inoculation de la petite-vérole (ces trois boutons font le sujet de la quatrième gravure) que l'inoculateur le plus expérimenté auroit à peine pu en saisir la différence. La seule qui existe, comme des observations ultérieures me l'ont appris, c'est que le fluide contenu dans les boutons de la petite-vérole des vaches inoculées, demeure limpide jusqu'à la dessiccation, & ne devient pas purulent, comme celui qui est contenu dans les boutons de petite-vérole ordinaire à l'endroit des incisions."
"22ᵉ Observation. Du bras de cette fille, j'inoculai le 12 avril quatre enfants, savoir John Macklove, âgé de 18 mois, Robert-François Jenner, âgé de 11 mois, Mary Pead, âgée de 5 ans, & Mary James, âgée de 6 ans. De ces quatre enfants, R. F. Jenner ne prit pas la maladie. Les autres l'eurent comme à l'ordinaire; mais craignant pour eux l'érysipele observé dans les cas précédens, j'essayai de le prévenir sur deux d'entre eux en recouvrant pendant six heures l'ampoule de l'inci\setcounter{page}{375} fion avec un caustique composé de parties égales de chaux vive & de savon, mais appliqué seulement 12 heures après le commencement de l'indisposition générale. Ce traitement réussit fort bien. Une demi-heure après l'application du caustique, dont les enfans s'apperçurent à peine, leur indisposition ceffa totalement, & il ne survint point d'érysipele \footnote{L'auteur demande ici dans une note quel feroit l'effet d'un pareil traitement dans la Petite-vérole inoculée? Une application caustique faite sur l'incision quelques heures après le commencement de la fièvre éruptive, la feroit-elle cesser, & si elle cessoit, feroit-ce sans préjudice à l'influence préfervatrice de l'inoculation? C'est ce qu'il est difficile de décider a priori. J'avois inoculé un de mes enfans âgé de 2 mois. Lorsque l'apparence des bras me fit préfumer que la fièvre n'étoit pas bien éloignée, je voulus examiner soigneusement quand & comment elle commenceroit. Pour cet effet, dans la soirée du septieme jour, je lui touchai le pouls de quart d'heure en quart d'heure. Il fut parfaitement naturel, & à 90 ou 100 jusqu'à dix heures du soir. Alors je le trouvai tout d'un coup à 130 ou 140, & dès ce moment il se soutint dans cet état de fièvre. Cette observation m'avoit fait préfumer que le pus formé dans l'incision, s'absorbe en un moment donné, & qu'à l'instant où il atteint le cœur, la fièvre commence, c'est-à-dire que le pouls devient très-fréquent. Mais les autres symptômes fébriles, l'abattement, le mal de tête, le dégoût, les maux de cœur &c. ne surviennent que le lendemain, & ce n'est qu'au troisieme ou quatrieme jour que se fait l'éruption générale. Tous ces symptômes m'avoient jusqu'à présent par l'effet du pus absorbé, plutôt que de celui qui est encore dans l'incision. Si donc on arrête l'inflammation du bras par quelque application caustique lorsque l'absorption est faite, lorsque la fièvre a commencé, est-il à présumer que la fièvre cesse ? J'en doute. Mais il arrive si fréquemment que l'expérience renverse les théories les plus plausibles, qu'il feroit très-possible qu'il en fût de même ici ; & j'avoue que ce que l'auteur raconte de la Petite-vérole des vaches ébranle un peu mon opinion, d'autant plus que pendant la fièvre de la Petite-vérole inoculée l'inflammation des bras fait de grands progrès, & qu'enfin lorsque la fièvre cesse, le bouton de l'incision s'entoure pour l'ordinaire d'une aréole moins rouge au centre que vers les bords. Quoiqu'il en soit, c'est l'ensemble de ces symptômes qui constitue le succès de l'inoculation, & il feroit fort étrange qu'on pût les arrêter tout d'un coup sans perdre le fruit de cette opération. Si cela était ainsi, ce serait une des plus belles & des plus intéressantes découvertes médicales qu'on ait faites depuis longtemps. Elle surpasserait bien par son utilité celle de l'influence préservatrice de la Petite-vérole des vaches. (O)}. Il est vrai qu'il\setcounter{page}{376} qu'il n'en survint point non plus à Mary Read, sur le bras de laquelle on n'avoit point appliqué de caustique.
23ᵉ. Observation. Enfin, du bras de M. Pead, j'inoculai J. Barge, enfant de 2 ans. La maladie qui en fut la suite eut exactement la même apparence, & les mêmes symptômes que dans les cas précédens. Elle eut aussi la même influence préservatrice contre la petite-vérole\setcounter{page}{377} vérole ordinaire, pour laquelle il fut inoculé avec beaucoup de foin, mais sans succès. Il n'en résulta qu'une inflammation passagere autour des incisions."
"La petite-vérole des vaches ne perd donc point ses propriétés primitives en passant successivement d'un corps humain dans un autre, puisque cet enfant étoit le cinquieme à qui elle avoit été successivement transmise de cette maniere."
Aux vingt-trois observations dont nous venons de parcourir rapidement les détails, l'auteur ajoute des remarques générales qui contiennent encore quelques faits intéressans & des réflexions ingénieuses. En voici l'abregé.
"Je crois avoir bien démontré, dit-il, que la petite-vérole des vaches est un préservatif assuré contre la petite-vérole ordinaire. S'il étoit nécessaire d'apporter de nouvelles preuves de ce fait, je pourrois ajouter qu'il a été amplement confirmé par le témoignage de Mr. Dolland, chirurgien, demeurant dans un Comté fort éloigné du Glocestershire, également fameux par ses laiteries. Mylord Sommerville, Président du bureau d'Agriculture, à qui Sir Joseph Banks, avoit communiqué mes observations, s'étoit adressé à ce chirurgien pour avoir des informations sur cet objet. Elles ont été parfaitement conformes à ce que j'ai vu. La petite-vérole\setcounter{page}{378} des vaches & la propriété qu'elle a de garantir de la petite-vérole ordinaire font bien connues aussi dans ce pays-là."
"Il paroît encore suffisamment prouvé que la petite-vérole des vaches doit son origine au javart des chevaux, & je suis parfaitement convaincu que jamais elle ne se manifeste que dans les endroits où un cheval étant atteint du javart, les domestiques qui pansent ses ulcères font en même temps occupés à traire les vaches, à moins qu'elle ne leur ait été communiquée par des domestiques employés à traire une vache venue d'ailleurs & déjà infectée, ou qu'ils ne la leur aient apportée eux-mêmes d'une autre ferme. J'aurois voulu faire des expériences directes pour vérifier ce fait. J'en avois projeté quelques-unes pour le printemps de 1797, mais la sécheresse de la saison rendit la chose impossible. Il n'y eut dans mon voisinage aucune vache infectée, aucun cheval atteint du javart. On fait que cette maladie a pour cause principale les pluies froides auxquelles les chevaux sont souvent exposés au printemps. Quand cette saison n'est pas pluvieuse, il est rare qu'ils en soyent atteints.
"Le fluide contenu dans les ulcères d'un cheval atteint du javart, subit dans le pis de la vache une modification qui augmente certainement son activité. Car d'un côté, il est rare que les domestiques qui pansent les ulcères\setcounter{page}{379} d'un cheval malade en foyent affectés; & de l'autre il n'est pas moins rare que dans le nombre des personnes employées à traire une vache infectée, il y en ait une seule qui échappe à l'infection."
"Le moment où le virus du cheval a le plus d'activité est le commencement de la maladie, lorsqu'elle ne se manifeste point encore par des ulcères purulens, mais par des fentes qui se forment aux talons de l'animal, & dont il suinte un fluide transparent, mais d'une couleur foncée, semblable à celui qui se forme dans les ampoules produites par un érysipèle. Je suis porté à croire que ce fluide perd son activité quand il devient purulent \footnote{Cela me paraît fort extraordinaire. Car quoique dans la petite-vérole le pus le plus limpide m'ait toujours paru le plus actif, cependant tout le monde fait qu'on peut encore inoculer avec succès lorsqu'il est extrêmement épais & prêt à sécher. Les croûtes même sont contagieuses , & qui pis est , les petites écailles qui se forment sous ces croûtes , lorsqu'elles sont tombées , le sont probablement aussi. Van Swieten raconte ( Commentar. in H. Boerhaave Aphorism. 1403. T. V. p. 150) que dans le Collège de Thérèse, grande maison d'éducation près de Vienne, où l'on séquestrait sur le champ les jeunes gens atteints de la petite-vérole, & où l'on était fort intéressé à abréger la durée du séquestre , on avait observé qu'il fallait le prolonger au moins 60 jours , (per novem septimanas) pour que le convalescent pût communiquer avec les autres enfans sans leur faire courir le risque de la contagion. C'est sans doute par cette raison que la petite-vérole se communique avec tant de rapidité. Car long-temps avant l'expiration des 60 jours, les malades fortent, vont & viennent, se mêlent dans la société, & fréquentent les promenades, les églises, les collèges & tous les autres lieux publics, sans qu'on se défie d'eux. (O)}. Car j'ai\setcounter{page}{380} souvent essayé d'inférer du pus pris sur de vieux ulcères au talon d'un cheval dans une incision faite avec une lancette sur le pis d'une vache, sans jamais produire par cette opération d'autre effet sur la vache qu'une inflammation locale & passagere."
Je soupçonne aussi que les vaches ne sont susceptibles de recevoir l'infection d'un cheval qu'au printemps & au commencement de l'été, saison pendant laquelle elles font fort sujettes à des éruptions spontanées, qu'elles n'ont point dans d'autres temps. Mais ce n'est là qu'une conjecture qu'il faudroit soumettre à une suite d'expériences directes. Quoiqu'il en soit, il est bien certain que dès que la maladie est formée dans la vache, elle peut se communiquer dans toutes les saisons de l'année indifféremment.
"Le virus du cheval ou de la vache est-il susceptible d'infecter la peau humaine par le simple contact, lorsqu'elle est parfaitement saine? C'est ce que je n'ai pu déterminer encore\setcounter{page}{381} d'une manière exacte. Je soupçonne que non, à moins que le contact n'ait lieu dans des endroits où l'épiderme est extrêmement mince, comme sur les lèvres. J'ai vu une jeune fille qui pour apaiser la chaleur douloureuse qu'elle éprouvait à l'un de ses doigts, où la maladie s'était manifestée, le portait souvent à sa bouche, & soufflait dessus. Il en résulta des ulcères semblables sur les lèvres. Mais à cette exception près, je n'ai connu aucun exemple bien prouvé de communication par le simple contact sans excoriation. Et si les domestiques qui trouvent une vache infectée échappent si rarement à l'infection, c'est parce que la nature de leurs occupations les expose à avoir presque toujours ou quelque coupure sur les doigts, ou quelque piqûre d'épine, ou quelqu'autre accident pareil qui met à nu leur peau.
"Il est singulier que la petite-vérole des vaches qui rend le corps humain inaccessible à la petite-vérole ordinaire, ne lui donne point cette sécurité relativement à elle-même, & qu'elle soit susceptible de revenir plusieurs fois. J'en ai déjà cité un exemple; voyez la neuvième Observation. En voici un autre. Elizab. Wynne, qui avait eu la petite-vérole des vaches en 1759 fût inoculée en 1797 avec du pus variolique sans succès. Mais en 1798 elle prit la petite-vérole des vaches pour la seconde fois. Je la vis au huitième jour de l'infection. Elle se\setcounter{page}{382} plainoit de lassitude & de faiblesse générale; & avoit alternativement des frissons & de la chaleur, le pouls fréquent & irrégulier, les extrémités froides. Ces symptômes avoient été précédés par un engorgement douloureux à l'aisselle. Elle avoit sur la main un bouton ulcére semblable à ceux que j'ai décrits ci-dessus.
"C'est aussi une chose très-remarquable que le virus du cheval, dont les effets sont précaires, incertains & indéterminés, avant que d'avoir passé par le pis de la vache, devienne alors non-seulement plus actif, mais encore invariablement doué de cette propriété spécifique de produire constamment sur le corps humain; des symptômes semblables à ceux de la fièvre variolique, & d'opérer sur lui ce changement qui le rend pour toujours inaccessible à la contagion de la petite-vérole. Ne peut-on pas raisonnablement conjecturer de-là que la petite-vérole doit son origine à un fluide engendré par quelque maladie dans le cheval, & successivement modifié ensuite par des circonstances accidentelles dont le concours lui a donné enfin cette forme contagieuse & maligne sous laquelle nous le voyons faire tant de ravages? Et quand on considère le changement que le virus du cheval subit dans le corps de la vache, ne peut-on pas imaginer que le virus de la plupart des maladies contagieuses qui regnent parmi les hommes peut avoir été accidentellement\setcounter{page}{383}  produit par des causes plus compliquées qu'on ne l'imaginerait d'abord et avoir successivement subi plusieurs modifications, auxquelles ces maladies, très-différentes peut-être dans l'origine de ce qu'elles font aujourd'hui, doivent enfin leur apparence actuelle. Il est aisé de concevoir, par exemple, que la rougeole, la fièvre rouge, et cette éruption analogue dont le caractère principal est d'être accompagné d'ulcères dans la gorge, peuvent toutes avoir eu la même origine, qui modifiée ensuite par différentes combinaisons nouvelles, en a fait autant de maladies spécifiquement distinctes les unes des autres \footnote{Certaines maladies contagieuses ne pourraient pas aussi avoir une origine végétale, et provenir primitivement de l'attouchement de quelque plante vénimeuse. Il y a quelques années que dans plusieurs terrasses et jardins de notre territoire, on cultivait un assez joli arbrisseau. C'était, si je ne me trompe, le Rhus Toxicodendron Linn. On l'émondeait toutes les années au printemps, et alors il y avait toujours quelques-uns des jardiniers et des enfants qui l'avaient touché, qui prenaient des rougeurs, des échauboulures et des boutons. Quant on s'en aperçut, on recommanda aux enfants de s'abstenir soigneusement d'y toucher, et les jardiniers ne le tailleraient plus que gantés. En se promenant un jour sur une terrasse où il y avait quelques-unes de ces plantes, une Dame à qui on les faisait voir eut la curiosité d'en couper une feuille et de s'en frotter le bras. Elle n'éprouva rien d'abord, mais quelques jours après, elle aperçut sur son bras de sa rougeur & de l'inflammation, puis un petit amas de boutons en suppuration & ayant une apparence dartreuse. Ces boutons se communiquèrent à l'avant bras avec lequel ils étoient en contact dans les mouvemens de flexion. Ils la répandirent de-là par le contact sur tout le corps, & furent accompagnés d'une forte de fiévre bilieuse, dont elle eut bien de la peine à se remettre, & qui dura deux mois. Cette maladie étant le produit d'une vraie inoculation auroit certainement pu se communiquer à d'autres personnes, & de l'une à l'autre, on conçoit facilement qu'elle auroit pu prendre une forme régulière & devenir générale.}.
\setcounter{page}{384} Indépendamment des différentes nuances dont la petite-vérole est susceptible, selon qu'elle est distincte ou confluente, elle peut certainement se montrer sous plusieurs formes différentes. Il y a environ sept ans qu'il régna dans la plupart des villes & des villages du Comté de Glocester, une espèce de petite-vérole qui étoit si bénigne, qu'on auroit dit que tous les malades avoient été inoculés. Pas un n'en mourut ou ne l'eut même confluente, au moins à ma connoissance. Cette bénignité ne tenoit certainement ni à la sécheresse ou à l'humidité de l'air, ni à sa température. Car je suivis les progrès de cette maladie pendant plus d'un an, sans jamais appercevoir aucune variation dans son apparence générale. Et mon respectable ami, le Dr. Hicks alors médecin de l'hôpital de Glocester, & qui se propose\setcounter{page}{385} de publier les nombreuses observations qu'il eut occasion de faire sur cette maladie, fit la même remarque. On ne peut donc la considérer que comme une espèce de petite-vérole différente de la petite-vérole ordinaire, ou plutôt selon le langage des botanistes, comme une variété, c'est-à-dire, une modification particulière de la petite-vérole, peut-être accidentelle, mais susceptible de se propager \footnote{C'est peut-être quelque épidémie de ce genre qui faisoit dire aux Professeurs De Haën & Van Swieten qu'il n'y avoit pas une grande différence de mortalité entre la petite-vérole naturelle & l'inoculée, puisque sur 220 malades à la Haye dont le premier avoit pris note, il n'en étoit mort que 5, & que sur 355 dont le second avoit pu recueillir l'histoire à Vienne & dans les environs, il n'en étoit mort que 7; si du moins ces assertions méritent quelque confiance. Mais on peut à juste titre soupçonner quelque illusion dans ces sortes de calculs faits dans des maisons particulières & sur des malades choisis. J'ai connu un médecin qui pendant une forte épidémie de petite-vérole que nous eûmes à Genève en 1776 & qui emporta 210 malades, en traita 49 avant que d'en perdre un seul, quoiqu'il y en eût quelques-uns dont la petite-vérole paroissoit très-confluente & très-maligne. Si raisonnant à la manière de Mrs. De Haën & Van Swieten, il avoit conclu de-là qu'en général sur 49 malades de petite-vérole il n'en meurt pas un, la suite de sa pratique l'auroit cruellement détrompé. Le 50^e malade mourut. Le 51 mourut aussi. Le 57 mourut encore. Enfin quand il en eut vu 73, il en étoit mort 7, outre quelques autres qu'il vit aussi dans le même temps, mais qu'il refusa de traiter, parce que c'eût été trop tard, & qu'il les jugeoit mourans.}.
\setcounter{page}{386} "Il a souvent été question dans les observations précédentes d'inoculations faites avec un pus bien choisi. C'est en effet une précaution très-importante, & dont l'oubli a eu de funestes conséquences. J'ai connu autrefois un inoculateur qui portoit constamment dans sa poche un petit flacon bouché & rempli de coton ou de lin imprégné de pus variolique en état de fluidité, avec lequel il ne se faisoit aucun scrupule d'inoculer, & souvent après l'avoir gardé fort long-temps sur lui. Il en résultoit à la vérité de l'inflammation aux incisions, des engorgemens à l'aisselle, de la fièvre, & quelquefois une éruption. Mais cette maladie étoit-elle bien la petite vérole? Pouvoit-elle mettre les malades à l'abri de la reprendre? Non certainement. On ne peut pas supposer que du pus placé dans les circonstances les plus propres à le faire entrer en putréfaction, pût conserver ses propriétés spécifiques, & donner une petite vérole véritable, aussi plusieurs de ses inoculés exposés ensuite à la contagion prirent la petite vérole; & quelques-uns moururent victimes de la malheureuse sécurité qu'il leur avoit inspirée. J'ai eu connoissance de malheurs semblables arrivés à d'autres inoculateurs, probablement aussi en conséquence de la maniere\setcounter{page}{387} peu foigneufe dont ils gardaient le pus variolique fans le mettre à l'abri de la putréfaction\footnote{C'est une remarque que fait aussi Mr. Hufeland d'après quelques praticiens d'Allemagne, & sur laquelle il insiste beaucoup. (Voyez la Bibli. Germanique N°. 1.) Quant à moi je n'ai jamais eu occafion de rien obferver de femblable. J'ai fouvent inoculé avec du pus ancien , mais féché fur un verre , fur un fil de coton , fur un morceau de toile , ou fur une éponge ; je le délayois dans de l'eau au moment de m'en fervir & la feule différence que j'aye vu entre ce pus & le pus frais dont je ne me fuis jamais fervi qu'au moment où je venois de le prendre , c'eft que le premier manque beaucoup plus fouvent que le fecond. Mais quand il réuffit la maladie eft exactement la même. Puifque l'occafion s'en préfente je dirai un mot d'une autre erreur dont j'ai vu deux exemples , & fur laquelle un jeune inoculateur ne fauroit être trop circonfpect. Il arrive quelquefois que les boutons de la petite-vérole volante reffemblent fi fort à ceux de la vraie petite-vérole , qu'on a pris l'une pour l'autre & qu'on a inoculé avec le pus de ces boutons. Dans les deux cas dont je parle , il en résulta de l'inflammation au bras & une maladie éruptive affez femblable à celle de la vraie petite-vérole. Heureusement qu'on s'aperçut de la différence , & qu'on s'en défia. On fit une feconde inoculation avec de la petite-vérole bien choisie & elle réuffit. Comme ce n'étoit pas moi qui avoit fait les premières , je n'ai pas pu juger exactement s'il y avoit eu quelque différence d'action locale entre les deux virus ; mais je fus averti qu'il y en avoit eu une très-sensible dans la nature de l'éruption , & en allant à fa source , on reconnut l'erreur. (O)}
\setcounter{page}{388}"En parlant d’inoculation, je ne puis m’empêcher de faire aussi mention d’une autre circonstance que je regarde comme d’une grande importance. J’ai de fortes raisons de croire que si les piqûres ou les incisions font assez profondes pour traverser la peau, & atteindre le tissu cellulaire, le risque de produire une maladie violente est beaucoup augmenté. Avant que les Suttons eussent introduit la méthode des piqûres ou des incisions très légères, j’ai connu un inoculateur qui faisoit une incision assez profonde pour laisser voir un peu de graisse, (c’étoit son expression) & qui y inféroit le pus variolique. Mais indépendamment des accidents locaux, de l’irritation, de la douleur, de l’inflammation aux bras, & des abcès qui résultoient de cette pratique, on auroit peine à croire combien de malades avoient entre ses mains une petite vérole confluente & même mortelle. Un autre inoculateur, que je me rappelle bien, passoit avec une aiguille & retenoit sous la peau un féton imprégné de pus variolique. Les suites de cette méthode n’étoient pas moins déplorables. A peu près dans le même temps, le Dr Hardwick de Sudbury, avait au contraire un succès presque aussi grand que celui qu’ont eu depuis les Suttons. Il faisait sur la peau une incision aussi superficielle que possible, & y plaçoit le fil. Pendant la fièvre éruptive & durant tout le cours de la\setcounter{page}{389} maladie, il tenoit ensuite, conformément à la théorie reçue dans ce temps-là, tous ses malades au lit, & avoit foin qu'ils éprouffaffent constamment une chaleur douce? Ne peut-on pas conclure de cette différence de succès entre les deux méthodes, que celui des inoculateurs modernes tient probablement plus à la manière de placer superficiellement le virus sur la peau, qu'au traitement subséquent de la maladie? \footnote{Je suis porté à croire que le véritable siége de préférence pour que la petite-vérole soit bénigne, est le tissu muqueux entre l'épiderme & la peau. Car si le virus est appliqué sur l'épiderme sans incision, je ne saurois voir aucune différence entre cette manière de communication, & celle de la petite-vérole naturelle, à moins qu'on ne suppose que celle-ci se communique plus fréquemment par les exhalaisons du malade que par le simple contact: quoiqu'il en soit, j'ai vu un cas d'inoculation naturelle, dans lequel la différence fut frappante. J'avois inoculé un petit enfant qui eut une éruption assez abondante; cet enfant fut pendant toute la maladie porté par sa Bonne. Cette fille n'avoit point dit qu'elle n'eût jamais eu la petite-vérole. Elle s'étoit contentée de dire qu'elle ne la craignoit pas. Elle se trouvoit accidentellement avoir une petite écorchure au col, qui fut constamment en contact avec les bras de l'enfant. Quelques jours après elle se plaignit d'une légère inflammation à cet endroit. Il s'y forma un bouton qui vint en suppuration, produisit de la fièvre, & s'entoura enfin d'une aréole érysipélateuse, au moment où l'éruption générale se fit. La maladie fut très-heureuse, très-bénigne & parfaitement semblable à la petite-vérole inoculée ; non-seulement par le peu d'intensité des symptômes, et par l'apparence du bouton qui en avoit été le précurseur, mais encore par un autre caractère spécifique qui n'a point été assez remarqué, quoiqu'il suffise seul' pour distinguer la petite-vérole naturelle la plus bénigne, de la petite-vérole inoculée la plus abondante ; c'est que la déssiccation des boutons commença au moins trois jours plutôt dans la petite-vérole inoculée que dans la petite-vérole naturelle.(O)}
\setcounter{page}{390}"Ce n'est pas que je prétende insinuer par là que l'air frais et les boissons froides pendant l'ardeur de la fievre, ne tendent point à l'appaifer et à diminuer utilement le nombre des boutons. Je veux dire seulement qu'il est impossible d'expliquer le succès non interrompu de la méthode du Dr. Hardwick, et les tristes accidens qui suivoient celle des inoculateurs dont je viens de parler, sans l'attribuer à la différence de leur manière d'inférer la petite-vérole, puisque les uns et les autres avoient recours dans la maladie subséquente au même traitement.
"Et puisque ce n'est pas le pus que l'on infère dans l'incision qui s'absorbe dans le sang, mais celui qui, par une opération particulière à l'économie animale, y est engendré, on peut concevoir que quoique la peau, la membrane adipeuse et la membrane muqueuse soyent toutes capables d'engendrer ce pus, lorsqu'elles\setcounter{page}{391} font stimulées par le contact du pus variolique, l'action de chacune de ces parties peut cependant produire quelque variation dans la qualité du pus qu'elle engendre. Et comment pourroit-on expliquer autrement la différence qui se trouve entre la petite-vérole inoculée et la petite-vérole naturelle? Après tout, il y a lieu de croire que lorsque l'absorption a lieu, les particules varioliques absorbées, ou n'ont point encore leurs qualités spécifiques et contagieuses, ou qu'elles les perdent en conséquence de quelques changemens dans leurs principes par cette absorption même, puisqu'il a été prouvé que quoique le pus variolique prodigieusement dilayé dans de l'eau, et probablement autant et plus qu'il le feroit dans le sang, s'il y exittoit en nature, donne fort bien la petite-vérole, le sang lui-même des malades ne peut jamais la donner\footnote{J'avois un jour quatre enfans à inoculer. Il n'y avoit point de petite-vérole dans la ville. Mais un officier de santé que j'avois chargé de m'avertir quand il y en auroit à la campagne, m'ayant indiqué une maison à la distance de trois kilomètres de la ville, en m'assurant que j'y trouverois ce que je cherchois, et que le malade étoit un enfant très-fain, je pris une voiture et j'y conduisis les quatre enfans pour les inoculer. Il se trouva que la petite-vérole étoit entièrement séche, au point que je ne pus découvrir un seul bouton en fuppuration. Mais en visitant bien le malade, j'aperçus entre le pouce et l'index une grosse ampoule pleine d'une sérosité limpide & jaunâtre. Je l'ouvris & j'inoculai les quatre enfants avec cette sérosité. Mais cette inoculation fut complètement inutile. Il n'y eut aucune apparence d'action ni locale, ni générale, & il fallut les réinoculer. Les fluides hors des boutons ne sont donc pas imprégnés d'un virus capable de donner la maladie. (O)}. — Mais cette digression est trop\setcounter{page}{392} éloignée de mon sujet pour la prolonger. Je reviens à la petite vérole des vaches.
"On ignore quand on a commencé à y faire attention dans ce pays. Nos plus vieux fermiers l'ont connue dans leur enfance. Ils se rappellent qu'elle avait alors les mêmes symptômes & la même marche qu'aujourd'hui. Mais ils ignoraient à cette époque ses rapports avec la petite vérole ordinaire; c'est probablement l'introduction générale de l'inoculation qui les a fait découvrir.
"Son ancienneté dans le pays n'est probablement pas bien grande. Car il est vraisemblable qu'on n'employait pas des hommes à traire les vaches, ni des femmes à panser les chevaux malades. Dans plusieurs autres pays de laiteries, ce n'est point la coutume de confondre, ainsi les occupations des deux sexes. En Irlande le plus chétif valet se croirait déshonoré si on l'occupait à traire une vache. Aussi la maladie y est-elle absolument inconnue.
"Au reste, la plupart de nos fermiers ignoraient autrefois que la maladie provînt de cette source.\setcounter{page}{393} Ce n'est que depuis quelques années qu'ils le savent; & cette connoissance a déjà produit de bons effets. Il est probable que les précautions qu'ils sont disposés à prendre à l'avenir, détruiront enfin cette maladie, ou la rendront du moins extrêmement rare.
"Que si l'on demande à quoi peuvent servir ces recherches sur ses rapports avec la petite-vérole, je répondrai que malgré les bienfaits incontestables de l'inoculation, perfectionnée comme elle l'est aujourd'hui, il est pourtant vrai qu'entre les mains les plus habiles on voit quelquefois des accidents à la suite de la petite-vérole inoculée; des cicatrices difformes; quelquefois même la mort. Il est impossible que ces malheurs n'inspirent pas un sentiment d'inquiétude & de terreur à l'idée de l'inoculation. Mais comme d'une part la petite-vérole des vaches est contre la petite-vérole ordinaire un préservatif aussi sûr que la petite-vérole inoculée, & que de l'autre elle n'est jamais accompagnée d'aucun accident qui puisse donner des craintes pour la vie, même dans les circonstances les plus défavorables, lorsqu'elle produit par exemple, une inflammation & des ulcères étendus sur les mains, ne peut-on pas conclure de-là que l'on pourrait en tirer parti pour introduire une manière d'inoculation préférable à celle qui est généralement adoptée aujourd'hui, sûrtout dans les familles où l'on\setcounter{page}{394} auroit lieu de craindre quelque mauvaise prédisposition héréditaire? Ce qu'il y a de plus à redouter dans la petite-vérole, c'est une éruption trop abondante. Or dans la petite-vérole des vaches, jamais il n'y a d'éruption.
"Jamais aussi elle ne se communique par les exhalaisons du corps malade. Ce n'est que par le contact. Et même il ne paraît pas qu'un simple contact entre le virus & l'épiderme suffise. Il faut qu'il soit appliqué sur la peau, mise à nud & dépouillée de la furpeau pour produire son effet. D'où il résulte que dans une famille on peut donner la maladie à un seul individu sans courir le risque d'infecter les autres, & de répandre la terreur dans le pays. J'ai vu plusieurs exemples de cette impossibilité de communiquer la petite-vérole des vaches par les exhalaisons des malades ou par le simple contact. Le premier enfant auquel je l'inoculai coucha pendant toute la maladie avec deux autres enfants qui n'avaient eu ni cette maladie, ni la petite-vérole ordinaire, sans infecter ni l'un ni l'autre. — Une jeune femme affectée de la petite-vérole des vaches, au point d'avoir sur les mains & sur les poignets plusieurs ulcères étendus & en état de suppuration avec beaucoup de fièvre, coucha de même pendant toute sa maladie avec une de ses compagnes qui n'avait jamais eu ni l'une ni l'autre espèce de vérole, sans lui communiquer aucune\setcounter{page}{395} maladie. -- Une troisieme femme qui en était atteinte au même degré, nourrissait un enfant, & ne l'abandonna point, sans que son nourrisson eût aucun mal.
"A d'autres égards encore l'inoculation de la petite-vérole des vaches aurait quelque avantage sur celle de la petite-vérole ordinaire. Nous voyons souvent celle-ci dans les personnes d'un tempéramment disposé aux scrofules exciter & mettre en activité cette cruelle maladie, & cela arrive tout aussi fréquemment lorsque les boutons font en petit nombre & d'une bonne nature que lorsque la petite-vérole est confluente ou d'un mauvais aspect. Peut-être que la petite-vérole des vaches n'aurait pas cet inconvénient. — On voit un assez grand nombre de personnes qui par quelque particularité de leur tempéramment ne peuvent prendre la petite-vérole par inoculation, & qui font toute leur vie tourmentées de la crainte de la prendre naturellement. Il est vraisemblable que l'inoculation de la petite-vérole des vaches pourrait servir à dissiper ces terreurs. Car comme on peut l'avoir plusieurs fois, & comme la propriété qu'elle a de garantir de la petite-vérole ordinaire n'est que jusqu'à un certain point réciproque, il est probable que cette inoculation donnerait la première, & les mettrait par-là pour toujours à l'abri de la seconde. Il serait fort extraordinaire que ces personnes là\setcounter{page}{396} se trouvant tout à la fois inaccessibles à l'une & à l'autre.
"Aussi dans le seul cas que j'aye vu où l'action du virus de la petite-vérole des vaches ne se manifesta que par des symptômes locaux sans aucune indisposition générale, la malade se trouva cependant susceptible de prendre la petite-vérole ordinaire. Voici le fait. Elisabeth Sarfenet servoit dans une laiterie de Newpark, paroisse de Berkeley, lorsque la maladie se déclara dans le troupeau, & affecta tous les domestiques employés à traire les vaches. Ils eurent tous des symptômes d'indisposition générale. Elle seule n'en eut aucun, pas même d'engorgement aux glandes axillaires, quoi-qu'elle eût plusieurs ulcères sur les doigts. Elle fut dans la suite accidentellement exposée à la contagion de la petite-vérole ordinaire, & la prit. Hannah Pick, une de ses compagnes, eut dans le même temps tous les symptômes de la petite-vérole des vaches complète, ulcères sur les doigts, tumeur & douleur sous les bras, fièvre, mal de tête, &c. Aussi plusieurs essais que je fis ensuite pour lui donner la petite-vérole ordinaire par inoculation, furent inutiles.
"On pourroit peut-être encore faire servir cette susceptibilité du corps humain de prendre la petite-vérole des vaches toutes les fois qu'il y est exposé, au soulagement ou à la guérison de plusieurs maladies chroniques dans lesquelles si on l'inoculoit comme remède, tout nous annonce qu'une diversion de ce genre pourroit être utile.
"Je terminerai ces recherches que je me propose de poursuivre lorsque l'occasion s'en\setcounter{page}{397} présentera, par l'exposé d'un fait récent qui semble prouver que le virus de la petite-vérole des vaches peut s'engendrer non-seulement bas de la jambe des chevaux, mais encore sur d'autres parties de leur corps. Un érysipele étendu se manifesta sans aucune cause apparente sur le haut de la cuisse d'un jeune poulain qui tétoit encore, & qui appartenoit à Mr. Millet, fermier de Rockhampton, près de Berkeley. Cette maladie dura plusieurs semaines, & se termina par trois ou quatre petits abcès. Le domestique qui les pansoit étoit aussi employé à traire les vaches. Il y en avoit 24 dans la ferme. Elles eurent toutes les symptômes de la maladie des vaches. La femme du fermier, le domestique qui pansoit le cheval, & une servante de la ferme, qui étoient les seules pérsonnes employées à les traire, les prirent aussi ; les deux domestiques en furent légèrement affectés, parce que le valet avoit eu autrefois la petite-vérole ordinaire & la servante celle des vaches. Mais la fermiere qui n'avoit jamais eu ni l'une ni l'autre maladie en fut très-cruellement incommodée. La nature des symptômes qu'elle éprouva ne permit gueres de douter que la maladie communiquée par le poulain aux vaches, & par les vaches aux pérsonnes employées à les traire ne fût bien la vraie & non la fausse petite-vérole des vaches (1). Mais pour le prouver\footnote{Voyez le premier Extrait, page 268. Ne pourroit-on pas tirer parti de cette observation pour essayer si un érysipele artificiellement produit sur la peau d'un cheval ne donneroit pas la maladie ? Si le javart est, comme on le dit, une maladie accidentelle, & qui n'a rien de spécifique, il semble probable que de quelque maniere que fût produite la férosité qui donne lieu à la petite-vérole des vaches, elle pourroit toujours avoir cet effet; que c'est une propriété particulière à la férosité accidentellement épanchée par quelque cause que ce soit dans le tissu muqueux des chevaux; & si, comme je le présume, le corps humain est aussi propre à la modifier que celui des vaches, on auroit par là un moyen facile de faire l'expérience dont je parlois plus haut, page 370 à la note. (O)}\setcounter{page}{398} complètement, il auroit fallu inoculer la petite-vérole ordinaire à la fermiere. C'est ce que des circonstances particulières ne permirent pas."
Tel est le sommaire, ou pour mieux dire la presque totalité de l'intéressant ouvrage du Dr. Jenner. Il est le premier qui ait fait connoître au public la maladie qui en fait l'objet, & la possibilité de garantir de la petite-vérole par l'inoculation de cette maladie. Mais un auteur anonyme qui ne paroît pas avoir eu connoissance de ces faits, vient de hazarder une idée analogue en proposant une suite d'expériences pour essayer si l'inoculation du Claveau, maladie des brebis qui a beaucoup de ressemblance avec la petite-vérole, ne pourroit pas utilement remplacer l'inoculation de la petite-vérole elle-même. Il ne cite aucun fait à l'appui de cette conjecture. Ainsi les observations à faire sur ce sujet auroient le mérite de la nouveauté\footnote{Voyez la feuille du Cultivateur du 12 frimaire N°. 15; on lit aussi dans celle du 7 frimaire N°. 14, des expériences très-curieuses du Cit. Coste. Il avoit éprouvé par l'effet du claveau une mortalité considérable dans ses moutons; & ayant observé que les plus jeunes bêtes avoient toujours la maladie plus bénigne, il imagina de l'inoculer à ses agneaux. Il fit frotter pour cet effet la surface du bercail qui recevoit ordinairement son troupeau avec la peau d'un mouton mort de la maladie. Quelque grossier que fût ce mode d'inoculation, il réussit; sur 350 bêtes inoculées, il n'en perdit que deux, tandis qu'auparavant il en avoit perdu 230 sur le même nombre ; les 348 qui en furent par ce moyen préservées, prospérèrent & se vendirent de 125 à 150 centimes par bête au-dessus du prix ordinaire, en considération de ce qu'elles avoient eu la maladie. Au reste, ce n'est pas la première fois qu'on ait inoculé le claveau. Le Prof. Venel & le Cit. Tessier avoient déjà tenté cette expérience avec succès ; & l'un de nos compatriotes, le Cit. Lullin, qui passionné pour l'Agriculture marche depuis longtems avec honneur dans cette carrière sur les traces de son ayeul, le célèbre Dechâteauvieux, nous informe que menacé un jour de perdre par l'effet du claveau qui s'y étoit manifesté, les trois quarts d'un grand troupeau de brebis qu'il soignoit, il inocula par l'incision toutes les bêtes encore intactes, & réduisit par ce moyen sa perte à un huitième.};\setcounter{page}{399} & quelqu'en fut le résultat, fut-il même absolument négatif; ces observations ne pourroient manquer d'être utiles sous plus d'un rapport. Ceux de nos lecteurs qui voudront se mettre au fait de cette maladie, de sa marche, de ses symptômes, de son analogie, pour ne pas dire de son identité avec la petite-vérole, ainsi qu'avec le claveau des dindons, & des différens moyens de guérison dont elle est susceptible, peuvent consulter l'excellente instruction sur le claveau des moutons, rédigée par le Cit. Gilbert, professeur de l'école vétérinaire d'Alfort, & publiée par le Conseil d'Agriculture sous le ministère du Citoyen BENEZECH.
\setcounter{page}{285} 
\chapter{Chimie}
\section{OBSERVATIONS ET EXPERIMENTS, &c. Observations & expériences sur la formation du fer; par Mr. SMITH: & annonce d'une découverte sur la faculté qu'ont les terres simples d'absorber l'oxygène; par HUMBOLDT.}
Mr. SMITH, après avoir montré, dit-il, dans des mémoires précédens ( qui ne nous sont pas connus ) que les terres calcaire & argilleuse; la terre végétale, ou le terreau proprement dit; enfin que le soufre & le phosphore peuvent se retirer de l'air commun, & que la force vitale tant animale que végétale est l'agent principal de la formation de ces substances, se persuade maintenant que cette même force, en décomposant, de diverses manieres, l'air & l'eau, forme ou produit, toutes les substances terrestres sans exception.
On fait que le fer se trouve partout & en particulier dans les matières animales & végétales; & à cet égard, l'auteur se demande si ce métal est, comme l'argile, formé par l'action vitale, ou bien s'il est produit dans l'acte même de la décomposition de ces substances? Il penche vers cette dernière opinion, & cherche à l'établir par quelques expériences.
\setcounter{page}{286} Dans la première, il fait croître cinq plantes de cresson dans un mélange d'argile & de terre siliceuse, humectées d'eau; & après avoir deffiché & incinéré ces plantes, & leffivé leurs cendres, il trouve, par l'essai du prussiate de potasse, qu'il se précipite du bleu de prusse dans la lessive, & que par conséquent elle contient du fer. Cependant les terres dans lesquelles ces plantes ont végétén'ont perdu aucune partie sensible de leur poids.
Mais, comme il feroit, à la rigueur, possible que ces plantes eussent extrait de ces terres une quantité de fer si petite qu'elle eût échappé aux instrumens employés pour s'assurer du poids de celles-ci avant & après l'expérience, l'auteur s'y prend d'une autre manière pour obtenir un résultat qui ne prête pas à cette objection. Nous transcrirons mot à mot son procédé.
"Je pris, dit-il, un quart d'once de crin de cheval, cuit au four ainsi qu'on le prépare pour garnir les meubles. Je le tins pendant douze heures dans une température de 105 degrés (32 3/4 R.) au moyen d'une lampe à esprit-de-vin. J'en pesai ensuite soixante grains. Je choisis cette substance afin que les racines des plantes que j'y ferois croître puissent en être dégagées sans danger de les rompre, ce qui feroit arriver si j'eusse employé du coton, de la flanelle ou des chiffons de toile. Je mis ce crin dans un vaisseau plat & j'y semai soixante grains de moutarde.\setcounter{page}{287} Je mis le tout dans une ferre chaude. Les plantes s'élevèrent très-haut, dans l'efpace de huit jours, à raifon de la chaleur & du défaut de renouvellement dans l'air. Je les fortis à cette époque, & après avoir foigneufement dégagé toutes les racines je defféchai le crin à la même température que précédemment, & je lui trouvai précisément le même poids. Les plantes pesoient deux onces & un quart, & je leur avois fourni quatre onces & demie d'eau de pluie dans divers arrosements. Je pris une once de ces plantes que je fis sécher & que je réduisis en cendres, comme dans la première expérience; je versai sur six grains de ces cendres un quart d'once d'eau distillée, & je décantai cette lessive. Elle verdissoit le syrop de violettes, comme dans la première expérience. Vingt gouttes d'acide marin versées dans la liqueur produisirent de même une effervescence \footnote{Nous dirons, en faveur de ceux de nos lecteurs qui ne connoîtroient pas les conséquences qu'on tire à l'ordinaire des effets de ces deux réactifs, que la conversion des couleurs bleues végétales en vert, indique la présence d'un alkali, ( qui sans doute étoit ici la potasse ); & que l'effervescence qui suit l'affusion d'un acide dans un mélange, annonce la présence d'un sel composé d'une base quelconque & d'acide carbonique ou air fixe, lequel se dégage sous forme élastique. (R.)}. Je fis deux parts de cette lessive. Dans l'une je versai de la solution de prussiate de potasse, & il parut un précipité abondant\setcounter{page}{288} de bleu de Prusse ; je versai dans l'autre portion quelques gouttes d'une forte infusion de noix de galles, & le mélange noircit un peu. Le reste de la liqueur lorsque je la saturai d'alkali caustique, fournit un précipité abondant de chaux sous la forme de flocons blancs, mais il restoit un résidu dont j'ignorois la nature. Pour la découvrir je fis croître de la même manière une plus grande quantité de ces plantes & après les avoir desséchées, brûlées, &c. je trouvai que ce résidu, étoit pour la plus grande partie, de la terre siliceuse\footnote{On fait que cette même terre forme, dans les nœuds du bambou, des concrétions qu'on nomme Tabasheer aux Indes & qu'on employe en médecine. Et dans un très-beau travail que viennent de faire les célèbre chimistes Français, Fourcroy & Vauquelin sur l'analyse des calculs de la vessie, ils ont trouvé que la terre siliceuse entroit quelquefois comme partie constituante dans ces concrétions. (R)}.
Pour découvrir ensuite si ce fer & cette terre siliceuse existoient dans la plante avant l'incinération, l'auteur traita à l'acide nitreux & à l'acide marin, tant à froid qu'à l'aide de la chaleur, une certaine quantité de ces mêmes plantes fraîches, & il ne put point en extraire de fer\footnote{Il nous semble que le fer auroit pu exister dans les plantes fraîches sans être décelé par l'action des eaux acides sus-désignées. Il suffiroit pour cela de supposer qu'il fût dans leur composition par des affinités supérieures à celles de ces acides pour les diverses parties constituantes de ces végétaux. (R)}. Il en conclut que l'incinération\setcounter{page}{289} est l'un des procédés qui, en succédant à l'influence préalable de la vitalité, peut produire le fer. Il est convaincu qu'il existe un autre procédé naturel qui décompose les végétaux & produit, dans certaines circonstances, les eaux ferrugineuses. On fait que ces eaux abondent dans les contrées où se trouvent aussi les mines de houille & où l'on rencontre en général des matières d'origine végétale en état de décomposition.
Ces expériences, comme toutes celles qui présentent des résultats extraordinaires, & celles-là surtout dans lesquelles on est acheminé à opérer sur de très petites quantités de matière & à fonder en même temps d'importantes conclusions sur les résultats, demandent à être répétées & variées. Nous nous abstenons de porter un jugement sur celles-ci; mais nous invitons les Chimistes & les amateurs à s'occuper de ces recherches; elles font en rapport immédiat avec celles qui ont procuré les savantes analyses des C. C. Fourcroy & Vauquelin & qui ont fait faire des progrès très-rapides à cette branche de la chimie si nouvelle & si curieuse, qu'on désigne sous le nom de chimie animale.
Nous faifirons cette occasion d'annoncer, quelques résultats remarquables qu'a obtenus l'actif & infatigable Humboldt, en s'occupant dernièrement à Paris d'expériences sur la constitution chimique de l'atmosphère.
\setcounter{page}{290} On fait, qu'indépendamment des mélanges accidentels, l'air atmosphérique est essentiellement composé de deux fluides très-différens, le gaz oxigene, & le gaz azote ; que le premier de ces fluides, le seul qui entretienne la vie des animaux respirans, entre pour un peu plus d'un quart dans le mélange, & que les trois autres quarts ne font pas respirables. Ces proportions varient, dans certaines limites, selon les saisons & les localités ; & les procédés eudiométriques tendent à faire reconnoître dans un moment & dans un lieu donné quelles sont les quantités respectives d'oxigene & d'azote qui composent un certain volume d'air qu'on soumet aux expériences. Nous dirons, en passant, que ce physicien plein de génie a beaucoup simplifié & perfectionné ces procédés.
Il avoit observé, ainsi que le Dr. Ingenhousz, que la terre végétale, ou le terreau ordinaire, avoit la propriété d'absorber peu-à-peu l'oxigene ; & que l'air atmosphérique, en contact avec ce terreau humide en vases clos, se réduisoit finalement à l'azote pur mêlé de quelques centiemes d'acide carbonique. Ce fait est d'une grande importance dans l'histoire de la végétation & tend en particulier à expliquer l'influence des fréquens labours sur la fertilité de la terre.
En variant les expériences, il a mis aussi en contact avec l'air atmosphérique en vases clos ;\setcounter{page}{291} chacune des cinq terres suivantes, que les chimistes nomment élémentaires parce qu'ils n'ont pas encore pu parvenir à les décomposer; savoir la chaux, la baryte, l'alumine, la magnésie, & la silice. Elles étoient abondamment humectées & remplissoient partiellement des flacons bien bouchés.
Il a observé, au bout de quelques jours, une diminution plus ou moins considérable dans l'air commun qui avoit séjourné sur les trois premières de ces terres; & les épreuves eudiométriques annonçoient que l'absorption se portoit exclusivement sur l'oxigène, qui entroit dans quelque nouvelle combinaison avec les matières terreuses. La présence de l'eau étoit une condition nécessaire au succès de l'expérience. Si ces essais, faits sur de très-petites quantités, se trouvent confirmés par les expériences que leur auteur, ou d'autres physiciens, mis sur la voie, répéteront sans doute en grand, la découverte à laquelle ils auront donné lieu fera une importante addition aux connoissances chimiques actuelles.

\setcounter{page}{148}
\chapter{Chimie}
\section{ACCOUNT OF A VIOLENT EXPLOSION, &c. Détails sur une explosion violente qui eut lieu dans un magasin de farines à Turin, le 14 décembre 1785, auquel on a joint quelques observations sur les inflammations spontanées, par le Comte MOROZZO. (Tiré du Repertory des Arts & Manufactures.)}
"L'ACADÉMIE de Turin, ayant exprimé son désir d'avoir un détail particulier de l'explosion dont je lui parlai peu de jours après cet accident, je me suis empressé de remplir son vœu, en recherchant avec la plus grande attention toutes les circonstances du fait, de manière à pouvoir le raconter avec exactitude."
"Je prendrai la liberté d'ajouter à ces détails quelques notices abrégées sur les différentes inflammations spontanées qui ont eu lieu dans diverses substances & qui ont occasionné de très-grands malheurs. Quoique ces phénomènes soient déjà, pour la plupart, connus des Physiciens, je me persuade qu'on me saura quelque gré de les avoir réunis dans ce Mémoire; car on ne saurait donner trop de notoriété à des faits qui ont des rapports si immédiats avec le bien public."
\setcounter{page}{149}
"Le 14 décembre 1785, vers six heures du soir, il y eut dans la maison de Mr. Giacomelli, boulanger de cette ville, une explosion qui jeta dans la rue les croisées de sa boutique; le bruit fut aussi fort que celui d'une piece d'artillerie & on l'entendit à une distance considérable. Au moment de l'explosion on apperçut dans la boutique, une flamme très-brillante qui ne dura que quelques secondes, & l'on observa immédiatement, que l'inflammation provenoit du magasin de farines, situé dans l'arriere-boutique, & où un enfant etoit occupé à en remuer une certaine quantité, à la lumiere d'une lampe. Cet enfant eut le visage & les bras échaudés par l'explosion; ses cheveux furent brûlés, & il se passa plus de quinze jours avant que les brûlures de la peau ne fussent guéries. Il ne fut pas la seule victime de cet accident; un autre enfant qui se trouva monté, sur une espèce d'échafaud dans une petite chambre, de l'autre côté du magasin, voyant la flamme qui s'etoit fait chemin de ce côté, & croyant que la maison etoit en feu, se précipita de son échafaud, & se cassa la jambe."
"Pour découvrir comment cet événement avait eu lieu, j'examinai avec beaucoup d'attention le magasin & ses dépendances; & c'est d'après cet examen & les détails que m'ont fourni les témoins oculaires, que j'ai tâché de rassembler toutes les circonstances dont je vais faire part."
\setcounter{page}{150}
"Le magasin de farine, qui est situé au-dessus de l'arriere-boutique est haut de six pieds, large d'autant, et long d'environ huit pieds. Il est divisé en deux parties par une muraille ; l'une et l'autre sont plafonnées en voute, et le pavé d'un côté est plus élevé d'environ deux pieds que celui de l'autre. Au milieu du mur qui forme ces deux partitions, est une ouverture de deux pieds et demi de large et de trois pieds de haut, par laquelle on fait passer la farine de la division supérieure dans l'inférieure."
"Le jeune garçon occupé dans la chambre inférieure à ramasser de la farine pour fournir au blutoir placé au-dessous, creusoit vers les côtés de l'ouverture, dans le but de faire tomber la farine de la chambre supérieure dans celle dans laquelle il était ; et tandis qu'il creusoit assez profondément, il y eut une forte d'avalanche de farine, suivie d'un nuage épais qui s'enflamma tout-à-coup à la lampe qui était suspendue à la muraille, et causa l'explosion violente dont on a parlé."
"La flamme se montra dans deux directions, elle pénétra par une petite ouverture de la chambre supérieure du magasin dans une très-petite chambre située au-dessus, et dont la porte et les fenêtres étaient fortes et bien fermées : elle ne produisit là aucune explosion : c'est-là où le pauvre garçon dont on a parlé se cassa\setcounter{page}{151} la jambe. La plus grande inflammation au contraire eut lieu dans la plus petite chambre, & prenant la direction d'un escalier étroit qui mène dans l'arrière-boutique, elle produisit là une explosion violente qui pouffa, ainsi qu'on l'a dit, toutes les fenêtres dans la rue. Le boulanger lui-même, qui se trouvoit alors dans sa boutique, vit la chambre tout en feu quelques instants avant que de sentir la secousse de l'explosion."
"Le magasin contenoit à l'époque de cet accident, environ trois cent sacs de farine."
"Je soupçonnois que cette farine avoit été mise humide dans le magasin ; & je cherchai à vérifier cette conjecture. L'examen que je fis m'apprit qu'elle étoit parfaitement sèche, & qu'il n'y avoit dans sa masse aucune apparence de fermentation, ni dégagement de chaleur sensible."
"Le boulanger me dit qu'il n'avoit jamais eu de la farine aussi sèche que celle de cette année, qui fut effectivement une année de grande sécheresse, car il ne plut pas en Piémont pendant cinq à six mois. Il attribuoit l'accident arrivé dans son magasin à la sécheresse extraordinaire du grain."
"Ce phénomène, tout frappant qu'il étoit à l'époque où il eut lieu, n'étoit pas entièrement nouveau pour le boulanger. Il m'apprit que dans sa jeunesse il avoit été témoin d'une in-\setcounter{page}{152} flammation semblable; elle avoit eu lieu dans un magasin de farine où l'on verfoit cette substance par un conduit de bois affez long dans un blutoir, tandis qu'il y avoit tout auprès, une lampe allumée. Mais dans ce cas, l'inflammation ne fut pas accompagnée d'une explosion."
"Il me fit part de plusieurs autres circonstances, que je crus devoir examiner de plus près. Dans l'un de ces cas, chez la veuve Ricciardi, boulangere de cette ville, dont le magasin de farine étoit contigu à la forge d'un ferrurier, la farine s'échauffa à un tel degré, qu'un jeune homme qui entra dans le magasin n'y put demeurer à cause de la chaleur qui brûloit ses pieds; cette farine étoit devenue d'un brun foncé, & tandis qu'on l'examinoit on commença à appercevoir des étincelles & le feu se mit au tas sans produire aucune flamme, à la maniere des pyrophores\footnote{Je désirois fort effayer s'il seroit possible d'amener la farine seule à l'état de pyrophore; mais quoique je la calcinaffe par une forte chaleur dans une petite retorte, avec les mêmes précautions qu'on employe pour faire les autres pyrophores, je ne pus jamais réussir à lui faire prendre feu par l'exposition à l'air. En y joignant l'alun, j'obtins un vrai pyrophore ainfi que l'avoit fait Lémery. (A)}."
"Il m'apprit auffi qu'une inflammation du même genre que celle dont j'ai parlé, avoit eu lieu\setcounter{page}{153} chez un boulanger de cette ville nommé Jofeph Lambert; elle fut occafionnée par l'action de fecouer auprès d'une lampe allumée, de grands facs qui avoient été remplis de farine; mais la flamme quoiqu'affez vive, ne fit de mal à perfonne \footnote{Ne feroit-ce point, dans ce cas, le nuage de farine qui auroit pris feu? Cette matiere féchée & pulvérulente peut dans certains cas s'allumer & faire un éclair à la maniere du Lycopode : l'auteur admet enfuite cette explication dans un cas different. (R)}.
"Il me paroît, d'après les détails qui précedent, qu'il n'eft pas difficile d'expliquer le phénomene en queftion. Voici l'idée que j'en ai conçue. A mefure que la farine tomboit, une grande quantité d'air inflammable qui avoit été confiné dans fes interftices, fe mettoit en liberté; cet air en s'élevant s'allumoit par le contact de la lumiere, & fe mêlant immédiatement avec une quantité fuffifante d'air atmolphérique, l'explofion eut lieu du côté où il y avoit le moins de réfiftance. Quant à la brûlure de la cheveux & de la peau du jeune homme qui étoit dans le magafin, il faut l'attribuer à l'inflammation des fines molécules de farine qui, flottant dans l'atmosphère s'allumoient par l'intermede de l'air inflammable, comme la pouffiere des étamines de certains végétaux (en particulier du fapin & de quelques mouffes) lorsqu'on la jette en l'air, s'allume si l'on en\setcounter{page}{154} approche la flamme d'une chandelle."
"Mais on peut objecter que comme la farine n'étoit pas du tout humide, & qu'elle n'avoit aucun degré sensible de chaleur, il ne pouvoit y avoir dans son intérieur aucune forte de fermentation; & qu'en conséquence il ne pouvoit pas s'en dégager de l'air inflammable: à cela je réponds."
1°. "Que la farine n'est jamais entièrement exempte d'humidité, ainsi que le prouve évidemment la distillation."
2°. "Que quoique le degré de chaleur ne fût pas assez considérable pour dégager l'air inflammable par la fermentation, une quantité suffisante de ce fluide se trouvoit libérée par, ce que je pourrois appeler un moyen mécanique, pour s'enflammer au contact de la lumière; & pour dégager, en même temps, tout celui qui communiquoit avec l'air atmosphérique \footnote{Mais comment s'étoit produit cet air inflammable qui se dégageoit ainsi mécaniquement. Nous ne voyons pas que l'eau se décompose, & donne son hydrogene, par l'intermède des matières végétales ou animales, sans une fermentation préalable, ou sans une haute température. (R)}."
3°. Il faut se rappeler que la farine donne aussi de l'air inflammable alkalin, qui est produit par sa partie végéto-animale, ou son gluten. Et nous savons que cette espèce d'air inflammable est très-susceptible d'inflammation.
\setcounter{page}{155}
"Après avoir décrit cet événement singulier je demande la permission de raffembler ici tous les faits qui ont rapport à des inflammations spontanées & qui font parvenus à ma connoissance. Les membres de l'administration doivent prendre un intérêt particulier aux détails circonstanciés de ces phénomènes, non-seulement parce que cela peut les conduire à prévenir les malheureux accidents qui en résultent, mais aussi parce que des innocents peuvent être soupçonnés d'avoir contribué à des événements qui feraient l'effet de causes naturelles, & être injustement persécutés par suite de ces soupçons."
"Je ne parlerai pas des inflammations causées par la foudre, par les feux souterrains, & par d'autres météores ; ces phénomènes ne font pas de la même nature que ceux dont je vais entretenir l'Académie ; mais je ne dois pas passer sous silence les combustions spontanées qui ont eu lieu quelquefois dans le corps humain. Quoique les accidents de ce genre soient très-rares, nous en trouvons quelques exemples dans les Transactions Philosophiques & dans les Mémoires de l'Académie de Paris & de Copenhague. On y raconte qu'une Dame Italienne, (la Comtesse Cornelia-Bandi) fut entièrement réduite en cendres les jambes seules exceptées : qu'une femme Anglaise, nommée Grace Pitt, fut presqu'entièrement consumée par une inflammation qui eut lieu dans ses entrailles;\setcounter{page}{156} & enfin, qu'un Prêtre de Bergame fut confumé de la même manière. On a attribué ces accidens à l'abus des liqueurs spiritueuses, mais les personnes qui en furent les objets n'appartenoient pas à la classe des victimes de l'intempérance, quelque nombreuse qu'elle soit d'ailleurs."
"Les inflammations spontanées des huiles essentielles, & celles de quelques huiles grasses lorsqu'on les mêle à l'acide nitreux, sont bien connues des Physiciens; on connoit aussi l'inflammation du charbon pulvérisé mêlé à froid avec cet acide; (découverte de Mr. Proust). On connoit la combustion du phosphore, du pyrophore, & de l'or fulminant. On ne trouve ordinairement ces substances que dans les laboratoires des chimistes, qui connoissent très-bien les précautions nécessaires pour prévenir les malheureux accidens qu'elles peuvent occasionner."
"L'incendie d'une frégate appartenant à l'Impératrice de Russie dans le port de Cronstadt, bâtiment sur lequel on étoit sûr qu'il n'y avoit point eu de feu allumé, montre que le noir de fumée imprégné d'huile de chenevis peut s'enflammer spontanément; ceci a été prouvé par les expériences que l'Académie de Pétersbourg fit faire à cette occasion par ordre de l'Impératrice; & quoique les Académiciens ne pussent réussir, à faire allumer du chanvre ou\setcounter{page}{157} des cordages en les imprégnant de la même huile, il est encore très-probable que le terrible incendie qui eut lieu dans le grand magasin des cordages à Pétersbourg, fut l'effet de l'inflammation spontanée de ces substances; ainsi que l'événement du même genre qui eut lieu à Rochefort en 1756."
"L'incendie d'un magasin de voiles qui eut lieu à Brest en 1757, fut causé par l'inflammation spontanée de quelques toiles cirées qui, après avoir été peintes d'un côté & mises à sécher au soleil, furent entassées encore chaudes. On prouva que cela avoit eu lieu ainsi, par des expériences subséquentes \footnote{Voy. Mém. de l'Acad. des Sc. de Paris 1760. (A). Voyez aussi ce Journal T. II, Sc. & Arts, p. 180. (R)}."
"Les végétaux, qu'on a fait bouillir dans l'huile ou la graisse, & qu'on laisse ensuite en tas, s'enflamment en plein air. Cette inflammation arrive toujours lorsque les végétaux ont conservé un certain degré d'humidité; si l'on les dessèche bien complètement au préalable, ils se réduisent en cendres sans aucune apparence de flamme. Nous devons ces observations à Mrs. Saladin & Carette." (Journal de Physique 1784.)
"Les monceaux de chiffons qu'on entasse dans les papeteries, & dont on accélère la préparation en les laissant fermenter, s'allument quelquefois si l'on n'y prend garde."
\setcounter{page}{158}
"On connoît depuis bien des siecles l'inflammation spontanée du foin; elle a souvent réduit en cendres des granges & des habitations rustiques. Elle est surtout à craindre lorsque le foin est entassé encore humide, parce que la fermentation est alors très considérable. Cet accident arrive rarement au premier foin (selon l'observation de Mr. de Bomare); mais il est beaucoup plus fréquent dans les regains entassés; & si par inattention, on laisse un morceau de fer dans une meule de foin en fermentation, elle s'enflamme presqu'infailliblement. On peut consulter à ce sujet, un excellent Mémoire de Mr. Senebier, (Journal de Physique 1781). Le blé entassé a aussi produit quelquefois des inflammations de ce genre.
Vaniere dit dans son Pradium rusticum:
Quae vero (gramina) nondum satis insolata recondens Imprudens, subitis pariunt incendia flammis.
"Le fumier aussi s'enflamme spontanément, dans certaines circonstances."
"Nous avons encore des exemples d'inflammations spontanées dans les produits du regne animal. Des pieces d'étoffe de laine qui n'avoient pas été dégraissées, ont pris feu dans un magasin. La même chose est arrivée à des tas de laine filée; on a vu des pieces d'étoffe s'enflammer tandis qu'on les charrioit au foulon; ces inflammations arrivent toujours quand les\setcounter{page}{159} matieres entassées conservent un certain degré d'humidité, qui est nécessaire pour exciter une fermentation; la chaleur qui s'en dégage dessèche les huiles & les dispose insensiblement à l'ignition, & la qualité de l'huile plus ou moins siccative peut y contribuer puissamment \footnote{Nous croirions plus volontiers que ces inflammations font l'effet de l'hydrogene dégagé de l'eau par l'action des matieres en fermentation, & élevées à une haute température par la chaleur sensible qui se dégage aussi dans ce procédé naturel. (R)}. On trouve aussi dans les substances appartenantes au regne minéral, des exemples d'inflammations spontanées. Les Pyrites entassées, si on les humecte & qu'on les expose à l'air prennent feu. La houille aussi, mise en monceaux, s'enflamme spontanément dans certaines circonstances. Mr. Duhamel raconte deux inflammations de ce genre qui ont eu lieu dans les magasins de Brest en 1751 & 1757." (Mém. acad. des Sciences.)
"Des bateaux chargés de chaux vive ont pris feu, chemin faisant, & la chaux mouillée, a souvent mis le feu à des substances qui se trouvent dans son voisinage."
"Des tournures de fer qu'on avoit laissées dans l'eau & qu'on exposa ensuite en plein air donnerent des étincelles & allumoient les combustibles voisins. C'est à Mr. Charpentier que nous devons cette observation."
\setcounter{page}{160}
"L'explofion d'un moulin à poudre en 1784, dans la manufacture royale de Turin, explofion dont on ne put point découvrir la caufe, pourroit peut-être avoir été occafionnée par l'inflammation spontanée des ingrédiens dont est compofée la poudre, ainsi que le foupçonna le comte de Saluces. Je ne nie cependant pas la poffibilité qu'il y a que cette inflammation eût été caufée par le météore auquel on l'attribua. Car il sort continuellement de ces ingrédiens une espece d'air hépatique, lorsqu'ils font humectés, & la flamme la plus légere peut allumer cette vapeur aériforme."
"Il est très-évident, d'après les faits que j'ai rapportés, que les inflammations spontanées étant très-fréquentes, & leurs causes très-variées, on ne peut mettre trop d'attention & de vigilance à prévenir leurs funestes effets. On doit donc survailler avec le plus grand soin les magasins publics, furtout ceux qui appartiennent à l'artillerie, ou ceux dans lesquels on renferme du chanvre, des cordages, du noir de fumée, de la poix, du goudron, des toiles cirées &c. substances qu'on ne doit jamais entasser, furtout si elles ont conservé de l'humidité. Pour prévenir les accidens il faudroit les examiner fréquemment, & fi l'on appercevoit qu'il s'y développât quelque chaleur, il faudroit y pourvoir sans retard. Il faut faire cet examen de jour, car il y a des inconveniens\setcounter{page}{161} à porter de la lumière dans les magafins, à raifon des vapeurs qui s'exhalent de la fermentation, & qui font souvent inflammables : l'approche d'une lampe allumée peut, dans ces cas là, occasionner une explosion subite."
"Il arrive souvent, que les fubftances en fermentation ne peuvent pas s'allumer d'elles mêmes ; mais le simple contact de la flamme suffit pour les mettre très-promptement en feu. On pourroit, en quelque sorte, faire une classe séparée, des substances dans lesquelles l'inflammation ne peut pas naitre spontanément, mais qui s'allument à l'approche de la flamme ; l'accident qui a donné lieu aux rapprochements que nous venons de faire, étoit dû à des fubftances de ce genre."
"L'ignorance des circonstances mentionnées ci-dessus, & une coupable négligence dans les précautions qu'on auroit dû prendre, ont souvent causé plus de pertes & de malheurs, que ne l'auroit pu faire la malice la plus ingénieuse. Il eft donc très-important de faire connoitre ces faits, pour que le public puiffe en tirer avantage.
Les rédacteurs de l'ouvrage périodique duquel nous avons tiré l'article qui précède, ajoûtent, qu'il faut faire une obfervation, que l'auteur paroît avoir ignorée ; favoir, que le noir de fumée qui occafionna l'incendie de la frégate Russe mentionnée dans ce mémoire,\setcounter{page}{162} étoit de nature végétale; préparé avec la fumée de sapin ou d'autres arbres résineux. Il a paru par des expériences qui ont été faites à Londres, qu'un mélange d'huile & de noir de fumée animal, tel qu'on l'employe communément dans cette ville, ne pouvoit s'allumer spontanément; tandis qu'un mélange semblable fait avec du noir de fumée végétal, venant de Russie, & qu'on s'étoit procuré pour en faire l'essai, ne manquoit jamais de s'enflammer & de brûler avec beaucoup de violence.
On pourroit aussi comparer avec ce qui précède, les détails que nous avons donnés ( T. II. Sa & Arts p. 180 ) sur une combustion spontanée qui eut lieu dans l'un des arsenaux de la Compagnie Anglaise dans l'Inde.

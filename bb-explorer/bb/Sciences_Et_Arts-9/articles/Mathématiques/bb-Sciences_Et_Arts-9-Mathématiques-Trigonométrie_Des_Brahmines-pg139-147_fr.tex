\setcounter{page}{139}
\chapter{Mathématiques}
\section{OBSERVATIONS ON THE TRIGONOMETRICAL TABLES OF THE BRAHMINS. Observations fur les tables Trigonométriques des Brahmines, Par le Dr. PLAYFAIR, Prof. de Mathématiques à l'Université d'Edimbourg. Extrait des Transactions de la Société Royale d'Edimbourg, pour 1795.}
LA Trigonométrie est une des parties des mathématiques qui a le plus contribué à leur perfection. Sans elle, l'arpenteur, réduit à se fier à l'adresse de ses mains & à la justesse de ses instrumens, ne pourroit exécuter que des opérations très-limitées. Sans elle, la géographie feroit encore remplie d'incertitudes & d'erreurs, la grandeur & la figure de notre globe seroient encore ignorées ; le navigateur, privé des secours qui justifient son audace, ne fauroit se hasarder sans témérité à perdre les côtes de vue. Sans elle, l'astronomie feroit encore au berceau : & nous ne connoîtrions ni les espaces qui séparent de nous les corps planétaires, ni leurs distances respectives. Sans elle en un mot, presque toutes les parties des mathématiques, ou seroient encore à naître, ou seroient réduites à leurs premiers élémens.\setcounter{page}{140} Nous devons donc être bien moins furpris de trouver des traces de cette partie de la géométrie dans les annales les plus reculées des peuples qui ont cultivé les sciences exactes, & même dans leurs temps fabuleux, que nous ne devons l'être de voir son origine remonter chez les Grecs seulement, à un siecle environ avant l'ère-chrétienne.
Le Dr. PLAYFAIR, un des Professeurs les plus distingués de la célèbre Université d'Edimbourg, ayant médité le bel ouvrage de l'ingénieux & infortuné auteur de l'astronomie Indienne, s'étoit convaincu de la solidité des preuves qu'il donne de l'antiquité de cette science dans l'orient, & de l'étendue qu'elle y avoit acquise; il les avoit rapprochées dans un beau mémoire lu en 1789 à la Société dont il est un des Membres les plus actifs. Ce travail lui fournit l'occafion d'étendre ses recherches sur les progrès des Indiens dans les mathématiques pures, & il a donné naissance au mémoire que nous avons sous les yeux. Nous espérons qu'il sera suivi d'autres dissertations, non moins intéressantes, & non moins propres à nous éclairer sur cette partie peu connue de l'histoire des sciences.
Le Surya siddhanta que le Dr. Davis vient de faire connoitre dans les recherches sur l'Asie, est un des livres sacrés des Indiens produits par l'inspiration divine. Il en font remonter l'origine à leur âge d'or : savoir à trois ou\setcounter{page}{141} quatre millions d’années. Parmi les fictions dont est rempli cet ouvrage remarquable, il contient un système raisonné de calculs astronomiques, & les règles de la trigonométrie; science qui paroît bien éloignée des fictions de la fable, & des charmes de l’imagination poétique.
La division de la circonférence admise par les Indiens, est la même que la nôtre, & elle est aussi celle qui étoit en usage chez les Grecs. À l’avantage qu’à le nombre 360 d’admettre un grand nombre de diviseurs, & en particulier d’être divisible par les trois nombres premiers 2, 3, 5; se joint celui que le mouvement journalier & apparent du foleil s’éloigne peu d’être une 360ᵐᵉ partie de la circonférence; peut-être même cette correspondance approchée a-t-elle donné lieu à notre division. Les Chinois, ayant plus d’égard à cette correspondance qu’à la facilité des calculs, ont divisé la circonférence en 365 ¼ parties égales dont chacune approche beaucoup de répondre au mouvement diurne. On fait que les avantages de la numération décimale, ont engagé les mathématiciens Français à l’introduire dans la division du cercle, & que déjà des tables sont construites d’après la division du quart de la circonférence en cent parties égales.
La construction des tables Indiennes les distingue d’une manière bien remarquable des tables\setcounter{page}{142} des Grecs & de celles des modernes. Les Grecs ont divisé le rayon en 60 parties égales, sans s'occuper de son rapport avec la circonférence. De là, les cordes se trouvent exprimées en 60mes. parties du rayon. Les modernes ont pris le rayon pour l'unité ou l'ont divisé en parties égales dont le nombre est exprimé par quelque puissance de 10. Dans les tables Indiennes, le rayon est exprimé en parties de la circonférence, de sorte que cette dernière étant divisée en 21600 minutes, le rayon se trouve exprimé par 3438. En partant des rapports approchés de la circonférence au diamètre, on trouve que le rayon vaut à-peu-près 3437 minutes & trois quarts, auquel nombre fractionnaire les calculateurs des tables Indiennes ont substitué le nombre entier le plus voisin 3438. Ce degré d'exactitude est le plus grand qu'on puisse obtenir, en poussant la division de la circonférence seulement jusqu'aux minutes. En particulier, il est plus grand que celui qu'on obtient par le rapport approché du diamètre à la circonférence, de 7 à 22, assigné par Archimede; & il approche beaucoup de celui de 113 à 355 assigné par Adrien Métius. On peut juger par ce rapport des progrès qu'avoient fait les Indiens dans la géométrie & dans le calcul.
Les tables Indiennes sont des tables de Sinus; elles different à cet égard d'une manière bien\setcounter{page}{143} essentielle des tables grecques qui contentaient les cordes. La substitution des Sinus aux cordes, a été jusqu'à présent attribuée aux Arabes : nous ignorons s'ils en sont les auteurs, ou s'ils n'ont fait que nous transmettre ce qu'ils tenaient des peuples orientaux.
Les tables Indiennes ne contiennent pas les tangentes & les sécantes, dont l'introduction en Europe ne remonte qu'au milieu du seizième siècle. D'un autre côté : il est remarquable qu'elles comprennent les Sinus versés, dont on ne trouve aucune mention ni chez les Grecs ni chez les Arabes. Cet avantage au reste doit être regardé comme peu important ; vu la facilité avec laquelle se calcule le Sinus versé au moyen du cosinus.
L'étendue des tables Indiennes est bien moindre que celle des tables modernes. Le plus petit arc qu'elles comprennent est celui de 3°. 45' soit 225'' qui est la vingt-quatrième partie du quart de la circonférence. Elles comprennent seulement les multiples de cet arc jusqu'à 90°. Leur étendue est donc limitée à 24 arcs qui croissent comme les nombres naturels : tandis que nos tables comprennent les Sinus des arcs au moins de minute en minute, & qu'il est aisé d'obtenir ceux des arcs intermédiaires, exprimés en secondes au moyen des tables proportionnelles.
Dans les tables Indiennes chaque Sinus est\setcounter{page}{144} exprimé, ainsi que le rayon, dans les minutes de la circonférence; savoir dans le nombre entier de minutes le plus voisin de la vraie valeur du Sinus. Cette exactitude n'est pas comparable à celle de nos tables usuelles, qui la pouffent au moins jusqu'à une dix-millionieme du rayon. Elle pourroit même donner des préventions contre l'habileté des mathématiciens Indiens; mais elles sont détruites par l'exposition des deux principes sur lesquels est fondée la construction de leurs tables. L'un de ces principes se trouve dans le Surya siddhanta lui-même; & l'autre dans un commentaire sur cet ouvrage.
Par le premier de ces principes: connoissant le sinus, & partant, le cosinus d'un arc, on connoît le sinus de sa moitié. Ce principe est précisément une des propositions fondamentales de notre trigonométrie: le double du quarré du sinus de la moitié d'un arc est égal au produit du rayon par le Sinus verse de cet arc. On fait que ce theorême est une suite immédiate du theorême de Pythagore. Par son moyen, connoissant le sinus de 30° (qui est la moitié du rayon), on connoît successivement les Sinus de 15°, 7°. 30', 3°. 45', qui est le plus petit arc des tables Indiennes. On trouve que le sinus de 3°. 45' soit 225' vaut environ 224' 44'' ; d'où il suit que cet arc & son Sinus diffèrent l'un de l'autre, environ d'un quart de minute seulement; & puisque l'exactitude des tables\setcounter{page}{145} Indiennes se borne aux minutes; cet arc & fon sinus font pris pour égaux entr'eux.
Ayant ainsi obtenu le sinus de 3°. 45'. on calcule par le second principe, les sinus de ses multiples jusqu'à 90°. La regle à suivre pour cela est énoncée dans l'ouvrage Indien d'une manière particulière, que notre auteur généralise comme il suit. Les sinus de deux arcs successifs étant connus: divisez le plus grand d'entr'eux par 225'; prenez le quotient entier le plus voisin du vrai (si ce quotient exact n'est pas entier). Soustrayez ce quotient approché de la différence entre les deux sinus. Le reste est la différence entre le premier sinus & celui de l'arc qui suit immédiatement. La démonstration de cette regle n'est pas donnée dans la partie de l'ouvrage Indien que le Dr. Davis fait connaître. Peut-être l'auteur, conservant le ton de l'inspiration, croit-il devoir exercer la foi de ses disciples, plutôt que les éclairer.
Le Dr. Playfair montre son accord avec le théorème suivant, bien connu des mathématiciens modernes : soient 3 arcs en proportion arithmétique : Le sinus de l'arc moyen, est à la somme des sinus des arcs extrêmes, comme le sinus de la différence de deux arcs qui se suivent, est au sinus du double de cette différence (qui est la différence des arcs extrêmes). Comme dans les tables Indiennes, le sinus de la différence de deux arcs successifs est 225',\setcounter{page}{146} & que le finus du double de cette différence est 449'; on a la proportion 225, est à 449, comme le finus de l'arc moyen est à la somme des finus des extrêmes. Delà : le finus du troisieme arc, est l'excès des ⁴⁴⁹/₂₂₅ du finus du second arc, sur le finus du premier.
Par ce théorème, le calcul des finus est ramené à la doctrine des suites récurrentes dont les modernes se sont occupés avec tant de frui & de profondeur ; & dont on ne s'attendoit pas à trouver une trace dans les fastes de la plus haute antiquité. Cette propriété des tables des finus, si propre à en faciliter la construction, n'a été remarquée par les mathématiciens d'Europe que depuis deux cens ans environ. Cependant il est aisé de la ramener à la 97ᵐᵉ. prop. des Données d'EUCLIDE (1) ( ouvrage trop peu connu des mathématiciens du Continent ). Ptolémée qui vivoit environ 400 ans après Euclide, & qui sans doute connoissoit cet ouvrage, n'a pas vu l'application de cette proposition au calcul des tables des cordes ; non plus que la multitude des mathématiciens qui l'ont suivi. Lorsque Viéte, à la fin du 16ᵐᵉ. siecle, fit connoître ce théorème, il passa pour en être l'auteur ; honneur que le Dr. Playfair attribueroit plutôt à son ingénieux compatriote Anderson d'Aberdeen.
Ce qui vient d'être dit fur les connoissances Trigonometriques des Indiens, & en particulier\setcounter{page}{147} sur les principes de la construction de leurs Tables, suffiroit pour faire juger des progrès qu'ils avoient fait dans les sciences exactes; & cela en remontant à la plus haute antiquité. Leurs connaissances astronomiques (que ce n'est pas ici le lieu de rappeler), prouvent qu'on ne saurait rapprocher la date du Surya siddhanta, de moins que 2000 ans avant l'ère Chrétienne. L'application heureuse de l'arithmétique à la géométrie, & la réduction en Tables des rapports des côtés des triangles d'après la connaissance de leurs angles, supposent dans l'une & l'autre de ces sciences un degré de perfection qui ne peut être que le fruit de leur culture pendant plusieurs siècles antérieurs. On peut donc, avec l'ingénieux & infortuné Bailly, faire remonter au moins à trois mille ans avant J. C. l'époque où les sciences mathématiques & en particulier l'astronomie, (dont la trigonométrie est une base indispensable), étaient en vigueur chez les Indiens; tandis que les observations astronomiques des Grecs, les plus anciennes, ne remontent guère au-delà de 1200 ans avant l'ère Chrétienne.
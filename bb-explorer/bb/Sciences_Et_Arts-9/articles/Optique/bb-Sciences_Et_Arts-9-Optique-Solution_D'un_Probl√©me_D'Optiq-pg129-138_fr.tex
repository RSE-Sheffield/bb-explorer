\setcounter{page}{129}
\chapter{OPTIQUE.}
\section{An OPTICAL PROBLEM, PROPOSED BY Mr. HOPKINSON &c. Problème d'optique proposé par Mr. HOPKINSON, & résolu par Mr. RITTENHOUSE. (Tiré des Transactions de la Société Américaine de Philadelphie T. II.)}
Nous avons occupé dernièrement nos lecteurs, des découvertes du Dr. Blair fur les propriétés de divers milieux transparents pour changer la direction de la lumière, & pour séparer les rayons diversement colorés. Cet auteur n'a touché qu'en passant à une faculté, qu'ont les corps solides, de fléchir & de décomposer aussi la lumière lorsqu'elle passe très-près de leur surface. Nous trouvons dans les Transactions Américaines un article assez curieux sur ce sujet : c'est la solution d'un problème d'optique suggéré par une expérience que chacun peut répéter & varier à volonté. Cette solution repose sur la propriété dont nous venons de parler, sur l'inflexion de la lumière ; & Mr. RITTENHOUSE, à qui l'on doit les développemens qu'on va lire, pouvait aller de pair, par l'étendue & la profondeur de ses connaissances, avec les premiers savants d'Europe: la mort, qui l'a enlevé\setcounter{page}{130} il n'y a pas long-temps à l'Amérique, à cause à cette contrée une perte qui n'y fera peut-être pas reparée dans le cours d'un siecle.
\subsection{LETTRE DE MR. HOPKINSON À MR. RITTENHOUSE.}
Mon cher Mr.
Je prends la liberté de vous demander de vouloir bien donner votre attention au problème d'optique suivant. Je le crois entièrement nouveau; sa solution pourra vous intéresser, & elle s'instruira sûrement.
J'étois sur la porte de ma maison, dans une soirée de l'été dernier. Je sortis de ma poche un mouchoir de soie, & après en avoir étendu une partie entre mes deux mains, je le tins verticalement devant mes yeux, en regardant, au travers du tissu, l'une des lampes de la rue, à la distance d'environ 100 verges. Je m'attendais à voir les fils de la trame du mouchoir fort grossis en apparence. Effectivement ils me parurent tels, & me sembloient être autant de gros fils d'archal; mais je fus très-surpris de trouver que, lors même que je faisois mouvoir le mouchoir à droite & à gauche, les petits barreaux noirs que représentoient ces fils ne se mouvoient point, mais demeuroient parfaitement fixes devant mes yeux. Si ces barreaux étoient l'effet de l'interposition des fils grossis;\setcounter{page}{131} entre l'œil et la lumière de la lampe, il semble qu'ils auraient dû se mouvoir et se succéder l'un à l'autre à mesure que les fils eux-mêmes passaient devant l'œil, mais il en arrivait tout autrement.
L'explication de ce phénomène surpasse mes connaissances en optique. Veuillez répéter l'essai, et s'il vous réussit, comme je n'en doute point, vous m'obligerez fort en me donnant une solution de ce problème, fondée sur des principes physiques.
J'ai l'honneur d'être, &c.
\section{RÉPONSE DE Mr. RITTENHOUSE}
Mon cher Mr.
L'expérience dont vous parlez, faite avec un mouchoir de soie et la flamme d'une lampe éloignée, est beaucoup plus curieuse qu'on ne l'imaginerait d'abord. Car l'objet que nous voyons n'est point le tissu du mouchoir, grossi en apparence, mais c'est un objet très-différent, ainsi qu'on peut inférer des considérations suivantes.
1°. Des rayons parallèles ou à-peu-près, tels que ceux qui viennent d'un objet distant, ne peuvent former l'image distincte d'un objet placé fort près de l'œil; car tous ces rayons, passant par l'ouverture de la pupille se rassembleront au fond de l'œil, et là, forment l'image du point lumineux. Les fils du mouchoir\setcounter{page}{132} n'ont d'autre effet que celui d'intercepter une partie des rayons, & de rendre l'image moins brillante.
2°. Si ces barreaux croisés que nous voyons étoient les images des fils de soye, ils devroient passer sur la rétine à mesure que les fils eux-mêmes passent devant la pupille ; mais cela n'arrive point, ainsi que vous l'avez observé ; car ils demeurent stationnaires.
3°. Si l'image sur la rétine étoit une peinture de l'objet placé au-devant de l'œil, cette image seroit plus ou moins fine ou grossière, à raison des changemens du tissu lui-même ; mais ni lorsqu'on employe un tissu plus ou moins grossier, ni lorsqu'on l'éloigne plus ou moins de l'œil, on ne produit aucun changement dans les apparences ; & le nombre des fils paroît le même soit qu'il en passe 10, 20, ou 30 devant la pupille dans le même temps. Il faut donc que l'image que nous voyons soit formée d'une manière très-différente ; elle doit être l'effet de cette inflexion qu'éprouve la lumière en passant près des surfaces des corps, & dont nous devons la découverte à Newton.
On démontre fort bien en optique que les différens points d'un objet placé à une certaine distance de l'œil, envoyent à l'organe des rayons disposés en forme de pinceaux, inclinés les uns aux autres sous des angles sensibles, & qui rendus ensuite convergens par la force\setcounter{page}{133} réfringente de l'organe se réunissent au fond sur la rétine & y forment l'image de l'objet.
L'effet des lunettes d'approche consiste à augmenter ces angles régulièrement dans un certain rapport, & à faire en sorte que les rayons qui étoient parallèles avant d'entrer dans la lunette, le soient encore après l'avoir traversée.
L'image agrandie que nous voyons dans cette expérience, doit donc être produite par des pinceaux de rayons qui, avant d'entrer dans l'œil, étoient considérablement inclinés l'un à l'autre. Mais comme ces rayons venoient de la flamme d'une petite lampe placée à une distance assez considérable, ils étoient à peu-près parallèles en arrivant sur le mouchoir de soye; c'étoient donc les fils de ce mouchoir qui leur donnoient des directions si différentes.
Avant que le tissu soit placé devant l'œil, les rayons de lumière parallèles formeront une seule tache brillante, comme en A, dans la figure ci-jointe; & une tache semblable sera \comment{figure not included in transcript} produite par ceux d'entre les rayons qui, passant par les intervalles du tissu, n'éprouveront point d'action de la part des fils qui le composent.\setcounter{page}{134} Mais supposons que les fils verticaux de ce tiffu, par leur influence fur les rayons en flechiffient quelques-uns, d'un degré vers la droite & vers la gauche, & en flechiffient d'autres de deux degrés, il y aura alors quatre nouvelles images formées, favoir, deux de chaque côté de l'image primitive en A. Par une influence femblable de la part des fils horifontaux, cette ligne, compofée de cinq points lumineux fera partagée en cinq autres lignes, favoir, deux au-deffus & deux au-deffous, ce qui formera un quarré de 25 taches brillantes féparées par quatre lignes obfcures verticales & autant d'horifontales. Ces taches brillantes & ces lignes obfcures ne changeront point de place lorfqu'on fera mouvoir le tiffu devant l'œil, dans une direction parallele à l'un de ces fils. Car le point de la rétine où tombera l'image, eft déterminé par l'incidence des rayons relativement à l'axe de la vifion, avant leur entrée dans l'œil, & non par la partie de la pupille qu'ils traversent.
Pour faire fur ce fujet des expériences plus exactes, je préparai un quarré d'environ poucè de côté, garni de cheveux tendus paralléllement les uns aux autres. Pour obtenir ce parallélifme je fis tarauder par un horloger deux fils de métal en vis à pas très-fins; il y en avoit 106 dans un poucè. Je mettois 50 à 60 cheveux dans chaque pas de la vis. Je regardois au travers\setcounter{page}{135} de cet appareil, une petite ouverture de \frac{1}{10} de pouce de large & de 3 pouces de long, faite au volet d'une chambre obscure, en tenant les cheveux parallèlement à l'ouverture : je voyois trois lignes parallèles presqu'également lumineuses, & de chaque côté, quatre ou cinq autres lignes de plus en plus foibles, colorées, & indistinctes, à mesure qu'elles s'éloignoient d'avantage de la ligne du milieu, que je faisois être formée par les rayons qui passoient entre les cheveux fans éprouver leur influence. Ne trouvant pas mon appareil aussi parfait que je l'aurois desiré, j'enlevai les cheveux & j'en substituai d'autres un peu plus gros ; 190 faisoient un pouce, & par conséquent les intervalles entr'eux avoient environ \frac{1}{190} de pouce de largeur. Les trois lignes lumineuses du milieu, n'étoient plus aussi brillantes qu'auparavant, mais les autres étoient plus fortes & plus distinctes, & j'en pouvois compter fix de chaque côté de la ligne centrale, également distantes les unes des autres, en estimant ces distances depuis le centre de chacune au centre de sa voisine. La ligne du milieu étoit toujours bien terminée & sans couleur ; les deux suivantes étoient encore assez bien terminées mais un peu plus larges ; leurs bords intérieurs étoient colorés en bleu & les bords extérieurs en rouge. Les autres étoient moins distinctes & montraient chacune & dans le même ordre, les couleurs\setcounter{page}{136} prismatiques, qui en s'étendant de plus en plus, paraissaient se toucher réciproquement vers la cinquième ou sixième ligne; mais celles qui étaient plus voisines du centre, étaient séparées les unes des autres par des lignes très-obscures, beaucoup plus larges que ne l'étaient les lignes brillantes.
Trouvant que le rayon de lumière qui venait du volet était ainsi divisé en un si grand nombre de pinceaux distincts, je voulus essayer de déterminer les angles que ces pinceaux faisaient entre eux. J'employai dans ce but, une petite lunette prismatique & un micromètre, que me prêta le Dr. Franklin. Je fixai mon cadre de cheveux parallèles, devant l'objectif, de manière à ce qu'il en occupât toute l'ouverture. Ensuite, en regardant dans la lunette, je mesurai l'espace entre les deux premières lignes de chaque côté, & je trouvai que la distance angulaire entre leurs bords intérieurs était de 13', 15". Et du milieu de l'une au milieu de l'autre, 35', 30"; enfin l'angle entre leurs bords extérieurs était de 17', 45". Dans le premier cas, j'apercevois une trace légère colorée en bleu au milieu de l'objet, & dans le dernier, une ligne rouge. Les autres lignes étaient trop faibles, même vues au travers de la lunette, pour qu'on pût mesurer avec précision les angles qu'elles sous-tendaient; mais d'après quelques essais, il me parut, qu'en comptant de la seconde\setcounter{page}{137} ligne d'un côté, à la seconde du côté opposé, & ainsi de suite, ces angles étaient doubles, triples, quadruples, &c. des premiers.
On peut conclure de ces apparences, qu'une portion très-considérable du faisceau de rayons, passait entre les cheveux sans éprouver de leur part aucune influence : qu'une autre portion moins considérable était fléchie d'environ 7', 45'' de part & d'autre ; les rayons rouges un peu plus, & les rayons bleus un peu moins. Une autre portion encore moindre, se fléchissait de 15', 40'' ; une autre, de 23', 15'' ; & ainsi de suite ; mais aucune portion sensible de lumière ne se fléchissait sous un angle moindre de 6' ; ni aucun rayon de couleur particulière ne se dirigeait sous un angle intermédiaire entre ceux produits par le doublement, le triplement, &c. de l'angle de flexion qui avait lieu dans les premières lignes latérales.
J'observai avec surprise, que les rayons rouges fussent le plus détournés de leur direction primitive, & que les rayons bleus éprouvassent la moindre inflexion. Ce phénomène est contraire à ce qui a lieu dans la réfraction qu'éprouvent les rayons en passant obliquement d'un milieu dans un autre. Mais mon observation s'accorde avec celles qu'a faites Sir Isaac Newton, sur les franges qui bordent les ombres des cheveux & d'autres corps solides. Voici ses expressions. "Et, par conséquent, le cheveu, en produisant\setcounter{page}{138} ces franges agissoit de même & sur les rayons rouges ou les moins réfrangibles, à une plus grande distance; & sur les rayons violets, ou les plus réfrangibles, à une distance moindre: & par ces influences il dispersoit le rayon rouge en franges plus larges & le rayon violet en franges plus étroites".
En suivant ces expériences, il est probable qu'on feroit des découvertes intéressantes sur les propriétés de cette substance si étonnante, la lumière, qui anime toute la nature dans les yeux de l'homme, & qui le dispose peut-être plus qu'aucun autre bienfait à reconnoître la bonté de l'Etre qui l'a créé. Mais le défaut du loisir nécessaire me force à quitter le sujet.
Je suis mon cher Mr. &c.
D. RITTENHOUSE.
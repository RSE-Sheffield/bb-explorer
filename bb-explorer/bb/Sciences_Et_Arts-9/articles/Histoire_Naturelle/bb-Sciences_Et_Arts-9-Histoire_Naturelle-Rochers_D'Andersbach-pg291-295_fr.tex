\setcounter{page}{291}
\chapter{Histoire Naturelle}
\section{DESCRIPTION OF THE ROCKS, &c. Description des rochers d'Andersbach en Bohème.}
On découvre de très-loin cette forêt de rochers: on diroit une armée de Géants rangés en bataille.\setcounter{page}{292} dans une immense plaine. A mesure qu'on approche, leurs dimensions apparentes augmentent, & peu-à-peu les groupes deviennent innombrables. Chacun de ces piliers est isolé, & on en voit qui ont de cent à cent-cinquante, & jusqu'à deux cents pieds de hauteur. Ils sont de forme plus ou moins conique & si rapprochés les uns des autres qu'un homme peut à peine passer dans l'entre deux. Ils occupent un espace de trois milles & demi d'Allemagne en circonférence, & forment un labyrinthe dont il seroit difficile de se tirer sans guide.
La substance de ces rochers est un grès siliceux très-tendre & presque friable lorsqu'il est imbibé d'eau. C'est une sorte de pierre à filtrer imparfaite. Le rocher attire puissamment l'humidité de l'air ; les pluies s'insinuent dans son tissu, il imbibe les rosées & les brouillards, & lorsqu'il est ensuite frappé des rayons du soleil l'eau suinte de toutes parts & emporte avec elle des particules de la substance du rocher. De là vient que les sentiers tortueux qui serpentent autour de ces masses isolées sont pour la plupart occupés par un ruisseau dont l'eau est très-limpide, parce qu'elle découle comme d'un filtre, & que le sable qu'elle charrie se dépose immédiatement sous la forme d'un gros gravier. On voit dans quelques endroits l'eau sortir du pied des cones, & on peut distinguer les filets de\setcounter{page}{293} fable qu'elle charrie. Quand ces petites fontaines s'ouvrent sous le niveau du ruisseau principal, l'effet est encore plus distinct et plus frappant.
Il est très probable que cette colonnade de rochers formait autrefois le noyau d'une montagne que les pluies et les torrents ont délavée ; le rocher mis à nu et exposé à l'action combinée de l'eau et de la gelée a pris un tissu tendre et spongieux. Cette dégradation s'opère sous les yeux mêmes de l'observateur. Les eaux charrient continuellement vers la base le sable qu'elles détachent du sommet. De-là résulte dans le niveau des sentiers, un exhaussement, très sensible au bout d'une dixaine d'années. Partout où l'eau ne tombe pas verticalement depuis le sommet des cones, mais coule le long d'un plan incliné, elle dépose des sédiments si copieux que les curieux entrent jusqu'à mi-jambe dans ces sables accumulés.
Les eaux ont attaqué par la base quelques-uns de ces cones de telle manière, que leur masse entière, toute énorme qu'elle est, demeure en équilibre sur un pivot qui a, tout au plus, un pied cube de solidité. Ces cones ont des indices marqués d'une origine commune. Toutes leurs couches sont continues, ou parallèles entr'elles et à l'horizon.
Lorsqu'on pénètre dans ce labyrinthe on y trouve les sites les plus variés et des traces très pittoresques.\setcounter{page}{294} toresques de la violence des torrens qui ont précipité des arbres & des rochers dans les abymes profonds dans lesquels on les voit encore ensevelis.
Les rochers d'Andersbach présentent pour ainsi dire le squelette d'une montagne. Lorsqu'on pénètre bien avant dans leur intérieur on arrive à la partie de la montagne qui est encore garnie de terre couverte de forêts, & dont la destruction commence d'une manière insensible. Cette circonstance met le comble à la singularité du spectacle. On peut voir dans le même lieu tous les degrés successifs de dégradation dont une montagne peut être susceptible.
Dans cette grotte profonde qui n'a ni toit ni voute qui la recouvre, la température est cependant à-peu-près uniforme & ne varie que peu avec les saisons. On y éprouve en été la fraîcheur la plus agréable & on s'y apperçoit à peine du froid de l'hiver.
Il y a vers les confins de ce groupe gigantesque un écho remarquable. Il répete sept syllabes jusqu'à trois fois sans confondre les sons. Le centre phonique est à une petite distance des côtés du grand cone dans lequel est le principal foyer des sons réfléchis. Les mots prononcés à voix basse sont répétés distinctement à la distance requise; mais lorsqu'on s'avance ou qu'on recule de quelque pas, la voix la plus forte ne produit aucun écho. Nous en fîmes l'essai par des mots articulés plus ou moins fortement, & par des coups de pistolet.
La nature offre de singulières variétés jusques dans ses phénomènes les plus simples. L'écho est, sans contredit, du nombre de ceux-ci; cependant, sans parler des différences relatives au nombre de syllabes qu'ils répètent,\setcounter{page}{295} à leur articulation plus ou moins distincte, & aux intervalles de retour plus ou moins prolongés, nous observâmes de plus, dans tous les essais que nous fimes avec nos voix dans ces montagnes, que ces sons réfléchis acquéroient un caractere ou timbre particulier.
J'ai vérifié depuis, par des observations plus particulieres, qu'aucun écho ne répete le son précisément tel qu'il a été articulé. L'écho d'Andersbach est prompt, & sec dans sa réflexion, parce que les surfaces qui répercutent sont dégarnies de terre & de végétaux. C'est un violon dont le corps feroit d'acier.
L'écho de Kinaft transformoit le bruit d'un coup de pistolet en longs roulemens interrompus par des éclats & des intonations variées. Sa réflexion est en partie produite par des rochers nus & en partie par des forêts. Mais rien n'égale l'effet touchant & mélancolique de l'écho de Zobtenberg dont les réflexions sont adoucies par un bois épais situé à une assez grande distance.

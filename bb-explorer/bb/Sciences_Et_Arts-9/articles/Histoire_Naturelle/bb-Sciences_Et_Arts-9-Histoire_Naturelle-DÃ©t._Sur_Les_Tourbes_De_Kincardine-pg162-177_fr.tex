\setcounter{page}{162}
\chapter{Histoire Naturelle}
\section{AN ACCOUNT OF THE PEAT-MOSSES, &c. Détails sur les tourbes de Kincardine & Flanders, en Perthshire. Par W. Christophe Tait, Pasteûr de Kincardine. (Transactions de la Société d'Edimbourg).}
LES tourbes de Kincardine & Flanders sont situées dans la grande plaine de Carse qui commence à Borrowstounness au midi du Frith de\setcounter{page}{163} Forth, et un peu au-dessus de Kincardine de l'ouest, du côté du nord. Cette plaine s'étend sur les deux bords du Frith, et ensuite de la rivière Forth jusqu'à Cardroff, à vingt-deux milles à l'ouest du point où elle commence. La largeur de cette plaine, vis-à-vis de Falkirk, où elle est la plus considérable, est d'environ sept milles, en y comprenant l'espace occupé par le Frith. A Stirling, la plaine n'a qu'un quart de mille, et sa largeur moyenne, depuis là jusqu'à Cardroff, est de trois milles. Le sol est une glaise bleue dont on n'a jamais sondé la profondeur. Un banc de gravier, cependant, se montre près de la surface et occupe l'espace d'un mille, entre Blair Drummond et Ochtertyre, en s'enfonçant du côté du Forth, à raison d'un pied sur cent. Presque toute cette étendue est une plaine unie : il n'y a de hauteurs que celles d'Airth, de Dunmore, de Craigforth, et de Dript, qui sont peu considérables. On ne voit de roc que sur ces éminences, et dans deux endroits de la rivière, savoir à Dript, et à Craigforth ; ce dernier forme une obstruction considérable dans le courant, qui empêche la marée de remonter plus haut. Dans tout le reste de la plaine, on ne voit pas une seule pierre dans le sol. Des bas-fonds de coquilles, surtout des huîtres, se montrent dans plusieurs endroits, comme dans des fossés qu'on a creusés à une certaine profondeur, et sur les bords du Forth\setcounter{page}{164} ou de ses branches. On voit un lit très épais de ces coquilles auprès du pont de la Goody, petite rivière qui coule dans le Forth. On en voit encore un autre au Sud de la route de Polmouth à Borrowstouness. Lorsque le Forth ronge ses bords, il découvre des troncs d'arbres très gros, enterrés à diverses profondeurs dans la glaise.
Quant à la rivière, il faut remarquer que la marée remonte jusqu'au banc de rochers de Craigforth, à 300 verges au-dessous de la réunion du Teith & du Forth. Au-dessus de ce point, la surface de la rivière est de quatre pieds & demi plus haut qu'elle n'est au-dessous, même dans les hautes marées du printemps. L'autre chute de la rivière, jusqu'à Cardroff, se trouve à Frew, à huit milles environ de la première: elle est de trois pieds.
Pour donner une idée du niveau presque horizontal de cet espace de terrain, je dirai que dans une reconnaissance que l'on a faite de la rivière, avec le projet de la rendre navigable, il a été reconnu qu'une digue de quatre pieds, élevée à la pointe de Craigforth accroîtrait la profondeur de la rivière de plus de trois pieds jusqu'au bacq de Frew; & qu'une digue de cinq pieds, élevée à Frew, augmenterait aussi la profondeur de trois pieds, au moins, jusqu'à Cardroff. Donc, la différence de hauteur de la surface de l'eau: à Cardroff, & des hauteur\setcounter{page}{165} tes marées à Craigforth, est de moins de dix pieds, & cela dans une distance de 40 milles, si l'on suit les sinuosités de la riviere, ou de 18 milles en ligne directe. La surface de la riviere est d'environ 21 pieds au-dessous du niveau de la glaise de part & d'autre, & ce- pendant le pays est souvent recouvert par les eaux, dans les inondations, jusqu'à une grande distance.
Une partie du pays est couverte de marais tourbeux, ou plaines de tourbe. La premiere de ces plaines du côté de l'est est celle de Kincardine, qui occupe l'angle entre le Forth & le Teith, & s'étend à l'ouest jusqu'à Burnbank. La plaine est ensuite exempt de tourbes dans un espace de deux milles & demi. Le marais tourbeux de Flanders vient ensuite, & regne jusqu'à Cardroff, en occupant des deux côtés du Forth une grande partie de la plaine. Le marais de Kincardine occupoit', il y a 25 ans, 1800 acres : les opérations dont je parlerai tout-à-l'heure l'ont réduit à 1500.
Ces deux espaces tourbeux font précisément de la même nature. Les marais de Froth, Dunmore, & Kinnaird leur ressemblent également. La longueur de tous ces espaces tour- beux pris ensemble, depuis Cardroff à Kin- naird, en déduisant les intervalles non tour- beux, est de 15 milles, & j'estime leur surface totale à 9000 acres. La plus grande épaisseur\setcounter{page}{166} de la tourbe au-dessus de la glaise est de quatorze pieds & demi.
A une certaine distance, toute la surface des espaces tourbeux semble couverte de bruyères, mais dans le fait il n'y en a que quelques buissons épars & séparés par des plantes qui croissent naturellement dans la mousse. Il y a aussi de place en place des espaces marécageux, où l'on enfonce, & où il ne croit pas un brin d'herbe.
Lorsqu'on creuse dans la tourbe, on trouve les débris des mêmes plantes qui croissent à la surface ; lesquels débris sont plus ou moins pourris, selon la profondeur, l'humidité, & la pression qu'ils ont éprouvée. Au fond de la tourbe, & à la surface de la glaise sur laquelle elle repose, on trouve une couche de morceaux de bois pourri, & de temps en temps un peu de terre & un peu de bruyère : celle-ci est beaucoup mieux conservée que celle qu'on trouve dans les couches supérieures de la tourbe. On voit également sur la surface de la glaise, ou au fond de la tourbe, un nombre infini de troncs d'arbres renversés auprès de leurs racines, lesquelles sont encore dans la glaise & situées comme elles l'étaient pendant la végétation. Les racines des buissons de bruyère sont également fixées dans la glaise, & semblent avoir végété autrefois dans cette terre, avant que la tourbe la recouvrit.
\setcounter{page}{167} Dans le marais de Kincardine, on voit un espace considérable que l'on nomme tourbe fluide. (flow-moff.) La surface de cet espace est unie, & avant un défféchement qui a eu lieu récemment, cette partie étoit fi faturée d'eau, à caufe de la fituation élevée du terrain environnant, qu'en effet la tourbe y étoit presque fluide. Dans les autres espaces, la tourbe a la confistance nécessaire pour pouvoir être coupée en morceaux; du moins vers le fond de la couche, car dans les parties supérieures de la masse tourbeuse, la putréfaction n'est pas suffisamment avancée, ou la compression n'est pas assez grande pour que l'on puisse la couper en morceaux solides.
On s'y est pris de diverses manières pour rendre ces marais productifs. Dans quelques endroits, après avoir enlevé la tourbe pour la brûler, on a réduit en cendres les débris pour la mêler ensuite à la glaise inférieure, avec la charrue. Dans les endroits où la tourbe a une grande épaisseur, & où ce mode d'amélioration ne pouvoit pas avoir lieu, on a brûlé la surface pour labourer ensuite dans la tourbe même, ou bien l'on a rapporté des terres fur la surface de la tourbe pour mêler ensemble les deux substances à la charrue. Les progrès que l'on a faits avec ces méthodes, pour la culture de la tourbe, n'ont, au reste, pas été bien considérables; & l'usage de se débarrasser\setcounter{page}{168} par les eaux de toute la masse de la tourbe, à quelques pouces près au-dessus de la glaise, a généralement prévalu aujourd'hui. Le sol qu'on a ainsi débarrassé de la tourbe qui le recouvroit, passe pour très-bon, & s'afferme aisément à raison de 15 shillings l'acre. Cette opération de flotter la tourbe est facile, parce que la glaise est de niveau avec le pays environnant, & que la masse de tourbe a été superposée à la glaise.
On dit que dès le commencement de ce siècle, on s'est servi de cette méthode entre Forth & Carron pour rendre ces terrains à la culture. On calcule qu'environ 600 acres y ont été ainsi débarrassés de la tourbe, au moyen de l'eau rassemblée dans l'enceinte du marais seulement, & sans le secours d'aucun ruisseau tiré des hauteurs voisines. On avoit fait des essais partiels sur les marais de Kincardine & Flanders, mais sans les rapporter à un plan général; mais en 1770 Lord Kames propriétaire de 1500 acres dans le marais de Kincardine & d'une grande partie de celui de Flanders, adopta cette méthode, & la perfectionna beaucoup. Aujourd'hui elle est généralement suivie: en voici le détail.
Sur le bord de la masse tourbeuse dont on veut se débarrasser, on creuse un fossé de deux pieds de profond, & de dix huit pouces de large. On y fait entrer un ruisseau d'un pied\setcounter{page}{169} de profondeur environ. L'ouvrier enlève ensuite avec une pelle de bois, une bande de six pieds de large, & d'un pied de haut, le long du fossé, & la jette dans l'eau, par morceaux, à mesure qu'il travaille. Si la pente est suffisante, l'eau en emmène autant que six hommes peuvent en couper. Lorsque l'ouvrier est arrivé à l'extrémité de la bande qu'il veut enlever, il reprend une seconde couche d'un pied, & ainsi de suite, jusqu'à ce qu'il ne reste que six pouces de tourbe sur la glaise. Cette couche de six pouces séchée au soleil, s'enlève, se brûle, s'enterre à la charrue, & le terrain se trouve prêt pour une récolte d'avoine.
Au fond de la masse de tourbe, ainsi enlevée, on trouve un nombre infini de troncs d'arbres & de racines qui ne laissent aucun doute que ces lieux n'aient été autrefois occupés par des forêts. Il n'est pas rare, je le sais, de trouver des arbres dans la tourbe, mais il n'est pas commun de les trouver en si grande abondance. Il y sont aussi nombreux que l'on peut supposer possible qu'ils aient été nourris par le sol; & ce qui est bien remarquable, les racines se trouvent fixées dans la glaise, avec la direction naturelle qu'elles avoient pendant l'époque de la végétation, & sont en nombre, en force, & en espèce proportionnés aux troncs d'arbres couchés tout auprès.
Les arbres sont des chênes, des bouleaux,\setcounter{page}{170} des noisetiers, des aulnes, & des faules. Dans quelques endroits il y a des sapins.
Le chêne eft le bois le plus abondant, furtout dans la partie occidentale des marais. On y a trouvé dernièrement quarante gros chênes couchés à côté de leurs racines, aufli près les uns des autres que l'on peut fuppofer qu'ils aient pu croître. Un de ces chênes avoit 50 pieds de long & plus de trois pieds de diametre : on compta dans une des fouches trois cent quatorze cercles concentriques, c'eft-à-dire autant d'années de croiffance. Dans une autre partie du marais, on a trouvé un tronc de chêne de quatre pieds de diametre; & l'on m'a affuré que l'on découvrit à Roff', il y a quelques années, dans la partie méridionale du marais, une racine qui avoit 15 pieds de diametre à la furface de la glaife. L'arbre de cette racine avoit 22 pieds de long & 4 pieds huit pouces de diametre à la partie où il avoit été coupé, c'eft-à-dire à 3 pieds du fol.
Le chêne eft noir, & encore fain, furtout dans la partie qui touche à la glaife. Il eft propre à beaucoup de chofes, & feroit probablement plus utile encore fi les gens entre les mains defquels il tombe avoient la précaution de le faire fécher convenablement. Comme on le féche trop brufquement, il fe fend, de maniere qu'on n'en peut pas faire des planches.
Les racines tiennent à la glaife dans leur\setcounter{page}{171} position naturelle, & les fouches dépassent la terre d'environ trois pieds. Elles sont très-peu pourries, & il faut beaucoup de travail pour les enlever.
Les autres arbres sont en général plus détruits : il est plus difficile de faire, à leur égard, des observations satisfaisantes. Leurs racines sont également fixées en terre, mais les fouches ne dépassent la glaise que d'un pied & demi, à-peu-près.
Les faits que je viens d'exposer peuvent indiquer eux-mêmes la cause qui a produit la disposition régulière des arbres & des racines, ainsi que le temps où cette disposition s'est réalisée. Car, d'abord, ces faits ne sauraient s'accorder avec la supposition que les arbres soient tombés de vétusté: parce que dans ce cas ils se trouveraient rompus à diverses hauteurs; & les troncs, comme les racines, auraient eu le temps de se pourrir avant que la tourbe eût pu les recouvrir : on ne les trouverait pas sains aujourd'hui.
Les faits ne sauraient non plus se concilier avec la supposition que les vents ont renversé ces forêts, car dans ce cas, les arbres auraient été rompus à différentes hauteurs, & plusieurs d'entr'eux auraient été arrachés dans leur chute. Je n'ai pas vu une seule exception qui pût conduire à cette hypothèse. On dit néanmoins que dans quelques endroits des marais, on a trouvé\setcounter{page}{172} des racines arrachées; mais ce qu'il y a de singulier c'est qu'alors on n'y a point trouvé le tronc adhérent aux racines.
On ne peut pas observer à l'appui de la supposition de l'effet des vents, que les arbres sont plus généralement couchés sur le sol dans la direction du Sud-ouest au Nord-est; car, par quelque cause que ces arbres soient tombés, ils doivent être dirigés comme ils le sont, attendu que les vents de Sud-ouest font les plus fréquents & les plus violents dans ce canton-là.
La situation la plus admissible des difficultés que cette question présente est donc que les arbres ont été coupés. La hauteur des fouches, qui est généralement de deux pieds au-dessus du sol, favorise cette opinion, parce qu'à cette hauteur un arbre est ordinairement beaucoup plus mince que vers le sol, & que l'ouvrier peut appliquer toute sa force en employant la hache. L'état fain des troncs & des racines ne peut non plus s'expliquer d'aucune autre manière que par cette supposition.
On observe dans la partie inférieure de quelques-uns de ces troncs, les traces de la hache, par une entaille de deux pouces & demi. Ce qui explique que ces traces ne soient pas visibles partout, c'est que la hache avait des dimensions très-peu considérables, & que d'ailleurs le temps pendant lequel ces arbres sont\setcounter{page}{173} reftés exposés à l'air avant d'être couverts par la tourbe, doit avoir été très-long.
Mais, demandera-t-on, dans quel but peut-on avoir entrepris de couper des forêts d'une fi grande étendue ? Ce n'étoit pas pour employer les bois, puisqu'on les a laiffés fur place; ce n'étoient pas non plus pour les défrichemens que l'on avoit en vue, fans quoi l'on auroit débarraffé la furface du fol. Mais en confidérant la fituation locale de ces tourbes, & en nous rappelant l'hiftoire d'Angleterre depuis le regne de Domitien à celui de Caracalla, nous trouverons de fortes raifons d'attribuer aux Romains l'opération qui a été faite fur ces forêts.
C'eft un fait bien connu que depuis le moment où Céfar envahit la Grande Bretagne jufqu'au déclin de l'Empire, les Bretons, incapables de tenir la campagne devant les troupes Romaines, ufoient de la reffource de fe retirer dans les bois & dans les marais pour inquiéter enfuite les armées Romaines par des excurfions. Les généraux Romains, depuis Agricola du moins, employerent non-feulement leurs foldats, mais une partie des Bretons foumis, à enlever aux Bretons indépendans leurs lieux de refuge, en abattant les bois, en defféchant les marais, ou en y faifant des routes. Il paroît que c'eft à ces aviliffans travaux que Galgacus faifoit allufion lorfque dans fon difcours aux Calédoniens\setcounter{page}{174} prêts à combattre Agricola, il leur dit: "Corpora ipsa ac manus, sylvis ac paludibus emuniendis, inter verbera ac contumelias conterunt."
(Tacit. in vit. Agric. cap. 31.)
On dit que Sévère employa de la même manière une grande partie de ses troupes, non seulement pour bâtir la muraille qui porte son nom; mais aussi à abattre les forêts, dessécher les marais, & jeter des ponts sur les rivières qui s'opposoient à sa marche vers le Nord de l'Angleterre. Mais, quoique dans cette marche il doive avoir passé sur le terrain même qui est occupé par les marais de Kincardine & de Frosk, je suis porté à croire que les forêts qui avoisinoient le Forth ont été détruites par ses prédécesseurs, parce que ceux-ci avoient le projet de faire un mur depuis le frith de Forth à celui de Clyde, pour marquer les limites de l'Empire Romain; car Sévère retira ses troupes d'entre les deux murailles, & ajouta de nouveaux travaux au mur d'Adrien ou en bâtit un lui-même dans une direction semblable.
Il est certain que les forêts dont il est question devoient embarrasser les Romains plus qu'aucune autre forêt de l'Isle, soit par leur proximité de la province Romaine, soit parce que la seule route par laquelle on pouvoit pénétrer dans la Calédonie traversoit le Carse; c'est-à-dire les marais de Kincardine & de Frosk:
Le marais de Kinnaird qui sans doute étoit\setcounter{page}{175} autrefois réuni à celui de Frosk, n'est qu'à un mille & demi de la rivière de Carron, laquelle rivière semble avoir été la borne de la province Romaine, dans l'endroit où elle cesse d'être navigable, c'est-à-dire où elle entre dans la plaine du Carse. Le marais de Kincardine n'est distant que de douze milles de la station de Camelon. Les forêts situées comme celles-là l'étaient, fournissaient donc aux Calédoniens un abri très-avantageux, soit pour faire des incursions dans la province Romaine, soit pour inquiéter les armées qui tentaient de pénétrer vers le Nord.
Il y a un fait qui démontre qu'il a existé dans cette partie de l'Isle un peuple plus avancé dans la civilisation que ne l'étaient les anciens Calédoniens, & cela à une époque plus reculée que celle de la formation des tourbes de Kincardine. Ce fait est la découverte d'une chaussée pratiquée à la surface de la glaise, & au-dessous d'une masse de tourbe de l'épaisseur de huit pieds. La partie de la route déjà déblayée & reconnue a soixante-dix verges de long; sa largeur est de quatre verges. Elle est construite avec des arbres de neuf à douze pouces de diamètre, disposés en longueur, dans la direction de la route. D'autres arbres, de moitié moins gros, ont été disposés à angle droit des premiers; & le tout avoir été recouvert de fascines. L'épaisseur qu'occupent les matériaux de cette route varie selon les lieux. Les arbres\setcounter{page}{176} qui font dans le fens de la route font en général, à la furface de la glaise; mais dans les parties humides & baffes, on les trouve enfoncés d'environ deux pieds dans l'argile.
Cette route traverse une piece de terre qui eft plus baffle que le terrain environnant. Sa direction eft depuis le Forth vers la partie la plus étroite du marais, du côté d'une autre route que l'on croit Romaine, & qui paffle entre la riviere Teith & le marais. Les traces de cette derniere route ont été reconnues & fuivies depuis Camelon jufqu'à quatre milles au Nord-ouest du pont de Dript, où il y avoit autrefois un bacq. Cette route Romaine eft faite avec une épaisseur d'un pied de gravier, fous lequel il y a, dans quelques endroits, une couche de pierres. Elle paroît avoir 20 pieds de large, ce dont on ne peut pas s'affaûrer exactement, parce que le terrain a été foumis à la charrue. Sa direction au-delà du Forth, à Dript, eft au Nord - ouest en ligne droite, vers le paffage de Leny, qui eft la voie naturelle pour communiquer aux montagnes, & par lequel paffle aujourd'hui le chemin qui conduit au fort William. Il eft donc très-probable que cette route étoit déstinée aux troupes que les Romains employoient contre les Calédoniens; elle conduisoit aussi, vraisemblablement, aux autres routes qui pénétroient plus loin vers le Nord par Dumblane, & la ftation bien\setcounter{page}{177} connue d'Ardoch. On doit également présumer que cette route tenoit à celle qu'on a découverte au fond de la tourbe, & que celle-ci est également un ouvrage des Romains.
Voici les conséquences que l'on peut déduire des faits & des raisonnemens ci-dessus. Il paroît qu'avant Agricola (le premier d'entre les généraux Romains qui ait couvert les frontières du Nord par une chaîne de postes) la plus grande partie du plat pays étoit occupée par des forêts ; qu'à-peu-près dans ce temps-là, une grande partie de ces forêts fut abattue par les Romains, parce qu'étant situées sur leur frontière elles fournissoient aux gens du pays des retraites sûres d'où ils venoient sans cesse désoler la province Romaine par des incursions ; enfin que les arbres abattus, se pourrissant dans ces terrains marécageux, ont donné lieu à la formation de cette masse énorme de tourbe qui les recouvre aujourd'hui. La production de la tourbe par les débris de la végétation des forêts n'est point une circonstance qui doive embarrasser, dans l'explication de ce phénomène : les observations des naturalistes l'ont mise hors de doute \footnote{Voy. le Mémoire de Lord Cromarty, sur la tourbe. Trans. philos. Vol. XXVII. p. 296. (A) Sciences & Arts. Vol. 9. N°. 2. an VII. M}, du moins pour les pays froids, & cela sert à expliquer certains phénomènes semblables que l'on observe dans différentes parties de l'Angleterre.
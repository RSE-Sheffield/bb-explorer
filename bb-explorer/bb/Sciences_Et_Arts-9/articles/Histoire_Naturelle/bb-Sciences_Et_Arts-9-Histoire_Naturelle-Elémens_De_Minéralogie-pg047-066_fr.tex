\setcounter{page}{47}
\chapter{Histoire Naturelle}
\section{ELEMENTS OF MINERALOGY, &c. Élémens de minéralogie, par Richard Kirwan Esq. Membre des Acad. de Stockholm, d'Upfal, de Berlin, de Manchester, de Philadelphie, &c. seconde édition : corrigée & augmentée. Vol. II. Elmsly. (Second Extrait.)}
L'Auteur commence la quatrième partie de son ouvrage, dans laquelle il s'occupe exclusivement des substances métalliques, en donnant un résumé des propriétés qui distinguent cette classe de substance, dont l'homme civilisé à su tirer tant d'avantages. Voici cette introduction :
"Les substances métalliques, chacune dans son état parfait, & sans alliage réciproque, présentent les caractères suivants :"
Leur couleur est ou un orangé rougeâtre plus ou moins clair ; ou rouge jaunâtre ; ou jaune rougeâtre ; ou blanc pur ; ou blanc rougeâtre ; ou blanc bleuâtre plus ou moins clair ; ou gris blanchâtre ; ou enfin, gris bleuâtre.
" Fracture, grenue, ou grenue en feuillets ; ou striée.
" Fragments, 4, 3.
\setcounter{page}{48}
" Dureté de 5 à 12.
" Pesanteur spécifique de 6 à 23.
" Tous les métaux sont solubles ou dans l'acide nitreux, ou dans l'eau régale, & on peut les en précipiter par les alkalis, ou caustiques, ou aérés \footnote{Les précipités sont souvent redissous. (A)} ou bien (sauf le Platine) par l'alkali prussien (prussiate de potasse). Tous sont fusibles, à un degré de chaleur propre à chacun, & lorsqu'ils sont en fusion, ils prennent une surface convexe, ou, s'ils sont en petite quantité, globulaire. Lorsqu'ils sont calcinés, s'ils ne sont pas trop volatils, ils donnent au Borax ou au sel microcosmique une couleur particulière après la fusion, ou bien ils les rendent opaques."
"Lorsqu'ils sont parfaitement fondus, ils sont, pour la plupart, miscibles ou combinables les uns avec les autres, mais (à l'exception du fer) ils refusent de s'allier avec leurs propres chaux (oxydes) ou avec les autres substances non-métalliques, à l'exception du soufre, du phosphore, & du charbon; & des acides sulfurique & phosphorique qui s'unissent à la plupart d'entr'eux."
"On connoit actuellement dix-sept substances susceptibles de prendre l'apparence métallique: L'or, le platine, l'argent, le cuivre, le fer, le plomb, l'étain, le mercure, l'antimoine, le régule d'arsenic.\setcounter{page}{49} D'arsenic, le bismuth, le cobalt, le nickel, le régule de manganese, l'uranite, le Sylvanite & le Titanite. Il y en a trois autres dont la faculté de passer à l'état métallique est jusqu'à un certain point, douteuse. Ce sont le Molybden, le Wolfranc \footnote{Nous nous étonnons que l'auteur en plaçant ici le Wolfranc, & en ne nommant point le Tungstène fasse abstraction des travaux de Scheele & des frères d'Elhuyar, qui paroissent avoir bien prouvé que le Tungstène est un véritable métal, & que le Wolfranc est cette même substance unie au fer. (R)}, & le Ménachanite.
"Parmi ces substances, les trois premieres, & le mercure, sont ordinairement appelées métaux nobles & parfaits, parce que lorsqu'ils ont été calcinés, ils repassent à l'état de régule, c'est-à-dire qu'ils recouvrent toutes les propriétés métalliques par le seul effet de la chaleur, & sans l'addition d'aucun combustible. Tandis que le cuivre, le fer, le plomb & l'étain ne peuvent se réduire bien sans cet interméde. On appelle ceux-ci, par cette raison, métaux imparfaits. Mais tous ceux que nous venons de nommer, en y comprenant même le mercure, ont pour propriété commune d'être malléables à un très-haut degré, & sont, par cette raison, appelés métaux entiers ou complets; tandis que le zinc, le régule d'antimoine, &c. sont ou très-peu malléables ou ne supportent point le marteau sans se briser. On les nomme à\setcounter{page}{50} caufe de cela, demi-métaux. Quelques minéralogistes rejettent cette distinction, parce qu'on observe une diminution progressive & aucune transition bien décidée dans cette propriété : mais si cette objection étoit fondée, on devroit aussi rejeter la distinction des sels & des pierres, & même celle des plantes & des animaux par la difficulté d'établir des limites bien précises. Le sens commun repousse ces scrupules. Il y a même une différence marquée entre les métaux nobles & les métaux imparfaits & les demi-métaux, c'est que le mercure, allié à ceux-ci, les volatilisle avec lui dans la distillation, tandis qu'il n'emporte point les premiers. On soupçonnoit il y a quelques années & d'après des apparences assez plausibles que certaines terres simples, étoient aussi susceptibles de réduction ou métallisation ; mais l'illusion fut découverte & dissipée par Savarefi, Klaproth, Westrumb & Thiauski."
"On appelle métaux natifs ceux que la nature présente à l'état de régule, c'est-à-dire sous forme métallique ; lors même qu'ils sont alliés entr'eux. Mais ceux qu'on trouve (& c'est leur apparence ordinaire dans les mines,) combinés avec quelque substance non-métallique, sont dits minéralisés par cette substance, on la désigne elle-même par l'épithete de minéralisateur & l'ensemble porte le nom de mine ; expression qui désigne aussi les terres ou pierres qui renferment\setcounter{page}{51} des matieres metalliques en quantité notable, minéralisées ou non.
Lorsque le minéralisateur est de nature saline, & rend le métal soluble dans moins de 20 fois son poids d'eau, on place d'ordinaire le composé parmi les sels : c'est le cas des vitriols de fer, de cuivre & de zinc. Mais lorsqu'on les considère surtout sous le rapport de leur base métallique, comme par exemple, les vitriols de cobalt & de nickel, on les range parmi les mines. Le plus commun des minéralisateurs est l'air pur (oxigene) dans son état concret, & l'air fixe (acide carbonique) s'y trouve souvent réuni. Le soufre, & l'arsenic à l'état de chaux, tiennent ensuite les premiers rangs ; & les acides vitriolique, marin, phosphorique, arsenical, & molybdique, se rencontrent moins communément. Le soufre & l'arsenic laissent, pour l'ordinaire, le brillant métallique aux métaux qu'ils minéralisent ; mais les autres minéralisateurs leur donnent, tout au plus une apparence saline, sans laquelle il seroit bien difficile de les reconnoitre.
"Dans quelques cas le soufre s'unit aux substances métalliques non calcinées, dans d'autres il se combine avec leurs chaux (oxides) ainsi que je l'ai déjà indiqué en traitant des pyrites. Cette distinction a son utilité lorsqu'il est question de choisir entre les diverses matieres d'opérer, & d'estimer la richesse d'une\setcounter{page}{52} mine donnée, avant de l'analyser. Quelques minéralogistes du premier mérite étoient diffusés à ne point admettre l'arsenic parmi les minéralisateurs, à raison de ce qu'il est lui-même un métal, mais comme il est maintenant certain qu'il est minéralisateur lorsqu'il se rencontre à l'état d'acide, je ne vois pas pourquoi sa chaux, qui est certainement demi-acidifiée, ne joueroit pas le même rôle."
"Lorqu'on observe la pesanteur spécifique d'une mine, il faut la séparer soigneusement de toutes les molécules de terre ou de pierres qui peuvent lui être étrangères."
"En donnant les résultats de l'analyse des mines, je n'ai pas désigné la quantité de métal parfait à laquelle correspond une quantité donnée de chaux ou de sel métallique, parce que ces rapports se trouvent réunis dans les tables qui sont à la fin de ce volume."
Après ces généralités l'auteur entre en matière. Quelques nomenclateurs en traitant des métaux, les ont rangés selon l'ordre dans lequel ils paroissent acquérir successivement des propriétés métalliques de plus en plus parfaites; on peut y arriver ainsi comme insensiblement en passant de la plombagine au molybdène, de celui-ci au Tungstène où à l'arsenic. D'autres ont suivi un ordre inverse, & commencent par les plus parfaits. C'est la méthode qu'a adopté Mr. Kirwan, & en conséquence il commence par l'or.
\setcounter{page}{53} Nous ferions un volume si nous prétendions suivre notre savant auteur dans les détails qu'il donne sur chaque métal. Il en décrit d'abord les propriétés physiques & chimiques dans l'état de règle. Il le considère ensuite sous l'état de mine & classe avec beaucoup d'ordre ces composés très variés; en introduisant toutes les subdivisions nécessaires; savoir, en espèces, tribus, familles & variétés. Chaque métal fait le sujet d'un chapitre. Ensuite, dans son chapitre 21°. intitulé: De l'analyse & de l'essai des mines métalliques, & qui remplit seul un quart du volume. Il indique, avec les détails suffisants, tous les procédés de métallurgie pratiqués par les meilleurs auteurs pour le traitement des mines. Il cite toujours ses autorités, & son érudition paroît être aussi vaste que son jugement est vain. C'est surtout des écrivains Allemands & Suédois qu'il emprunte les lumières quand sa propre expérience ne les lui a pas fournies. Nous allons prendre ça & là quelques morceaux pour faire connaître sa manière.
\subsection{CHAPITRE II. Platine.}
"Ce métal, dans son état le plus parfait, a les propriétés suivantes:
Sa couleur est blanche, d'une teinte intermédiaire entre l'étain & l'argent."
Son lustre 4. Sa dureté 7,5: Sa pesanteur\setcounter{page}{54} spécifique, de 20,6 à 23. Celle d'un coin dont l'amiral Gravina m'a fait présent est de 20,663 & celle d'une barre que le Roi d'Espagne envoya au Roi de Pologne, de 20,722. Les variations dépendent du plus ou moins de refoulement que lui a procuré l'action du marteau & de la quantité de flux qu'il peut avoir retenu à la fonte."
"Il est malléable, ductile, & laminable comme l'or; il n'est point affecté par l'action de l'air; il n'est soluble que dans l'eau régale ou dans l'acide oxy-muriatique, qui prend lorsqu'il en est saturé, une couleur d'un rouge foncé; on l'en précipite par le Tartarin (la potasse) & plus difficilement par la soude\footnote{Nous avons observé que le précipité de platine employé comme couleur dans la peinture à l'émail fournit des teintes gris de fer plus ou moins foncé qui ressemblent à l'encre de la Chine. (R)}. Il n'est pas visiblement affecté par l'alkali Prussien pur,\footnote{Mr. Berthollet a trouvé que lorsqu'il est disoùs il s'acidifie dans un degré très-marqué, ce qui explique plusieurs phénomènes singuliers qu'offre ce métal. (A)} & point du tout par une solution étendue, de vitriol de fer; ces propriétés le distinguent de l'or. Il s'amalgame avec le mercure."
"Il est à-peu-près infusible dans le feu de nos fourneaux, mais il cède à l'action des fortes lentilles & à celle du chalumeau avec un jet d'air pur."
"Le Platine cru ou commun nous vient du Pérou, en grains; d'une couleur de fer grisâtre,\setcounter{page}{55} de figure irrégulière & plus ou moins aplatis, mêlés de sables ferrugineuses & quartzeux en différentes proportions. Sa pesanteur spécifique dans cet état, est de 6 à 11. Il renferme souvent des particules d'or ou de mercure. Lorsqu'il est purgé des particules pierreuses étrangères, sa pesanteur spécifique est de 12 à 16 \footnote{Nous l'avons trouvé de 15,725. (R)}. Le platine en cet état, est encore intimement combiné avec le fer & agit sur l'aiman; il est extensible jusqu'à un certain point sous le marteau & contient entre un tiers & un quart de son poids de fer. Cependant cet alliage n'est soluble que dans l'eau régale ou dans l'acide oxy-muriatique."
"On le sépare du fer en le dissolvant dans environ huit fois son poids d'eau régale, en aidant la dissolution sur la fin par la chaleur. On précipite le fer par l'alkali Prussien, ou, encore mieux, par une solution concentrée de sel ammoniac qui ne précipite que le platine. Ce précipité séché & chauffé au plus haut degré pendant deux heures au feu de forge, donne une masse agglutinée qui supporte le marteau."
" L'esprit de sel lui enlève aussi beaucoup de son fer. L'acide phosphorique dans l'état glacial le rend aussi fusible dans un creuset garni \setcounter{page}{56} de charbon, ainsi que l'a découvert M. Pelletier."
"Mais la meilleure méthode pour traiter ce métal est celle de Mr. Jeannetty. Je vais la décrire parce qu'elle n'est pas encore bien connue."
"1°. Il triture d'abord le platine ordinaire avec de l'eau pure pour le séparer du fer & des autres matières qui s'y trouvent mêlées."
"2°. Il en mêle ensuite 1,5 livre avec 3 d'arsenic blanc, & une de potasse purifiée; il prépare un creuset capable de contenir vingt livres de cette matière."
"3°. Quand le creuset & le fourneau sont bien chauds, il y met d'abord un premier tiers du mélange, puis donne chaud; il en jette un second & donne encore un bon coup de feu, puis il projette la dernière dose."
"4°. Quand le tout est bien fondu, il enlève le creuset; il le laisse refroidir; casse le régule, qui attire alors l'aimant; il le fond une seconde fois, puis une troisième, s'il est encore magnétique; il le laisse enfin refroidir & casse le régule."
"5°. Il prépare alors des creusets capables de contenir des régules de trois pouces & demi de diamètre; il met dans chacun une livre & demi du régule en morceaux avec poids égal d'arsenic, & demi livre de potasse; il fond le tout & le laisse refroidir dans une position horizontale, afin que le régule soit partout d'épaisseur égale."
\setcounter{page}{57}
"6°. Il met ces régules sous une moufle ; il les chauffe jusqu'à ce qu'ils commencent à s'évaporer : il ferme le fourneau pour modérer la chaleur, qui pourroit volatiliser le tout, & il les tient dans cet état pendant fix heures."
"7°. Il les chauffe dans l'huile commune au degré où l'huile s'évapore à siccité : il continue cette opération pendant toute la durée de l'évaporation de l'arsenic.
"8°. Finalement il les met dans l'acide nitreux (probablement pour détruire les derniers restes de l'arsenic) & pour libérer ensuite les régules de cet acide il les fait bouillir dans l'eau : il les chauffe au rouge dans un creuset pour prévenir toute absorption de matière étrangère, vu qu'ils font encore très-spongieux. Enfin, il les fait chauffer à feu nud & les forge en barreaux."
"Le platine n'a point de mines connues ; mais on le trouve dans l'état métallique parmi les mines d'or d'alluvion, mêlée au sable, dans les paroisses de Novita & de Citaria, au nord de Choco dans le Pérou. On le sépare de l'or, partie par le lavage & partie par l'amalgame."
"Il y a dans le cabinet de l'Académie de Bergara, un morceau de platine qui est de la grosseur d'un œuf de pigeon."
"Mr. Tillet a trouvé que le platine combiné avec une certaine portion d'argent, est soluble dans l'acide nitreux."
\setcounter{page}{58}
"Mr. Richter nous affure qu'on peut le débarrasser complètement du fer, en ajoutant du Tartarin (potaffe) à la folution, jufqu'à-ce que le précipité paroiffe : on ajoute enfuite du tartre vitriolé (fulfate de potaffe,) jufqu'à-ce que le tout foit précipité; on lave enfuite jufqu'à-ce que le vitriol de fer, beaucoup plus foluble que ne l'eft celui de platine, foit diffout par le lavage, & on le reconnoît par l'alkali pruffique. On fait fécher le refte du précipité, & on le mêle avec une fois & demi fon poids de foude defféchée : on le place enfuite dans un creufet, qu'il ne rempliffe pas tout-à-fait ; enfuite on le couvre & on le chauffe par degrés jufqu'à-ce qu'il foit fondu. On trouve alors, affure-t-il, un régule couleur d'argent."
On voit , d'après cette defcription que le platine poffede dans un degré éminent toutes les propriétés qui caractérifent une fubftance métallique ; qu'il eft en quelque forte le plus métal des métaux: paffons à l'autre extrémité de l'échelle & décrivons avec l'auteur le Titanite.
"C'eft l'infatigable Klaproth qui a découvert que cette fubftance étoit de nature métallique. On l'avoit nommé fchorl rouge , & je l'avois défigné fous ce nom dans mon premier volume. Il me femble être une fubftance intermédiaire entre un demi métal & une terre ; mais des expériences ultérieures , & en particulier , fa faculté de s'allier avec d'autres fubftances métalliques décideront la question."
\setcounter{page}{59}
"On le trouve, d'après le baron de Born, à Rhonitsz ; selon d'autres à Boinick en Hongrie \footnote{Nous en avons de beaux échantillons venant du St. Gothard. (R)}."
"Sa couleur eft d'un rouge brunâtre."
"On le trouve cristallifé en prismes rectangulaires à quatre faces, rayés ou fillonnés longitudinalement, longs d'environ ½ pouce, ou moins ; souvent en aiguilles & placés sur du mica schisteux alternant avec le quartz \footnote{Il affecte volontiers l'apparence d'un tissu formant des mailles rhomboïdales. (R)}."
"Lustre 3. Transparence 0 excepté en très-petits fragmens : alors, 1."
"Sa fracture en travers eft feuilletée : la fracture longitudinale tient le milieu entre la feuilletée & la raboteuse ; dans quelques endroits elle approche de la forme conchoïde."
"Fragmens 3. sa poudre eft couleur de brique ; ou rouge orange."
"Dureté 9 ; fragile ; pesanteur spécifique 4,18.
"Exposé à une chaleur de porcelaine dans un creuset d'argile il ne change point. Seulement sa couleur brunit un peu. Mais dans un creuset de charbon il s'eft rompu en fragmens anguleux, a perdu son lustre & sa couleur, & a passé au brun pâle."
"Au chalumeau, il résiste au sel microcosmique,\setcounter{page}{60} mais il cède au borax && à la soude, && donne avec le premier un verre couleur d'hyacinthe, && avec le second, un verre rouge blanchâtre."
Aucun des acides minéraux, ni l'eau régale, ne l'attaquent, même aidés de la chaleur :
"200 grains de cette substance, fondus dans un creuset de porcelaine avec 5 fois son poids de Tartarin doux (carbonate de potasse) fortement après le refroidissement une masse grisâtre, assez blanche, dont la surface étoit cristallisée en aiguilles && la structure fibreuse. Dissoute dans l'eau bouillante, elle ne tarda pas à précipiter une substance blanche qui édulcorée && doucement séchée pesoit 328 grains && avoit une apparence terreuse. Je l'appelle en cet état chaux de Titanite. La liqueur alkaline, étant ensuite saturée d'acide marin déposoit un mélange argillosiliceux, à la quantité d'environ 8 grains && qui provenoit vraisemblablement du creuset."
Propriétés de cette chaux. Elle est soluble dans les acides vitriolique, nitreux, && marin, mais par l'évaporation spontanée, elle forme avec tous ces acides une masse plus ou moins gélatineuse, que Klaproth attribue à la terre siliceuse qui s'y trouve mêlée, car la solution nitreuse ainsi évaporée laisse voir quelques cristaux."
"Ces solutions se précipitent en flocons blancs\setcounter{page}{61} légers, par le Tartarin doux ou le sel alkali caustique, ils sont d'un vert brunâtre par l'alkali prussien & d'un rouge brunâtre par la teinture de Galles."
"L'étain change la solution muriatique en rouge bleuâtre & le zinc en bleu indigo. Cette dernière cependant blanchit lorsqu'on la chauffe peu-à-peu, & laisse un précipité blanc rediffoluble dans l'acide marin, & avec le zinc il redonne une couleur bleue, comme auparavant. On pourroit présumer delà, que la chaux est plus complètement déoxigénée par l'étain que par le zinc, la couleur bleue étant la transition ou la teinte par laquelle passent plusieurs substances avant que de devenir rouge. Il s'en suit que le rouge est la couleur de ce métal dans son état métallique & qu'il est à cet état de régule sous l'apparence sous laquelle on lui avoit donné le nom de schorl."
"50 grains de cette chaux blanche furent réduits par l'ignition à 38. Elle étoit jaunâtre tant qu'elle étoit chaude, mais , comme la chaux de zinc, elle blanchissoit en se refroidissant. Après l'ignition elle devient insoluble dans les acides, probablement à raison de la cohésion qui unit ses molécules. Echauffée sur le charbon elle devient couleur de rose, & ensuite bleu d'ardoise ; & finalement se fond en un verre imparfait, dont la surface est finement striée. Cette chaux se conduit avec les flux comme\setcounter{page}{62} le fait le Titanite natif. Elle est irréductible par les méthodes ordinaires."
"Elle me paroît se rapprocher par un grand nombre de ses propriétés de la chaux de ménacha-nite; la différence entre ces deux substances peut provenir du fer contenu dans la derniere."
"Nous supprimons les détails sur les mines de Titanite. Ils n’offrent gueres que la répéti-tion de ce qu’on a lu tout-à-l’heure, puis que ce métal s’offre naturellement sous l’état de régule. Au surplus, en donnant à cette substance le nom de métal, on la met dans une classe à la-quelle elle ne paroît avoir jusqu’à présent que bien peu de titres."
Le fer occupe 40 pages dans l’ouvrage de Mr. K. c’est, sans contredit, le premier des métaux, sous le rapport de l’utilité; & c’est aussi celui dont les transformations sous l’état de mine sont les plus variées; & il est répandu avec une telle profusion à la surface du globe que sa découverte ne pouvoit échapper dès les premiers pas de la civilisation. Nous disons sa découverte, car autant il est commun sous le masque de la minéralisation, sous lequel il est, d’ordinaire, absolument méconnoissable, autant il est rare dans l’état natif. On a même soup-çonné que les échantillons de ce genre qu’on a trouvés en fort petit nombre, étoient le ré-sultat de quelque réduction artificielle produite par l’action du feu. L’auteur paroît cependant\setcounter{page}{63} persuadé que le fer natif d'Eibensteck en Saxe, celui de la montagne du grand Gilbert en Dauphiné, & celui trouvé en gros bloc dans les plaines d'Etumpa au Pérou, sans parler de la grande masse observée par Pallas en Sibérie, que tous ses échantillons, disons-nous, sont des produits naturels soit par la voie sèche, soit par la voie humide.
L'auteur classe les mines de fer en douze especes. La seconde de ces especes savoir le fer minéralisé par l'air pur, ou oxidé, est la plus nombreuse. Il la subdivise en quatre Tribus dont l'une a jusques-à six familles, & une de ces familles renferme jusques à 5 variétés. Ce systême fournit trente-huit divisions ou sous-divisions qui nous paroissent heureusement & ingénieusement distribuées pour qu'on puisse se reconnoitre parmi ces nombreux composés qui tous ont le fer pour base; & remarquons que ce nombre feroit bien plus grand encore si l'auteur n'en eût exclu les pyrites soit sulfureuses, soit arsenicales. Il a placé les premieres parmi les combustibles & les secondes parmi les mines d'arsenic.
Le chapitre 24 qui traite, ainsi que nous le disions tout-à-l'heure, de l'analyse & de l'essai des mines est un abrégé de métallurgie très-complet & très-instructif. Il commence par une table assez étendue dont l'objet est d'aider les commençans à juger par les caracteres extérieures.\setcounter{page}{64} rieurs d'une mine, de l'espece de métal qu'on doit en attendre par l'analyse. L'auteur réunit les mines qui se ressemblent le plus sous les deux caracteres visibles les plus frappans, savoir, la couleur & le lustre, abstraction faite du métal auquel elles appartiennent. Chaque couleur forme un chef de division, & renferme, quant au lustre, trois subdivisions ou gradations, savoir, le lustre métallique, le lustre ordinaire & le lustre nul. Le commençant trouve dans cette ordonnance particuliere ses termes de comparaison rapprochés les uns des autres & peut ainsi se former plus promptement le coup-d'œil, qualité si essentielle dans l'étude des minéraux.
Suit un tableau des pesanteurs spécifiques des mines de diverses substances métalliques : les pesanteurs extrêmes seulement, font indiquées pour chaque métal.
"Lorsqu'on a approché jusqu'à un certain point ( dit ensuite l'auteur ) en observant la couleur & le lustre de la mine, de la détermination du genre du métal auquel elle appartient, on peut circonscrire sa recherche en comparant les caracteres extérieurs de la mine à examiner, avec ceux de l'espece de mine dont elle paroît se rapprocher ; on arrive ensuite encore plus près en examinant les résultats de l'action des acides & du chalumeau, & en les comparant avec ceux connus dans l'échantillon qui sert de comparaison. Si toutes ces apparences s'accordent\setcounter{page}{65} à faire rapporter la mine inconnue à l'échantillon connu, alors on procéde à l'essai par la voie sèche, ou à l'analyse par la voie humide, ainsi qu'on le verra dans les sections suivantes.
L'auteur traite une à une dans ces sections toutes les variétés de mines, métal par métal, & indique d'une manière détaillée, les procédés suivis par les meilleurs métallurgistes modernes. L'embarras du choix, & le peu de chance que nous aurions d'être utiles en transcrivant le traitement de l'une quelconque de ces mines, nous arrête.
On trouve à la fin du volume, des tables très étendues qui renferment les proportions des divers ingrédients qui composent les chaux métalliques. La première colonne désigne l'oxide dont il est question & son précipitant; les quatre suivantes indiquent la proportion du métal, de l'oxigène, de l'eau, & de l'air fixe sur cent parties de chaque oxide. La dernière colonne contient les noms des chimistes qui ont fourni les résultats. La principale différence entre ces tables & celles du même genre qui ont déjà paru, consiste, indépendamment de leur plus grande étendue, dans la distinction essentielle introduite par l'auteur, entre la quantité d'oxigène contenu dans les oxides & celle de l'eau qui s'y trouve souvent associée. Ces tableaux sont au nombre de trois. Nous venons de décrire le premier. Le second donne la proportion\setcounter{page}{66} des ingrédiens dans les précipités acido-métalliques provenant des solutions dans l'acide nitreux. Le troisieme indique la proportion du soufre & du métal sur 100 parties de chaque sulfure métallique. En voici le résultat abrégé.
\comment{table}
100 parties d'argent peuvent prendre 15 parties de soufre
. . . . . . . de cuivre . . . . . . . . . . . . 25,4
. . . . . . . de fer . . . . . . . . . . . . . . . 56
. . . . . . . d'étain . . . . . . . . . . . . . . . 18
. . . . . . . de chaux d'étain . . . . . . . . . 40
. . . . . . . de plomb . . . . . . . . . . . . . . 15
. . . . . . . de mercure . . . . . . . . . . . . . 25
. . . . . . . de régule d'antimoine . . . . 29,8
. . . . . . . d'antimoine natif . . . . . . . . 35
. . . . . . . de bismuth . . . . . . . . . . . . . 17,6
Vingt pages de notes explicatives accompagnent ces trois Tableaux.
L'ouvrage entier porte l'empreinte du génie méthodique de son savant auteur, de ses connoissances très-étendues & de son amour pour la vérité. Il est au-dessus de tout préjugé national, & les écrits des savans du Continent paroissent lui être aussi familiers que ceux de ses compatriotes.
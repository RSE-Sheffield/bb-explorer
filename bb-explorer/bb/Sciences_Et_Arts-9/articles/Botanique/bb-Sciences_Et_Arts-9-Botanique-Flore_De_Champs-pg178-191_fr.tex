\setcounter{page}{178}
\chapter{Botanique}
\section{FLORA RUSTICA exhibiting , &c. Flore des champs , accompagnée de figures exactes des plantes qui sont utiles ou nuisibles à l'Agriculture. Par Th MARTIN, B. D. & F. R. S. &c. Londres , 4 vol. in-8. 144 pl. col. \large{Article communiqué.}}
Les Arts sont enfans des sciences : celles-ci doivent donc diriger les opérations des premiers , & c'est en se prêtant un mutuel appui que les uns & les autres s'élèvent au degré de perfection dont ils sont susceptibles. Plus un Art est compliqué & plus ceux qui le pratiquent sentent la dépendance où ils sont , & s'empressent de suivre les directions que les savants leur fournissent ; c'est ainsi que l'horloger, le navigateur , l'aéronaute , sont entièrement soumis au géomètre , à l'astronome, au physicien ; mais dans les arts qui sont peu compliqués , c'est-à-dire qui altèrent peu les corps naturels , on oublie facilement cette dépendance ; la routine décide uniquement le choix des procédés & des matieres employées par les Artistes , & de là résulte que les arts les plus simples sont les moins perfectionnés ; l'agriculture est un exemple frappant de ce que j'avance\setcounter{page}{179} elle est la source où tous les arts viennent puiser leurs matériaux; mais elle est restée en arrière au milieu des progrès de la civilisation, & cela tient en partie à ce que le flambeau des sciences ne l'a pas encore suffisamment éclairé. Ceci est également un reproche aux agriculteurs qui n'ont point cherché à appliquer à la culture les découvertes scientifiques, & aux savans, qui ne se sont point assez rapprochés des premiers. La physique végétale peut rendre à l'agriculture des services éminens; & la botanique elle-même peut lui être fort utile en indiquant de nouvelles plantes à cultiver, en faisant connoître plus précisément celles qui font la richesse de l'agriculteur: cette application de la botanique à l'agriculture a été faite avec plus de soin en Angleterre que sur le continent; j'en appelle aux ouvrages de Rays, de Stillingfleet, de Miller, de Curtis; & à l'ouvrage même dont je vais donner une idée.
Je considérerai cet ouvrage sous deux points de vue, 1º sous le rapport de la botanique, 2º sous celui de l'agriculture.
La Flora Rustica donne la description de 144 plantes choisies parmi toutes celles qui sont les plus communes dans les champs, les prés & les lieux cultivés; elle n'a point été achevée, comme on le voit par le petit nombre de plantes qu'elle renferme; à des descriptions exactes & concises, sont jointes des figures qui\setcounter{page}{180} n'ont pas un grand degré de perfection. La petitesse du format a forcé le peintre à ne présenter qu'un fragment de la plante trop petit pour en donner une idée toujours suffisante. Les couleurs sont mises le plus souvent avec peu de soin, & dans les plantes à découpures menues, comme les ombellifères, on ne retrouve point dans les figures cette élégance & cette légèreté qui les caractérisent.
Sous le point de vue botanique on trouve de l'exactitude, mais peu de choses nouvelles. Dans cet ouvrage; le genre des Trifolium est celui qui est traité avec le plus de soin; l'auteur en décrit sept espèces qu'il distingue, avec des caractères plus précis qu'on ne l'avait fait jusques à présent.
1°. Le Tr. Alpestre ( fig. 1. ) a les épis fermés, les corolles presqu'égales, les stipules setacées divergentes, les folioles lancéolées, les tiges droites, raides & simples.
2°. Le Tr. medium a les épis lâches, presque globuleux, velus & sessiles, les corolles inégales, les stipules en alène divergentes; les tiges un peu flexueuses, branchues & pubescentes ( fig. 2. )
3°. Tr. pratense a les épis fermés; les corolles inégales, quatre dents du calice égales, stipules terminées en pointe aiguë, tiges courbées au bas. ( fig. 3. )
4°. Tr. rubens a les tiges droites, les feuilles finement dentées en scie, les épis longs &\setcounter{page}{181} villeux, les corolles monopétales. ( fig. 9. )
5°. Tr. flexuosum ( fig. 13. ) a les épis de fleurs laches ; corolles presqu’égales, stipules en alene, convergentes ; tiges flexueuses branchues, c’est le Tr. medium Lin. Suec.
6°. Tr. lupinaster ( fig. 16. ) a les têtes de fleurs partagées par le milieu, les feuilles quinées & sessiles, les légumes polyspermes.
7°. Le Tr. repens ( fig. 34. ) a les têtes de fleurs en petites ombelles ; les légumes tétrapermes, les tiges étalées.
Comme ces espèces excepté la 6e. sont originaires de notre pays, j’ai cru qu’il seroit agréable à ceux qui étudient les végétaux indigènes de trouver ici leurs caractères ; mais je me hâte d’en venir aux applications de la botanique à l’agriculture.
Parmi les plantes qui nous sont utiles, on place au premier rang la nombreuse famille des graminées ; ces végétaux fournissent notre nourriture la plus essentielle, & celle des bestiaux & des animaux qui nous servent. Mais toutes ne possèdent pas ces qualités au même degré, & il est important de les distinguer : les caractères de ces plantes peuvent paroître minutieux & inutiles à ceux qui ont peu réfléchi sur ce sujet, mais ils n’en font pas moins réels, constants, & il seroit à désirer que dans la culture des prairies, on eût moins négligé le choix des gramens qu’on y introduisoit. On ne\setcounter{page}{182} mélange pas dans les prairies artificielles le trefle, la luzerne & le fain-foin, pour quoi donc semer ensemble le *phleum pratense*, la *briza media*, l'*alopecurus pratensis*, &c. qui ont des habitudes tout auffi différentes. Entrons dans quelques détails sur les divers gramens.
*L'anthoxanthum odoratum* (fig. 23.) qui croît dans tous les terrains, paroît plaire aux moutons; il a l'avantage de fleurir de bonne heure, enforte qu'il peut se refermer avant qu'on le fauche. Curtis en recommande fortement la culture.
*L'alopecurus pratensis* (fig. 6.) réunit tous les avantages que peut desirer le cultivateur, l'abondance, la bonne qualité, une prompte maturité & des semences faciles à recueillir. *L'alopecurus agrestis* (fig. 22.) qui n'en differe que par ses glumes glabres & mousses, lui est inférieur en bonté; l'*alopecurus geniculatus* (*fig. 97.*) est mangé par le bétail; mais les plantes coudées ont le désavantage de perdre un peu de place.
Le *phleum pratense* (fig. 5.) est cultivé depuis 1750 environ, sous le nom de *Timothy-grafs* parce que Timothée Flawson l'a mis le premier en usage à New-York; on l'a apporté en Angleterre en 1763; mais sa réputation a été de courte durée, son foin est dur; il ne vient que dans les terrains humides. Haller dans une dissertation sur les plantes à fourrage de la Suisse en 1770, (*Mém. Econ. de Berne*) recommande qu'on le distingue du *phleum nodosum* dont le foin est encore plus maigre.
\setcounter{page}{183} Le phaleris canariensis (fig. 17.) a été naturalisé en Europe dans ce siècle; ce fourrage est cultivé avec succès dans le Comté de Kent près de Sandwich, & dans l'Isle de Thanet, on le sème à la fin de février à la quantité de six gallons par acre; Miller croit que la moitié suffiroit, la largeur de ses feuilles indique l'abondance qu'il doit procurer, & ce seroit une acquisition intéressante pour notre pays. Peut-être la phalaris aquatica (fig. 18.) qui est originaire d'Egypte pourroit-elle être utile.
Holcus lanatus (fig. 118.) est vanté par Haller pour les chèvres; mais n'est pas estimé en Angleterre, ses tiges servent à faire des cordes de bateaux dans l'Isle de Skye.
Melica nutans (fig. 65.) sert, dit Pennant, dans l'Isle de Rusa en Écosse, à faire des cordes pour des filets à poissons, qui restent long-temps sans se pourrir; la Melica uniflora (fig. 64.) est trop rare pour faire un objet de culture.
Dactylis glomerata (fig. 14.) a une tête si pesante que les fortes pluies l'écrasent; elle est d'ailleurs productive, mais son foin est dur.
Cynofurus cœruleus (fig. 20.) originaire de nos montagnes, est recommandable pour la culture, surtout à cause de la promptitude avec laquelle il fleurit.
Lolium perenne (fig. 44.) est le fameux Ray-grass. Le premier auteur qui parle de cette culture est Plot en 1677, en décrivant le Comté\setcounter{page}{184} d'Oxfort; on l'a recherché à cause de la facilité avec laquelle il se seme; Haller dit que les brebis le préfèrent à la paille; Curtis dit, au contraire qu'il ne leur convient pas. Il croit dans les terrains forts, argilleux & humides, & vit trois ans.
L'elymus & l'arundo arenaria ne servent au bord de la mer qu'à retenir les sables.
Parmi les hordeum, l'hordeum pratense (fig. 108.) qui est la rar. $\beta$ de l'hordeum murinum de Linné est un bon fourrage, mais peu productif; mais le vrai hordeum murinum (fig. 43.) est dangereux pour les chevaux au rapport de Curtis à cause des pointes de ses épis. Les autres ne font pas utiles dans les prairies.
Le triticum repens (fig. 124.) est un fléau des jardins & des terres de labour; on l'extirpe en en occupant les racines; ces racines utiles en médecine servent à Naples à nourrir les chevaux. On remplace cette racine en Italie avec celle du Panicum Dactylon (fig. 77.)
Les bromus sterilis, asper & mollis (fig. 99, 125, 126.) sont ou inutiles ou dangereux.
Parmi les festuca, il faut remarquer la Festuca pratensis (fig. 84.) que Curtis recommande comme fourrage à cause de la largeur de ses feuilles, & l'abondance de ses graines, & la Festuca fluitans (fig. 113.) dont nos ruisseaux sont garnis; les chevaux, les vaches, & les truies en sont friands. Ses semences qui sont douces & nour\setcounter{page}{185} riffantes sont récoltées en Allemagne & en Pologne, sous le nom de semences de manne; & on les estime dans les soupes & les gruaux, à cause de leur délicatesse; bien plus, ils la réduisent en farine & en font un pain, un peu inférieur à celui de froment. Le son est donné aux chevaux qui ont des vers, en les empêchant de boire pendant quelques heures. Les poules d'eau & les poissons, & en particulier les truites sont plus abondantes dans les ruisseaux où croît cette plante. Schreber dit que la manne se tire aussi du *Panicum sanguinale* qu'on cultive dans ce but en Allemagne. Pourquoi n'imitons-nous pas ces exemples d'industrie & d'économie?
Le *Poa annua* (fig.98.) est, selon la remarque de Curtis, la seule plante annuelle qui pousse des drageons d'où s'élèvent de nouvelles tiges; cette herbe est très-incommode dans les allées de jardins, d'où on a de la peine à l'extirper. Curtis conseille pour cela de les arroser avec de l'eau bouillante. Cette plante peut à ce qui paraît devenir utile; on la cultive dans le Suffolk, & elle y sert aux besoins journaliers des troupeaux, auxquels son feuillage tendre est agréable; elle n'acquiert jamais une grande hauteur; on recueille sa semence dans les mois de mai & de juin & on peut encore la multiplier en éclatant ses racines\footnote{On a à la Rochelle des prairies naturelles d'une espece de Poa qu'on y nomme Myfote; il ne semble pas devoir donner un fourrage abondant, vû que ses feuilles sont presque linéaires, & que sa tige courte ne paroît pas taller beaucoup. (R)}.\setcounter{page}{186} Les Brifa ne méritent pas d'être cultivés. L'Avena elatior (fig. 67.) qui a été pris à tort pour le ray-grafs, & qui est le fromental des Français, est un excellent fourrage. \footnote{Haller dit dans les Mém. de la Soc. Economiq. de Berne, que Stanislas fit cultiver le fromental en Lorraine en écartant les troupeaux des champs où il étoit semé; ce fourrage dure 10 ans, se fauche trois fois par an, & produit 18000 livres de foin par arpent; il n'est point cultivé en Suisse quoiqu'il y croisse spontanément. (R)} L'avena flavescens (fig. 112.) peut devenir un pâturage de brebis.
Telles sont les principales graminées dont on peut tirer parti dans l'agriculture. Presque toutes sont originaires de nos regions, & plusieurs sont totalement négligées; nous allons chercher au loin des végétaux étrangers à notre sol, avant d'utiliser ceux qu'il nous a fourni! je n'ai point parlé dans cette rapide énumération des gramens décrits dans la Flora rustica qui n'offroient pas d'utilité. Passons maintenant à l'examen abrégé de la 2e. famille d'où les agriculteurs tirent leurs richesses, je veux parler de la famille des Légumineuses.
J'ai déjà eu occasion de parler des trèfles fous le point de vue botanique & je ne trouve rien\setcounter{page}{187} de particulier sur la culture de ces plantes\footnote{L'énumération des trèfles cultivés que fait Martyn n'est pas complète ; 1º. le Tr. hybridum est cultivé en France au rapport de Haller ; ce fourrage est bon, mais annuel 2º. le Tr. fragiferum est cultivé en Irlande où ses tiges s'élèvent dit encore Haller jusqu'à sept pieds, 3º. le Tr. Stellatum est cultivé aux pieds des Pyrénées sous le nom de faronche ; il est annuel ; on le seme en automne & on le récolte en avril ; dans le même champ, on seme du froment ; ce faronche se seme en le jetant sur le terrain ; son herbe fraîche est aussi bonne pour les chevaux que l'avoine, 4º. le Tr. melilotus officinalis est commun dans tous les prés & s'y fait distinguer par son odeur exquise. (R)}. Le genre des Medicago offre plusieurs espèces intéressantes pour les prairies artificielles ; tout le monde connoît la Medicago fativa ou Luzerne (fig. 48.) originaire d'orient d'où elle a été apportée en Grèce du temps de la guerre de Darius. Elle n'a été introduite en Angleterre qu'en 1550 : Miller & Tull conseillent de la semer en la jetant avec la main. La M. lupulina (fig. 16.) est usitée en Angleterre, ainsi que la M. Falcate (fig. 86.) Mais Haller objecte contr'elle la dureté de ses tiges. La M. arborea (fig. 100) est cultivée en Italie & paroît être le Citise vanté par Virgile.
L'Hedisarum onobrychis ou sain-foin (fig. 47.) est originaire des sols calcaires & se trouve surtout sur les éminences ; il étoit déjà cultivé en 1540 en Angleterre & c'est en effet un fourrage\setcounter{page}{188} précieux; dans le même genre l'H. coromarium si remarquable par sa beauté est cultivé en Italie avec un grand succès à cause de son abondance & de la bonté de son herbe\footnote{N'est ce point le fulla du royaume de Naples? On le seme en octobre dans les chaumes qu'on brûle après, & on le récolte au mois de juin; Haller considere encore l'Hedifarum alpinum & l'H. obscurum. (R)}. Mais il est temps de dire un mot d'une plante dont la culture a été recommandée par tous les Botanistes-Agricoles, mais qui a toujours été négligée; c'est la coronilla varia (fig. 15); elle étoit cultivée au milieu du siecle passé au rapport de Parkinfon. Pourquoi ne pas remettre cette culture en vigueur; cette plante est commune le long de nos haies seches; l'abondance de ses graines promet une femaifon facile à l'agriculteur laborieux, & l'abondance de ses feuilles promet de le recompenfer de ses foins; le seul inconvénient de cette plante comme fourrage, est qu'elle est couchée, ce qui la rendroit difficile à faucher, la culture la redresseroit peut-être.
Les Lathirus latifolius (fig. 8.) & pratensis (fig. 52.) sont indiqués par Martyn, & je pense, avec raison, comme de bons fourrages dignes d'être effayés par les agriculteurs qui aiment à enrichir leur pays & la société de nouvelles branches d'industrie. Il indique encore Vicia cracca (fig. 117.) à cause de l'abondance de ses\setcounter{page}{189} graines & recommande Vicia sativa comme fourrage pour le bétail & les chevaux. Le Lotus Corniculatus est égal ou supérieur en bonté au trèfle (1)\footnote{La liste des légumineuses qui peuvent être utiles est plus étendue que celle que donne Martyn, & il l'auroit sans doute augmentée s'il eut fini son utile ouvrage. Bohadsch a conseillé le robinia pseudo-acacia pour le fourrage des bestiaux, & il a été cultivé en Allemagne. Haller recommande l'orobus luteus, l'astragalus pilosus, & vesicarius; tout le genre des astragales semble propre à la culture; mais j'y distingue surtout l'astragale à feuilles de réglisse, a. glycyphyllos; tout annonce une récolte abondante & un bon fourrage, en cultivant cette plante indigène; elle est vivace; elle pousse des tiges longues, nombreuses & ramenses; ses feuilles sont grandes & nombreuses, ses graines se forment abondamment & seroient faciles à recueillir. Les troupeaux paroissent aimer cette herbe, pourquoi donc la reléguer parmi les plantes inutiles.}.
Outre les graminées & les légumineuses, il est encore quelques autres plantes intéressantes par leurs usages économiques & ruraux; Martyn a fait entrer dans son ouvrage toutes celles qui sont très-communes dans les lieux cultivés; nous ne le suivrons point dans des détails qui n'offriroient aucun intérêt & nous n'en extrairons que quelques mots sur les plantes qui offrent des utilités nouvelles ou peu connues.
Le Plantago lanceolata (fig. 67) a été conseillé comme prairie sous le nom de Rib-grass. Le grand accroissement qu'il prend dans les\setcounter{page}{190} terrains humides sembloit indiquer sa culture; mais l'expérience n'a point eu le succès qu'on avoit annoncé; Haller vante l'alchemilla & lui attribue la bonté des prés alpins; Dickenfort l'a semée & recoltée, mais aucun animal n'a voulu manger de ce foin.
L'agrimonia eupatorium (fig. 37) est inutile aux troupeaux; mais elle est bonne sous forme de thé contre les hemorragies & les obstructions; les Canadiens employent sa racine contre la fievre brûlante. Hill la conseille pour la jaunisse. Si on la cueille au moment de la floraison elle donne une teinte nankin, & dans le temps de la mâturité ce jaune est foncé & de bon teint \footnote{Pennant rapporte qu'à Cannai dans les Isles Hébrides on se sert de l'écorce de saule & des racines de tormentille pour tanner, & qu'à Rum, on se sert de cette derniere contre la dissenterie.}.
Cichorium Intybus (fig. 144), ou la chicorée sauvage a été introduite dans la culture en 1788 par Arthur Young: en Lombardie elle passe pour augmenter le lait & la chair des troupeaux; elle offre une récolte abondante, soit qu'on la coupe à mesure que les besoins de la ferme l'exigent, soit qu'on l'établisse en coupe reglée; elle en fournit 3 pendant 3 ou 4 ans, & ensuite n'en donne plus que 2; la culture aggrandit les feuilles de cette plante.
Martyn décrit (fig. 10. II. 12.) trois variétés du chêne rouvre (Quercus robus) l'une\setcounter{page}{191} à feuilles pétiolées dès leur naissance ; la 2e. à feuilles aussi pétiolées mais très-printanières, & ayant des feuilles moins épaisses & d'un vert clair ; celle-ci est le vrai Robur des anciens. La 3e. variété a les feuilles sessiles ; l'agriculteur ne doit point négliger ces considérations peu importantes en apparence, puisqu'elles lui fourniront un moyen de reconnaître la qualité du bois ; la variété à feuilles sessiles a le bois meilleur, plus fort & plus compact que les chênes à feuilles pédiculées\footnote{Cette observation coincide avec celles que j'ai eu occasion de faire sur les chênes de notre pays. Il m'a paru que les chênes à feuilles sessiles font très-communs dans la plaine, tandis que ceux à feuilles pétiolées le font sur les montagnes ; j'avois soupçonné que peut-être cette absence de parenchime dans la partie inférieure de la nervure principale, tenoit à une diminution de la sève dans les terrains pierreux des montagnes ; la circonstance que leur bois est aussi moins compact, semble justifier cette idée. C'est au reste au printemps qu'on observe le plus facilement la différence entre ces variétés de chêne. (R)}.
Ici je termine cet extrait de la Flora Rustica qui comme on a pu en juger par ce que j'en ai dit, est un ouvrage utile à répandre parmi les agriculteurs qui aiment à mettre de la précision dans leurs idées, & qui ne craignent pas de tenter quelques essais pour augmenter la masse de nos richesses réelles. Il est fâcheux que cet ouvrage n'ait pas été achevé, étant aussi avancé qu'il l'étoit.
A. P. DECANDOLLE.
\setcounter{page}{97}
\chapter{ASTRONOMIE}
\section{ON THE DISCOVERY, &c. Découverte de quatre nouveaux Satellites au Georgium sidus (la planète d'Herschel). Annonce du mouvement retrograde de ceux déjà découverts & explication de la cause qui les fait disparaître à certaines distances de la Planete. Par W. HERSCHEL, &c. (Tiré des Trans. Phil. pour 1798. Part. I.)}
On croiroit, aux découvertes nombreuses & belles qui ont illustré le nom d'HERSCHEL, que cet Astronome, indépendamment des talens qui le distinguent comme observateur, jouit d'un ciel pur & d'un climat favorable : on feroit dans l'erreur. La Nature, qui lui donna beaucoup, mais qui ne cumule jamais ses dons sans mesure, lui a refusé ce dernier bienfait. Nous tenons de lui-même, que déduction faite des nuits d'été, des crépuscules plus ou moins longs, des clairs de lune, & de toutes les chances atmosphériques, il ne lui reste qu'environ quatre-vingt-dix heures d'observation proprement dite sur une année : comment fait-il donc pour employer un temps aussi court avec tant de profit pour la science? Voici son secret.
\setcounter{page}{98} Il épie avec une vigilance soutenue tous les instants favorables ; & ses instrumens étant très-variés, il choisit dans chaque occasion, ceux qui conviennent plus particulièrement à la circonstance atmosphérique dans laquelle il se rencontre. Il arrive à son télescope l'imagination calme ; il réunit en quelque sorte, tous ses sens, toutes ses forces, dans ses yeux seuls ; il cherche uniquement à voir, à décrire exactement ce qu'il voit. Il dicte à mesure ; Caroline Herschel sa sœur écrit fidèlement sous sa dictée ; elle enregistre les apparences, certaines, douteuses, & jusqu'aux soupçons vagues. Les instrumens, l'observateur, le secrétaire ne forment pour ainsi dire, pendant toute la durée d'une belle nuit, qu'une foule machine à perceptions, à laquelle rien n'échappe, & qui peut tout retracer.
Nous avons vu ces registres : ils forment des volumes, des trésors de faits qu'aucun préjugé, aucune prédilection n'a influencés, & sur lesquels le génie peut méditer ensuite dans les heures tranquilles : là reposent les germes des découvertes, jusqu'à-ce que d'heureux rapprochements leur donnent l'existence ; là, lorsqu'Herschel ne sera plus, lorsque le temps aura détruit ses instrumens admirables, les Astronomes en possession de ces recueils pourraient voir encore avec ses yeux, se former à son école, & faire, par des combinaisons qui\setcounter{page}{99} lui avoient échappé de son vivant, comme des découvertes posthumes, qui honoreroient encore sa mémoire.--Mais écoutons l'exposé qu'il fait lui-même de celles que nous venons d'annoncer.
"Occupé, comme je l'ai été depuis quelque temps, dit-it, à perfectionner mes tables des mouvemens des satellites du Georgium sidus, je fus dans le cas de calculer de nouveau toutes les observations que j'en avois faites. En parcourant la série entière de ces observations depuis l'année où ils furent découverts (1787) jusqu'au temps présent, je les trouvai tellement étendues, surtout dans une classe particulière, que je résolus de faire de celle-ci l'objet d'un examen attentif."
"Ces observations ont rapport à la découverte de quatre nouveaux satellites; à des soupçons sur l'existence d'un grand & d'un petit anneau appartenant à cette planète & situés à angles droits l'une de l'autre ; au degré de lumière & à la grosseur de ces satellites ; enfin à leur disparition lorsqu'ils se trouvent à une certaine distance de la planète."
"Je fus fort aidé dans ce travail par une suite de théorèmes faciles, que j'avois imaginés pour calculer toutes les particularités relatives aux mouvemens des satellites ; comme, par exemple; étant donné l'angle de position, trouver la longitude du satellite, ou vice versà ; étant donnés l'angle de position & la longi-\setcounter{page}{100} tude, trouver l'inclinaison de l'orbite, l'apogée, la plus grande élongation, &c. J'avois encore calculé des tables de réduction de la position du point de la plus grande élongation, & de la distance de l'apogée, ou de l'ouverture de l'ellipse : j'avois aussi trouvé moyen d'employer ce globe ordinaire d'une manière expéditive pour contrôler les calculs de ce genre. Je fis disparaître par ces procédés beaucoup de difficultés qui m'avoient précédemment embarrassé."
"J'étois en état, au moyen de ces tables & de ces théorèmes, d'examiner les observations qui pouvoient se rapporter à de nouveaux satellites; d'après la supposition que leurs orbites étaient dans le même plan que celles des deux satellites déjà connus, & que la direction de leur mouvement étoit aussi la même."
"Je dois annoncer à cette occasion, que le mouvement des satellites du Georgium sidus, est retrograde."
"C'est ici un exemple remarquable de la grande variété qui existe dans les mouvemens des corps célestes. On avoit remarqué jusques à présent, que toutes les planetes & tous les satellites du système solaire se mouvaient dans le même sens & selon l'ordre des signes : les rotations même, soit les mouvemens diurnes tant des planetes principales, que du soleil & de fix des planetes secondaires, s'exécutent dans la même direction; mais voici deux corps céleftes\setcounter{page}{101} affez confidérables, qui fe meuvent dans un fens retrograde."
"Je reviens à l'examen des obfervations mêlées, dont le réfultat eft affez important & fera l'objet de ce mémoire. Elles prouveront l'exiftence de quatre fattelites additionnels à notre nouvelle planete. Les obfervations qui tendent à établir l'exiftence d'anneaux autour de fon difque ne paroiffant pas affez pofitives, il faudra, ou rejeter en entier les foupçons conçus à cet égard, ou chercher feulement à les vérifier lorfqu'on aura en main des inftrumens plus puiffans. On verra auffi le phénomene remarquable de la difparition des fattelites, & la caufe de cette difparition fera indiquée."
"Je rapporterai d'abord les obfervations qui appuient çes conclufions ; j'y joindrai quelques courts argumens qui montreront que mes réfultats découlent immédiatement des faits obfervés."
"Pour être plus clair, je clafferai les obfervations fous trois différens chefs : je commencerai par çelles qui fe rapportent à la découverte des nouveaux fattelites."
"Je ne ferai que rapporter un grand nombre d'obfervations fur de prétendus fattelites que je trouvois enfuite être des étoiles, ou dont je ne pouvois déterminer l'efpece, faute d'un temps affez clair ; car, entrer dans la defcription détaillée de chacune de ces apparences,\setcounter{page}{102} ou retracer les figures destinées à les faire reconnaitre, ce feroit perdre du temps sans utilité pour l'objet qui nous occupe maintenant. Je dois avertir cependant qu'à l'égard des observations qui font ici simplement rapportées, on a pris à-peu-près les mêmes précautions que dans celles dont on lira le détail extrait en nature du journal lui-même. Les premieres seront données sous le titre de Rapports, les dernieres sous celui d'Observations."

\subsection{RECHERCHE DE NOUVEAUX SATELLITES \large{Rapport.}}
Sans entrer ici dans les détails que renferme le mémoire, nous dirons seulement que le premier de ces Rapports date du 6 février 1782. Le Dr. Herschel commença ce jour-là à soupçonner qu'une très-petite étoile voisine de la planete étoit un satellite; mais cette étoile ayant conservé sa position le 7 & le 8 février le soupçon fut détruit. Après avoir donné d'autres Rapports analogues, jusques à la date du 1 novembre 1784, l'auteur nous dit que son télescope ( alors sous la forme Newtonienne ) ne lui paroissant pas assez puissant pour ce genre de recherches, il les abandonna en partie. Mais il ajoute que le gain de lumiere qu'il fit en introduisant sa méthode particuliere d'observer, dans laquelle il a supprimé le second\setcounter{page}{103} miroir,\footnote{Voyez la description du Télescope d'Herschel, T. I, Sc. & Arts. (R)} lui donna bientôt les moyens de les reprendre avec plus de succès. Voici le Rapport du 11 janvier 1787.
"Janvier 11. 1787. On a observé trois satellites supposés; un premier, un second & un troisième. — Janvier 12, le premier & le second ne se sont plus retrouvés dans l'endroit où j'avais marqué leur place; mais le troisième y était encore & c'est, par conséquent, une étoile fixe."
Ces deux satellites étaient ceux-là même dont la découverte fut plus positivement constatée par des observations postérieures.
Tous les Rapports des années 1787—1789, au nombre de 16, ne renferment que des soupçons de l'existence d'un troisième & quatrième satellite; soupçons détruits à mesure, par les observations subséquentes.
Nous avons transcrit un Rapport, nous allons donner une Observation, tirée de celles que renferme le mémoire; on jugera de la méthode de l'observateur.
"Mars 26, 1794. 9 heures 35'. ( temps sidéral \footnote{Les astronomes Anglais emploient pour l'ordinaire le temps sidéral; c'est celui qui est indiqué par la révolution diurne de la Terre rapportée, non au Soleil, mais aux Étoiles fixes. Une horloge réglée sur le temps fidéral indique toujours la même heure lorfque la même Etoile paffe au méridien; & marque 12 heures lorfque l'un ou l'autre équinoxe y arrive. Elle avance tous les jours d'environ 3'. 56". fur le tems folaire moyen. (R)} avec l'oculaire qui grossit 480 fois,\setcounter{page}{104} Je vois beaucoup mieux le premier fatellite qu'avec l'oculaire 320. Je foupçonnois, avec celui-ci, un troisieme fatellite précifément au nord de la planete, un peu plus loin d'elle que le premier, & l'oculaire 480 vérifie prefque mon foupçon (voyez fig. 5 )."
"9 heures 44'. Avec l'oculaire qui groffit 600 fois je conferve le même foupçon, mais je ne puis me fatisfaire fur la réalité."
"11 h. 32'. Je vois actuellement très-bien le troisieme fatellite fuppofé. Il eft beaucoup plus petit que le premier & fur la même ligne que celui-ci & la planete; enforte que c'eft probablement une étoile fixe, puifque lorfque je l'ai vu la premiere fois, il précédoit le premier d'une quantité plus confidérable que celle que le mouvement plus rapide du premier fatellite pourroit expliquer. Si c'eft une étoile fixe, elle forme avec q r un triangle prefque rectan-gle, dont le petit côté eft 3 r; ou bien elle fe trouve prefque fur la même ligne avec q & n."
"NB. Les lignes font indiquées plus juftè dans la defcription qu'elles ne le font dans la figure; celle-ci n'eft deftinée qu'à défigner les étoiles dont il eft queftion."
\setcounter{page}{105}
"Mars 27. 8 heures 37' oculaire 320. La même petite étoile que j'ai observée la nuit dernière à 11 heures 32' a disparu de la place où je l'avois vue. D'après la lumière qu'elle avoit alors, comparée à r qui eft, ce foir, très-voisine de la planete & à peine visible, je fuis certain qu'elle devroit être affez brillante pour être aperçue immédiatement fi elle étoit dans le lieu défigné par ma description."
"10 heures 20': La planete eft fort éloignée de l'étoile r."
"11 h. 41'. J'ai entrevu plusieurs petites étoiles ou fatellites fuppofés: l'un d'eux dans un lieu qui s'accorde avec le troisieme fatellite de la nuit dernière (en fupposant qu'il eût fait route avec la planete) c'eft-à-dire qu'il eft un peu plus éloigné, & après le premier: un autre en avant du premier mais plus près: quelques autres au fud, à une bonne diftance. Mais je n'ai pu en voir aucune d'une manière fuivie; j'entrevoyois seulement par intervalles des points lucides."
Après avoir donné textuellement celles d'entre les pieces tirées de fes registres qui peuvent, par leurs rapprochements, faire découvrir ce qu'il cherche, l'auteur les discute, & voici comment il raifonne.
"D'après la comparaifon des Rapports d'un grand nombre de fatellites fuppofés, avec les Obfervations choifies & tranfcrites dans leur\setcounter{page}{106} entier, on doit voir avec évidence que la méthode de chercher les objets difficiles à appercevoir & de les désigner par des lignes & des angles, en y ajoutant toutes les circonstances particulières qui pourront les faire retrouver, que cette méthode, dis-je, a été pratiquée dans toute son étendue. Ainsi en garde contre les illusions, nous devons convenir que l'action d'entrevoir seulement une très-petite étoile, est déjà un argument assez fort en faveur de son existence. Ce que j'appelle ensuite vérifier son soupçon (ce que je fais d'ordinaire en employant un oculaire plus fort que celui avec lequel le soupçon a pris naissance) c'est obtenir une perception plus durable de l'objet en question; c'est-à-dire, le voir de manière à pouvoir fixer l'œil sur l'objet, le comparer à d'autres objets environnants, & établir ainsi d'une manière précise & satisfaisante sa position relative."
En discutant, d'après ces principes, les observations dont il a donné les détails, l'auteur établit l'existence 1°. d'un satellite intérieur, c'est-à-dire plus voisin de la planète principale que ne le sont l'un ou l'autre des deux satellites qu'il avoit découverts en 1787 : c'est sur-tout l'observation du 5 mars 1794 qui lui paroît concluante à cet égard; & celle du 27 du même mois achève de lever tous les doutes. 2°. D'un satellite intermédiaire, c'est-à-dire dont l'orbite est placée entre celles des deux anciens\setcounter{page}{107} fatellites. 3°. D'un fatellite extérieur à ceux-ci ; 4°. Enfin d'un fatellite le plus éloigné de tous ; découvert le 28 février 1794, & confirmé le 27 mars 1794, & le 28 mars 1797.
"Si j'ai autant tardé, dit le célebre Aftronome, à publier ces obfervations, ce n'eft pas que j'euffe la moindre incertitude fur l'existence de ces deux fatellites, mais c'étoit feulement dans l'efpérance de pouvoir donner en même temps quelques détails ultérieurs fur leurs révolutions périodiques. Il n'étoit point fatiffant pour moi d'annoncer un fatellite, à moins que je ne puffe en même temps marquer plus précisément la place où d'autres aftronomes pourroient le rencontrer ; 'mais' comme il s'eft écoulé maintenant plus de temps que je ne m'en étois accordé pour terminer la théorie de ces aftres fecondaires, j'ai cru devoir ne pas différer plus long-temps l'annonce de leur exiftence."
"Voici donc la disposition relative des quatre nouveaux & des deux anciens fatellites."
"1°. Le fatellite intérieur. 18 janvier 1790."
"2°. Le fecond ( le plus voifin de la planete entre les deux anciens ) 11 janvier 1787."
"3°. Le troifieme. Intermédiaire. 26 mars 1794."
"4°. Le quatrieme. Le plus éloigné des deux anciens. 11 janvier 1787."
"5°. Le cinquieme. Extérieur. 9 février 1790."
\setcounter{page}{108}
"6°. Le sixieme. Le plus éloigné de tous: 28 février 1794."
La seconde partie du Mémoire que nous avons sous les yeux est intitulée "Observations & Rapports relatifs à la découverte d'un ou plusieurs anneaux appartenant au Georgium sidus & à l'applatissement de ses régions polaires;" nous choisissons quelques-unes des Observations, dans lesquelles on trouvera le pour & le contre.
"Novembre 13, 1782. Télescope de 7 pieds; oculaire 460. Je ne vois aucun applatissement vers les régions polaires."
"Avril 8 1783. Je soupçonne un applatissement."
"Février 4, 1787. Télescope de 20 pieds; oculaire 300. Le disque bien terminé. Aucune apparence d'anneau. Crépuscule très-fort."
"Mars 4. Je commence à soupçonner que la planete n'est pas ronde. Lorsque je la vois distinctement elle paraît avoir deux pointes doubles & opposées. Peut-être un double anneau, c'est-à-dire deux anneaux à angles droits." &c.
Après avoir revu fréquemment cette apparence, ou quelque chose d'analogue, l'auteur découvre enfin le 26 février 1792 qu'elle est due à une illusion optique provenant d'une cause extérieure. Voici son observation finale à cet égard:
"Mars 5, 1792. Je viens d'observer le Georgium sidus avec un miroir nouvellement poli &\setcounter{page}{109} d'une figure excellente. La planète m'a paru bien terminée & sans aucun soupçon d'anneau. Je l'ai vue successivement avec des oculaires de 240, 300, 480, 600, 800, 1200, & 2400, tous ces foyers étoient très-distincts ; & je suis assez convaincu que le disque est aplati. La lune étoit assez voisine de la planète."
Il termine sa discussion sur cette classe d'observations par les réflexions suivantes.
"Sans entrer plus avant , dit-il, dans un sujet nécessairement enveloppé d'incertitude, j'ajouterai seulement, que l'observation du 26 mars paroît très-décisive contre l'existence d'un anneau. Lorsque je formai ce soupçon pour la premiere fois, je crus convenable de supposer que la position de cet anneau pouvoit être telle qu'il en devint presqu'invisible, & qu'en conséquence il ne falloit pas abandonner les observations, jusqu'à-ce qu'il se fut assez écoulé de temps pour qu'on put obtenir une position plus favorable à mesure que la planète s'éloigneroit de son nœud. Dix ans d'intervalle ont dû suffire à cet égard ; car, quelle qu'eût été la situation de l'anneau, pourvu qu'elle fut demeurée la même, on auroit dû le voir avec plus ou moins d'avantage. Accordant donc une grande confiance à l'observation du 5e. mars 1792 confirmée par celles que j'ai faites depuis, j'ose affirmer que la planète n'a aucun anneau qui ressemble le moins du monde à celui, ou plutôt à ceux de Saturne."
\setcounter{page}{110}
"L'aplatissement de la planète vers ses pôles paraît suffisamment prouvé par plusieurs observations. Le télescope de 7 pieds, celui de 10 pieds & celui de 20 pieds le confirment également; & la direction désignée dans l'observation du 26 février 1794 paraît s'accorder avec ce qu'on pourrait conclure par analogie tirée de la position de l'équateur de Saturne & de celui de Jupiter."
"Si l'on admet cet aplatissement, on peut en inférer sans hésiter, que le Georgium sidus a aussi une rotation sur son axe & que cette rotation est très-rapide."
La troisième partie du Mémoire porte le titre de "Rapports & Observations sur la lumiere & la grosseur relative des satellites du Georgium sidus & sur leur disparition à certaines distances de la planète."
Il résulte effectivement de ces extraits du Journal des Observations, que ces satellites sont tantôt visibles & tantôt invisibles, abstraction faite des circonstances atmosphériques qui peuvent avoir à cet égard de l'influence. Voici quelques-unes des réflexions de l'auteur sur ce sujet.
"On a vu d'après un grand nombre d'observations que toutes les étoiles très-petites perdent beaucoup de leur lustre lorsqu'elles se trouvent dans le voisinage de la planète. — On fait bien en général qu'une lumiere plus forte\setcounter{page}{111} semble éteindre une lumière plus faible. Mais il y a certaines circonstances qui accompagnent l'action de la lumière sur l'organe de la vue lorsque les objets sont à peine perceptibles; ces circonstances sont si remarquables qu'on ne doit point les passer sous silence."
"Après m'être accoutumé à suivre les satellites de Saturne & de Jupiter jusqu'au bord même de leur planete, jusqu'au point de mesurer le diametre apparent de l'un des satellites de Jupiter par son entrée sur le disque (1); j'espérois qu'une occasion semblable se présenteroit pour le Georgium sidus, non pas, à la vérité, pour mesurer les satellites; mais la planete elle-même, au moyen du passage du satellite sur son disque. Je me flattois aussi d'établir les époques des satellites, d'après leurs conjonctions & leurs oppositions, avec plus d'exactitude que je n'avois pû le faire en les observant dans d'autres points de leurs orbites; la cause qui m'a empêché de réussir à ces divers égards mérite d'être approfondie."
"Nous pouvons remarquer que les satellites disparoissent régulièrement lorsqu'après leur plus grande élongation ils se trouvent à une certaine distance de la planete."
Ici l'auteur discute quelques unes de ses observations pour rechercher quelle est la distance
\footnote{Phil. Trans. pour 1797. Part. II. p. 335.}\setcounter{page}{112} à laquelle cette disparition a lieu, & il trouve que le premier satellite ne s'aperçoit plus, pour l'ordinaire, lorsqu'il se trouve à 18" de distance apparente de la planète; & le second, à la distance d'environ 20 secondes. Dans des nuits très-belles & très-rares, le premier a été aperçu une fois à la distance de 13'',8 & le second à 17'',3. Mais dans aucun temps on ne les a vus plus près.
Pour expliquer ce fait, l'auteur rejette d'abord la supposition qui l'attribuerait à l'effet de quelque atmosphère dense qui appartiendrait à la planète & intercepterait la lumière des satellites; car cette lumière paraît se perdre également lorsque le satellite est dans la portion antérieure de son orbite, ou en deçà de cette prétendue atmosphère. Voici comment il raisonne ensuite sur ce qu'il croit être la véritable cause de la disparition.
"La lumière de Jupiter ou de Saturne, dit-il, à raison de son éclat, est répandue d'une manière presqu'uniforme dans un espace de plusieurs minutes autour de ces planètes. Leurs satellites aussi, ayant une lumière assez brillante & se mouvant dans une sphère assez complètement illuminée, ne peuvent pas être fort affectés par leurs diverses distances de leur planète principale. Ils ont beaucoup de lumière à perdre & n'en perdent comparativement que peu."
"Mais, au contraire, le Georgium sidus est\setcounter{page}{113} très-peu lumineux, & l'influence de sa faible lumière ne peut s'étendre autour de cette planète avec une certaine égalité. Cette circonstance nous permet de voir les objets les moins lumineux lors même qu'ils ne font qu'à une minute ou deux de distance de la planète. Ses satellites sont à-peu-près les objets les moins lumineux qu'on puisse appercevoir dans la voute céleste, enforte qu'ils ne peuvent supporter aucune diminution notable de leur lumière par l'effet du contraste avec un objet plus lumineux, sans disparoitre. Si donc la sphere d'illumination de notre nouvelle planète est limitée à 18 ou 20", nous pouvons très-bien expliquer pourquoi les satellites disparoissent lorsqu'ils arrivent à cette distance, car ils ont très-peu de lumière à perdre & ils la perdent d'une manière assez soudaine,"
"On peut se prévaloir des observations faites sur la distance à laquelle disparoissent les divers satellites, pour déterminer leur lumière respective. Le second satellite paroît, en général, plus brillant que le premier; mais comme le premier de ces deux disparoit d'ordinaire plus loin de la planète que ne le fait l'autre, nous pouvons admettre que le premier satellite est réellement plus brillant que le second. On ne verra gueres le premier des nouveaux satellites que vers les plus grandes élongations. Il ne paroît pas être fort inférieur en lustre aux deux\setcounter{page}{114} autres; & s'il existoit quelques autres satellites intérieurs on ne les verroit probablement pas par la même raison pour laquelle les habitans du Georgium fidus ne découvriroient peut-être jamais l'existence de notre Terre, de Vénus & de Mercure."
"Le second des nouveaux satellites, celui qu'on a nommé intermédiaire, est beaucoup plus petit que l'un ou l'autre des anciens satellites. Les deux extérieurs, soit le 5e. & le 6e. sont les plus petits de tous, & on doit les chercher surtout dans leurs plus grandes élongations."
L'auteur termine son Mémoire par une indication approximative des révolutions périodiques de tous les satellites. Elle est tirée des rapports observés entre leurs distances apparentes à la planète dans leurs plus grandes élongations. Il avertit soigneusement que ces déterminations ne font encore que des à-peu-près, jusqu'à ce que les observations aient été multipliées. Voici ces périodes.
\comment{table}
....Jours heur. min.
Le nouveau Satellite intérieur aux deux anciens. 1. 21. 25.
intermédiaire. . . . . 10. 23. 4.
extérieur le moins éloigné. 38. 1. 49.
le plus éloigné de tous. . 107. 16. 40.
\setcounter{page}{94}
\chapter{Correspondance}
\section{Aux Citoyens Rédacteurs de la Bibliothèque Britannique.}
Paris 17 Fructidor an 6.
Citoyens,
Dans la partie des essais du Comte de RUMFORD; que vous venez de publier dans votre intéressant Recueil, j'ai vu avec plaisir la note (pag. 108) dans laquelle vous rappelez quelques phénomènes qui démontrent une attraction évidemment Élective, & qui semblent par conséquent se refuser à l'explication que propose ce Savant de tous les effets apparents d'affinité chimique, par la seule combinaison des effets de la chaleur avec ceux des changements de pesanteur spécifique, sans avoir recours à l'action d'une attraction particulière.
Les adhésions de l'eau aux corps qu'elle mouille; du mercure à l'or, à l'étain, &c. font bien certainement, comme vous l'observez des exemples de ces attractions électives, & il ne paraît pas que le feu influe dans cette espèce d'affinité, puisqu'on ne remarque point de changement dans la température des corps entre lesquels elle s'exerce.
\setcounter{page}{95} Mais il s'en faut bien, à mon avis, que vous donniez à cet argument toute la force dont il est susceptible ; j'oserai même dire que vous me paraissez en détruire les conséquences les plus directes et les plus importantes lorsque vous mettez en opposition les corps gras comme ne s'attachant pas à l'eau, le fer comme ne s'attachant pas au mercure, lorsque vous supposez surtout répulsion entre ces substances.
J'ai publié, il y a vingt-cinq ans, dans le Journal Physique (T. I. pag. 172 & 460) & depuis dans les Élémens de chimie de Dijon, dans le Dictionnaire de chimie de l'Encyclopédie méthodique, article adhésion, des expériences qui prouvent non-seulement qu'il n'y a pas répulsion entre l'eau et l'huile, entre le mercure et le fer ; mais encore qu'il y a par le seul contact de ces substances une force attractive que l'on peut déterminer par des poids. Cette attraction est assez puissante pour produire l'ascension de l'eau entre deux lames de suif rapprochées parallèlement à une petite distance. J'ai fait voir que l'adhésion ne dépendait pas, comme on l'avoit soupçonné, de la pression de l'air; puisqu'elle varioit suivant les différens corps, les surfaces exposées à la pression étant semblables; puisqu'elle avoit lieu sous le récipient de la machine pneumatique comme dans l'air. Il résulté enfin de ces expériences que l'ordre des adhésions du mercure aux métaux qu'il dissout, est d'accord avec l'ordre de ses affinités déterminé par les précipitations de ces métaux les uns par les autres.
Je suis bien éloigné de contester au Comte de Rumford, l'influence, si manifesté en tant de circonstances, des changemens de température et de pesanteur spécifique sur les combinaisons chimiques; mais les phénoménes les plus familiers des dissolutions, et particulièrement les faits que je viens de rappeler me\setcounter{page}{96} semblent résister à ce que l'on considere ces changemens comme cause unique. Ce ne font réellement que des forces adjuvantes ou perturbatrices. La rupture d'équilibre, les compositions nouvelles, les échanges de principes s'opèrent toujours en vertu de l'attraction universelle, dont les effets sont naturellement modifiés par la figure des molécules et la distance qu'elle engendre; qui est souvent contrebalancée efficacement par la force d'agrégation ou de cohésion; qui s'accroît à raison de la quantité des points de contact, ou pour parler plus exactement, des surfaces susceptibles d'entrer dans la sphère de son activité; qui, à un certain degré d'accroissement, produit des adhésions plus ou moins fortes, mais superficielles et sans dissolution; qui, dans les degrés ultérieurs, détermine des affinités plus ou moins puissantes.
Si vous jugez de quelque importance de ramener l'attention de vos Lecteurs sur ce sujet, vous pouvez, Citoyens, faire de cette Lettre l'usage qu'il vous plaira.
Recevez l'assurance de la haute estime et des sentimens fraternels avec lesquels je suis
Votre Concitoyen
L. B. GUYTON.
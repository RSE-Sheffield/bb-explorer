\setcounter{page}{337}
\chapter{GÉOLOGIE}
\section{ON THE PRIMITIVE STATE OF THE GLOBE, &c. Sur l'état primitif du globe & la catastrophe qui lui a succédé. Par R. KIRWAN, Esq. des Soc. Roy. de Londres & d'Irlande. ( tiré des Transactions de l'Acad. d'Irlande, T. VI. ) \large{( Dernier extrait. )}}
A près avoir, dit l'auteur, établi dans l'essai précédent la confiance due à Moyse d'après des raisonnemens purement philosophiques & indépendans de toutes considérations théologiques, je ne me ferai aucun scrupule de le prendre pour guide dans le développement des circonstances de la plus terrible catastrophe à laquelle l'homme, tous les animaux & le globe lui-même aient été exposés depuis la création. Son témoignage est, à la vérité, confirmé par les traditions de plusieurs anciens peuples\footnote{Nous trouvons dans une très-curieuse Dissertation sur le Déluge, publiée en latin en 1768 par le C. Pigot, l'un des Professeurs actuels de l'Acad. de Genève, quelques détails sur ces diverses traditions ; on les a retrouvées en Chaldée ; en Syrie ; en Egypte ; en Perse ; dans l'Inde ; en Italie ; en Chine ; en Grèce ; en Scandinavie ; & jusques en Amérique. — Toutes s'accordent sur le fait principal, & elles ne different que dans les circonstances. (R)}\setcounter{page}{338}. — Mais je m'attache plus particulièrement aux preuves géologiques, & j'en trouve de très-concluantes.
Il cite d'abord, les coquillages trouvés au Pérou, par D. Ulloa à la hauteur de 14,220 pieds : & comme il a montré dans l'Effai précédent, que les poiffons n'ont commencé à exifter que lorfque les eaux furent defcendues à 8500 pieds feulement, au-dessus de leur niveau actuel, les coquillages adventifs trouvés à de fi grandes hauteurs, ont dû y être portés par quelque inondation, qui, pour atteindre jufques là n'a pu être partielle, mais a dû s'étendre fur tout le globe.
Enfuite ; on a trouvé dans les plaines de la Sibérie des os d'éléphant & de rhinocéros, & même la carcaffe entiere d'un de ces derniers animaux. Or, comme ils ne peuvent vivre dans un climat aussi froid, il faut qu'ils y aient été transportés par quelque violente inondation qui provenoit de régions d'une température plus douce, separées de la Sibérie par des montagnes de plus de 9000 pieds de hauteur, par dessus lefquelles les eaux ont dû se frayer un chemin.
\setcounter{page}{339} Enfin on trouve souvent entassés dans les mêmes lieux des coquillages qu'on fait appartenir à des régions très-distantes les unes des autres. Il faut donc que l'une au moins des espèces ainsi réunies ait été amenée par quelque violente inondation.
L'auteur considère ces trois classes de faits comme offrant les preuves géologiques les moins équivoques d'un déluge universel. Il croit que les autres faits du même genre qu'on cite en preuve peuvent avoir une origine différente. De ce nombre sont les os fossiles d'éléphants qu'on rencontre ici & là en Europe; les os & dents de baleines trouvés près de Maestricht, &c.
On a formé diverses hypothèses pour expliquer ceux d'entre ces faits qui sont les mieux constatés : les uns supposent que dans une époque plus ou moins ancienne, l'axe de la terre était parallèle à l'écliptique, ce qui produisait sur toute la surface du globe un printems perpétuel & une température supportable aux éléphants. Mais les astronomes ont démontré qu'il est impossible que ce parallélisme ait jamais existé. D'autres imaginent que les pôles changent continuellement de place & qu'ils ont pu se trouver anciennement là où est maintenant l'équateur, &c. Mais les mouvements bien connus de l'axe de la Terre, mouvements très-peu sensibles & qui ont leurs limites & leurs périodes, ne s'accordent point avec\setcounter{page}{340} Cette hypothese; d'ailleurs les pyramides d'Egypte, encore très-régulièrement orientées, montrent que depuis trois mille ans, les pôles n'ont point changé.
Vient ensuite. L'hypothese de Buffon, qui comme on fait, suppose la terre dans un état de fusion ignée, & d'incandescence pendant un terme plus ou moins long, & se refroidissant ensuite graduellement. Elle étoit ainsi plus chaude que celle ne l'est à présent, lorsque les éléphans pouvoient vivre en Sibérie. — "Si Buffon, dit l'auteur, eût suivi les sentiers de l'expérience, il auroit vu que ce qu'il appelle son verre primitif, le quartz, loin d'être une substance vitreuse, est infusible, & qu'au grand feu il perd sa transparence, & une portion de son poids: il auroit vu que le mica, qu'il prend pour du quartz exfolié, est un composé absolument différent du quartz, &c. &c. Et sans parler des autres difficultés auxquelles cette hypothese est sujette, elle n'explique pas même les os d'éléphans, & la carcasse de rhinocéros trouvés en Sibérie; car, en supposant que le climat de ces régions ait pu une fois convenir à ces animaux, leurs dépouilles trouvées dans le pays ne peuvent point prouver qu'ils y ont vécu: 1°. parce qu'éloignées comme elles le sont, de plusieurs centaines de milles de la mer, elles sont accompagnées de platites marines, qui sans doute ont été charriées avec elles; 2°. parce qu'on les\setcounter{page}{341} trouve en monceaux ; & qu'on ne peut pas mieux préfumer que ces animaux se donnaffent à cette ancienne époque des cimetières communs , qu'on ne l'observe actuellement dans l'Inde où rien d'analogue n'a été remarqué ; 3°. parce que le rhinocéros a été trouvé entier, & non putréfié ; ce qui n'auroit pû arriver s'il eût péri dans un climat chaud ; 4°. enfin, parce que dans le Groënland, à une latitude qui n'est pas très distante de celle où les os ont été trouvés en Sibérie ( 55°. ) on rencontre sur les montagnes, des ossemens de baleines & non d'éléphans. Il faut donc que le climat y fût alors assez froid pour les baleines ; c'est-à-dire , extrêmement différent de celui qui convient aux éléphans : donc, les animaux dont on a trouvé les restes en Sibérie ne peuvent pas y avoir vécu.
"La quatrieme hypothese est celle de Mr. Edouard King , qui a été fort développée & agrandie par Mr. Deluc : ce philosophe , justement célèbre est persuadé que les continens actuels étoient , avant le déluge , le fond ou le lit de l'ancien océan , & que le déluge ne fut que le résultat de la submersion des anciens continens , lesquels forment le fond actuel des mers. Il résulte de cette supposition que nos montagnes actuelles ont dû être formées dans l'océan antédiluvien , & que les coquillages marins peuvent se rencontrer sur leurs sommets les plus élevés."
\setcounter{page}{342} "Selon cette hypothese les anciens continens doivent avoir existé dans les régions maintenant occupées par les mers atlantique & pacifique. Comment expliquer alors les os d'éléphans & le rhinocéros trouvés en Sybérie? Ces parties du globe étoient le fond d'une mer; & les eaux en se précipitant vers les continens qui s'enfonçoient, ont dû y porter & non en rapporter, des dépouilles animales quelconques."
Indépendamment de ces objections tirées des faits géologiques, notre auteur attaque l'hypothese de Mr. Deluc par un côté auquel ce savant ne se feroit sans doute pas attendu. Il ne la trouve point assez conforme au récit de Moïse.
L'écrivain sacré indique nominativement deux causes de la catastrophe : une pluïe de quarante jours ; & l'éruption des eaux du grand abîme. Il nous dit que les eaux s'élevèrent sur les continens pendant un certain nombre de jours ; qu'elles deméurèrent stagnantes pendant une seconde période, & qu'ensuite elles reprirent leur ancien lit. Puisqu'il y a eu dans leur marche, mouvement progressif & retraite, n'est-il pas évident que ces expressions ne sont applicables qu'à une portion identique & non à deux portions différentes du globe ?
"Mr. Deluc, dit notre auteur, répond qu'il est annoncé (13 verset chap. 6 de la Genèse) que la terre feroit détruite. — Mais, Moïse n'a pu entendre par cette expression une destruction\setcounter{page}{343} qui la fit difparoître à toujours, & il ajoute que les eaux s'élevèrent de 15 coudées au dessus des plus hautes montagnes; mais comme il n'a parlé nulle part des montagnes antédiluviennes & qu'il a fait mention, au contraire dans plufieurs endroits, de celles qui font postérieures au déluge, il eft évident que c'eft à celles ci que cette circonftance de fa narration fe rapporte."
La branche d'olivier eft difficile à expliquer dans l'hypothefe de Deluc. Ce favant la fait croître fur une des Ifles des mers antédiluviennes; mais alors, comment Noé auroit il pu confiderer l'apparition de ce rameau comme étant un figne de la diminution des eaux? S'il y eut alors des Ifles qui demeurerent intactes, leurs habitans auroient pu échapper à la catastrophe; fuppofition directement contraire au texte. La colombe revint, parce qu'elle ne trouva pas où se pofer; il n'y avoit donc pas d'Ifles; & le corbeau ne revint pas, parce qu'il dut trouver affez de carcaffes flottantes pour se nourrir.
Après avoir feulement nommé, en paffant, Burnet Woodward & Whifton, comme inventeurs de fystêmes géologiques qui ont été re futés suffisamment & à plusieurs reprises\footnote{Burnet fait de la Terre antédiluvienne une écorce plane qui recouvre l'abyme; cette écorce se rompt, se déprime inégalement, elle comprime les eaux; celles-ci viennent l'inonder partiellement et se retirent ensuite. — Woodward délaye, dissout même dans l'océan boueux produit par les pluies et l'éruption de l'abyme, toute l'enveloppe solide terrestre; ces matières se déposent ensuite et se stratifient pêle-mêle. — Whiston cherche dans les Cieux et dans l'abyme à la fois, la cause du bouleversement. Il fait approcher une comète, dont l'attraction produit des marées énormes; l'eau de l'abyme, fortement attirée, rompt l'enveloppe qui la recouvre et se répand partout avec furie; la queue même de la comète fait momentanément partie de l'atmosphère terrestre, et occasionne la pluie de quarante jours, &c. (R)}.
\setcounter{page}{344} L'auteur en vient au détail de cette grand révolution, pris dans son sens simple et littéral. "C'est le seul, ajoute-t-il, qui paroisse parfaitement d'accord avec tous les faits maintenant observés. Moïse attribue nettement la catastrophe à une cause surnaturelle. — Nous devons donc considérer le Déluge comme une inondation miraculeuse, provenant à-la-fois et de l'atmosphère et des eaux souterraines; et si les eaux, situées en parties au-dessus et en partie au-dessous de l'écorce du globe, ont pu suffire une fois à couvrir jusqu'aux plus hautes montagnes, ainsi que je l'ai montré dans mon premier Essai, elles peuvent avoir recouvert une seconde fois ces mêmes sommités, en supposant qu'une impulsion surnaturelle ait fait ressortir cette masse de liquide hors de ses réservoirs souterrains." Keil,\setcounter{page}{345} d'après la supposition que la profondeur moyenne de l'Océan n'excédait pas un quart de mille, avoit calculé qu'il faudroit vingt-huit Océans pour recouvrir les plus hautes montagnes ; mais, dit l'auteur, Mr. De la Place calculant la profondeur moyenne de la mer, non d'après des sondes partielles & imparfaites ; mais d'après ce qu'exige la théorie newtonienne des marées, démontre qu'une profondeur mobile de quatre lieues ne peut concilier cette théorie avec les phénomènes. Il n'en faut pas tant, à beaucoup près, pour que cette masse d'eau suffise à l'explication mofaïque du déluge.
" Après avoir ainsi établi la possibilité & la réalité de cette catastrophe, je vais essayer de reprendre son origine, ses progrès, & ses conséquences, qui sont encore sous nos yeux. Je crois que l'inondation vint du grand océan méridional, au-delà de l'équateur, & qu'elle s'étendit sur l'hémisphère septentrional." Voici les faits sur lesquels l'auteur fonde cette opinion :
1°. L'Océan méridional renferme la plus grande masse d'eau qu'il y ait actuellement à la surface du globe.
2°. On trouve bien dans les latitudes septentrionales, entre 45 & 55°, des dépouilles terrestres & marines des régions méridionales ; mais on ne rencontre pas dans les latitudes australes des débris d'animaux ou de végétaux qui aient appartenu aux régions boréales. Ici\setcounter{page}{346} l'auteur cite les faits rapportés par divers Naturalistes, & qui tous appuient fon affertion.
3°. On trouve encore dans beaucoup d'endroits des traces d'un choc ou d'une impression violente provenant du midi. Ici s'appliquent les observations de Patrin fur les chaînes de la Daourie; de Steller, fur celles du Kamschatka; de Storr, de Hæpfner & de Deauffure fur les chaînes voisines des Alpes; enfin celles de Lafius fur les montagnes du Hartz.
4°. On peut remarquer que tous les grands continens se terminent par un angle aigu du côté du midi. L'Amérique, l'Afrique, la Presqu'isle d'Asie, l'extrémité méridionale de la Nouvelle Hollande, présentent cette figure, résultat probable de l'action violente des eaux venant du midi, action à laquelle rien que de grandes chaînes de montagnes n'a pu résister.
C'est au milieu du vaste continent de l'Asie, vers les sources du Ganges & du Burrampooter que notre auteur place la première demeure de l'homme. Delà, à mesure que la température devint plus froide, la population descendit dans les plaines de l'Inde.
Il cherche à se former une idée des effets d'une pluie de quarante jours, en les rapprochant des observations recueillies par les physiciens modernes sur certains bouleversements produits par des pluies fortes ou longues. On a vu une montagne entière, de la province de\setcounter{page}{347} Vermeland en Suède s'écrouler après une à-verse d'une nuit; &c. Dans l'entassement de matieres de toute espece qui dût être le premier résultat du déluge, les substances animales durent fournir l'acide phosphorique qui minéralise quelquefois des métaux, on le trouve combiné avec la terre calcaire sous la forme de phosphate de chaux.
Suivons maintenant, avec une mappemonde ou un globe terrestre sous les yeux, la marche des eaux méridionales telle que notre auteur l'indique à grands traits.
"Le complément de la catastrophe fut sans doute, dit-il, l'éruption & l'invasion des eaux du grand abime, ainsi que le dit expressément l'Historien sacré. Ce réservoir est probablement, comme on l'a indiqué tout-à-l'heure, l'immense étendue de mer comprise entre les Philippines & le continent d'Asie d'un côté, l'Amérique méridionale de l'autre, & delà jusqu'au Pole antarctique. Cet océan est contigu à un autre non moins vaste, compris entre la côte orientale de l'Amérique méridionale d'une part, & la Nouvelle Hollande de l'autre, & se terminant au même pole. La premiere de ces deux masses d'eau, chassée du midi au nord avec une impétuosité irrésistible, balaya le continent qui unissoit probablement alors l'Asie & l'Amériques; il n'en reste de traces que quelques Isles clairsemées, jusques vers les 40 degrés de latitude\setcounter{page}{348} nord, où il paroît que les hautes montagnes de la Chine & de la Tartarie d'un côté & celles de la côte Américaine située vis-à-vis, diminuèrent sa violence : delà le courant se répandit à droite & à gauche sur les contrées collatérales ; la partie contenue par les grandes chaînes de Tartarie forma, en entraînant toute la partie végétale du sol, le désert de Coby ; le torrent intérieur ou moyen, tendoit droit au pôle, mais resserré encore par les montagnes contigues & élevées de la Sibérie orientale & de l'Amérique occidentale, il dut s'élever à une hauteur & acquérir une pression telles que rien ne put y résister & que les sommets des montagnes en furent bouleversés, ainsi que l'ont remarqué Patrin & Steller : ces eaux, chargées des dépouilles animales & végétales des contrées qu'elles venoient de ravager au Sud, les déposèrent dans les immenses plaines, doucement inclinées, de la Sibérie orientale. Telle est l'origine des ossements provenant d'éléphants ou de rhinocéros qu'on trouve dans les plaines ou les collines graveleuses & marneuses dans le Nord-ouest de la Sibérie. C'est du moins l'origine que leur assigne Pallas."
"Si, revenant vers le midi, nous contemplons les effets de l'arrivée des eaux contre l'Inde & l'Arabie, nous les verrons, là où les côtes n'étoient pas garanties par des chaînes de montagnes, creuser les golphes de Nankin,\setcounter{page}{349} de Tonkin & de Siam, le vaste golphe de Bengale, le golphe Persique & la mer rouge. On ne peut présumer que les foibles courans qu'on observe encore aux environs des promontoires qui séparent ces golphes, aient pu les former; mais la force d'une masse d'eau telle qu'on vient de la désigner paroît seule être en proportion avec ces effets prodigieux, & uniformes dans leur principal caractère."
"L'impulsion du courant paroît avoir été surtout dirigée du Sud au nord entre les 110 & les 200 degrés de longitude orientale, à compter du méridien de Londres. Son action paroît avoir été plus foible dans les régions plus occidentales. Je soupçonne que les plaines de l'Inde furent moins ravagées que le reste, ou peut-être leur fertilité subséquente est-elle due au grand nombre de rivières qui arrosent cet heureux pays. Il n'en est pas ainsi de l'Arabie : la base solide qui résista à l'inondation céda toute la croûte de terre végétale qui la recouvroit, cette contrée est encore maintenant déserte, & les régions intérieures plus montueuses ayant intercepté & recueilli ce sol enlevé, font encore de nos jours, renommées pour leur fertilité. On peut attribuer aussi à l'enlèvement du terreau antédiluvien les vastes déserts sablonneux de l'Afrique & la stérilité de la plupart des plaines de la Perse."
"Nous ne suivrons pas l'auteur dans tout le\setcounter{page}{350} déploiement de son système sur le continent de l'Asie. Nous remarquerons seulement qu'il attribue les grands lacs méditerranées tels que la mer Caspienne, & le Pont Euxin aux tournoiemens occasionnés dans le courant principal par les obstacles que lui opposoient les chaines de l'Asie septentrionale. Le mélange des coquillages & des animaux Indiens & Américains qu'on trouve en Europe & au nord de l'Afrique, comme à Fez, par exemple, appuie éminemment l'ensemble de ce système."
"La rencontre de ces masses énormes de liquide qui chemioient dans des directions opposées, dut secouer violemment certaines parties de la voûte qui les supportoit & peut-être les rompre. C'est à un effet de ce genre que l'auteur attribue la formation de la mer Atlantique depuis les 20 degrés de latitude Sud jusqu'au pôle arctique; "la seule inspection d'une carte suffit, dit-il, à montrer que cette portion considérable de la surface du globe a été creusée par l'impression des eaux; la protubérance qui s'étend du Cap Frio à la rivière des Amazones, correspond avec l'excavation de la côte d'Afrique depuis la rivière de Congo jusqu'au Cap Palmas ; & d'autre part, la faillie du continent Africain entre le détroit de Gibraltar & le Cap Palmas, répond à l'immense golphe entre New-York & le Cap St. Roque."
Les parages voisins de la grande rupture,\setcounter{page}{351} durent s'en ressentir plus ou moins, & les déchiremens latéraux formerent, ou préparerent, des Isles, & les côtes escarpées des faces occidentales d'Irlande, d'Ecosse, & de Norwège. "Il est possible, ajoute l'auteur, que les fissures des masses basaltiques de ces mêmes côtes, datent de l'époque de cette secousse violente."
C'est avec les débris de matières métalliques, bitumineuses, charbonneuses, &c. charriées pêle mêle par les eaux, & aux combinaisons nouvelles & aux dépôts, qui se formerent dans le liquide, que l'auteur forme certaines couches de houille, & les basaltes, substances entre lesquelles l'excellent observateur Werner a trouvé des rapports très-marqués. Les fluides élastiques qui s'en dégageoient laissent des cavités qui furent ensuite remplies par des infiltrations auxquelles on doit les calcédoines, les zéolites, les olivines, les basaltines, & des substances spathiques diverses, qu'on trouve encore fréquemment mêlées d'asphalte & de bitume. Ce dépôt fut la dernière scène de la grande catastrophe, c'est pourquoi les coquillages étant déjà stratifiés à cette époque, on n'en trouve point dans ces basaltes. Certaines couches de grès & d'argile datent du même temps.
Au demeurant, l'auteur ne donne qu'en passant & sans plus d'importance qu'elle ne peut en mériter réellement, cette hypothèse sur la formation des basaltes. Il reprend ensuite le\setcounter{page}{352} récit de Moïse & répond à quelques objections des critiques ; & en particulier à celle qui repose fur la difficulté que, dût éprouver Noé à raffembler toutes les especes d’animaux dans l’arche. Il croit d’après le rapprochement du v. 19 ; Chap. VI de la Genèse, avec le v. 30 Chap. IJ du même livre, qu’il ne s’agissoit que des animaux herbivores, & des granivores; c’est-à-dire de ceux qui font plus particulièrement utiles à l’homme. Il croit que les autres, furent créés dans une époque fubféquente; & ceux-là surtout, qui appartiennent exclusivement à l’Amérique, & aux Zones torride & glaciale.
L’atmosphere elle-même dût éprouver une modification particuliere par les effets du déluge. La proportion des végétaux qui purifient l’air, aux animaux qui le détériorent, étoit beaucoup plus grande dans les temps qui fuivirent de près la création qu’elle ne le fut ensuite: delà probablement la longévité des antédiluviens. L’effet inverse eut lieu après le déluge, & les innombrables décompositions de matieres animales absorberent une quantité très-considérable d’oxigene, ce qui amena peu-à-peu l’atmosphere aux proportions actuelles. La race humaine habita probablement long-temps les lieux élevés, pour éviter l’effet des exhalaifons nuisibles; ensuite, les descendans de Noé s’établirent dans le voisinage de la Chine; pays dont les traditions s’accordent avec ce titre d’ancienneté.
\setcounter{page}{353} Dans son TROISIEME ESSAI, l'auteur s'occupe des événemens postérieurs au déluge ; mais qui avoient été jusqu'à un certain point, préparés par cette catastrophe. Il met dans ce nombre la séparation totale des continens d'Amérique & d'Asie ; le rétréciffement de la Baltique; la séparation de la mer Cafpienne & du Pont Euxin & la jonction de cette dernière mer avec la méditerranée; enfin la séparation de l'Irlande avec la Grande-Bretagne, & l'ouverture complète du Canal de la Manche. Il examine ce que la tradition nous apprend, & ce que les apparences nous acheminent à préfumer fur ces événemens.
On n'a point de tradition fur la séparation de l'Afie avec l'Amérique; l'auteur conjecture que ces continens étoient réunis parce qu'il paroît que leurs habitans ont eu des communications réciproques. Il croit que la séparation finale fut l'effet d'éruptions volcaniques.
Il préfume le rétrécissement fucceffif de la Baltique d'après les plaines de plufieurs centaines de milles qu'on trouve dans la Ruffie méridionale depuis Pétersbourg jusqu'à Pultawa; ces plaines font encore marécageufes & couvertes de coquillages pétrifiés. Cette mer eft plutôt un lac formé par la décharge des eaux d'un baffin immenfe; fon eau n'eft que peu falée & elle n'acquiert cette qualité que par fa communication avec la mer d'Allemagne.
\setcounter{page}{354} Pallas a, sinon démontré, du moins rendue extrêmement probable l'ancienne communication qui existoit entre la mer Caspienne, le lac d'Aral & la mer Noire avant que le Bosphore de Thrace fût ouvert. Il conclut d'un assez grand nombre d'observations que le niveau primitif de la mer Caspienne étoit de quatre-vingt-dix pieds au-dessus de son niveau actuel, & qu'elle communiquoit ainsi avec le Pont Euxin par la mer d'Azoph. Ce n'est qu'environ 1800 ans avant notre Ere que la séparation a eu lieu, ainsi que le montre Mr. Forster dans un savant Mémoire inséré dans le magasin de Goettingue pour 1780.
"Buffon ne donne point les motifs de son opinion que la rupture de l'Isthme qui séparoit cette mer de la méditerranée a dû précéder le déluge," aucune tradition dit l'auteur, n'appuie cette hypothèse. La rupture de l'Isthme fut probablement soudaine & totale, & selon toute apparence, l'effet d'un tremblement de terre. Considérons, pour découvrir cet effet, l'état antérieur de la méditerranée."
"Avant son union avec la mer Noire & l'océan, elle formoit probablement un bassin beaucoup plus étroit & moins profond que ne l'est son bassin actuel ; car quoiqu'elle reçût plusieurs rivieres considérables telles que le Nil, le Rhône & le Pô, puisque même à présent, l'évaporation suffit à maintenir son niveau quoiqu'elle\setcounter{page}{355} reçoive les eaux de la mer noire & de l'océan, nous pouvons bien fuppofer que lorfqu'elle étoit isolée les eaux étoient beaucoup plus baffes. Lors donc que par la rupture de l'Ifthme de Thrace d'une part, & de celui qui joignoit Ceuta & Gibraltar de l'autre, les eaux fe jetèrent avec violence dans ce baffin; la preffion immenfe qui en réfulta fit crever fon fond, & dans ce tumulte, les Isles de Sicile, de Sardaigne, de Corfe, & celles de l'Archipel furent féparées du continent adjacent, & l'Italie prit la forme allongée qu'on lui voit aujourd'hui. Les rivages voifins de France & d'Efpagne, & furtout ceux d'Afrique, de Grèce & d'Afie furent inondés affez avant, & delà les imprégnations falines qu'on trouve encore dans les parties adjacentes de l'Afrique."
L'auteur nous paroît moins heureux dans fon explication des volcans fou-marins de la méditerranée. Il conjecture du filence d'Homère, que l'apparition de l'Etna fut poftérieure au temps où vécut ce Poëte. Il rappelle d'anciennes traditions fur la féparation de la Sicile & de l'Italie.\footnote{Zancle quoque juncta fuiffe Dicitur Italiæ, donec confinia pontus Abftulit & media tellurem reppulit unda. Ovid. Métam. ....Trinacria quondam. Italiæ pars una suit, sed pontus & æstus. Mutavere situm, rupit confinia nereus. Victor, & abscisso interluit æquore montes. Claudian. de rapt. Proserp.}\setcounter{page}{356}
"Les côtes escarpées qu'on observe entre Gênes & Livourne, & que Ferber a décrites dans sa 22e. Lettre sont l'effet de la rupture des couches; car les marées étant presque nulles dans cette mer, on ne peut leur attribuer rien de pareil. La rapidité du Rhône & de la plupart des rivieres qui se jettent dans cette mer, du côté d'Europe, indique la grande inclinaison des pays intérieurs vers cette même mer, effet naturel de la dépression des terres qui leur servoient d'appui. On observe aussi dans les montagnes de Suisse les vestiges d'un choc venant du Sud-ouest; mais j'en abandonne le développement aux excellens géologues qu'on trouve en nombre dans ce pays."
Ce fut à l'époque de l'ouverture du détroit des Dardanelles que la mer Noire, baissant de niveau, se trouva séparée de la mer Caspienne, le niveau de celle-ci qui ne reçoit que deux rivieres considérables (le Volga & l'Ural) baissa ensuite par l'évaporation, & laissa à sec les déserts salés qu'on trouve encore aujourd'hui dans son voisinage.
La Grande-Bretagne ne fut séparée du continent, que long-temps après le Déluge; & le Canal de St. George est encore de plus fraîche\setcounter{page}{357} date. C'est à un tremblement de terre que l'auteur attribue la rupture de l'Isthme entre Calais & Douvres; la brèche fut ensuite agrandie par l'effet des marées & des courans. L'Irlande, protégée par l'Écosse contre la violence du choc septentrional, ne s'est séparée d'elle que graduellement & dans des temps postérieurs à sa séparation d'avec l'Angleterre, séparation qui date probablement du déluge, & qui fut l'effet d'un choc venant du midi.
Tous ces changemens se sont opérés au moins trente-six siècles en arrière du temps où nous sommes, & l'auteur ne voit pas de raison de supposer que le niveau moyen de l'océan ait changé depuis lors. Mais il n'en est pas de même du niveau des continens.
Deux causes ont constamment tendu à l'abaisser. La première est la diminution des eaux dont ils étoient primitivement imprégnés; la seconde est l'éboulement, ou le glissement des couches argileuses inférieures : sans parler des causes accidentelles comme les tremblemens de terre & les inondations.
"Après le déluge, nous dit l'auteur, la terre sur laquelle les eaux avoient séjourné 150 jours & d'où elles se retirerent lentement, dut être bien profondément humectée. La végétation en fut d'autant plus active; & on fait que vingt siècles en arrière de nous la plus grande partie de l'Europe étoit couverte de forêts; cette circonstance\setcounter{page}{358} devoit contribuer à entretenir l'humidité & produire une infinité de ruisseaux qui ne tarissoient point & charrioient continuellement dans de grandes rivieres, & delà à la mer, les particules du sol délayées par leurs eaux. Nous voyons de notre temps, les plus hautes sommités attaquées par l'action combinée de l'air & de l'eau se détruire graduellement ; & sans doute ces sommités cédent depuis long-temps à cette influence & s'abaissent d'autant. A mesure que la population s'est accrue sur la surface du globe les forêts ont disparu, & le sol, en se desséchant, a du se contracter. Il est évident à l'aspect des collines horizontales qui bordent le lit de la plupart des grandes rivieres, que la masse d'eau qu'elles charrioient étoit beaucoup plus considérable autrefois qu'elle ne l'est actuellement."
L'auteur cite quelques faits modernes qui appuyent cette partie de son système ; entr'autres la descente ou le glissement d'une partie considérable de la montagne de Ziegenberg en Bohême. Elle descendit de 38 brasses, jusques à l'Elbe, & les arbres qui la couvroient demeurèrent en partie debout & en partie inclinés. Ailleurs au contraire la surface du sol s'est élevée par les dépôts des pluies, qui ont même enseveli quelques monumens de l'antiquité. Les ravages de la guerre ont souvent aidé la nature dans ce genre de travail.
\setcounter{page}{359} L'influence des volcans pour changer la face du globe paroît être beaucoup plus circonscrite, selon notre auteur, que plusieurs écrivains modernes ne l'ont affirmé. Il croit qu'un très-petit nombre de montagnes leur doivent l'existence, & qu'en particulier, ni le Vésuve ni l'Etna ne font des produits de volcans; il en atteste leurs bases, qui sont des pierres Neptuniennes ou formées par les eaux. — Écoutons comment il critique la disposition de certains observateurs à tout volcaniser.
"Il n'est aucune illusion contre laquelle les physiciens se soient tenus moins en garde que contre celle qui multiplie les volcans à mesure que l'imagination s'échauffe : sans doute l'étonnement que cause les magnifiques phénomènes de leurs éruptions a influé jusques sur l'entendement de quelques-uns des spectateurs. Chaque pierre noirâtre qu'ils rencontrent ensuite est une lave ; l'excellent observateur Sir W. Hamilton n'est point hors de l'influence de cet enthousiasme ; il nous dit dans une lettre à Sir John Pringle, (mai 1776) que "partout où l'on trouve des colonnes basaltiques semblables à celles du pavé des Géans en Irlande, on ne peut douter qu'il n'ait existé des volcans, car ces colonnes sont de la lave pure." — Je crois pourtant qu'aujourd'hui personne ne prétendra que l'origine volcanique de ces colonnes soit fait à l'abri du doute\footnote{C'est là une grande question, & l'auteur, s'il était appelé au recensement des opinions sur l'origine des basaltes, trouverait encore peut-être plus de volcanistes qu'il ne l'imagine. (R)}.
\setcounter{page}{360} Il nous dit que le Vésuve & l'Etna ont été formés par une série d'éruptions volcaniques; quoiqu'il n'y ait aucune preuve du fait pour la première de ces deux montagnes, & qu'on puisse démontrer que la seconde existait comme montagne avant que de devenir un volcan. Le Père della Torre, qui a donné une bonne description du Vésuve soutient que sa charpente n'est point volcanique, mais une simple continuation des Apennins: ce nombre de pierres neptuniennes qu'il vomit dans ses éruptions, & que Gioeni a rassemblées dans sa lithographie du Vésuve, confirme cette opinion; sans doute, les couches calcaires sont recouvertes d'une croûte de lave très-profonde; mais il n'est point prouvé que la masse entière du Vésuve en soit composée. On dit qu'en creusant un puits au bord de la mer, on a trouvé des couches de lave à une assez grande profondeur; mais qui fait si on n'a pas pris le hornblende, soit compacte, soit schisteux, pour la lave elle-même qui doit son origine à ce genre de pierre?
"Quant à l'Etna, il ne peut y avoir de doute. Dolomieu a trouvé des amas énormes de coquillages marins sur ses flancs, au Nord-est, environ 2000 pieds au-dessus du niveau de\setcounter{page}{361} KIRWAN, SUR L'ÉTAT PRIM. DU GLOBE, &c. 365, de la mer. Il en conclut avec raison que ce volcan existoit comme montagne avant que la mer l'eût mis à découvert; il ajoute, qu'à la hauteur d'environ 2400 pieds, on trouve des bancs réguliers de glaise grisâtre remplie de coquillages marins. Ces couches doivent avoir été déposées à l'époque où la montagne se formoit dans le sein de la mer: on y trouve aussi dit-il de la lave prismatique; mais le mot de lave, surtout avec l'épithète de prismatique, n'en impose plus maintenant à personne."
"Il affirme de plus, que dans certaines parties de cette montagne on trouve des bancs calcaires, sous la lave. D'autre part, le Comte de Borch dans ses Lettres sur la Sicile & sur Malte, nous dit, que la pierre fondamentale de l'Etna est un granit mêlé de jaspe, substances qui n'ont pas de rapport avec des laves; il ajoute qu'on y trouve des mines de plomb, & de cuivre en abondance, & ces mines ne se rencontrent jamais dans les laves, sauf quelques fragments accidentellement enveloppés. Ce dernier géologue prétend que l'Etna a au moins 8000 ans d'antiquité; il fonde cette opinion sur les couches de terreau qu'il a observées entre diverses couches de lave. Cependant Dolomieu nous dit expressément (Isles Poncés, p. 472) qu'il n'existe rien de pareil entre les couches de cette substance. Cette observation ne permet pas d'admettre les conséquences du calcul de\setcounter{page}{362} Borch ; & quand on trouvoit de la terre végétale entre les couches de lave, on ne pourroit rien en conclure sur leur date, car il y a de ces couches qui font beaucoup plus promptement décomposées & plus fertiles que d'autres : ainsi le Chevalier Gioanni trouva en 1787 des laves vomies en 1766, déjà en état de végétation, tandis que d'autres beaucoup plus anciennes n'en montroient aucun indice (Dolomiieu, Ponçes p. 493) : On sait d'ailleurs que les couches composées de cendres volcaniques & de pierres ponces végétent beaucoup plus promptement que les autres."
Ces détails n'entroient pas dans le plan primitif de l'auteur ; il se justifie de les avoir introduits, en observant que l'opinion qui fait remonter jusqu'à des temps très-reculés, l'état du globe tel à-peu-près que nous le voyons, se trouveroit en opposition avec les récits de Moïse, récits dont il a cherché à démontrer l'authenticité. Par ses savans rapprochemens & les concordances frappantes qui en résultent, il nous semble avoir atteint le but qu'il s'étoit proposé ; & là aussi il s'arrête.
Mais en lui accordant que la cause du Déluge fut surnaturelle, nous regrettons que son génie n'ait pas cherché à faire un pas de plus, & à deviner quel fut le mode d'action de cette cause. Les physiciens occupés de ces hautes recherches ont toujours un fil qui dirige & contient leur\setcounter{page}{363} imagination, d'ailleurs facilement égarée; c'est la loi du minimum d'action qui s'observe dans tous les grands phénomènes de la nature; c'est donc toujours à la plus prochaine, à la plus simple qu'il faut recourir quand on a l'ambition de remonter aux causes. Quel est le plus simple des moyens que pouvoit employer l'auteur de la nature pour produire cette submersion dont nous venons d'étudier les circonstances ? Le philosophe qui aura répondu juste à cette question aura probablement deviné le secret que l'Historien ignora; ou qu'il ne crut pas ses contemporains dignes d'apprendre. Voici la solution que nous donne l'ingénieux auteur de la dissertation que nous avons citée \footnote{P. 337. (nott.)}.
Il suppose qu'à l'époque du déluge, le mouvement de rotation de la terre sur son axe fut retardé dans une certaine proportion: tous les faits s'expliquent alors de la manière la plus simple.
On fait que la force centrifuge, née de la rotation, donne à la terre la forme d'un sphéroïde relevé sous l'équateur; & que la différence du rayon de l'équateur au demi axe polaire, ou la taille de la région équatoriale en vertu de la force centrifuge, est de plus de 10000 toises.
Si l'on suppose que la vitesse de rotation diminue aussi & la\setcounter{page}{364} gravité augmente proportionnellement. Pour nous faire une idée de l'effet, anéantissons totalement la force centrifuge, c'est-à-dire arrêtons tout-à-fait le mouvement de la terre ; la gravitation agissant alors seule, tendra à faire du sphéroïde une sphère ; les eaux des zones équatoriales cherchant leur niveau, se répandront de part & d'autre vers les pôles ; & leur saillie étoit assez considérable pour qu'il doive en résulter un mouvement prodigieux dans leur chute ; ce mouvement est plus considérable que celui qui suffit à expliquer le déluge ; mais on peut le modérer à volonté à mesure qu'on diminue la quantité du retardement supposé ; on explique encore, si l'on veut, des oscillations dans la masse du fluide, en supposant que le mouvement de rotation ne fut pas retardé uniformément, mais comme par secousses. On peut donc rendre raison de tous les grands faits par cette hypothèse.
L'auteur ne l'a pas même développée autant qu'elle auroit pu l'être : car indépendamment des autres influences atmosphériques qu'il examine, & dont il discute l'effet avec sagacité, un changement brusque dans la densité de l'air put seul produire la pluie ; effet que nous imitons en petit lorsque nous condensons l'air dans nos récipients ; on voit paroître incontinent une vapeur aqueuse qui se précipite.
Un autre effet encore plus marqué dut se\setcounter{page}{365} combiner avec l'accroissement de gravitation qui donna aux eaux équatoriales une tendance à se ruer de part & d'autre vers les pôles ; c'est le résultat de l'inertie de toute la masse des mers, qui se trouvoit en mouvement par l'effet de la rotation de la terre, avec des vitesses données pour chaque point de leur surface, vitesses proportionnelles aux rayons de rotation, ou aux cosinus des latitudes. Car, à l'instant où l'on suppose que le ralentissement du mouvement diurne a lieu, les eaux, conservant leur vitesse acquise, ( qui sous l'équateur approche de celle d'un boulet de canon ) se meuvent avec une extrême violence d'occident en orient sur leur fond qui demeure en arrière ; & leur quantité de mouvement relatif, qui varie avec les latitudes des zônes dans lesquelles on le considere, se combinant avec la tendance polaire qui résulte de la transformation du sphéroïde en sphere, & avec l'effet des obstacles locaux qu'offroit la disposition primitive des continens & des chaînes de montagnes, a pu produire très-facilement toute la serie d'événemens dont Moïse nous a présenté le tableau.
Veut-on ensuite que tout revienne à sa place; il suffit de supposer que la même main qui retarda le mouvement rotatoire du globe lui redonne sa vitesse premiere ; les eaux retournent des pôles vers l'équateur ; elles reforment\setcounter{page}{366} le sphéroïde. Leur inertie produit des courants d'orient en occident qui se combinent aussi avec ceux qui tendent, en retour, des pôles à l'équateur jusqu'à ce que ces mêmes eaux aient acquis toute la vitesse du noyau solide qui les entraîné. — Nous laissons les détails, qui tous cadrent merveilleusement avec les faits & avec les traditions Indiennes. Qu'un LAPLACE faisait l'idée principale ; qu'il la méditât ; qu'il lui appliquât ses puissants moyens d'analyse , elle prendrait, nous le croyons, cette consistance, que les sciences exactes seules, peuvent donner aux matières conjecturales \footnote{Le Prof. Picot ajoute dans une note, que l'Abbé Le Brun avait eu une idée analogue à la sienne, en cherchant à expliquer le Déluge par une accélération dans le mouvement diurne ; il augmentait ainsi la force centrifuge & faisait sortir de leurs cavernes les eaux souterraines. Il avait même construit une machine qui représentait la suite des événements qu'avait dû produire cette cause. Voici ce que nous mande, sur le même sujet, notre savant correspondant, le C. Wild. "Je me suis rappelé en lisant la première partie du Mémoire de Kirwan dans votre Journal, la Géogonie de Silbershlag & un appareil qu'il me fit voir, au moyen duquel il démontrait le déluge conformément au récit de Moïse. C'était un secteur de globe, en forme conique, dont la surface sphérique contenait des montagnes à peu près dans la proportion requise. L'intérieur était très ingénieusement distribué en cavités, siphons, communications, &c. — arrivoit la pluie — les bondes des Cieux s'ouvroient; les plaines étaient inondées; l'eau absorbée entroit par mille petites ouvertures; enfin les abymes regorgeoient par la pression, & les plus hautes montagnes étoient recouvertes jusques à 15 coudées au-dessus de leurs sommets; enfin l'eau rentroit dans le sein de la terre. — Et voici le plus remarquable; c'est que tout se passoit, soit pour la quantité relative de liquide, la vitesse, les temps proportionnels, &c. conformément au récit de Moïse. — J'eus lieu d'admirer sincèrement le génie de ce Mécanicien.}.
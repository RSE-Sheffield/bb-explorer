\setcounter{page}{228}
\chapter{GÉOLOGIE.}
\section{ON THE PRIMITIVE STATE, &c. Sur l'état primitif du globe & la catastrophe qui lui a succédé. Par R. KIRWAN. Esq. des Sociétés Royales de Londres & d'Irlande. ( Tiré des Transactions de l'académie d'Irlande, T. VI.)}
Si, comme nous l'avons observé à l'occasion de l'écrit précédent, les fciences & les arts trouvent leur avantage à s'entr'aider, il n'est pas moins vrai que les fciences considérées à part,\setcounter{page}{229} & celles là même dont les rapports font le moins évidens, gagnent aussi à se prêter mutuellement des secours. Cette alliance, système dont l'application des mathématiques à la physique fut un des premiers & des plus utiles exemples, a essentiellement contribué à leurs progrès dans le siècle précédent & dans celui qui va finir.
Mais pour découvrir leurs relations réciproques, pour moyenner ces traités d'échange, il faut, comme dans la diplomatie des nations, que ceux qui sont appelés à y jouer un rôle y portent, outre une vocation naturelle, qui tient au caractère de l'individu, un vaste assortiment de connaissances bien digérées ; un esprit net & conciliant ; une tête fertile en aperçus, & froide dans l'examen ; de la pénétration, de la sagacité, & une logique sévère. Sans doute, le monde savant n'offre qu'un bien petit nombre d'hommes qui réunissent cet ensemble de qualités éminentes ; mais, en traçant cette esquisse, nous avions pourtant dans l'esprit un modèle : c'était Ri KIRWAN ; & son nom s'est présenté si souvent dans nos feuilles, que peut-être ici nos lecteurs l'avaient deviné avant que nous le nommassions.
Ils ont pu remarquer, en effet, que la chimie, la physique, la minéralogie, l'agriculture, la géologie, en un mot, que presque toutes les sciences naturelles ont dû des progrès à\setcounter{page}{230} ses travaux soutenus; & l'écrit qui lui mérita une palme dans l'Académie d'Irlande \footnote{Essai sur les engrais. Voyez T. II. Sc. & Arts, p. 28.}, montre d'une manière frappante, ce que peuvent gagner deux sciences à être rapprochées par une main habile: la chimie & l'agriculture, long-temps inconnues l'une à l'autre, & maintenant associées, ne se sépareront plus; & en se perfectionnant ensemble, ces deux sciences procureront par les moyens les plus directs, le bien de l'espèce humaine.
Le succès de cette tentative a sans doute encouragé Mr. Kirwan à en hasarder une autre à la réussite de laquelle il a pu croire que l'humanité gagneroit sous le rapport moral. Nous trouvons, dans le dernier volume des Transactions de l'Académie d'Irlande un travail dont il est l'auteur & qui est rempli des plus ingénieux rapprochemens; son objet est de raccorder les monumens géologiques de l'état primitif du globe avec ceux tirés de l'histoire, & en particulier avec le texte des Livres sacrés. Il n'est pas entre les philosophes modernes le premier qui ait étudié profondément ce sujet: Wallerius avoit déjà pris le récit de Moïse pour base de son système géologique; & le savant DELUC a publié ses idées à cet égard dans une suite de lettres au Prof. Blumenbach, qui ont paru à\setcounter{page}{231} Londres depuis près de trois ans : ces lettres viennent d'être traduites, & réunies, avec des additions; en un volume in 8°. imprimé à Paris. Les deux géologues, quoique d'accord sur le but principal & les idées fondamentales, ne le sont pas toujours dans le mode d'argumentation; il ne nous appartient pas de décider entr'eux; & l'amitié, qui nous lie à notre célebre compatriote feroit d'ailleurs un motif suffisant de récufation. Nous venons au mémoire de Mr. Kirwan.
Cet auteur pose d'entrée, les principes logiques auxquels il raménera tous ses raisonnemens. Nous allons les énoncer textuellement:
"Dans la recherche des faits anciens qui dépendent de causes naturelles, on doit, dit-il, s'attacher invariablement à certaines regles de raisonnement. La premiere est de n'attribuer aucun effet à une cause dont l'énergie connue soit insuffisante pour le produire. La seconde est de n'admettre aucune cause dont l'existence ne soit pas prouvée, ou par l'expérience ou par un témoignage incontestable. Il y a un grand nombre de phénoménes naturels qui ont eu lieu & s'observent encore dans des endroits distans de nous & dont on ne pourroit sans absurdité rejeter la preuve par témoignage : ainsi les habitans du Nord de l'Europe qui n'ont jamais éprouvé de tremblement de terre & n'ont jamais vu de volcans, doivent admettre\setcounter{page}{232} par simple témoignage, l'existence passée & actuelle de ces phénomènes."
"La troisième règle est, de n'attribuer à une cause présumée, que l'influence qu'elle possède réellement d'après l'observation, & dans les circonstances dans lesquelles elle a exercé son action."
"Je me propose de me conformer strictement à ces règles dans la recherche que j'entreprends, & elle n'aura de mérite qu'autant que j'y ferai fidèle. Je les appliquerai aux divers systèmes dont j'aurai l'occasion de parler."
"Et les personnes qui feraient tentées de considérer cette recherche comme superflue, & prétendraient que l'état présent du globe mérite seul l'attention des naturalistes; ces personnes, dis-je, voudront bien observer que son état primitif est si intimément lié avec les apparences actuelles, qu'on ne peut se faire des idées justes de celles-ci sans remonter aux temps antérieurs. Et de plus, des preuves récentes nous ont convaincus que, l'obscurité qui a juste présent enveloppé les premiers âges du monde, a trop favorisé l'apparition de divers systèmes d'athéisme; & que ces systèmes ont à leur tour trop multiplié le désordre & l'immoralité, pour qu'on ne doive pas chercher à dissiper cette obscurité en s'aidant de toutes les lumières qu'ont fourni les recherches des naturalistes modernes. Ainsi la géologie se trouvera\setcounter{page}{233} arriver par une transition naturelle (pour nous exprimer en minéralogistes) à la religion: comme celle-ci, par un passage aussi naturel, se lie au système de la moralité.
"Il faut dire encore, que les recherches des géologues modernes ont été tellement étendues et heureuses, qu'elles sont si évidemment liées avec l'objet qui nous occupe, que depuis l'époque où les anciennes traditions ont été obscurcies ou effacées, il n'y a point eu de période plus favorable que ne le sont les temps actuels, à l'examen de l'état primitif du globe, malgré le long intervalle qui nous en sépare. On n'avoit point jusqu'à présent, parcouru la surface dans nombre de directions différentes; on n'avoit pas jusqu'à nos jours des données exactes sur sa forme, sur les proportions et les gisemens relatifs des continents et des mers, sur la densité moyenne de la masse totale. On n'avoit pas étudié ses parties constituantes et leurs rapports mutuels; on n'avoit rien approfondi comme on l'a fait depuis vingt-cinq ans; et jamais non plus, les preuves testimoniales n'avoient été plus soigneusement pesées et soumises à une critique plus sévère, qu'elles ne l'ont été dans la dernière moitié de ce siècle."
"Je sais que l'admission de la preuve testimoniale en matière de physique pure, répugne à beaucoup de gens; mais ils ne font point assez d'attention à la nature de l'objet dont il\setcounter{page}{234} s'agit. Toutes les recherches physiques sont fondées sur l'expérience ou l'observation, séparément ou collectivement, & sur les conséquences immédiates qu'il est possible d'en déduire. Lorsqu'on ne peut avoir recours à l'expérience comme dans le cas présent, alors il faut se borner à l'observation seule ; mais comme celle-ci même ne se présente pas tous les jours, & que les faits sont souvent séparés par de grands intervalles de temps & de lieu, on est forcé d'admettre les preuves bistoriques & par conséquent le témoignage. Ainsi, en astronomie, science d'ailleurs purement physique, les philosophes n'ont jamais hésité à admettre les observations d'un Hipparque ou d'un Ptolomée. En effet, les anciens événemens géologiques étant du domaine de l'histoire, tous les efforts par lesquels on prétendroit les approfondir en argumentant seulement d'après leurs conséquences actuelles, & en excluant toute preuve testimoniale bien avérée, feroient aussi absurdes que ceux d'un écrivain qui voulant faire l'histoire de l'ancienne Rome ne consulteroit que les médailles & les ruines de ses monumens publics, en écartant volontairement les témoignages d'un Tite - Live, d'un Salluste, d'un Tacite. On convient de toutes parts que de grands changemens ont eu lieu à la surface du globe depuis qu'il a commencé d'exister, & que ces changemens ne se sont pas répétés\setcounter{page}{235} depuis quelques milliers d'années. Pourquoi ces grands événemens & leurs circonstances ne feroient-ils pas susceptibles de preuves testimoniales ? Ce n'est sans doute pas qu'ils soient improbables, ou contredits par les observations actuelles, puis qu'au contraire, leur réalité est accordée sans conteste ; je n'en fais voir qu'une raison, relativement à quelques-uns de ces événemens ; & au premier énoncé, cette raison est plausible, savoir que leur existence a précédé celle de l'espece humaine : cela prouve certainement que les connoissances de l'historien qui les rapporte n'avoient pas été acquises, quant à ces faits, par des moyens humains. Mais si dans une serie de faits découverts par une recherche particuliere, à laquelle le témoin est absolument étranger, on remarque accord parfait avec sa relation, non pas seulement quant au matériel des faits qu'il a décrits, mais quant à l'ordre & à la succession de ces mêmes événemens, alors on est forcé de reconnoitre que la relation est véritable, quelle que soit la source où l'historien a puisé ses documens \footnote{On ne sent pas, au premier abord, toute la force de cet argument, parce qu'on ignore quelle est cette recherche collatérale & indépendante de l'historien, par laquelle on découvre & les faits & leur succession nécessaire, tels, & dans le même ordre où il les présente. La lecture du mémoire donne la clef de ce genre de raisonnement. Il revient à celui d'un Juge qui, interrogeant un témoin sur les principales circonstances d'un certain événement, circonstances dont le Juge auroit obtenu d'ailleurs la connoissance, recevroit des réponses parfaitement conformes à ce qu'il sait être la vérité. Il n'hésiteroit pas alors, à ajouter foi au témoin sur les circonstances même qui, par leur nature, n'auroient pu être connues que de ce dernier. (R)}. Si cette source n'a pu être\setcounter{page}{236} humaine, elle doit avoir été surnaturelle, & doit mériter toute croyance, même dans les cas auxquels l'observation n'a pu atteindre, ou qui ne sont peut-être pas susceptibles de ce genre de confirmation. Maintenant j'ose dire que ces caractères s'appliquent à la relation que nous a donné Moïse de l'état primitif du globe, & je prétends le démontrer dans le cours de cet Essai."
Nous ne nous proposions pas d'abord, de transcrire en entier l'espèce de préface qu'on vient de lire; mais les principes de l'auteur & sa méthode y sont si nettement exposés que nous n'aurions pu, sans lui faire tort, en retrancher une ligne. Le reste de son écrit est divisé en deux Essais distincts; le premier traite de l'état primitif du globe; le second, de la catastrophe du Déluge. Nous réserverons l'analyse de celui-ci pour un numéro prochain.
\subsection{PREMIER ESSAI}
Sur l'état primitif du globe.
LA forme sphéroïdale du globe terrestre\setcounter{page}{237} bien établie par les astronomes ; c'est-à-dire , son renflement sous l'équateur & son applatissement vers les pôles , prouvent que sa surface , jusqu'à une certaine profondeur , doit avoir été liquide ou du moins dans un état qui permit à ses molécules de se mettre en équilibre en obéissant à la fois à la gravitation & à la force centrifuge née de la rotation. Quelques observations géologiques montrent aussi que certaines substances actuellement très - dures doivent avoir été originairement dans un état de mollesse. Ainsi les cailloux siliceux qui font partie des poudingues argileux des montagnes de Quædliæ & Porthscellet en Norvège , sont réduites par la compression à l'épaisseur d'un quart de pouce dans la partie inférieure de la montagne , & se trouvent plus gros & plus arrondis à mesure qu'on s'élève. L'auteur cite d'autres faits du même genre , sur lesquels nous n'insisterons pas , parce que le fait principal est généralement admis.
On ne connoît rien de l'intérieur ; mais on peut présumer & on s'accorde assez à admettre , qu'à l'époque de la création & long-temps après , il y existoit d'immenses cavités , dont les voûtes étoient assez solides pour résister à l'énorme pression des matières liquides qui les recouvroient.
Cette liquidité , qui appartenoit même aux parties maintenant les plus solides , ne pouvoit\setcounter{page}{238} provenir que d'une fusion ignée, ou d'une solution aqueuse. L'auteur renvoye, sur la premiere de ces deux hypotheses, à un Mémoire inséré dans les Transactions d'Irlande, dans lequel il a prouvé fort au long qu'elle étoit absolument inadmissible. Reste la solution aqueuse, dont la supposition s'accorde beaucoup mieux encore qu'il ne l'avoit imaginé d'abord, avec les propriétés & les caracteres qui distinguent les solides actuels, si l'on en excepte ceux qui sont décidément d'origine volcanique.
Mais les solides actuels ne sont presque point solubles dans l'eau; & où trouveroit-on tout le liquide qui a dû être nécessaire pour les tenir en dissolution? Cette difficulté a embarrassé beaucoup de Naturalistes. Les uns ont recours à un dissolvant particulier maintenant dissipé ou combiné; les autres, à une très-haute température. L'auteur résout fort simplement le problème. Pourquoi oublier, dit-il, le fait duquel nous sommes partis & qui est démontré, c'est que la terre étoit dans l'origine une masse liquide? pourquoi chercher des solides à faire dissoudre, tandis qu'il n'en existoit point, & que les molécules destinées à former ces solides étoient primitivement suspendues dans une masse ou bouillie hétérogène, qui contenoit les élémens de tout ce qui a existé depuis?
L'eau de cette bouillie étant liquide, devoit\setcounter{page}{239} avoir une chaleur exprimée au moins par le zéro du thermomètre commun, & peut-être posséder une température beaucoup plus élevée. Ensuite, elle devoit contenir les huit terres simples; toutes les substances métalliques; tous les principes chimiques simples. Cette masse devoit en un mot, former dans son ensemble, un composé plus compliqué qu'aucun de ceux qui ont existé depuis, & doué par conséquent de propriétés très-différentes de celles que nous reconnoissons dans nos liquides actuels.
Ainsi le feu élémentaire dut être contemporain de la création du chaos; & les loix de la gravitation & des attractions électives paroissent devoir être de même date.
Nous remarquerons que le système de Deluc ne diffère jusqu'ici de celui de notre auteur qu'en ce que le premier suppose ce chaos d'abord dans un état de congélation & n'en admet la liquidité qu'après la création de la lumière, laquelle produisit ensuite le feu nécessaire à la liquéfaction de l'eau.
Ce fluide chaotique ne contenoit sans doute pas les différens ingrédiens dont il étoit composé, disséminés dans sa masse d'une manière uniforme. Certaines régions renfermoient certaines terres en plus grande proportion que d'autres; ailleurs certains métaux dominoient. La terre calcaire existoit alors, & , selon notre auteur, a précédé l'animalité. Buffon & Hutton\setcounter{page}{240} qui ont foutenu le contraire, ignoraient sans doute que les analystes modernes ont reconnu la terre calcaire dans un nombre de pierres dont la formation appartient à cette premiere époque. Hutton il est vrai, suppose que le monde actuel provient des ruines d'un monde antérieur, & celui-ci d'un précédent. "S'il entend nous conduire ainsi jusqu'à l'infini, dit notre auteur, je, ne prétendrais pas l'y suivre; mais s'il s'arrête quelque part, à moins qu'il ne suppose aussi son globe absolument différent de celui que nous habitons ( & alors je n'ai rien à en dire ) il y retrouvera mon argument."
L'action des affinités électives dut produire dans un fluide constisté comme celui qu'on vient de décrire, des cristallisations entre les élémens homogenes qui se trouvoient dans sa masse. Les groupes de ces cristaux se déposoient & s'entassoient à mesure sur le noyau solide. Dans les régions où la terre siliceuse & l'argile prédominoient, & ces régions étoient de beaucoup les plus étendues, le granit & le gneifs ont dû se former les premiers. L'auteur explique ingénieufement d'après l'analyse connue du quartz du feldspath & du mica qui entrent comme cristaux intégrans dans la composition de ces deux agrégés, quel a dû être l'ordre de succession de ces cristaux; d'abord le quartz, puis le feldspath, ensuite le mica.
Cependant l'eau dégagée par l'effet de ces\setcounter{page}{241} cristallisations s'élevait à mesure & en facilitait de nouvelles qui s'attachaient aux précédentes ; & le mode d'agrégation variant, à raison des proportions dans les molécules élémentaires, il se faisait là du granit, ici du gneiss, plus loin des roches micacées ; & l'on observe actuellement ces roches entremêlées dans les montagnes primitives qui datent de cette époque. On doit peu s'étonner si l'on y rencontre aussi quelquefois des substances métalliques & en particulier du fer, comme aussi des traces de carbone & de plombagine.—Ici, comme en beaucoup d'autres endroits, l'auteur cite notre illustre ami De Saussure. Nous nous étonnons qu'il n'ait pas cité La Méthérie qui a expliqué depuis long-temps, la formation des montagnes primitives par cristallisation.
Dans d'autres régions où se trouvèrent les mêmes terres élémentaires, mais sous d'autres proportions que celles des ingrédients granitiques, il se forma des masses siliceuses ; savoir des porphyres, des jaspes, des schistes siliceux, &c. Ailleurs encore, les argillites, l'hornblende, les schistes primitifs, & d'autres roches d'ansique formation parurent, à raison de tels ou tels ingrédients qui dominaient dans la masse, & décidaient des cristallisations partielles & imparfaites.
Les substances métalliques ; que l'auteur considère comme élémentaires & contemporaires\setcounter{page}{242} de tous ces ingrédients, & le fer en particulier, durent souvent s'unir avec le soufre, ( aussi de même date ) & former les sulfures métalliques qui constituent la plupart des minéralisations. Le pétrole, plus léger que l'eau mais retenu dans la masse épaisse du fluide chaotique, pût là aussi s'unir au soufre & au carbone & se précipiter avec eux. Voilà les houilles primitives.
Jusques ici, tout se passe tranquillement, dans le sein d'un liquide en repos. Mais bientôt la scène va changer.
On fait maintenant par expérience, qu'il se dégage beaucoup de calorique dans l'acte de la cristallisation. Quelle chaleur immense ne dut pas se manifester à l'époque où des masses aussi énormes devinrent solides ; quelle évaporation à la surface du fluide, en conséquence de cette chaleur ; quelles ondulations dans le fluide élastique dégagé, à raison de son inégalité dans les diverses régions où la cristallisation avoit un différent caractère!
Et ici l'effet put renouveler la cause : car l'évaporation diminuant la quantité absolue & la pesanteur spécifique du liquide chaotique, dut, ainsi que cela arrive dans les solutions de nos laboratoires, le disposer à de nouvelles précipitations ; & lorsque les molécules ferrugineuses très-abondantes, & d'ailleurs très-peu solubles dans l'état métallique, furent brusquement abandonnées, elles durent décomposer\setcounter{page}{243} l'eau, avec laquelle elles se trouvoient en contact à une très-haute température. Voilà de prodigieuses quantités de gaz hydrogene qui se produisent; elles rencontrent l'oxigene mis à l'état élastique par la chaleur qui alloit jusqu'à l'incandescence, & forment d'épouvantables inflammations. La chaleur, redoublant par ce fait même, doit avoir dégagé du fluide chaotique tout l'oxigene; & ce principe s'unissant avec le reste du fer encore métallique, & en partie avec les substances carboniques & bitumineuses, aura produit une conflagration prodigieuse, qui aura même atteint la base solide du fluide chaotique & l'aura crevassée en beaucoup d'endroits.
Ce n'est point une supposition forcée, que celle de flammes qui sortiroient du fond des eaux. L'auteur en cite des exemples récens. On a vu des isles se former à la suite de ces éruptions.
Ces convulsions, qui paroissent avoir surtout agité à cette époque l'hémisphere méridional, eurent d'importantes conséquences.
1°. La grande chaleur communiquée au liquide en dégagea l'oxigene & les fluides non respirables: & ainsi, se forma peu-à-peu l'atmosphere.
2°. L'oxigene rencontrant le carbone incandescent produisit l'acide carbonique; & ce fluide d'abord mêlé à l'atmosphere, s'en sépara par le refroidissement, & absorbé ensuite par le\setcounter{page}{244} liquide, s'unit à la terre calcaire (qui, comme plus soluble que les autres, n'avoit pas encore abandonné le menstrue), & la fit cristalliser. Telle fut l'origine des couches calcaires primitives. Cette formation explique pourquoi elles sont presque toujours dégagées de tout mélange, & pourquoi lorsqu'elles alternent avec d'autres substances, c'est avec des gneiss, ou d'autres roches primitives.
On peut établir d'après l'analyse chimique, que la formation de l'air fixe fut postérieure à celle de la plupart de ces roches; car elle nous apprend que la terre calcaire, qui entre comme ingrédient en petite quantité dans les substances primitives, y est dans l'état caustique, c'est-à-dire privée d'acide carbonique.
L'analogie tirée de ce que nous observons dans nos laboratoires conduit plus loin l'ingénieux auteur de ces développemens. Après la cristallisation des grandes masses primitives au fond du liquide, & après l'évaporation que produisit la chaleur dégagée, la cristallisation put aussi s'opérer à la surface; ainsi qu'on l'observe dans plusieurs sels & dans l'eau de chaux. Des couches successives purent s'y former & se déposer ensuite, tantôt horizontalement, tantôt inclinées & même verticales, à raison des accidens de rupture qui devoient être fréquens dans ces couches comme suspendues. On observe ces variétés d'inclinaison dans les montagnes primitives.
\setcounter{page}{245} Les plaines primitives durent réfulter du dépôt fuccellif des parties folides encore contenues dans le liquide chaotique, mais peu difposées à la criftallifation, ou trop diftantes entr'elles pour s'atteindre. Les argiles mêlées de filice, d'autres terres, & de fer, en fe dépofant, formerent des plaines. Celles-ci purent enfuite être recouvertes de débris des montagnes primitives environnantes, mais ce fut dans une époque poftérieure.
Un océan univerfel recouvre encore ces montagnes & ces plaines ; mais on va les voir paroître au-dehors : de vaftes cavernes exiftoient dans le noyau folide du globe ; les crevaffes occafionnées dans fon écorce par l'action des feux dont on a parlé, donnent ( & particulièrement dans l'hémifphere auftral ) accès à l'eau fupérieure, dans ces grands réfervoirs. Elle s'y jette d'abord avec furie, puis moins violemment à mefure que fa profondeur diminue ; les fommets font dégagés, puis les plaines ; voilà les continens : ils fe deffèchent & fe confolident peu-à-peu.
Leur apparition paroît avoir été fuccellive. Les premieres couches mifes à fec étoient élevées au moins de 8500 à 9000 pieds ( 2742 mètres ) au-deffus des mers actuelles, ainfi les fommítés des principales chaînes de l'Afie orientale, des Alpes, les Pyrénées, en Europe, &\setcounter{page}{246} des Cordillieres en Amérique, durent se trouver fort au-dessus du niveau de l'Océan.
À cette époque, & non auparavant, le fluide chaotique, devenu un liquide à-peu-près homogene, est une véritable mer & se peuple de Poissons.
Ici l'auteur prouve que les eaux antérieures ne contenoient pas d'animaux marins, en citant, avec son érudition ordinaire, les rapports des géographes & des physiciens qui ont déterminé les gîsemens de ces chaînes & leur hauteur, & ceux des Naturalistes qui assurent unanimément qu'on n'y découvre point de dépouilles marines, & que les couches calcaires mêmes qui s'y trouvent sont primitives. Pallas a trouvé des sources & des lacs salés dans les plaines de ces hautes régions; il y a même rencontré des veines de houille, mais jamais de traces organiques, à l'exception de quelques coquillages roulés dans les fentes des granits, dépôts accidentels & qu'il attribue avec raison, à l'événement du déluge \footnote{Nous invitons, à cette occasion, les jeunes naturalistes qui font des courses & commencent des collections, à distinguer toujours bien soigneusement ce qui appartient essentiellement au sol qu'ils parcourent, ce qui est renfermé par exemple dans la masse des rochers contemporains de ce sol, de ce qui n'est qu'adventif, ou déposé à la surface par l'effet de causes qui ont agi postérieurement à la formation de ces rochers. L'inattention à cet égard introduit dans les rapports une consusion qui fait reculer la science. (R)}.
\setcounter{page}{247} Et celles-là même, d'entre les montagnes purement calcaires, qui surpassent la limite que l'auteur assigne à l'ancienne mer, ne contiennent point de dépouilles marines. Il cite à cet égard la Peyrouse pour les Pyrénées, & Defaulfure pour les Alpes.
Et réciproquement, aucune des montagnes qui contiennent dans leur masse des coquillages pétrisiés n'atteint la hauteur de 8600 pieds (2620 mètres). L'auteur montre même d'après de nombreuses citations, que la plupart de ces montagnes sont d'une hauteur beaucoup moindre; & deux seulement; l'une désignée par Deluc, le mont Grenier dans les Alpes, l'autre la montagne de Terglore en Carinthie indiquée par le baron de Zoïts, approchent de la limite qu'il a fixée: encore soupçonne-t-il que les coquillages trouvés par Deluc puissent être adventifs & provenir du déluge; mais comme nous avons observé nous-mêmes certaines sommités voisines du Grenier, à une hauteur à-peu-près semblable, & entièrement composées de coquillages en masse, nous ferions disposés à croire que ceux qu'a vus Deluc appartenoient de même aux bancs calcaires de ces hautes régions, qui furent par conséquent long-temps sous-marines.
\setcounter{page}{248} Mais les contrées encore supérieures, & celles que les eaux avoient abandonnées, durent bientôt se peupler d’animaux & de végétaux ; car le froid sévère qui les rend maintenant inhabitables n’existoit pas pour elles alors, attendu qu’elles occupoient les couches inférieures de l’atmosphère qui n’étoit pas encore descendue à son niveau actuel ; & qu’elles étoient voisines des mers ; circonstances qui, comme l’auteur l’a montré ailleurs, influent principalement sur la température d’un climat.
Il prouve ensuite, que les eaux ne se retirerent que graduellement du reste des continens maintenant découverts : voici ses argumens.
1°. On trouve de part & d’autre du plateau altaïque, des montagnes coquilleres soit calcaires, soit argileuses. Ces masses ont dû être l’effet du séjour de la mer dans ces régions pendant un grand nombre d’années.
2°. On rencontre çà & là, dans les plaines de l’ancien & du nouveau monde, des amas énormes de coquillages marins. Parmi les nombreuses citations de l’auteur à cet égard, on distingué le fameux Fahlun de Touraine. On trouve-là, à 100 milles de la mer, & à huit ou neuf pieds de profondeur, une masse de vingt pieds d’épaisseur composée presqu’uniquement de coquillages marins, dont les analogues vivans habitent en grande partie, les mers voisines.\setcounter{page}{249} La plupart de ces coquillages sont déposés sur leur face plate, ce qui prouve qu'ils n'ont pas été accumulés par l'effet de quelque bouleversement. On voit des entassements du même genre mais moins considérables en Angleterre. Il est à remarquer qu'on en trouve rarement en creusant au-dessous du niveau des mers actuelles. Les pays les plus bas, comme le Brabant & la Hollande en contiennent la plus grande quantité. Ils sont aussi fort abondants dans certaines contrées de Russie, où la mer a séjourné plus long-temps qu'ailleurs.
3°. On a trouvé dans nos continents à d'assez grandes profondeurs, & même sous des montagnes, des arbres & divers végétaux ensevelis. Ces arbres n'ont pu croître que dans des régions mises à sec, tandis que celles où on les trouve étoient couvertes par les eaux; ce qui prouve une retraite graduelle. La situation de ces dépouilles ensevelies souvent sous des couches régulières de matières pierreuses, ne permet pas d'attribuer leur dépôt à une convulsion telle que le déluge.
4°. On a souvent trouvé des arbres ensevelis, à des hauteurs où la température qui y règne maintenant ne leur aurait pas permis de végéter. Ils y ont donc vécu à une époque à laquelle ces sommités étoient moins élevées au-dessus de la mer, & à une moindre distance de ses rivages.
\setcounter{page}{250}
5°. Enfin, on observe dans les deux continens, des montagnes stratifiées, de hauteurs diverses au-dessous de 8000 pieds, & dans les couches desquelles on trouve diverses substances d'origine marine & terrestre, qui s'y rencontrent ou pétrifiées, ou dans leur état naturel. La régularité de ces couches indique une action uniforme & long-temps continuée. Nous ne connoissions que les marées qui aient ce caractere. On trouve bien çà & là des bouleversements, mais ils sont dus à des causes accidentelles. Les couches ne sont pas superposées dans l'ordre de leurs pesanteurs spécifiques, mais les matériaux qui composent chacune d'elles prise à part sont ordinairement disposés dans cet ordre.
Après avoir ainsi prouvé que la seconde retraite des eaux s'opéra très-lentement & que les montagnes secondaires furent formées dans leur sein, seulement depuis que ces eaux eurent été peuplées d'animaux, l'auteur s'occupe du mode de formation de ces montagnes, & il observe :
Que leurs matériaux durent être 1°. les débris des montagnes primaires qui existoient sous les eaux ; leurs sommités battues des vagues & détachées par des tremblemens de terre purent rouler & s'amenuiser suffisamment ; 2°. dans quelques endroits, des volcans sous-marins vomirent en abondance des matières, que les\setcounter{page}{251} eaux travaillèrent ensuite de la même manière. Tous ces fragments durent se déposer à mesure, contre le pied & les flancs des montagnes primaires; ils purent recouvrir même des sommets primitifs inférieurs; ils ensevelirent à diverses époques & les coquillages & les végétaux pêle mêle. Les arbres prirent la situation que leur donnèrent les courants qui les précipitoient des montagnes, & delà cette uniformité qu'on remarque dans leur position. Ces avalanches boueuses surprirent aussi les poissons dans certains parages, & on les y trouve encore dans des masses maintenant pierreuses: \footnote{Voyez la description des poissons renfermés dans la pierre, à Monte-Bolca, Tom. II. Sc. & Arts, p. 68. (R)}
D'autres grands faits géologiques s'expliquent aussi heureusement dans cette hypothèse. L'observation constante, que les montagnes à couches secondaires reposent toujours sur les matieres primitives. Le Dr. Hutton seul, nie le fait pour l'Angleterre, parce qu'en la traversant, il a rarement trouvé le granit en masse. Cela ne prouve rien sinon que ce granit étoit plus ou moins profondément recouvert par les couches secondaires. D'ailleurs d'autres naturalistes tels que Mr. Everman & le Dr. Ash, nous apprennent que l'Ecosse presqu'entière repose sur le granit \footnote{Ainsi, dans nos Alpes, le Granit des chaînes centrales plonge sous les chaînes secondaires collaterales, mais reparaît quelquefois dans les plaines & en particulier là où des rivières ont découpé profondément le sol. Nous l'avons retrouvé de cette manière à Tain en Dauphiné, au bord du Rhône & au pié des collines où croît le fameux vin de l'hermitage. (R)}.
\setcounter{page}{252}
2°. L'observation de Deaussure, que les couches des montagnes secondaires font ordinairement inclinées parallèlement aux flancs des chaînes primitives, même à une grande distance, & paroissent s'élever contr'elles \footnote{Cette belle observation est surtout frappante sur le sommet du Cramont où son savant auteur nous invita à la répéter avec lui il y a vingt ans. (R)}. Cela indique une formation originaire en appui contre ces flancs; mais de vastes découpures formées par des événemens postérieurs ont creusé des vallées intermédiaires & n'ont laissé subsister que les grands traits de cette disposition générale.
Ainsi Schreiber a observé dans la montagne de la Gardette que les couches calcaires étoient toujours superposées au gneifs, & que là où il y avoit pénétration entre ces substances, c'étoit constamment le gneifs qui se projettoit dans le calcaire, comme consolidé antérieurement à l'arrivée de cette dernière substance. Notre auteur cite encore ici une observation de Deaussure sur la situation des poudingues & des grès, qui appuie son système.
\setcounter{page}{253} La retraite graduée de la mer continua jusques à une époque qui dut précéder le déluge de quelques siècles. L'auteur présume que les continens avaient été mis à sec long-temps avant cette catastrophe, parce que sans elle les montagnes n'auraient pu acquérir la consistance nécessaire pour lui résister, ainsi que nous voyons qu'elles l'ont fait. Ce n'est pas que les couches secondaires n'aient eu déjà une certaine dureté à l'époque même de leur formation; mais elle n'approche jamais de celle que procure à ces matières le séjour à l'air et l'infiltration lente des sucs lapidifiques. Cette infiltration n'est point une hypothèse, et l'auteur cite ici de belles observations de Werner et de Bergman sur ce procédé de la Nature.
Après avoir ainsi établi, d'après les connaissances positives que nous fournissent la chimie et la géologie dans leur état actuel de perfectionnement, la série, en quelque sorte nécessaire, des grands phénomènes ante-diluviens, l'auteur compulse le récit de Moïse, de la manière suivante:
"Au commencement Dieu créa les Cieux et la Terre." — C'est-à-dire, le premier événement de l'histoire de la Terre est sa création, et celle des corps célestes semés dans l'espace.
"Et la terre était sans forme et vide." C'est-à-dire d'abord qu'il n'existait pas de globe antérieur, des ruines duquel la terre actuelle eût\setcounter{page}{254} été formée. L'expression sans forme & vide est imparfaitement traduite de l'hébreu Tobu Bohu; Ainsworth remarque que Tobu désigne un état de confusion, & Bohu un vide. Elle étoit donc en partie dans un état de chaos & en partie vide, ainsi que l'auteur a montré que cela devoit être, d'après les phénomènes.
"Et les ténèbres étoient sur la face de l'abime."
La lumière n'existait donc pas encore; Méde & Eftius observent que l'expression hébraïque traduite par l'abime ou les profondeurs, désigne un mélange confus de terre & d'eau. David dit Ps. CIV. v. 6. "Tu l'avois couverte de l'abime comme d'un vêtement & les eaux se tenoient sur les montagnes."
Il paroît donc que l'eau fut créée liquide, & que par conséquent le calorique existait à cette premiere époque.
"Et l'esprit de Dieu (ou plutôt un esprit de Dieu) se mouvait sur les eaux." Le mot traduit par esprit, désigne en général un fluide élastique invisible. — De Dieu; les Hébreux désignaient ainsi tout ce qui étoit vaste. — Se mouvait, plus exactement recouvrait, les eaux, ou planoit sur elles. Ici l'auteur voit les produits de l'évaporation que procura le calorique dégagé par la cristallisation dans le sein des eaux. David vient de nous apprendre un fait essentiel, qu'il tenoit sans doute de la tradition mosaïque; c'est que les eaux surpassaient les mon\setcounter{page}{255} zagnes; donc celles-ci furent formées dans le sein d'un liquide. Il dit aussi "il a fondé la terre sur ses bases, tellement qu'elle ne sera point ébranlée à perpétuité." Ceci dénote l'époque où les montagnes se formèrent & s'affirent sur le noyau du globe.
"Et Dieu dit," que la lumiere soit, & la lumiere fut. "Ici l'auteur remarque que les faits seulement, furent révélés à l'Ecrivain sacré; & que les expreffions durent toujours être afforties tant à ses propres notions qu'au langage du peuple auquel il transmettoit ces faits. Ainsi, *Dieu dit*, *Dieu vit que cela étoit bon*, &c. sont des phrases anthropologiques, qui ne fignifient autre chose sinon que tels événemens, naturellement possibles, ont eu lieu en vertu des loix ou des forces établies par le Créateur.
Cette production de la lumiere a lieu à l'époque où notre auteur a montré, que dans l'ordre des événemens naturels, les flammes des éruptions volcaniques ont dû paroître. Il n'a pas fait attention que cette époque est assez nettement désignée v. 4 du même pseaume qu'il a cité & qui paroît être un abrégé de la tradition mosaïque. "Il fait des vents ses anges & du feu brûlant ses serviteurs." Moïse appelle jour cette lumiere, d'après sa ressemblance avec le jour, qui, dans le sens ordinaire, désigne l'effet de la présence du soleil, & ne put avoir lieu que dans une époque subséquente.
\setcounter{page}{256} "Et Dieu dit," qu'il y ait une étendue entre les eaux, & qu'elle sépare les eaux d'avec les eaux. "Cette étendue ne peut être que l'atmosphère, qui contient les nuages, & sépare ainsi bien véritablement les eaux d'avec les eaux."
"Puis Dieu dit; que les eaux qui font au-dessous des Cieux foient rassemblées en un lieu, & que le sec paroiffe; & il fut ainsi." — Voilà le cinquieme événement que Moïse place dans l'ordre que lui ont assigné des considérations purement physiques, & indépendantes de son récit.
Les événemens subséquens n'ont pas de rapport avec la géologie, à l'exception de la création des poissons, fait très-important dans la théorie de la terre.
Moïse, & les considérations physiques, nous apprennent que cette formation fut postérieure à la séparation des eaux d'avec les premiers continens & les montagnes primitives. L'auteur sacré nous dit aussi que la création des animaux terrestres fut subséquente à celle des poissons: ce fait est également indiqué par les observations géologiques; car on trouve toujours les débris d'animaux près de la surface du sol, tandis que les poissons occupent les couches inférieures; Buffon ne se contente pas d'admettre cet ordre de succession, il en fait une des bases de son système.
\setcounter{page}{257} "Voilà donc, dit notre savant auteur, sept à huit faits principaux en géologie, qui d'une part, font énoncés par Moïse, & de l'autre, déduits a posteriori des observations les mieux constatées par les naturalistes modernes. Les deux séries s'accordent parfaitement, non-seulement quant au matériel des faits, mais dans l'ordre des événemens successifs. Si nous admettons l'une de ces deux séries, sa concordance avec l'autre est une preuve de la vérité de celle-ci. Si nous n'admettons ni l'un ni l'autre, alors leur accord reste à expliquer. Cet accord, rend infinie l'improbabilité que l'une & l'autre font fausses; l'une des deux doit donc être véritable, & par conséquent aussi, l'autre."
"Lorsqu'il ne feroit question que de deux événemens seulement; en supposant les sources où l'on en puise la connoissance, totalement distinctes & indépendantes l'une de l'autre, il feroit déjà très improbable, si ces faits font faux, que les récits fussent d'accord, soit pour le matériel, soit quant à l'ordre de succession de ces événemens: représentons cette improbabilité, seulement par la fraction 1/10; alors l'improbabilité de l'accord de sept événemens, dans le cas où ils feroient faux, sera exprimée par la fraction 1^7/10^7, c'est-à-dire, par le rapport de l'unité à dix millions; & cette improbabilité feroit encore beaucoup augmentée si l'on eut fait entrer dans le calcul la considération de\setcounter{page}{258} l'ordre de succession de ces événemens.
Ici l'auteur termine la premiere partie de son intéressant Mémoire. Nous avons promis l'analyse de la seconde dans un prochain numéro; c'est avec plaisir que nous tiendrons parole.

\setcounter{page}{192}
\chapter{Arts}
\section{SPECIFICATION OF THE PATENT, &c. Patente accordée à James WILSON, Esq. pour une diffpolition particulière qu'il a inventée dans les armes à feu, au moyen de laquelle la poudre est mise à l'abri des effets du mauvais temps, beaucoup plus sûrement & à moins de frais, que par aucune méthode connue jusqu'à présent.}
"A tous ceux qui les présentes verront, &c. Moi J. W. déclare que mon invention est décrite comme suit :"
"Son objet est de mettre à l'abri de l'humidité, la poudre dans le bassinet des armes à feu ; d'une manière facile , & à peu de frais."
"D'abord, pour l'adapter aux platines ordinaires déjà construites , je propose d'appliquer , soit en queue d'hironde , soit par vis ou par foudure, dans le côté de la platine & du bassinet vers le canon de l'arme , une piece de laiton ou de fer , disposée en forme d'arc , d'environ une ligne de largeur , qui recouvre la lumière & s'élève du côté qui touche le canon , avec un talus en dessus , sous un angle d'environ 60 degrés; par ce moyen , l'eau qui pourroit s'infinuer entre le canon & la batterie , dans l'endroit\setcounter{page}{193} où la poudre est la plus exposée, est conduite de part & d'autre de l'arc par la gouttiere que forme le bizeau ou talus en contact avec le canon. Ce bizeau fait tout l'avantage de cette invention, parce qu'il empêche l'eau d'arriver dans le bassinet, attendu qu'elle doit prendre plus volontiers la direction du conduit qui l'invite à descendre de part & d'autre de la batterie, qu'elle ne montera par dessus le bizeau. Lorsque le vieux bassinet l'exige, j'en forme un nouveau par une piece que j'adopte dessus, & qui, dans ce cas, se joint avec l'arc & en éleve les côtés d'environ ½ ligne ; ce qui approfondit l'espace destiné de part & d'autre, au passage de l'eau ; cet espace s'augmente encore en rabattant le devant & le derriere entre le bassinet & son rebord, autant qu'on peut le faire sans nuire à la platine, qui permet souvent assez de profondeur pour la sûreté sans qu'on éleve le bassinet. Lorsque cette derniere piece est construite originairement d'après mon principe, il ne faut rien de plus que l'arc & la batterie, car l'élévation du bassinet, ou la créusure d'une gouttiere deviennent inutiles puisque l'eau s'écoule immédiatement. De plus, la batterie est excavée ou creusée en dessous, de maniere à ressembler le plus qu'il est possible au bassinet inférieur ordinaire, en le supposant d'une profondeur modérée ; & elle dépasse entierement le même bassinet, dont les bords font\setcounter{page}{194} limés en tranchant; la batterie ferme exactement, mais pas au degré qui l'empêche de joindre lorsqu'il y a la moindre faleté dans les parties frottantes, ainsi que cela arrive dans les batteries à recouvrement. La batterie descend jusques vers le bas du talus, contre le canon, avec un rebord creux, ou en chamfrein, de manière à ne pas s'appuyer dans le haut contre le talus de l'arc, pour prévenir les inconvéniens qu'occasionneroit la pression & l'accumulation des faletés lorsqu'on auroit tiré fréquemment. Il est bon d'observer aussi que les côtés du baffinet étant taillés en tranchant, & se trouvant en contact avec le creux de la batterie, les surfaces opposées l'une à l'autre sont très-peu considérables & favorisent la séparation des faletés, plutôt que de les ramasser, ainsi que cela a lieu dans les surfaces planes. On a reconnu cet avantage après avoir tiré 47 coups sans rien essuyer. Lorsqu'on fait les platines à neuf, il faut préférer le baffinet à jour, & on dispose l'arc ainsi qu'on l'a décrit. On peut ajouter l'arc en talus au canon, au lieu de le mettre à la platine; & on peut aussi l'y visser, ou l'y inférer à queue d'hirondelle. Cette disposition a l'avantage de prévenir tous les inconvéniens qui résulteroient de la négligence à visser la platine bien ferrée contre le canon ; & pour éviter qu'il ne reste des doutes sur la jonction entre le canon & l'arc, celui-ci se projette d'environ 1/4 de ligne en dehors de la platine, & est inféré dans le canon. En foi de quoi, &c.
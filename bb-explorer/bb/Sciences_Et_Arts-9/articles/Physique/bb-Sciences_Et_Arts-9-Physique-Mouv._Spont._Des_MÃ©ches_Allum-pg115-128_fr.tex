\setcounter{page}{115}
\chapter{Physique}
\section{AN ACCOUNT OF CERTAIN MOTIONS, '&c. Description de certains mouvemens que de petites mèches allumées & nageant dans un baffin d'huile acquierent spontanément. Observations sur les phénomènes qui tendent à expliquer les principes desquels dépendent ces mouvemens, par P. WILSON, Professeur d'Astron. à Glasgow, membre de la Société Royale d'Edimbourg. (Tiré des Transactions de cette Société Tom. IV. )}
Il y a ceci de particulier dans les observations de Physique, c'est qu'on ne peut apprécier chacune d'elles, ni prévoir son influence, au moment où elle se présente : les plus brillants résultats font souvent cachés sous des faits en apparence minutieux. Il y a loin sans doute de l'attraction d'un morceau de fer par une pierre, à la découverte d'un monde nouveau ; de la chute d'une pomme dans un verger, aux grandes loix qui régissent le système planétaire ; du mouvement d'un brin de paille, aux appareils qui commandent le tonnerre ; de l'effervescence d'un acide versé sur un métal, à ces superbes aérostats qui transportent l'homme dans les airs ;\setcounter{page}{116} cependant ces découvertes fi difparates étoient intimément affociées; elles formoient, deux à deux, les extrémités d'une férie qui n'eût pas exiflé fi fon premier & prefqu'imperceptible chaînon n'eût pas exiflé lui-même. Ainfi un fait nouveau eft, pour un Phyficien, comme un billet dans une loterie où il joue à coup für, & où les lots deviennent des billets à leur tour; car une découverte conduit à une autre, par les rapports innombrables qui lient tout l'ensemble des événemens naturels.
Qui fait donc où nous conduira peut-être la petite lampe flottante du Dr. WILSON ? Il communique cette découverte dans une lettre à John PLAYFAIR fon collègue dans la Société Royale d'Edimbourg, le même dont nous avons analyfé le beau travail fur l'Aftronomie des Brahmines. Nous allons donner cette lettre à - peu - près textuellement.
"Les phénomenes dont je vais parler, dit l'auteur, m'étoient entièrement nouveaux il y a quelques mois, & je ne crois pas que perfonne y ait jamais fait attention, ou les ait décrits. Ce que j'appelle la lampe hydrostatique, n'eft autre chofe qu'une petite rondelle de papier ordinaire à écrire, d'environ ⅜ de pouce de diametre, portant au centre une mèche de coton filé, d'environ un quart de pouce de long : cette mèche eft enfilée dans un trou pratiqué au centre du papier. Ce petit appareil acquiert,\setcounter{page}{117} lorsque la mèche eft allumée & qu'il furnage dans de la très-bonne huile, certains mouvemens dont j'ai cherché à étudier la nature & la caufe."
"Un baffin de verre peu profond & cylindrique eft commode pour faire ces expériences. A l'inftant où la lampe eft allumée, elle fe meut affez rapidement dans une direction quelconque jufqu'à-ce qu'elle rencontre les parois du vafe; elle prend enfuite un mouvement circulaire, en s'appuyant toujours contre ces parois, & elle fait ainfi un grand nombre de révolutions."
"Quelquefois cette circulation fe fait de droite à gauche, d'autrefois dans la direction oppofée, felon que le point de la circonférence du papier qui a toujours été à l'avant dans la navigation libre & directe, fe trouve un peu à la droite ou à la gauche du point de cette même circonférence qui eft en contact avec les parois du vafe. On peut obferver cette tendance dans ce que j'appellerois volontiers le point directeur de la bafe, en remarquant la rotation partielle de la lampe autour de la mèche comme fur un axe, dès qu'elle arrive contre les bords du vafe. Quelquefois, mais rarement, le point directeur lui-même s'attache aux parois, & forme là comme un lien par l'effet bien connu de l'attraction capillaire qui agit en élevant le liquide à la fois contre la lampe & contre les côtés du vafe: lorfque ce lien correfpond ainfi au point\setcounter{page}{118} directeur, la lampe cesse de se mouvoir & paroît n'avoir plus aucune tendance à circuler."
"Lorsque la petite mèche est placée un peu excentriquement sur sa base, la lampe se meut de manière que la partie de la base qui se trouve la plus voisine de la mèche devient l'arrière ; & si la base est coupée de forme ovale & que la mèche, soit placée excentriquement sur le grand axe, l'extrémité de la base qui est la plus voisine de la mèche est postérieure aussi, lorsque la lampe traverse spontanément le bassin. De même, si la base est un triangle équilatéral & que la mèche, soit placée dans la perpendiculaire abaissée d'un angle sur le côté opposé, le sommet de cet angle ou la base seront en arrière selon que la mèche sera placée auprès de l'un ou de l'autre. Les lampes ainsi disposées circulent aussi à leur arrivée contre les bords du vase, quand le point directeur est situé à l'opposé, ainsi que cela arrive d'ordinaire."
"Quelle que soit la cause qui fait mouvoir la lampe droit en avant, la circulation perpétuelle qui a lieu dès qu'elle atteint les parois, semble provenir de la force qui avoit mis l'appareil en mouvement, & qui continue à agir, mais dans une direction inclinée à celle de l'attraction capillaire qui forme le lien : & il est évident que cette inclinaison sera plus grande ou moindre, selon que le point directeur sera plus ou moins éloigné du vase. Lorsque le point\setcounter{page}{119} directeur & le lien coincident, il sembleroit que ces deux forces devroient pouffer la lampe dans une direction perpendiculaire au côté du vase, & dans ce cas elle doit devenir immobile. C'est ce qu'on observe effectivement."
"J'ai eu l'occafion de remarquer ensuite que lorfque la lampe prend fon mouvement rectiligne, il paroît y avoir une répulfion très-active entre fa partie poftérieure & l'huile qui lui eft contigue. Cet effet eft très-vifible lorfqu'on a préalablement répandu un peu de pouffiere de charbon fur la furface de l'huile. A mefure que la lampe chemine, elle laiffe une plage derriere elle entièrement dégagée de pouffiere, parce que les particules font chaffées en arriere & latéralement avec un mouvement plus rapide que celui qui réfulteroit du fimple déplacement du mobile."
"Je désirois éprouver comment fe feroit la difperfion de la poudre de charbon, dans le cas où la lampe feroit stationnaire. Je conftruifis, dans ce but, une lampe avec une oublie à cacheter très-mince; j'y adaptai une mèche excentrique faite d'un fil de coton doublé; & pour empêcher que mon appareil ne prît feu, je garnissois d'une feuille d'or fa furface fupérieure. Lorfqu'on maintenoit cette lampe d'une maniere fixe à la furface de l'huile, la pouffiere fe retiroit dans toutes les directions & laiffoit l'efpace autour de la lampe parfaitement nettoyé. Mais\setcounter{page}{120} on pouvait observer que cette dispersion de la poussière par l'effet de la répulsion apparente de la base de la lampe, était beaucoup plus rapide du côté le plus voisin de la mèche qu'elle n'était partout ailleurs, & surtout, du côté diamétralement opposé à celui-là."
"Les circonstances dont on vient de faire mention paraissent expliquer suffisamment & le mouvement progressif de la lampe, & la loi générale de ce mouvement telle qu'on l'a décrite tout-à-l'heure. Car en considérant cette dispersion de la poussière seulement sous un point de vue général, & comme étant l'effet de quelque répulsion entre la base & l'huile qui la touche, les faits qu'on vient de rapporter indiquent clairement que, dans tous les cas, cette répulsion est la plus forte dans cette partie de la base qui en est la plus voisine de la mèche ou de la flamme. Et comme l'action & la réaction font toujours égales, & en sens opposé, la lampe doit être poussée dans la direction d'une ligne tirée au travers de la mèche vers la partie de la base qui en est la plus éloignée & où la réaction est la moindre."
"Mais pour obtenir une connaissance plus complète de la cause physique de ces mouvements, il me semblait nécessaire de faire quelques recherches plus particulières sur cette répulsion apparente qui s'exerce entre la base de la lampe & l'huile environnante, telle qu'elle\setcounter{page}{121} est annoncée par la dispersion de la poussiere dont on a parlé : & ici les considérations suivantes vinrent s'offrir."
"L'huile, tranquille dans un bassin, & d'une température uniforme, a toutes ses parties dans un état d'équilibre & de repos. Lorsqu'on allume la lampe, il est évident qu'on introduit une cause très-active qui tend à détruire cet équilibre : cette cause est la flamme qui s'alimente d'une partie de l'huile, & qui n'en est séparée que par une feuille de papier ou un pain à cacheter. L'huile, dans ces circonstances, à raison de la chaleur violente qu'elle éprouve doit se dilater subitement, & sa pesanteur spécifique se trouvant ainsi diminuée elle doit être soulevée par une force suffisante pour élever une partie de ce liquide au-dessus du niveau du reste. Mais cette portion chauffée, rencontrera, dans son effort pour se soulever, une résistance égale au poids de la lampe qui repose dessus, ce qui la forcera à s'échapper en glissant de dessous la base sous la forme d'un courant superficiel très-mince ; il paroît suivre de-là avec une égale certitude, que ce courant constant coulera le plus volontiers & le plus abondamment vers le côté de la base de la lampe où la résistance est la moindre, ou là où il y a le moins de chemin à faire pour pousser en avant ; c'est-à-dire depuis la mèche ou la flamme vers le bord de la base\setcounter{page}{122} qui se trouve le plus voisin, & c'est effectivement ce qui a lieu. Mais, d'après les lois du mouvement, il est certain que la réaction de ce courant d'huile raréfiée, sortant ainsi avec rapidité & abondance d'un côté particulier de la base doit pousser la lampe dans la direction contraire & la faire mouvoir dans celle qu'on lui a vu prendre. On peut encore remarquer que l'huile échauffée, en se retirant ainsi de dessous la flamme & cherchant à s'élever au-dessus du niveau commun, à raison de la diminution de sa pesanteur spécifique, peut soulever plus ou moins le côté de la base le plus voisin de la mèche, & aider à la réaction du courant qui recule en faisant voguer la lampe dans une direction opposée & pour ainsi dire le long d'un plan incliné."
"Les faits suivans paroissent prouver d'une maniere indubitable que l'huile raréfiée sous la base a réellement une tendance constante à s'élever au-dessus du niveau commun : savoir, 1°. lorsqu'une des lampes a brûlé pendant quelque temps & que sa base est bien garnie d'huile, si on éteint sa flamme, elle s'enfonce incontinent. 2°. Une lampe dont la base est faite d'une lame mince de talc vogue très-bien malgré sa pesanteur spécifique jusqu'à ce que sa flamme soit éteinte : alors elle s'enfonce immédiatement."
"En confirmation de l'explication que j'ai\setcounter{page}{123} tenté de donner, je trouvai que lorsqu'on produisoit à la surface de l'huile avec chaleur locale, en approchant très-près de cette surface l'extrémité d'un fer obscurément rougi, on ne tardoit pas à voir une effluence qui rayonnoit de tous côtés en s'écartant du fer; la poussière de charbon étoit chassée sous la forme d'un cercle qui s'étendoit de plus en plus jusqu'à ce qu'enfin toutes ses particules se trouvoient réunies contre les bords du bassin."
"Lorsque l'huile, dans cette expérience étoit peu profonde, en mêlant à ce liquide des feuilles d'or réduites en paillettes, on pouvoit observer un courant inférieur qui se mouvoit dans un sens opposé au précédent, c'est-à-dire, en se rapprochant du fer rouge dans toutes les directions, & en s'élevant ensuite. Mais cette tendance générale par laquelle toutes les parties du fluide se meuvent pour chercher l'équilibre, se prouve d'une manière assez récréative par l'expérience suivante. On prend une tasse ordinaire qu'on remplit, à-peu-près, d'eau pure: on verse sur cette eau une petite cuillerée de bonne huile très-claire dans laquelle on a mêlé des paillettes d'or. Si l'eau est froide, l'huile versée lentement & de suite sur le centre du vase s'étend sur sa surface sous la forme d'une lentille & n'atteint pas les parois. Si l'on met sur cette lentille une petite lampe allumée, elle vogue en circulant comme elle le fait dans un plus grand\setcounter{page}{124} bassin. Si on la fixe un moment, on voit les paillettes continuellement chauffées de la lampe vers les bords de la lentille, puis revenant ensuite par le fond jusques sous la lampe, où elles montent, puis sont entrainées de nouveau rapidement par le courant divergent supérieur, jusqu'à-ce qu'arrivées à une certaine distance elles s'enfoncent & reviennent vers le centre \footnote{Cette expérience confirme la théorie par laquelle nous avions essayé d'expliquer un fait singulier observé par le Comte de Rumford; savoir, que lorsqu'on fait fondre du suif en plaçant un peu au-dessus de sa surface, un boulet rouge, le bassin de suif solide qui renferme la partie liquide, au lieu d'être de forme concave, est convexe au milieu. Voyez T. VIII. Sc. & Arts, p. 228 (R)}.
"Lorsqu'une rondelle de papier, ou une oublie, ou un corps léger quelconque surnage à l'huile du bassin, si l'on en approche la pointe d'un fer chaud, ce corps quitte à l'instant sa place, & s'enfuit comme s'il éprouvoit une répulsion; mais, en réalité, par l'effet du courant divergent produit à la surface de l'huile par la chaleur qui lui est communiquée."
"De même, si l'on jette sur de l'huile de térébentine, sur de l'éther, de l'alcool, ou l'un quelconque des fluides inflammables & légers, une oublie fort échauffée, elle se mettra tout de suite en mouvement jusqu'à-ce qu'elle\setcounter{page}{125} soit refroidie ; alors, le courant qui paroissoit sortir de l'oublié en abondance , cesse d'exister. D'autres liquides, tels que le rum, le suif fondu, la cire d'abeilles & la résine font aussi paroître une effluence continuelle à leur surface, par l'application locale de la chaleur, & on y observe les mêmes phénoménes qu'offrent les petites lampes allumées & nageant sur l'huile. Il est à remarquer cependant , que cette propriété semble appartenir à tous les fluides de nature inflammable , & qu'elle ne se manifeste point dans l'eau lorsqu'on applique une chaleur locale à sa surface."
"Car si l'on approche un fer rouge vers la surface de l'eau d'un bassin, couverte de poussière de charbon, ces molécules ne fuient point comme elles le sont lorsqu'elles surnagent à l'huile, mais elles paroisssent seulement acquérir un mouvement circulaire lent & irrégulier qui s'étend peu-à-peu, tandis que ces mêmes molécules conservent à-peu-près leurs places relatives. La même chose arrive lors même qu'on fait arriver la pointe du fer jusqu'à l'eau, qu'on entend frémir au moment de l'immersion."
"Je ne sais trop comment expliquer ce fait, à moins qu'il ne soit la conséquence de la moindre expansibilité de l'eau par la chaleur, comparée à celle des liquides inflammables ; la dilatation peut être assez peu considérable pour ne point rompre l'équilibre & ne produire aucun\setcounter{page}{126} mouvement dans le liquide sous le solide chaud qu'on approche. Il se pourroit encore, que les particules d'eau transmettent à leurs voisines l'excédent de chaleur qu'elles reçoivent, plus rapidement que ne le font les particules des fluides inflammables en pareilles circonstances, & qu'elles résistent ainsi à acquérir la haute température nécessaire au degré d'expansion qui romproit l'équilibre & produiroit un courant divergent. Il n'est pas besoin de dire que le maximum de cette température ne peut dans aucun cas (sous la pression atmosphérique ordinaire) dépasser le terme de l'eau bouillante, soit le 212° degré du thermomètre de Fahrenheit."
"Je ne pus douter cependant, que l'équilibre entre les molécules de l'eau ne fût troublé par l'application locale de la chaleur ; mais dans un degré beaucoup moindre qu'on ne l'observe dans les liquides inflammables : voici l'expérience qui me conduisit à ce résultat. Je fis nager sur l'eau une petite tasse, formant une lampe garnie d'huile, & dont la mèche étoit placée en dedans, un peu au-dessous du bord de la tasse, & de niveau avec l'eau extérieure. Cet appareil ainsi disposé acquéroit spontanément un mouvement, peu rapide il est vrai, mais dans lequel la partie de la lampe dont la flamme étoit la plus voisine, formoit toujours l'avant de la petite flottaison. La même tasse,\setcounter{page}{127} fortie de l'eau & mise à voguer dans un baffin rempli de fort rum, se mouvoit beaucoup plus vite, & selon la même loi."
J'ai l'honneur d'être, &c.
PAT. WILSON.
P. S. "Si vous aviez l'intention de répéter quelques-unes des expériences, il faudroit donner quelqu'attention aux observations suivantes. Le fil de coton que j'employois pour les mèches étoit de cette espèce fine & flexible dont on se sert pour broder la mousseline. Après avoir percé la base de la lampe on passe dans le trou un bout de ce fil, on le coupe fort court par dessous & on force avec une épingle la bavure de la carte contre le coton pour qu'il se maintienne en place. On coupe ensuite le fil au-dessus, en laissant à la mèche une longueur d'environ un quart de pouce; & la lampe est construite. On la pose doucement sur l'huile, en la tenant par la mèche, pour que la base de papier ne soit point tourmentée & conserve sa forme plane; on touche la mèche avec une goutte d'huile & la lampe est prête à être allumée. Un bout de ficelle qu'on a préalablement trempée dans l'huile, est très-commode pour allumer & ne laisse rien tomber sur la surface comme le feroit une chandelle ou une bougie."
"Pour que la lampe prenne un mouvement\setcounter{page}{128} de rotation bien décidé, il faut que l'huile soit très-pure, & en plein contact avec les parois du vase. L'huile & le bassin qui le renferme doivent être à la même température, entre 55 à 60 de Fahrenheit (11 & 12 ½ R.) Car si une partie du bord du vase est de beaucoup plus chaude que le reste, la lampe, en y arrivant, abandonnera cet endroit, poussée par le courant qui semblera partir du lieu le plus échauffé."
"Quelquefois la lampe en voguant dévie un peu de sa direction ordinaire, parce que la forte chaleur de la flamme fait voiler le papier, & changeant ainsi sa forme, rend le courant plus abondant dans certains endroits que dans d'autres."
"On peut observer quelquefois dans le suif fondu qui environne la mèche d'une chandelle ordinaire, des atômes que les mouchettes y ont laissé & qui se meuvent continuellement en s'éloignant & se rapprochant de la flamme. Quelques Physiciens ont considéré ces mouvements comme étant l'effet d'attractions & répulsions électriques. Il semblerait plutôt qu'on doit les attribuer à des courants opposés qui ont lieu, dans la graisse fondue, à sa surface & immédiatement au-dessous; ainsi qu'on l'a expliqué précédemment."

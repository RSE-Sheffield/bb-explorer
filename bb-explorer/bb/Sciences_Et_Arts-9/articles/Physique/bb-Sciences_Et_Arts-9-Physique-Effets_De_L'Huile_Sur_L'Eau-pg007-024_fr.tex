\setcounter{page}{3}
\chapter{PHYSIQUE}
\section{SOME OBSERVATIONS ON THE PHENOMENA WHICH TAKE PLACE BETWEEN OIL ET WATER, &c. Observations sur les phénoménes qui ont lieu quand on jette de l'huile sur de l'eau, adressées en forme de lettre au Dr. THOMAS PERCIVAL, Membre de la Société Royale de Londres, &c. par le Dr. MARTIN WALL, Professeur de Chimie dans l'Université d'Oxford; avec une réponse du Dr. PERCIVAL, & une replique du Dr. WALL; tirées du 2d. vol. des mémoires de la Société Littéraire & Philosophique de Manchester. — On a joint à cet Extrait celui d'une lettre relative au même sujet, adressée par Mr. RICHARD PATTERSON au Dr. BENJ. RUSH de Philadelphie & tirée du 3e. vol. des Transactions Américaines.}
Si l'on verse de l'huile sur de l'eau agitée; l'huile se répand à l'instant sur la surface de l'eau & la recouvre comme une fine pellicule, dont l'épaisseur est proportionnée d'une part à la quantité de l'huile, & de l'étendue de la surface qu'elle tapisse. Mais ce qu'il y a de très-singulier, c'est que l'agitation de l'eau...
\comment{pages 4-5 missing}
\setcounter{page}{6}
Le principal agent de la nature dans toutes ses opérations paroît être l'attraction que toutes les parties de la matiere ont les unes pour les autres. Mais les phénoménes de la gravitation, de la cohésion, de l'électricité, du magnétisme, & des affinités chimiques, montrent que ce principe est susceptible de différentes modifications, qu'il n'obéit pas toujours aux mêmes loix.
On voit souvent deux corps qui s'attirent mutuellement, être repoussés par un troisieme, & même en certaines circonstances se repousser entr'eux. Pour expliquer ces disparates, dont l'électricité & la magnétisme offrent surtout plusieurs exemples, les philosophes ont eu recours à un principe agissant en sens contraire de l'attraction, & dont ils supposent que l'action commence là où celle de l'attraction finit. Ce principe qu'on a appelé le principe de répulsion a été admis même en chimie; & quand deux corps ne montrent aucune disposition à se réunir, on leur attribue assez généralement une faculté répulsive l'un à l'égard de l'autre. Ce seroit pousser peut-être l'hétérodoxie un peu trop loin que de nier totalement en physique l'existence d'un principe de répulsion; mais il y a lieu de croire qu'en chimie; on en a porté l'application trop loin, & que dans les cas où l'on n'apperçoit aucune affinité entre deux corps, il n'est pas nécessaire de supposer entre eux un principe contraire.
\setcounter{page}{7} Ainsi, par exemple quoique l'huile ne se mêle point avec l'eau sans un intermede, il ne s'ensuit pas qu'il y ait entr'eux une vraie répulsion. C'est peut-être tout simplement un de ces cas auxquels on a donné le nom d'astraction élective; & pour expliquer cette immiscibilité, il suffit de supposer que les parties de l'eau s'attirent entre elles avec plus de force qu'elles n'attirent celles de l'huile, & réciproquement. On sent qu'il doit nécessairement résulter de cette supposition que si l'on verse de l'huile sur de l'eau, ou même si l'on agite fortement ensemble ces deux fluides, les particules de l'huile tendront à se réunir entr'elles & à se séparer de l'eau, surtout si celles de l'eau ont la même tendance à se séparer de l'huile.
Appliquons cette théorie à quelques-uns des phénomènes qui résultent du contact de ces deux fluides. Le Dr. Nash nous apprend dans son histoire du comté de Worcester (Nash's hist. of Worcestershire; vol. I. pag. 300) que dans la fabrique de sel commun à Droitwich; on jette dans la chaudière qui contient la saumure un morceau de résine pour rendre le grain du sel moins grossier. Il ajoute que Mr. Romney l'un des principaux propriétaires de cette fabrique lui a assuré que plus on mettoit de résine, plus le grain du sel étoit fin, à tel point que si l'on en mettoit seulement la grosseur de deux noix, le sel ne se cristalliserait\setcounter{page}{8} point du tout. Le beurre, le suis & toute autre substance huileuse susceptible de se liquefier par la chaleur nécessaire pour faire bouillir la saumure, ont le même effet. Elles se fondent sans se mêler à l'eau, mais se répandent sur sa surface, sous la forme d'une pellicule d'autant plus épaisse qu'on en a mis davantage. Or on sait que la cristallisation ne se fait bien que lorsque l'eau s'évapore lentement & présente une grande surface à l'air, parce que l'air est aussi nécessaire que l'eau à la formation des cristaux, dont il est comme elle une des parties constituantes. Si donc la communication de l'air avec l'eau est interrompue, comme elle doit l'être par l'interposition de la lame huileuse dont nous venons de parler, la cristallisation se fera mal, le sel se déposera en très-petits cristaux au fonds de la chaudiere, & son grain sera d'autant plus fin que l'interception sera plus complète.
Le fait cité par le Dr. Franklin & si bien décrit par lui du calme qui résulte de l'huile répandue sur la surface de la mer agitée, est susceptible d'une explication parfaitement analogue. On sait que l'air a pour l'eau une grande attraction. Quand ces deux fluides sont en contact, ils adhérent & se réunissent l'un à l'autre avec la double force d'une affinité chimique & d'une cohésion mécanique, tellement qu'on les trouve rarement l'un sans l'autre, l'air contenant\setcounter{page}{9} toujours plus ou moins d'eau, & l'eau plus ou moins d'air. C'est pourquoi, lorsque dans une tempête, une masse considérable d'air est poussée avec une grande force sur la surface de l'eau, l'air se saisit pour ainsi dire de l'eau, l'entraîne avec lui dans son cours, & ne s'en sépare que lorsque la masse d'eau soulevée réagit par son poids, & retombe pour prendre son niveau. C'est par ces impulsions & ces réactions fréquemment répétées que l'eau contracte enfin cette violente agitation qui constitue la tempête. Mais si on jette de l'huile sur la surface de l'eau, elle se répand sur le champ au loin en une pellicule non interrompue, qui intercepte absolument la communication de l'eau avec l'air, ensorte que ce dernier fluide n'ayant que peu ou point d'attraction pour l'huile ne mord point sur elle, glisse sur sa surface, & n'y occasionne aucune agitation, aucun tumulte.
Il n'est pas besoin de supposer pour cela aucune force répulsive entre l'huile & l'eau. Il suffit que les parties de l'huile ayent entr'elles une forte attraction, & qu'elles n'en ayent que peu ou point pour l'eau ou pour l'air. On peut même conclure directement de la facilité & de la promptitude avec laquelle l'huile se répand sur la surface de l'eau qu'elle n'en est pas repoussée. Si elle l'étoit, elle fujroit, pour ainsi dire, le contact de l'eau, elle se réuniroit en globules, & fujroit vers les bords, au lieu\setcounter{page}{10} que si l'on ne suppose aucune répulsion pareille: elle doit naturellement par sa légèreté spécifique, surnager sur l'eau, & suivant les lois de l'hydrostatique, se répandre également au loin pour reprendre son niveau.
Cette propriété que l'huile a de calmer l'agitation de l'eau sur laquelle on la répand est accompagnée d'une autre propriété non moins remarquable, qui est de la rendre plus transparente, soit parce que la plus légère ondulation de l'eau tendant toujours à troubler sa limpidité, l'huile la rétablit en faisant cesser l'agitation, soit parce que l'addition de l'huile modifie la réfraction de la lumière. Quoiqu'il en soit le fait est certain. Bien que peu observé par les philosophes, il paroît avoir été connu par tous les peuples du monde. Pline en parlant de la mer dit expressément que "l'huile a la propriété de calmer son agitation & d'augmenter sa transparence". Il ajoute que" c'est" par cette raison que les plongeurs ont soin" d'en tenir dans leur bouche pour la répandre sur l'eau. \footnote{Omne oleo tranquillari, & ob id urinantes are spargere, quoniam mitiget naturam asperam, lucemque deportet.} "Plutarque en fait le sujet d'une de ses questions de philosophie naturelle." Quelle est la cause du calme & de la transparence de l'eau arrosée d'huile"? \footnote{Δια τι της θαλαττης ελαιω καταρραινομενης γινεται καταβενεια και γαληνη.} On\setcounter{page}{11} trouve dans le naufrage d'Erafmé ce paffage fingulier. En parlant des divers efforts que faifoient les matelots dans la tempête. "Quelques-uns, dit-il, se prosternoient, adorant la mer, rappelant de vieux contes & jetant à grands flots fur les ondes tout ce qu'ils pouvoient trouver d'huile fur le vaisseau."\footnote{Nonnulli procumbetes in fabulas adorabant mare, quicquid erat olei effundentes in undas.} Dans l'édition des Elzevirs, ce paffage est éclairci par une note dans laquelle on remarque que "l'huile a la propriété de donner de la lumiere, & de répandre partout le calme, même fur la mer, le plus implacable des élémens."\footnote{Ea natura est olei, ut lucem afferat, ac tranquillet omnia, etiam mare, quo non aliud elementum implacabilius.}
Et ce n'est pas ici un objet de pure spéculation. Les pêcheurs de la côte de Provence, ceux qui habitent les bords du Tage près de Lisbonne, les habitans des isles Hébrides, & jusqu'à ceux de St. Kilda, la plus reculée de ces isles, ont recours au même moyen pour distinuer avec plus de facilité les moules & autres coquillages au fonds de la mer. Enfin dans le Register annual\footnote{Journal historique très-estimé qui se publie à Londres toutes les années.} pour l'an 1760, c'est-à-dire 15 ans avant la publication du mémoire du Dr. Franklin, on trouve un paffage curieux,\setcounter{page}{12} probablement copié des papiers du jour. "Dans le terrible incendie qui eut lieu il n'y a pas longtemps sur les bords de la Tamise, on remarqua que l'huile qu'on avoit jeté dans la rivière pour empêcher les progrès du feu, avoit visiblement calmé l'agitation des vagues. Il paroît que cette propriété de l'huile de calmer la surface de l'eau, est connue depuis long-temps. Une ancienne loi prescrivoit dans les cas de tempête, où l'on est obligé de jeter les marchandises à la mer pour alléger le vaisseau, de commencer par l'huile, s'il s'en trouvoit à bord. Les habitants de Raguse font encore aujourd'hui dans l'usage, lorsqu'ils vont à la pêche au harpon, de jeter de l'huile sur l'eau avec un arrosoir pour mieux voir jusqu'au fonds. Les ouvertures que forment ainsi sur l'eau les gouttes d'huile, portent dans leur pays le nom de fenêtres."
Telle est la substance du Mémoire du Dr. Wall. Voici celle de la réponse du Dr. Percival.
1°. Il commence par rappeler le phénomène décrit par le Dr. Franklin relatif à l'agitation de l'eau sous l'huile. Si l'on verse de l'eau dans un verre ordinaire jusqu'au tiers de sa hauteur, & qu'on y verse ensuite autant d'huile; qu'on suspende le verre par une ficelle nouée à la distance d'un pied au-dessus du verre, & qu'en tenant la ficelle par le nœud, on lui fasse éprouver de légers balancemens pareils à ceux\setcounter{page}{13} qu'il éprouveroit dans un vaisseau dans lequel il feroit suspendu au plafond, on verra l'huile à la surface parfaitement unie & tranquille, tandis que l'eau paroîtra au-dessous dans une grande agitation. Mais si l'on verse l'huile, & qu'on la remplace par de l'eau; la totalité du fluide paroîtra immobile, & la surface demeurera parfaitement tranquille. "J'ai montré cette expérience à plusieurs personnes," dit le Dr. Franklin. "Ceux qui n'avoient qu'une connoissance superficielle des principes de l'hydrostatique ont cru d'abord être en état d'expliquer le phénomène. Mais l'explication qu'ils en ont donnée, a beaucoup varié & m'a paru inintelligible. Ceux qui étoient plus profondément versés dans cette science ont trouvé le fait singulier, & ont promis d'y réfléchir. Il mérite en effet qu'on y réfléchisse. Car un phénomène inexplicable par nos vieux principes peut nous en fournir de nouveaux dont on tirera peut-être quelque parti pour la solution d'autres problèmes obscurs dans les sciences naturelles." Le Dr. P. attribue ce phénomène, non sans quelque défiance, à la répulsion qui a lieu entre les particules de l'huile & celles de l'eau. Il rapporte cette répulsion aux oscillations de l'éther subtil qui, suivant Sir Isaac Newton, traverse tous les corps, oscillations qui en augmentant sa densité, doivent aussi augmenter sa force élastique.\setcounter{page}{14} Le Dr. Franklin ne trouveroit peut-être pas cette explication fort intelligible. Mais ce qui l'est davantage, c'est le parti que le Dr. tire du fait même pour expliquer l'effet des matieres huileuses qu'on répand sur la surface de la saumure en ébullition. L'agitation de la saumure lui paroît devoir être augmentée, comme celle de l'eau dans le verre, par la lame huileuse dont on la recouvre, & il n'en faut pas davantage pour rompre la cristallisation du sel.
2°. Tout le monde connoît les effets suffocans de l'air chargé des exhalaisons de la graisse brûlée, ou de la mouchure de chandelle. Quand on inspire ces vapeurs, on éprouve dans la poitrine une sensation de conflit très-différente de celle que procure l'inspiration de l'acide carbonique ou de l'air inflammable. L'explication la plus simple qu'on puisse donner de ce phénomene n'est-elle pas que l'air abandonnant les parties huileuses pour se réunir aux parties aqueuses avec lesquelles il entre en contact, une grande force répulsive succede à l'attraction qui le réunissoit aux premieres? Car comme on voit en algebre les quantités négatives commencer lorsque les quantités affirmatives s'évanouissent, de même aussi en physique la répulsion commence quand l'attraction finit.
3°. On a un grand nombre de preuves de\setcounter{page}{15} cette force de répulsion. Tout le monde connoît l'éclat que jettent les gouttes de pluye qui tombent fur les feuilles de chou. On a démontré qu'il provient d'une abondante réflexion de lumière par la surface inférieure & aplatie de ces gouttes. Or c'est ce qui ne pourroit pas avoir lieu s'il y avoit un contact parfait & sans intervalle entre la feuille & la goutte. Puis donc qu'il y a un intervalle entre elles, il y a répulsion, & c'est ce qui explique la volubilité de ces gouttes, & leur chûte le long de la feuille, sans laisser aucune trace d'humidité après elles. --- Un autre exemple encore très-évident de répulsion, est celui d'une aiguille d'acier poli qu'on peut faire flotter sur l'eau, parce que si on la pose doucement, elle n'entre pas en contact avec l'eau, mais se fait sur sa surface par sa force répulsive un lit dont la concavité est beaucoup plus grande que le volume de l'aiguille.
4°. Les attractions & les répulsions qui ont lieu entre la rosée & certaines substances font encore plus remarquables. On connoit les expériences par lesquelles Muschembroeck a prouvé que certains corps l'attirent avec beaucoup de force, tandis que d'autres la repoussent. Ces expériences ont été répétées par l'académicien Dufay qui a fait voir que la rosée tombe abondamment dans les vases de verre ou de porcelaine ; mais absolument point dans\setcounter{page}{16} ceux de métal, surtout si le métal est bien poli. Si l'on place à la même exposition, & à côté l'une de l'autre, deux soucoupes, l'une de porcelaine sur un plat d'argent, & l'autre d'argent sur un plat de porcelaine, la soucoupe & le plat de porcelaine se trouvent couverts de rosée, tandis qu'il n'en tombe pas une goutte sur la soucoupe ou le plat d'argent. Il y a plus. Deux verres de montre parfaitement semblables furent placés l'un au milieu d'un plat d'argent & l'autre au milieu d'un plat de porcelaine, en entourant le premier d'un anneau d'argent bien poli pour que le côté convexe du verre fût bien à l'abri de la rosée. Au bout de quelques jours le second verre se trouva contenir cinq ou six fois plus de rosée que le premier. Elle étoit dans celui-ci ramassée en petites gouttes dans le centre ; & ces gouttes devenaient toujours plus petites en approchant du bord, autour duquel, dans toute la circonférence intérieure du verre, il y avait un espace parfaitement sec. Cette expérience fut répétée 30 fois, toujours avec le même succès. Le Dr. Watson, aujourd'hui Evêque de Landaff, en a depuis peu confirmé les résultats par une suite d'expériences très-curieuses faites dans le but de déterminer la quantité des vapeurs qui s'élèvent dans un espace donné de la surface de la terre. Voici ses propres paroles : "au moyen d'un peu de cire je fixai un petit \setcounter{page}{17} écu au milieu du côté concave d'un verre, de manière qu'il en fut très-près sans le toucher. Je plaçai le verre sur du gazon, avec sa convexité en-déhors. Il fut à l'instant couvert de vapeurs partout ailleurs que dans le voisinage de l'écu. Non-seulement l'écu en était lui-même parfaitement exempt; mais il avait empêché les parties voisines de son bord d'en recevoir. Car entre la partie du verre située immédiatement au-dessous, il y avait tout autour de l'écu sur le verre, un anneau large d'un quart de pouce, parfaitement sec, comme si l'écu avait repoussé les vapeurs jusques-là. Un grand pain à cacheter rouge, un cercle de papier blanc, & plusieurs autres substances, dont il serait ennuyeux de faire l'énumération, eurent exactement le même effet &c. (Watson's chemical Essays: V. III. p. 64). Tels sont les faits d'après lesquels le Dr. se croit autorisé à regarder le principe de répulsion comme un puissant agent dans les opérations de la nature. C'est à lui, dit-il, que l'air que nous respirons doit probablement son existence & son élasticité, la lumière qui nous éclaire son mouvement rapide, & ses diverses inflexions; & le feu son énergie vivifiante. Cela posé, l'analogie permet-elle qu'on exclue la répulsion de cette branche de la physique que comprend la chimie? c'est une queffique que comprend la chimie? c'est une question\setcounter{page}{18} que je soumets de nouveau au génie philosophique du Dr. Wall.
Celui-ci accepte le défi, & le soutient avec un ton de civilité rare entre les Savans. Il commence par déclarer qu'il est fort éloigné de nier absolument le principe de répulsion en physique. Ce principe lui paroît incontestablement devoir être admis pour expliquer les phénomènes de l'électricité & du magnétisme. Mais il soutient qu'en chimie on n'en a pas besoin & que les répulsions apparentes peuvent toujours s'y résoudre en de simples attractions électives. 
Il écarte ensuite de la question le sentiment de suffocation, qu'on éprouve en respirant les vapeurs de l'huile brûlée, comme n'ayant rien de commun avec le principe de répulsion entre l'huile & l'eau, & comme s'expliquant très-naturellement par la phlogistification de l'air imprégné de ces vapeurs qui le rendent non-feulement peu propre à déphlogistiquer le sang & à se convertir en air fixe, mais encore très-irritant en lui-même. Le Dr. W. paroît ici, comme en quelques autres endroits de son mémoire, encore imbu de l'ancienne théorie. Il ne parle ni de la décomposition de l'air dans les poumons, ni de l'absorption de l'oxigène par le sang. Il suppose que le principal effet de la respiration est d'enlever au sang le phlogistique surabondant. Mais son raisonnement n'en est pas moins concluant. Il suffit d'admettre que l'air\setcounter{page}{19} Impégné de ces vapeurs ne peut servir à la respiration, & qu'elles sont elles-mêmes très-irritantes; & c'est ce qui ne peut gueres se nier, pour expliquer parfaitement bien le phénomene, lors même qu'au lieu de répulsion on supposeroit une attraction très-forte entre ces vapeurs & l'humidité naturelle des poumons.
Le Dr. W. n'admet point en chimie les lois de l'attraction mécanique, encore moins les propriétés des quantités algébriques, & il regarde avec le Dr. Lewis comme un point très-important de doctrine, la distinction des deux espèces d'attraction qu'on observe l'une dans les phénomenes mécaniques, & l'autre dans les phénomenes chimiques. On ne peut sans s'exposer à de grandes erreurs appliquer à l'une les lois de l'autre.
Les gouttes de pluie qui tombent sur la feuille de chou sans la mouiller l'ont d'abord ébranlé en faveur du principe de répulsion; mais en y réfléchissant, il a vu que le même effet a lieu lorsque l'eau tombe goutte à goutte sur des substances en poussière avec lesquelles elle a d'ailleurs beaucoup d'affinité; comme la farine, surtout si ces substances sont sur un plan incliné: Qui fait si les feuilles de chou ne sont pas ordinairement recouvertes de la poussière qui flotte dans l'air, ou de quelque farine résineuse, ou de quelques gouttes imperceptibles de rosée huileuse, qui par leur sphéricité empêchent les\setcounter{page}{20} gouttes d'eau d'entrer en contact avec la feuille & les font rouler en globules par l'effet de l'attraction intrinsèque & réciproque de leurs parties? \footnote{Cette conjecture se trouve vérifiée par les belles Observations de Dumont de Courset sur les substances glauques, observations qui ont été publiées par Boucher dans le Journal de Physique pour le mois d'Avril 1798, page 279. (R)} C'est ainsi que le mercure roule en globules, non-seulement sur le bois ou le marbre, mais encore sur d'autres substances avec lesquelles il a beaucoup d'affinité, telles que l'étain & le cuivre, & même l'or & l'argent, quoique l'affinité que ces deux derniers métaux ont avec lui, surmonte beaucoup plus promptement l'attraction intrinsèque & réciproque de ses parties, qui le force à se mettre en globules. En général c'est à cette attraction réciproque que toutes les parties d'un liquide quelconque ont entr'elles, qu'est due leur tendance à se mettre en globules, lorsque la quantité en est peu considérable; l'attraction même plus grande que d'autres corps ont pour elles ne surmonte cette tendance & ne force les gouttes à s'aplatir & à se répandre sur ces corps qu'au bout d'un temps plus ou moins long, proportionnément à la force de cette dernière attraction.
Quant à l'expérience de l'aiguille flottante, le Dr l'a examinée avec la plus grande\setcounter{page}{21} attention, tant à l'œil nud qu'avec une bonne loupe, & il s'est convaincu qu'il n'est pas exact de dire que l'aiguille ne touche l'eau en aucun point. Il y a plusieurs points de contact ; mais il est vrai que le contact n'est pas général, ce que notre auteur explique, d'après les principes de l'ancienne théorie chimique, en faisant remarquer que l'eau a peu d'attraction pour les corps qui contiennent beaucoup de phlogistique, particulièrement lorsque la surface de ces corps est bien polie, de manière à ne mettre aucune entrave à celles de leurs propriétés qui dépendent du phlogistique. Tous les métaux, mais surtout l'acier bien poli font dans ce cas là. L'eau a de la peine à les mouiller. Elle n'adhère point à leur surface, ou si elle y adhère, ce n'est que par places & d'une manière interrompue. Que le fait dépende ou non du phlogistique, peu importe. Il n'en est pas moins curieux. Faut-il pour l'expliquer avoir recours à un principe de répulsion ? L'auteur n'en voit pas la nécessité. Une attraction foible ou nulle lui paroît suffisante pour en rendre raison.
Enfin quant au fait de l'agitation de l'eau sous l'huile, cette agitation contribue sans doute à rompre la cristallisation du sel, mais elle ne suffit pas pour expliquer l'utilité de l'huile sur la saumure, parce que si l'agitation seule de l'eau étoit la cause de la finesse du grain, il suffiroit de tenir constamment la faumure agitée\setcounter{page}{22}  par l'ébullition pour produire le même effet ; ce qui est contraire à l'expérience.
Au surplus, il n'est pas besoin de recourir à un principe de répulsion entre les deux fluides pour résoudre le problème proposé à cet égard par Franklin. Il s'explique facilement par leur différence de gravité spécifique qui leur fait recevoir l'impulsion dans des proportions inégales, et par la légèreté de l'huile, légèreté qui lui maintient constamment sa position à la surface, quelle que soit l'agitation de l'eau au-dessous.
Mais, dit à cette occasion Mr. Patterson, si cette explication était juste, il s'ensuivrait qu'en mettant sous l'huile du mercure au lieu d'eau, l'agitation de ce fluide à la moindre impulsion serait beaucoup plus grande, puisqu'en raison de sa gravité spécifique bien supérieure à celle de l'eau, il recevrait l'impulsion dans une proportion beaucoup plus grande. C'est précisément le contraire de ce qui arrive. L'agitation du mercure sous l'huile est très-légère. De plus, si l'on verse deux fluides de gravité spécifique inégale dans deux verres différents, et qu'on les mette en mouvement avec le même degré de vitesse, on trouvera l'agitation du fluide le plus pesant, moindre que celle du fluide le plus léger; le premier aura plus de tendance à maintenir son niveau; l'explication du Dr. ne peut donc pas se soutenir. Mais on peut en donner\setcounter{page}{23} une autre beaucoup plus plausible.
Lorsque la gravité spécifique d'un corps plongé dans un fluide n'est pas de beaucoup supérieure à celle de ce fluide, sa tendance de bas en haut dans le fluide est peu inférieure à sa tendance de haut en bas, & une légère impulsion suffit pour lui donner la première direction. C'est le cas de l'eau plongée dans l'huile. Une impulsion aussi légère que celle des balancements d'un vaisseau, suffit pour la faire ondoyer & lui donner l'agitation dont parle le Dr. Franklin.
Si cette explication est juste, l'agitation du fluide inférieur sera d'autant plus grande, que sa différence de gravité spécifique avec le fluide supérieur sera plus petite & réciproquement. Aussi voit-on que l'agitation du mercure sous l'eau est très-petite; & celle de l'eau sous l'huile beaucoup plus grande; mais elle est surtout très considérable entre l'huile & l'esprit-de-vin, parce que la gravité spécifique de ces deux fluides est à-peu-près égale. D'après ce principe, elle doit être presque nulle entre l'eau ou l'huile & l'air, & c'est ce qui, beaucoup mieux que la ténacité de ses parties, explique la tranquillité de la surface supérieure de l'huile dans l'expérience de Franklin. Car si on fait flotter sur la surface de l'huile un liquide beaucoup plus pesant que l'air, mais un peu plus léger que l'huile, tel que l'éther, la surface de l'huile est agitée à la moindre impulsion\setcounter{page}{24} comme le feroit celle d'un autre fluide.
Voilà pourquoi, ajoute Mr. P., lorsqu'en hiver nous mettons notre biere auprès du feu pour la réchauffer, il suffit de balancer un instant le pot qui la contient pour mêler parfaitement les parties froides avec les parties chaudes, & pour rendre sa température uniforme, parce qu'il n'y a qu'une très-petite différence de gravité spécifique entr'elles.
Qu'il nous soit permis de remarquer en finissant que c'est probablement là ce qui a fait pendant si long-temps illusion aux Physiciens sur le prétendu pouvoir attribué par eux aux fluides comme aux solides de conduire la chaleur, illusion que le Comte Rumford vient de dissiper, en démontrant que les fluides sont absolument privés de ce pouvoir. Mais puisque la moindre impuslion suffit pour mêler leurs différentes parties, ils n'en ont pas besoin.

\setcounter{page}{297}
\chapter{PHYSIQUE}
\section{ESSAIS ON THE MICROSCOPE, &c. Essais sur le Microscope \footnote{Voyez le titre détaillé, pag. 197 de ce volume, en tête du premier extrait.}, par feu G. ADAMS, seconde édition, avec des additions considérables par F. KANMACHER, in-4°. grand format, 714 pag. avec trente-deux planches folio. Londres W. & S. Jones, Holborn, 1798; prix 1 liv. sterl., 8 sh. broché. \large{(Dernier Extrait.)}}
LES huit derniers chapitres de l'ouvrage d'ADAMS, c'est-à-dire, au moins les deux tiers du livre, devroient porter un autre titre que celui d'Essais sur le microscope. Ils renferment l'exposé des principales découvertes faites par divers observateurs à l'aide de cet instrument: c'est plutôt une Contemplation de la nature microscopique: & sous ce point de vue, l'ouvrage peut intéresser un plus grand nombre de lecteurs. Nous allons en extraire les faits les plus remarquables, & l'innombrable tribu des insectes nous les fournira presque seule.
\setcounter{page}{298} On décrit plutôt qu'on ne définit un insecte en disant que "c'est un animal dont la tête est garnie d'antennes; qui n'a point d'os, mais dont la peau est plus ou moins dure; qui a fix pieds ou davantage; & qui respire par des ouvertures latérales."
Il y a cette différence très-notable dans l'organisation générale des Insectes, comparée à celle des quadrupèdes, des oifeaux; & des poiffons; c'est que ces trois classes d'animaux ont un fquelette offeux auquel les muscles font attachés; au lieu que tout l'intérieur de l'insecte est composé de matiere molle, & que les muscles s'attachent à un fquelette extérieur qui remplit le double office d'os & de peau.
La plupart des Naturalistes considerent quatre parties principales dans le corps d'un insecte; favoir: la tête; le tronc ou corcelet; l'abdomen ou le ventre; & les membres. La structure de chacune de ces parties fournit des caracteres particuliers qu'on emploie dans la classification.
La tête seule, offre de singulieres variétés dans son organisation; tantôt elle est jointe au thorax par une forte d'articulation, tantôt elle ne forme avec lui qu'une seule piece; elle est le siege principal des sens & contient le cerveau, ou ses rudimens. Elle est très-grosse dans certains insectes, à proportion du reste de leur corps; & la tête n'est pas d'une grosseur constante dans le même insecte. Elle est ordinairemont\setcounter{page}{299} très-petite dans certaines chenilles avant qu'elles changent de peau; y la dureté de cet organe ne lui permet pas de croître à proportion du reste du corps; mais lorsque l'insecte est disposé à revêtir une enveloppe nouvelle, il retire vers le premier anneau du col la substance molle qui remplissoit l'intérieur de sa tête, & là elle trouve une place suffisante à de plus grandes dimensions: on la voit avec étonnement venir double en grosseur de celle que l'animal a laissée dans sa vieille peau; & comme l'insecte ne mange ni ne croît pendant que sa tête se forme, cet organe, & le reste du corps, ont des époques alternatives d'accroissement: pendant que la tête grossit, le corps n'augmente point, & vice versâ.
Il y a dans la structure de la bouche des insectes un art merveilleux: Elle varie dans sa forme & sa disposition à raison des espèces. Elle occupe ordinairement la partie antérieure & inférieure de la tête; mais quelquefois elle est sous la poitrine: tantôt elle est armée de pinces plus ou moins fortes pour saisir, retenir, & dépécer une proie; d'autres fois elle porte un dard propre à blesser & à sucer le sang: dans certains insectes la mâchoire est sillontée de dents quelquefois assez fortes pour qu'à leur aide l'animal puisse charrier des fardeaux, percer la terre, les bois les plus durs, & jusqu'aux pierres pour y faire son habitation & celle de sa progéniture.\setcounter{page}{300} D'autres portent une sorte de tube qui remplace la langue, & sert à sucer les fleurs. Enfin certains insectes, tels que les autres, n'ont point de bouche apparente.
La trompe, que possèdent certains insectes est un organe très-curieux & d'une structure très-compliquée. Quelques Naturalistes ont cru qu'elle remplissoit à la fois les fonctions de bouche, de nez, & de trachée artère. L'auteur décrit avec beaucoup de détail celle de l'abeille, du taon, du papillon, & du cousin: on peut se faire une idée assez nette des deux dernières sans le secours des figures. Nous allons parler d'après lui.
"La trompe du papillon est roulée en spirale comme le ressort d'une montre; elle fait ordinairement huit révolutions. On peut la développer & l'étendre, avec la pointe d'une épingle. Son diamètre diminue de l'origine à l'extrémité, & là, elle se divise en deux petits tubes qui ont chacun leur organe de suction. C'est probablement par là que l'insecte extrait le suc des fleurs :comme il n'a point de bouche il n'a pas d'autre organe alimentaire que sa trompe. Lorsqu'on la sépare de l'insecte, elle conserve la faculté de se rouler & se dérouler pendant un temps assez long."
"La trompe du cousin (culex) est formée d'un grand nombre de pièces extrêmement délicates qui concourent toutes au même but.\setcounter{page}{301} C'est un stilet propre à percer la peau; & un suçoir qui fait arriver le sang de l'animal attaqué, jusques dans l'intérieur de l'insecte. On ne voit à l'œil nu que l'enveloppe ou le fourreau qui renferme toutes les pieces. C'est un tube cylindrique, ouvert dans sa longueur de maniere que l'insecte peut se séparer du dard, & le relever à mesure qu'il enfonce celui-ci."
"Le dard est composé de cinq à six lames ou lancettes excessivement aigues, assemblées en faisceau. Les unes sont taillées en forme d'épée à deux tranchans, & les autres sont dentelées à leur pointe en fer de flèche. A l'instant où l'insecte enfonce dans la peau ce faisceau de dards jusqu'à l'intérieur de quelque très petite veine, il l'accompagne d'une ou deux gouttes d'un fluide qui atténue le sang; & le faisceau de lames agissant alors à la maniere des tubes capillaires, le sang s'élève entr'elles & arrive jusques dans le corps de l'insecte. Le fluide injecté cause la démangeaison si désagréable qu'on éprouve à la suite des piquures de ces petits animaux."
L'auteur indique, dans une note, quelques remedes pour ces piquures, souvent très incommodes. L'alkali volatil appliqué à l'instant; l'eau fraiche; se frotter le soir avec un peu de terre à foulon détrempée d'eau; ces trois spécifiques tendent à diminuer l'inflammation. Il propose aussi comme préservatifs, de se laver\setcounter{page}{302} avec du vinaigre, ou bien avec une solution de sucre de Saturne (acétite de plomb) avant de s'exposer à l'attaque des cousins.
Les antennes, ces fils minces, quelquefois articulés, mobiles dans toutes fortes de directions, dont la tête d'un assez grand nombre d'insectes est armée sont un organe dont l'usage est encore mal connu. Quelques naturalistes ont cru qu'elles remplaçoient le nez; d'autres leur attribuent une sensibilité assez exquise pour faire apercevoir à l'animal les moindres mouvemens ou changemens dans le fluide qui l'environne. On voit avec évidence que certains insectes non pourvus d'ailes, employent cet organe à tâter le terrain & à assurer leur marche. La forme des antennes, excessivement variée à raison des espèces, a été employée avec succès, par les naturalistes comme moyen de classification.
Mais c'est surtout l'organe de la vue qui présente chez les insectes les variétés les plus admirables. La plupart ont deux yeux: dans le monocle, ces deux yeux sont si rapprochés l'un de l'autre qu'ils paroissent n'en former qu'un. Le gyrinus en a quatre; le scorpion six, l'araignée huit, & le scolopendre trois. Cet organe est fixe dans presque toutes les classes d'insectes, & chez un assez grand nombre, ces protubérances qu'on aperçoit des deux côtés de la tête sont des yeux, composés d'un très-grand nombre\setcounter{page}{303} de facettes dont chacune est un œil complet. C’est furtout dans les yeux de l’insecte appelé Demoiselle (libellula) que le microscope fait découvrir cette admirable structure. La surface convexe de la protubérance est divisée en un nombre de cellules exagones qui renferment chacune un œil dont la surface est extrêmement polie; si l’on place une chandelle allumée dans le voisinage, chacune des faces en réfléchit l’image en même temps qu’elle fait pour l’insecte l’office d’une lentille transparente. La multiplicité de ces facettes & leurs directions très-variées suppléent à l’immobilité des protubérances auxquelles elles appartiennent, & font que l’animal peut voir en même temps de tous les côtés. Swammerdam a montré que chacune de ces facettes forme la base d’une pyramide exagone composée d’un grand nombre de fibres longitudinales & transversales, fans doute essentielles à la perfection de l’organe. Leuwenhoeck a compté 12,544 exagones pareils dans les yeux d’une Demoiselle. Quelques mouches éphémères ont quatre protubérances garnies d’yeux à réseau; & l’organe lui-même chez d’autres insectes du genre des mouches ne le cède pas en couleur & en vivacité aux pierres précieuses. La couleur varie avec les especes. Ici les yeux font verts, là rouges; d’autres fois changeans du pourpre au verd & à la couleur de cuivre.
L’araignée a ordinairement huit yeux très\setcounter{page}{304} beaux à voir au microscope, quelque désagréable que soit l'aspect de l'insecte lui-même. Ces yeux sont disposés de la maniere la plus favorable pour voir de tous côtés: Deux sont sur le sommet de la tête et voient en haut: deux sont à la partie antérieure, et quatre autres dans les côtés, regardant les uns en avant, les autres en arriere. Mr. Baker essaya d'appliquer un de ces yeux à un trou d'épingle fait à une carte, pour s'en servir comme de lentille: il trouva qu'il grossissoit beaucoup, mais que les objets étaient peu distincts. Peut-être ce défaut de transparence provenoit-il de ce que l'insecte était mort depuis long-temps lorsqu'on fit cet essai.
Les ailes ne sont pas l'organe le moins curieux à observer dans les insectes: les uns n'en ont que deux, d'autres quatre: chez quelques-uns, ces quatre ailes sont égales, chez d'autres inégales. On sait que Linné en a fait l'une des principales bases de sa classification. \footnote{Les ailes des insectes ont servi, il est vrai, à la formation des classes du célèbre Linné: en conséquence, en tirant ses étymologies du Grec, il a nommé Aptère tout insecte destitué d'ailes; Coléoptère tout insecte qui a deux ailes dures, et deux autres membraneuses, comme le cerf-volant; Hyménoptère, celui qui a quatre ailes nues et dépourvues des écailles qui ornent celles des papillons, comme l'abeille, &c. Mais jusqu'à présent aucun auteur n'avait étudié avec assez d'attention l'organisation de ces parties pour en faire une application méthodique. Le Cit. Jurine, connu dans le monde savant par ses belles expériences physiologiques, & qui a particulièrement étudié les insectes, dont il possède une collection très-précieuse, va publier incessamment un ouvrage sur les Hyménoptères, qui offrira aux Naturalistes une nouvelle méthode pour fixer avec sûreté leurs caractères génériques, & qui leur offrira sans doute de nouveaux sujets d'admiration, par la régularité qu'ils remarqueront dans la distribution des nervures qui divisent comme en cellules, la surface des ailes.
Cette méthode sera appliquée aux diptères & peut-être aux coléoptères. Il a observé que l'aile antérieure des hyménoptères s'articule avec une pièce un peu mobile fixée au corcelet, & il a soupçonné que c'étoit leur bras, qui étoit très-raccourci, & que la nature avoit construit sur un modele différent de celui de l'homme, à raison des fonctions qu'il avoit à exécuter. Les deux filets gros & durs qui s'étendent depuis ce moignon de bras jusqu'au point de l'aile, dans sa partie antérieure, représentent les os du radius & du cubitus : il a considéré enfin le point de l'aile comme le Carpe ou le poignet, à raison du mouvement articulaire dont il jouit; pour s'en convaincre que l'on jette les yeux sur un hanneton au moment où il se pose après avoir volé; on le verra plier à l'instant, dans cette articulation son aile membraneuse pour la cacher sous l'écailleuse. Or si l'avant-bras n'étoit pas articulé avec le bras; si le poignet n'avoit pas une charniere, & si les nervures de l'aile n'étoient pas équivalentes aux muscles mis en action par la volonté, comment cette opération s'exécuteroit-elle? Une seule partie de l'aile antérieure ou de la grande aile a suffi à ce Naturaliste pour y prendre le premier des caracteres génériques qui constituent sa méthode; c'est la partie qui s'étend depuis l'extrémité de l'avant-bras & le poignet jusqu'au bout de l'aile; l'on y voit fréquemment deux rangs de cellules: il a nommé cellules radiales celles qui partent du point, & cubitales celles qui sont au-dessous. A ce caractere générique il en a associé deux autres qui sont les mandibules ou dents, & les antennes, où l'on suppose que réside le sens de l'odorat. — Cet ouvrage sera accompagné de gravures exécutées avec le plus grand soin; chaque genre sera représenté par un insecte dont les caracteres génériques seront grossis; ainsi qu'on les verroit au microscope. (R)}\setcounter{page}{305} Les variétés dans la forme & la structure de cet organe chez les insectes sont presqu'infinies: la beauté de son coloris; l'art avec lequel l'aile est articulée avec le corps; la manière curieuse dont quelques-unes se reployent sous un couvercle solide & se déployent au moment du besoin; les articulations déliées qui permettent\setcounter{page}{306} ce jeu; les nombreuses ramifications ou nervures qui donnent à l'organe sa solidité, en même temps qu'elles servent à conduire les sucs nutritifs sur toute sa surface; cette merveilleuse organisation déploie au microscope toute son élégance. Mais il n'est pas besoin de cet instrument pour reconnoître les élytres, ces étuis\setcounter{page}{307} qui recouvrent les ailes & dont la force & la dureté sont si bien adaptées à leur fonction; ils servent en même temps de bouclier à l'insecte; & par les brillantes couleurs dont la plupart sont décorés, ils l'emportent en lustre sur les ouvrages de l'art les plus soignés. On est tenté en les considérant, de croire que le luxe n'est point une invention ou un abus purement humain, mais que la nature en a donné l'exemple, & que dans des sujets de goût ou d'ornement elle a été quelquefois jusqu'à fournir des modèles.
On trouve dans la micrographie du Dr. Hooke des recherches fort curieuses sur la vitesse absolue des vibrations des ailes de quelques insectes. Il a trouvé au moyen du son que produisent ces vibrations & dont il cherchoit l'unisson sur une corde tendue, qu'il se fait plusieurs centaines & au-delà d'un millier de ces vibrations dans une seconde de temps. Si, comme on a lieu de le présumer, la rapidité des sensations de l'animal est en rapport avec la vitesse des mouvemens qu'il peut exécuter, sa vie, qui nous paroît si courte, est relativement aussi longue que la nôtre, & peut-être davantage, à raison du nombre de perceptions & d'idées dont ils peuvent être capables dans un temps donné.
Et s'ils sont distingués sous ce point de vue dans le règne organique, ils ne le sont pas\setcounter{page}{308} moins par ces étonnantes transformations que subissent tous les infectes ailés sans exception, avant d'atteindre toute la perfection dont leur espèce est susceptible. Ce n'est point un simple changement de couleur, de grosseur, de proportion dans les diverses parties de l'animal; c'est une métamorphose entière & complète dans ses formes, ses proportions, ses organes, ses mouvemens & toutes ses habitudes. Ces changements sont quelquefois si soudains, qu'on pourroit croire que les anciens, à qui ils n'étoient pas tout-à-fait inconnus, avoient puisé dans l'observation de quelques faits de ce genre leurs fictions poétiques de métamorphoses & peut-être même l'opinion de la transmigration des ames \footnote{L'emblême d'un papillon, qu'on trouve sur des monumens funéraires, désignoit, sans doute, leur opinion que l'ame survivoit au corps, & qu'elle étoit appelée à une régénération dans cette nouvelle, existence. (R)}. Quoiqu'il en soit, les faits relatifs à ces transformations n'avoient été observés que très imparfaitement jusqu'à la fin du siècle dernier, époque où Malpighi & Swammerdam levèrent presqu'entièrement le voile dont la nature s'étoit enveloppée: l'infatigable & clairvoyant Réaumur vint ensuite, & on ne trouve plus gueres qu'à glaner après ces auteurs dans le champ des découvertes entomologiques. Notre texte nous entraineroit ici dans de grands\setcounter{page}{309} détails ; nous ne pouvons qu’indiquer rapidement les principaux traits qui caractérisent ces opérations naturelles.
La vie de l’un des insectes soumis aux transformations peut être divisée en quatre époques ou périodes.
Le germe dans l’état d’œuf, possède déjà un principe de vie qui n’attend que les circonstances favorables, & pour l’ordinaire une certaine température, pour se développer & passer à l’état de larve, soit dans certaines espèces, à l’état de ver ou chenille. Il se transforme ensuite en chrysalide ; enfin il devient insecte ailé & arrive alors au terme du perfectionnement qu’il peut atteindre. Il jouit quelque temps de toute la plénitude de son existence ; c’est à cette époque qu’il la femelle pond les œufs qui doivent reproduire l’espèce, puis l’insecte meurt, quand cette tâche est terminée. Entrons dans quelques détails sur ces divers modes d’exister dans un animal qui conserve son identité au travers des transformations les plus étranges.
Et remarquons d’abord que la même intelligence qui a ordonné ces changemens, a pourvu l’insecte, dans chacune de ces périodes, d’un assortiment d’organes parfaitement adaptés à son état passager & à toutes ses convenances momentanées.
Prenons l’histoire de la chenille pour exemple & observons ses métamorphoses : la femelle\setcounter{page}{310} qui pond ses œufs, choisit toujours, guidée par un instinct particulier, le lieu qui sera le plus convenable à sa progéniture. Le principe de vie demeure latent dans ces œufs, & peut supporter la température la plus rigoureuse, jusqu'à ce que le soleil du printemps le mette en action & renforce le petit animal assez pour qu'il perce son enveloppe & cherche ensuite par son propre instinct à entretenir l'existence qu'il vient d'acquérir & qu'une nourriture convenable ne tarde point à consolider.
Le corps de la chenille est divisé en douze anneaux; la tête, qui est dure & crustacée, tient au premier. Aucune chenille de papillon ni de phalène, n'a moins de huit jambes, ou plus de seize. Dans celles qui en ont douze, les six premières sont crustacées & adhérentes aux trois premiers anneaux; elles forment l'enveloppe des six jambes du futur papillon. Les six autres sont molles & flexibles ou membraneuses; elles accompagnent exclusivement l'état de larve. On remarque dans la partie latérale du corps, neuf petits trous de chaque côté; ce sont probablement, les organes de la respiration; & cinq ou six taches noirâtres qu'on observe des deux côtés de la tête, sont, selon toute apparence, les yeux de l'insecte.
Il est déjà soumis à des changemens considérables dans son état de chenille; il prend en grossissant, & à plus d'une reprise, une peau\setcounter{page}{311} nouvelle : & fi l'on examine fa dépouille, on remarque que les principales parties folides de la tête, le crâne, les mâchoires, & même des parties crustacées intérieures qui fervoient d'attache à certains muscles, les enveloppes des jambes ; en un mot, que toutes les parties visibles de l'animal se retrouvent dans cette dépouille...
Ce n'est qu'avec beaucoup de peine qu'il s'en débarrasse : cette crise est ordinairement précédée & suivie d'un jeune plus ou moins long ; mais notre célèbre compatriote Bonnet a observé que certaines chenilles s'empressoient de manger leur dépouille après l'avoir déposée. — Nous regrettons d'être ici forcés d'omettre les curieux détails qui présentent l'insecte, comme un modèle de patience, d'industrie & d'adresse. On sait que le ver à soie subit quatre changemens pareils, qui sont, pour lui, comme autant de mues.
Enfin le moment approche, où l'insecte va se trouver par une transformation beaucoup plus complette, que ne l'est un simple changement de peau, dans un état moyen, en quelque sorte, entre la vie & la mort. Il va perdre presque tous les organes des sens, & la faculté de se mouvoir ; & par conséquent celle d'éviter tous les dangers: dangers qu'il lui seroit inutile de prévoir puisqu'il ne pourroit s'y dérober. Mais le même instinct qui lui annonce l'arrivée prochaine de cette période de demi-existence, lui suggere les\setcounter{page}{312} moyens de se garantir d'une partie des inconvéniens dont il est menacé dans l'état de chrysalide.
Après avoir cherché une retraite convenable il employe l'une des quatre méthodes suivantes pour s'y mettre à l'abri. Certaines chenilles, telles que le ver-à-soie, se filent un tissu très-ferré qui les enveloppe comme un œuf dans la coquille; d'autres se cachent dans de petites cellules creusées en terre: d'autres se suspendent par leur extrémité postérieure: d'autres enfin s'accrochent aussi en l'air, mais par une espece de ceinture dont elles se sont environnées.
Le passage de la chenille à l'état de chrysalide est précédé d'un assez long intervalle de repos & de jeûne, qui donne aux organes extérieurs & intérieurs situés sous la peau, le temps de se développer. Puis, l'animal se contracte & se retire dans la partie antérieure de son enveloppe; là par des efforts répétés il fend le crâne en trois pieces, & rompt aussi les premiers anneaux du tronc; & après s'être donné beaucoup de mouvement il sort enfin de cette dépouille, qui le constituoit reptile, & qui semble lui être devenue odieuse.
Le voilà chrysalide: c'est un petit magot orné quelquefois des plus belles couleurs; & en particulier d'une teinte dorée de laquelle il a tiré son nom \footnote{χρυσοσ (chrysos) en grec signifie de l'or.}. Toutes les parties du futur\setcounter{page}{313} papillon existent, & quelques-unes peuvent être aperçues sous l'enveloppe mince mais crustacée, de la chrysalide; le fluide vifqueux dont elle est recouverte à son apparition ne tarde pas à se dessécher, & contribue à consolider cette écorce qui est en même temps le tombeau de la chenille & le berceau du superbe insecte ailé qui la remplacera. La chrysalide du papillon blanc ordinaire, vue au microscope lucernal pour les objets opaques, y donne un très-beau spectacle.
L'insecte, sous cette forme, est ordinairement immobile, & attend patiemment l'époque de sa dernière métamorphose. Il est comme enseveli dans un sommeil profond; il se réveille au contact des objets extérieurs & donne alors quelques signes de vie, mais c'est là tout. Ses organes sont enveloppés & fixés dans la situation qu'ils avaient prise en sortant de l'état de chenilles; quelques-uns sont susceptibles d'un léger mouvement, mais très lent, & en apparence pénible.
Cet état dure plus ou moins longtemps à raison de l'espèce de la chenille; & d'autres circonstances encore mal connues & dont probablement la température moyenne de la saison est une des principales. Tantôt quelques semaines suffisent au développement de la chrysalide, & d'autres fois ce développement exige plusieurs mois. L'auteur cite un exemple de chrysalides provenant des mêmes œufs, dont les chenilles avaient \setcounter{page}{314} été nourries de la même manière & avoient filé leur coque en automne à peu de jours de distance, & dont le déploiement à l'état de papillon eut lieu à des périodes très-distantes : savoir, pour un tiers d'entr'elles au printemps suivant; pour un second tiers une année après; & pour le troisième, encore l'année suivante. Celles-ci avoient donc vécu trois ans dans l'état de chrysalide.
Le papillon sort plus facilement de l'enveloppe de la chrysalide que celle-ci ne s'est débarrassée de sa peau de chenille. Cette enveloppe est si desséchée & devenue si fragile à l'époque où l'animal parfait est prêt à en sortir, qu'il ne faut, à sa part, qu'un léger effort pour la rompre; & cette rupture se fait toujours au même endroit. L'insecte en sort humecté & encore engourdi; bientôt ses ailes s'étendent & se renforcent en se desséchant : elles grandissent, exactement à vue d'œil, & en peu de minutes leur surface devient cinq fois plus grande qu'elle n'était d'abord; le papillon les essaye & s'envole.
La nourriture de la chenille était solide & grossière, & l'animal la cherchait en rampant & exposé à mille dangers. Il la triturait péniblement entre ses dents pour que son estomac pût la digérer. Devenu papillon il se nourrit du nectar des fleurs; l'air devient son élément & son séjour habituel; il frappe nos regards\setcounter{page}{315} par l'élégance de ses formes & la beauté de ses couleurs. Il lui falloit des journées entieres pour arriver d'un buisson au buisson voisin, & maintenant, d'un vol léger & en un clin-d'œil il dépasse des arbres les plus élevés; il ne s'occupe que d'amour & de voyages, & semble jouir, en folâtrant, de toute la plénitude de ses facultés nouvelles.
Si quelques-uns de nos lecteurs à qui ces faits seroient connus & familiers nous reprochoient de suivre, dans leur exposé, notre auteur avec trop de complaisance, nous les inviterions à se rappeler que l'un des principes que nous ne perdons jamais de vue en travaillant à ce recueil, est d'écrire moins pour les gens profondément instruits, que pour la classe nombreuse des amateurs qui, sans s'être attachés à aucune branche d'étude particuliere, cherchent à meubler leur mémoire & leur entendement, de ce genre de connoissances qu'on aime à se retracer parce qu'elles donnent matiere à réfléchir.
Les insectes ailés ne font pas les seuls qui soient soumis aux transformations: la punaise & la puce, ces deux especes si incommodes pour l'homme, passent aussi par l'état de larve & de chrysalide.
Parmi les faits remarquables que cite notre auteur, & qui pourroient n'être pas connus des Naturalistes du continent, nous choisirons\setcounter{page}{316} une anecdote de l'histoire des pucerons.
On fait, d'après les curieuses recherches de Bonnet, qu'un puceron, mis dès l'instant de sa naissance dans la solitude la plus parfaite, se trouve au bout de quelques jours au milieu d'une famille nombreuse d'individus de son espèce ; que si l'on met à part un de ces pucerons, il ne tarde pas à produire seul une seconde génération, dont chaque individu possède aussi la faculté de faire naitre une troisième génération, &c.
"Mr. Bonnet, dit l'auteur, avoit répété l'expérience jusqu'es à la sixieme génération, & avec des résultats uniformes, lorsqu'un soupçon de Mr. Trembley son ami \footnote{Le célèbre auteur de la découverte des polypes d'eau douce & l'un des meilleurs observateurs que ce siecle ait produits. (R)} l'engagea à reprendre cette suite d'observations & à pousser plus loin l'épreuve. Il poursuivit jusqu'à la dixieme génération ces pucerons solitaires, en tenant une note exacte du jour & de l'heure de la naissance de chacune. Il découvrit qu'il naissoit finalement des mâles & des femelles : les premiers ne parurent qu'à la dixieme génération, & en petit nombre : l'accouplement de ceux-ci avec les femelles suffissoit à produire dix générations successives d'insectes, toujours vivipares, excepté la premiere qui provient d'oeufs\setcounter{page}{317} fécondés : tous ces individus font des femelles, à l'exception de quelques mâles, qui paroissent seulement à la dixieme génération, & donnent naissance à une nouvelle série semblable."
Le Dr. Richardson a repris ces expériences ; & a choisi parmi la grande variété d'espèces que renferme le genre des pucerons ou aphides, ceux du rosier, comme étant les plus gros & ceux dont la carrière est la plus longue. Ils paroissent de bonne heure au printemps & on en trouve encore en automne.
Dès le mois de février, si le soleil est chaud, on les voit sortir de petits œufs noirs déposés l'année précédente aux environs des boutons à fleur. L'inclémence de la saison en détruit souvent un grand nombre, & ceux qui lui réussissent n'atteignent qu'en avril toute leur grossesse ; c'est alors qu'ils commencent à le propager, après avoir changé deux fois de peau. Il n'y a que des femelles ; elles font toutes vivipares & leurs petits naissent à reculons. Le printemps produit deux générations pareilles ; mais la fécondité augmente dans les mois d'été ; ils fournissent cinq générations, dont les individus changent trois ou quatre fois de peau selon que la saison est plus ou moins chaude. Ce changement est d'autant plus fréquent, que l'accroissement de l'insecte est plus rapide.
Quelques individus de la troisième génération, nés en mai, après leur dernier changement\setcounter{page}{318} de peau, se trouvent pourvus de quatre ailes, plus longues que leur corps. On observe la même variété dans les générations suivantes, mais toujours sans diversité de sexe. Trois autres générations se succèdent pendant les mois d'août et de septembre, et c'est dans la dernière seulement (qui est la dixième, à dater de celle qui a commencé par l'œuf) qu'on retrouve des mâles; mais toujours beaucoup moins nombreux que les femelles.
La multiplication de ces insectes pendant les mois d'été est si grande qu'ils nuiraient excessivement à la végétation des pousses tendres auxquelles ils s'attachent s'ils n'étaient partiellement détruits par de nombreux ennemis. Mais ils ne font pas sans amis, si l'on peut donner ce nom aux parasites qui les affligent officiellement pour profiter des aubaines que les pucerons leur fournissent. La fourmi et l'abeille, ces deux classes, d'ailleurs si laborieuses, font de ce nombre, et ramassent le miel que produisent les pucerons avec assez d'abondance.
Il y a cependant cette différence entre les procédés de ces deux classes; c'est que l'affiduité des fourmis est constante, tandis que les abeilles n'ont recours au miel des pucerons que quand les fleurs sont rares. Celles-ci l'enlèvent aux pucerons à mesure qu'ils le produisent; les fourmis le prennent sur les feuilles où les pucerons l'ont déposé.
\setcounter{page}{319} L'histoire des abeilles elles-mêmes est assez étendue dans notre auteur ; elle nous mèneroit trop loin & feroit, après tout, imparfaite, car il ne connoissoit pas les découvertes de notre ingénieux compatriote F. Huber\footnote{Nouvelles observations sur les abeilles ; adressées à Mr. Charles Bonnet par François Huber. Genève chez Barde & Manget. 1792.
L'histoire de cet ouvrage n'est pas moins remarquable que ne le font les découvertes qu'il renferme. Qu'on nous permette de transcrire ici quelques phrases de la préface, elles présentent un fait que nous croyons unique dans les annales des sciences.
"En publiant, dit l'auteur, mes observations sur les abeilles, je ne dissimulerai point que ce n'est pas de mes propres yeux que je les ai faites. Par une suite d'accidens malheureux je suis devenu aveugle dans ma première jeunesse; mais j'aimois les sciences, & je n'en perdis pas le goût en perdant l'organe de la vue. Je me fis lire les meilleurs ouvrages sur la physique & sur l'Histoire naturelle: j'avois pour lecteur un domestique (F. Burnens, né dans le Pays-de-Vaud) qui s'intéressoit singulièrement à tout ce qu'il me lisoit: je jugeai assez vite par ses réflexions sur nos lectures, & par les conséquences qu'il savoit en tirer, qu'il les comprenoit aussi bien que moi, & qu'il étoit né avec les talens d'un observateur.
Ce n'est pas le premier exemple d'un homme, qui, sans éducation, sans fortune, & dans les circonstances les plus défavorables, ait été appellé par la nature seule à devenir naturaliste. Je résolus de cultiver son talent & de m'en servir un jour pour les observations que je projetois: dans ce but, je lui fis répéter d'abord quelques-unes des expériences les plus fimples de la phyfique; il les exécuta avec beaucoup d'adreffe & d'intelligence; il paffa enfuite à des combinaifons plus difficiles. Je ne poffédois pas alors beaucoup d'instruments; mais il favoit les perfectionner, les appliquer à de nouveaux ufages, & lorfque cela devenoit néceffaire, il faifoit lui-même les machines dont nous avions befoin. Dans ces diverfes occupations, le goût qu'il avoit pour les fciences devint bientôt une véritable paffion & je n'hésitai plus à lui donner toute ma confiance, parfaitement affuré de voir bien en voyant par fes yeux."}.
\setcounter{page}{320} Rien de plus curieux & de plus varié que l'industrie avec laquelle la plupart des insectes pourvoient à l'emplacement de leurs œufs & à la nourriture de l'embryon qui doit en provenir. Nous en avons déjà cité quelques exemples dans ce Recueil, en parlant des ælstres\footnote{Voyez T. VII, pag. 47, Sc. & Arts, les curieuses observations de Mr. Bracy Clark fur ce genre d'insectes; notre auteur rend à cet excellent naturaliste toute la justice qu'il mérite. (R)}. Le coufin, la mouche éphémère, la demoiselle volent au-dessus des eaux dormantes pour y déposer leurs œufs, qui doivent éclorre dans le fein du liquide. La masse d'œufs pondus par la femelle du coufin reffemble à un petit bateau; chaque œuf a la forme d'une quille & ils font réunis d'une manière fingulière. L'insecte ne pond qu'un œuf à la fois; il croife fes jambes poftérieures pour le recevoir dans la fourche qu'elles
\setcounter{page}{321} forment; un second œuf vient s'attacher au premier, puis un troisieme, & ainsi de suite jusqu'à ce que le petit bateau qu'ils doivent former soit achevé & puisse flotter tout seul. Plusieurs insectes ont pour leurs œufs & pour les larves qui en proviennent, les mêmes soins que l'instinct de la maternité inspire à la plupart des vivipares. L'araignée elle-même, si vorace & si cruelle, soigne ses œufs avec l'intérêt le plus inquiet. L'araignée-loup, les charrie sur son dos dans une petite bourse de soie tissue par elle; & si on les lui enleve elle donne des signes de la plus grande détresse. Cependant les individus de cette espece s'attaquent quand ils se rencontrent, & le plus fort dévore le plus foible. Ils portent jusque dans leurs amours, une défiance & des précautions, qu'on n'observe jamais dans les autres especes. Et l'industrieuse fourmi, au milieu des soins économiques dont elle semble être constamment occupée, donne à ses œufs des attentions toutes particulieres; elle les redouble lorsque les larves ont paru; si le temps est beau elle apporte du fond de son nid ces petits magots au soleil pour qu'ils jouissent de sa bénigne influence; puis elle les remporte dès que l'air se rafraichit. Si le nid se trouve dérangé par quelqu'accident, & souvent par la malice d'un passant, l'alarme est universelle; les meres courent ça & là, dans l'apparence du désespoir; elles recueillent les larves dispersées, elles les entreposent\setcounter{page}{322} sous le premier abri en attendant qu'elles aient réparé le dommage, puis elles vont les chercher pour les remettre dans leur nid.
La fécondité des insectes surpasse toute idée; l'auteur cite ici sur cet objet, les expériences & les calculs de Réaumur, de Lyonnet, de De Geer, &c. La chenille à brosse, par exemple, d'après les calculs de Lyonnet, peut fournir à la troisième génération 1,492,750 chenilles produites par un seul papillon; De Geer, d'après ses propres observations porte ce nombre à quatre millions. Mais ce sont aussi les espèces dont la multiplication est la plus rapide qui ont le plus grand nombre d'ennemis: cette compensation entretient l'équilibre tel que nous l'observons autour de nous.
Parmi les anecdotes que cite l'auteur sur les habitudes & l'industrie des insectes parasites, ou qui se nourrissent aux dépens d'autres espèces vivantes, nous choisirons le fait suivant. "J'observois un jour ( dit le Naturaliste qui le lui a communiqué ) quelques chenilles qui paroissoient faire un excellent repas d'une feuille de chou. Je fus frappé des procédés d'une petite mouche qui, en bourdonnant autour des chenilles paroissoit chercher à se poser : celles-ci commencèrent incontinent à s'agiter avec des signes d'une violente inquiétude & en s'efforçant par leurs contorsions d'esquiver la mouche chaque fois qu'elle s'approchoit plus particulierement\setcounter{page}{323} de l'une d'elles : enfin l'insecte ailé fit son choix & s'abattit sur l'une des plus grosses & des mieux nourries ; vainement le malheureux reptile cherchat-il à déloger la mouche ; la terreur qu'il avoit montrée à son approche n'étoit rien au prix des angoisses que les piqûres de l'insecte ailé lui faisoient éprouver : la pauvre chenille se tordoit à chaque blessure , essayoit d'atteindre la mouche en courbant son corps en-dessus , mais tous ses efforts furent vains ; & la mouche après avoir donné trente ou quarante coups de stilet , s'envola en triomphe. Je présumai qu'elle avoit déposé un œuf dans chacune des plaies qu'elle venoit de faire , & j'emportai la chenille chez moi pour observer les suites de l'événement : je lui donnai à manger , elle parut se guérir au bout de quelques heures & pendant quatre à cinq jours elle mangea comme à son ordinaire ; mais ces œufs ne tarderent pas à donner naissance à de petits vers , oblongs & très-voraces , qui furent à peine éclos qu'ils commencerent à se nourrir dans l'intérieur de la chenille , de la substance même de l'animal , sans attaquer cependant les organes de la respiration ni ceux de la digestion ; & parvenus à toute leur grosseur ils sortirent de la chenille en la perçant à jour , supplice auquel elle succomba. L'insecte ainsi attaqué par la larve de cette mouche n'échappe jamais ; mais la larve , qui a besoin de lui pour\setcounter{page}{324} toute la durée de sa propre existence, ne touche point jusqu'au dernier moment aux parties essentielles à la vie ; cette larve ne produit point de papillon ; les vers, qui se font nourris de la substance du malheureux infecte, se filent une coque, dès qu'ils l'abandonnent, & ils sortent ensuite de cette enveloppe, sous la forme de mouches qui, avec le même instinct féroce, vont chercher à placer leurs œufs comme l'ont été ceux qui les ont produites."
Ces mouches parasites, ou ichneumones, offrent un très-grand nombre d'espèces : Linné en a reconnu soixante-dix-sept, & il n'est pas probable qu'aucune ne lui ait échappé. Les habitations des insectes tant aquatiques que terrestres ; la structure de leurs nids, dans laquelle ils développent tant d'adresse & d'industrie ; tout cela fournit à notre auteur des articles intéressans que nous regrettons d'être forcés d'omettre. Il les termine par une réflexion qui mérite d'être méditée. Après avoir montré par des comparaisons qui font toutes à l'avantage de l'infecte, combien l'homme lui est inférieur dans tous les procédés que l'infinitude suggère ou commande ; il trouve dans cette infériorité même le signe de sa perfection réelle, ou plutôt d'une perfectibilité indéfinie & illimitée, qui le place infiniment au-dessus de tout animal dont les facultés ont dans leur développement, une marche rigoureusement circonscrite,\setcounter{page}{325} & un terme fixe qu'elles atteignent toujours, il est vrai, mais qu'elles ne dépassent jamais.
A mesure que notre extrait se prolonge, nous éprouvons le pénible embarras du choix entre des citations qui nous semblent offrir un égal intérêt. Nous allons prendre le parti de les exclure toutes, & arriver, avec notre auteur, aux confins de l'animalité : nous terminerons ce morceau, déjà long, par quelques-uns des traits les plus saillans de l'histoire des polypes d'eau douce : c'est-là que l'Auteur de la nature a diversifié ses modèles avec le plus de variété, & qu'il a surtout mis en évidence l'inconcevable fécondité de ses moyens pour la conservation des especes.
Ces polypes appartiennent à l'ordre des zoophytes, ou animaux-plantes : leur corps paroît consister en un simple tube qui porte à l'une de ses extrémités de longs bras avec lesquels il saisit la proye dont il se nourrit. Cet animal paroît ne posséder aucun des organes qui sont communs à tant d'autres ; il n'a ni tête, ni cœur, ni estomac, ni intestins, il n'a point de sexe, & se reproduit d'une manière très-singulière ; si on le coupe en morceaux, chacune des portions séparées devient en peu de temps un animal parfait dans son espece. Cette propriété si caractéristique & qui rappelle la fable de l'hydre de Lerne les a fait nommer\setcounter{page}{326} Hydres par Linné : le Dr. Hill les a appelées Biota, d'ún mot grec qui signifie vie, à raison de la prodigieuse vitalité dont cet animal paroît être doué.
Ceux qui prétendent qu'on trouve chez les anciens, des germes de prefque toutes les découvertes modernes citent Aristote qui dit \footnote{Aristote de Historia animalium, T. I. L. IV. C. 7. pag. 824.} qu'il y a des animaux qui multiplient de bouture, & par des fections, à la maniere des plantes. St. Augustin rapporte qu'un de ses amis fit devant lui l'expérience de couper en deux un polype, & que les deux portions se fauverent chacune de leur côté \footnote{St. Augustin. De quantitate animæ, C. 62. pag. 431. col. 1.}.
Leuwenhoeck avoit aussi publié quelque chose sur ces animaux en 1703 dans les Tranfact. Philos. ; mais personne ne les avoit étudié & n'avoit rendu compte de leurs diverses especes, de leurs habitudes, de leur maniere de se nourrir & de se propager jusqu'à ce que le hasard les fit tomber sous les yeux d'un Naturaliste Genevois qui réunissoit au plus haut degré les talens & les qualités dont l'ensemble constitue l'observateur. C'étoit feu Mr. Trembley, & son nom depuis lors est en quelque sorte identifié avec sa découverte. L'ouvrage dans lequel il a donné la suite de ses observations sur ces animaux,\setcounter{page}{327} est un chef-d'œuvre de fagacité & de logique. Il écrivoit à Mr. Bonnet en janvier 1741. "J'ai étudié ces petites créatures depuis le mois de juillet dernier; je leur trouve des propriétés qui appartiennent à la plante, & d'autres qui font de l'animal. On les prend pour des plantes au 'premier coup-d'œil, mais si ce font des plantes, elles font sensibles & ambulantes; si ce font des animaux, on peut les multiplier de bouture comme les plantes" ce ne fut qu'au mois de mars de la même année qu'il put décider la question.
Ce fut le célèbre Buffon qui le premier communiqua à la Société Royale de Londres cette curieuse découverte : quelques métaphysiciens ne pouvant accorder les faits avec des raisonnemens qu'ils avoient pris l'habitude de croire sans replique, nierent bonnement les faits ; mais il fallut les admettre à la fin, quand une nuée de témoins les eut constatés.
On compte sept especes d'hydres ou de polypes d'eau douce. Celles sur lesquelles on a le plus multiplié les observations sont le polype vert, le polype gris, & le polype rougeâtre.
Le premier que découvrit Trembley fut un polype vert. La couleur verte est très-décidée dans cette espece. Il observa dans les bras, ou ramifications, de cette petite créature des indices d'un mouvement spontanée ; l'animal les allongoit ou les raccourcissait : il les recourboit\setcounter{page}{328} en tout sens ; & au plus léger contact ces bras se contractaient tellement qu'ils disparaissaient tout-à-fait, en se réduisant à une petite saillie en forme d'un grain verdâtre. Il découvrit ensuite le polype gris, & lui vit avaler & digérer des vers beaucoup plus gros que n'était le polype lui-même. Ces espèces ont depuis six jusqu'à douze ou treize bras ; on en a compté dix-huit dans un polype gris. Ils se contractent tellement au fortir de l'eau qu'on les prendrait pour une goutte de gelée. Ces animaux peuvent dilater séparément ou leurs bras, ou leur tronc, & se ployer dans toutes les directions imaginables. Ils montent & descendent à volonté dans l'eau & demeurent suspendus depuis la surface par un procédé singulier : ils font sortir, pour un moment, l'extrémité de leur queue au-dessus du niveau du liquide ; elle s'y dessèche très-promptement, & par sa résistance à s'humecter de nouveau elle soutient le polype, & forme autour du point de suspension une petite concavité comme celle qu'on observe autour d'une aiguille lorsqu'on la fait surnager à l'eau. Quelquefois le polype s'attache si fortement, par le simple contact de sa queue, aux plantes aquatiques, qu'on ne peut pas facilement lui faire lâcher prise, surtout quand avec ses bras, qu'il emploie comme autant d'ancres, il a multiplié ses points d'adhérence.
\setcounter{page}{329} Sa bouche est située à l’extrémité antérieure du tronc à l’endroit d’où partent tous les bras ; cet organe prend toutes les formes qui conviennent aux besoins de l’animal, & elle est susceptible d’une très-grande dilatation ; car le volume de la plupart des animaux dont se nourrit le polype est à la dimension ordinaire de sa bouche, ce que feroit une pomme aussi grosse qu’une tête d’homme, pour une bouche humaine.
C’est un animal très-vorace ; lorsqu’il a faim, il étend ses bras comme un pêcheur déploie ses filets, & il attend patiemment que le hasard lui amène quelque proie. Tout petit insecte qui passe à portée est saisi par quelqu’un des bras & porté à la bouche. On ne lui connoît point d’yeux ; cependant il paroît avoir quelque sens qui remplace celui de la vue & qui l’informe de l’approche de sa proie.
Lorsqu’il attaque un petit ver, il le tue avec tant de promptitude que Fontana attribue, d’après ce fait, au polype une influence vénéneuse très-active. A peine l’animal a-t-il touché le ver, de l’un de ses bras, que celui-ci périt, sans qu’on observe qu’il ait reçu aucune blessure.
La transparence du polype permet de suivre dans son intérieur la digestion des fragmens qu’il a avalés ; on facilite cette observation en lui donnant à manger quelque ver rougeatre. On voit alors les sucs nutritifs se répandre\setcounter{page}{330} non-seulement dans le tronc mais dans les bras.
Qualquefois deux polypes attaquent le même ver par ses deux extrémités ; alors ils avalent chacun de leur côté jusqu'à ce que leurs bouches se rencontrent ; puis ils s'arrêtent quelque temps, & le ver se rompant pour l'ordinaire, chacun conserve sa moitié. D'autres fois les choses ne se passent pas si paisiblement ; ils se disputent la proie ; & l'un des deux ouvrant la bouche avec plus d'avantage, avale l'autre, avec une portion du ver. L'affaire tourne moins mal qu'on ne le présumerait pour le polype dont une portion a été avalée ; son antagoniste le revomit pour l'ordinaire sain & sauf au bout d'une heure ; d'où il parait que l'estomac du polype, qui dissout si promptement les matieres animales, n'attaque point la substance d'un autre polype \footnote{L'un de nos amis, le Cit. Trembley, neveu du naturaliste célèbre à qui nous devons presque toutes ces observations si curieuses, vient d'y ajouter quelques faits nouveaux. Voici ce qu'il nous mande à ce sujet. J'ai lu avec bien de l'intérêt l'extrait que vous avez donné dans votre dernier N°. de la Biblioth. Bris. de l'ouvrage d'Adams. — Vous annoncez qu'il a consacré un chapitre entier aux polypes d'eau douce. Comme j'ai observé dans le courant de l'été dernier cette espèce de ces singuliers animaux & que j'ai été témoin d'un fait qui avoit échappé aux observations de mon oncle, je serois très-curieux de savoir s'il en est question dans l'ouvrage d'Adams : voici ce dont il s'agit."
"Mon oncle exprime ses regrets au commencement de son second Mémoire de ce qu'après fix mois d'observations sur le polype vert il n'a pas pu découvrir comment il se nourrissait."
"On a vu, dit-il, dans le Mémoire précédent, que ce sont les polypes verts que j'ai trouvés les premiers. Je les ai observés pendant plus de fix mois ; & quelques soins que je me soie donnés pour découvrir comment ils se nourrissent, je n'ai pu en venir à bout. Lorsque j'ai eu lieu d'être persuadé qu'ils étoient des animaux, & que leur structure m'a été un peu connue, j'ai soupçonné que cette ouverture qui se faisoit remarquer à leur extrémité antérieure étoit leur bouche ; mais tous ceux que j'avois sont péris avant que j'ai pu pousser plus loin mes recherches ; & tous les soins que j'ai pris pour en trouver d'autres, depuis le mois d'avril 1741, jusqu'à présent, (janvier 1744) ont été inutiles."
"J'ai eu le bonheur de trouver de ces animaux dans une eau stagnante très-voisine de ma demeure ; & après avoir répété quelques-unes des expériences qui tiennent aux étonnantes reproductions de ces animaux, j'ai dirigé toute mon attention sur les faits qui pouvoient m'indiquer la maniere dont ils se nourrissent ; & mes recherches n'ont point été infructueuses.
J'avois mis divers insectes aquatiques dans une grande soucoupe où je tenois des polypes ; je remarquai que ceux-ci appliquoient leurs bras avec une forte avidité sur des vers rouges de tipules : les vers se débatttoient & entraînoient les polypes avec eux. Ceux-ci paroissoient sucer le ver. J'essayai un jour de couper en deux un de ces vers, & j'eus alors le plaisir de voir un polype avaler en entier une de ces portions. Après l'avoir gardée pendant un certain temps & en avoir tiré toute la substance qui lui convenoit, il vomit le résidu. Depuis ce moment je vis tous les jours des polypes avaler de petits vers ou des portions de vers ; on apperçoit ceux-ci dans le corps du polype à travers sa peau ; il les rejette ensuite sans les avoir déformés ; mais il est à présumer qu'il en a tiré tout le suc, parce que si l'on offre à un autre polype un de ces vers rejetés, il ne le saisit pas."
" Je vis un jour un spectacle très-singulier : ordinairement les polypes saisissent les vers par l'une de leurs extrémités ; mais j'offris à l'un d'eux une portion d'un ver assez gros ; il appliqua sa bouche sur le milieu du ver, puis il la dilata peu-à-peu au point de couvrir toute sa surface ; les bras du polype étoient redressés en dehors & fort contractés, ce qui formoit une espèce de petite corbeille dont l'extrémité postérieure du polype occupoit le centre : on peut dire que les trois quarts de son corps s'étoient convertis en une grande bouche qui suçoit le ver sans l'avaler. Il m'auroit été impossible de le reconnoître, sous cette forme singulière, si je n'avois suivi de l'œil ce changement depuis l'origine. Lorsque le polype eut bien sucé le ver, il reprit dans un instant sa première forme. Les polypes verts se nourrissent indistinctement des vers rouges dont j'ai parlé, & d'une autre espece fort analogue, mais qui est d'un blanc jaunâtre. Ces vers forment de petits tuyaux avec des débris de plantes ou autres matières spontanément ordinairement renfermés. Ces deux especes sont très bien décrites dans le premier Mémoire du IVe. vol. de Réaumur sur les insectes. Il est affez singulier que la conformation des vers blancs l'ait porté à leur donner le nom de vers polypes, quoiqu'à cette époque il ignorât l'existence des polypes, & par conséquent, que ces vers leur servoient de nourriture. Je montrai mes polypes verts à Mr. Favre de Rolle, qui a beaucoup de goût pour l'Histoire naturelle; il en prit quelques uns chez lui, & découvrit au bout de quelque temps, que ces mêmes vers qui étoient dévorés par les polypes les mangeoient à leur tour; ils appliquent leur bouche contre le polype, qui diminue peu à peu, se change en une sorte de bouillie, & paffe en entier dans le corps du ver où l'on voit la matière verdâtre dont eft composé le polype. C'eft un fait affez singulier dans l'Histoire naturelle, que de trouver des insectes d'especes différentes qui se mangent réciproquement."
" J'ai essayé de garder des polypes sans leur donner à manger; ils ont vécu près de 4 mois; ils ont diminué à mesure que le jeune se prolongeoit, & ont fini par disparaître tout-à-fait. La privation d'air les tue au bout de dix à douze jours, & beaucoup plus tôt si l'eau dans laquelle ils sont est très-échauffée. Quelques heures ont suffi pour les tuer dans un flacon que j'avais mis dans la poche de mon gilet"......
" En observant quelques animaux aquatiques j'ai vu une sorte de ver qui a un grand rapport avec les larves de cousins, dont il diffère cependant à quelques égards dans la partie postérieure. Sa tête est aplatie de dessus en dessous, & son contour est arrondi ; le corps est longuet, la tête est bien détachée du premier anneau, auquel elle tient par une espèce de col. Ce petit animal fait sortir de sa tête deux espèces de houpes dont l'agitation & le jeu attirent vers sa bouche l'eau & les petits corps qui nagent au-dessus. Il avale ensuite ceux qui lui conviennent. Pendant que je considérais cette petite manœuvre, je fus très-étonné de voir la tête de ce petit insecte tourner comme sur un pivot horizontal, de manière que le dessus se trouva dessous, quoique le reste du corps n'eût point bougé : je la vis tourner et retourner plusieurs fois sans que le jeu des petites houpes cessât. Cela présente un spectacle assez singulier. Je l'ai montré à Mr. Senebier qui a vu la chose comme moi. Je n'ai rien trouvé qui y eût rapport dans les descriptions de Reaumur. J'ai suivi l'insecte dans l'état de nymphe ; elle a beaucoup de rapport avec celle du cousin. Mr. Jurine à qui Mr. Senebier a envoyé une mouche provenue d'un de ces vers a assuré que ce n'étoit point une tipule ni un cousin."....
" Il y a encore un fait singulier relatif aux polypes verts, et dont je n'ai pu découvrir la cause : c'est leur extrême rareté dans certains temps et dans certains lieux. Vous avez lu que mon oncle fut plus de trois ans sans pouvoir en retrouver ; et mon frere m'écrit de Berlin que j'ai eu une espece de bonne fortune en trouvant des polypes verts, et qu'il en avoit souvent cherché avec mon oncle aux environs de Genève, sans pouvoir en découvrir."}.
\setcounter{page}{331} Toutes ces allures ont des rapports plus ou moins marqués avec ce que nous connoissons de l'animalité en général ; mais fous le rapport\setcounter{page}{332} de la propagation, le polype est une plante ; car il multiplie par rejettons. On voit paroître à sa surface de petits tubercules qui s'allongent.\setcounter{page}{333} peu à peu en forme de branche, deviennent un polype parfait, & se séparent ensuite du tronc primitif par des secousses volontaires.\setcounter{page}{334} La chaleur de la saison influe beaucoup sur la rapidité de ces développements : on observe à cet égard des différences qui s'étendent depuis un jour jusques à quinze. Le jeune polype pendant son adhérence au tronc principal, le fait profiter de la nourriture qu'il fait déja se procurer avec ses petits bras.
Le principe de vie est si universellement & si intimément attaché à toute la substance de cette classe d'animaux, que les mêmes procédés qui tuent tous les autres, non-seulement n'anéantissent point ceux-ci, mais accélèrent, immédiatement\setcounter{page}{335} leur multiplication. On peut varier ces épreuves de mille manieres. Si on coupe un polype en morceaux, soit en long, soit en travers, chaque fragment devient bientôt un animal entier ; même une petite portion de la peau produit aussi un animal parfait. Si l'on fend un polype en deux, depuis le sommet de la tête jusqu'au milieu du tronc, on produit un polype à deux têtes qui font chacune leurs fonctions ; si on le divise en six ou sept parties, on obtient un polype à six ou sept têtes. Qu'on refende chacune d'elles, on a un polype à douze ou quatorze têtes ; qu'on coupe toutes\setcounter{page}{336} ces têtes, chacune fournit seule un nouveau polype, & de nouvelles têtes croissent sur tous les troncs. Jamais les récits fabuleux de l'Hydre de Lerne n'ont approché de ces merveilles.
On peut enter deux ou plusieurs polypes les uns aux autres, en rapprochant leurs extrémités fraîchement coupées: on met si l'on veut, la tête de l'un sur le corps de l'autre & elle y demeure. On peut réunir deux polypes en un seul en faisant avaler, l'un des deux par l'autre jusqu'à la tête; le nouveau polype a deux fois plus de bras qu'un autre & s'en sert également bien. Enfin si l'on retourne un polype sur lui-même, comme on feroit un sac ou un gant, il s'accoutume très-bien à ce mode d'existence; son estomac devenu épiderme, l'épiderme converti en estomac, s'acquittent très-bien de leurs fonctions respectives. Si à l'époque de l'opération il y avoit de jeunes polypes à la surface extérieure de l'animal ainsi retourné, ceux qui avoient déjà acquis un certain développement font leur chemin en-dedans du côté de la bouche pour sortir par-là lorsqu'ils se détacheront; ceux qui étoient plus petits prennent le parti de se retourner aussi & ils atteignent ainsi de nouveau la surface extérieure, en traversant la substance même du sac dans lequel ils se trouvoient renfermés.
Mais ces détails entraînants nous font dépasser\setcounter{page}{337} les bornes & le genre d'un extrait. — Nous supprimons & les réflexions qui naissent en foule, & d'autres faits qui appartiennent au monde microscopique ; peut-être y reviendrons-nous une fois : nous aimons à l'espérer.
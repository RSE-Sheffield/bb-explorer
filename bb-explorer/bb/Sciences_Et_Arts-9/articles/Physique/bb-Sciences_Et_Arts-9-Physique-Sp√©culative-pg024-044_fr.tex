\setcounter{page}{24}
\chapter{Physique}
\section{SUFFRAGES BRITANNIQUES FAVORABLES A LA PHYSIQUE SPÉCULATIVE.\footnote{Voy. Vol. 8. Sc. & Arts, p. 140.}}
Ce siècle, surnommé philosophique, qui se dit si bien affranchi, du joug de toute Autorité; y est cependant encore asservi, à l'égard de certaines Questions délicates: savoir; quand il croit pouvoir se dispenser de les examiner;\setcounter{page}{25} en s'en rapportant passivement à quelques grands noms, sur ce qu'il doit en penser. Et cela doit être ainsi : Vû l'immense Multitude d'Écrits, qui s'offrent actuellement à notre curiosité, sans que pourtant notre vie soit plus longue que celle de nos dévanciers.
Si donc, ces respectables Autorités, faisaient (par exemple) pour avoir proscrit certaines Recherches, comme infructueuses à tous égards. Il étoit assez naturel, que la foule des Lecteurs, les regardât du même œil.
Seulement. On aurait dû s'assurer auparavant, si le Fait, n'était pas exagéré, ou même hazardé : C'est-à-dire, si ces premiers Juges, n'avaient point modifié leur sentence, par quelques restrictions; ou si même, ils n'avaient pas été d'un Avis contraire à celui qu'on leur prête.
Mais: Les mêmes occupations ou distraction qui avaient détourné tant d'Esprits superficiels, d'examiner le fond même des Questions; les ont détournés aussi, de vérifier (par une lecture soigneuse des livres originaux), si leurs Auteurs avaient réellement soutenu les Opinions que le bruit public leur attribue.
Non-seulement; ils n'ont point eu occasion, de se guérir de leurs Préjugés : Mais encore; ils s'y sont confirmés de plus en plus; soit, par la répercussion mutuelle des Echos; soit par le silence, (pris pour approbation) du petit\setcounter{page}{26} nombre d'Esprits éclairés qui auraient pu détruire ces Préjugés. Il ferait bien temps enfin; que quelqu'un qui fût écrire avec un peu d'énergie, réclamât hautement, contre ces Citations incomplètes (ou même infidèles), qui ont entrainé tant de gens inattentifs. Et en attendant mieux. Il ne sera pas entièrement inutile; qu'on ébauche du moins de gros en gros, une telle Réclamation: Ce qui donnera peut-être l'Éveil, à des Ecrivains capables de la développer convenablement.
\subsection{Chapitre I. Le Chancelier BACON.}
M'étant occupé, pendant toute ma jeunesse, de la recherche des Causes générales: Mais, ayant ensuite vu affirmer (& répéter jusqu'à l'ennui), dans presque tous les Livres modernes de Physique; que le grand BACON, avait condamné cette Recherche, comme désespérée & comme inutile. Je voulus enfin savoir; sur quelles Raisons, ce Législateur de la Philosophie appuyait deux Opinions, qui ne me paraissent point raisonnables. Car ces Ecrivains, ne rapportaient jamais que ses simples Assertions. Et il me parut indispensable, de m'en informer par moi-même: Parce que je soupçonnais fort; qu'il n'y avait qu'un petit nombre d'entr'eux, qui eussent lû l'Original: Puisqu'ils s'accordaient tous, à en citer les mêmes Passages, courts & très-peu nombreux; comme\setcounter{page}{27} s'ils les eussent copié les uns sur les autres.
Or : Je fus très amplement dédommagé, de la petite Fatigue que donne quelquefois la Nomenclature de ce Génie incomparable ; par la multitude des Vuës fublimes & fort judicieuses dont fes Ouvrages philosophiques font parsemés. Et j'y vis entr'autres avec fatisfaction. Que ; comme on ne possédoit point encore , quelque Art de découvrir sûrement la Vérité, du moins en Physique ; il n'était pas surpris, qu'il y eut tant de sceptiques sur cette matière. A quoi il ajoutait : Qu'au lieu de jeter le manche après la coignée, ces Philosophes auraient dû observer ; que le Mal provenait, de ce qu'on était entré dans de mauvaises Routes. Il se proposait donc d'y suppléer : Non (disoit-il modestement) par la force de son génie ; mais, en indiquant quelques nouvelles Règles, pour discerner la Vérité. Et même , il ne développait pas assés ces Règles ; pour qu'il fut aisé à des Esprits ordinaires , d'en faire tout de suite quelques Applications.
Son coup-d'oeil perçant, lui fit bien concevoir : Qu'on pourrait aller très-loin, dans la recherche des Causes naturelles, même les moins à portée de nos Sens ; pourvû qu'on entrât dans de nouvelles Routes. Mais , il avouait modestement ; qu'il ne se faisait presque aucune idée déterminée , de ce que pourraient être ces Routes : Excepté cependant, une forte d'Induction;\setcounter{page}{28} plus propre que l'Induction ordinaire ; à avancer dans cette Recherche : savoir des Exclusions ou Rejections ; dont il pressentait vaguement les avantages, sans en déterminer la Marche. J'articule (de préférence) cette Route-là : Parce que j'ai toujours été fort surpris ; qu'aucun des Physiciens qui semblent avoir lu les Ecrits de BACON, ne se soit avisé de cultiver cette Méthode ; que moi Commençant, j'avais déjà essayé d'appliquer à la Physique.
Cet incomparable Philosophe, fut prévoir aussi : Qu'on réussirait beaucoup mieux, à découvrir les Causes naturelles ; en disséquant & subdivisant la Matière, comme avait fait DEMOCRITE ; qu'en partant, comme ARISTOTE, de quelques Qualités considérées abstraitement.
Il attachait aussi un très-grand prix ; à la Configuration des Particules, & à leur Mouvement. Il disait même, sur ce dernier Chef : Qu'il n'était point étonné ; que la connaissance des Causes n'eût fait aucun progrès, tant qu'on ne les avait cherchées que dans des Principes tranquilles ; puisque c'était sans doute dans des Principes agités, qu'il aurait fallu les chercher. Mais, il ne pouvait pas insister en détail là-dessus : Vu l'ignorance où l'on était alors, des Règles de la communication du Mouvement.
En général : Les Préceptes de BACON, étaient trop relevés, pour être goûtés & suivis.\setcounter{page}{29} pär son siècle.\footnote{On a remarqué que les jeunes-gens tiraient peu de fruit de la lecture des Reflexions de Morale fondées sur une longue Expérience; parce qu'ils n'en avaient pas encore éprouvé eux-mêmes l'importance & la vérité. Il en est de même des Reflexions d'une Logique un peu profonde, comme sont la plupart de celles du Chancelier BACON: On en sent peu la solidité & le prix; quand on n'a pas encore beaucoup lu & médité, sur les Recherches dont il nous indique tant les Écueils, que les bonnes Avenues.} Ausi ne fit-il point d'Enthousiastes, ni par conséquent de Secte. La Vérité, consistant ordinairement dans des Milieux tempérés; que les Esprits communs ne saisissent guère, & que les Esprits ardens-goûtent fort peu: Elle étend plus lentement son empire; que ne font les Opinions extrêmes & tranchées. D'autant plus: Que ceux qui l'annoncent, étant ordinairement des Gens sans intrigue ni passions; ils ne cabalent & ne machinent point en sa faveur.
Il dut furtout bien déplaire aux Philosophes de son temps; quand il prononça itérativement: Qu'un Physicien calomnie la Nature; lorsqu'il multiplie les Principes d'explication, uniquement parce qu'il ne fait pas expliquer tous les Phénomènes par peu de Principes: Et qu'il calomnie l'Esprit humain; quand il le juge incapable d'aucune autre Ressource, que de celles dont ce Physicien-là a su s'aviser lui-même.
\setcounter{page}{30} Je ne m'étends pas davantage, sur ce qu'il dit des Moyens de perfectionner la Physique. Parce que, ce n'est pas là-dessus principalement, qu'on a tronqué ou défiguré ses Pensées : Mais bien plutôt, à l'égard de la prétendue Inutilité d'une Connaissance des Causes générales ; ce que, par cette raison, je vais examiner plus au long.
Et : Comme ce sont surtout les Réflexions favorables à l'Utilité de cette Connaissance ; qu'ont supprimé le plus constamment, les Ecrivains qui rapportent d'autres Passages de ce respectable Législateur : J'aurai soin de traduire bien littéralement, la plupart de ces Réflexions-là.
Enfin : Pour ne pas disperser l'attention du Lecteur, sur tant de Citations éparses & diverses ; en la faisant fautiller trop souvent, d'une certaine Considération, à une Considération toute différente, pour revenir ensuite à cette première (comme le fait très-souvent nôtre Auteur) : Je rapprocherai les uns des autres, les Passages qui ont le plus d'analogie entr'eux ; en les groupant ensemble, à part de ceux qui ont moins de ressemblance avec eux.
Mais ; avant que de passer à ces divers Groupes de Passages favorables à ma Thèse : Je vais répondre aux deux Raisons qu'on allegue en faveur de la Thèse contraire.
1°. Donc, on observe : Que BACON s'est\setcounter{page}{31} beaucoup plus étendu, sur les avantages de l'Expérience, que sur ceux de la recherche des Causes. A quoi je réponds : Que cela était absolument nécessaire de son temps; vu qu'alors, on négligeait beaucoup trop les Observations, pour courir après des Subtilités scholastiques.
2°. L'on répète jusqu'à satiété, la Période suivante du Novum Organum (laquelle fait partie du 10me Aphorisme du second Livre) : Neque fingendum aut excogitandum; sed inveniendum, quid Natura faciat aut ferat; Comme si elle nous interdisait la recherche des Causes. Mais; je n'y vois qu'une Distinction, sur la manière dont on doit procéder à cette Recherche; & voici comment je traduirais cette Periode : "Il ne faut pas fabriquer ou imaginer gratuitement, des Explications ou Causes; mais on doit, procéder méthodiquement, à découvrir les Opérations & Productions de la Nature." D'ailleurs : Cette Règle-là, ne doit pas être prise, généralement : Puisqu'elle n'est prescrite, que dans l'une des trois Ministrations de la 1re Partie des Indices sur l'Interprétation de la Nature.
Au reste : Les Lecteurs fort économes de leur temps, ne doivent pas appréhender; de ne trouver ici, que des Passages uniquement propres à soutenir la Thèse pour laquelle j'écris, tout ceci. Ces mêmes Passages, étant entremêlés aussi, d'autres excellentes Réflexions,\setcounter{page}{32} qui peuvent avoir échappé à quelques-uns de ces Lecteurs.
\subsection{ARTICLE I. Inconvéniens qui résultent; de ce qu'on se livre exclusivement aux Expériences.}
§. 1. LA Considération des Antipathies & Sympathies, ainsi que des Propriétés cachées & particulières; est une espece de Magie ou de Grimoire: Qui endort l'Entendement humain; par le murmure & le jargon de ces Vertus occultes & spécifiques; en les transmettant respectivement & mystérieusement aux Lecteurs trop dociles, comme si elles étoient descendues du Ciel. De sorte que; la Crédulité, se reposant sur ces Traditions oisives; les hommes ne sont plus excités; à une Recherche vigilante & approfondie; des Causes proprement dites.
2. La plupart des Physiciens; renoncent à une Connoissance universelle des Choses; & à la recherche des Principes: Ce qui est extrêmement nuisible à l'avancement des Doctrines. Car, quand on veut découvrir au loin, on doit monter sur des Tours, ou sur des lieux fort élevés: Et il est impossible à qui que ce soit, de fonder les parties éloignées & intimes de quelque science; s'il s'arrête au Rez-de-chaussée de cette science, au lieu de s'élever au Faîte, comme sur un Observatoire.
3. Tant que l'Astronomie se borne, à l'Extérieur\setcounter{page}{33} vient des Choses célestes (comme font, leur nombre, leur situation, leurs mouvemens, leurs périodes); sans y joindre les viscères de ces Choses (savoir les Causes physiques): Elle n'instruit l'Entendement humain; que comme une Peau empaillée nous donne la connoissance d'un Animal. Et il nous manque une autre Astronomie, qu'on puisse appeler organisée & vivante.
4. Pour étendre le Pouvoir & les Opérations de l'homme; il ne suffit pas (ou il n'importe pas beaucoup) de connoître de quoi les Choses font composées; si l'on ignore les Moyens & les Voyes des Changemens qu'elles subissent. C'est cependant sur ces Principes morts; que travaillent le plus souvent les spéculatifs: Comme s'ils ne se proposoient, que de contempler le cadavre de la Nature, sans rechercher les Facultés qui constituent sa Vie. De sorte qu'on ne s'occupe des Principes moteurs, que presque en passant & fort négligemment; quoiqu'ils soient la Chose la plus considérable & la plus utile de toutes.
5. Quand un Physicien, n'aura recherché d'entre les Causes; que celles des Phénomènes tels qu'ils se présentent au Vulgaire (c'est-à-dire, composés de plusieurs); sans les avoir réduits, à une véritable décomposition ou simplicité, comme par Distillation.
Il pourra bien (s'il est conséquent d'ailleurs);\setcounter{page}{34} ajouter quelque chose de passable, & même d'ingénieux, aux découvertes d'autrui. Mais : Il n'ouvrira aucune Route majeure, & comme spéculaire ; & il ne méritera pas le titre d'Interprète de la Nature.
6. En philosophant uniquement d'après l'Expérience ; on tombe encore plus aisément dans des Opinions irrégulières & contraires au cours de la Nature ; qu'en philosophant uniquement a priori & spéculativement. NB. Voici quelques-uns des Abus auxquels on s'expose, dans la première de ces Routes.
L'Instruction que nous fournissent nos sens, étant souvent en défaut & même trompeuse ; l'Observation sur laquelle on se fonde, pouvant avoir été faite avec peu de soin ou de régularité, & comme au hasard ; les Témoignages, étant ordinairement légers & mal rapportés, les Pratiques, dirigées vers le leurre, & asservies à la Routine ; la Marche des Expériences, aveugle, déraisonnable, vague & irrégulière ; enfin ; l'Histoire-naturelle étant superficielle & incomplète : Ces Moyens-là, n'ont pu fournir à l'Entendement, que des Matériaux très-défectueux, pour l'avancement de la Philosophie & des Sciences.
7. On se tromperoit du tout au tout sur mon Intention ; quand je recommande aux Physiciens, de rassembler des Expériences concernant les Arts : Si l'on se figuroit ; qu'il s'agit seulement,\setcounter{page}{35} d'en venir à les mieux perfectionner. Car : Quoique, dans plusieurs cas ; je ne méprise pas complétement ce Perfectionnement des Arts : Cependant, mon But est entièrement ; que les Ruisseaux de toutes les Expériences mécaniques, se rendent enfin de toutes parts, dans l'Océan de la Philosophie. Je le repéte donc : Ce n'est pas pour les Faits eux-mêmes, que je propose d'en faire des Collections, Et il ne convient point ; d'en mesurer l'Importance, d'après eux ; mais, d'après leurs Conséquences, & leur influence sur la Philosophie.
8. En général. Les hommes ne pourront tirer parti de toutes leurs Forces : Que, quand ils ne se livreront pas tous à-la-fois, à un seul & même genre de Recherches : Mais que les uns, en embrasseront d'une certaine sorte ; & les autres d'une espèce différente ( suivant la diversité de leurs goûts & de leurs talents ).
\subsection{ARTICLE II. Préférence que méritent, les Expériences utiles à la Philosophie, sur celles qui sont utiles aux Arts.}
§. 1. Il faut toujours avoir présent à l'esprit ; ce que j'inculque perpétuellement : Qu'on doit rechercher les Expériences propres à nous éclairer : avec encore plus de soin, que celles dont on peut retirer quelque autre Avantage.
\setcounter{page}{36}
2. La première Conféquence qu'il faut tirer d'une Expérience quelconque ; c'est la Connoiffance des Causes, ou des Propositions générales. Et l'on doit s'attacher aux Expériences lucifères , plutôt qu'à celles qui font fructifères.
3. Le bon Ordre des Expériences, confifte d'abord, à allumer le flambeau qui doit éclairer la Route ; ensuite, à la parcourir. En commençant, par quelque Expérience régulière, réfléchie, & exempte de tâtonnemens ; en en déduisant ensuite, des Propositions générales ou Principes ; & en passant delà, aux nouvelles Expériences que suggéreront ces Propositions.
4. On pourra concevoir une Espérance bien fondée, sur l'avancement ultérieur des Sciences : Quand on introduira & qu'on accumulera dans l'Histoire-naturelle, beaucoup de ces Expériences ; qui n'ont aucune Utilité par elles-mêmes ; mais qui servent seulement, à trouver des Propositions théorétiques & des Causes : Expériences, que j'ai accoutumé d'appeler lucifères, pour les distinguer des fructifères. Or ; ces Expériences lucifères, possèdent une merveilleuse Prérogative : savoir ; de ne jamais tromper ou frustrer notre attente. Car : Comme on les employe ; non, pour produire quelque Effet ; mais, pour contribuer à constater la réalité de quelque Cause soupçonnée : Quel que soit leur Résultat ; elles répondent toujours également à notre But ; puisqu'elles décident la
\setcounter{page}{37}
\subsection{Question affirmativement ou négativement.}
5. Il y aura beaucoup de Choses, dans l'Histoire-naturelle telle que je la propose ; qui paroistront subtiles, & plus curieuses qu'utiles ; non-seulement aux Esprits vulgaires, mais même à un Entendement quelconque accoutumé à l'état présent de la Physique. C'est pourquoi ; je dis & répète: Que je requiers qu'on débute, par des Expériences lucifères & non fructifères..... La connoissance des Principes simples, bien examinée & déterminée ; est comme une Lumiére ; qui éclaire toutes les Opérations ; & qui a la force, d'embrasser (ou de mener à sa suite) des multitudes Pratiques, ainsi que d'être la source de Propositions éminentes : sans (cependant) que cette Connoissance, considérée en elle-même, soit d'une grande Utilité.
6. L'Histoire-naturelle que je propose ; n'est point celle, qui amuserait par la Variété des objets, ou qui apporterait quelque Profit immédiat par des Expériences avantageuses : Mais plutôt celle, qui puisse éclairer la Recherche des Causes, & allaiter l'enfance de la Philosophie. Car : Quoique je prêche principalement, la Pratique & la partie active des Sciences : Cependant; je crois qu'on doit attendre la saison la plus propre à la Récolte, & ne point moissonner le bled en herbe. Car je fais parfaitement : Que les Propositions solidement établies, menent à des légions de Pratiques ; non éparses,\setcounter{page}{38} mais groupées. Au lieu que, je blâme & déconseille entièrement ( comme la pomme d'or qui retarda la course d'Atalante ) ; ce Desir prémâturé & puéril, de capter promptement quelques Gages de nouvelles manipulations.
7. Si quelqu'un ; qui n'aurait fait aucune Découverte particulière, matériellement utile ; avait allumé un flambeau, propre à éclairer les bords des Choses qui touchent à celles qu'on connoissait déjà : Et ensuite ; ce Flambeau ayant été exhaussé ; il vienne à faire voir & à dévoiler les Choses les plus abstruses ; Un tel Penseur ; me paraît étendre l'empire de l'homme sur l'Univers, ainsi que le délivrer de ses fers & entraves.
8. Le But le plus élevé de l'Histoire-naturelle, est : De fournir les moyens & matériaux, d'une Induction solide & concluante ; & de tirer, des objets qui tombent sous nos sens, autant de Connaissances qu'il en faut pour instruire notre Entendement.
Car, quant à l'autre But de cette Etude ; qui se borne, à nous amuser par des Descriptions, où à nous procurer quelque Avantage matériel : Il est certainement ; d'un Ordre inférieur & d'un genre subalterne ; en comparaison de celui qui est propre à préparer notre esprit pour établir la Philosophie.
L'Histoire-naturelle que je recommande principalement ici ; est donc celle : Qui sert de\setcounter{page}{39} Base solide & durable, à une Philosophie réelle & active; & qui fournit à l'étude de la Nature, la première étincelle d'une lumière pure & exempte d'illusions.
9. Ce qui a perdu la Physique expérimentale: C'est que les hommes ont recherché principalement les Expériences frustifères, & même plus promptement que les lusifères: Et qu'ils se sont entiérement attachés, à produire quelque Ouvrage éclatant; plutôt qu'à manifester les Oracles de la Nature; ce qui cependant, ferait l'Ouvrage des Ouvrages, & renfermerait toutes les Puissances humaines....... Leur Méprise & Erreur à cet égard, provient de ce qu'ils se sont figuré, que l'office de la physique consistoit à plier & réduire les Faits qui arrivent rarement, à ceux qui nous font familiers. Au lieu que cet Office, consiste plutôt; à déterrer, les Causes de ces choses familières même, & les Causes reculées de ces Causes.
\subsection{Avantages généraux, de la connaissance des Causes.}
§. 1. DE même que, l'Abeille ne se borne pas, à puiser dans les fleurs, la Matière de sa cire & de son miel; mais, qu'elle la digère & la transforme encore, par une Faculté qui lui appartient. Ainsi: Le travail de la véritable Philosophie, consiste; à déposer les Matériaux\setcounter{page}{40} de l'Histoire-naturelle & de l'Expérience; non, dans la Mémoire seulement, & tels qu'ils se présentent; mais, dans l'Entendement, après les avoir transformés & façonnés. Desorte qu'on peut augurer les plus heureux succés; d'une Alliance plus étroite & plus sage, que formeront nos successeurs, entre ces deux genres de travaux.
2. Celui qui connaît bien, les Qualités universelles de la Matière, & parlà, dequoi elle est capable: Ne pourra pas ignorer non plus, le Passé ni le Présent ni l'Avenir; au moins, quant aux Résultats généraux.
3. La connaissances des Causes physiques; répand du jour sur les Objets analogues, & donne prise à de nouvelles Inventions.
4. Celui qui connaît les Formes (c'est-à-dire, les Qualités fondamentales); en déduit, & découvre (ce qu'on n'a point fait jusqu'à présent) des Choses que, les vicissitudes de la Nature, ni les Expériences les plus industrieuses, n'auroient jamais mis au jour; & qui ne feraient jamais entrées dans la Pensée des autres hommes.
5. C'est avec raison, qu'on dit: Que, pour savoir véritablement les Choses, il faut en connaître les Causes. Car il n'est pas vraisemblable: Qu'on puisse savoir véritablement une chose, avant que l'Esprit soit entièrement affermi dans l'application de ses Causes.
6. Ce qui fait manquer les Opérations; c'est\setcounter{page}{41} surtout l'Ignorance des Causes. Car, ceux qui connoissent bien les Causes ; feront aussi de bons juges, des Effets & des Opérations.
7. Celui qui connoît les Formes (c'est-à-dire, les différentes Qualités des différentes substances ) ; est en état de classer ensemble, par quelque Conformité naturelle, les Matières même les plus dissemblables en apparence.
Ainsi : Il peut découvrir & produire, des Choses qui n'avoient point encore été faites ; & qui n'auroient jamais été réduites en acte, soit par les vicissitudes de la Nature, soit par des Expériences ingénieuses, soit même par le hasard ; enfin, qui ne feraient jamais entrées dans la Pensée de l'homme.
\subsection{ARTICLE IV. Considérations plus particulières, sur la Recherche des Causes.}
§. 1. Je propose une Entreprise, beaucoup plus grande que l'Astronomie vulgaire. Car ; je ne pense pas seulement, aux Calculs & aux Prédictions, qui composent ordinairement cette science ; mais, à la Philosophie. Savoir celle; qui puisse instruire l'Entendement humain ( par des Causes naturelles & des Raisons solides ), de, tout ce qui concerne les Corps célestes : Non seulement, de leur Mouvement & de leurs Périodes ; mais aussi, de leur substance & de toutes leurs Qualités.
\setcounter{page}{42}
2. Il n'y a prefque perfonne, qui ait fait des Recherches fur les Caufes phyfiques ; par exemple, fur la fubftance des Chofes céleftes, foit des Etoiles, foit de ce qui occupe leurs Intervalles. C'eft pourquoi j'affirme ; que la Partie phyfique de l'Aftronomie, eft une chofe encore à faire.
3. La recherche de la première Conftitution des Atomes, eft fi importante : Que je doute ; fi fon Utilité, n'eft pas abfolument la plus grande de toutes. Vû qu'elle eft ; la Règle Suprême de l'action & de la force, la véritable Déterminatrice de nos efpérances, & la Directrice de nos opérations.
4. Si la doctrine de DEMOCRITE fur les Atomes, n'eft pas vraye : On peut du moins s'en fervir, pour faire comprendre certaines Vérités. Et l'on ne doit point s'en effrayer ni s'en défier, à caufe de la Subtilité qu'elle fuppofe dans la Nature. Car on doit comprendre que les chofes les plus petites font foumifes au calcul comme les plus grandes. Et perfonne ne doit fe figurer ; que cette Doctrine, foit une Spéculation plus curieufe qu'utile. Car on peut remarquer : Que prefque tous les Philofophes ( qui fe font fort occupés d'Expérience & d'Objets particuliers, & ont comme difféqué la Nature jufqu'au vif) ; tombent enfin tout naturellement, dans de femblables Recherches ; quoiqu'ils n'en viennent pas heureufement à bout.
5. Quant à l'Utilité de cette Recherche fur la\setcounter{page}{43} première Constitution des Atomes : Je ne fais, si elle n'est pas de la plus haute importance; puisqu'elle doit servir de Règle, pour juger en dernier ressort, de toute activité & force des Corps.
6. Il faut bien, décomposer les Corps : Pourvû que ce ne soit pas à l'aide du Feu; mais, par la Raison, & par une Induction proprement dite, avec les Expériences auxiliaires. Il faut aussi, comparer les Corps entr'eux, les réduire à de simples Elémens; enfin, démêler les Qualités qui s'y rencontrent & s'y combinent. Il faut, en un mot, quitter VULCAIN pour MINERVE; si l'on veut répandre la Lumière, sur les Tissus & Configurations intimes des Corps : Choses; d'où dépend toute Propriété & Vertu, & d'où l'on déduit toute Règle des Altérations & Transformations considérables.
7. Les hommes ont accoutumé; de borner leurs recherches sur les Causes, à ceci : De rapporter les Phénomènes rares, à ceux qui sont fréquens; en donnant ceux-ci, pour Causes de ceux-là : Mais; de ne point rechercher les Causes, de ce qui arrive souvent; le regardant, comme suffisamment admis. Ainsi : Ils ne recherchent pas les Causes, de la Pesanteur, ni des autres Phénomènes journaliers : Mais; en partant de ces Faits, comme manifestes & reçus; ils discutent les autres Choses, moins familières; & ils prononcent, qu'elles sont des Conséquences de ces Faits.\setcounter{page}{44} Au lieu que je suis sûr: Qu'on ne peut porter aucun jugement, sur les Choses rares, ou remarquables (&, bien moins encore, en mettre au jour de nouvelles); avant que d'avoir examiné & trouvé, les Causes des Choses ordinaires, & même les Causes de leurs Causes.

\setcounter{page}{197}
\chapter{Physique}
\section{ESSAYS ON THE MICROSCOPE, CONTAINING, &c. Effais fur le microfcope, contenant une defcription des microfcopes les plus parfaits; une hiftoire générale des infectes; leurs transformations, leurs habitudes; leur Économie. Détails fur les diverfes efpèces & les propriétés fingulières des hydres, & des rotiferes; defcription de trois cent quatre-vingt-trois animalcules différens, & catalogue d'objets intéreffans; expofition de l'organifation du bois & des criftallifations falines. Par feu G. ADAMS; feconde édition avec des additions confidérables par F. KANMACHER, Membre de la Société Linn. In-4° grand format, 714 p. avec trente-deux planches in-folio. Londres W. & S. Jones, Holborn, 1798. prix 1 liv. sterl. 8 shell. broché.}
LES sciences & les arts ont des rapports nombreux, une dépendance réciproque & intime, qui les fera toujours marcher de front vers la perfection ou la décadence. Le favant, guide & affure les pas de l'artiste, celui-ci multiplie, à fon tour, les moyens du favant; il fabrique\setcounter{page}{198} fes armes ; la conquête de vérités nouvelles eft pour l'ordinaire , le fruit de leurs efforts réunis ; ils partagent alors la gloire du fuccès , & triomphent enfemble fans jaloufie.
Mais la marche eft bien plus rapide , elle préfente les facultés humaines fous un jour plus brillant , lorfque le même individu peut manier la plume & la lime tour-à-tour ; paffer de l'obfervatoire à l'attelier , ou du cabinet au laboratoire ; quant il fait à la fois inventer , fabriquer , & décrire ; lorfqu'il poffede , en un mot , & la tête qui médite , & la main qui exécute. Point de partage alors , point de dépendance réciproque ; l'homme eft en poffeffion de fa force toute entière ; il chemine vers le but fans détours inutiles.
George ADAMS , auteur de l'ouvrage que nous avons fous les yeux , réuniffoit bien des traits du modele dont on vient de tracer l'efquiffe. Héritier d'une réputation déjà acquife par fon pere , dans l'art de conftruire les inftrumens d'aftronomie & de phyfique , il l'avoit beaucoup étendue ; & ces appareils divers , conftruits dans fes atteliers & continuellement perfectionnés fous fes directions , faifoient pour lui l'objet d'un commerce confidérable dans les deux mondes. L'ambition de l'Artifte devoit fans doute être fatisfaite ; mais entraîné par goût vers la carrière des sciences , animé dans leur culture par un efprit public , une forte de loyauté\setcounter{page}{199} qui ne lui permettoit de jouir qu'à demi des découvertes jusqu'à ce qu'il eût contribué à les répandre, il écrivoit avec facilité & avec un zèle infatigable. Il a publié à diverses époques, sur l'astronomie, la géographie, la géométrie, l'optique, & sur la physique expérimentale, des ouvrages précieux\footnote{Ces ouvrages, dont on trouve les dernières éditions chez & Jones, Holborn, forment une collection qui mérite toute l'attention des amateurs; ils portent les titres suivans. Essais Géométriques & Graphiques; contenant 1°. un choix de Problèmes géométriques dont plusieurs n'ont point été publiés. 2°. La description & l'usage des instrumens dont on garnit à l'ordinaire les Etuis de mathématiques; on y a joint la description & l'usage de plusieurs instrumens nouveaux, employés dans la Géométrique pratique. 3°. Un système complet d'Arpentage, avec le détail de plusieurs perfectionnemens importans introduits dans cet art utile; on y a joint la description des Théodolites, planchettes, & autres instrumens d'arpentage. 4°. Les méthodes de Nivellement, pour la conduite des eaux, avec une description du Niveau à bulle d'air perfectionné. 5°. Un Cours de Géométrie pratique militaire, telle qu'on l'enseigne aux écoles de Woolwich. 6°. Un Essai abrégé de Perspectives 2 vol. 8°. avec 35 planches. -- 14 shell. broc. Essai sur l'Electricité, dans lequel on développe les principes de cette science; on décrit les divers instrumens qui ont été imaginés soit pour en éclaircir la théorie, soit pour en rendre la pratique intéressante. On y a joint une lettre à l'auteur, par Mr J. Birch, sur l'Électricité médicale. 4e. édit. 1 vol. 8°. avec 6 planches. Prix 6 shell. broché.
Essai sur la vision ; dans lequel on décrit en abrégé la construction de l'œil & la nature de la vision ; cet ouvrage est particulièrement destiné aux personnes dont la vue s'est affaiblie, & il leur indique les moyens de la conserver, en désignant les cas dans lesquels il convient de faire usage de bésicles, & les moyens de les choisir sans risquer d'altérer sa vue. 8°. seconde édit. avec fig. Prix 3 shell. broch.
Essais Astronomiques & Géographiques ; contenant, 1°. une exposition étendue des principes généraux de l'astronomie, avec un détail de toutes les découvertes du Dr. Herschel. 2°. L'usage des Globes, avec la solution d'un grand nombre de problèmes arrangés en classes distinctes & accompagnés d'informations curieuses. 3°. La description & l'usage des Orrerys & Planétaires. 4°. Une introduction à l'Astronomie pratique, 3e. édit. 8°. avec 16 planches. Prix 10 sh. 6 d. broché. NB. L'introduction à l'astronomie pratique ou l'usage du Quart de Cercle & de l'Equatorial, avec 2 planches, se vend séparément 2 sh. 6 d.
Leçons del Physique théorique & expérimentale, 2de. édit. en 5 vol. 8°. avec plus de 40 grandes planches.
Essais sur le Microscope, &c.. C'est celui dont nous donnons l'analyse.}, & il en préparoit d'autrès\setcounter{page}{200} lorsqu'une mort prématurée l'a enlevé ; âgé de 45 ans seulement, aux sciences, aux arts, & à ses nombreux amis. Son frère, qui avoit fait à Genève une partie de ses études, lui a succédé dans la direction de son commerce & de ses ateliers ; & il y porte un talent &
\setcounter{page}{201} une activité qui lui promettent des succès constants.
Nous donnerons une idée du genre de l'ouvrage que nous annonçons, par un exposé rapide, tiré de la préface même de l'auteur ; & nous y joindrons quelques morceaux détachés, qui feront connoître sa maniere, & apprécier son travail.
Le premier chapitre contient une histoire abrégée de l'invention du microscope, & des perfectionnemens successifs de cet appareil. On y trouve le procédé du P. Della Torre pour faire ses globules de verre qui avoient acquis tant de célébrité. Le second chapitre traite de la vision ; & l'auteur y développe très-nettement la raison de l'avantage optique qu'on obtient dans l'usage des verres lenticulaires pour grossir les objets. Il n'a introduit dans ce chapitre que les connoissances strictement nécessaires à l'exposé des vérités qu'il lui importoit de rendre saillantes.
Dans le troisieme chapitre il décrit les microscopes les plus perfectionnés & quelques autres, qui font d'un usage assez général. Les avantages respectifs de chacun de ces instrumens sont indiqués, afin que le lecteur puisse choisir ceux qui lui conviendront le mieux dans un cas donné. Les gravures qui représentent ces divers instrumens sont si belles & si nettes qu'on peut suivre les descriptions avec beaucoup de facilité.
\setcounter{page}{202}Le quatrieme chapitre renferme l'indication des méthodes à employer dans la préparation des objets qu'on se propose d'observer, & les précautions à suivre dans l'usage des divers microscopes.
L'auteur, après avoir, dans ces quatre chapitres, familiarisé le lecteur avec la construction & l'usage de tous les appareils microscopiques, & l'avoir, en quelque sorte, muni de tous les moyens d'étude dans ce genre, commence, dès le cinquieme, un cours d'observations, dirigées en particulier vers l'infectiologie. Ce cours occupe plus des deux tiers de l'ouvrage & voici comment l'auteur explique & justifie cette déviation du plan qu'il avoit d'abord adopté.
"Lorsque j'entrepris, dit-il, les Essais que je publie, je me proposois seulement de réimprimer l'ouvrage de mon pere intitulé, Micrographia illustrata: mais je vis bientôt que cet ouvrage & celui de Baker, étoient l'un & l'autre, imparfaits. A l'époque où ces auteurs écrivoient, l'Histoire Naturelle étoit bien éloignée du degré de perfection qu'elle a atteint depuis, & leurs erreurs sont excusables par cette considération. J'ai dono cherché d'abord à rectifier ces erreurs; & après avoir dans mon 5e. chapitre, fait quelques réflexions générales sur l'utilité de l'Histoire Naturelle, j'ai traité ce sujet dans un ordre systématique, & en ex\setcounter{page}{203} Posant aux amateurs les principes de la classification Linnéenne des insectes, je leur ai fourni le moyen de les distinguer les uns des autres, d'étudier leur organisation, d'éviter les erreurs & de propager les découvertes dans ce genre."
Comme les transformations de ces animaux constituent l'une des branches principales de leur histoire, & qu'elles fournissent au microscope un grand nombre d'objets curieux, j'en ai donné une description fort étendue ; j'y ai été acheminé par la confédération des erreurs qu'avoient commises à cet égard plusieurs Naturalistes, pour n'avoir pas étudié ces métamorphoses avec assez d'attention. Je voulois m'en tenir-là ; mais entraîné par les charmes de mon sujet, j'ai décrit avec détail les traits saillants de l'organisation de ces petites créatures ; & si mes lecteurs éprouvoient, en parcourant cette partie de mon ouvrage, autant de plaisir que j'en ai eu à l'écrire ; si ce sentiment leur donnoit quelque prédilection pour cette branche des études naturelles ; s'il les décidoit seulement à entreprendre la lecture de l'ouvrage étonnant de l'immortel Swammerdam ; un de mes souhaits les plus ardents feroit accompli."
L'auteur donne dans le chapitre suivant, quelqu'idée de la structure intérieure des insectes. Il rend justice aux travaux du célèbre Lyonnet & il emprunte de sa description de la chenille du saule, tous les curieux détails qui, en développant\setcounter{page}{204} l'organisation singuliere de cet insecte, montrent en même temps jusqu'où peut atteindre l'industrie d'un observateur muni de bons instrumens & doué d'une patience à toute épreuve.
Le chapitre suivant est consacré en entier à l'histoire des polypes d'eau douce. Les propriétés de cette classe d'insectes sont tellement extraordinaires, qu'on les considérera dans le principe, comme aussi opposées au cours régulier de la nature, qu'elles l'étoient aux opinions reçues sur l'animalité. Que dire, en effet, 'animaux qui multiplient par rejettons & de bouture, comme une plante; qu'on peut enter comme une branche sur une autre; qu'on peut retourner comme un gant, & qui ne cessent point de vivre & d'exercer les diverses fonctions auxquelles leur organisation fut appropriée? Ce chapitre est terminé par quelques détails sur les vorticelles dont Linné a donné l'énumération.
On se croiroit ici aux derniers confins de l'animalité, & l'œil humain y trouveroit du moins ses limites, sans le secours du microscope. Cet instrument lui découvre dans les infusions des végétaux un Océan peuplé de myriades d'êtres organisés, de formes bizarres & infiniment variées, qui par leur transparence presque parfaite semblent appartenir encore à demi à l'élément aqueux dans lequel ils nagent,\setcounter{page}{205} & qui paroissent, par la vivacité de leurs allures, vouloir se hâter de jouir d'une vie que la nature ne leur prêta que pour de courts instans. L'auteur dans son huitieme chapitre, l'un des plus curieux de l'ouvrage, passe en revue, d'après le système de classification du laborieux Müller, trois cent quatre-vingt-trois espèces de ces animalcules; en ajoutant aux descriptions de ce Naturaliste des détails intéressans sur ceux de ces insectes qu'on se procure le plus facilement & qui peuvent par cette raison, se présenter le plus fréquemment aux curieux. Ce chapitre occupe 158 pages.
L'auteur expose dans le neuvieme, l'organisation des végétaux & celle des bois en particulier, telle qu'on l'observe sur leurs tranches minces coupées, soit transversalement, soit dans d'autres sens. Il décrit l'instrument avec lequel on se procure ces tranches; c'est un art nouveau, que Mr. Custance a porté à sa perfection. Cet Artiste prépare des assortimens qui comprennent jusqu'à 72 échantillons de bois différens, & que MM. Kapmacher & Jones, éditeurs de l'ouvrage d'Adams, joignent, pour l'ordinaire, aux grands microscopes qui sortent de leur atelier.
On trouve dans le dixieme chapitre des détails sur la cristallisation des substances salines; les Editeurs y ont ajouté un catalogue raisonné des objets qu'on peut conserver en forme de\setcounter{page}{206} collection pour exercer & instruire les amateurs.
Le onzieme chapitre renferme dans un ordre syftématique la defcription des coquillages microfcopiques , principalement tirée du bel ouvrage publié en 1784 par M. Walker. Les
figures repréfentent chaque individu d'abord dans fes dimenfions naturelles , & enfuite tel qu'on le voit au foyer de l'inftrument. On y a joint le catalogue defcriptif d'un grand nombre de graines, qui , pour la plupart , font très-curieufes à voir au microfcope. Les figures qui les repréfentent , ainfi groffies , font d'un fini parfait.
Enfin les Editeurs donnent dans le douzieme & dernier chapitre , des inftructions fort détaillées & très-claires , fur les moyens de recueillir & de conferver les infectes; ils y ont joint un catalogue fort étendu d'objets microfcopiques dont la collection pourroit former un cabinet d'un genre nouveau & très-inftructif. Ils y font entrer quelques-uns des produits les plus fubtils de l'induftrie humaine ; & les réflexions que leur fuggere la comparaifon de ces produits avec ceux de la Nature , terminent l'ouvrage. Nous allons tranfcrire cette péroraifon.
"--Quel contraste humiliant! cherchons-nous la beauté , l'harmonie , la perfection , nous les trouvons fans fin dans les œuvres de la Nature: & les produits de l'art , ceux qui nous paroifsent,\setcounter{page}{207} à l'œil nu, être, le fruit d'une adresse consommée dans l'ouvrier, & dont la beauté & la délicatesse nous charment; ces chefs-d'œuvres n'offrent, sous le microscope, que la preuve humiliante de l'infériorité des produits humains comparés aux glorieuses productions de la Nature. Les plus beaux tissus, les ouvrages de l'aiguille les plus parfaits, paroissent tellement grossiers, à cette épreuve, que nos belles les dédaigneroient autant qu'elles les admirent, si elles étoient jamais tentées de les examiner avec cet instrument. Mais en revanche, plus nous étudions de près les ouvrages de la Nature & mieux nous y découvrons la main puissante qui les ordonna. Considérez cette fleur: là les fibres qui échappent par leur ténuité à l'organe simple de la vue, se trouvent, à l'aide du microscope, composées d'autres fibres de plus en plus atténuées, dans une progression dont l'œil le mieux armé n'atteint pas le terme: l'ensemble de cette organisation présente une radiance céleste, une richesse de coloris qui semble destinée aux vêtements d'un ange, car ceux d'un monarque de l'Orient, dans toute sa gloire, sont loin d'en approcher. — Jamais donc, ni dans les œuvres de ses mains, ni dans celles de son génie, l'homme mortel n'atteindra la perfection."
On présume volontiers que des Artistes qui s'expriment ainsi ne doivent pas être facilement\setcounter{page}{208} satisfaits sur le mérite de leurs propres ouvrages. Ce sentiment est une sorte de garant en leur faveur. — Reprenons notre exposé ; & passons aux détails que nous avons promis.
On ne peut fixer avec précision l'époque où le microscope fut inventé. Si l'on donne ce nom à une simple lentille ou à une sphérule, il n'est point improbable que les Grecs & les Romains n'en aient connu l'usage. Sénèque disoit litteræ quamvis minutæ & obscuræ per visteam pilam aqua plenam majores clarioresque cernuntur. Nat. Quæst. L. 1. C. VII \footnote{"On voit les lettres, même très-petites & obscures plus claires & agrandies au moyen d'une boule de verre pleine d'eau."}. Mais les premiers Artistes qui aient fait des microscopes proprement dits, paroissent être Zacharie Jansens & son fils, qui travailloient avant 1619, année dans laquelle Cornelius Drebbel porta en Angleterre un instrument de ce genre fabriqué par eux. Fontana, qui écrivoit en 1646 disoit, de son côté, avoir fait des microscopes en 1618. C'est donc dans les premieres années du dix-septieme siecle que cet instrument fut inventé.
Le microscope simple, qui est composé d'une seule lentille, grossit d'autant plus que le foyer de cette lentille est plus court. Il faut beaucoup d'habitude pour savoir s'en servir, à cause de la petitesse du champ & de la difficulté de\setcounter{page}{209} porter fur l'objet, s'il eft opaque, une lumiere fuffifante. C'eft cependant cet inftrument que Leuwenhoeck, Swammerdam, Lyonnet & Ellis ont employé dans leurs profondes & patientes recherches. Vers l'an 1740 le Dr. Lieberkun le perfectionna beaucoup, en ajuffant la lentille au milieu d'un petit miroir concave percé au centre & très poli: ce miroir réflécit fur l'objet affez de lumiere pour qu'il puiffe être vu diftinctement, malgré la grande difperfion que caufe l'action de la lentille.
Nous décrirons, d'après l'auteur, les microscopes que les découvertes de Leuwenhoeck avoient rendus célébres dans toute l'Europe.
Ce n'étoit autre chofe, dit-il, qu'une très petite lentille double convexe, logée dans une monture compofée de deux plaques rivées enfemble & percées d'un petit trou. On plaçoit l'objet à l'extrémité d'une aiguille d'argent qu'on pouvoit approcher à volonté pour chercher le point de vue ; on fixoit l'objet, s'il étoit folide, avec un peu de colle ; & s'il étoit fluide on en mettoit une goutte fur une lame très mince de talc, ou bien de verre foufflé, qu'on colloit enfuite à l'aiguille. Les lentilles étoient de forces très différentes, & adaptées chacune au genre de l'objet à examiner; mais aucune de celles dont il fit préfent à la Société Royale, ne groffiffoit autant que les globules de verre qu'on avoit employés au même usage.\setcounter{page}{210} Il dit s'être convaincu par une expérience de quarante ans, que les lentilles les plus avantageuses pour faire des découvertes microscopiques sont celles qui, n'ayant qu'une force moyenne, laissent aux objets beaucoup de lumière & de netteté. Il consacrait toujours une de ses lentilles à un ou deux objets seulement; & il en avait constamment quelques centaines sous sa main."
On commença en 1665 à faire usage de petites sphérules de verre solide au lieu de lentilles, dans le microscope simple. Ces globules grossissent prodigieusement & font d'excellens microscopes lorsqu'on fait s'en servir. Mr. Adams décrit en grand détail le procédé du P. Della Torre pour fondre ces globules ensorte qu'ils soient bien transparents & bien sphériques; il faut pour cela une lampe d'émailleur & un attirail assez considérable. On les fait tout aussi bien, & beaucoup plus simplement par la méthode suivante qu'on trouve indiquée dans le Journal de physique de Nicholson ( juin 1797) & transcrite à la fin de l'ouvrage d'Adams.
"On prend une lame de verre à vitres, mince, & de moins d'une ligne de largeur. On la tient verticalement suspendue par son extrémité supérieure auprès d'une chandelle, à la distance convenable pour que la pointe de la flamme soufflée latéralement avec un chalumeau, puisse l'atteindre à environ un pouce au-dessus\setcounter{page}{211} de l'extrémité inférieure de la lame. Le verre s'amollit & le poids du morceau inférieur le fait filer en bas d'environ deux pieds; le filet de verre se réduit au diametre d'environ $\frac{1}{500}$ de pouce. On le sépare du morceau qu'il tient suspendu, & sans faire usage du chalumeau, on approche latéralement ce fil si fin vers la flamme bleue qui entoure vers le bas la mèche de la chandelle : aussitôt son extrémité rougit & se fond en un globule qu'on fait grossir successivement en continuant d'avancer l'extrémité du fil vers la flamme, jusqu'à ce qu'il ait atteint la dimension qu'on veut lui donner. On coupe alors le fil excédent, & l'on a un globule bien transparent & parfaitement sphérique."
On peut faire ces globules d'une maniere encore plus simple, mais moins durable, en prenant une goutte d'eau à l'extrémité d'une épingle & en la tenant suspendue auprès d'un très petit trou pratiqué dans une lampe de métal. Deux gouttes d'eau, séparées en partie par une lame mince de laiton, mais qui se touchent dans leur axe commun, forment aussi une lentille double-convexe, qui peut servir de microscope simple.
Mais de tous ces moyens extemporanés le plus efficace, lorsqu'il est question d'observer des animalcules dans un liquide transparent, c'est de suspendre au bout d'une pointe obtuse de métal une goutte de ce liquide lui-même, en quantité suffisante pour qu'elle forme un peu\setcounter{page}{212} plus qu'un hémisphère : les corps étrangers qu'il renferme paroissent énormément grossis lorsqu'on approche ce globule de l'œil ; & on peut surtout les observer avec avantage à la lumière de la chandelle.
Nous transcrivons avec plaisir ces détails d'une pratique facile, & qui tendent à mettre les moyens des découvertes à la portée d'un plus grand nombre d'observateurs, ou d'amateurs.
Ceux d'entr'eux qui ne sont pas familiarisés avec les principes de l'optique accueilleront peut-être quelques procédés extrêmement simples pour déterminer la force amplificative des divers microscopes. Nous les tirons du second chapitre.
On a peine à se persuader que la lentille interposée entre l'œil & l'objet qu'elle fait paroître sous de si grandes dimensions, n'a d'autre effet réel que de permettre à l'organe de percevoir distinctement un objet qu'il examine à une très petite distance ; effet que la structure ordinaire de cet organe n'admet pas à une distance plus rapprochée que celle d'environ huit pouces.
Pour s'assurer de cette vérité, il suffit de placer l'œil fort près d'une lentille de verre, dans le but de regarder au travers, un objet qu'on en approche ou qu'on en éloigne, jusqu'à-ce qu'on le voye distinctement & bien terminé ; qu'alors, en éloignant la lentille, on lui substitue.\setcounter{page}{213} une plaque mince de métal percée d'un petit trou, on verra l'objet aussi distinct & sous d'aussi grandes dimensions que lorsqu'on le regardoit au travers de la lentille; seulement il paroîtra moins éclairé. Son diametre fera agrandi, dans ce cas, autant de fois que la distance de l'objet à la lentille, ou au trou qui l'a remplacée, est contenue dans la distance à laquelle la vision est distincte à l'œil nu.
Ainsi, par exemple, si l'on a une lentille convexe des deux côtés, dont le foyer, soit d'un demi pouce, elle grossira 16 fois, environ, les dimensions linéaires des objets, parce qu'un demi pouce est contenu 16 fois dans huit pouces, distance ordinaire de la vision distincte. Plus on raccourcira le foyer des lentilles, ou plus on augmentera la courbure de leurs surfaces, & plus il y aura de différence entre ce foyer & la distance constante de huit pouces; par conséquent, plus elles grossiront.
Si au lieu de se servir d'une lentille, on employe un globule de verre; comme le foyer de ces globules est à la distance de $\frac{3}{4}$ de leur diametre, ce globule grossira autant de fois que cette quantité est contenue dans huit pouces. Ainsi, supposant que ce diametre soit de $\frac{1}{10}$ de pouce\footnote{Environ 25 décimillimètres. Voyez le tableau de réduction des Mesures anglaises aux nouvelles mesures françaises, T. VIII de ce Recueil, p. 480. (R)}, les $\frac{3}{4}$ feroient $\frac{3}{10}$ de pouce & le\setcounter{page}{214} diametre réel de l'objet seroit à son diametre apparent, comme 40/3 à 8 ou comme 3 à 320 ou comme 1 à 106.
Mais comme nous n'avons considéré qu'une dimension dans ce rapport, & que les surfaces augmentent comme les quarrés de leurs côtés semblables, une telle lentille rendroit la surfacé apparente d'un objet, plus grande qu'elle ne le paroit à l'œil nu, dans le rapport du quarré de 106 à 1; c'est-à-dire, onze mille deux cent trente-six fois.
On est quelquefois dans le cas de calculer l'agrandissement non seulement de la surface, mais du volume entier ou de la solidité: comme, par exemple, lorsqu'on veut savoir combien d'animalcules font contenus dans une goutte d'un liquide donné. Alors il faut élever au cube le rapport de l'amplification linéaire, ce qui, dans ce cas, donneroit le rapport de onze cent quatre-vingt onze millions seize mille, à un.
Le microscope composé est formé de plusieurs verres convexes disposés de maniere que c'est à l'image d'un objet, & non à l'objet lui-même, que s'applique la force amplificative de la lentille. Celui de ces verres qui se trouve le plus près de l'objet & qui forme l'image (qu'il présente renversée) se nomme la lentille objective. Le verre qui la suit est destiné à augmenter le champ de l'instrument ou la quantité de surface qu'on pourra voir à la fois; enfin la lentille\setcounter{page}{215} oculaire, qui permet à l'Oeil de s'approcher beaucoup, & par conséquent de voir l'image fort agrandie, & grossit proportionnellement l'objet. Cette dernière lentille est ordinairement composée de deux ou plusieurs verres lenticulaires, combinaison destinée à corriger l'aberration de sphéricité & à augmenter encore le champ de l'instrument.
La méthode la plus simple pour mesurer la force amplificative du microscope composé, est de placer à son foyer une règle qui porte de très-petites divisions & qui déborde de beaucoup ce foyer d'un côté. Alors, en mettant un oeil au microscope & tenant l'autre ouvert, on vient facilement à estimer quelle étendue occupe sur la règle, regardée à l'œil nu, l'une de ses divisions vue dans le microscope ; & le rapport de cette division à l'étendue apparente qu'elle occupe sur la règle est la mesure de la force amplificative linéaire de l'instrument. On élève ensuite ce rapport au quarré, s'il s'agit de surfaces, & au cube s'il est question de volumes, comme nous l'avons indiqué tout-à-l'heure.
Les Éditeurs ont joint à ce chapitre un détail sur l'application qu'on a faite au microscope du micromètre, c'est-à-dire d'un appareil propre à mesurer immédiatement les dimensions réelles des objets qu'on observe. On s'y est pris de plusieurs manières, & on a successivement\setcounter{page}{216} simplifié les procédés. Déjà depuis plusieurs années Mr. Martin & d'autres opticiens plaçaient au foyer de l'instrument, des lames de verre, d'ivoire, ou de corne, qui portaient des divisions très-fines auxquelles on comparait les objets observés en les mettant sur ces divisions, ou à côté d'elles ; mais on remarquait que l'épaisseur du verre nuisait à la clarté de la vision, & que l'ivoire & la corne éprouvaient des variations hygrométriques considérables. On doit à Mr. Cavallo l'heureuse idée d'adapter pour micromètres aux lunettes ou télescopes, des lames très-minces de nacre de perles fort exactement divisées \footnote{Trans. Phil. 1791.}. On a appliqué cette Invention avec le même succès aux microscopes. La nacre de perle est très-propre à cet usage ; elle est suffisamment transparente lorsqu'elle est amincie à l'épaisseur d'une feuille de papier ; elle a assez de consistance, & les plus fines divisions s'y tracent très-nettement. On donne pour l'ordinaire à ces lames, une vingtieme de pouce Anglais (environ treize décimillimètres) de largeur, & les Artistes Anglais, ont poussé si loin l'art des divisions, que Mr. Coventry , & d'autres, préparent de ces lames, dont les divisions donnent depuis 1/100 jusqu'à 1/5000 de pouce. Mrs. Jones joignent à leurs microscopes un assortiment d'une demi douzaine de micrometres\setcounter{page}{217} en verre & en ivoire, outre un ou deux en nacre de perles. Ils sont divisés en lignes & en quarrés, depuis 1/100 jusques à 1/1000 de pouce. Les plus petites divisions dont on fasse usage commodément avec le micrometre de verre font les 4000mes de pouce; & comme elles peuvent être croisées à angles droits par des lignes semblablement effacées, on a ainsi des quarrés qui paroissent très-distincts sous le microscope, & dont la surface est la seize millionieme d'un pouce.
Les micrometres de verre disposés en quarrés, & appliqués au microscope solaire, divisent l'image des objets projettée sur l'écran, d'une maniere très-commode pour les définir dans leurs dimensions exactes.
Entre les microscopes de toute espece qui sont décrits dans le troisieme chapitre, on doit distinguer, à ce qu'il nous semble, celui auquel Mr. Adams a donné le nom de Lucernal parce qu'on l'emploie à l'aide d'une lampe. Voici les principaux avantages qui caractérisent cet instrument, tels que l'auteur les développe.
"Comme la très-grande pluralité des objets qu'on peut se proposer d'observer sont opaques, & qu'il n'y en a qu'un très-petit nombre qui soient assez transparents pour être bien vus dans les microscopes ordinaires, on a long-temps desiré un instrument qui pût se prêter facilement à l'examen des objets opaques. Et même\setcounter{page}{218} lorsqu'on observe des objets transparents on perd souvent, dans les torrents de lumière qui les traversent, les parties les plus déliées & les plus curieuses ; tandis que d'autres paroissent sous la forme de lignes ou de taches noires, parce qu'elles ne laissent point passer du tout de lumière. Le microscope lucernal n'a point ces inconvénients ; on y voit les objets opaques d'une manière extrêmement nette ; leurs couleurs naturelles, loin de rien perdre, sont rehaussées & conservent toutes leurs teintes ; & les parties ou faillantes ou creuses sont rendues avec tout leur effet."
"Cet instrument a un autre grand avantage, qui lui est particulier : c'est la facilité avec laquelle les objets opaques s'y adaptent. Comme la texture & les parties délicates des plantes ou des animaux s'altèrent souvent par les préparations nécessaires pour les soumettre à l'observation lorsqu'on employe les microscopes ordinaires, on peut les examiner d'abord avec le microscope lucernal qui ne les altère nullement, & les étudier ensuite comme objets transparents."
"Ce microscope ne fatigue nullement la vue ; l'objet est représenté absolument au naturel, & offre un spectacle qui satisfait l'esprit. On peut le regarder des deux yeux à la fois sans être obligé d'en cligner un comme on le fait ordinairement."
"Enfin, au moyen de cet instrument, les\setcounter{page}{219} perfonnes les moins habituées au dessin peuvent tracer elles-mêmes les contours exacts des objets observés ; & les dessinateurs exercés peuvent travailler avec beaucoup plus de précision & de diligence en en faisant usage. La plupart des figures qui accompagnent cet ouvrage ont été dessinées par ce moyen, & j'espère que leur exécution témoignera en sa faveur \footnote{On en trouverait difficilement de plus belles.}. Sous ce point de vue, cet instrument ferait d'un grand secours à l'anatomiste, au botaniste, à l'entomologiste, &c. en les mettant à portée non-seulement d'étudier avec facilité les objets de leurs recherches, mais d'exécuter eux-mêmes des dessins exacts de ceux qu'ils se proposent de faire connaître."
Il faudrait un grand développement de figures pour décrire cet instrument : nous en indiquerons seulement la disposition générale. La lumière vive d'une lampe d'Argand est réfléchie, par un miroir concave, sur l'objet mis au foyer du microscope ; & l'image agrandie de cet objet est portée sur une glace dépolie, au-delà de laquelle, & dans l'axe de l'instrument, est placé l'observateur, qui voit l'objet peint sur la glace avec toutes ses couleurs. On peut par une autre disposition, faire de l'instrument une sorte de lanterne magique, lorsqu'il est question d'objets transparents ; enfin on peut encore\setcounter{page}{220} s'en servir à la façon de microscope ordinaire ; c'est donc un instrument universel, dans ce genre. Mrs. Jones l'ont encore, à ce qu'il paroît, fait avancer d'un degré vers la perfection. Cet appareil curieux coûte de dix-huit à douze guinées, selon qu'on l'exige plus ou moins complet.
Le sujet du quatrieme chapitre ; savoir, "les instructions générales pour l'usage du microscope & la préparation des objets," se subdivise en trois chefs. Sous le premier, sont indiquées toutes les dispositions qui ont rapport à l'instrument en particulier : le second traite de la quantité de lumiere & de la meilleure maniere de la diriger sur les objets à examiner ; on trouve sous le troisieme, les moyens de préparer & de conserver les différentes substances, selon leur organisation & leur texture. Nous transcrirons quelques-uns des conseils de l'auteur.
"Il faut, dit-il, toujours commencer par employer la lentille la moins forte ; on acquiert ainsi une idée exacte de l'ensemble & de la situation respective des parties, & l'on ne court plus de risque de se tromper lorsqu'on arrive aux verres les plus forts. — Les objets transparens supportent mieux les lentilles d'un foyer court, que ne le font les objets opaques."
"Il y a beaucoup de différence entre voir simplement un objet au microscope, ou l'étudier avec cet instrument. Dans le premier cas\setcounter{page}{221} nous nous contentons de recevoir l'impreffion d'une image formée par le jeu des verres; dans le fecond, nous formons un jugement d'après l'afpect de cette image; & c'eft ici qu'eft la difficulté: il faut connoitre fon fujet, être doué de patience, répéter les obfervations; car il y a des cas où les apparences font femblables malgré la différence totale des fubftances qui les font naître; c'eft à la pénétration de l'obfervateur qu'il appartient alors de diftinguer. -- Qu'il ne fe hâte point de prononcer; qu'il apprécie l'effet de toutes les circonftances qui peuvent influer fur ce qu'il voit; extenfion forcée, contraction occafionnée par le defféchement, &c."
"On peut être trompé fur la vraie couleur des objets lorfqu'on employe de très-fortes lentilles: l'amplification donne lieu à des jeux de lumière accidentels, qu'il faut favoir reconnoître."
Sur la quantité de lumière, l'auteur prefcrit de la modifier felon la nature des fubftances qu'on étudie, & il paroît préférer la lumière de la chandelle, ou d'une lampe, à celle du jour. On peut mieux difpofer de celle-là à fon gré; & fous ce rapport encore, le microfcope lucernal a l'avantage fur les autres. En parlant de la préparation des objets, "perfonne, dit-il, ne fut à cet égard plus habile & plus heureux que Swammerdam. Dans l'adreffe des disseclions\setcounter{page}{222} dans la persévérance des recherches, dans l'art de déployer aux regards des curieux les richesses microscopiques de la création, il n'a point eu & n'aura peut-être jamais d'égal. Profondément frappé & constamment animé par les scènes étonnantes que lui dévoiloient ses observations, aucun mécompte, nulle difficulté ne l'arrêtoit dans la poursuite de la vérité. — Boerhaave a examiné avec beaucoup d'attention tous ses manuscrits, dans l'espérance d'y trouver quelques détails sur ses procédés. — Il paroît que son moyen principal de dissection étoit l'emploi de ciseaux extrêmement fins & tranchans ; il les préféroit aux bistouris & aux lancettes qui n'ayant pas, comme les ciseaux, un point d'appui réciproque, déchirent souvent les organes très-délicats & déplacent des filamens qu'ils entraînent. Ses instrumens pointus étoient si déliés qu'il ne pouvoit les aiguiser qu'à l'aide d'une loupe ; mais aussi il disséquoit, par leur moyen, les intestins des abeilles avec autant d'aisance & de précision que peut le faire un anatomiste qui travaille sur les mêmes organes chez de grands animaux. Il se servoit de même, avec beaucoup d'adresse, de petits tubes de verre tirés en pointe très-fine à l'une de leurs extrémités, pour souffler dans les plus petits vaisseaux, & les injecter au besoin avec des liqueurs colorées. — Les insectes ont presque tous à l'intérieur plus ou moins de graisse, dont la présence rend\setcounter{page}{223} leur dissection d'autant plus difficile : Swammerdam découvrit que cette substance était parfaitement dissoluble dans l'essence de térébenthine, ce qui lui fournit les moyens de dégager très-complètement leurs viscères pour les étudier & les démontrer. — On l'a vu fréquemment employer une journée entière à dissoudre ainsi la graisse de l'intérieur d'une chenille pour découvrir la véritable structure du cœur de cet insecte. — Sa manière d'enlever leur peau à l'époque où elles sont prêtes à filer leur coque, mérite d'être citée : il les plongeait un instant, suspendues par leur fil, dans l'eau bouillante ; l'épiderme s'enlevait alors facilement. Il les mettait ensuite dans un mélange à parties égales, de vinaigre distillé & d'esprit-de-vin : cette immersion donnait aux divers organes une consistance suffisante pour qu'on pût les séparer de leurs enveloppes sans les déformer. Il montrait ainsi la nymphe renfermée dans la chenille, & le papillon emmaillotté dans la nymphe. On ne peut lire les ouvrages de ce grand Naturaliste sans éprouver un extrême plaisir, soit que l'on considère l'étendue de ses travaux & son ardeur infatigable, soit que l'on réfléchisse aux sentiments de piété qui se déployaient dans son âme chaque fois qu'un nouveau miracle de la création se découvrait à ses yeux ; l'Auteur de ces merveilles était toujours présent à sa pensée, & son\setcounter{page}{224} cœur bondissoit vers lui à chaque trait de l'admirable sagesse dont toutes ses œuvres portent l'empreinte."
Les procédés de Lyonnet sont aussi développés par Mr. Adams, & nous avons le regret de ne pouvoir le suivre dans tous ses détails. — Voici comment il prescrit de s'y prendre pour observer, soit la circulation du sang, soit la figure des molécules rouges qui entrent dans la composition de ce fluide.
L'un des animaux chez lesquels la circulation s'observe le mieux est une petite anguille ou un petit gougeon. On le lave bien, pour détacher la couche gluante qui le recouvre, puis on l'étend sur la gouttiere presque plate qui accompagne d'ordinaire les microscopes, & est destinée à ce genre d'observations; on l'y maintient par des fils qui lacent d'un côté à l'autre & l'on observe la queue presque transparente de l'animal. Ou bien on met le petit poisson dans un tube un peu plus gros que lui, & rempli d'eau. Leuwenhoeck donne dans sa 112e. lettre, une description exacte des vaisseaux sanguins dans la queue d'une anguille.
On a décrit aussi un appareil propre à observer la circulation dans le mésentère d'une grenouille vivante. "Mais, dit notre auteur, comme la cruauté de ce genre d'expériences priveroit le naturaliste sensible de toute la satisfaction qu'il pourrait en retirer, il se contentera\setcounter{page}{225} sans doute de savoir qu'elles ont été faites. Il y a amplement dans la nature de quoi alimenter sa curiosité sans faire éprouver à l'observateur des sensations aussi pénibles."

The poor beetle that we tread upon
In corporeal sufferance finds a pang as great
As when a giant dies.\footnote{Le pauvre Scarabée que nous écrasons sous nos pieds, meurt dans des souffrances aussi cruelles que peuvent l'être celles d'un géant à l'agonie.}
SHAKESPEARE.

Les Editeurs citent ici, en note, une réflexion de Montaigne que nous relevons aussi volontiers. C'est que "toutes les espèces de créatures animées ont droit à une forte de bienveillance et de compassion de la part de l'homme."Et, ajoutent-ils, il est fort à regretter qu'on ne cherche pas avec plus de soin qu'on ne le fait, à inculquer ce principe dans le cœur des jeunes gens. On devroit tâcher d'étouffer de bonne heure cette maligne disposition de la plupart des enfants à tourmenter les êtres sensibles qui font dans leur dépendance et sur lesquels ils exercent leur tyrannie. Le moindre inconvénient qui résulte de la négligence des parents ou des instituteurs à cet égard, est l'habitude que prennent les enfants de s'endurcir aux souffrances qu'ils n'éprouvent pas eux-mêmes; et trop souvent, ils passent de cette apathie aux actes de la cruauté la plus atroce. Le fénat\setcounter{page}{226} d'Athènes, punit, comme on fait, un enfant pour avoir crevé les yeux d'un oiseau; & l'inimitable Hogarth ce grand peintre du genre humain, a représenté admirablement dans ses cinq périodes de la cruauté, \footnote{La première des cinq estampes représente des enfants qui dans différents jeux se plaisent à tourmenter toutes sortes d'animaux. On voit dans la quatrième un homme qui cherche à ensevelir, à la clarté d'une lanterne, une femme grosse qu'il a égorgée & dont il a coupé les mains. (R)} les terribles conséquences de l'inattention des parents à cette fatale légèreté chez leurs enfants."
On ne s'attend gueres à recevoir des conseils d'éducation dans un Essai sur le microscope. — Revenons au sujet.
Pour voir les molécules rouges du sang, on peut s'y prendre de plusieurs manieres. Ou bien l'on en étend une très-petite goutte en couche aussi mince qu'il est possible, sur un verre plan. En la délayant un peu avec de l'eau tiède les molécules se subdivisent d'elles-mêmes en d'autres plus petites, qu'on peut voir isolées. Ou bien comme le prescrit Mr. Baker, on mêle à ce fluide un peu de lait & on l'aspire dans un tube de verre d'un diamètre extrêmement fin. Mr. Hewson, qui a particulièrement étudié ce sujet, soutient que c'est très-improprement qu'on a nommé globules les molécules du sang, attendu qu'elles ne sont\setcounter{page}{227} point rondes mais aplaties. Il le démontre en délayant un peu de sang caillé, dans la lymphe même, ou la sérosité qui s'en sépare par le repos. La lymphe, ainsi légèrement colorée, mise sous le microscope sur un plan de verre un peu incliné pour que la partie la plus fluide s'échappe doucement, laisse voir les molécules rouges sous la forme de disques avec un point noir au milieu\footnote{Ne pourrait-on point objecter que la formation préalable du caillot étant une forte de cristallisation, elle pouvait avoir dénaturé la forme primitive des molécules? (R)}.
L'auteur après avoir donné les directions nécessaires pour préparer les infusions dont on se propose d'observer les animalcules, et pour soumettre ces infiniment petits à l'action du microscope, termine ces détails par une réflexion que nous croyons devoir transcrire.
"L'observateur, dit-il, doit être très-réservé à former un jugement sur la nature, l'usage, et les opérations de ces animalcules, s'il prétend raisonner d'après les idées qu'il s'est formées en étudiant les propriétés des animaux ordinaires. Car ces moyens surnaturels que nous fournit le microscope, nous introduisent, en quelque sorte, dans un monde nouveau, où nous trouvons, non pas un petit nombre d'individus, mais des milliers d'espèces tellement singulières dans leurs apparences, leur formation,\setcounter{page}{228} leurs habitudes, que si cet aspect agrandit des notions que nous nous sommes formées de la toute puissance créatrice, il semble en revanche, par ce qu'il nous offre d'incompréhensible, nous montrer les limites de l'entendement humain plus rapprochées que nous n'aurions été disposés à le croire.
Mais nous dépassons sans nous en apercevoir, les bornes d'un extrait ordinaire. Nous n'avons fait cependant que parcourir très-rapidement les cinq premiers chapitres, soit la partie technique de l'ouvrage. Nous entretiendrons nos lecteurs dans un prochain numéro, de la nombreuse collection de faits qu'il renferme.

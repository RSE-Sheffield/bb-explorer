\setcounter{page}{99}
\chapter{ROMANS}
\section{WAVERLEY, ou IL Y A SOIXANTE ANS. (Edinburgh 1814) 3 vol. in-12°} \large{(Second Extracit. Voy. page 532 du volume précédent)}

WAVERLEY, dont la curiosité est vivement excitée, rassemble beaucoup d'informations sur les terribles habitans des montagnes voisines. Le régisseur Macwheeble lui fait le détail effrayant des dégats qu'ils commettent, et le Baron, tout en exhalant sa colère, raconte diverses anecdotes qui le mettent au fait de leurs mœurs. "J'avoue, dit ce dernier, que leurs chefs sont, en général, des gentilshommes fort considérés et d'une haute naissance, dont la parole a force de loi\setcounter{page}{100} parmi leur clan. Il ne leur convient pas, à la vérité, de mettre leur noblesse (un lignage dont toute la preuve repose sur les vaines rimes et les chansons de leurs Bardes ou Sennachies), de pair avec l'évidence des anciennes chartes et concessions royales, accordées aux maisons illustres de la plaine par divers Monarques écossais. Et cependant, telle a été dernièrement l'outrecuidance et la présomption de ces chefs, qu'ils ont osé ravaler ceux qui possèdent de tels actes, comme s'ils n'étaient que d'abjects tenanciers."
Ces paroles, bien plus que le vol du troupeau, expliquent à Waverley le profond ressentiment du Baron contre le chef de la tribu voisine, et la conversation continuoit encore, quand tout-à-coup la porte s'ouvre, et un Montagnard complètement armé et équipé s'offre à leurs regards. Cette introduction martiale auroit certainement paru hostile à Waverley, si le sommelier n'avoit pas fait l'office de maître des cérémonies, et si le Baron et sa fille n'avoient pas conservé un calme parfait; mais quoiqu'il en fût, il ne put se défendre d'un peu d'émotion. Le Gael\footnote{Montagnard d’Ecosse, parlant la langue gælique} étoit un homme noir, ramassé et\setcounter{page}{101} remarquable par un air de force qu'un manteau à plis très-amples rendoit plus frappant encore. Son kilt ou jupon court laissoit voir ses jambes nerveuses et bien tournées. Une bourse de peau de chèvre, dûment flanquée de ses défenses ordinaires, un poignard et un pistolet, étoit suspendue devant lui, et son bonnet, surmonté d'une plume courte, proclamoit le droit qu'il avoit d'être traité en homme bien né. A son côté pendoit une épée à deux mains (claymore) et son bouclier (target) étoit sur son épaule. D'une main il tenoit un long fusil de chasse; de l'autre il tira son bonnet, et le Baron qui savoit la manière dont on devoit parler à ces gens-là, lui dit d'un air de dignité, mais sans se lever, et comme un Prince qui recevroit une ambassade. "Soyez le bien venu, Evan Dhu Maccombich, quelles nouvelles de Fergus Mac Ivor Vich Ian Vohr?"
"Fergus Mac Ivor Vich Ian Vohr vous salue, baron de Bradwardine et de Tully Veolan, répondit l'ambassadeur en bon anglais, et il est fâché qu'un épais nuage élevé entre vous et lui, ait obscurci l'amitié \footnote{ou gallique, ou erse, dérivée du celte, et bien différente de l'écossais de la plaine, qui n'est qu'un anglais corrompu. (R)}\setcounter{page}{102} et l'alliance qui ont uni de tous temps vos ancêtres et les siens; il demande que le nuage se dissipe, et que le Clan Yvor et la maison de Bradwardine se remettent sur le pied d'autrefois, alors que les festins tenoient lieu de batailles. Il s'attend que vous direz de votre côté que vous êtes affligé de ce nuage, et à l'avenir, personne ne s'informera s'il est descendu de la montagne dans la vallée ou monté de la vallée à la montagne, car malheur à celui qui perd son ami pour la nuée orageuse d'un matin d'été."
"A cela le Baron répondit avec majesté: qu'il savoit que le chef des Mac Yvor étoit bien disposé pour le Roi, (mot équivoque qui s'entend ici du roi Jacques,) et qu'il seroit fâché qu'un nuage s'interposât entre lui et un noble pénétré d'aussi excellens principes, car," dit-il, "alors quelles partis se rassemblent, foible est celui qui n'a point de frère."
La paix se solemnise en buvant les santés des puissances contractantes, et l'ambassadeur celtique, après avoir réglé avec Macwheeble quelques articles secrets, apparemment relatifs au subside qu'exigeoit son chef, prend des informations sur le vol de la nuit précédente.\setcounter{page}{103} Waverley est très-frappé de la perspicacité et de la justesse d'esprit qu'il déploie dans cette enquête, et comme lui-même plaît beaucoup à Evan Dhu, ce dernier lui propose de l'accompagner à quinze milles de là dans les montagnes, au lieu où il suppose qu'on a caché le bétail du Baron. "Car," dit-il, "à moins que moi ou un de mes" pareils ne vous y menions, vous ne verrez" jamais rien de semblable en votre vie."
Malgré un peu d'effroi qu'éprouve Rose, Edouard accepte cette offre; le Baron insiste pour qu'il emmène son garde-chasse, et il part avec l'ambassadeur et sa suite, composée de deux montagnards, dont l'un portoit sur l'épaule une hâche au bout d'un pieu, Evan Dhu, frère de lait du chef, ayant jugé cette escorte nécessaire à sa dignité.
La caravanne s'enfonce dans les montagnes, à travers les torrens, les rochers et les précipices. Arrivés au défilé de Bally Brongh, qui jadis avoit été défendu par dix hommes du Clan Donnochie contre cent Écossais de la plaine, Eyan montre à Waverley la place où quelques-uns de ces braves avoient été enterrés. La nuit survient et trouve les voyageurs en chemin. On emmène le garde-chasse, parce que, dit-on, "Donald Bean," le brigand\setcounter{page}{104} qu'on croit être en possession du troupeau, n'aime pas les visites des étrangers à moins que ce ne soient des amis très-particuliers. Evan Dhu part lui-même pour préparer ce Donald à l'arrivée d'Edouard, et ce dernier, non sans quelque effroi, reste seul avec l'homme à la hache.
Ils traversent ensemble un bois de pins qui paraît sans fin à Waverley; c'est en vain qu'il questionne son guide; celui-ci ne lui répond qu'en langue gallique; enfin, étant parvenus au bord d'une large rivière ou d'un lac, cet homme lui fait signe de s'asseoir. La lune se levait alors, et on pouvoit confusément apercevoir une grande étendue d'eau et les formes indécises des montagnes. L'air frais d'une nuit d'été restaure les sens d'Edouard, après une marche rapide et fatigante, et il respire avec plaisir le parfum qu'extraient les bonteaux, tout humides de rosée.
Là, il a le temps de laisser une libre carrière à son imagination. Seul, à minuit, avec un homme d'un aspect étrange et sauvage, sur la rive d'un lac inconnu, prêt à visiter le repaire de quelque insigne brigand, il savoure le charme particulier attaché aux situations romanesques. Une seule circonstance lui déplaît, mais il en écarte l'idée,\setcounter{page}{105} c'est la cause de son voyage, le troupeau de vaches du Baron !
Tandis qu'il méditoit encore, son guide le secoue doucement et lui faisant signe de regarder du côté du lac, Edouard voit dans le lointain un point de lumière, lequel augmentant rapidement de grandeur et d'éclat paroît bientôt comme un météore de feu qui ondoie à l'extrémité de l'horizon. A l'instant un bruit de rames se fait entendre, et peu après, un sifflement aigu, auquel le compagnon d'Edouard répond de la même manière ; bientôt on distingue un bateau conduit par cinq ou six montagnards; deux de ces hommes s'élancent sur le rivage, enlèvent Edouard d'un bras robuste et, l'ayant placé dans le bateau, ils se remettent à la rame, et leur léger esquif s'éloigne de terre avec une grande rapidité.
Le profond silence que garde l'équipage, n'est interrompu que par le murmure monotone d'une chanson gallique que frédonne le pilote, en manière de sourd récitatif, et par le bruit cadencé des rames qui se règle sur cette musique. La lumière vers laquelle on se dirige, prend en s'agrandissant toujours, un éclat plus rouge et une forme plus irrégulière. C'étoit évidemment un feu énorme;\setcounter{page}{106} mais s'il étoit allumé dans une isle ou sur le rivage, c'est ce qu'Edouard ne pouvoit pas distinguer. Ce feu touchoit certainement la surface de l'eau, car, réuni à son image, il formoit un disque flamboyant, en apparence suspendu dans le voide, et ne ressembloit pas mal au char ardent sur lequel le mauvais Génie d'un conte oriental, traverse les terres et les mers.
Le bateau étant près du rivage, Edouard voit que ce feu est sans cesse entretenu par deux figures noires, auxquelles la réverbération de la flamme donnoit l'air de véritables démons, et qu'il est allumé dans la bouche d'une grande caverne qui admet l'eau du lac jusqu'à la moitié de sa profondeur. Arrivés vis-à-vis de cette ouverture, les bateliers donnent un violent coup de rame et retirant tout à coup leurs avirons, le bateau porté en avant par l'impulsion qu'il a reçue, dépasse la petite platte-forme sur laquelle le feu est allumé et s'arrête au fond de la caverne, devant de grands quartiers de roche qui y forment un espèce d'escalier naturel. On jette une énorme quantité d'eau sur le feu, qui s'éteint aussitôt avec un sifflement et une fumée terribles, et l'on reste dans une complète obscurité. Des bras nerveux enlevé\setcounter{page}{107} vent Édouard, le placent sur ses jambes et l'entraînent au fond de la grotte. Il fait quelques pas dans les ténèbres, s'avançant toujours vers des voix confuses qui semblent partir du centre même du rocher, et au tournant d'un passage oblique, Donald Bean et tout son établissement se découvrent subitement à ses yeux.
L'intérieur de la grotte, qui là avait beaucoup d'élévation, était illuminé par des torches de pin qui répandoient une vive clarté, accompagnée d'une odeur forte mais point désagréable. À cette lumière se joignoit l'éclat ardent d'un énorme brasier autour duquel étoient assis cinq ou six Montagnards armés; d'autres qu'on apercevoit indistinctement dans divers recoins de la caverne, dormoient enveloppés de leurs manteaux, et au fond d'une large niche, on voyoit une brebis égorgée suspendue par les pieds, et deux vaches récemment tuées. Le maître de cette singulière demeure, accompagné d'Evan Dhu, s'avance bientôt pour recevoir Waverley; celui-ci s'attendoit, d'après tous les objets qui frappoient ses regards, à voir dans la personne de Donald Bean une figure gigantesque, féroce, et telle qu'un Salvator l'auroit choisie pour la placer au centre d'un\setcounter{page}{108} groupe de bandits. Point du tout, le brigand Écossais est un petit homme pâle, maigre, qui a servi en France, et s'est assez ridiculement affublé d'un vieil uniforme rouge et bleu pour faire honneur à son hôte. Il étonne cependant Waverley par des connaissances très-précises et très-détaillées sur l'état politique et militaire du pays, il lui cite plusieurs particularités ignorées sur son propre régiment, et sait jusqu'au nombre de recrues qu'il a amenées des terres de son oncle.
"N'avez-vous rien de particulier à me dire?" lui demanda Donald à voix basse, d'un air significatif, et quand Édouard un peu étonné lui répondit qu'il n'étoit venu que par curiosité, le brigand le regarda fixement pour pénétrer s'il étoit sincère, et dit ensuite : "Vous auriez pu tout aussi bien vous fier à moi qu'au Baron de Bradwardine ou à Vich Ian Vohr, mais c'est égal, vous êtes toujours le bien venu chez moi." — Ces paroles laissèrent un peu de trouble à Waverley.
On sert un repas très-grossier mais abondant, qui est avidement dévoré par tous les convives, et pendant lequel Donald déploie une grande politesse. L'on donne pour lit à Waverley, un matelas de bruyère dont les\setcounter{page}{109} fleurs sont tournées en dehors, et on le couvre de manteaux écossais. Les montagnards vont et viennent toute la nuit dans la caverne, il en arrive de nouveaux, apportant le butin qu'ils ont enlevé; ceux-ci coupent avec leurs coutelas, pour leur souper, de larges tranches des animaux égorgés. Enfin toutes ces voix galliques s'affaiblissent pour Edouard, tous ces groupes fugitifs vacillent devant ses yeux, et il tombe dans un profond sommeil. A son réveil, il trouve la caverne vide. Voulant sortir de ce repaire, il arrive dans la grotte extérieure, mais là il ne voit point d'autre issue que le lac qui y pénètre. Enfin il découvre, à l'extrémité de la platte-forme où l'on avoit allumé le feu, quelques légères entailles, pratiquées dans le roc, et s'en servant comme d'escalier pour monter, il parvient en tournant au sommet du rocher, d'où il redescend, non sans peine, de l'autre côté. Sur le bord du lac sauvage qu'il n'avoit vu que de nuit, il tourne ses regards vers le lieu qu'il a quitté, et admire le choix heureux de cette retraite ignorée. Du côté de terre, il ne voit plus qu'un rocher, en apparence inaccessible; et du côté du lac, la voûte basse et enfoncée par laquelle il est entré, ne peut être distinctement aperçue de la rive opposée.\setcounter{page}{110} Ayant bientôt retrouvé Evan Dhu qui l'attendoit, celui-ci le conduit auprès de la fille de Donald Bean, Alice, une jolie Montagnarde qui leur avoit préparé un déjeûner à l'ombre d'un bouleau. Après un repas agréable, ils remontent dans le bateau et continuent leur voyage.
On trouvera peut-être que l'attente est excitée plus qu'elle n'est remplie. On suppose qu'il va arriver des scènes merveilleuses, et il n'y en a pas, on se croit dans les siècles barbares et l'on rentre brusquement dans la civilisation. Mais c'est ce contraste même que l'auteur a voulu faire ressortir. C'est le passé et le présent, devenus pour ainsi dire contemporains, qui rendent remarquable l'époque qu'il a dépeinte.
Au sortir du bateau, Evan annonce à Waverley qu'il le mène à Glennacoich, chez son chef, auquel cette visite est annoncée. Chemin faisant, les voyageurs causent ensemble. Waverley apprend de son guide que le troupeau du Baron est en route pour Tully Veolan, à l'exception de deux vaches qui, malheureusement, étoient déjà tuées, et comme il s'émancipe à appeler Donald Bean un voleur, Evan trouve cela très-mauvais : "Voleur, dit-il, pas du tout;\setcounter{page}{111} , il n’a jamais enlevé moins d’un troupeau
en sa vie...... Celui qui prend une vache à
une pauvre veuve ou une chèvre à un indigent, est un voleur, mais celui qui emmène le troupeau d’un Laird Saxon (Anglais) est un gentilhomme fourrageur. D’ailleurs, enlever un arbre de la forêt, une
truite de la rivière, ou des vaches de la plaine, c’est ce dont n’a honte aucun Montagnard. "
— "Mais quelle sera sa fin s’il est surpris à exercer son métier?,"
— "Eh bien, il mourra par la loi, comme bien d’autres braves gens l’ont fait avant lui. Il a déjà failli être pendu à la potence du Roi à Crieff, où est mort son père, où est mort son grand’père, où il faut espérer qu’il vivra assez pour mourir lui-même; s’il n’est pas tué dans quelque expédition auparavant. "
"Et vous espérez une telle mort pour lui : ami, Evan!,"
"Pourquoi pas? Voulez-vous que je lui souhaite de mourir sur un tas de paille dans sa caverne?,"
"Et sa fille Alice?,"
"Eh bien! si cet accident arrive, comme
son père n’aura plus besoin d’elle, je ne\setcounter{page}{112} "vois pas ce qui m'empêchera de l'épouser."
"Bravement parlé, mon cher Evan."
La route est longue, et à mesure qu'on approche de Glennaquoich, Evan s'attache davantage à relever la dignité de son chef.
"Je voudrois," dit-il, "qu'il vînt à votre rencontre avec sa queue !"
"Avec sa queue ?"
"Oui, avec tous ceux qui l'accompagnent quand il va visiter les chefs voisins. Il a d'abord, ajoute Evan, qui se redresse avec fierté et compte sur ses doigts, il a son bras droit (hanchman) puis son poète (bhaird) puis son orateur (bladier) pour faire des harangues aux grands personnages qu'il visite, puis son porteur d'armes (gilly more) puis son passeur d'eau ( gilly casflue ) qui le porte sur ses épaules à travers les torrens, puis le guide de son cheval ( gilly coms-traine ) qui conduit cet animal par la bride dans les sentiers escarpés et difficiles, puis son porteur de havre-sac ( gillie trushar-nish ) enfin le joueur de cornemuse et son garçon, sans compter une douzaine de jeunes gens qui n'ont point d'emploi particulier, et ne sont là que pour suivre le laird et pour exécuter ses ordres."
"Et entretient-il tous ces gens-là?"
" Tous\setcounter{page}{113} "Tous ces gens-là? Bien d'autres braves gens encore, je vous assure, qui ne sauroient où coucher sans la vaste grange de Glennaquoich."
Comme ils parloient, un coup de fusil est entendu, et au grand étonnement d'Evan, de ce que Fergus marchoit sans sa queue, ce chef, en habit de chasseur et suivi d'un seul jeune garçon, s'offrit tout-à-coup à leurs regards.
Dans le fait, Fergus Mac Yvor avoit beaucoup trop de connoissance du monde pour croire imposer du respect à un officier Anglais, par le vain étalage d'un cortège qu'il auroit pu ne trouver que ridicule. Sa figure parut à Waverley frappante de grâce et de dignité. Au-dessus de la taille moyenne et admirablement bien tourné, il portoit l'habit montagnard dans sa forme la plus simple, et lui donnoit une élégance toute particulière. Sa physionomie, décidément écossaise, offroit tous les traits caractéristiques du nord, mais sans rudesse et sans exagération, et dans tous les pays elle eut été reconnue pour belle. La plume unique qui surmontoit son bonnet avoit quelque chose de martial et de gracieux, et une abondance\setcounter{page}{114} de cheveux noirs et bouclés ajoutoit encore à l'agrément de cette figure.
A tous ces avantages, Fergus joignoit un air ouvert et affable, qui produisoit d'abord l'impression la plus favorable. Cependant, un physionomiste habile eût peut-être été moins satisfait de l'expression de sa figure à un second qu'à un premier examen. Son sourcil et sa lèvre supérieure sembloient déceler les habitudes d'une volonté despotique et d'une supériorité hautaine. Sa politesse même, quoique naturelle et franche, portoit l'empreinte du sentiment de son importance, et à la moindre contradiction, une étincelle subite mais fugitive, dans son regard, trahissoit un caractère violent et fier, qui pour être souvent contenu n'en devenoit pas moins à craindre. En un mot, tout l'extérieur de ce chef donnoit l'idée de ces jours clairs et sereins de l'été, dans lesquels nous connoissons à des signes très-légers mais indubitables, qu'il y aura de l'orage, et que la foudre éclatera avant la fin de la soirée.
Fergus Mac Yvor, (de beaucoup la figure la plus saillante du roman de Waverley) issu d'une race Ecossaise très-illustre, comptoit parmi ses ancêtres un guerrier fameux,\setcounter{page}{115} appelé Jean le grand, lequel, entraînant à sa suite une partie du clan de ses pères étoit venu trois cents ans auparavant s'établir dans le Perthshire, et y avoit bâti le château de Glennaquoich. Le père de Fergus, après avoir joué un rôle actif dans la malheureuse insurrection de 1715 en faveur des Stuarts, obligé de se réfugier en France, épousa une femme de qualité de ce pays, et en eut deux enfants, Fergus et Flora. A sa mort, sa terre de Glennaquoich, qui avoit été confisquée, fut rachetée à un bas prix pour son fils, et ce jeune homme revint, en conséquence, s'établir dans son pays natal. Là, l'esprit actif et pénétrant de Fergus, ses talens militaires et l'éducation française qu'il avoit reçue, lui firent déployer un caractère tout particulier, et qui ne pouvoit exister qu'à l'époque précise où il vivoit ; car soixante ans plus tôt, il n'eût point eu ce vernis d'usage du monde et d'élégance qui le distinguoient, et soixante ans plus tard, le feu de son ambition et de sa passion pour l'autorité, auroient manqué de l'aliment nécessaire.
La tribu, que gouvernoit Fergus n'étant pas très nombreuse, ce chef mettoit une habileté extraordinaire à fortifier son pou-\setcounter{page}{116} voir et à augmenter son importance. Ainsi il fomentoit entre les clans voisins, des divisions, dont il cherchoit à se rendre l'arbitre. Il maintenoit, à quelque prix que ce fût, dans toute sa grossière abondance, l'hospitalité illimitée qui étoit l'attribut le plus distingué d'un chef de clan ; il retenoit dans ses domaines le plus qu'il pouvoit de tenanciers hardis et vigoureux, soit de sa propre race, soit de la souche primitive, et même des étrangers disposés à suivre sa bannière et à prendre le nom de Mac Yvor.
Il avoit réussi à se faire nommer commandant en chef d'un des corps levés par le Gouvernement pour maintenir la tranquillité dans les montagnes de l'Ecosse ; et, en cette qualité, il agissoit avec beaucoup de vigueur et de célérité, et faisait régner un grand ordre dans le pays. Toutefois ses vues en cela n'étoient pas désintéressées ; il vouloit donner des habitudes militaires à ses vassaux en les faisant entrer successivement dans sa compagnie, et il usoit d'ailleurs fort arbitrairement du pouvoir qui lui avoit été confié. C'étoit, par exemple, avec une indulgence très-suspecte, qu'il traitoit les brigands qui lui rendoient hommage ou faisoient restitution, lorsqu'il l'exigeoit d'eux, tandis qu'il\setcounter{page}{117} livroit rigoureusement à la justice tous ceux qui bravoient ses avertissemens ou ses ordres. D'un autre côté, si des agens civils ou militaires du Gouvernement, poursuivoient des maraudeurs sur ses domaines, sans-demander sa concurrence, ils pouvoient être assurés d'essuyer quelque défaite signalée. Alors, il étoit le premier à se lamenter auprès des officiers de justice, et, tout en blâmant doucement leur témérité, il déploroit l'état d'anarchie auquel le pays étoit en proie.
Cependant ses doléances ne le mettoient pas à l'abri du soupçon, et le Gouvernement avoit fini par lui ôter son commandement militaire. Dès lors, quoiqu'il eût l'art de cacher son ressentiment, tout le pays avoit senti les effets de sa disgrace. Donald Bean et d'autres brigands de cette espèce avoient infesté la plaine, et les nobles, qu'on avoit désarmés, en qualité de Jacobites, se mettoient à l'abri des déprédations des montagnards, en payant tribut à Fergus, ce qui fournissoit aux frais énormes de son hospitalité féodale.
Cependant tout n'étoit pas ambition dans ce chef. Dès son enfance, il s'étoit dévoué de toute son ame à la cause de ses Rois exilés, et le desir ardent de les servir
H 3\setcounter{page}{118} loit à ses yeux celui de son propre agranissement. Pour eux, il disciplinoit son clan; pour eux, il entretenoit partout des intelligences; pour eux, il s'étoit réconcilié avec le baron de Bradwardine, qu'il savoit être de leurs partisans secrets. Ce zèle étoit récompensé de la part des princes par beaucoup de belles paroles, de temps à autre par quelques subsides de louis d'or, et sur-tout par une patente de Comte (un grand parchemin revêtu d'un énorme sceau de cire) accordée par un personnage qui n'étoit rien moins que Jaques III, Roi d'Angleterre, et huitième Roi d'Ecosse, à son féal, loyal et bien-aimé Fergus MacYvor de Glennaquoich, Comté de Perth, dans son royaume d'Ecosse.
Séduit par l'éclat de cette couronne de Comte, Fergus se plongea profondément dans les intrigues et les complots de cette période malheureuse, et il réconcilia sa conscience pour l'intérêt de son parti, à bien des manœuvres que son honneur ou son orgueil auroit sans cela réprouvées.
Les environs de Glennaquoich étoient agrestes et incultes; rien n'y sembloit accordé au luxe et à l'ornement. Il y avoit à peine des enclos, et malgré les cris et les courses de quelques bergers, le bétail et les\setcounter{page}{119} ponies sauvages pénétroient de partout dans un petit nombre de champs labourés. Mais l'objet des soins du possesseur, sa gloire et son espérance, c'étoit une centaine d'hommes complétement armés et équipés, qui paradoient devant le château, et exécutoient avec précision, au son de la cornemuse, des marches, des combats simulés et toutes sortes d'évolutions militaires.
"J'avois oublié," dit Fergus à Waverley, d'un air de négligence, "que je devois passer en revue ce petit nombre d'hommes de mon clan. Je n'en tiens sous les armes que ce qu'il en faut pour défendre ma propriété et celles de mes amis, de visites semblables à celle qu'a reçue dernièrement le baron de Bradwardine; et puisque le Gouvernement nous a enlevé tout autre moyen de défense, il faut bien qu'il consente à ce que nous nous protégions nous-mêmes." — "Mais avec des forces pareilles, vous auriez bientôt soumis toute la bande de Donald et bien d'autres." — "Sans doute, et ma récompense seroit un ordre de délivrer au commandant de Stirling le peu de largesse s é pées qu'ils nous ont laissées. Il n'y auroit pas à cela une trop bonne politique, ce me semble... mais venez, Capitaine.\setcounter{page}{120} , taine, le son des cornemuses nous invite ,, à dîner, que j'aie l'honneur de vous montrer ma sauvage demeure.,, La grande salle ou vestibule dans laquelle le dîner étoit servi, occupoit tout le rez-de-chaussée du château, bâti pas Jean le grand, et une immense table de bois de chêne en tenoit toute la longueur. Les apprêts du festin étoient simples jusqu'à la rudesse, et les convives nombreux jusqu'à en devenir incommodes. Au haut de la table siégeait le chef, ayant auprès de lui Edouard et deux ou trois étrangers des clans voisins. Venoient ensuite les principaux de son propre clan, qui tenoient de lui des terres en fief, suivis de leurs fils, de leurs neveux, et de leurs frères de lait; puis les officiers de la maison de Fergus, d'après leurs différens grades, et enfin les métayers et laboureurs qui cultivoient la terre de leurs propres mains. Au-delà même de ces longues files, on voyoit sur la prairie, à travers une grande porte ouverte à deux battans, une multitude de montagnards d'une condition encore inférieure, lesquels étoient néanmoins envisagés comme des hôtes, et participoient à la protection du chef et aux plaisirs de la journée. Enfin sur les derniers\setcounter{page}{121} confins de cette perspective, on apercevoit des groupes errans de femmes, de jeunes garçons, d'enfants vêtus de lambeaux, des mendians, et jusqu'à des chiens de différente espèce, qui tous jouoient un rôle plus actif dans l'action principale de la pièce. Cette hospitalité, en apparence sans bornes, avoit pourtant son genre d'économie. Quelques soins avoient été pris pour apprêter les plats de poisson et de gibier qui garnissoient le haut de la table, où siégeaient le chef et les étrangers; plus bas fumoient d'énormes pièces de bœuf et de mouton, et quoiqu'il n'y eût pas de viande de porc, qui est en abomination dans les montagnes d'Ecosse, on se rappeloit à cet aspect les festins grossiers des prétendans de Pénélope: au-dessous étoient servis en abondance des mêts plus communs encore, et cette inégalité ne choquoit personne. Chacun comprenoit que son goût devoit se régler sur la place qu'il occupoit à table, et prétendoit choisir de préférence la part que lui assignoit l'économie. Ainsi l'on voyoit des soupes, des oignons, des fromages, placés vis-à-vis des laboureurs, et ces mêts, joints aux reliefs du festin, régaloient aussi les nombreux enfants d'Yvor, qui mangeoient en plein air.\setcounter{page}{122} Mais le plat qui figuroit au centre de la table, le triomphe et l'orgueil du cuisinier c'étoit un agneau d'un an, rôti tout entier et dressé sur ses quatre-jambes, avec un bouquet de marjolaine dans la bouche. Le pauvre animal, vigoureusement assailli par tout le clan, avec des couteaux de chasse, des sabres, des poignards, n'offrit bientôt qu'un misérable spectacle. Trois joueurs de cornemuse faisoient sans interruption, retentir de leurs airs guerriers les voûtes de la salle, et ces sons extraordinaires, joints aux rudes accens galliques, étourdissoient Edouard au point qu'il en perdoit la tête.
Fergus porte la santé de Waverley; puis tout-à-coup, imposant silence aux cornemuses, il s'écrie ! "Où la chanson s'est-elle cachée, mes amis, que Mac Murrough ne peut pas la trouver." A ces mots, un vieillard, le Barde de la famille, se lève et commence à chanter d'une voix basse et rapide une abondance de vers celtiques qu'on écoute avec avidité. Ses regards sont d'abord baissés vers la terre, sa figure est presque immobile. Mais s'animant à mesure qu'il avance, ses yeux se lèvent sur ceux qui l'entourent pour demander et bientôt pour commander l'attention, ses accens graduellement renfor-\setcounter{page}{123} cès deviennent sauvages et terribles, des gestes passionnés les accompagnent. Il sembloit à Waverley, qu'il pleuroit les morts, apostrophoit les absens, exhortoit, supplioit, sollicitoit ceux qui l'entendoient. L'ardeur du poète se communique bientôt à l'auditoire. Les phisionomies agrestes et rembrunies des fils d'Yvor prennent une expression plus fière et plus animée, un feu nouveau étincelle dans leurs regards, tous se penchent vers le poète, plusieurs se lèvent saisis d'un transport subit, plusieurs portent la main sur la garde de leurs épées.....
Le Barde se tait, et après un silence imposé par l'émotion générale, Fergus prend une petite coupe d'argent placée devant lui, et l'ayant remplie de vin, il l'envoie à Mac-Murough, l'invitant, à boire ce jus et à garder la gourde qui le contient, en mémoire de Vich Ian Vohr. Le Barde, pénétré de la plus profonde gratitude, boit le vin, baise la coupe avec respect, et l'ayant soigneusement cachée dans le manteau qui couvroit son sein, il donna l'essor à sa reconnoissance dans une chanson vive et joyeuse.
Le chef, pendant cette scène, avoit plutôt observé que partagé l'enthousiasme de son clan. Il porta quelques toast guerriers et\setcounter{page}{124} nationaux, et quand Waverley lui demanda le sujet de la première chanson. — "Je vois," dit-il, que vous avez trois fois passé la coupe," et j’allois vous proposer de nous rendre auprès de la table de thé de ma sœur, qui vous" expliquera ces vers mieux que je ne puis le" faire. Quoique je ne veuille pas restreindre" la joie de mon clan, je prends peu de part" moi-même à leurs plaisirs actuels, et," ajouta-t-il, en souriant. " Je ne tiens point" d’Ours en réserve pour dévorer la raison de" ceux qui en peuvent faire un bon usage." Edouard ayant accepté de grand cœur cette offre, Fergus, après avoir fait une courte apologie à ses hôtes, quitta la table avec lui; et à peine eurent-ils franchi la porte, qu’ils entendirent éclater les bruyantes acclamations des convives, et l’énergique expression de leur dévouement pour leur chef.
Le salon qu’occupoit Flora Mac Yvor étoit meublé de la manière la plus simple, parce qu’à Glennaquoich on retranchoit les dépenses superflues, pour maintenir dans toute sa dignité l’hospitalité du maître et multiplier le nombre de ses adhèrens. Cependant aucune trace de cette économie ne s’apercevoit sur la personne de Flora. Ses habits élégans et même riches, offroient une alliage\setcounter{page}{125} heureux des modes françaises et du costume écossais, et ses cheveux, dont les boucles d'un noir de jay retomboient sur son cou, étoient retenus an-dessus de son front par un beau cercle de diamans.
Flora ressembloit singulièrement à son frère; même profil antique et correctement dessiné; mêmes yeux noirs, mêmes paupières, même teint clair et animé, seulement avec un degré de plus de blancheur et de transparence. Mais la régularité fière et presque sévère des traits de Fergus étoit dans Flora admirablement adoucie; sa voix qui rappeloit celle de son frère, avoit aussi un timbre plus pur et plus argenté, et tandis que l'œil vif et impatient de Fergus sembloit avide de gloire, de pouvoir, d'élévation, celui de Flora avoit une expression touchante et pensive, et ne s'enflammoit que pour la supériorité de l'âme. Elevée dans un couvent du premier ordre, aux frais et sous l'inspection de l'épouse du Prétendant, elle avoit ensuite passé deux années dans la famille de cette Princesse qui avoit admis Fergus au nombre de ses pages. Dès l'âge le plus tendre, une reconnaissance passionnée, un attachement religieux pour la famille exilée de ses rois, étoient les sentimens qui avoient animé le cœur de Flora.\setcounter{page}{126} A ses yeux, le devoir de son frère, du clan Yvor, de chaque honnête Anglais étoit de contribuer à tout prix, à travers tous les dangers, au rétablissement de cette maison illustre. Pour cette cause, elle étoit prête à tout entreprendre, à tout souffrir, à tout sacrifier. Son dévouement, qui surpassoit celui de son frère en fanatisme, étoit aussi bien supérieur en pureté, et la flamme de l'amour pour ses rois, brûloit sans mélange d'intérêt personnel dans son âme.
Ce qui remplissoit les loisirs de Flora, dans la retraite sauvage de Glennaqnoich, c'étoit la culture des arts et des lettres. L'étude des chefs-d'oeuvres français, anglais, italiens occupoit son esprit, et son talent ainsi que son enthousiasme pour la poésie, trouvoit un aliment continuel dans une nature pittoresque, et jusques dans les chants des Bardes Celtiques dont elle aimoit à reproduire les beautés.
Cet attachement héréditaire des chefs pour leur clan, qui prenoit chez Fergus une teinte d'ambition et de politique, remplissoit l'âme de sa sœur d'une affection pure et désintéressée. Son temps, ses soins, la pension particulière qu'elle tenoit de la princesse Sobiesky, étoient consacrés au soulagement des pauvres et des infirmes; aussi étoit-elle universellement\setcounter{page}{127} adorée. Les Bardes ne cessoient de l'exalter dans leurs chants, et le dernier qu'avoit composé Mac Murrough se terminoit par ces mots, que le plus beau fruit du pays pendoit à la plus haute branche.
Flora produit une très-vive impression sur Waverley. Il trouve en elle tout ce qui lui avoit manqué dans Rose, des manières très-imposantes, une sorte de supériorité douce et mélancolique, des talens brillans dans plus d'un genre, et cet enthousiasme contenu qui agit si fortement sur une imagination exaltée. Son enchantement fut à son comble quand, vers le soir, Flora accompagnée de deux jeunes personnes de son clan, se fit porter par l'une d'elles une petite harpe écossaise, au pied d'une cascade, dans un site extrêmement sauvage et romantique. Là, assise sur un banc de mousse, elle lui chante d'une voix harmonieuse sa traduction anglaise de la chanson que le Barde avoit entonnée à table, "car dit-elle, pour parler le langage poétique de mon pays, "le séjour de la Muse celtique est sur le rocher solitaire, et sa voix doit se confondre avec le bruit du torrent des montagnes." Le soleil couchant frappoit tous les objets d'une teinte pourprée. Les yeux noirs et le teint animé de Flora\setcounter{page}{128} brilloient d'un éclat presque surnaturel, et une expression pathétique ajoutoit encore à l'effet ravissant de sa figure. Aux premiers accens de sa voix, l'émotion de Waverley fut vive, au point d'en être pénible, mais bientôt l'espèce d'hymne qu'elle chante le captive et l'entraîne. C'est un appel extrêmement brillant et martial aux clans les plus illustres de la haute Ecosse. Le poète leur retrace leurs anciens exploits, il les invite à courir aux armes, à secouer le joug de l'étranger, et le génie de Walter Scott se déploie là dans son véritable genre.
Waverley, toujours plus épris de Flora, prolonge son séjour à Glennaquoich. Il assiste à un rendez-vous de chasse de plusieurs clans, lequel, sans qu'il le sache, n'est que le prétexte d'un rassemblement politique. La chasse est admirablement décrite. Au moment où tous les cerfs réunis dans une étroite enceinte, se font jour à travers les rangs des Montagnards, Edouard, qui ne comprend pas l'ordre donné en gallique aux chasseurs de se prosterner, est blessé, quoique Fergus expose sa vie pour le préserver. Ce chef, après avoir pourvu à ce qu'exige la situation d'Edouard, le quitte, et étant revenu au bout de quelques jours, il le trouve\setcounter{page}{129} trouve à-peu-près guéri, et reprend avec lui la route de Glennaquoïch. En arrivant, il annonce en gallique à sa sœur des nouvelles ( apparemment le débarquement du jeune Charles Stuart sur la côte d'Ecosse ) qui lui donnent la plus vive émotion. Edouard en apprend à son tour de fort importantes pour lui. Plusieurs lettres l'attendoient à Glennaquoich. Les unes, de sa famille, lui apprennent que son père a été disgracié par le Ministre, et lui enjoignent de quitter le service d'un Gouvernement aussi injuste. Une autre fulminante de son Colonel, lequel paroissoit lui avoir déjà écrit, et lui ordonnoit de rejoindre son corps dans trois jours, sous peine de destitution. Ce terme étant expiré, Fergus lui montre dans un papier public un article très-injurieux pour tous les Waverley et pour lui en particulier, annonçant sa destitution "pour cause d'absence sans congé" et la nomination d'un autre officier à sa place. On l'avoit donc condamné sans l'entendre; sans même prendre des informations; il est indigné. Fergus partage avec vivacité ses sentimens, mais il lui prouve l'impossibilité de se venger par un duel et l'exhorte à joindre avec lui les drapeaux des Stuarts. Edouard, sans se décider à cet égard, offre sa main à\setcounter{page}{130} Flora, proposition qui cause une grande joie à Fergus. Cependant, Flora demande du temps et finit par répondre que Mr. Waverley a certainement écarté un obstacle insurmontable à cette union en déposant la cocarde hanovrienne, mais que son cœur à elle, quoique libre de tout engagement, est beaucoup trop rempli d'un autre enthousiasme que celui de l'amour, pour qu'elle pût faire le bonheur d'un jeune homme exalté, qui auroit le besoin et le droit d'être l'objet de toutes ses pensées. D'ailleurs, elle dissuade Edouard de suivre l'avis de son frère, lui prouvant que sa réputation d'officier resteroit entachée s'il changeoit si promptement de parti; elle lui conseille d'aller à Londres, de se justifier, et après avoir rompu tous ses liens avec le Gouvernement, de suivre la route que lui indiquera sa conscience. Cette réponse douteuse jette Edouard dans une grande perplexité. Après une nuit agitée, il reçoit un exprès de Mlle. de Bradwardine. Cette jeune personne lui écrit dans un style fort naïf, qu'il doit se garder de paroître à Tully-Veolan, qu'un parti de soldats est venu l'y chercher et qu'il courroit de grands dangers s'il étoit arrêté. Elle ajoute que, sur le bruit d'une insurrection\setcounter{page}{131} dans les montagnes, on persécute les Jacobites, et que son père a pris la route du nord avec un corps de quarante cavaliers. Malgré ces motifs de crainte, Waverley se décide à suivre l'avis de Flora et il part pour l'Angleterre. Fergus, bien que très-occupé à armer le clan Yvor, l'accompagne pendant une partie du chemin, enflammant sa tête et son cœur pour les Stuarts et pour Flora. "Vous allez tomber dans de mauvaises mains lui dit-il, et vous verrez qu'il me faudra aller assiéger, pour vous délivrer, le château d'Edimbourg ou de Stirling." Arrivés au défilé de Bally Brough, les deux amis se séparent, et une scène nouvelle s'ouvre bientôt pour chacun d'eux.
(La suite au Cahier prochain.)
\setcounter{page}{359} deux ou trois jours après, étant assez bien pour recevoir compagnie, il entreprit un voyage à sa maison de campagne pour changer d'air. Le voilà donc revenu parfaitement rétabli.
\section{ROMANS \large{WAVERLEY, ou IL Y A SOIXANTE ANS. (Edinburgh 1814) 3 vol. in-12°} \small{( Quatrième et dernier extrait. Voy. p. 216 de ce vol. )}
PENDANT le combat de Preston, Waverley avait dérobé, à la fureur des Montagnards, un officier anglais d'un rang élevé, que le Prince confia ensuite à sa garde. Ce prisonnier se trouva être un colonel Talbot, un homme d'un mérite distingué, qui avait eu autrefois de grandes obligations à Sir Everard Waverley, et qui crut devoir désabuser son neveu de bien des espérances illusoires. Il avait déjà diminué l'ascendant de Fergus sur l'esprit d'Édouard, lorsque ce dernier obtint du Prince pour le Colonel la permission de retourner à Londres, où les intérêts les plus pressants le rappelaient. Cependant,\setcounter{page}{360} les fêtes ont recommencé à Holy-Rood. Flora, qui a découvert le penchant décidé de Rose de Bradwardine pour Waverley, fait en secret à son amie le sacrifice du commencement de goût qu'il lui avoit inspiré et le rebute par une froideur constante. De pareils soins ne tardent pas à réussir. Edouard éprouve bientôt du dépit de ce que Fergus fait la cour à Rose, et les défauts de son ami le frappent chaque jour davantage. Il se passe diverses scènes, dans lesquelles le caractère hautain et irascible de ce chef, les jalousies sans nombre que fait naître l'espoir d'une restauration, et l'adresse conciliatrice du Prétendant se développent d'une manière heureuse et frappante. Au commencement de novembre, Charles Édouard, à la tête de six mille hommes au plus, se décide à jouer le tout pour le tout, et cherche à pénétrer au centre de l'Angleterre, quoiqu'il n'ignorât pas que de formidables préparatifs étoient faits pour l'y recevoir. Le succès accompagne ses premiers efforts, il assiège et prend Carlisle, et poursuit vers le sud sa marche audacieuse. Fergus et Waverley envisagent d'un œil différent les progrès de l'armée. "Le premier, un composé d'air et de feu, défiant le monde entier sous les armes, ne voyoit dans\setcounter{page}{361} dans chaque pas en avant, que la diminution de la distance de Londres. Il ne sollicitoit, n'attendoit, ne desiroit d'autre secours, que celui des clans pour replacer les Stuarts sur le trône, et ne considéroit"chaque nouvel adhérent, que comme un compétiteur de plus aux faveurs du futur monarque. Edouard remarquoit, au contraire, que dans les villes où le Prince faisoit proclamer Jaques III, personne ne crioit VIVAT, et que la populace ébahie ouvroit des yeux étonnés, sans même donner des signes de cet esprit turbulent et tapageur, qu'elle se plaît à manifester pour la moindre cause. Le soulèvement des riches Jacobites, sur lequel on avoit compté, n'étoit point général, et si le premier aspect des Montagnards inspiroit l'effroi, leur pauvreté et leur petit nombre sembloit un augure certain du mauvais succès de leur entreprise."
Pendant la marche, Waverley s'explique avec Fergus au sujet de Flora. Il lui avoue que l'indifférence, plus encore que les refus de cette jeune personne, l'ont forcé à renoncer à elle. Cette déclaration paroît à Fergus un affront, dont son orgueil l'empêche seul de tirer vengeance; il laisse cependant percer\setcounter{page}{362}  un tel ressentiment, que Waverley piqué au vif quitte son clan et va s'engager sous les drapeaux du Baron. Le bruit se répand parmi les Mac-Yvor, que Waverley a abandonné leur bien-aimée Miss Flora, et ils prennent ce jeune homme dans une haine terrible. Bientôt Édouard reçoit de son ancien ami, Evan Dhu, l'avis secret de se tenir sur ses gardes, et un Écossais qui le sert lui confirme l'idée de son danger, "car", lui dit cet homme, "le Prince lui-même ne seroit pas en sureté si ces gens-là croyoient qu'il a blessé leur chef dans son honneur." Au moment même une balle siffle aux oreilles d'Édouard, et il met son cheval au galop pour aller droit à Fergus. — "Colonel Mac-Yvor", lui dit-il, "j'ai à vous apprendre qu'un de vos gens s'est mis en embuscade et a tiré sur moi." — "Comme c'est un plaisir que je compte me donner à moi-même, à la réserve pourtant de l'embuscade ", repartit Fergus," je voudrois savoir quel est l'homme de mon clan qui a osé me prévenir?" "Je serai à vos ordres quand il vous plaira, en attendant, le brave homme qui vous a remplacé, c'est votre page que voilà: c'est Callum Beg." "Sortez des rangs, Callum. Avez-vous fait feu sur Mr. Waverley?,"\setcounter{page}{363} "Non," répond sans se troubler Callum.
"Oui," dit le domestique Ecossais, "je vous ai vu de mes yeux comme je vois ce clocher qui est là."--"C'est faux," reprend Callum, et un duel entre les pages alloit précéder celui des maîtres, quand Fergus, du ton décidé qui lui est ordinaire, demande à Callum son pistolet... le chien en étoit abaissé, le bassinet et la bouche étoient noirs de fumée, il venoit d'être tiré.
"Voilà," dit Fergus en frappant de toute sa force sur la tête de ce jeune homme avec le pommeau du pistolet, "voilà pour avoir agi sans ordres et menti afin de le cacher."--Callum reçoit le coup sans faire le moindre mouvement pour l'éviter, et tombe privé de connoissance.--"Tenez-vous tranquilles, au nom du ciel," dit Fergus au reste du clan, "je brûle la cervelle du premier qui se mêle de ma querelle avec Waverley."--Aucun ne bouge, et Callum étendu sur la terre nageoit dans son sang, mais personne n'ose le secourir.
"A vous à présent, Mr. Waverley, ayez la bonté de tourner votre cheval pour aller à vingt pas d'ici sur la commune." Tout en s'éloignant ensemble, ils se disent des mots très-aigres, dans lesquels se trouvent mêlés les noms de Flora et de Rose, et à peine des\setcounter{page}{364} cendus de cheval, ils avoient déjà tiré l'épée lorsqu'ils furent séparés par le Baron. Bientôt après survint le Prince, qui parvint à les calmer, en intéressant à leur réconciliation, du moins apparente, la réputation des deux femmes qu'ils respectoient le plus.
Cependant la difficulté de pénétrer plus avant en Angleterre s'accroissoit chaque jour, et après un conseil de guerre, tenu à Derby le cinq décemb., les Montagnards, au grand regret du jeune Prince dont ils suivoient les étendarts, renoncèrent à cette résolution désespérée. Une fois la retraite résolue, on l'exécuta avec tant de promptitude, que l'armée écossaise échappa à la poursuite du Duc de Cumberland, qui marchoit sur ses pas avec une puissante cavalerie.
Ainsi furent abandonnées de gigantesques espérances. Nul n'avoit été plus ardent que Fergus; nul n'éprouva une mortification plus cruelle. Il s'étoit déclaré avec emportement contre la retraite dans le conseil de guerre, et quand son opinion fut rejetée, il versa des larmes de douleur et d'indignation. Depuis ce moment il ne fut plus le même. On l'auroit à peine reconnu pour cet esprit bouillant, audacieux, qui sembloit se trouver à l'étroit sur la terre. Déjà l'armée avoit poursuivi depuis plusieurs jours sa marche rétrogade.\setcounter{page}{365} grade, quand Waverley reçut la visite de ce chef, dans une chaumière où il s’était arrêté.
Comme depuis leur rupture, ces deux officiers ne s’étaient pas adressé la parole, Edouard attendoit avec quelque inquiété une explication, lorsque jetant les yeux sur son ancien ami, il fut frappé du changement qui s’était opéré en lui. Ses yeux étoient éteints, ses joues creuses, sa voix languissante, son pas moins ferme et moins relevé. Ses habits mêmes, jusqu’alors arrangés avec toute l’exactitude militaire sembloient jetés négligemment sur sa personne. Il invite Edouard à le suivre, et voyant celui-ci ceindre avec fierté son épée, il se prend à sourire d’un air mélancolique.—"J’ai eu tort," lui dit-il, aussitôt qu’ils furent seuls,—"Flora, à qui j’ai écrit au sujet de notre querelle, me répond qu’elle n’a jamais eu le dessein de vous donner des encouragemens... je me suis donc conduit comme un insensé."... Et, reprenant le ton d’une amitié qui n’eût point souffert d’interruption, il dit qu’il est le plus malheureux de tous les êtres, que sa sœur va devenir aussi à plaindre que lui. Il conjure Waverley d’épouser Rose de Bradwardine, et d’emmener la pauvre Flora avec eux en Angleterre, où leurs amis négocieront facilement leur pardon. Aux scrupules que lui oppose Edouard, il répond\setcounter{page}{366} "qu'il voit parfaitement clair dans l'avenir, que le vaisseau fait eau de toutes parts, et qu'il est plus que temps d'entrer dans la chaloupe, pour ceux qui peuvent le faire. —"Nos chefs se flattent," dit-il, "que les exécutions tomberont sur les riches de la plaine, et que leurs montagnes et leur pauvreté feront comme à l'ordinaire leur salut. Mais cette fois ils se trompent. John Bull a été trop sérieusement effrayé, et les Ministres Hanovriens méritent d'être pendus comme des imbécilles, si à présent qu'ils sont les plus forts, ils laissent subsister un seul clan en Ecosse qui puisse leur porter ombrage."
A ces vues si nettes sur les évènemens, Fergus, par une de ces foiblesses qui ne sont pas hors de la nature, associe dans son imagination frappée, le retour des superstitions de son pays et de son enfance. Il a vu le Bodach Glas, le spectre qui depuis un meurtre commis par Jean le grand, apparaît, à tous les descendans de ce guerrier, la veille des grandes calamités. "Avant que le jour de demain soit arrivé," dit-il, "je serai mort ou prisonnier."
Waverley obtient du Baron la permission de marcher pendant cette journée avec Fergus, dans l'espoir de calmer les esprits agités de ce chef. Les Mac-Yvor le voyant réa\setcounter{page}{367} concilié avec lui oublient leur ressentiment; et Callum, qui est rétabli, et plus dévoué que jamais à son maître, est le premier à l’accueillir. La retraite continue à s’exécuter sans obstacles; on croyoit avoir une journée d’avance sur l’ennemi. Après avoir dépassé un grand étang, les Mac-Yvor et un autre clan alloient entrer dans l’enceinte palissadée du village de Clifton, quand Waverley s’apercevant que le soleil étoit couché, commence à plaisanter Fergus sur les prédictions du Bodach Glas... "Les Ides de mars ne sont pas passées," répond le chef avec un sourire, "... et regardant au-delà de l’étang, un puissant corps de cavalerie lui paroît comme voltigeant au-dessus de la surface noire.
Les Montagnards se fortifient derrière les palissades, où ils sont bientôt attaqués. Les deux amis font des prodiges de valeur, et Fergus, voyant l’ennemi repoussé, reprend une impétuosité naturelle et entraîne tout son clan à la poursuite des dragons démontés. Cependant un rayon fugitif de lune découvre aux Anglais le petit nombre de leurs vainqueurs; honteux, ils se rallient, se retournent, reprennent l’offensive, et les Montagnards cherchent de nouveau à gagner leurs retranchemens. Waverley, qui s’étoit séparé de Fergus, l’aperçoit de loin qui combat en\setcounter{page}{368} désespéré avec Callum et Evan Dhu contre un corps de cavaliers, mais bientôt il le perd de vue. Lui-même s'égare dans la nuit, le son des cornemuses a cessé de le guider. Accablé de fatigue et d'inquiétudes, il arrive au bout de plusieurs heures, dans une chaumière. Là, il est long-temps enfermé par les neiges, et quand les communications sont rétablies, il apprend que le Duc de Cumberland, à la tête de l'armée anglaise, a repris Carlisle, et que le Prétendant a déjà gagné le nord de l'Écosse. Il ne peut pas songer à rejoindre son corps. Au milieu des tristes pensées qui l'accablent, il lui tombe entre les mains un papier public. Il y voit que Richard Waverley est mort de chagrin de la conduite d'un fils qu'il avoit, et que Sir Everard va être mis en jugement pour crime de haute trahison, à moins que le fils du défunt et l'héritier du Baronet, Edouard Waverley, actuellement au service du Prétendant, ne vienne de lui-même se livrer entre les mains de la justice.
Ces nouvelles plongent Edouard dans une douleur qui va jusqu'au délire. Ses sentiments et ses devoirs les plus sacrés l'appellent à Londres; il n'hésite pas à s'y rendre. Comme son signalement est donné partout sur la route, il n'échappe que par miracle.\setcounter{page}{369} à mille dangers, que nous ne nous arrêterons pas à détailler. Arrivé chez le colonel Talbot, où il demeure caché, il apprend que son père est mort en effet, mais non de chagrin, et que son oncle n'a jamais couru aucun danger. Lui-même a été justifié d'une partie des faits à sa charge, parce que Donald Bean, qui a été saisi et exécuté, a avoué, avant de mourir, que c'étoit lui, qui, sous le nom de Ruffin, avoit débauché ses soldats. Cet homme avoit ajouté, qu'encouragé par une récompense que lui avoit donnée une personne qu'il ne nommoit pas, il avoit ensuite délivré Edouard des mains de ceux qui l'emmenoient prisonnier. Waverley, disculpé comme officier, obtient son pardon comme rebelle par le crédit du colonel Talbot, et ayant donné sa parole de ne pas reprendre les armes, il retourne secrétement en Ecosse.
Le combat de Falkirk avoit jeté un dernier éclat sur les armes du Prétendant, mais la défaite de Culloden ruine à jamais ses espérances. Edouard apprend sur la frontière que la tête de Charles Edouard est mise à prix, que tous ses adhérens sont en fuite ou se cachent misérablement. Pénétré d'horreur il poursuit sa route jusqu'à Edimbourg; là on lui dit que Fergus a été fait\setcounter{page}{370} prisonnier, la nuit même où ils avoient combattu ensemble à Clifton, et que cet infortuné a été conduit à Carlisle où il doit être jugé avec les officiers de la garnison. Quant au Baron on le suppose en fuite. Waverley inquiet pour Rose, dont l'idée domine actuellement dans son cœur, se rend à Tully Veolan. Cette demeure vénérable est pleine de soldats. Les biens du Baron ont été confisqués, tout est saccagé, ruiné, les ours renversés jonchent la terre. Davie Gelattie chante encore, mais des airs lugubres. Pourtant à travers sa folie, il reconnoît dans Waverley un ami, et lorsque le soir est venu, il le conduit le long d'un torrent sauvage jusques dans la chaumière écartée où sa vieille mère cachoit toutes les nuits le Baron. La joie et la surprise est grande de part et d'autre, mais ce qui étonne surtout Edouard, c'est de reconnoître le lendemain, dans cette même chaumière, celle où il a été malade, et dans la mère de Davie Gelattie, cette bonne Janette qui le soignoit. Une foule de circonstances et les informations qu'il prend ne lui laissent aucun doute que ce ne soit Rose de Bradwardine qui a employé Donald Bean pour le délivrer et qu'il a entrevue à travers les fentes de sa prison. Décidé à lui consacrer sa vie il va\setcounter{page}{371} chercher le Baron dans une caverne sauvage où ce pauvre homme passoit les journées; entretenu par la tendresse compatissante de ses anciens vassaux. Ce brave officier, donc la grande figure étoit revêtue du costume le plus grotesque, gardoit l'entrée de la caverne; une large épée nue à la main. Edouard lui demande avec timidité sa fille en mariage. Pendant qu'il parle, le vieux noble cherche à rassembler toute sa dignité pour recevoir convenablement l'offre d'une alliance entre les maisons de Bradwardine et de Waverley, mais ses efforts sont inutiles; le père l'emporte sur le Baron, il ne peut retenir ses larmes et jettant ses bras autour du cou d'Edouard, "mon fils, mon fils," s'écrie-t-il, "si j'avois parcouru toute la terre c'est toi que j'aurois choisi....."
Edouard, tourmenté d'inquiétudes sur le sort de Fergus, se rend auprès du colonel Talbot, qui étoit alors à Edimbourg, pour chercher à l'intéresser en faveur de ce malheureux; mais il le trouve inexorable. — "Toutes mes sollicitations seroient infructueuses," lui dit le colonel, "et d'ailleurs" je me ferois scrupule d'employer pour lui" mon influence. La justice exige une expiation. Ils ont rempli toute la nation d'alarme et de douleur, et comme victime,\setcounter{page}{372} nul ne pouvoit être mieux choisi que ce chef infortuné. C'est avec une pleine connaissance de la nature de son entreprise qu'il a armé son clan. Le sort de son père ne l'a pas intimidé, la clémence du gouvernement, qui lui a rendu ses biens, ne l'a pas adouci. Qu'il soit brave, généreux, qu'il possède mille belles qualités, c'est ce qui le rend plus à craindre; qu'il soit rempli de lumières et de talens, c'est ce qui ôte toute excuse à son crime; qu'il ait été un fanatique dans une mauvaise cause, c'est ce qui le rend d'autant plus propre à en devenir le martyr...... Il savoit parfaitement ce dont il s'agissoit, il a joué la vie contre la mort, une couronne de Comte contre un cercueil; ... qu'il prenne sa chance, telle que le destin l'a amenée......"
Waverley se rend à Carlisle, non dans le moindre espoir de sauver Fergus, mais pour le revoir encore.
La commission d'Oyer et Terminer siégeoit encore, quand Waverley, que son extrême agitation fait prendre pour un parent des prévenus, perce une foule immense, et parvient au pied du tribunal. Le Verdict de GUILTY venoit d'être prononcé, et un silence imposant avoit succédé à ces paroles\setcounter{page}{373} terribles. Dans ce moment Édouard jette les yeux sur la barre. Il n'y avait pas de méconnaître la figure majestueuse et les nobles traits de Fergus, quoique ses habits fussent en lambeaux et son visage couvert des teintes livides d'une longue captivité. A son côté était Evan Dhu. En les regardant, Édouard se sent atteint d'une espèce de vertige, mais il est rappelé à lui-même par la voix de l'officier public prononçant ces mots solennels. "Fergus Mac Yvor de Glennaquoich, autrement appelé Vich Ian Vohr, et Evan Mac Yvor, autrement appelé Evan Dhu Maccombich, vous et chacun de vous êtes convaincus de haute trahison. Mais afin que vous mourriez conformément à la loi, je vous demande ce que vous avez à alléguer pour détourner la sentence que la cour va prononcer contre vous."
Fergus, au moment où le juge président se couvroit pour prononcer la sentence, plaça son propre bonnet sur sa tête, et fixant sur ce Magistrat un regard assuré, répliqua d'une voix ferme : "Je ne veux pas laisser supposer à cette nombreuse assemblée que je n'aye rien à répondre à un semblable appel. Mais ce que j'ai à dire, vous ne consentiriez pas à l'entendre,\setcounter{page}{374} car ma défense seroit votre condamnation. Hier et le jour qui l'a précédé, vous avez ordonné que le sang le plus loyal et le plus noble fût versé comme de l'eau — n'épargnez pas le mien. — Si celui de tous mes ancêtres eût été dans mes veines, je l'aurois hasardé pour cette cause."
Evan, qui avoit toujours regardé son chef avec beaucoup d'attention, se levant alors, parut desirer de parler; mais le tumulte du tribunal et la difficulté de s'exprimer dans une langue presque étrangère, l'empêchoient de rompre le silence. A cet aspect, un murmure de compassion se fit entendre parmi les spectateurs, parce qu'on supposoit que ce pauvre homme vouloit alléguer pour son excuse l'autorité de son chef. Le juge commanda le silence et encouragea Evan à s'expliquer. "J'allois seulement dire," répondit Evan, d'une manière qu'il croyoit être insinuante, "que si vous, très-excellent seigneur, et l'honorable cour, consentiez à mettre Vich Ian Vohr en liberté, pour cette unique fois, afin qu'il retournât en France, où il ne troubleroit jamais le gouvernement du roi George, six des meilleurs de, son clan seroient tout prêts à subir le jugement à sa place; et si vous vouliez senlement me laisser aller à lennaquoich,\setcounter{page}{375} je vous les ramenerois moi-même; pour que vous puissiez les pendre ou les tuer, comme il vous plairoit, à commencer par moi tout le premier."
Malgré la solemnité de la circonstance, une sorte de rire, causé par la singularité de la proposition, fut entendu dans l'assemblée. Le juge ayant reprimé cette indécence, Evan regarda d'un air ferme autour de lui, et quand le murmure fut appaisé il reprit:
"Si les messieurs Saxons (Anglais) rient de ce qu'un pauvre homme tel que moi, pense que sa vie ou celle de six de ses pareils, vaille celle de Vich Ian Vohr, ils peuvent avoir bien raison; mais s'ils rient parce qu'ils pensent que je ne tiendrois pas ma parole, et que je ne reviendrois pas le racheter, ils ne connoissent ni le cœur d'un Montagnard, ni l'honneur d'un gentilhomme."
Personne 'n'eut plus envie de rire depuis ce moment.
Alors le juge prononça la sentence de mort attachée au crime de haute trahison, avec tous ses horribles accessoires. L'exécution en fut fixée au jour suivant. — "Pour vous Fergus Mac Ivor," continua-t-il," je n'espère point de grace. Vous pouvez, d'ici à demain, vous préparer pour votre\setcounter{page}{376} dernière souffrance dans ce monde et votre grande audience dans l’autre.
— "Je ne désire rien de plus, monsieur," répondit Fergus, du même ton ferme et décidé.
Le sombre regard qu’Evan avait jusque-là tenu fixé sur son maître fut alors adouci par une larme. — "Pour vous, pauvre ignorant," reprit le juge, "qui nous montrez d’une manière frappante, comment la fidélité, due seulement au roi et à l’état, peut, d’après vos malheureuses notions de clan, être transférée à un ambitieux qui vous rend l’instrument de ses crimes — pour vous, dis-je, je sens une telle compassion que si vous pouvez vous résoudre à demander grace.....
— "Grace pour moi, point de grace," interrompit Evan, "puisque vous devez verser le sang de Vich Ian Vohr," et continuant dans son langage vulgaire, il fit au juge une réponse insultante.
"Qu’on emmène les prisonniers," dit ce dernier, "que son sang retombe sur sa tête."
La séance est levée, et Waverley qui demande à voir Fergus apprend que personne ne peut être admis en sa présence, si ce n’est sa soeur et un prêtre. L’esprit public, lui dit-on, est en danger d’être perverti par les bruits qui se répandent sur les derniers\setcounter{page}{377} momens de ces rebelles. Cependant, il obtint à la fin de l’avocat du condamné, la promesse d’être introduit auprès de lui le lendemain avant qu’on détache ses fers pour l’exécution.
"Est-ce bien de Fergus Mac Yvor, qu’ils parlent," dit en lui-même Waverley? "De Fergus, le brave, le généreux, le preux chevalier, le chef puissant d’une tribu dévouée, l’amour des dames, le héros des chants? Lui que j’ai vu à la tête des chasseurs, le premier à l’assaut? Est-ce lui qui est enchaîné comme un mal-faiteur, qui sera traîné sur une charette à l’échaffaud public, pour mourir d’une mort lente et cruelle par les mains du dernier des misérables? Terrible en effet étoit le spectre qui a présagé un sort pareil au vaillant seigneur de Glennaquoich!"
D’une voix défaillante, Waverley demande à l’avocat d’annoncer sa visite à Fergus, et, de retour à son auberge, il écrit un billet à peine lisible à Flora Mac Yvor, pour la prier de le recevoir dans la soirée. Le messager lui rapporte une réponse d’une écriture belle et ferme, portant que, "miss Flora Mac Yvor ne pouvoit pas refuser de
\setcounter{page}{378} voir l'ami le plus cher de son bien-aimé frère, même dans ce moment d'une douleur inexprimable. "
Le soir, Waverley se rendit dans une maison catholique qui servoit d'asyle à miss Mac Yvor. Il fut à l'instant introduit dans un grand appartement meublé d'une tapisserie sombre, où Flora assise auprès d'une fenêtre grillée étoit occupée à coudre un vêtement de flanelle blanche. A quelque distance, une femme âgée en habit de religieuse, qui paroissoit une étrangère, lisoit à haute voix dans un livre de prières, mais elle quitta la chambre au moment, où Waverley y entra. Flora se leva pour le recevoir et lui tendit la main, mais elle n'osa pas même essayer de parler. Son beau coloris avoit tout à fait disparu; toute sa personne étoit prodigieusement amaigrie, et son visage et ses mains, aussi blancs que le marbre statuire le plus pur, faisoient un contraste sinistre avec son vêtement et ses cheveux noirs. Cependant rien n'étoit négligé ni en désordre dans son extérieur. Ses cheveux même quoique dépouillés de tout ornement, étoient arrangés avec son exactitude ordinaire. — Les premiers mots qu'elle prononça, furent: "L'avez-vous vu? "\setcounter{page}{379} "Hélas, non," répondit Waverley, "ils ont refusé de me laisser entrer." — "Celà s'accorde avec le reste," reprit-elle, "mais nous devons nous soumettre. Croyez-vous" d'en obtenir la permission?
" Oui, pour, pour...... le mot de destin, main ne put s'échapper des lèvres d'Édouard."
"Alors ou jamais, reprit Flora, ....." jusqu'au temps," ajouta-t-elle en regardant le ciel, "où je sais que nous nous retrouverons tous. ..... Mais j'espère que vous" le reverrez pendant que la terre le porte" encore. Il vous a toujours aimé au fond" de son cœur, malgré..... mais il est" inutile de parler du passé."
— "Inutile, il est vrai," répéta Waverley.
— Et même de l'avenir, mon excellent ami, du moins quant à ce qui concerne les événements terrestres. Combien de fois ne me suis-je pas représenté la possibilité de cette horrible issue, et n'ai-je pas pris à tâche d'envisager comment je me conduirois alors. ..... Et pourtant combien peu j'ai pu comprendre toute l'amertume de ce moment !
— Chère Flora, si votre force d'ame.....\setcounter{page}{380} - Ah, oui, c'est cela, répondit-elle avec une sorte d'égarement, ma force d'ame!..... Il y a, vous frémirez, il y a un démon acharné sur mon cœur, qui me dit tout bas, que c'est cette force d'ame dont Flora s'est glorifiée, qui a...... assassiné son frère !
- Grand Dien pouvez-vous?......
- Non, cela n'est pas, mais cependant il me poursuit comme un fantôme. Je sais que c'est une vision une imagination...... mais il s'obstine à rester. Il est là, il répand son horreur sur mon ame. Il me dit que mon frère aussi léger qu'ardent, auroit dissipé son énergie parmi une multitude d'objets; c'est moi qui lui ai appris à concentrer ses forces, à tout jouer sur ce coup unique et désespéré. Ah si je pouvois me souvenir que je lui eusse dit seulement une fois, "celui qui"frappe avec l'épée mourra par l'épée," que je lui eusse dit, restez chez vous, réservez vos talens, vos vassaux, votre vie, pour des entreprises à la portée de l'homme. Mais, ô mon digne ami, j'ai aiguillonné cet être fougneux, et la moitié au moins de son infortune est à la charge de sa sœur!......
Edouard chercha de mille manières à combattre cette horrible idée, et lorsqu'il\setcounter{page}{381} lui dit qu'elle avait obéi à ce qu'elle croyait son devoir; "Pensez-vous que je l'oublie ce devoir," s'écria-t-elle avec précipitation et en élevant son regard; "ô non, ce n'est point parce qu'elle était coupable que je regrette cette entreprise, mais parce qu'il était impossible que l'issue en fût différente."
A cela Waverley ne resta pas sans réponse, mais Flora ne l'entendait plus.
"Vous souvenez-vous," reprit-elle avec une sorte de phrénésie, "du jour où vous me trouvâtes arrangeant des rubans (les cocardes blanches qu'elle distribuait à tout son clan) que vous prites pour les rubans de noce de Fergus, aujourd'hui je cous son vêtement de mort! — Nos amis d'ici donneront dans leur chapelle un asyle sanctifié aux restes sanglants du dernier Vich Ian Vohr.
Mais ces restes, ils n'y reposeront pas tous. sa tête. sa tête.
Je ne presserai pas de mes lèvres les froides lèvres de mon cher Fergus."
Ici la malheureuse Flora fut atteinte de convulsions et s'évanouit. La religieuse entra, et Édouard, à qui elle dit de quitter la chambre, mais non la maison, ayant été rappelé, trouva au bout d'une\setcounter{page}{382} demi heure, Flora infiniment plus calme.
Il lui adressa de la part de Rose de Bradwardine l'instante prière de venir vivre avec elle et de la regarder comme sa sœur.
"J'ai reçu de ma chère Rose," répondit Flora, "une lettre dans ce même but. Ce que j'ai appris d'elle et de son père me donne le dernier rayon de joie qui puisse me parvenir. — Remettez ceci de ma part, Mr. Waverley, à cette amie bien aimée. C'est le seul ornement de prix que la pauvre Flora ait jamais possédé, et ce fut le don d'une princesse." Elle mit alors entre les mains d'Edouard un écrit contenant la chaîne de diamants dont elle avoit coutume d'orner ses cheveux. — "Pour moi," dit-elle, "je n'en ai plus que faire. La protection de mes amis m'a assuré une retraite à Paris dans un couvent des bénédictines Ecossaises, et demain, si je puis survivre à demain, je me mets en route avec cette vénérable sœur. — A présent donc Mr. Waverley, adieu. Puissiez-vous être aussi heureux avec Rose que le méritent vos aimables caractères à tous deux. — Pensez quelquefois aux amis que vous avez perdus, mais ne cherchez pas à me revoir, ce seroit de la bonté mal entendue."\setcounter{page}{383} À ces mots, elle lui tendit sa main, que Waverley inonda de larmes. De retour à son auberge, il apprend par une lettre de l'avocat de Fergus qu'il pourra le lendemain voir son ami depuis le moment où les portes du château s'ouvriront jusqu'au moment fatal où l'on viendra chercher les damnés.
Après une nuit agitée, l'aube du jour trouva Edouard sur l'esplanade, en face de l'entrée gothique de la forteresse; mais il la parcourut longtemps dans tous les sens avant l'heure où la règle de la garnison permit de baisser le pont et d'ouvrir les portes.
Il montra sa permission au sergent de la garde et on le laissa entrer. C'était dans un appartement voûté, au centre de la tour énorme qui forme l'antique château de Carlisle, que Fergus était retenu prisonnier. Le bruit des barres et des verrous qu'on tira pour introduire Waverley fut bientôt suivi d'un cliquetis de chaînes, lorsque le chef infortuné traîna ses fers pesans sur le pavé de pierre de sa prison, pour se jeter dans les bras de son ami.
"Mon cher Edouard," dit-il d'un ton ferme et presque gai, "voilà de la vraie bonté, j'ai appris votre bonheur futur avec\setcounter{page}{384} le plus vif plaisir. Et comment est Rose? Et comment va notre vieil ami, cet original de Baron? Bien, je le vois à votre air. Et comment réglerez vous la préséance entre les trois hermines passant et l'ours, et le tirebotte? > \footnote{Armoiries des Waverley et des Bradwardine. Un tire botte en sautoir avec une épée; avoit été ajouté par le Prince à l'écu du Baron, lors de la cérémonie,}
"Et comment, oh! comment, mon cher Fergus, pouvez-vous parler de ces choses-là dans un pareil moment.>
"Oui, certainement nous étions entrés dans Carlisle sous de plus heureux auspices, le 16 novembre dernier, par exemple, lorsque marchant à côté l'un de l'autre, nous plantâmes le pavillon blanc sur ces antiques tours. Mais je ne suis pas un enfant, pour me coucher par terre et pleurer, parce que le sort a tourné contre moi. Nous avons joué avec hardiesse, il faut payer avec fermeté. Et, puisque le temps me presse, venons à ce qui m'intéresse le plus. - Le Prince? A-t-il échappé aux assassins?
- "Oui, et il est en sûreté....
- Que le ciel en soit loué! Donnez-moi les détails de sa fuite. >>\setcounter{page}{385} Edouard raconta, autant qu'il les savoit, les particularités de cette remarquable histoire, et Fergus l'écouta avec le plus vif intérêt. Il le questionna ensuite sur plusieurs de ses amis, et, quand il vint à prendre les informations les plus précises sur le sort de son clan, Edouard put lui prouver, d'après diverses circonstances, que les Mac Yvor avoient en général peu souffert, ce que Fergus apprit avec une extrême satisfaction. "Vous êtes riche," dit-il, "Waverley, et vous êtes généreux; quand vous entendrez dire que ces pauvres Mac Yvor sont inquiétés pour leurs misérables possessions, par quelque agent cruel du gouvernement, souvenez-vous que vous avez porté leur tartanne, et que vous êtes un fils adoptif de leur race. Le Baron qui connoît leurs mœurs, et vit près de leur pays, vous informera du moment et des moyens de les protéger. Voulez-vous faire cette promesse au dernier Vich Ian Vohr?"
Edouard, comme on peut le supposer; lui en donna sa parole; et il la tint si bien qu'il fut par la suite connu dans leurs montagnes, sous le nom de l'ami des fils d'Yvor. "Plut à Dieu," reprit Fergus, "que je pusse vous transmettre mes droits à l'amour\setcounter{page}{386} et à l'obéissance de cette brave et antique race — ou du moins que je pusse engager ce pauvre Evan à accepter la vie aux conditions qu'on lui impose, et à être pour vous, ce qu'il a été pour moi, le plus tendre, le plus courageux, le plus dévoué." Ici, les larmes que son propre sort n'avoit pu arracher à Fergus, tombèrent en abondance à l'idée de son frère de lait.
Mais, dit-il, en séchant ses pleurs, cela n'est pas possible. Vous ne pouvez pas être pour eux Vich Ian Vohr, le fils de Jean le Grand, et ces trois mots magiques, ajouta-t-il avec un foible sourire, sont le charme qui dispose de toutes leurs affections, et il faudra que le pauvre Evan suivre son frère dans la mort comme il l'a fait dans toute la vie. . . . . .
— "Et je suis sûr," s'écria ce brave vassal en se levant de terre où il s'étoit, par discrétion, tenu couché si tranquillement dans ce lieu sombre, qu'Edouard ne l'avoit pas aperçu. " Je suis sûr qu'Evan n'a jamais desiré ni mérité une meilleure fin que celle précisément de mourir avec son chef. . . . . .
— "Et à présent," reprit Fergus après un silence, "que dites-vous du Bodach Glan? Je l'ai revu cette nuit. Il étoit dans le rayon\setcounter{page}{387} de l'une qui tombait sur mon lit de cette fenêtre étroite et élevée. Pourquoi le craindre, dois-je, ai-je pensé, demain, long-temps avant cette heure, je serai aussi immatériel que lui. — Esprit mensonger, lui ai-je dit, es tu venu terminer tes courses sur la terre et jouir de ton triomphe sur le dernier descendant de ton ennemi? Le spectre a paru faire un signe d'approbation et s'est évanoui.
— Que vous en semble Édouard? Mon confesseur, un homme excellent et plein de sens, m'a dit que l'église ne niait pas la possibilité de ces sortes d'apparitions; mais il m'a défendu d'arrêter là-dessus mes pensées, de peur des tours étranges que joue l'imagination. Que pensez-vous de cela?
— "Je pense," répondit Édouard qui voulait briser là-dessus, "beaucoup comme votre confesseur. Un bruit léger à la porte annonce ce brave prêtre. Waverley se retire pendant qu'il administre les sacrements aux deux condamnés, et lorsqu'au bout d'une heure on l'admet de nouveau en leur présence, il est bientôt suivi d'un forgeron qui vient enlever leurs fers, escorté d'une file de soldats.
"Vous voyez, dit Fergus, l'hommage qu'ils rendent à la force et à la valeur de nous\setcounter{page}{388} autres Montagnards. Ils nous enchaînent comme des bêtes féroces, et quand ils brisent nos fers, la peur que nous ne prenions le château d'un coup de main, les porte à nous faire garder par la force armée...... Bientôt on entend le tambour de la garnison qui bat aux armes.—" C'est là le dernier appel," dit Fergus, " auquel j'obéirai. Et à présent, mon cher, bien cher Edouard, avant que nous nous séparions, parlons de Flora, le sujet qui touche à ce qu'il y a de plus tendre, de plus sensible dans mon cœur."—" Mais nous ne nous séparons pas ici," interrompit Waverley.—" Oui, nous nous séparons ici," dit Fergus, " vous n'irez certainement pas plus loin...." Et entrant alors avec un calme, mêlé d'une sorte d'ironie, dans la discussion des circonstances les plus horribles de sa mort, " à le prendre comme on voudra," ajouta-t-il, " cela ne peut pas durer longtemps. Mais, dans ce qu'un mourant peut supporter avec calme et fermeté, il y a de quoi tuer l'ami plein de vie qui le regarde."—Un bruit de chars et de chevaux se faisant alors entendre, il reprit: " Je vous l'ai déjà dit, vous n'irez pas plus loin ; et comme ce bruit m'avertit que le temps presse, dites-moi comment vous avez trouvé la pauvre Flora? "\setcounter{page}{389} Edonard, d'une voix suffoquée par l'agonie qu'il éprouvoit, lui donna pourtant quelque idée de l'état de sa sœur...
"Pauvre Flora," répondit Fergus, "elle auroit supporté tout ceci pour elle-même... mais pour moi !... vous, Waverley, vous connoîtrez bientôt le bonheur d'une affection mutuelle dans le mariage, long-temps, bien long-temps, puissiez-vous en jouir avec Rose.... Mais vous ne concevrez jamais la pureté du sentiment qui unit deux orphelins tels que Flora et moi, laissés seuls, pour ainsi dire, dans l'univers, et nous tenant lieu de tout l'un à l'autre depuis notre plus tendre enfance. Mais l'idée énergique qu'elle a de ses devoirs, du dévouement qu'on doit à ses Rois, rendront par la suite du nerf à son ame... Alors elle ne pensera plus à Fergus que comme aux héros de notre race, dont elle aimoit à se rappeler les exploits,"
"Elle ne vous reverra donc pas, elle sembloit pourtant...."
"Il a été nécessaire de la tromper, de lui éviter cette séparation effroyable.... Je n'aurois pu la quitter sans larmes, et je ne puis supporter que ces hommes se croient le pouvoir de m'en faire verser... Je lui ai écrit, et cette lettre, que lui remettra mon confesseur, lui apprendra que tout est fini,"\setcounter{page}{390} Un officier parut alors, et annonça que le grand scheriff et sa suite attendoient devant le château, réclamant les corps de Fergus Mac-Yvor et d'Evan Dhu Maccombich. - "Je vous suis, " dit Fergus. En conséquence, soutenant Edouard de son bras, il descendit l'escalier de la tour. Evan et le prêtre les suivoient immédiatement, et les soldats faisoient l'arrière-garde.
Un escadron de dragons et une compagnie d'infanterie, rangée en bataillon carré, occupoient la cour, et au centre de ces troupes on voyoit la charette destinée à emmener les condamnés à un mille de Carlisle, au lieu de l'exécution. Ce tombereau, de couleur noire, étoit traîné par un cheval blanc; à l'une des extrémités, étoit assis le bourreau, tenant dans sa main une large hache; c'étoit un homme d'un aspect horrible et féroce, bien assorti à sa profession. A l'autre bout, et plus près du cheval étoit un siège vide à deux places. L'on apercevoit à travers les arches gothiques du passage long et obscur, qui s'ouvroit sur le pont levis, le grand scheriff et son cortège à cheval, auxquels le cérémonial usité entre le pouvoir civil et militaire, n'avoit pas permis d'aller plus loin. "Cela n'est pas trop mal arrangé pour un dénouement, " dit Fergus en sou-\setcounter{page}{391} riant avec dédain à cet appareil de terreur. — "Ce sont," s’écria Evan Dhu avec une sorte de vivacité en regardant les dragons, "ce sont les mêmes coquins qui ont pris le galop devant nous à Gladsmuir, avant que nous ayons pu en tuer une douzaine. Ils n’en ont pas l’air moins fier à présent.".... Le prêtre lui imposa silence.
Alors le tombereau s’approcha; Fergus se retourna vers Waverley, le serra contre son cœur, et après l’avoir embrassé à deux reprises, s’élança légèrement à sa place. Le prêtre suivoit dans une voiture à part. — Comme Fergus agitoit encore sa main, en signe d’amitié pour Edouard, les soldats se serrèrent autour de la charette, et toute la procession se mit en marche. Elle s’arrêta un moment sous le passage voûté, pendant que le gouverneur du château délivroit les personnes des condamnés entre les mains du pouvoir civil. Dieu sauve le Roi George!, s’écria le grand sheriff après cette formalité. — "Dieu sauve le Roi Jaques!" répondit Fergus d’une voix haute et ferme et se tenant debout sur sa charette... Ce furent les derniers mots que Waverley lui entendit prononcer.
La procession se remit en marche, et la charette disparut de dessous le portail où\setcounter{page}{392} elle s'étoit arrêtée. La marche des morts ainsi qu'on l'appelle, fut à l'instant entendue, mêlant ses sons mélancoliques à ceux d'une cloche drapée qu'on sonnoit de la cathédrale voisine. A mesure que la procession s'éloigna, la musique militaire s'affoiblit par degrés, et le triste tocsin des cloches retentit seul dans l'air.
Le dernier des soldats de la file avoit disparu, la cour même du château étoit déjà vide, que Waverley y restoit encore atterré et immobile, les yeux fixés sur les voûtes sombres où il avoit aperçu son ami pour la dernière fois. Enfin, une femme du peuple vient le réveiller de cet état de stupéfaction. Il enfonce son chapeau sur ses yeux, traverse rapidement les rues devenues désertes, et de retour à son auberge, au bout d'une heure et demie d'angoisses inexprimables, le bruit du tambour et des fifres jouant une marche animée, et le bourdonnement confus de la foule, lui apprennent que tout est fini.
Dans la soirée, le prêtre lui fit une visite, et lui répéta les dernières paroles de son ami expirant. "Fergus Mac-Yvor meurt comme il a vécu, se souvenant jusqu'à la fin de son amitié pour Edouard Waverley." Il ajouta, qu'il avoit aussi vu Flora, dont l'ame sembloit\setcounter{page}{393} bloit moins bouleversée depuis qu'elle n'avoit plus rien à redouter. Il se proposoit de quitter le lendemain Carlisle avec elle et la sœur Thérèse, pour se rendre au port le plus proche et de là gagner la France. Waverley força cet honnête ecclésiastique à accepter une bague de prix, et une somme d'argent destinée à payer les services de l'église en faveur de l'ame de Fergus Mac-Yvor. Il vouloit en cela donner une dernière consolation à sa sœur, "et pourquoi d'ailleurs," pensa-t-il, "ces actes de souvenirs ne feroient-ils pas partie des honneurs que, dans tous les cultes, on se plaît à rendre aux amis qu'on a perdus?"
Après des scènes aussi fortes, le reste paroît sans couleur. Waverley épouse Rose, le Baron obtient sa grace, et ses amis rachètent pour lui ses domaines à son insçu. On fait une surprise agréable à ce brave vieillard, on le conduit dans sa terre, tous les ours sont remis à leur place, la coupe même est retrouvée, les principaux personnages du roman sont heureux, mais on ne s'en soucie plus.
Dans un dernier chapitre, intitulé : Postscriptum, qui devroit être une Préface. L'auteur avance, que les liseurs de romans sautent toujours les préfaces et commencent les
Littérat. Vol. 58. N⁰. 3. Mars 1815. Ff\setcounter{page}{394} livres par la fin, et en conséquence, il place à la suite de son ouvrage les remarques qu’il veut faire lire avant l’ouvrage même.
"Aucune nation européenne," dit-il, "n’a subi, dans l’espace d’un peu plus d’un demi siècle, un changement aussi complet que la nation Écossaise. Les suites de l’insurrection de 1745, c’est-à-dire, l’abolition de l’autorité patriarcale des chefs de clans dans les montagnes, celle de la juridiction héréditaire des Barons dans le plat pays, l’extinction totale du parti jacobite, qui se glorifioit de maintenir les anciennes coutumes, l’accroissement des richesses et les progrès du commerce, ont également contribué à rendre les Écossais actuels aussi différens de la génération qui vient de s’éteindre, que les Anglais de 1814 le sont des Anglais du temps d’Élisabeth. La marche du nouvel ordre de choses, bien que constante et rapide, a cependant été graduelle, en sorte que c’est en portant les yeux derrière soi à quelque distance, qu’on peut le mieux s’en apercevoir… Cette vieille race, qui conservoit un attachement sans espoir pour la maison de Stuart, a totalement disparu, et avec elle sans doute bien des préjugés absurdes, mais aussi bien des exemples touchans de loyauté, de désintéressement, de généreuse hospitalité, de dévouement\setcounter{page}{395} ment pour ses Rois, et d'antique foi écossaise.
Des scènes imaginaires, des personnages fictifs m'ont servi à faire revivre des évènemens dont je tenois les récits de ceux mêmes qui y avoient joué un rôle.
Le combat de Preston et l'escarmouche de Clifton sont dépeints d'après les narrations de témoins oculaires les plus dignes de foi, confrontées avec les relations des meilleurs historiens... et même dans la partie romanesque de cet ouvrage il y a beaucoup de vérité....
Après avoir fait une part honorable à quelques descriptions des antiques mœurs de son pays dans un genre différent, l'auteur exprime le vœu que la tâche d'en conserver le souvenir eût été confiée à la seule plume digne de le retracer. Et puisqu'il a déjà interverti l'ordre accoutumé, il finit par une dédicace à l'Addisson Calédonien, à l'auteur du Man of feeling et du Mirror, le respectable vieillard, Mr. Henry Mackenzie.
\setcounter{page}{490}
\chapter{ROMANS}
\section{MANSFIELD-PARK.}
Le succès assez général qu'a obtenu le roman de Pride and prejudice, Orgueil et préjugé, nous engage à en extraire un second du même auteur.— Nous reviendrons à la méthode employée précédemment pour quelques ouvrages du même genre; celle de remplir l'intervalle des scènes qui méritent d'être traduites en entier, par un récit abrégé qui fasse suivre le fil des évènemens, et montre les personnages les plus intéressans sous les divers points de vue où l'auteur a voulu les placer.\setcounter{page}{491} (Trois sœurs assez mal partagées du côté de la fortune, mais douées de beauté, ont des chances fort différentes dans la loterie du mariage. Miss Ward l’aînée, a le bonheur de plaire à un riche Baronet, Sir Thomas Bartram. Elle devient sa femme, et partage avec lui la possession d’une riche terre nommée Mansfield-Park dans le comté de Northampton. Les espérances qu’une alliance aussi brillante donne à ses deux sœurs, miss Marie et miss Francis, ne se réalisent pas. L’une fait une folie d’amour en épousant Mr. Price, lieutenant de marine sans fortune, sans éducation, sans crédit, et se brouille avec sa famille à cette occasion. L’autre se trouve heureuse, après six ans d’attente trompée, de donner sa main à Mr. Norriss, le vicaire qui dessert la cure de Mansfield-Park.
L’opposition qui se trouvoit entre le caractère de lady Bartram et celui de mistriss Norriss, loin de nuire à la bonne intelligence, la favorisoit. Lady Bartram étoit une personne complétement apathique. Le soin de sa famille, composée de quatre enfants, deux fils et deux filles, ne l’occupoit guères. Elle croyoit avoir tout fait en appelant auprès d’elle une gouvernante bien recommandée, pour ses filles, et quant aux garçons, c’étoit l’affaire de Sir Bartram. Elle passoit\setcounter{page}{492} sa vie assise ou couchée sur un sofa. Un petit chien favori étoit le seul objet de sa sollicitude, et lorsqu'elle avoit fait quelques festons ou quelques points de tapisserie dans la journée, elle étoit très-satisfaite de l'emploi de son temps. Mistriss Norris, au contraire, avoit une surabondance d'activité qui ne trouvoit son application que dans le ménage de lady Bartram, c'étoit elle qui se chargeoit d'en surveiller l'économie. Elle gouvernoit, et se mêloit de tout dans la maison de sa sœur, dont elle flattoit les enfans et grondoit les domestiques. Sir Thomas, qui étoit un homme de sens, auroit bien voulu que sa belle-sœur eût eu un peu de la paresse que sa femme avoit de trop, mais il aimoit la paix, et supportoit ce qu'il ne pouvoit changer.
Au bout de plusieurs années, pendant lesquelles mistriss Price n'avoit conservé presqu'aucune relation avec ses sœurs, elle essaya de rentrer en grace, et en communiquant à lady Bartram la naissance d'un neuvième enfant, dont elle lui demandoit d'être marraine, elle consultoit Sir Thomas sur la manière de placer son fils aîné âgé de onze ans. Cette lettre réveilla quelques sentimens d'affection et de pitié pour mistriss Price. On envoya une layette pour l'enfant, et un peu d'argent.\setcounter{page}{493} d'argent. Mistriss Norris se chargea d'écrire et de transmettre les avis demandés. Une fois qu'elle eut recommencé à s'intéresser à sa sœur, elle ne s'en tint pas là. Le desir de se faire valoir d'une réconciliation entière sans qu'il lui en coûtât rien, lui fit naître l'idée d'engager Sir Thomas à se charger d'un des enfans de sa sœur. Elle réussit à lever toutes les objections, à aplanir toutes les difficultés. Mistriss Price consentit avec joie à confier à ses sœurs la petite Fanny, l'aînée de ses filles. Mistriss Norris prit tous les arrangemens relatifs au voyage de l'enfant, et fut elle-même à sa rencontre jusqu'à la ville de Northampton. )
La petite fille arriva sans accident à Northampton. Elle y trouva sa tante, qui, avec son officieuse activité, fut charmée d'être la première à recevoir cette enfant et à la présenter à la famille.
Fanny Price avoit dix ans. Sa figure n'avoit rien de remarquable au premier coup-d'œil, quoique ses traits fussent agréables. Elle étoit un peu pâle, et plutôt petite pour son âge. Sa timidité, quoiqu'extrême, ne lui donnoit pas de la gaucherie. Le son de sa voix étoit fort doux, et quand elle parloit, sa physionomie étoit gracieuse.\setcounter{page}{494} Sir Thomas et lady Bertram l'accueillirent avec bonté. Le premier sur-tout, voyant qu'elle avait grand besoin d'être encouragée, chercha de son mieux à la mettre à son aise, mais il avait naturellement un air de gravité, qui inspirait le respect plutôt que la confiance, tandis que lady Bertram, sans se donner la moindre peine, et seulement avec un sourire bienveillant, fut d'abord pour sa nièce, la personne la moins redoutable de la famille.
Les jeunes cousins et cousines de Fanny montraient aussi l'intention de la bien recevoir. Les deux fils, âgés de seize et dix-sept ans, étaient à ses yeux des hommes faits. Il y eut quelqu'embarras dans la manière de leurs sœurs au premier moment, parce qu'elles se sentaient observées par Sir Thomas, qui leur avait fait à cette occasion une leçon assez solennelle. Cependant elles étaient trop accoutumées à l'indulgence ou à la flatterie de tout ce qui les entourait pour avoir de la timidité, et celle de Fanny augmentant leur confiance, elles commencèrent à l'examiner de la tête aux pieds avec un air d'indifférence supérieure.
C'était une belle famille que celle de Sir Thomas. Les fils étaient grands et bien faits, les filles décidément jolies, et fort déve-\setcounter{page}{495} loppées, ce qui rendoit le contraste aussi frappant sous les rapports extérieurs que sous ceux de l'éducation. La différence d'âge n'étoit cependant pas considérable. Julia n'avoit que douze ans, et Marie treize. La pauvre petite étrangère étoit aussi malheureuse que possible. Craintive et déconcertée, elle n'osoit lever les yeux, et répondoit à peine aux questions qu'on lui adressoit, tant elle avoit peur de ne pouvoir s'empêcher de pleurer. Mistriss Norris ne l'avoit entretenue tout le long du voyage que du bonheur extraordinaire auquel elle étoit appelée, et de la reconnoissance qu'elle devoit en concevoir. Fanny se reprochoit de ne pouvoir être contente, et n'en étoit que plus à plaindre. En vain Sir Thomas s'efforçoit de l'encourager; en vain mistriss Norris répétoit que Fanny étoit une grande fille, bien raisonnable, et qui voudroit faire honneur à ses parens. Les caresses même de lady Bartram, qui lui fit place sur son sofa à côté du petit chien, ne purent surmonter sa tristesse. La vue d'une excellente tourte aux confitures n'eut pas plus d'influence: il fallut que son petit cœur se dégonflât: elle se mit à pleurer. Le repos paroissant le meilleur remède à son chagrin et à la fatigue d'un long voyage, on la fit mettre au lit, et le sommeil vint bientôt tarir ses larmes.\setcounter{page}{496} "Ce commencement ne promet pas beaucoup," dit mistriss Norris, quand Fanny eut quitté la chambre. "Ce n’est pas faute de l’avoir exhortée chemin faisant, je vous en réponds. Je lui ai fait sentir sur-tout combien il étoit important que la première impression fût en sa faveur; mais elle pourroit bien tenir un peu de sa pauvre mère, qui étoit fort boudeuse. Cependant, il ne faut pas nous rebuter; au fond il est assez naturel qu’elle ait du chagrin d’avoir quitté ses parens: elle ne sait pas encore ce qu’elle gagne au change. Il faut lui donner le temps de se reconnoître."
Il fallut en effet du temps, et plus que mistriss Norris n’étoit disposée à en accorder, pour accoutumer Fanny à la nouveauté de sa situation et à l’éloignement de tous ceux qu’elle aimoit. Il y avoit bien chez les habitans de Mansfield une intention générale de la traiter avec amitié, mais elle avoit une sensibilité vive, et rarement elle étoit comprise.
Le lendemain de l’arrivée de Fanny, il y eut suspension complète de leçons. On imagina que ce seroit un moyen de la familiariser avec ses cousines, mais cela ne produisit pas grand’chose. Les demoiselles Bartram prirent une fort mince opinion de Fanny en\setcounter{page}{497} voyant qu'elle ne savoit pas un mot de français, et ne paroissoit point comprendre le mérite d'un duo de piano qu'elles lui firent la faveur de lui jouer.
Enfin, lasses d'essayer en vain de l'égayer ou d'exciter son admiration, elles lui abandonnèrent généreusement les joujoux dont elle faisoient le moins de cas, et reprirent leurs passé-temps ordinaires.
Fanny ne trouvoit nulle part de quoi remplacer ce qu'elle avoit perdu en quittant sa famille. Le silence de lady Bartram, l'air imposant de son mari, les réprimandes de mistriss Norris augmentoient également sa timidité naturelle. Ses cousines répétoient souvent leurs remarques sur la petitesse de sa taille. Miss Lee, la gouvernante, ne cessoit de s'étonner de son ignorance, les domestiques faisoient des plaisanteries sur ses vêtemens; enfin, la pauvre enfant, loin de se plaire dans les vastes et somptueux appartemens du château, craignoit toujours de salir ou de gâter les meubles et n'osoit toucher à rien. On se doutoit peu de tout ce qu'elle éprouvoit, mais le soir, quand elle étoit dans son lit, elle se livroit sans contrainte à son chagrin et s'endormoit rarement sans pleurer.
Un jour son cousin Edmond la trouva toute en larmes dans un endroit écarté du jardin.\setcounter{page}{498} Edmond avait un excellent naturel; il s'approcha d'elle, et la pressa avec amitié de lui dire ce qui la chagrinoit. A toutes ses questions, sur la cause possible de ses pleurs, à toutes ses offres de services, elle répondoit, "non, non, je vous assure, non, je vous remercie," mais enfin quand il lui parla des parens qu'elle avait quittés, le redoublement de ses sanglots lui fit comprendre qu'il avait touché la corde sensible. "Vous regrettez, ma bonne petite Fanny, d'être séparée de votre maman, c'est bien naturel; mais pensez pourtant que vous êtes ici avec de bons amis, qui desirent de tout leur cœur de vous rendre heureuse. Allons, faisons un petit tour de promenade ensemble, et nous causerons." Fanny se calma peu-à-peu. Edmond lui parla en détail de ses frères et sœurs. Il comprit que son frère William, l'ainé de la famille, et qui avait un an de plus qu'elle, était celui qu'elle aimait le mieux. C'était à lui qu'elle avait recours toutes les fois qu'elle avait besoin de protection contre la turbulence ou la malice des cadets; c'était lui qui plaidait pour elle, quand il y avait quelque grace à demander à sa mère, dont il était le favori; c'était lui qui avait témoigné le plus de regrets en la voyant partir.\setcounter{page}{499}
"Eh bien!" dit Edmond, "je gage qu’il vous écrira."
"Il me l’a bien promis," répondit Fanny, "mais il m’avoit recommandé d’écrire la première; et je n’ai ni papier ni plume."
"Oh! s’il ne tient qu’à cela, je vous donnerai tout ce qu’il vous faut. Serez-vous bien contente de pouvoir écrire à William ?"
"Oh oui! bien contente. Mais qui est-ce qui portera ma lettre à la poste ?"
"Ne vous inquiétez pas de cela, ma petite cousine, j’en aurai soin ; elle ira avec les autres lettres, et elle ne coûtera rien à William, parce que papa l’affranchira."
Fanny fut un peu en peine, de penser que la lettre passeroit sous les yeux de son oncle, mais elle en prit son parti. Edmond rentra à la maison avec elle, lui raya son papier, lui tailla sa plume, lui corrigea son orthographe, avec la complaisance d’un bon frère ; mais ce qui la toucha plus que tout le reste, fut un mot d’amitié qu’Edmond ajouta pour William de sa propre main. Il glissa en outre une demi guinée dans la lettre et se chargea de l’expédition.
Fanny ne savoit comment remercier son cousin, mais sa physionomie et quelques mots sans suite et sans arrangement, suffirent à Edmond pour comprendre que Fanny n’étoit\setcounter{page}{500} point ingrate. De ce moment il fit plus d'attention à elle, et son intérêt s'accrut en trouvant qu'elle avoit un cœur excellent, et une intention constante de bien faire. Il s'établit entr'eux une relation qui remplaça un peu pour Fanny celle de son frère. Edmond l'encourageoit, l'avertissoit doucement, et la faisoit valoir auprès des autres.
Elle prit insensiblement plus de confiance en elle-même. Elle commença à oser lever les yeux devant Sir Thomas. La voix de sa tante Norris ne la faisoit plus tressaillir de crainte. Ses cousines trouvoient qu'une troisième compagne ajoutoit quelquefois aux plaisirs de leurs jeux, et convenoient que Fanny étoit assez bonne enfant. Tom, l'aîné des fils de Sir Bartram, avoit les dispositions assez ordinaires aux jeunes gens qui entrent dans le monde avec la perspective d'une grande fortune, et se croient nés pour jouir de la vie sans en supporter les obligations. Un enfant de l'âge de Fanny, ne pouvoit être pour lui qu'un jouet ; il étoit persuadé, que pourvu qu'il lui fit de temps en temps de petits présens, il avoit le droit de s'en moquer toutes les fois que cela pouvoit le divertir mieux qu'autre chose.
Sir Thomas et mistriss Norris s'applaudissoient mutuellement de la bonne œuvre qu'ils\setcounter{page}{501} avoient faite, et commencèrent à espérer qu'elle leur donneroit, à tout prendre, plus de satisfaction que de peine. Fanny savoit lire, écrire, et coudre, mais son éducation n'avoit pas été plus loin, et ses cousines, voyant qu'elle ignoroit beaucoup de choses qui leur étoient depuis long-temps familières, en concluoient qu'elle n'avoit aucun talent. Elles arrivoient de temps en temps dans le salon pour faire part à leur mère de leurs découvertes dans ce genre, et toujours avec de nouvelles exclamations. Imaginez donc, maman, que Fanny ne sait pas assembler les pièces de la carte d'Europe! ,,--`` Concevez-vous que Fanny ne connoisse pas les notes de musique? " -- "Fanny croit bonnement que le soleil tourne autour de la terre. Peut-on être plus bête! "
``Ma chère," répondoit mistris Norris, d'un ton doctoral, ``cela est en effet bien ridicule, mais vous ne devez pas vous attendre à trouver beaucoup d'enfans développés au point où vous l'êtes. "
``Mais, ma tante, c'est qu'on n'a pas d'idée de son ignorance. Savez-vous ce qu'elle nous disoit l'autre jour? c'est que pour aller en Irlande on passoit à l'isle de Wight. Elle l'appelle l'isle tout court; comme s'il n'y en avoit point d'autre dans le monde. C'est pour\setcounter{page}{502} tant de ces choses qu'on n'a pas besoin d'apprendre. Je me souviens que je savais tout cela longtemps avant d'avoir dix ans. A quel âge est-ce que je récitais déjà toute la chronologie des Rois d'Angleterre et les principaux événements de leur règne?
"Il est vrai, mes chères amies, que vous avez une mémoire étonnante, et votre pauvre cousine en a probablement fort peu. Vous ne devez vous apercevoir de cette différence que pour plaindre ceux qui n'ont pas reçu de la nature des talents aussi distingués que vous. La modestie, mesdemoiselles, relève l'éclat des plus brillantes qualités."
C'était ainsi que mistress Norris travaillait à former le jugement de ses nièces. Il n'est pas étonnant qu'avec des talents aussi précoces, et tant d'instruction prématurée, il y eut un déficit sur d'autres points. Le caractère avait été un peu négligé. Elles ne savaient ce que c'était que générosité, connaissance de soi-même, devoirs envers les autres : on leur enseignait tout, excepté ce qui pouvait les rendre aimables et heureuses.
Il se passe quelques années sans qu'aucun changement ait lieu dans la situation de Fanny. Mr. Norris meurt; il est remplacé dans ses fonctions par le Dr. Grant, qui vient avec\setcounter{page}{503} sa femme prendre possession de la cure de Mansfield. Le fils aîné de Sir Thomas fait des dettes. Son père, pour réparer cette brèche dans sa fortune, et éloigner son fils de quelques liaisons dangereuses, prend le parti d'aller avec lui aux grandes Indes, soigner des propriétés qu'il a dans les colonies anglaises. Edmond se destine à l'église, et ses études l'éloignent de Mansfield une partie de l'année, mais il est toujours l'ami et le protecteur de Fanny; et les temps de vacances qui le ramènent à la maison paternelle, sont les plus heureux pour Fanny. Maria Bartram fait la conquête d'un jeune homme riche et sot. L'idée de l'indépendance et de la fortune qu'elle acquerra par ce mariage lui fait illusion sur le peu d'amabilité de l'époux, et la chose se conclut au retour de Tom Bartram, chargé d'apporter le consentement de son père. Ce dernier est obligé de prolonger encore de quelques mois son séjour aux Indes).
Les choses en étoient là, et Fanny avoit atteint sa dix-huitième année lorsque la société du voisinage reçut une addition agréable par l'arrivée d'un frère et d'une sœur de mistriss Grant, la femme du nouveau vicaire. Flora et Henri Crawford, orphelins de bonne heure, avoient été élevés par un oncle riche. Il y avoit une grande différence d'âge entr'eux et\setcounter{page}{504} mistriss Grant, qui les avoit perdus de vue depuis plusieurs années. Elle se faisoit un plaisir extrême de recevoir sa sœur chez elle, mais elle craignoit que l'habitude des plaisirs de Londres ne lui rendît bien insipide le séjour de la campagne. Miss Crawford de son côté, n'étoit pas sans quelqu'inquiétude à ce sujet, et elle fit promettre à son frère, qui l'accompagnoit et n'étoit pas beaucoup plus amateur qu'elle de la vie champêtre, de revenir bien vite la chercher si elle s'ennuyoit.
On se revit de part et d'autre avec beaucoup de satisfaction. Miss Crawford fut charmée de trouver que sa sœur n'avoit ni la pédanterie d'une ménagère, ni la rusticité d'une campagnarde, et que le Dr. Grant se présentoit comme un gentilhomme.
Flora Crawford étoit remarquablement jolie. Son frère, sans avoir de beaux traits avoit une figure très-agréable. Tous deux prévenoient singulièrement en leur faveur par la grace et la vivacité de l'expression. Mistriss en fut enchantée dès le premier moment. Sa sœur sur-tout étoit pour elle un objet d'espérance et de sollicitude maternelles. Même avant son arrivée, elle s'étoit occupée de lui chercher un mari. Dans son opinion, rien n'étoit plus brillant pour miss\setcounter{page}{505} Crawford, puisqu'entre toutes ses aimables qualités elle devoit apporter en dot à son époux, vingt mille livres sterling. Ses prétentions se dirigèrent donc tout naturellement sur le fils aîné du Baronet Sir Thomas Bartram. Comme elle étoit très-communicative, il ne se passa pas plus de deux ou trois heures avant que Flora ne fût informée de ce projet.
Flora avoit vu Tom Bartram à Londres, et tout en plaisantant sur la prévoyante activité de sa sœur, elle saisit l'idée avec assez de vivacité. Henri ne tarda pas à pénétrer ce petit secret. "Ensuite," dit mistriss Grant, "je vous dirai que j'ai pensé encore à une chose, qui, si elle réussissoit, ne me laisseroit plus rien à désirer. J'ai la passion de vous fixer tous deux dans ce pays, et j'ai dans l'esprit, mon cher frère, que vous seriez très-heureux d'obtenir la main de miss Julia Bartram. C'est une personne charmante, gaie, vive, remplie d'esprit." Henri fit une inclination de tête et remercia mistriss Grant.
"Ah ma sœur," dit Flora, "si vous venez à bout de fixer les résolutions de ce jeune homme et de le déterminer au mariage, vous serez bien habile. Toute mon éloquence y a échoué. J'avois trois amies qui soupiroient pour lui. Les mères, les\setcounter{page}{506} tantes s'en sont mêlées. Vous n'avez pas d'idée de tous les ressorts qu'on a fait jouer pour le prendre au piège. Il a une coquetterie abominable. Vos demoiselles Bertram pourraient bien y être attrapées, je vous en avertis; elles ne seraient pas les premières."
" Est-il possible !", interrompit mistriss Grant, "je crois qu'elle vous calomnie."
" Voilà une bonne ame ! "dit Henri, "qui ne se hâte point de croire le mal. Je suis prudent; je ne veux pas hasarder mon bonheur par une résolution précipitée; et c'est précisément parce que je respecte infiniment l'état du mariage que je veux attendre d'en être plus digne."
" Oui, oui ! croyez cela ! Regardez un peu son air hypocrite. — Je vous assure que c'est un véritable vaurien. Les leçons du général l'ont tout-à-fait perverti."
" J'ai tant vu de ces jeunes gens ," dit mistriss Grant, "professer de l'éloignement pour le mariage, que je ne m'y prends plus. Cela veut dire seulement qu'ils n'ont pas encore rencontré l'être prédestiné."
Les nouveaux venus furent accueillis avec joie au château. La jolie figure de miss Crawford ne lui fit point de tort auprès des demoiselles Bertram. Elles avaient assez de\setcounter{page}{507} beauté pour ne pas craindre la comparaison. Elles lui permettoient d'avoir des yeux noirs, animés et spirituels, un teint agréable quoiqu'un peu brun, et une tournure gentille parce que les avantages qui les distinguoient elles-mêmes étoient d'un genre différent.
Quant à Henri, on commença par trouver qu'il n'avoit rien de remarquable que ses manières, qui étoient celles d'un homme comme il faut. Le jour suivant, on découvrit qu'il étoit mieux qu'on n'avoit cru d'abord : il avoit beaucoup de physionomie, les dents parfaitement belles, un ensemble gracieux. Enfin à la troisième entrevue, il fut jugé tout d'une voix le plus aimable jeune homme qu'on eût jamais connu. L'engagement de Marie ne laissoit pas lieu à rivalité entre les deux sœurs, en sorte que Julie regarda cette conquête comme lui étant réservée, et avant qu'Henri eût séjourné une semaine à Mansfield, elle étoit convaincue d'avoir une passion dans toutes les formes.
Les idées de Marie sur le sérieux de sa position étoient très-vagues. Elle préféroit n'y pas penser. Quel mal pouvoit-il y avoir à se plaire dans la société d'un homme aimable? Tout le monde savoit que sa main étoit promise. Mr. Crawford n'avoit qu'à prendre garde à lui.\setcounter{page}{508} Mr. Crawford redoutoit peu cette espèce de danger. Il voyoit qu'il seroit fort bien venu à faire sa cour à deux jolies femmes. Ses vues n'alloient pas au-delà; et d'ailleurs ses notions de délicatesse dans ce genre n'étoient pas très-rigoureuses.
"Elles sont vraiment charmantes, vos demoiselles Bartram, ,, dit-il à sa sœur, "après les avoir reconduites à leur voiture, un jour qu'elles avoient fait visite au presbytère.
Je savois bien ,, répondit mistriss Grant, , qu'elles vous plairoient beaucoup ; mais vous préférez Julia. "
"Il n'y a pas de doute, je préfère Julia."
"Parlez vous bien sérieusement? on pourroit trouver Marie plus belle."
"Il est certain que Marie a des traits plus réguliers, une tournure plus noble; mais je préfère Julia, car cela est décidement convenable."
"Je vous l'ai dit, mon frère, que vous finiriez par la préférer."
"Je fais bien mieux puisque je commence."
"D'ailleurs, mon cher, pensez toujours que Marie est engagée. Son cœur n'est plus libre."
"Eh bien , croiriez-vous qu'elle ne m'en plaît que mieux. Une femme engagée est d'une\setcounter{page}{509} société beaucoup plus agréable qu'une demoiselle à marier. Elle a plus de confiance en elle-même, elle craint moins de se compromettre, et d'ailleurs on ne court pas les mêmes dangers à lui adresser des hommages."
"Quant au danger, il n'y en a pas assurément. Mr. Rushworth est un trop bon parti pour qu'on le traite avec légèreté."
"Flora a l'air de croire," dit Henri, "que miss Bartram n'en fait pas tout le cas qu'il mérite. Voilà comme vous vous jugez entre amies, vous autres femmes. Mais moi, je ne veux pas supposer qu'elle donne sa main sans son cœur."
"Vous êtes trop malin pour moi," dit mistriss Grant, "on ne sait jamais si vous êtes de bonne foi; qu'en dites-vous, Marie? Est-ce qu'il ne se moque point de nous?"
"Je crois qu'il faut l'abandonner à sa persiversité. Les bons conseils ne lui profitent pas. Il finira par être attrappé comme tous les autres."
"Comment! tous les autres."
"Eh oui! tous ceux qui se marient."
"Mais, ma chère, vous n'y pensez pas."
"Mon Dieu, oui! j'y pense. Il n'y a aucune affaire dans laquelle on court plus de\setcounter{page}{510} chance d'être attrapé. On ne s'y engage qu'avec l'espoir d'y trouver son propre avantage sans s'embarrasser d'apporter sa part dans le lot commun."
" Ah, ma chère !" dit mistriss Grant en soupirant, "je crains que vous n'ayez été à une mauvaise école chez notre oncle le général."
" Ma pauvre tante," répondit Flora, "n'était pas payée assurément, pour dire du bien de l'état conjugal; mais je ne parle que d'après mes propres observations, et j'ai toujours vu faire plus de marchés que de véritables mariages."
" En vérité vous exagérez, ma sœur. Il y a beaucoup d'exceptions; et les unions heureuses ne sont pas si rares que vous le dites. On sait bien qu'il y a partout le pour et le contre. L'état du mariage a ses épines, mais il faut en voir aussi les bons côtés."
" Allons ! j'admire votre esprit de corps. Quand je serai mariée, je ferai bonne contenance comme une autre."
" Vous ne valez pas mieux que votre frère, mon enfant," dit mistriss Grant, "mais vous deviendrez tous deux plus sages, si nous pouvons vous garder encore quelque temps."
(Flora et Henri prolongent en effet leur\setcounter{page}{511} séjour au presbytère, et ne pensent plus à retourner à Londres.
Henri fait de grands progrès dans les bonnes graces des demoiselles Bertram, et Flora, tout en jetant ses filets pour le frère aîné, trouve que le cadet vaut bien la peine qu'on fasse quelques frais pour lui plaire. Elle fait venir sa harpe de Londres et Edmond passe des matinées délicieuses à l'entendre. Elle prend la passion de monter à cheval, et Edmond l'accompagne dans ses promenades.
Fanny voudroit aimer miss Crawford, car elle a l'habitude de partager tous les sentiments de son cousin, et elle voit que sa société lui est fort agréable, mais cependant elle reçoit souvent des impressions défavorables à Flora, dont le petit manège de coquetterie ne lui échappe point, et dont la légèreté de principes la scandalise quelques fois.
L'époux de Marie, Mr. Rushworth, a des projets d'embellissemens pour sa terre de Sotherton située dans le voisinage de Mansfield. Il invite la famille Bertram et les Crawford à y passer une journée, pour avoir leur avis sur ses plans.
Fanny est de la partie, par grande faveur, car comme sa tante Bertram ne va jamais\setcounter{page}{512} nulle part et que Fanny lui est très-dévouée, c'est toujours elle qui est chargée de lui tenir compagnie. Mistriss Grant s'offre à la remplacer dans cette occasion et lady Bartram y consent).
Bientôt après le déjeuner, la calèche fut attelée. Henri en étoit le conducteur. Il y avoit deux places sur le siège, Mistriss Grant trouva moyen de faire accepter l'autre à Julia. Marie de fort mauvaise humeur de cet arrangement, monta dans la calèche et se piqua peu d'entretenir la conversation avec ses compagnons de voyage. La gaieté d'Henri et de sa sœur, dont elle entendoit souvent les éclats de rire, lui donnoit une certaine inquiétude qu'elle dissimuloit à peine.
Le temps étoit superbe; le pays à parcourir charmant. Tout étoit nouveau pour Fanny. Elle jouissoit en silence et ne regrettoit que de n'avoir pas son cousin pour lui communiquer ses observations. Edmond à cheval, suivoit ou devançoit la voiture. Lorsqu'il passoit près de la portière, Fanny et Flora avançoient la tête de son côté, et disoient quelquefois toutes deux ensemble: "Le voilà!" C'étoit le seul point sur lequel elles se rencontrassent. Le spectacle de la nature qui avoit tant de charme pour Fanny,\setcounter{page}{513} laissait miss Crawford tout-à-fait indifférente. Le tour d'esprit de Fanny la portoit à la réflexion, celui de Flora à la plaisanterie; en sorte que sans mistriss Norris, la conversation auroit été fort languissante. En approchant de Sotherton, cependant, la curiosité de Marie fut excitée sur ce qu'elle alloit voir. Mistriss Norris admiroit tout ce qu'on lui disoit appartenir à cette terre; la vanité de Marie étoit flattée en pensant qu'elle en alloit devenir la maîtresse, et Fanny même étoit écoutée lorsqu'elle disoit son petit mot d'éloge.
On entra enfin dans une longue et belle avenue qui conduisoit au château. Rushworth se présenta pour recevoir les dames, sa mère les introduisit dans un salon où une collation élégante étoit préparée. On parcourut ensuite les vastes appartemens du château où la richesse brilloit plus que le goût, et l'on finit par la chapelle qui en dépendoit.
Fanny dont l'imagination s'étoit montée sur des descriptions de chapelles gothiques fut fort déçue en ne trouvant ni ces inscriptions à moitié effacées, ni ces bannières que les vents agitent, et qui font entendre comme des gémissemens lugubres, ni ces tombeaux qui attestent la brièveté de la vie.
Je vous en fais mon compliment de contentement\setcounter{page}{514} doléance, "lui dit Edmond à qui elle se plaignit de ce mécompte" mais ceci est moderne, et ne doit point ressembler aux chapelles qui appartenoient à des monastères ou à d’anciens seigneurs.
Mad. Rushworth commença alors à faire en détail l’histoire de la chapelle de Sotherton, de l’époque où elle avoit été bâtie ; de celle où les bancs avoient été réparés, et les coussins de cuir noir, changés en coussins de velours cramoisi. "Autrefois," dit-elle, "on y lisoit les prières en famille, mais Mr. Rushworth à abandonné cet usage."
"La civilisation fait tous les jours des progrès," dit miss Crawford à Edmond, tandis que Mad. Rushworth s’éloignoit pour répéter son discours à ceux qui ne l’avoient pas entendu.
"Il me semble," dit Fanny, "que ces dévotions de famille devoient avoir quelque chose de touchant et de respectable. C’étoit un lien de plus entre les pères et les enfans, les maîtres et les domestiques."
"Oui," répondit Flora, "c’étoit fort beau; mais croyez-vous de bonne foi que les pauvres femmes-de-chambres et les laquais, à qui on faisoit faire des exercices de piété, comme récréation après le travail, en fussent fort édifiés, tandis qu’ils voyoient leurs\setcounter{page}{515} maîtres s'en dispenser sous les plus légers prétextes?"
"Ce n'est sûrement pas ainsi que Fanny l'entend," dit Edmond, "l'exemple seul est efficace en pareil cas."
" Pour moi, "ajouta miss Crawford," je trouve qu'il ne faut pas vouloir gouverner les consciences; tolérance et liberté en matière de religion, c'est ma devise, si les bonnes gens qui se croyoient obligés de venir à l'heure fixe, se mettre à genoux et bâiller sur ces galeries, avoient prévu qu'il viendroit un temps où l'on pourroit avoir la migraine à l'heure du service, ils auroient bien envié notre bonheur. Figurez-vous un peu combien de fois les miss Bridget de la famille sont venues ici avec des mines dévotes, tandis que leur imagination trottoit ailleurs; sur-tout si le chapelain ne valoit pas la peine d'attirer leurs regards; et j'imagine que c'étoit d'assez pauvres espèces que ces chapelains: encore pires que ceux d'aujourd'hui."
Fanny rougit et regarda Edmond, comme pour l'inviter à répondre. Elle même étoit trop indignée de cette plaisanterie pour en dire son avis. 〟
"Vous avez beaucoup de talent, mademoiselle," dit Edmond,"pour faire ressortir\setcounter{page}{516} le côté plaisant des choses, et vous amusez toujours si vous ne persuadez pas; mais croyez-vous que les dévotions particulières de ceux qui ne savent point se gêner dans l'observation du culte public, ni fixer leur attention à volonté, fussent plus sincères et plus ferventes?"
"Mais, oui. — Dans la retraite du cabinet on a moins d'objets de distraction, et puis l'épreuve n'est pas si longue."
Julia vint alors interrompre leur conversation en disant: "Regardez donc Marie et son futur époux! Les voilà au pied de la chaire comme si la cérémonie alloit s'accomplir."
Henri s'approcha de l'oreille de Marie, et lui dit: "Je n'aime pas vous voir si près de l'autel."
Marie tressaillit involontairement et s'en éloigna quelques pas; mais Julia continua sa plaisanterie. "Qu'est-ce donc qui nous manquerait, ajouta-t-elle, tous les parens sont rassemblés, les paroles données, les cœurs à l'unisson; ce serait charmant. — Et la bénédiction nuptiale qui est-ce qui la donnerait? Voilà à quoi je ne pensois pas. Quel dommage, Edmond, que vous n'ayez pas encore pris les ordres!"\setcounter{page}{517} A ces mots, miss Crawford, qui n'avoit point soupçonné qu'Edmond fût destiné à l'église, parut extrêmement déconcertée. Fanny comprit la cause de son embarras et la plaignit véritablement.
" Comment donc," dit Flora en s'efforçant de reprendre son assurance ordinaire. Vous allez entrer dans l'état ecclésiastique."
" Aussitôt que mon père sera de retour," répondit Edmond, "à Noël probablement."
" Si j'avois su cela, j'aurois parlé avec plus de respect de la confrairie," dit miss Crawford, en s'acheminant vers la porte de la chapelle, "tout le monde la suivit. La gaieté générale fut un peu obscurcie. Marie n'avoit point goûté les plaisanteries de sa sœur, et Flora sentoit qu'elle avoit fait une école avec Edmond.
On se dispersa dans les jardins. Flora découvrit l'entrée d'un bosquet très-agréable par sa fraîcheur, Edmond et Fanny l'y suivirent. Henri de son côté ne crut pas être trop indiscret en rompant le tête à tête de Marie et de Rushworth ; et sous prétexte de donner ses avis pour les embellissemens projettés, il les accompagna dans leur promenade. Julia, retenue par politesse auprès des\setcounter{page}{518} deux dames, faisait pénitence et rachetoit son triomphe du matin.
Fanny étoit peu accoutumée à marcher; la séance dans les appartemens et la chapelle l'avoit déjà fatiguée; elle proposa de s'asseoir; mais bientôt miss Crawford, qui n'aimoit pas à rester long-temps en place, éleva une question sur l'espace à parcourir pour arriver à un certain endroit du bosquet, et ne voulut être persuadée qu'en jugeant par ses propres yeux. Elle emmena Edmond, en disant : "Nous allons revenir, miss Price sera bien aise que nous la laissions reposer encore quelques momens."
Un quart d'heure, vingt minutes se passèrent et Fanny toujours espérant les voir arriver, ne voulut pas risquer de les manquer en prenant un autre sentier. Enfin, elle entendit des voix, mais c'étoit Marie et ses deux adorateurs.
"Miss Price, toute seule!" s'écrièrent-ils. Fanny fit son histoire. "Pauvre Fanny!" dit Marie, "ils sont bien impolis. Vous auriez beaucoup mieux fait de venir avec nous." Alors ils s'assirent, et la discussion sur ce qu'ils avoient vu, recommença. Marie goûtoit fort toutes les idées de Henri, et Rushworth, qui étoit accoutumé à adopter celles\setcounter{page}{519} des autres, faisoit chorus. Après quelques minutes de conversation, Marie remarquant une grille en fer qui séparoit le bosquet du park, demanda s'il ne vaudroit pas mieux aller de ce côté là pour dominer une plus grande étendue de terrain et juger de l'effet des plans proposés. Henri saisit cette idée avec beaucoup de vivacité, et insista sur l'impossibilité de rien déterminer sans avoir vu le château de tous les points de vue environnans; mais la grille étoit fermée. Rushworth regrettait de n'avoir pas pris la clef, et se promettoit pour une autre fois de l'avoir en poche. La fantaisie de miss Bartram devint d'autant plus vive, que la chose paroissait plus difficile, et il fallut enfin que Rushworth se décidât à aller chercher la clef.
"Comment trouvez-vous Sotherton?" dit Marie, quand il fut parti. "Là, sans compliment, cela répond-il à ce que vous attendiez?"
"Mais, oui, pas mal. Il y a de la grandeur. C'est assez beau dans son genre, quoique ce ne fût peut-être pas celui que je préférerois. A vous dire le vrai, ajouta-t-il en baissant un peu la voix, "je ne reverrai plus cet endroit-ci des mêmes yeux. L'été prochain, tout y sera changé pour moi: et quels\setcounter{page}{520} que soient les embellissements qu'on pourroit y faire, je n'y retrouverai pas le même plaisir.
Après un moment d'embarras, Marie répondit: "vous autres gens du monde, vous réglez votre opinion sur la mode, et si c'est une chose reconnue que Sotherton a gagné à ces changements, vous serez du même avis... Ce n'est pas là la question," dit Henri en soupirant: "D'ailleurs, je ne suis pas un homme du monde autant que vous le supposez; malheureusement pour moi. Je n'ai pas, du moins, cette faculté si commode dans certains cas, d'oublier le passé et de commander au sentiment."
Il y eut un moment de silence, puis Marie reprit: "vous paroissiez bien gai ce matin pendant la route. Il m'a semblé qu'entre Julia et vous, c'étoit à qui riroit du meilleur cœur."
"Réellement! avons-nous été si gais?... Oui, c'est vrai. Je m'en souviens à présent; mais je vous jure que je serois bien embarrassé à dire de quoi nous avons ri... Ah! voici ce que c'est! Je faisois à votre sœur une histoire comique, d'un Irlandais qui étoit palfrenier chez mon oncle... Elle aime beaucoup à rire, miss Julia."\setcounter{page}{521} "Vous la croyez plus gaie que moi."
"Plus aisée à amuser, du moins, par conséquent meilleure compagnie," dit Henri en souriant d’un air significatif. "Je gage que je n’aurois point réussi auprès de vous de la même manière."
"Je suis naturellement aussi vive que ma sœur, mais vous comprenez, que dans ce moment-ci, toutes mes pensées ne sont pas couleur de rose."
"Ah, sans doute, je le comprends. Vous avez trop de sensibilité pour que la circonstance où vous êtes ne vous rende pas sérieuse. Mais rien au moins ne doit vous attrister.— La perspective qui s’offre à vous est des plus riantes."
"Ne parlons que de celle qui est en effet sous nos yeux," interrompit Marie. "Je meurs d’envie de m’élancer dans le parc. Cette barrière m’est insupportable. Je dis comme le serin de Sterne : je ne puis pas sortir.— La clef n’arrive point." En même temps Marie s’avança vers la grille avec un air d’impatience.
Henri la suivit en disant : "Je comprends que pour rien au monde vous ne voudriez franchir cet obstacle sans l’aveu et la protection de Rushworth, car, d’ailleurs, il ne se\setcounter{page}{522} roit pas impossible de se passer de la clef. Il y a là un petit passage où, avec mon aide, vous traverseriez lestement de l’autre côté, si réellement vous en avez la fantaisie et qu’un léger scrupule ne vous arrête pas."
"Moi, du scrupule! et pourquoi? Mr. Rushworth n’a aucun droit à le trouver mauvais. Il va être ici dans l’instant, et nous serons encore en vue. Il ne peut manquer de nous atteindre. D’ailleurs, s’il tardoit, miss Price auroit la bonté de lui dire que nous avons pris les devants et qu’il nous trouvera là-bas sous ce gros chêne.—Voyez-vous—vers cette éminence. "
Fanny, qui blâmoit l’étourderie de sa cousine, essaya de la retenir en lui disant: " mais je vous assure que vous faites une imprudence, chère Marie, vous risquez de vous accrocher à ces pointes de fer. Tout au moins, votre robe est en danger d’être déchirée.— Croyez-moi, attendez encore un moment. "
Marie n’attendit pas même la fin du raisonnement de Fanny, et lui répondit de l’autre côté de la grille." Bien obligé de l’avis, chère Fanny, mais voilà qui est fait. Vous voyez que ma robe et moi, nous nous en sommes tirés à merveilles. "
Fanny se retrouva seule de nouveau avec\setcounter{page}{523} Des réflexions peu agréables. Marie et Crawford, ayant pris une route assez indirecte pour arriver au gros chêne, disparurent bientôt à ses yeux. Edmond et sa compagne ne revenoient point. Fanny auroit cru qu'ils étoient sortis du bosquet, si elle avoit pu se persuader que son cousin, ordinairement si bon et si attentif pour elle, l'eût complétement oubliée dans cette occasion.
Au milieu de sa rêverie, elle entendit des pas précipités et crut que c'étoit Rushworth, mais elle fut surprise de voir Julia toute essoufflée et toute rouge, qui s'écria, en la trouvant seule." Que faites-vous ici, Fanny?
Et les autres, où sont-ils allés? Je croyois Marie et Crawford avec vous."
Fanny lui raconta ce qui s'étoit passé.
" Belle invention en vérité! Y a-t-il long-temps? on ne les voit plus. Mais ils ne peuvent pas être bien éloignés. Si Marie a passé, je passerai bien, moi; et sans aide même."
" Mais, attendez donc un moment, Julia. Mr. Rushworth va venir avec la clef."
" Oh! je ne m'embarrasse pas de cela. J'ai eu mon compte de la famille ce matin. Je viens d'échapper à la mère dans cet instant. Pendant que vous faisiez des idylles ici tout à votre aise, j'ai été en pénitence. Chacun\setcounter{page}{524} son tour! — Vous auriez bien pu, ma mignonne, rester un peu avec ces bonnes dames; mais vous avez toujours soin de vous esquiver dans ces cas-là. "
Fanny sentit l'injustice de ce reproche, mais elle passoit à Julia ses momens de caprice, parce qu'ils ne duroient pas. " N'avezvous point vu Mr. Rushworth?" lui dit-elle.
"Oui, oui! je l'ai vu, courant à perdre haleine, le pauvre garçon! Il n'a eu que le temps de nous dire pourquoi, et où nous le trouverions.
"C'est dommage qu'il ait pris tant de peine pour rien. "
"Quant à cela, c'est l'affaire de Marie. Je ne suis pas obligée de me punir pour ses sottises. La mère m'a assez ennuyée: il n'est pas juste que j'aie encore le fils sur les bras. "
En disant cela, Julia sauta légèrement de l'autre côté de la grille, sans écouter Fanny.
L'attente du retour de Rushworth, et le sentiment du tort de Marie occupa Fanny et l'empêcha de penser autant à celui d'Edmond. Elle cherchoit comment elle pourroit adoucir la communication qu'elle avoit à faire à Rushworth, lorsqu'elle le vit arriver. Il parut en effet, très-mortifié, mais son regard seul exprima son mécontentement, et sa surprise.
\setcounter{page}{525} "Ils m'ont prié de vous dire, ajouta Fanny, que vous les trouveriez là-bas, vers cette petite éminence ou aux environs."
"Certes! je n'irai pas plus loin," répondit Rushworth avec humeur. "Je ne les aperçois point. Je suis horriblement fatigué." Il s'assit à côté de Fanny et garda un morne silence. Enfin il lui dit: "trouvez-vous, je vous prie, que ce Crawford soit un personnage si distingué? Il y a des gens qui le vantent d'une manière.... Pour moi, je ne vois rien en lui de si merveilleux."
"Je suis assez de votre avis," dit Fanny; "il me semble qu'on exagère son mérite."
"On dit qu'il est bel homme, par exemple," ajouta Rushworth, "je voudrois bien savoir ce qu'on peut admirer en lui. Je parie qu'il n'a pas cinq pieds neuf pouces. Ces Crawford ne nous vont point. Tout alloit mieux avant qu'ils vinssent ici."
Un léger soupir de Fanny répondit à la tirade de Rushworth.
"Si j'avois fait quelque difficulté d'aller chercher cette clef," continua-t-il, "on auroit pu dire que c'étoit ma faute, mais au moment où elle l'a voulu, je me suis mis à courir."
"Rien ne pouvoit être plus obligeant que\setcounter{page}{526} votre 'procédé', et vous avez surement fait toute la diligence possible, mais il y a loin d'ici au château, et vous savez que lorsqu'on attend, les minutes semblent des quarts d'heure."
Rushworth un peu calmé, se laissa enfin persuader d'aller chercher Marie; et Fanny se mit aussi en mouvement pour retrouver ses premiers compagnons. Ils s'annoncèrent par des éclats de rire, et au détour d'un sentier, elle les rencontra. Il étoit évident que le temps avoit passé pour eux d'une manière fort agréable, et qu'ils ne s'étoient point aperçus que leur absence eût été aussi longue.
L'heure du dîner rassembla les convives, et Fanny s'aperçut bientôt, en regardant Julia et Rusworth, qu'elle n'étoit pas la seule qui eût éprouvé des mécomptes dans cette journée.
Comme on avoit une course de dix milles à faire pour retourner à Mansfield-Park, l'intervalle qui suivit le dîner jusqu'au départ ne fut pas long. Mistriss Norris se fit pourtant un peu attendre, parce qu'elle avoit saisi l'occasion de se faire de bons amis de la gouvernante et du jardinier, ensorte qu'elle emportoit force recettes, des œufs de faisans, des grains pour son jardin, et d'autres bagatelles de ce genre, Enfin on prit congé. Henri\setcounter{page}{527} approcha de Julia, et lui dit : " j'espère obtenir de vous la même faveur que ce matin, si vous ne craignez pas que l'air soit trop frais à l'heure qu'il est." Julia, surprise agréablement de cette proposition, ne se fit pas trop presser. Marie fit un peu la mine mais la conviction d'être au fond la personne préférée lui fit prendre cette petite mortification en patience.
" Éh bien, Fanny, " dit mistriss Norris entrant dans l'avenue de Mansfield-Park, " avouez que voilà une réunion de plaisirs comme vous n'en avez eu guères dans votre vie. Vous pouvez nous remercier, votre tante Bartram et moi, pour vous avoir ménagé ce jour de fête. "
", " Mais vous même, ma tante, " dit Marie, " vous n'avez pas mal employé votre temps et vous revenez chargée d'une quantité de bonnes choses. — Ce que je sais du moins, c'est qu'il y a un certain panier qui m'a étorché le coude tout le long du chemin. "
", " Ah! ce sont des plantes de bruyère que ce brave jardinier a voulu absolument me faire prendre, mais si cela vous gêne, ma chère, je vais le prendre sur mes genoux. — Fanny se chargera de ce petit paquet.— Prenez bien garde, mon enfant; ne le laissez pas tomber. C'est un de ces petits fromages.\setcounter{page}{528} à la crème comme celui que nous avons mangé à dîner. - Cette pauvre mad. Whitaker, m'a tellement pressée d'en emporter un qu'il n'y a pas eu moyen de lui refuser. Quel trésor que cette femme pour madame Rushworth! - On n'a pas d'idée de l'ordre et de l'économie qu'elle maintient dans la maison. - C'est là que les domestiques sont sur un bon pied. - Prenez garde au fromage Fanny. - Elle a mis dehors deux femmes-de-chambre qui se donnoient les airs de porter des robes de perkale blanche. - A présent donnez-moi seulement le panier. Je puis me tirer d'affaires toute seule. "
" Qu'est-ce donc que vous avez encore attrapé là, ma tante? " dit Marie.
" Attrapé! ma chère. - Ce sont des œufs de faisans. Il a bien fallu les prendre. La bonne dame Whitaker les a mis de force dans la voiture. Quand elle a su que je vivois seule, elle m'a assuré que ces petites bêtes me seroient une ressource infinie. Je les ferai couver à la meilleure de mes poules, et si je réussis, vous en aurez dans votre basse-cour. "
Quand mistriss Norris eut cessé de parler, il y eut un silence complet. On étoit fatigué, et il auroit été difficile à chacun de dire si le plaisir, ou la peine l'avoit emporté dans cette journée.
\small{( La suite à un autre Cahier. )}
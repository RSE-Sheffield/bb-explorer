\setcounter{page}{216}
\chapter{ROMANS}
\section{WAVERLEY, ou IL Y A SOIXANTE ANS. (Edinburgh 1814) 3 vol. in-12° \large{( Troisième extrait. Voy. p. 99 de ce vol.)}}

APRÈS nous être arrêtés quelque temps sur des peintures de mœurs qui nous ont paru piquantes, nous allons parcourir avec rapidité des scènes plus importantes peutêtre dans le roman, mais moins caractéristiques. Arrivé dans la plaine, Waverley remarque chez les habitans beaucoup de trouble et d’inquiétude; le peuple redoute une descente des Montagnards, et de toutes parts on s’agite, on s’attroupe, on fabrique des armes. Lui-même et son guide attirent tous les regards, et, dans une petite ville qu’ils traversent, une dispute et bientôt une émeute s’étant élevées, Edouard, contre lequel un forgeron s’élance, un fer rouge à la main, étend cet homme sur la place d’un coup de pistolet. Bientôt, traîné devant le juge, il est obligé de livrer ses papiers, et après qu’on les a examinés, le forgeron, qu’on avoit cru mort, se trouve parfaitement rétabli, mais\setcounter{page}{217} une accusation de haute trahison succède à celle de meurtre. Les lettres des parens d'Edouard, une pièce de vers composée par Flora, et jusqu'au manuscrit, resté intact, de Mr. Pembroke, tout conspire à rendre suspecte sa fidélité au gouvernement. Il monte un cheval de Fergus. Il a assisté au rendez-vous de chasse des clans rebelles. Il est ce même Waverley qu'on cherche partout, parce qu'il s'est manifesté des mouvemens insurrectionnels dans son régiment, et parmi ses propres recrues, à la suite d'ordres donnés par lui et scellés de son sceau. Ces dernières circonstances qu'ignorait Edouard, lui rappellent qu'après la nuit qu'il a passée chez Donald Bean, il n'a plus retrouvé son cachet, et il soupçonne que ce brigand, qui étoit aussi un recruteur du parti rebelle, s'est servi de ce moyen pour suborner sa troupe. L'affaire prenant de jour en jour une tournure plus sérieuse, le juge se décide à envoyer l'accusé au commandant de Stirling, pour qu'on lui fasse son procès en forme. Comme les routes étoient alors peu sûres, on confie Waverley à une escorte de Caméroniens (presbytériens exagérés; dont le fanatisme vulgaire est fort bien dépeint). Ces sectaires, quoique marchant sous un drapeau, n'observoient aucune discipline. Leur chef,\setcounter{page}{218} nommé Gilfillan, n'est occupé qu'à déployer son éloquence grossière et souvent énergique; il se fâche de ce qu'un petit tambour qui les accompagne ne sait que des marches profanes et veut absolument lui faire battre le psaume 119.
Au bout de quelques heures de route, les Caméroniens sont joints par un colporteur à l'issue d'un chemin de traverse. Cet homme écoute le chef avec intérêt et flatte son amour-propre. Étant arrivé le premier avec lui au haut d'une éminence couronnée d'un bois, il se met à siffler, comme pour appeler un petit chien qu'il avoit perdu; ce que Gilfillan trouve très-mauvais.—"Mais," dit le colporteur, "si votre seigneurie vouloit simplement considérer Tobie et son chien,.."—"Tobie et son chien," répond le presbytérien furieux, "sont des apocryphes et des païens, et il n'y a que des papistes qui puissent les citer:... je commence à croire que je me suis trompé à votre égard, mon ami."—"Très-vraisemblablement," repart le colporteur d'un grand calme, et, se mettant à siffler de plus belle, cinq ou six Montagnards cachés dans le bois fondent sur les Caméroniens, leurs claymores à la main. Gilfillan, alors séparé de la plus grande partie de sa troupe, se défend vaillamment, mais\setcounter{page}{219} il est bientôt mis hors de combat. Edouard lui-même ayant été blessé dans la mêlée, est emporté avec une promptitude inconcevable par les Montagnards, et les Caméroniens qui s'étoient enfin tous réunis, voyant leur chef hors d'état de les guider, craignent une autre embuscade, et continuent leur route sans tenter de reconquérir le prisonnier.
Les Montagnards construisent à la hâte un brancard, sur lequel ils placent Waverley et poursuivent leur course rapide. A toutes les interrogations que leur adresse le blessé, ils ne répondent que ces mots, " niel sassenagh ,, point d'Anglais ; Edouard, à sa grande surprise, ne reconnoît point sur leurs tartannes les couleurs du clan de Fergus. Enfin, après que ces hommes ont traversé une forêt et un torrent, ils s'arrêtent dans une pauvre chaumière, et Waverley, pour la seconde fois, est confié aux soins d'un Esculape des montagnes. Pendant son traitement, qui est long parce qu'il lui survient de la fièvre, personne ne lui dit un seul mot qu'il comprenne ; mais une vieille femme nommée Janette, le sert pourtant avec zèle et affection. Il est couché dans une espèce d'alcôve de bois, fermée, et l'on se cache de lui avec le plus grand soin. Quelquefois il croit\setcounter{page}{220} apercevoir, au travers des fentes de la cloison, une figure de femme qui n'est point une paysanne, et son imagination active lui fait espérer que c'est Flora, mais il découvre bientôt qu'il s'est trompé. Lorsqu'il est en état de marcher, quelques-uns des Montagnards qui l'avoient amené, viennent un soir le chercher, et lui font comprendre qu'il faut se mettre en route. Au milieu des préparatifs du départ, il sent son bras doucement pressé et reconnoît Alice, la fille de Donald Bean, qui lui fait signe de se taire. Est-ce elle qu'il a entrevue pendant sa maladie? Est-ce au pouvoir de son père qu'il a été? Il quitte la chaumière sans avoir éclairci ses doutes, mais non sans avoir généreusement récompensé la vieille Janette.
Les Montagnards fournissent des armes à Edouard, en lui indiquant qu'elles pourront lui être utiles. On marche dans les ténèbres et à chaque instant on fait halte, on observe, on écoute, et l'on s'avance, tantôt avec une extrême rapidité, tantôt avec une lenteur timide. Bientôt l'on entend l'appel de sentinelles anglaises, qui se répondent de distance à distance, et ces sons meurent ensuite dans le silence de la nuit. Comme on venoit de passer non loin d'un grand bâtiment éclairé, Duncan Duroch, le chef de la bande, s'arrête\setcounter{page}{221} tête, flaire le vent, ainsi qu'un épagneul, et renvoie tous ses camarades, à l'exception d'un seul. Alors se mettant à quatre, si bien enveloppé dans son manteau, qu'on ne le distingue pas des bruyères sur lesquelles il se traîne, il va faire une reconnoissance. De retour au bout d'un moment, il oblige ses deux compagnons à prendre la même posture et se met de nouveau à leur tête. Edouard qui trouve cette allure très-incommode, sent bientôt une forte odeur de fumée, que l'organe plus exercé de son guide avoit dès long-temps découverte. Cette odeur partoît de l'intérieur d'un enclos de pierres sèches, tel qu'on les construit en Ecosse pour renfermer les troupeaux. Les fugitifs se traînoient le long du mur de cet enclos, quand tout-à-coup Duncan Duroch se relève, et, soit pour faire parade d'habileté, soit pour avertir Edouard de son danger, il l'engage par signes à se relever aussi. Waverley le fait, et regardant par dessus la muraille, il voit en plein, un poste de soldats anglais couchés autour du feu de leur bivouac. La sentinelle seule étoit éveillée; elle alloit et venoit lentement, portant sur l'épaule un fusil dont l'acier réfléchissoit les rayons rouges du feu, et fixant les yeux de temps à autre, comme\setcounter{page}{222} avec impatience, sur la partie du ciel où la lune alors obscurcie se cachoit dans les nuages.
Après une minute ou deux, par un de ces changemens subits d'atmosphère si fréquens dans les contrées montagneuses, il s'élève un air de bise, qui balaie toutes les vapeurs dont se couvroit l'horizon, et la lune, dans son plein, verse sa brillante lumière sur la bruyère découverte et marécageuse au milieu de laquelle étoit situé l'enclos.
La fuite devenait impossible. Edouard et ses guides ne pouvaient rester cachés qu'en se tenant prosternés contre terre, et le moindre mouvement les exposait à une découverte inévitable.
Duncan Duroch prend bientôt son parti, il dit un mot à l'oreille de son compagnon, et se mettant de nouveau à quatre, il se dirige vers l'endroit même d'où il étoit venu. Edouard le voit qui profite avec l'habileté d'un sauvage, du moindre buisson, de la moindre inégalité du terrain qui pent le dérober aux regards, et ne traverse jamais les places les plus en évidence, que quand la sentinelle a les yeux tournés. Enfin, lorsqu'il a gagné les taillis qui bordent le marais dans cette direction, il disparaît tout-à-fait. Mais bientôt il ressort d'un côté dif-\setcounter{page}{223} fèrent, et s'avançant avec intrépidité sur la Bruyère découverte, comme pour provoquer les regards; il s'approcha, ajusta sa pièce et fait feu sur la sentinelle. Une blessure au bras interrompt d'une manière funeste les observations météorologiques de ce pauvre homme et l'air populaire qu'il siffloit, et il lâche un coup inutile. Cependant l'alarme est donnée, et tous les soldats se mettent à la poursuite de Duncan Duroch. Celui-ci, après s'être distinctement offert à leur vue, s'enfonce de nouveau dans les Buissons; et pendant ce tems, Edouard et son compagnon prennent le large. Ils ont le tems de traverser le marais avant le retour des soldats, et au bout d'une demi-heure de course, ils se retrouvent dans un chemin creux. Leur troupe qui les attendait avec des chevaux arrive. Bientôt arrive Duncan Duroch lui-même. Il était à la vésité, hors d'haleine, et avait tout l'air d'un homme qui venoit d'échapper à la mort, mais pourtant il rioit avec fracas du succès de sa ruse de guerre, et de la manière dont il avait trompé les Hanovriens.
Tous les périls de la fuite étant alors surmontés, Waverley et son escorte continuent paisiblement leur voyage, et les premiers rayons du soleil éclairent pour eux le château.\setcounter{page}{224} de Donne, sur lequel flotte le pavillon blanc des Stuarts. Des cocardes blanches ornent les bonnets à plume de tous les Montagnards qui en forment la garnison. Le gouverneur du château, qui portoit lui-même le costume montagnard, reçoit Edouard avec beaucoup de civilité, mais sans répondre à aucune de ses questions.
Après lui avoir fait prendre quelque repos, il le remet à une autre escorte, qui gagnoit le nord, et lui donne un cheval. A quelque distance d'Edimbourg le bruit de la guerre se fait entendre; des coups de gros canon qui retentissent par intervalles, annoncent que l'œuvre de la destruction s'exécute, et bientôt Waverley découvre la capitale de l'Ecosse, s'étendant le long d'une crête de rochers, qui s'abaisse du côté de l'est. Des nuages roulans de fumée enveloppent le château, et à mesure que ces tourbillons ascendants s'éclaircissoient vers leurs bords, de nouvelles bouffées de vapeurs épaisses venoient en obscurcir le centre. Ce spectacle imposant produit une profonde impression sur Waverley; toutefois le feu se ralentit et cesse enfin tout-à-fait. Le chef de la petite troupe évite néanmoins de braver l'artillerie de la citadelle assiégée, qui tiroit sur les renforts destinés au prince Stuart, et se dès\setcounter{page}{225} tournant vers le sud, il s'approche de l'ancien palais d'Holy-Rood, et délivre son prisonnier à la garde de ce bâtiment vénérable.
On introduit Waverley dans une longue galerie, garnie de grandes images de Rois, lesquels s'ils avoient jamais régné, florissoient dans un temps où l'art de peindre étoit presque inconnu. Cette pièce servoit d'antichambre à l'appartement que l'aventureux Charles Edouard occupoit alors dans le palais de ses ancêtres. Plusieurs officiers Écossais, dans les deux costumes, passoient et repassoient en grande hâte, d'autres avoient l'air d'attendre des ordres. Des secrétaires signoient des passeports, dressoient des listes, des rôles de toute espèce. Chacun étoit affairé; tous les esprits sembloient tendus vers quelque but important, et Waverley debout, dans une embrasure de fenêtre, rêvoit, sans qu'on le regardât, à sa destinée, dont la crise paroîssoit s'approcher rapidement.
Comme il étoit profondément plongé dans sa méditation, un froissement d'étoffes écossaises excite un bruit léger derrière lui; tout à coup un bras ami frappe son épaule, une voix amie s'écrie : "A-t-il eu raison le prophète Ecossais? Croira-t-on désormais à ses prédictions.\footnote{Littéralement la seconde vue serait-elle comptée}"\setcounter{page}{226} tourne, et il est dans les bras de Fergus.
— "Mille fois le bienvenu dans Holy Rood!" dans Holy Rood rendu à son légitime souverain ! N'ai-je pas dit que nous prospérerions, que vous tomberiez dans les mains" des Philistins si vous vous sépariez de nous?"
— "Cher Fergus, quel temps il y a que je" n'ai entendu la voix d'un ami ! Et où est" Flora?"
— "En sûreté et la spectatrice triomphante" de nos succès. "
— "Ici?"
— "Oui, à Edimbourg du moins, et vous la verrez. Mais il faut auparavant que je vous mène vers un ami auquel vous ne songez guère, et qui s'est souvent informé de vous."
Disant ces mots, il entraîne Waverley hors de la chambre, et, avant que celui-ci sût où on le conduisoit, il se trouve dans une salle d'audience arrangée avec une apparence de pompe royale.
Aussitôt, un jeune homme à cheveux blonds, également remarquable par la noble expression et par la régularité de ces traits, sort du cercle d'officiers et de chefs Ecossais.
On appelle en Ecosse, seconde vue, un pressentiment des événemens futurs, dont le peuple est intimément convaincu que quelques personnes sont douées. Ceux qui sont censés posséder ce don se nomment Seers des voyans.\setcounter{page}{227} qui l'entouroient, et, à sa grâce parfaite, Waverley crut qu'il auroit reconnu sa haute naissance, lors même que l'étoile et la jarretière, dont il étoit décoré, ne l'auroient pas indiquée.
"Que votre Altesse Royale, dit Fergus en s'inclinant profondément, me permette de lui présenter.......
Le descendant d'une des plus anciennes et des plus loyales familles de toute l'Angleterre," répondit le Prince en lui coupant la parole, "Pardon, mon cher Mac Ivor, si je vous ai interrompu, mais il n'est besoin d'aucun maître des cérémonies pour présenter un Waverley à un Stuart."
En disant ces mots, le Prince tendit la main à Edouard avec une courtoisie charmante, et lui fit des espèces d'excuses sur ce que ses partisans l'avoit à son insu amené par force; l'assurant qu'il étoit entièrement libre, et que s'il vouloit rejoindre l'Electeur d'Hanovre on lui fourniroit un passe-port. Alors, sans laisser à Edouard le temps de se remettre, le Prince lui prouva qu'il étoit déjà au nombre des victimes désignées par le gouvernement, et excitant tour à tour ses passions, il flatta si adroitement son orgueil, enflamma si fortement, tantôt son indignation, tantôt son enthousiasme, que Waverley, comme entraîné par un torrent d'impulsions irrésistibles.\setcounter{page}{228} se jetta aux pieds de Charles Edouard et lui dévoua son cœur et son épée.
Le Prince, embrassa Waverley avec une effusion de reconnaissance, et le présentant aussitôt au brave Lochiel \footnote{Cameron de Lochiel, l'ami le plus fidèle, le partisan le plus zélé du Prétendant. Il le seconda dans toutes ses entreprises, partagea tous ses périls, et le suivit dans son exil en France, où le Roi lui donna le commandement d'un régiment. (R)} et aux autres nobles dont il était entouré, il voulut relever aux yeux de tous le prix de ce nouvel adhérent, et eut avec Edouard une conversation particulière. Il lui exposa la situation de ses affaires, lui développa les deux avis qui partageoient son conseil, celui de marcher en avant et celui d'attendre des secours de France, et lui offrit un grade élevé; mais Edouard refusa cette faveur jusqu'à-ce qu'il l'eût méritée, et demanda de servir comme volontaire sous son ami Mac Yvor. Le Prince, à qui cette réponse fut évidemment agréable, ceignit avec beaucoup de grâce sa propre épée à Waverley et invita les deux amis à revenir le soir. "C'est," dit-il, "peut-être la dernière nuit où nous habiterons ces antiques salles, et comme nous nous battrons avec des consciences pures, nous voulons passer gaîment la veille du combat."
"Comment vous a-t-il plu, " demanda Fergus à son ami, en descendant le grand escalier?\setcounter{page}{229} Comme un Prince pour lequel il faut vivre et mourir ," répondit Waverley.
L'habile chef des Mac Yvor savoit ce qu'il faisoit en exposant son ami à cette entrevue inattendue. Comment Waverley qui avoit été rejetté , calomnié , menacé par les ennemis des Stuarts , auroit-il resisté aux flatteuses sollicitations d'un prince jeune , valeureux , que sa situation périlleuse, que le dévouement de tant de braves chefs , que le palais même qu'il habitoit rendoit plus intéressant encore? Fergus mit Waverley au fait de toutes les intrigues qui agitoient déjà cette nouvelle cour ; il le loua beaucoup d'avoir refusé un grade." N'ai-je pas moi-même , lui dit-il , été obligé de supprimer une patente de Comte obtenue pour de très-anciens services , de crainte d'exciter des jalousies. Il nia d'avoir eu aucune part à la délivrance d'Edouard des mains des Carmeroniens , et conjectura qu'elle étoit due à Donald Bean. Mais quel avoit été le but de ce brigand , et pourquoi il ne l'avoit pas volé , c'est ce qu'il ne concevoit pas. Il mène Edouard chez le baron de Bradwardine qui étoit à Edimbourg, et le soir , Waverley complétement équipé dans un uniforme montagnard aux couleurs du clan Yvor , et très-beau sous ce costume , se rend avec Fergus au palais d'Holy-Rood , où se donnoit une fête splendide.\setcounter{page}{230} Deux femmes assises auprès l'une de l'autre y attiroient tous les regards ; l'une, Flora Mac Ivor ; fière, imposante, au maintien noble et gracieux ; l'autre, Rose de Bradwardine, éclatante de fraîcheur et de jeunesse, et plus animée, plus vive dans son expression qu'elle ne l'avoit encore été. Waverley fut reçu par le Prince avec une distinction marquée, par Rose avec une émotion et un plaisir visibles, et par Flora avec une nuance très-légère de trouble que voilà bientôt une sorte d'affectation polie. "Je le reçois ainsi qu'un second frère," dit-elle à Fergus qui le lui présentoit comme un fils adoptif d'Yvor, et elle pesa sur ce mot de frère d'un ton qui glaça Waverley. Celui-ci piqué ou excité par ces différens accueils, devint extrêmement brillant d'esprit et d'imagination, et produisit une sensation marquée. L'exaltation de toutes les têtes, l'activité de toutes les espérances, l'idée de la gloire et même celle du danger, répandirent une sorte d'ivresse sur les plaisirs de cette soirée qui se prolongea bien avant dans la nuit. Quand les musiciens eurent joué cet air du départ si bien connu des Ecossais, le Prince se leva et s'adressant à l'illustre assemblée qui alloit se séparer : "Adieu," dit-il, "belles Dames qui avez si dignement honoré un Prince\setcounter{page}{231} exilé et proscrit, adieu mes vaillants amis, adieu, et bonheur à tous. Puisse la félicité dont nous avons joui ce soir, être le présage d'un retour prompt et triomphant dans cette demeure de nos pères, et celui de mille et mille réunions de joie et de plaisir dans ce palais d'Holy Rood! ,"
Quand le Baron de Bradwardine raconta dans la suite cet adieu du Prétendant, il ne manqua jamais d'ajouter:
\small{Audiit et voti Phoebus succedere partem\\Mente dedit; partem volueres dispersit in auras.}
\small{Une part de ce vœu du Dieu fut exaucée\\Dans la vague des airs l'autre fut dispersée.}
Le conflit d'une foule de passions diverses avoit épuisé les forces de Waverley, et bientôt il fut plongé dans un profond sommeil. Il rêvoit à Glennaquoich, et avoit transporté dans les salles de Jean le grand, la brillante réunion qui venoit d'orner celles d'Holy Rood. Il entendoit distinctement la cornemuse et cela du moins n'étoit pas une illusion, car le chef de la musique du clan Yvor souffloit déjà dans cet instrument, en parcourant d'un pas solennel la cour de l'auberge où couchoient les deux amis, et ces sons perçans, qui d'abord avoient été en\setcounter{page}{232} harmonie avec le songe d'Edouard; finirent par le dissiper. On part. Waverley demande en vain un cheval. Vich Jan Vohr à pied et convert de ses armes, marche à la tête de son clan, et le Prince lui-même suit son exemple. Parvenus au sommet de la petite colline de St. Léonard, d'où l'on a une vue complète des rochers sur lesquels est bâtie Edimbourg, et de la campagne environnante, le spectacle le plus singulier et le plus animé s'offre à leurs regards. Une multitude de Montagnards, qui dormoient sur la terre enveloppés de leurs manteaux bigarrés, se lèvent et s'agitent de partout, comme des abeilles excitées; tandis que les rochers et le ciel lui-même retentissent du bruit assourdissant des cornemuses, qui les appellent aux armes. L'éclat ondoyant des plumets, des tartannes, des drapeaux, brille confusément sur la prairie. Mais bientôt cette foule en désordre se sépare par masses distinctes. Chaque clan se range sous sa bannière, au son de sa musique particulière, et marche derrière son chef. De tant de mouvemens irréguliers et confus, il résulte un ordre admirable, et bientôt des manœuvres militaires, auxquelles un général expérimenté eût été forcé d'applaudir.\setcounter{page}{233} Les clans séparés se placent à la suite les uns des autres et forment une longue colonne, à la tête de laquelle flotte l'étendart de Charles Édouard, une croix rouge sur un fond blanc, avec ces mots: *tandem triumphans*. Quelque peu de cavalerie, composée en grande partie de gentilshommes de la plaine, forme l'avant-garde, et l'on voit leurs drapeaux; peut-être en trop grande quantité pour une si petite troupe; se déployer à l'extrémité de l'horizon. Des traîneurs de ce corps, qui s'efforcent de le rejoindre et poussent leurs chevaux à travers la foule, malgré les cris et les malédictions qu'ils excitent, ajoutent à la vie et au mouvement, mais non à l'ordonnance régulière de cet ensemble. Toutefois, ce qui donne à l'impression qu'en reçoit Waverley quelque chose de sérieux, d'imposant, et même de terrible, c'est le son du gros canon de la citadelle, qui tire de distance à distance sur tout ce que les artilleurs espèrent encore atteindre.
En examinant de près cette armée, Waverley sentit diminuer un peu la confiance que lui avait inspirée Fergus. Il est vrai que les premiers en rang de chaque clan, bien armés d'une large épée, d'un bouclier, d'un fusil et d'un poignard, auquel s'ajoutoit souvent un pistolet d'acier, se presentoient\setcounter{page}{234} comme les guerriers les plus beaux et les plus robustes qu’on eût pu choisir parmi les troupes de toute la chrétienté, et leur esprit indépendant et fier, malgré leur parfaite obéissance pour leur chef, leur courage indompté et leur habileté à tirer parti du moment, les rendoient également redoutables. Mais c’étoit là une élite particulière, composée de parens plus ou moins proches du chef. Derrière ces hommes, se trouvoient de véritables paysans, qui cependant dédaignoient ce titre, et prétendoient, non sans quelque fondement, être issus de races plus anciennes que ceux auxquels ils obéissoient. Ainsi chaque clan un peu considérable avoit des espèces d’ilotes à sa suite; les descendans de Fingal servoient les Stuarts d’Apine, ceux de Macbeth, les Murray; etc. Mais ces fils de héros avoient une misérable apparence. Lors de l’acte du désarmement, les chefs de clan avoient trouvé moyen de soustraire quelques armes pour leurs proches, mais il n’en étoit pas resté pour ces pauvres gens, qui n’en suivoient pas moins la bannière de celui dont ils tiroient leur subsistance. En conséquence, ils portoient, l’un un fusil cassé, l’autre une épée sans fourreau, l’autre une faulx, et quelques-uns même n’avoient, outre leur poignard, que des pieux arrachés aux sapins\setcounter{page}{235} des montagnes. Cependant ils augmentoient la masse, et par cela même la force des bataillons, et leurs cheveux hérissés, leurs habits en désordre, leur physionomie farouche et hagarde, portoient la terreur chez les habitans de la plaine. Mais à tout prendre, la vue de cette armée, de quatre mille hommes au plus, abattit un moment l'espoir de Waverley. "Y a-t-il bien là," se disoit-il en lui-même," de quoi renverser le gouvernement et changer les destinées d'un puissant Empire?"
Cependant un petit canon de campagne, le seul qu'on possédât, donne le signal du départ. Des cris de joie remplissent aussitôt les airs, et toute la ligne est en mouvement. Waverley se hâte de gagner la tête de l'armée qu'occupoit son clan. Les braves Mac Yvor, qui n'avoient point encore vu leur nouveau frère dans son uniforme, le reçoivent avec acclamation, et leurs cornemuses le saluent par un air de triomphe. Portant alors ses regards derrière lui, Édouard voit la longue colonne des clans tourner autour des bases de la montagne d'Arthur Seat, et il la perd de vue lorsqu'elle arrive sous la crête basaltique si remarquable qui fait face au petit lac de Duddington.
L'armée quitte les terres basses qui bordent\setcounter{page}{236} la mer et s'enfonçant dans l'intérieur du pays, elle occupe le sommet de cette fameuse colline de Carberry, où l'infortunée Marie Stuart se rendit prisonnière entre les mains de ses sujets rebelles. Là, Charles Édouard fit faire une halte à ses troupes, soit dans l'espoir de fondre avec avantage sur l'armée du général Cope, qui devait côtoyer la mer dans sa marche vers Édimbourg, soit parce que c'était une position centrale, de laquelle on pouvait observer et déjouer les desseins de l'ennemi. Bientôt Fergus est mandé auprès du prince. La cavalerie du baron de Bradwardine a rencontré des partis anglais, et il vient d'arriver des prisonniers. Waverley sort de la ligne par curiosité, il voit des estafettes couverts de poussière, qui galopent pour annoncer que l'armée du général Cope est en pleine marche le long de la côte. Tout semble présager un combat prochain, mais un incident funeste donne un nouveau cours aux idées de Waverley. Des cris qui partent d'une chaumière l'obligent à y entrer. Il entend une voix mourante répéter la prière dominicale avec l'accent de sa province, et il découvre, à travers l'obscurité, un dragon Anglais blessé et presque dépouillé, qui lui demande en grace "une seule goutte d'eau,"—"Vous en aurez,\setcounter{page}{237} aurez," répond Waverley qui le porte dans ses bras à l’entrée de la chaumière, et lui fait boire de la liqueur de son flacon. Je crois reconnaître cette voix," dit le dragon, "mais regardant avec effroi les habits d’Edouard, il s’écrie: non, ce ne peut pas être notre jeune seigneur!
Ce nom, qui étoit celui qu’on donnoit à Waverley dans la terre de son oncle, réveille chez lui mille souvenirs confus. Il considère plus attentivement l’homme qu’il a devant lui, et dans des traits déjà défigurés par la mort, il reconnoît avec horreur le sergent de sa petite troupe." Houghton!" s’écrie-t-il, "ô ciel! est-ce toi?" —"Ah! je n’aurois jamais cru entendre encore des mots anglais. Mais, ô capitaine! comment avez vous pu nous abandonner, et nous laisser séduire par ce démon de l’enfer, ce Ruffin?. . . nous vous aurions suivi à travers le feu." —"Ruffin? je vous assure, Houghton, que vous avez été indignement trompés." —"Je l’ai toujours cru, quoiqu’ils nous aient montré votre propre cachet. . . Aussi Trim a-t-il été fusillé et moi, dégradé." . . . Mais déjà ce malheureux étoit à l’agonie. Au bout d’un quart d’heure, il rend le dernier soupir entre les bras de Waverley, le conjurant de prendre\setcounter{page}{238} soin de ses pauvres parens et de ne plus combattre" avec ces sauvages en jupons contre la vieille Angleterre."
Cette scène mélancolique excite une émotion douloureuse et mêlée de remords dans l'âme de Waverley. Les mots souvent répétés par Houghton - ô capitaine, pourquoi nous avez-vous abandonnés? retentissent comme un tocsin au fond de son cœur. "Oui," dit-il, " j'en ai agi envers vous avec une légèreté cruelle. Je vous ai tirés de vos champs paternels, soustrait à la protection d'un seigneur bon et généreux, et après vous avoir livrés à toute la rigueur de la discipline militaire, j'ai évité de porter ma part du fardeau commun, j'ai négligé les devoirs que j'avois embrassés, exposant à la fois et ma réputation et les hommes dont je devois être l'appui, aux attaques des méchans et des traîtres... O indolence et indécision, si vous n'êtes pas vous-mêmes des vices, quelles poignantes douleurs n'entraînez-vous pas souvent à votre suite!, "
Malgré la rapidité de leur marche, le soleil étoit déjà bas lorsque les Montagnards arrivèrent sur les hauteurs, où ils se rangèrent en ordre de bataille. De là, ils dominoient sur la vaste plaine qui s'étend au nord jusqu'à l'Océan, et offre aux regards divers vil-\setcounter{page}{239} lages, parmi lesquels on distingue celui de Preston. Bientôt ils voient au-dessous d'eux défiler l'armée anglaise, qui longe le rivage de la mer et se forme sur une ligne parallèle à la leur. A cet aspect ils poussent un cri terrible, et ils seroient aussitôt tombés sur l'ennemi, si leurs chefs, redoutant l'avantage qu'eût donné à l'artillerie anglaise le terrain entrecoupé et marécageux par lequel il leur auroit fallu descendre, n'eussent réprimé leur ardeur. Semblables à deux gladiateurs qui épient réciproquement le moment favorable pour s'attaquer, les deux armées sont quelque temps immobiles en face l'une de l'autre. L'aspect si différent qu'elles présentent, leurs évolutions, tous les détails de la guerre sont décrits d'une manière extrêmement animée: Il y a encore du talent dans la peinture d'un de ces brusques changemens dans les dispositions de l'âme, si connus des gens à imagination. Les Mac-Yvor ont été envoyés par Charles-Édouard, prendre possession du cimetière de l'église de Preston, et bientôt un détachement de cavalerie anglaise s'avance pour les en chasser. Waverley qui regarde par-dessus le mur, reconnoît tout près de lui les dragons de son régiment. Il voit flotter le drapeau qu'il suivoit autrefois, il entend sonner la charge à laquelle il a si\setcounter{page}{242} Ces mots répandent une joie universelle. Chaque commandant dispose sa troupe sans bruit, et bientôt l'armée, marchant avec un silence et une promptitude étonnante, s'engage toute entière dans un sentier détourné qui traverse le marais. Le brouillard n'avoit pas gagné les hauteurs, en sorte que les Montagnards profitèrent quelque temps de la clarté des étoiles, mais cet avantage fut perdu pour eux, quand, à mesure qu'ils descendirent, ils se plongèrent dans l'Océan de vapeurs épaisses qui s'étendoit sur la plaine et sur la mer. Il y eut à vaincre quelques obstacles, inséparables d'une marche nocturne dans un chemin étroit, rompu et marécageux, mais de toutes les troupes existantes, c'étoit à des Montagnards Ecossais que ces inconvéniens pouvoient être le moins sensibles, et ils poursuivirent leur mouvement d'un pas aussi ferme que rapide.
Alors que le clan Yvor s'approcha du terrain solide, le qui vive d'une vedette ennemie fut entendu à travers le brouillard. "Paix," s'écria Fergus, "à que personne ne réponde s'il veut vivre, en avant, marche" et ils pressent leurs pas en silence.
La vedette fait feu de sa carabine, et ce bruit est aussitôt suivi de celui des fers de son cheval qu'il met au galop en s'éloignant.\setcounter{page}{243} "Hylax in limine latrat," dit le baron de Bradwardine, "ce misérable donnera l'alarme."
Cependant le clan de Fergus, bientôt suivi du reste de l'armée, étoit sorti des terres marécageuses; déjà le jour commençoit à poindre, et comme l'on n'avoit pas compté sur l'effet de la surprise, cet incident ne répandit aucun trouble, et ne fit que hâter les préparatifs du combat.
Les dispositions en furent très-simples. L'armée du prince Stuart occupoit l'extrémité orientale de la plaine, et s'étendoit sur deux lignes entre le marais et la mer, dans un champ moissonné et parfaitement uni et découvert. La première ligne devoit attaquer l'ennemi, la seconde servoit de réserve, et entre deux étoit placé un petit corps de cavalerie, commandé par le Prince en personne. Il avoit d'abord voulu charger lui-même à la tête de la première ligne, mais les instantes prières des chefs l'en avoient enfin dissuadé.
Les deux lignes, formées par les phalanges des clans, s'avancent en même temps, la première pour engager à l'instant le combat. "À bas votre manteau, Waverley," s'écrie Fergus en jetant le sien, "nous gagnerons de l'étoffe pour nos tartannes avant que le soleil\setcounter{page}{244} soit au-dessus de la mer. " A ces mots tous les Mac-Yvor dépouillent leurs manteaux, préparent leurs armes, et il y a une pause solennelle d'environ trois minutes, pendant laquelle ces braves gens, après avoir ôté leurs bonnets, lèvent leurs yeux vers le ciel et prononcent une courte prière. Dans ce moment, Waverley sentit battre son cœur comme s'il alloit éclatter dans son sein. Ce n'étoit pas de l'effroi, ce n'étoit pas de l'ardeur, c'étoit un composé de tous deux, une émotion nouvelle et profondément énergique, dont la violence commença par le glacer et l'étourdir, puis l'éleva à un état d'enthousiasme et de délire. Tout concouroit à exalter son ame; les cornemuses donnoient en plein, et bientôt les clans s'élancèrent tout entiers comme des masses épaisses. Mais ils réglèrent mieux leurs pas à mesure qu'ils avancèrent, et le murmure toujours grossissant de leurs voix devint un cri féroce et sauvage.
Dans cet instant, le soleil qui étoit déjà au-dessus de l'horizon, dissipa les brouillards, les vapeurs s'élevèrent comme un rideau, et laissèrent voir les deux armées au moment où elles alloient se rencontrer. La ligne des troupes réglées Anglaises brilloit de tout l'éclat d'une armée bien équipée, flanquée de cavalerie et d'artillerie; mais son aspect n'inspira\setcounter{page}{245} pira pas d'effroi aux Montagnards. "En avant, fils d'Yvor!" s'écria Fergus, ou les Camerons (un autre clan) "verseront le premier sang." — Ils se précipitèrent en avant, poussant un hurlement terrible.
Le reste est connu..... La cavalerie Anglaise qui devoit charger les clans pendant leur attaque, reçut le feu de leurs fusils, tirés en pleine course, et saisie d'une terreur panique, vacilla d'abord, puis s'arrêta, se débanda, et galopa au loin. Les artilleurs abandonnés par la cavalerie, s'enfuirent quand ils eurent déchargé leurs pièces. Alors les Montagnards qui avoient jeté leurs fusils après avoir fait feu, saisissant leurs larges épées, tombèrent tête baissée sur l'infanterie avec une furie irrésistible.
Les vieilles troupes Anglaises, formées dans les guerres de Flandres, en se défendant avec valeur sur la droite de l'armée, prolongèrent quelque temps le combat, mais leurs rangs trop étendus furent enfoncés en divers endroits par les phalanges compactes des clans, et dans la lutte corps-à-corps qui s'ensuivit, les armes particulières, ainsi que l'audace et l'activité prodigieuses des Montagnards, donnèrent à ces fiers enfans du nord une supériorité décidée.
Déjà le cri du triomphe éclatte dans tous\setcounter{page}{246} les clans. La bataille a été livrée et gagnée; tout le bagage, l'artillerie, les munitions des troupes de ligne restent au pouvoir des insurgés. Jamais victoire ne fut plus complète, et à l'exception de la cavalerie, qui avoit pris la fuite au premier abord, à peine échappat-il un seul homme.
La joie de Waverley ne fut pas longtemps sans mélange. Vers la fin du combat, il aperçut de loin le brave G. son ancien colonel, qui, après avoir été abandonné de ses propres cavaliers, s'étoit mis à la tête d'un petit corps de fantassins, et soutenoit avec eux, adossé contre un mur, une lutte désespérée. Les efforts d'Edouard pour sauver cet excellent homme ne servirent qu'à le rendre témoin de sa chûte. Il vit son vieux colonel jeté à bas de son cheval avec une faux par un Montagnard, et percé de plus de coups qu'il n'en eût fallu pour arracher dix vies\footnote{Ce colonel, qui n'est désigné dans le roman que par la lettre initiale de son nom, s'appeloit Gardiner. Il mourut exactement comme on vient de le lire, appuyé contre la muraille de son propre parc, vis-à-vis duquel se donnoit le combat. La connoissance qu'il avoit des localités le mit à portée de donner au général Cope de bons avis, que celui-ci, pour son malheur négligea de suivre. Ce combat, que les Ecossais appellent celui de Gladsmuir, porte le nom de Prestonpans chez la plupart des historiens. (R)}.\setcounter{page}{247} Quand Waverley arriva près de lui, il n'avoit pas perdu tout-à-fait connoissance. Le guerrier mourant parut se rappeler ses traits, et jetant sur lui un regard de reproche et pourtant d'affection, il sembla vouloir s'efforcer de parler. Mais déjà il étoit aux prises avec la mort. Forcé d'abandonner son dessein, il joignit ses mains en attitude de prière et résigna son ame à son créateur. Cet œil mourant qui se fixa sur Waverley, ne frappa point aussi fortement ce jeune homme, dans ce moment de confusion et de trouble, que quand il se présenta par la suite à son imagination.
Cependant l'ivresse est à son comble dans l'armée du Prince. Les Montagnards jouissent avec de bruyants transports de leur gloire et de leur richesse, l'espoir des chefs est exalté au plus haut degré, et chacun fait éclater sa joie. Le baron de Bradwardine seul est un peu triste. Il regrette que la fuite précipitée des dragons Anglais l'ait empêché de donner un modèle du véritable Poelium equestre, et veut du moins faire preuve de zèle en s'acquittant du service anciennement attaché à sa Baronnie, (exuendi seu detrahendi calligas regis post battaliom). En conséquence il se décide à tirer en grande pompe devant toute\setcounter{page}{248} la cour, les bottes ou plutôt le soulier écossais de Charles Edouard qui représente Jaques III son père, en faisant dresser sur place par le baillif Macwheeble un procès-verbal de cet acte féodal. La gravité imperturbable du Baron dans cette occasion, la noble condescendance du Prince, les railleries de Fergus, comparées au récit emphatique de cette cérémonie, que publia le lendemain la gazette d'Edimbourg, fournissent le sujet de quelques chapitres d'un genre original, où une plaisanterie un peu fortement prononcée, est néanmoins agréable.
(La fin au Cahier prochain.)
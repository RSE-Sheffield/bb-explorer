\setcounter{page}{383}
\chapter{MÉLANGES.}
\section{FRAGMENS du PATRONAGE, par MARIA EDGEWORTH, Londres 1814.\footnote{La traduction de cet ouvrage se trouvant déjà annoncée nous nous contenterons d'en tirer quelques fragmens.}}
LADY GRANVILLE étoit d'une haute naissance et possédoit une grande fortune. Elle avoit toujours vécu dans la société de la Cour, soit en Angleterre, soit sur le continent; elle avoit les manières du grand monde; elle s'étoit formé des principes sur la conduite qu'on doit y tenir; et elle étoit dans l'usage de débiter\setcounter{page}{384} en toute occasion des maximes sur ce sujet.
Milady faisait profession de mépriser l'instruction et les principes qu'on trouve dans les livres: elle appeloit tout cela de la théorie non applicable. Elle avoit un tact très-sûr concernant ses propres intérêts, et ne se trompoit guères sur ses vraies convenances. Aussi avoit-elle une confiance en son jugement, qui lui faisoit donner ses avis comme autant d'oracles.
Elle étoit assez mécontente de la manière dont les Percy avoient répondu à ses lettres; mais comme ils étoient ses parens, elle persistoit à vouloir les diriger. Elle s'invita chez eux, et y arriva un beau jour, en dépit des mauvaises routes qui l'en séparoient. Le premier jour, elle étoit si fatiguée et si occupée des dangers qu'elle disoit avoir courus pour arriver à Hill, qu'elle ne parla guères d'autre chose. Le lendemain matin après déjeûner, elle entra ainsi en matière: "mon cher Mr. Percy, j'ai été singulièrement étonnée du parti que vous avez pris de vous séparer du monde. Voulez-vous me permettre de vous dire franchement mon opinion là-dessus?"
"Tant qu'il vous plaira, milady; et j'espère que vous ne serez pas fâchée si je conserve ma façon de penser."\setcounter{page}{385} "Assurément ! il n'y a rien de si juste. Je n'exige jamais qu'on soumette son opinion à la mienne. Je ne puis cependant me dissimuler que j'ai quelque connaissance du monde. Se mêler des affaires de famille est une sorte d'impertinence, qui doit nécessairement amener des brouilleries. J'évite cela avec le plus grand soin. Mais quand on est lié comme nous le sommes, cela fait exception. Voici un bon moment pour causer un peu : les enfans sont sortis. Dites-moi, je vous prie, mon cher Mr. Percy, comment l'entendez-vous donc ? Que comptez-vous faire de vos filles?"
"Faire de mes filles? je n'entends pas, milady, ce que vous voulez dire."
Ah ah ! vous avez l'esprit bien peu ouvert aujourd'hui. Je veux dire, Mr. Percy, que je voudrois savoir comment vous en disposerez."
"Mais.... je ne compte pas en disposer, milady."
"Eh bien, mon cher ami, je vous annonce que vous vous en repentirez. Vous vous en repentirez, c'est moi qui vous le dis. Vous pouvez prendre confiance en moi parce que j'ai quelque connaissance du monde. Il faut mettre vos filles en avant. Il faut tirer parti de leurs avantages de naissance, de relations et de talens. Il faut qu'elles soient à\setcounter{page}{386} portée de faire des connaissances agréables; en un mot, il faut les mettre à la mode: il n'y a que cela pour réussir. Il faut leur faire des protections."
"Quoi! des protections pour mes filles aussi?"
"Eh mais, assurément! on ne se tire point d'affaires sans protection. Lorsqu'elles entreront dans le monde, voici ce qu'il y a à faire. . ." Milady s'arrêta court, parce que Caroline et Rosamonde rentrèrent.
"Continuez, milady," lui dit Mr. Percy.
"Pourquoi priver mes enfants de l'avantage de vous entendre sur ce qui les intéresse?"
"Puisque vous le voulez, je dirai donc devant elles ce qui est vrai, c'est que la mode fait tout. Combien ne voyons-nous pas de jeunes personnes sans figure, sans talent, sans esprit, faire leur chemin dans le monde, uniquement parce qu'on a réussi à les mettre à la mode. Il ne faut pour cela qu'une bonne protection. Tenez, par exemple, vous connaissez mistriss Cotterel? Non..! Eh bien! lady Peppercorn? pas davantage? . . Voilà l'inconvénient de ne pas vivre dans le monde! mais enfin, vous connaissez tous la famille Falconer?"
"Tout au plus," dit Mr. Percy, "quoique nous soyons cousins. Je connais cependant un peu le commissaire,\setcounter{page}{387} "Le commissaire est un homme qui n’est point sans mérite, dans sa partie; mais sa femme est une personne très-distinguée, tout-à-fait distinguée! Croyez-moi, tirez parti de cette femme-là. Je n’en connois pas qui puisse être plus véritablement utile à vos filles, à moins que la jalousie de mère ne s’en mêle. Ah! je ne voudrois pas répondre de ce point là; mais elle a une adresse et un tact très-remarquables. Voyez donc comme ses filles sont recherchées. Cependant pour l’esprit, elles sont fort au-dessous de Rosamonde que voilà, et votre Caroline a des avantages de figure cent fois supérieurs. Mais pour faire valoir ces choses là, il faut du savoir-faire. Ecoutez, voulez-vous me charger de la chose? Je les mettrai sous sa protection. Voulez-vous vous fier à moi? Je ne manque pas de moyens, comme vous savez."
"Je suis tout-à-fait de votre avis, milady, quant aux moyens."
"J’en étois sûre, que nous nous entendrions à merveilles. "
"Mais je n’ai point pour mes filles les vues que vous me supposez. "
"Enfin, mon cher monsieur, vos vues ne peuvent être que de bien marier vos filles. Voyons: Mad. Percy, avez-vous quelque objection à cela?"\setcounter{page}{388} "Non, assurément!" dit Mad. Percy,
"mais permettez-moi de vous demander, milady, ce que vous appelez bien mariées,,"
"J'entends par-là ce que tout le monde entend, 1°. de la fortune.,"
"Qu'est-ce que c'est que de la fortune précisément?,"
"Et mais, voici du nouveau! Mr. et Mad. Percy, quand on vit dans le monde, il faut parler comme tout le monde, et entendre les expressions comme tout le monde les entend. Vous comprenez qu'on ne finiroit pas, s'il falloit consulter un dictionnaire philosophique pour chaque mot. D'ailleurs, avec les définitions, on n'apprend rien du tout. Quand on a bien défini, on n'est pas plus avancé qu'auparavant. C'est ce qui fait que les livres en général n'apprennent rien, absolument rien. Ce que je sais fort bien sans l'avoir appris dans les livres, c'est qu'en ménage, on a beau avoir du bon sens, du mérite, de l'amour et toutes ces choses-là, il faut du pain et du beurre par dessus le marché.,"
"Assurément!" dit Mad. Percy; "et cela signifie toutes les choses nécessaires à la vie."
"Les choses nécessaires et beaucoup de choses superflues aussi, car cela rentre à présent dans le nécessaire.,"
\setcounter{page}{389} Entendez-vous un équipage à quatre chevaux?"
"Oh! non, non, ma chère madame; je suis de bon sens et je m'exprime avec mesure. Un équipage à quatre chevaux pour des demoiselles qui n'ont rien, c'est absurde à espérer."
"Je l'espère aussi peu que je le souhaite pour mes filles, milady; et mes filles, j'en suis sûre, ne le désirent pas plus que moi."
"A la bonne heure; mais enfin si la chose se rencontroit par aventure; vous ne refuseriez pas."
"Cela dépendroit du personnel de celui qui l'offriroit; mais dans tous les cas, je desire que mes filles se décident par elles-mêmes."
"Mais, ma chère madame, vous parlez comme s'il étoit question de choix. Et croyez-vous, je vous prie, qu'il viendra des barons et des chevaliers chercher vos filles dans ce désert?"
"Vous voudriez donc, milady," interrompit Mr. Percy, "qu'elles allassent elles-mêmes à la quête des chevaliers?"
"Il faut se montrer, vous dis-je; il faut se produire, nous ne sommes plus au temps de la chevalerie, où les demoiselles de châteaux\setcounter{page}{390} passaient leur vie à faire de la tapisserie, et n'osoient se montrer qu'à leur balcon. Il faut qu'elles sortent, je vous dis: il faut qu'elles sortent, et personne n'a droit de leur demander ce qu'elles vont chercher. L'usage aujourd'hui est que les demoiselles se mettent en avant. Savez-vous qu'à Bath l'hiver dernier c'étoient les femmes qui prioient les hommes pour danser: ma parole d'honneur! c'étoit à ce point là: elles y étoient obligées.
"Obligées!" dit Mr. Percy.
"Oui obligées, sous peine de rester comme des idiotes toute la nuit sur des banquettes."
"Ce qu'il y a de plus triste," dit Mr. Percy, "c'est qu'à présent, pour avoir un mari, il faut également le chercher. C'est au moins ce qu'assure Mad. Chatterton. Elle est amusante: elle se plaignoit l'autre jour qu'elle avoit promené inutilement ses filles à toutes les eaux. "Je n'ai rien à me reprocher," disoit-elle: "je les ai menées à Bath, à Londres, à Tunbridge, à Weymouth, à Cheltenham, enfin partout. Je ne puis plus rien faire pour elles." Je n'exagère pas; elle a dit ses propres paroles au milieu d'un cercle nombreux."
"Dire ces choses-là dans un cercle, c'est\setcounter{page}{391} une sottise; mais c'est qu'au fait, il n'y a rien de si bête que cette dame Chatterton, et qui pis est, elle a les manières communes. Ah ! je pourrois vous promettre, par exemple, que si vous me donniez votre fille Caroline, je mènerois mieux les affaires. Écoutez-moi: voici mon petit projet; et puis vous ferez ce que vous voudrez. Je la mène à Tunbridge. Je la fais voir un peu, et pas trop. En général, c'est une mauvaise marche que de débuter à Londres, sans s'être fait entrevoir ailleurs. Il faut avoir pour soi quelques prôneurs à la mode, qui fassent circuler d'avance dans la société, des rapports avantageux.
Mr. et Mad. Percy s'expliquèrent tour-à-tour et dans le même sens. L'un et l'autre professèrent un grand éloignement pour de pareils moyens. Non seulement parce que leur délicatesse y répugnoit, mais aussi parce que de telles finesses sont si connues, qu'elles excitent la défiance et le dégoût de ceux qu'on prétend gagner ainsi. "D'ailleurs," ajouta Mad. Percy, "il est bien difficile que, des mariages décidés à la hâte sûr une connoissance faite au bal, puissent être heureux: c'est une vraie cotterie. Il n'y a raisonnablement d'espérance de bonheur, que lorsqu'on a eu le temps de\setcounter{page}{392} se connoître et de s'apprécier réciproquement."
"Allons! à votre aise! vous voulez enterrer vos filles ici. C'est comme vous voudrez: ce ne sont pas mes affaires."
Mr. Percy répondit que leur intention, au contraire, étoit de saisir toutes les occasions que leur position et leur fortune permettroit de mettre leurs filles en relation avec des gens de mérite."
"Et je vous prie, ces gens de mérite vont sortir de terre apparemment? Qui imaginez-vous donc qui puisse venir vous chercher ici ?"
"Mais, milady, puisque vous nous faites l'honneur de venir nous voir, quoique nous ayions perdu notre place et notre fortune, nous pouvons espérer de voir ici des gens comme il faut."
"Ce que vous dites-là est fort poli, Mr. Percy, mais il ne s'agit pas de cela. Voyons, ne me donnez donc pas le chagrin de laisser cette jolie petite Caroline passer ses beaux jours dans la solitude comme la fleur du désert."
"Non, milady," reprit Mr. Percy, d'un ton animé. "Nous n'enverrons point notre enfant au marché de la capitale; à ce marché\setcounter{page}{393} où celles qui réussissent à attraper un mari paient souvent de leur bonheur ce triste succès et où celles qui manquent le but, sont condamnées au mépris. En restant au sein de leur famille, en n'allant point à la chasse des maris, mes filles au contraire, n'éprouveront ni humiliations ni mécomptes. Mariées ou non, elles seront heureuses et respectées. Dans le sanctuaire de la maison paternelle, elles seront toujours des objets d'affection. Nous ne voulons leur faire courir aucune chance étrangère à de telles espérances.
"Verbiage et déraison! mon cher cousin. Je vous en demande pardon, mais tout ce que vous avez dit là n'a pas le sens commun. Au terme de cette sainte réclusion dans la maison paternelle, vos demoiselles se trouveront de vieilles filles, entendez-vous? Ma chère Mad. Percy, au nom du bon sens, ne laissez donc pas sacrifier vos enfants à cette belle philosophie de votre mari! Quel que soit sont avis et son sentiment là-dessus, je suis sûre que vous avez horreur de cette idée de voir vos filles vieillir sans se marier."
"Le vœu de mon cœur," répondit Mad. Percy, "c'est de voir mes filles heureuses dans le mariage, comme je suis moi-même; mais.\setcounter{page}{394} j'aimerais mieux les voir l'une et l'autre dans le tombeau, que de les donner à des hommes indignes d'elles, et uniquement pour qu'on puisse dire qu'elles ont trouvé un mari, ou pour éviter le malheur imaginaire du célibat."
Mad. Percy, qui étoit une personne fort douce, prononça ces mots avec tant de vivacité et d'émotion, que milady en fut tout étourdie, et garda le silence. On lui proposa un tour de promenade qui rompit le cours de ses idées, et elle fut sur le point d'abandonner la partie. Cependant, après le thé, elle renouvela une tentative.
"Voyons, Mr. Percy," dit-elle, "encore un mot pendant que les enfans causent entr'eux."
"Nous sommes dans l'usage de tout dire devant nos enfans. Nous n'avons rien de caché pour eux, et c'est un de nos moyens d'être heureux ensemble. Je leur dis mes projets sur eux, et en quelque sorte toutes mes pensées. Comment pourrois-je, sans cela, espérer qu'elles me doivent les leurs."
"A la bonne heure! on sait fort bien que c'est le devoir des enfans de tout dire à leurs parens : la reconnaissance les y oblige."
"Le devoir, la reconnaissance sont de fort bonnes choses, mais il y a encore un\setcounter{page}{395} point qui a son importance. Souvenez-vous de ce que le duc d'Epernon répondoit à son Roi, qui lui reprochoit de manquer d'affection pour lui: Votre Majesté peut disposer de mes services et de ma vie, mais l'amitié ne se gagne que par l'amitié."
"Je conviens de la chose; mais ce n'est pas précisément d'amitié qu'il s'agit entre parens et enfans."
"Je serois bien fâché que vous apprissiez cette doctrine aux miens, milady."
"Allons, allons, ne chicanons pas sur les mots. Qu'on appelle amitié ou autrement, le sentiment des enfans pour leurs parens, ce qu'il y a de sûr c'est que ceux-ci sont mieux aimés lorsqu'ils sont indulgens sur certains points. Caroline! venez un peu ici. N'ayez donc pas l'air de penser à autre chose, car vous voyez bien que c'est de vous qu'il est question. Voyons! jetez-vous à genoux devant ce père barbare pour plaider votre cause et la mienne. Je veux vous mener avec moi à Tumbridge, ma chère petite, et ce méchant homme ne le veut pas. Cependant, je réponds de votre succès."
"Qu'est-ce que vous entendez, milady, par mon succès?" lui dit Caroline, d'un air très naïf.
"Or ça, n'allez-vous pas aussi prendre les\setcounter{page}{396} aires philosophiques de votre père? Nous autres qui vivons dans le monde, nous ne sommes pas faits à ces manières là. Mais en bon Anglais, je vous dirai, ma chère, au risque de vous faire un peu de chagrin, que si vous venez avec moi, vous serez passablement admirée, passablement enviée, et voici le plus triste, passablement mariée."
"Passablement mariée !" s'écria Mad. Percy, "nous ne serons contentes ni l'une ni l'autre. Nous voulons être très-bien mariées,"
"Ah ah! vous êtes en bonne disposition. On ne sait sur quoi compter avec vous autres femmes philosophes: Ce matin, vous ne vouliez pas d'un équipage à quatre chevaux. A présent vous voilà à l'autre extrême: Être assez bien mariée, c'est avoir deux mille livres de rente: être très-bien mariée, c'est en avoir dix mille."
"Voilà donc le taux du marché aux maris. J'ignorois le sens précis de ces expressions. J'entendois plus que cela en disant bien mariée. Je voulois un homme de mérite, qui eût de l'affection pour ma fille, et qui fût digne d'en être aimé."
"Ma chère amie, permettez-moi de vous donner un conseil. Si vous montez la tête de cette enfant sur le ton romanesque, vous en aurez du chagrin, c'est moi qui vous le\setcounter{page}{397} dis. Croyez-vous donc qu'il va descendre des génies tout exprès du ciel pour épouser vos filles?"
"Il faut faire comme tout le monde, et alors on trouve un mari qui vous rend aussi heureuse qu'une autre."
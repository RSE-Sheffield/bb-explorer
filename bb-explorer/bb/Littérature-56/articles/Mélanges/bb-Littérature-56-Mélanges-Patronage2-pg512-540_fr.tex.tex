\setcounter{page}{512}
\section{FRAGMENS du PATRONAGE, par MARIA EDGEWORTH. \large{(Second extrait. Voy. p. 383 de ce vol.)}}
\subsection{Le Diplomate.}
PEU de jours après, Mr. Percy et Mr. Falconer allèrent ensemble faire leur visite à lord Oldborough. Mr. Percy auroit à peine reconnu son ancien ami tant il étoit changé. Autrefois il étoit gai, prévenant; son abord étoit ouvert; il avoit de l’abandon et des manières franches; il aimoit le monde et la dissipation.
Mr. Percy le retrouvoit sérieux et réservé. Les soucis avoient imprimé leurs traces sur son front. Sa physionomie annonçoit de l’habileté et de la décision, mais on n’y voyoit pas le calme qui naît de la véritable fermeté. Sa politesse étoit roide et cérémonieuse. La conversation sembloit être pour lui un moyen d’étudier les opinions des autres, et non d’exprimer les siennes. On sentoit que ses pensées et toutes ses facultés se portoient vers quelque objet en perspective, et que la jouissance du présent étoit peu de chose pour lui.\setcounter{page}{513}
Mr. Percy et Mr. Falconer observoient lord Oldborough avec des vues très-différentes. Le premier, dans un but philosophique, cherchoit à découvrir les changemens que l'ambition avoit opérés en lui, et l'influence que cette passion avoit sur son bonheur. Le second l'étudioit avec l'intention de tirer parti de ses qualités et de ses défauts, afin d'acquérir une protection utile à ses projets d'avancement. Mais il vit bientôt que l'entreprise étoit délicate, et demandoit tout son savoir-faire. Lord Oldborough étoit sur ses gardes, et sembloit ne pas se soucier d'entrer en relation avec le Commissaire.
On arrivoit promptement à un certain point dans les bonnes graces du lord, on obtenoit aisément son attention, mais l'accès à sa confiance étoit chose difficile, pour ne pas dire impossible.
Mr. Falconer avoit en ressource beaucoup de petits moyens accessoires pour faire valoir son zèle et ses services. Il résolut de commencer par les dépêches que le hasard avoit fait tomber entre ses mains et qu'il avoit déchiffrées; et dans ce but, il mit la conversation sur le naufrage et sur Mr. de Tourville; mais lorsqu'il crut avoir suffisamment excité l'attention et la curiosité de lord Oldborough, il parla d'autre chose, car il ne\setcounter{page}{514} vouloit pas mettre Mr. Percy dans le secret, et il sortit du salon avec le colonel Hanton, neveu de lord Oldborough, sous prétexte de voir ses chevaux, qui devoient figurer aux courses de New-Market.
Lord Oldborough resté seul avec Mr. Percy se départit un peu de la froide réserve qu'il avoit montrée jusques-là. Il ne lui étoit pas difficile de voir que Mr. Percy n'avoit aucune vue intéressée; et que sa visite n'étoit pas pour l'homme de cour, mais pour l'ancien ami. Aussi oublia-t-il pour un moment ses précautions accoutumées et sans s'inquiéter s'il avoit le jour en face ou s'il lui tournoit le dos, de manière à échapper aux observations, il rapprocha sa chaise de Mr. Percy et lui dit: Il y a bien long-temps, monsieur, que nous ne nous sommes vus. Cependant, je ne puis croire, en vous regardant, qu'il se soit passé vingt-cinq ans. Vous êtes toujours le même. La vie paisible que vous avez menée est aussi favorable à la santé qu'au bonheur. Je ne puis vous dire combien je vous envie la tranquillité dont vous jouissez."
"Assurément, milord," répondit Mr. Percy en souriant, personne ne pourroit croire, que parvenu comme vous l'êtes, au sommet de la fortune et des honneurs, vous puissiez envier l'humble situation d'un gentilhomme campagnard."\setcounter{page}{515} "Ah, Mr. Percy, je n'en suis pas encore là. Je suis loin d'avoir atteint ce point où vous me supposez; et quand j'y serois arrivé, n'est-ce pas alors que je devrois trembler? Le poste le plus éminent est aussi le plus périlleux." Un profond soupir accompagna ces paroles, et lord Oldborough se mit à traiter des sujets de politique générale sur lesquels Mr. Percy s'exprima avec sa franchise accoutumée, en conservant néanmoins les égards dus à la place qu'occupoit lord Oldborough. Celui-ci, sensible à cette attention, sourioit, approuvoit par un geste, un signe de tête, lorsque Mr. Percy parloit des hommes publics, et des mesures prises par le gouvernement, mais s'il exprimoit un sentiment énergique de patriotisme et de dévouement au bien, lord Oldborough, paroissoit distrait et prenoit du tabac. Si Mr. Percy énonçoit un principe favorable à la liberté et en particulier à celle de la presse, les prises de tabac se succédoient plus rapidement encore, parce que sa seigneurie vouloit que sa physionomie n'exprimât rien dont on pût conjecturer son opinion. Si Mr. Percy continuoit à traiter le même sujet, milord détournoit les yeux en serrant les lèvres, pour marquer sa désapprobation, et pour faire comprendre qu'il avoit de la peine\setcounter{page}{516} à s'empêcher de répondre. Il ne prononça pas un seul mot; et après un silence de quelques instans, il demanda en souriant à Mr. Percy comment on se portoit chez lui. Il tourna la conversation sur ses intérêts domestiques. Il témoigna son étonnement de ce qu'un homme doué de talens aussi distingués que Mr. Percy, vivoit dans la retraite. Il parot oublier ce qu'il venoit de dire sur les peines et les dangers de l'ambition; il mit de côté tout ce que Mr. Percy avoit dit lui-même de son goût pour la vie retirée; et s'exprima comme s'il eût été convenu que Mr. Percy auroit été bien aise de briller dans la carrière des affaires s'il en avoit eu la possibilité. En conséquence, lord Oldborough lui indiqua adroitement de quelle manière il pourroit parvenir; il mit en avant des assurances de bienveillance. Il le complimenta sur ses talens; enfin, il le sonda comme quelqu'un qui se croyoit certain de lui trouver de l'ambition au fond du cœur. Lorsqu'il fut bien convaincu qu'il n'y en avoit point, il fut véritablement surpris. Son opinion sur le jugement de Mr. Percy baissa considérablement. Il le regarda comme un être hors de rang, qui agissoit par des motifs inexplicables; mais il ne put se défendre d'avoir du respect pour lui en qualité d'homme, vraiment honnête et indépendant,\setcounter{page}{517} chose à laquelle un homme qui est dans la carrière politique ne croit guères. Il se flattoit aussi, que Mr. Percy avoit quelque considération pour sa personne, et comme cette considération étoit désintéressée, il en faisoit un cas particulier. D'ailleurs, Mr. Percy vivoit hors des affaires, et n'avoit aucune liaison avec les meneurs des partis. C'étoit un homme sûr, et qui ne pouvoit jamais se trouver sur son chemin. Tout cela fit une impression profonde sur lord Oldborough; et il mit de côté toute réserve. Ce changement surprit beaucoup Mr. Percy; mais il éprouva de tout cela un sentiment pénible. Il voyoit une ame noble, que l'ambition avoit corrompue, et chez laquelle les bons principes et les mouvemens généreux se trouvoient étouffés; il voyoit un esprit naturellement vigoureux et droit, que l'intérêt avoit rétréci et faussé. Un homme enfin, dont l'existence sembloit concentrée dans les plaisirs de la vanité. Lord Oldborough pénétra la pensée de Mr. Percy. "Vous me plaignez," lui dit-il, "je suis en effet à plaindre. Je sens le vide de tout ce que l'ambition peut donner, et je ne puis me soustraire à cette passion. J'ai besoin de ce mouvement de l'espérance. Ce jeu des intrigues politiques est devenu un besoin pour moi."\setcounter{page}{518} "Milord, un homme de caractère peut
se soustraire à un esclavage, quel qu'il soit."
"Oui, mais il faut qu'il en ait la vo
lonté."
"Est-il donc possible de perdre complé
tement."
Milord l'interrompit; "de perdre complé
tement l'amour de la liberté, vous voulez
dire? oui, très-possible. Vous me re
gardez en pitié," ajouta milord en se levant.
"Je ne suis pas accoutumé à exciter ce senti
ment là, et je ne puis pas dire que cela me soit
agréable."
Après un moment de silence,
la physionomie de lord Oldborough reprit
son expression diplomatique. La conversa
tion languit, et ils furent charmés de voir
rentrer le Commissaire et le Colonel.
\subsection{L'intrigant.}
Mr. Falconer étoit un politique à projets;
après avoir posé les fondemens de son édi
fice, il bâtissoit étages sur étages. Il avoit
réussi à placer son fils aîné; il s'occupoit de
mettre les deux autres dans le chemin de la
fortune. Il destinoit Buckhurst à l'Eglise, et
John à la carrière des armes. Quoique le
commissaire eût obtenu de lord Oldborough
une audience secrète de plusieurs heures,
et qu'il lui eût fait agréer son fils; pour se\setcounter{page}{519} crétaire, il sentoit bien qu'il n'avoit pas fait de grands progrès dans la faveur du noble lord; mais il s'en inquiétoit peu parce qu'il avoit en réserve des moyens de se rendre nécessaire, et d'assurer ainsi l'avancement de ses fils.
Cependant une autre avenue à la fortune s'ouvrit à lui tout-à-coup. Le neveu de lord Oldborough, le colonel Hauton étoit resté en l'absence de son oncle au Park Clermont pour être à portée des courses de chevaux qu'on annonçoit devoir être très-brillantes. Buckhurst avoit été camarade d'école du colonel et lui avoit souvent aidé à faire ses tâches, car celui-ci avoit toujours été un fort médiocre écolier et avoit fini par prendre une aversion décidée pour les livres. Le commissaire imagina de tirer parti de cette relation de son fils et il l'engagea à aller faire une visite au colonel, qui le reçut fort bien.
Il seroit difficile de peindre le colonel Hauton par des traits qui le fissent distinguer de tous les autres jeunes gens du même rang; excepté peut-être qu'il étoit plus arrondi que personne dans son égoïsme. Cependant cela ne perçoit point à la première vue, et on le croyoit généralement d'un naturel bon et sociable. Il avoit tellement besoin de trouver dans les autres les ressources qui lui manquoient en lui-même qu'il étoit bien obligé\setcounter{page}{520} de rechercher la société et de s'y rendre agréable. Le défaut absolu de dignité dans le caractère et dans les manières passoit chez lui pour de la bonhomie: son ambition étoit de ressembler à son cocher, mais il n'avoit pu réussir qu'à attraper la tournure de son palfrenier.
Quoiqu'il fit sa société habituelle des jockeys et des gens d'écurie, personne n'étoit plus hautain et plus despotique; mais ses égaux le contentoient et ne s'embarrassoient guères qu'il prît sa revanche sur ses inférieurs.
Son plaisir favori, ou plutôt sa grande affaire, étoit les courses. Buckhurst ne pouvoit pas aider le colonel dans cette partie comme dans les thèmes de latin, mais il s'informoit, il prenoit intérêt à la chose, il savoit admirer, et le colonel qui dans ce cas se sentoit l'homme instruit, faisoit l'entendu et jouissoit de sa supériorité. Il pressa Buckhurst de rester au Parc-Clermont et de l'accompagner aux courses. — Deux chevaux fameux devoient courir: le highblood du colonel, et le wildfire d'un gentilhomme nommé Burton.
Les jours qui précédèrent ce grand événement furent employés aux préparatifs des hommes et des chevaux. On y mit toute l'importance et toute la pédanterie imaginables. Aux heures fixées, le colonel et son ami assistoient à l'opération d'étriller, bonnets\setcounter{page}{521} chonner, brosser, laver, le fameux high-blood. Ils voyoient donner les rations, présidoient à l'exercice qui devoit le tenir en haleine, et s'assuroient que la litière étoit bien faite.
Après le cheval venoit le jokei, Jack-Gile. Il avoit aussi sa part des soins. On le pesoit, on lui faisoit faire diète, suer, boire de l'eau-de-vie, puis on le repesoit de nouveau; et ce qui étoit encore plus difficile, on le maintenoit en gaieté pendant toute la cure. Aussi le garçon d'écurie rendit-il témoignage à l'activité de ces messieurs, et il convint qu'il étoit impossible de mieux travailler qu'ils ne l'avoient fait pendant toute la semaine.
Enfin, arriva le grand jour, et Jack-Giles fut pesé pour la dernière fois, en présence de témoins ainsi que son rival Tom Hand. On amena les chevaux qui devoient faire la course, et les spectateurs pleins d'impatience, se pressèrent en foule, sur-tout ceux qui étoient engagés dans de fort paris pour ou contre. Le signal fut donné et on entendit répéter de tous côtés les noms des chevaux, des jokeis et des maîtres. Les amateurs à cheval se soulevoient sur leurs étriers pour mieux voir. Les dames montées sur l'échafaudage tendoient le cou pour être vues, et personne ne les regardoit. Deux homme\setcounter{page}{522} furent renversés et foulés aux pieds sans qu'on y prit garde; deux femmes s'évanouirent et leurs voisins continuèrent à parier: on n'avait pas le temps d'être poli. La victoire, d'abord douteuse, fut remportée par Mr. Burton. Ses amis l'entourèrent avec de bruyantes acclamations. Tom Hand et Wildfire furent portés aux nues.
Le colonel, dans son dépit, lâcha un mot qui indiquoit un soupçon de fraude. Mr. Burton n'étoit pas d'humeur de supporter cette insinuation contre son jokey, et encore moins, contre lui-même. Il jura qu'il n'y avoit pas en Angleterre un plus honnête jokey que Tom Hand, et qu'il le soutiendroit contre qui voudroit l'attaquer. Le colonel, comptant sur son rang, sur la place de son oncle, sur ses amis qui l'entouroient, se permit de traiter le squire trop légèrement; et celui-ci lui répondit avec une plaisanterie un peu grossière. Le colonel qui n'avait pas la raillerie à son commandement, se mit en fureur quand il vit que son antagoniste rioit, et faisoit rire les autres. Il devenoit insolent. Alors Mr. Burton se fâcha tout de bon: il dit qu'il ne souffriroit aucune impertinence ni du neveu d'un lord, ni d'un fat de colonel; et tout en parlant d'un ton haut et menaçant, il se mit à agiter le fouët qu'il tenoit à la main. Dans ce moment\setcounter{page}{523} critique, on assuré que le colonel fit quelques pas en arrière. D'autres prétendent que le jeune Falconer se jetta en avant de lui, sauta sur Burton et en arrêtant le coup qui étoit destiné à un autre, lui arracha le fouët de la main. Il s'en suivit un duel après lequel il reparut en public pour recevoir les félicitations des dames; et il fut ensuite complimenté par son père sur cet incident comme le plus heureux qui pût lui arriver, "mon ami," lui dit-il, "c'est un coup de partie! lord Odborough regarde le colonel comme son héritier; et qui a l'oreille du neveu & le cœur de l'oncle."
: "Le cœur n'a rien à faire là, mon père,..
: "A la bonne heure, mais ce qu'il y a à faire, c'est un bon bénéfice."
: "Comment donc, un bon bénéfice?"
: "Mais, voyons! seroit-ce par hasard pour recevoir des coups de fouët, ou une balle au travers du corps, que tu t'es mis en avant de cette manière?"
: "Ma foi! c'est par étourderie. Je me suis mis en avant, je ne saurois vous dire, moi ; mais vous comprenez que je ne pouvais pas faire autrement. J'étois avec lui. Il a hésité, et moi j'ai agi. J'avais parié pour lui, je ne pouvois pas l'abandonner. Quand on est du parti de quelqu'un on ne peut pas lui voir\setcounter{page}{524} donner des coups de fouet tout tranquillement."
" Allons! voilà qui est fort généreux; mais comme tu es un peu trop âgé maintenant pour agir en écolier, il faut voir comment nous pouvons tirer parti de cela."
" Je n'ai jamais pensé à en tirer parti."
" Mais considère donc l'importance de la chose. Voilà un colonel au service de Sa Majesté, à qui tu as épargné des coups de fouet en public. Tu as sauvé toute la famille de la honte. Ce sont des choses qu'on ne peut pas oublier: c'est-à-dire, quand on force les gens à s'en souvenir; car je connois le monde moi: si on les laissoit faire, dans huit jours ils n'y penseroient plus. Je me charge de suivre la chose avec l'oncle: tu la soigneras avec le neveu. Il faut de temps en temps lui rappeler cela délicatement. Je m'en fie à ton jugement et à ton adresse."
" Oh, ne vous fiez ni à mon jugement, ni à mon adresse, car je n'ai ni de l'un ni de l'autre, pour ces choses là. J'agis de première impulsion: bien ou mal, comme cela se rencontre. La finesse n'est pas mon fort à moi. Je laisse cela à Cunningham: c'est son métier. Il a du génie pour cela: on dit qu'il ne faut pas qu'il y ait deux génies dans la même famille."
\setcounter{page}{525} "C'est assez plaisanté : Parlons sérieusement, si tu ne veux pas employer ton adresse et ton bon sens à t'avancer, je t'avancerai : laisse-moi faire. Tu vois comme j'ai poussé ton frère. Sois tranquille ; je te pousserai bien aussi."
"S'il ne faut qu'être tranquille, je serai bien tranquille, je vous le promets. Ce n'est pas que je ne fusse bien aise d'avoir une bonne place ; mais je ne veux rien faire de sâle pour l'obtenir. ,,
"Cela va sans dire ; mais je vois des gens qui n'ont ni mérite ni relations, et qui par faveur ont obtenu des bénéfices beaucoup meilleurs que ce que tu as en vue."
"Moi ! je n'ai rien en vue du tout."
"Comment donc! le Bénéfice de Chipping-Friars, qui est à la nomination du colonel, et dont le desservant a eu une attaque d'apopléxie ? Tu n'as pas cela en vue ? ,,
"A vous dire le vrai, mon père, je n'ai pas du goût pour l'Eglise. ,,
"Il ne s'agit pas d'Eglise ; il s'agit de neuf cent livres sterling de rente, Monsieur le philosophe, entendez-vous ?"
"Ah ! j'aurois du goût pour un pareil revenu, par exemple ; mais n'y a-t-il donc pas moyen d'avoir le revenu autrement , que par l'Eglise?"
Littérature. Vol. 56. No. 4. Août 1814.
N n\setcounter{page}{526} ,, Non pas pour toi. J'ai bien réfléchi et combiné la chose. Il faut prendre les ordres; il n'y a pas d'autre parti. Je vais en écrire à mon ami l'évêque dès ce soir. ,,
"Je vous prie , mon père , de ne pas trop vous presser ; car décidément je n'ai pas du goût pour l'Église. ,,
Mr. Falconer ouvrit de grands yeux , et lui dit : " qu'est-ce que cela signifie donc? n'as-tu pas passé comme un autre , à l'université?"
"A la bonne heure , mais... ... je ne suis pas ... ... comment dirai-je ? je ne suis pas assez moral ; je suis ... ... je suis trop crâne , trop étourdi pour l'Église. Je ne voudrais pas pour rien au monde entrer dans l'Église et ne pas faire honneur à l'état que j'aurais embrassé."
"Et pourquoi ne ferais-tu pas honneur à l'Église ? Je ne vois pas que tu sois plus mauvais sujet que tant d'autres. Par exemple D. , à ton âge était un franc étourdi , et bien il a mordu à la chose : le voilà devenu un doyen gros et gras. Qui dirai-je encore ? L'évêque qu'on vient de nommer. Il était tout comme toi étant jeune. Allons , allons ! il ne faut pas y faire tant de façons. Ne me parles donc plus de tes scrupules. ," ... ...\setcounter{page}{527} \subsection{La coquette}
Le château de Hungerford était un bel et ancien édifice au milieu d'un vaste parc. Les maîtres de la maison avaient souvent leurs amis à demeure. Ceux-ci y trouvaient une réception cordiale et une parfaite liberté. Il y avait toujours des appartemens prêts à recevoir les intimes de la maison, et qui portoient leurs noms. La famille Percy comptoit parmi les plus désirés et les mieux accueillis.
En arrivant ils s'aperçurent à l'empressement des anciens domestiques qu'on avait du plaisir à les voir et qu'on s'était occupé d'eux. Cependant aucune affectation d'égards ne leur rappela qu'il s'était fait un changement dans leur fortune, il n'y avait ni plus ni moins qu'à l'ordinaire; mais tout était d'accord avec les anciennes habitudes de confiance et d'amitié.
Lady Angélica Headingham faisait partie de la société rassemblée au château. Elle venoît d'hériter d'une grande fortune et jouait un rôle brillant dans le monde. Pendant son enfance elle avait été soumise à une gêne excessive. Aussi jouissait-elle bien complétement d'une liberté nouvellement acquise. Elle parut tout-à-coup sur la scène comme une beauté.\setcounter{page}{528} un bel esprit, comme la protectrice des gens à talens; et quoiqu'elle n'eût fait aucune étude de ces différens rôles, elle s'en acquittôit avec une confiance si parfaite qu'on eût dit qu'elle y avoit été élevée.
A un certain égard cependant, elle n'avoit pas atteint le plus haut point d'élégance: elle ne paroîssoit pas assez indifférente à la louange. A mesure qu'elle recevoit plus d'enbens, elle s'en montroit plus avide; et pour mieux en savourer le parfum elle oublioit quelque fois l'attitude superbe qui convient à une des divinités de la mode.
Son cœur ne paroîssoit pas avoir été touché des hommages d'aucun de ses admirateurs, mais elle donnoit à tous ce qu'il falloit d'espérance pour leur tourner la tête. Deux des plus persévérans, Sir James Harcourt, et Mr. Barclay l'avoient suivie au château de Hungerford.
Le premier avoit une fort belle figure. C'étoit un homme à la mode, il avoit beaucoup d'usage du monde et le ton de la cour. Il s'étoit ruiné en briguant inutilement dans deux élections de Comtés. Le ministre lui avoit fait espérer d'année en année une place ou une pension; mais comme ses créanciers s'étoient lassés d'attendre, il avoit pris le parti de devenir amoureux de lady Angélica.\setcounter{page}{529} L'autre admirateur étoit un homme de beaucoup de sens. Il jouissoit d'une excellente réputation. Il avoit un bon nom et une grande fortune. Il étoit arrivé à l'âge où il désiroit jouir tranquillement du bonheur domestique; mais il avoit vu une si grande masse de misères résulter des mauvais mariages dans le cercle de ses connoissances, qu'il craignoit de former un attachement, ou du moins un engagement quelconque. La connoissance qu'il avoit des mœurs du grand monde lui donnoit également la crainte de s'établir; et il s'étoit très-bien défendu de toute inclination jusqu'au malheureux moment où il avoit rencontré lady Angélica. Sa raison et son jugement avoient en vain résisté à l'ascendant de ses charmes. Il avoit cependant eu la prudence de ne faire aucun aveu; mais elle en savoit plus que lui sur son sentiment, elle le tourmentoit de sa coquetterie. Il supportoit tout avec la patience d'un martyr.
Lorsque la famille Percy vit lady Angélica pour la première fois, elle étoit dans toute sa gloire; sa toilette avoit bien réussi; elle avoit une parfaite sécurité sur l'effet qu'elle produisoit, et un renouvellement d'ambition pour enchanter des gens avec lesquels elle alloit faire connoissance. Quoiqu'elle eût\setcounter{page}{530} passé la fleur de la jeunesse ; elle avait encore une beauté éclatante , et l'expression d'une personne qui exige l'admiration parce qu'elle y est accoutumée ; ses gestes , ses attitudes , ses regards , ses sourires étaient toujours un peu en exagération , et laissaient percer le projet. Elle était entourée d'hommes qui faisaient la conversation avec elle : le chevalier Harecourt , Mr. Barclay , Mr. Seebright, jeune poète , Mr. Gray , un savant , et de quelques personnages muets qui l'écoutaient. Elle soutenait l'intérêt , et les amusait tous parfaitement , quoique ce fut le moment de la journée le plus critique , c'est-à-dire la demi-heure qui précède le dîner. Les beautés sont curieuses des beautés , comme les gens d'esprit le sont de leurs pareils. Lady Angélica avait ouï dire qu'une des Dlles. Percy était fort belle. Son regard passa avec la promptitude de l'éclair sur Mad. Percy et Rosomande qui entrèrent les premières , et se fixa sur Caroline. "Belle en effet , se dit-elle avec inquiétude , mais bien simple ! , et elle se rassura. Caroline n'avait aucune grace d'après, aucun sourire de commande : elle entra dans le salon comme elle serait entrée dans sa chambre. Les dames Pembroke allèrent au-devant d'elle et la firent asseoir entre elles deux. Sa physionomie exprimait ce qu'elle sentait, c'est-à-dire , le plaisir de les revoir.\setcounter{page}{531} "Jolie créature!," dit tout bas lady Angélica au chevalier Harcourt, qui avançait la tête pour savoir ce qu'elle en disait.
"Mais......oui, elle est bien: rien de très remarquable cependant."
Le salon était si grand que lady Angélica pouvait causer sans être entendue de Caroline qui faisait la conversation de son côté.
"Il y a quelque chose de bien intéressant dans cette simplicité;" dit-elle à Mr. Barclay. "Ne trouvez-vous pas que la simplicité a un charme inconcevable? C'est bien dommage que cela ne puisse pas durer. C'est un peu comme ces couleurs délicates qui plaisent au premier moment, et qui semblent s'affadir pendant qu'on les regarde: j'y ai été prise vingt fois."
Pendant qu'elle parlait, Mr. Barclay fixait Caroline, et il répondit: "C'est bien dommage que cela passe!"
"Infiniment dommage!," reprit-elle. A la campagne, si l'on y mettait beaucoup de soin, cela pourrait durer quelques années, peut-être; mais à Londres, c'est l'affaire d'un hiver."
"Ce que vous dites-là est trop vrai," reprit Mr. Barclay.
"Aussi, je dis moi," poursuivit lady Angélica, "qu'il vaut mieux avoir ces choses qui\setcounter{page}{532} restent : par exemple, d'être une personne à la mode, ou avoir des connoissances, ou de l'esprit, ou du génie. Toutes ces choses-là ont leurs inconvéniens; je le sais, mais cela vaut mieux qu'un souvenir de fraîcheur et de simplicité."
"Quand c'est passé c'est déplorable, il n'y a pas de doute :" dit Mr. Barclay. Sir James serra les épaules et déclara que pour lui il n'aimoit pas beaucoup la simplicité, parce qu'il la trouvoit fade. . . . . .
Les demoiselles Percy passoient une partie des matinées, à dessiner avec les nièces de Mad. Hungerford, lady Mary Pembroke et sa sœur Mad. Mortimer. Elles avoient leur établissement dans un cabinet atténant à la bibliothèque, et souvent elles joignoient l'amusement de la lecture ou de la musique à celui du dessin.
Quant à lady Angélica, elle ne savoit se fixer à aucune occupation suivie ; et elle trouvoit souvent le temps long malgré la fertilité de ses ressources de coquetterie, pour attirer l'attention, et varier l'effet de ses grâces. Elle se joignoit quelquefois aux promenades des jeunes personnes pourvu que les hommes en fussent.
L'enthousiasme pour les beautés de la nature est à la mode aujourd'hui. Lady An-\setcounter{page}{533} gélica savoit jeter dans la conversation quelques phrases bien composées sur les accidens de lumière, sur les contours hardis des montagnes, sur les teintes riches des bois. Caroline au contraire jouissoit en silence, et n'avoit pas besoin de faire admirer son admiration d'un beau paysage pour trouver du plaisir à le contempler.
Mistriss Mortimer faisoit planter des bosquets, et chacun donnoit son avis. Caroline y mettoit un véritable intérêt et montroit beaucoup de goût pour l'arrangement des jardins. Lady Angélica saisisssoit l'occasion de parler des belles maisons de campagne de ses amis; mais au moment où elle n'occupoit plus exclusivement l'attention de la compagnie, elle ne manquoit pas de se plaindre du chaud, du froid, de la fatigue, et de ramener les promeneurs au château.
Un jour que les jeunes dames étoient établies à dessiner, Sir James entra dans le cabinet pour chercher un livre. Il s'approcha de Caroline et parut surpris de la beauté de son travail. Mad. Hungerford, qui étoit présente, dit à Sir James : "miss Percy ne fait point parade de ses talens, et sa modestie ajoute un prix infini à tous les avantages qui la distinguent. Lady Angélica arriva.\setcounter{page}{534} vant dans cet instant à la porte de la bibliotheque, entendit cet éloge de Caroline, et pour couper court à un sujet de conversation qui lui déplaisoit, elle poussa un cri plaintif, en disant qu’elle avoit une crampe au pied. "Appuyez-vous sur moi," s’écria Mr. Barclay, "et posez doucement votre pied à terre."
"Impossible," répondit lady Angélica, "je suis au supplice. Sir James, venez donc. Portez-moi tous deux sur le sopha, je ne puis me soutenir." Milady cependant se trouva bientôt soulagée et arrangea son joli pied sur des coussins avec toute la grace imaginable. Si elle avoit été susceptible de quelque sensibilité, elle auroit été reconnoissante de l’intérêt vif que Mr. Barclay avoit témoigné en la voyant souffrir; mais sa vanité seule étoit excitée, et elle prit ce moment pour s’amuser à tourmenter le seul homme qui l’aimoit véritablement, en réveillant sa jalousie. Elle fit mille coquetteries à Sir James; elle lui permit de rattacher les cordons de son soulier. Elle prétendit être assez bien remise pour faire un tour dans le parc, et envoya Sir James lui chercher son schall; mais lorsque Mr. Barclay lui offrit de l’accompagner et qu’elle vit les dames se disposer à la suivre, elle renonça à son projet, prit le schall des\setcounter{page}{535} mains de Sir James, se drapa avec prétention, et lui demanda s'il n'avoit jamais vu milady Hamilton jouer la pantomime tragique. Sir James répondit qu'oui, mais qu'il ne croiroit point avoir l'idée de la perfection dans ce genre, jusqu'à ce que lady Angélica eût consenti à la lui donner.
Personne n'appuya la demande de Sir James. Mr. Barclay parut sérieux; mais lady Angélica oubliant l'observation qu'on venoit de faire sur l'étalage des talens, ne songea qu'à complaire à la fantaisie de Sir James. Elle passa dans la chambre voisine, et reparoissant un moment après, elle exprima successivement, par ses attitudes, ses gestes et sa physionomie, la crainte, l'espérance, l'amour et la jalousie. Elle réussissoit à merveille. Les applaudissemens des spectateurs l'encourageoient, et l'émotion alloit croissant, lorsqu'un incident vint nuire à l'illusion de la scène. En s'enveloppant dans le schall pour jouer la mélancolie, lady Angélica détacha une fausse tresse des plus beaux cheveux châtains, et c'en étoit fait peut-être de sa charmante coëffure grecque, lorsque Caroline, prévoyant le danger, se hâta d'aller au secours de la belle actrice, et prévint, avec beaucoup d'adresse, la petite humiliation dont elle étoit menacée.\setcounter{page}{536} Caroline reçut un léger remerciement de lady Angélica, mais elle se fit une bien bonne note dans l’esprit de ceux des spectateurs qui comprirent son motif, et en particulier de Mr. Barclay.
Pour lady Angélica, elle n’étoit occupée que de ses projets sur Barclay, et elle s’applaudissoit du succès de son petit manège pour l’inquiéter et le rendre jaloux de Sir James. Dans l’enchantement de ce qu’elle croyoit un triomphe, elle se pencha vers Mad. Hungerford et lui dit tout bas. " Le pauvre homme est pétrifié: il voit bien que ce n’est pas pour lui que j’ai pris la peine de faire mes attitudes, mais un sourire me le ramènera quand je voudrai." Mad. Hungerford secoua la tête en signe de doute et de désapprobation. Cependant les bravo de Sir James retentissoient encore aux oreilles de lady Angélica et l’empêchèrent de faire attention à cette réponse muette. Bientôt la société se sépara, car c’étoit l’heure de la toilette. Lady Angélica, fatiguée de son rôle resta la dernière au salon avec Mad. Hungerford.
" Je n’en puis plus," dit-elle en se jetant sur un sopha. " Mes forces ne comportent pas la vivacité que je mets à tout. Vous trouvez peut-être que j’ai trop d’abandon, trop\setcounter{page}{537} d'entraînement, même quelquefois un grain de folie; mais que voulez-vous, on ne peut pas se refondre. "
" Ma chère Angélica, " répondit Mad. Hungerford d'un ton grave, mais affectueux. " J'ai été l'amie intime de votre mère, je voudrois être aussi la vôtre. Mon âge et mon expérience me placent à votre égard d'une manière qui me permet plus de liberté avec vous que personne ici n'en pourroit avoir. C'est ce qui me donne le courage de vous dire la vérité, car je déplore sincèrement qu'une personne comme vous se laisse gâter par la flatterie. "
" Eh, mon Dieu! ma chère madame, vous êtes bien bonne assurément. Je suis prête à vous entendre faire la revue de tous mes défauts; mais pourtant, que je vous dise une chose avant que vous preniez la peine de commencer; vous allez me dire que je suis coquette, inconséquente, étourdie; eh bien, oui, j'ai tous ces torts-là, j'en conviens, mais je ne sais qu'y faire. Et si l'on me trouve bizarre, je répondrai que j'aime mille fois mieux être bizarre qu'insipide. De tous les caractères, celui que j'estime le moins, c'est un caractère commun. Je ne voudrois pour rien au monde passer pour une personne ordinaire. "\setcounter{page}{538} "Mais alors, ma chère, vous n’êtes pas d’accord avec vous-même, car la coquetterie est une chose très commune dans le monde. Pas un roman, pas une pièce de théâtre, qui n’en présente le modèle. Tous les petits airs, toutes les affectations ont été jouées sur la scène. Croyez-moi, ayez une ambition plus relevée et plus digne de vous. Faites plus de cas de votre propre estime que des louanges vulgaires. Avec votre beauté et tous vos moyens de plaire et d’obtenir l’admiration, vous ne risquerez jamais d’être une personne ordinaire, si vous n’imitez rien et que vous soyez toujours vous-même."
"Imiter," s’écria lady Angélica, "ah! pour cela, je vous arrête. Parmi tous mes défauts, celui-là au moins n’est pas du nombre, et si quelqu’un peut prétendre à l’originalité, je crois que j’y ai quelques titres. Connoîtriez-vous quelqu’un, par hasard, à qui l’expression d’originalité pût être mieux appliquée ?"
"Mais... peut-être n’irois-je pas bien loin," dit Mad. Hungerford, "pour trouver une femme qui, sans jamais chercher à se distinguer, montre toujours un caractère supérieur, un caractère à elle. Réunir la raison à la sensibilité, le bon sens à l’esprit, est un mérite assez rare pour pouvoir être appelé original."\setcounter{page}{539} "Oh, je vois bien de qui vous voulez parler. C'est de votre petite amie Caroline dont vous êtes si entichée. Je ne lui dispute point ses avantages. Elle est régulièrement belle; c'est un genre peu séduisant. Elle est fort instruite, je veux le croire. Elle paraît avoir de la simplicité, de la droiture; mais attendez que cela ait passé à l'épreuve. Un hiver à Londres peut produire d'étranges changements. Je sais ce que c'est moi. Il faut avoir été enivrée de louanges, avoir vu chez tous les hommes l'effet de l'admiration, chez toutes les femmes celui de l'envie, pour comprendre combien il est difficile de résister à cette influence."
"Si cette preuve n'est pas achevée pour Caroline, dit Mad. H. en souriant, du moins est-elle commencée, et vous avez déjà sous les yeux des objets d'observation qui peuvent vous satisfaire."
Mad. Hungerford se tût et lady Angélica garda le silence et pendant quelques moments parut assez agitée. Enfin elle dit à Mad. H. "Croyez-vous que Mr. Barclay m'aime réellement!"
"Je crois, répondit Mad. H. qu'il vous étoit très-sincèrement attaché, mais vous avez fait tout ce qui dépendoit de vous pour\setcounter{page}{540} affaiblir, et peut-être détruire son sentiment.
"Pardonnez-moi des vérités dures; mais qui ont votre bonheur futur pour objet. Je crains......."
"Ah, ne craignez rien, ma chère Madame, c'est mon affaire. Dites-moi seulement que vous croyez à sa sincérité. Je ne suis pas en peine du reste. Je sais si bien comment il faut s'y prendre avec ces pauvres hommes. C'est une petite guerre dont je connois toutes les finesses."
- Mad. Hungerford soupira et dit, "c'est un jeu où de plus habiles que vous, ma chère, ont perdu la partie sans retour. Souvenez-vous de cet avertissement et puiss t-il vous épargner des chagrins."
L'arrivée de miss Caroline interrompit la conversation, et lady Angélica alla faire sa toilette.
\section{CORRESPONDANCE.}
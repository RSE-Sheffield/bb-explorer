\setcounter{page}{112}
\chapter{ROMANS.}
\section{FRAGMENS du PATRONAGE, par MARIA EDGEWORTH. \large{(Troisième extrait. Voy. p. 512 du vol. préc.)}}
(ALFRED PERCY invité chez Mad. Faisalconer à un grand concert, arrive avant que l'assemblée soit formée : il est introduit dans le salon où l'on doit faire de la musique ). Miss Georgiana étoit assise au piano, entourée de musiciens , soit artistes ; soit compositeurs ; et des amateurs, qui devoient jouer un rôle dans la soirée. Il s'agissoit d'arranger d'avance l'ordre dans lequel les différens morceaux devoient se suivre. La consultation\setcounter{page}{113} tion des plus sérieuses, et à peine Alfred obtint-il un regard distrait de miss Falconet et un mot de politesse froide. Ensuite il n’entendit plus qu’un mélange confus de mots techniques, de noms d’acteurs et de titres d’opéras, qui lui fit sentir sa profonde ignorance musicale. Bientôt cependant, au milieu de ce cahos, il trouva matière à exercer son esprit d’observation. Les bonnes amies des demoiselles Falconer montroient autant de talent pour flatter celles-ci que pour médire des absentes, sur-tout de celles qui devoient figurer sur la scène lyrique.
"Nous aurons donc cette éternelle miss Bings, avec sa voix fêlée et ses antiques ariettes à passages," disoit l’une.
"Oh! ce n’est rien," répondoit une autre, "en comparaison de la petite miss Crotch. Elle a plus de voix qu’il n’en faut pour donner envie de se boucher les oreilles quand elle chante. Je ne conçois pas comment il y a des gens qui l’admirent. Elle est bonne musicienne; mais elle n’a pas l’ombre de goût. Quant aux demoiselles Barhams, c’est encore pire. L’aînée n’a pas plus d’oreille que cette table, la cadette n’a qu’un filet de voix, et bien aigre, par dessus le marché."\setcounter{page}{114} "Maman," dit Georgiana." Les dames Le Grand doivent-elles venir?" " Certainement, ma chère, vous savez bien que je n'ai pas pu éviter de leur proposer." " Cela étant," reprit Georgiana, en rougissant de dépit," je déclare que je ne chanterai pas." " Pourquoi donc, ma chère? assurément vous ne devez pas avoir peur des dames Le Grand." " Ah! par exemple," dit une des complaisantes de miss Georgiana." Vous leur feriez beau jeu si vous leur cédiez le terrain. Il ne tient qu'à vous de les écraser complètement: je vous en donne ma parole." " Il faut pourtant convenir," reprit Mad. Falconer, " que mesdames Le Grand chantent bien, mais leur genre n'est pas généralement goûté, et puis elles ne sont point musiciennes, c'est une chose reconnue." " A la bonne heure," répondit Georgiana, " mais je déteste faire de la musique avec elles. Elles ont la fureur de faire les seconds dessus, et c'est la partie qui convient le mieux à ma voix ; d'ailleurs, vous savez qu'elles ont inventé de chanter constamment mon air favori O giove omnipotente. Je le leur pardonnerois encore, mais elles m'ont\setcounter{page}{115} aussi volé Quanto e amor possente. Rien n'est odieux comme d'entendre ainsi gâter les airs qu'on aime. "
"Chut," dit Mad. Falconer, "voici Mad. Le Grand. "
"Ah! madame, que j'ai de plaisir à vous voir! c'est vraiment bien aimable à vous de venir chez moi. Comment va donc cette pauvre tête? et miss Le Grand, et la charmante Elisa? ah! les voilà! il nous les faut absolument: point de succès sans elles pour notre concert.
Après les dames Le Grand, arrivèrent miss Crotch et les demoiselles Barhams. Mad. Falconer les reçut avec les mêmes exclamations de joie; mais à ces complimens de commande succéda un silence et une réserve glacée parmi les jeunes personnes. Il étoit évident qu'elles se considéroient réciproquement comme rivales et que leur politesse étoit forcée.
Enfin les hommes arrivèrent. Le comte Altenberg parut. Il n'avoit pas été dans les coulisses, comme Alfred Percy, et tout lui parut charmant. Miss Georgiana se montra polie, animée, pleine de grace. Elle étoit mise avec un goût parfait, et sa mère étoit là pour faire valoir les moindres choses. \setcounter{page}{116} pour réparer les petites inadvertances, pour surveiller les premières impressions, et tendre ses filets en conséquence.
Le concert commença; mais qui pourrait décrire les anxiétés des mères rivales, la jalousie mal déguisée; les éloges destinés à faire ressortir un défaut, les complimens à haute voix, les critiques dites à l'oreille, et tout ce manège de fausseté dont personne n'est dupe et dont on fait cependant un échange continuel dans la société!
Mad. Falconer déployoit des qualités supérieures dans ce genre, sa physionomie ne laissoit jamais rien percer de ce qu'elle avoit intention de dissimuler. Elle faisoit les honneurs de chez elle avec une aisance inimitable, et tout en ne perdant jamais de vue son objet, elle avoit l'air de ne s'occuper qu'à faire briller les talens subordonnés des miss Crotch et des miss Barhams.
De leur côté, les jeunes personnes mettoient en usage tous les petits moyens connus de se contrarier réciproquement, de déjouer les petits projets, et de tourmenter la maîtresse de la maison. Celles qui étoient sûres d'être pressées de chanter assoient qu'elles n'étoient point en voix, et faisoient une résistance convenable. L'une se plaignoit qu'elle n'étoit pas accoutumée à l'instrument, l'autre,\setcounter{page}{117} qu'elle avoit la vue trop basse pour lire la musique sur le pupitre. Une troisième, que les bougies étoient mal placées. Il falloit pour les contenter ouvrir ou fermer tour-à-tour les fenêtres et les portes. Mad. Falconer se prêtoit à tout avec une complaisance parfaite, ne faisoit d'objection à rien, et cependant, manœuvroit avec tant d'adresse, qu'en ayant l'air d'entrer constamment dans les fantaisies des autres, elle ne faisait au fond que ce qui lui convenoit. Tandis que son mari tenoit le comte Altenberg engagé dans une discussion politique, elle eut soin de faire chanter les demoiselles Le Grand. Ensuite, pour laisser reposer l'admiration des auditeurs, elle mit sa fille Arabelle au piano, et fit passer les morceaux les plus insignifians, tout en meublant la scène par des politesses placées à propos aux parens et aux intéressés. Enfin, s'apercevant que le commissaire étoit au bout de ses moyens de conversation, elle invita le Comte à se rapprocher de l'orchestre, le plaignit beaucoup d'avoir manqué les bons momens; mais elle n'osoit pas abuser de la complaisance de ces dames, en leur redemandant de chanter avant qu'elles fussent reposées." Allons! Georgiana," dit-elle à sa fille, pour remplir bien ou mal\setcounter{page}{118} un intervalle, voyez ce que vous pourriez nous donner. Il faut mettre tout amour-propre de côté pour chanter après ces dames, je le sens bien ; mais enfin, faute de quelque chose de mieux, essayez *Giove onnipotente*."
" Oh! maman, c'est impossible. L'entreprise est au-dessus de mes forces. Priez miss Le Grand : elle le chante à ravir. "
" Cela est vrai, ma chère, mais il seroit trop indiscret de presser miss Le Grand de recommencer. Je comprends votre défiance de vous-même après ce que nous venons d'entendre; mais vous montrerez au moins votre bonne volonté, et je suis sûre que Mr. le Comte est indulgent. "
Georgiana se laissa gagner, et avec une charmante timidité, elle commença par *Giove onnipotente*. Peu-à-peu elle prit du courage, elle chanta aussi *O quanto e amor possente*. Jamais elle n'avoit été mieux en voix. Le Comte aimoit passionnément la musique : il fut enchanté. Pour ne point laisser refroidir cette impression, Mad. Falconer fit cesser la musique et on annonça le souper. Elle feignit de vouloir placer le Comte entr'elle et son fils John, mais au moment de s'asseoir, elle appela Georgiana, en lui disant : " mettez-vous\setcounter{page}{119} ici, ma fille. Vous savez que je ne soupe jamais, et je ferai mieux les honneurs en restant en-dehors. En effet, elle continua à surveiller les opérations du service; tout en distribuant les mots obligeans, les sourires, les complimens à droite et à gauche, et chaque fois qu'elle jetoit les yeux vers le Comte, elle s'applandissoit du succès de ses opérations. Il avoit l'air très-empressé, Georgiana étoit en beauté, la conversation ne tarissoit pas; tout alloit à merveilles. "
Alfred Percy, en revanche, se retira fort mécontent de l'emploi de sa soirée. Il avoit applaudi tout en bâillant, ainsi que beaucoup d'autres; et il ne se consoloit point de n'avoir pas pu entretenir un moment le Comte Altenberg, comme il s'en étoit flatté, en allant chez le commissaire. Le lendemain il écrivit à Rosamonde une lettre où la famille Falconer n'étoit pas peinte sous son plus beau jour.
\subsection{Portraits}
( Les Falconer projettent de monter un théâtre dans une maison de campagne, qui est peu éloignée de celle qu'habite la famille Percy. On forme une troupe d'acteurs. Alfred\setcounter{page}{120}  Percy écrit à ses sœurs, qui sont ceux qui doivent la composer, ainsi que la société invitée aux représentations).
"Les deux dames Arlington," disait Alfred, "sont sœurs. Elles ne demandent pas mieux que d'étendre leurs relations, et Mad. Falconer est charmée de les avoir, parce qu'elles sont parentes de milord Duc.
Je les ai souvent rencontrées, soit chez les Falconer, soit chez lady Angelica, et souvent aussi chez lady Jane Granville. Lady Anne affecte la langueur, et lady Francis, la vivacité : toutes deux sont de grandes dames fort enfants gâtés. La personnalité de lady Anne est passive et paresseuse, celle de lady Francis est active et hostile. Elle n'aime rien que par accès. On est une délicieuse créature, on est un ange tant qu'on fait toutes ses fantaisies. Ses caprices se succèdent avec tant de rapidité, et elle a tant d'inquiétude et d'impatience, qu'auprès d'elle, sa sœur a l'air tout-à-fait éteinte. Celle-ci ne se soucie de rien, n'aime rien. Elle se pique d'un superbe dédain pour tout ce qui l'entoure. Elle semble n'être dans ce monde que pour boire de l'eau d'orge ; et pourvu qu'elle évite constamment d'agir et de penser, elle est heureuse.—En voilà assez sur les dames Arlington."\setcounter{page}{121} Buckhorst Falconer ne peut pas être de la partie, le colonel Hauton le tient à son régiment, mais ses deux amis, MM. Clay sont engagés dans la troupe. Malgré les belles maximes de morale que Mad. Falconer a débitées sur le scandale que l'un d'eux vient de donner au public, elle les invite chez elle, et j'ai dans l'esprit que si elle manque son coup avec le Comte, elle se retournera sur l'un ou l'autre des frères pour en faire son gendre. Je me représente ce que diroit ma mère s'il étoit question de vous les donner pour maris, et je vais m'amuser à vous les peindre.
Clay l'Anglais, et Clay le Français, comme on les appelle ordinairement, ont hérité d'une grande fortune gagnée dans le commerce, et qu'ils prodiguent à l'envi avec une inconcevable extravagance, non pour satisfaire de véritables goûts; mais pour être reçus dans la première société. Clay le Français n'a rapporté de ses voyages qu'une fatuité ridicule. Il parle sans cesse de lui et de ses avantures sur le continent. Voici comment il amène une anecdote. "Quand je voyageais avec la princesse Orbitella. Quand j'étais à Paris chez la marquise de ***, etc., etc..... Au reste," je ne suis peut-être\setcounter{page}{122} pas tout-à-fait juste à son égard, parce que j'ai une antipathie décidée pour les Anglais qui ont la sottise de singer les Français. L'imitation est toujours si gauche, si ridicule, si méprisable. Clay parle sans cesse de tact, et il en manque absolument, il se pique de manier la plaisanterie; il croit avoir rapporté de Paris le véritable ton du persiflage, mais au fait, il n'a qu'une habitude de moquerie vulgaire, sans distinction et sans nuances. Il nous reproche beaucoup le manque de savoir vivre, tandis que lui-même est continuellement en défaut sur les plus simples égards de politesse. Il n'a dans ses manières ni la cordialité anglaise, ni le fini que donne l'éducation en France. Pour nous donner ce qu'il appelle l'esprit de société, il voudroit introduire chez nous le système de la galanterie française. Il affiche même les principes les plus relâchés sur ce point, et je lui ai entendu dire qu'il valoit toujours la peine de s'attacher à une femme mariée, pourvu qu'elle eût un des avantages requis en pareil cas: savoir, de l'esprit, de l'élégance, de la beauté, ou seulement une bonne table. — Il assure que son dernier procès pour cause de divorce ne lui a coûté que dix mille livres sterling, et que\setcounter{page}{123} c'est une bagatelle pour lui. Si l'on prononce le mot de vertu publique ou privée, il fait une mine de dédain. Le patriotisme, l'amour de son pays sont pour lui des chimères, des préjugés dont la philosophie doit nous détromper.
Il y a des gens charitables qui prétendent justifier Clay, en disant qu'il vaut mieux qu'il ne paroît; que ses vices sont empruntés, et ne tiennent pas à un mauvais caractère. Soit, je ne décide pas s'ils en sont moins haïssables.
Clay l'Anglais pousse l'affectation aussi loin que son frère, mais dans un autre genre. Il est froid, orgueilleux, réservé, ennuyé et ennuyeux à l'excès. L'importance qu'il met à ses plaisirs en fait disparoître la gaieté. Son dîner est une affaire grave et solennelle. Il se fait gloire d'une sensualité raffinée. Son ambition d'habileté consiste surtout à passer pour faire des marchés avantageux dans tous les genres; à avoir toujours ce qu'il y a de plus beau et de meilleur sans le payer plus cher qu'un autre. Ce n'est pas qu'il craigne la dépense, mais il redoute sur toute chose de passer pour dupe, dans quelle affaire que ce soit. Il n'adopte aucune mode étrangère, son luxe est absolument Anglais, et je crois\setcounter{page}{124} que son principal motif en cela est de ne ressembler en rien à son frère. Il aime peu l’Angleterre, mais il méprise souverainement tout ce qui appartient aux autres pays du monde. – Il ne se mêle presque point à la conversation, et il a l’air de se moquer de de ceux qui veulent bien se donner la peine d’être aimables pour les autres, ou qui risquent de se compromettre en amusant le public.
Il se monte pourtant quelquefois à une gaieté bruyante et grossière, mais il n’a aucune sociabilité dans l’esprit; et lors même qu’à la suite d’un dîner il fait des extravagances avec des fous de son espèce, c’est sans entraînement et sans passion.
Il a une maîtresse qui le ruine. Il la maudit cent fois par jour: elle le lui rend bien, à ce que dit Buckhurst, et cependant il vante les charmes d’une vie indépendante, et s’applaudit de n’être pas marié. Il fait profession de mépriser les femmes en général. La personne la plus belle et la plus accomplie ne lui feroit pas faire la sottise d’épouser, à moins, dit-il, qu’elle ne lui apportât 100 mille liv. sterl. dans chacune de ses poches: avec le costume du jour, il ne court pas grand risque.\setcounter{page}{125} On assure pourtant que sous cette enveloppe d'orgueil et d'égoïsme, il est resté quelques germes d'un bon naturel, qu'il est susceptible d'affection pour ses chevaux : ce seroit déjà quelque chose. On l'a vu prendre de l'amitié, de l'admiration même pour son cocher s'il a du talent. Tout homme habile à mener des chevaux a des droits à son respect. Il se fait le compagnon, l'élève même d'un palfrenier ou d'un jockey, s'il peut ainsi se perfectionner sur quelqu'un des points importants. Le coup de fouet, les attitudes, le dialecte dans toute sa pureté, tout cela mérite une étude à part. Il n'y a pas jusqu'à la manière de cracher qui ne soit un objet d'attention; on m'a dit même, que Clay l'Anglais s'est fait arracher une dent, pour cracher dans le vrai style : après cela, je ne réponds pas de la vérité du fait.
\subsection{Le Bal.}
Le jour fixé pour le bal de Mad. Falconer arriva enfin, et toute la bonne société du voisinage se rassembla à Falconer-Court. Mad. Falconer avoit meublé son salon à neuf pour cette occasion avec goût et magni\setcounter{page}{126} ficence. Personne n'entendait mieux qu'elle la direction d'une fête et l'art d'en faire les honneurs. Elle embrassoit d'un coup d'œil' toute une assemblée, et au milieu d'une' multitude de voix, son oreille distinguait et saisissait tout ce qu'il lui importait d'entendre. Quoique son attention se portât sur un grand nombre d'objets à la fois, aucune nuance ne lui échappait; elle mettait à ses démonstrations d'égards et de politesse des gradations si fines qu'il fallait être bien habile observateur pour découvrir le projet, sous cette apparence de bienveillance générale, ou si l'on s'en doutait par rapport aux autres, chacun croyait faire exception, et n'être accueilli d'elle qu'à raison d'un mérite personnel.
Les Dlles. Falconer fixées dans une embrasure de fenêtre à l'extrémité du salon, semblaient avoir adopté un système tout opposé à celui de leur mère. Elles se piquaient de jouer le premier rôle dans leur coterie, et tout ce qui était en dehors de ce petit cercle n'avait aucune part à leurs bonnes grâces. Elles laissaient à leur mère toute la peine et toute la responsabilité du succès de la fête. Leur seule ambition était d'être regardées et de faire effet.\setcounter{page}{127} En attendant l'arrivée des musiciens, les Dlles. Falconer et quelques personnes de leur clique, se placèrent directement vis-à-vis de la porte d'entrée du salon, de manière à voir jusqu'au fond de l'antichambre par laquelle on devoit passer. C'étoit de là qu'elles dirigeoient le feu de leur batterie sur les arrivans, et dès le premier coup-d'œil, l'arrêt étoit prononcé, sans appel à l'indulgence et à la charité.
"Je m'étonne que nous ne voyons point encore les Percy," dit Miss Georgiana.
"Sir Robert Percy at-t-il chez lui quelqu'un de sa famille?" répondit une autre jeune Demoiselle.
"Je ne parle pas de sir Robert, c'est une autre branche. Ils sont tout-à-fait ruinés ceux-là; mais maman s'est cru obligée de les inviter parce qu'ils sont nos parens, quoique nous ne les connoissions point."
" Ce qu'il y a de cruel," ajouta Arabelle Falconer, "c'est que cela nous prive d'avoir sir Robert, car ils sont brouillés."
"Il est vrai que c'est abominable d'être obligé de se gêner pour des gens qui ne vivent point dans le monde et dont on n'entend plus parler depuis des siècles : il n'y a que lord Olborough qui a, dit-on, quelques relations avec le père."\setcounter{page}{128} "Ce qu'il y a de sûr, c'est qu'ils feroient bien mieux de rester chez eux, puisqu'ils aiment tant à vivre en famille. Buckhurst nous a raconté que dans le temps même où ils étoient riches, ils ne voyoient personne; et lady Jane Granville nous a affirmé qu'après la perte de leur fortune, ils avoient refusé l'offre qu'elle leur faisoit de prendre avec elle une de leurs filles et de la mener à Tunbridge. Conçoit-on un pareil entêtement. Jugez quel avantage pour une jeune personne qui n'aura pas le sou ! Aussi lady Jane a-t-elle bien juré de les laisser là avec leurs systèmes et leurs bizarreries."
"Prends donc garde comment tu en parles, dit Georgiana, à entendre certaines gens, ce sont des sages, des savans, des philosophes, que sais-je encore? et pour moi je déclare qu'ils me font tous une telle peur que je n'oserai pas ouvrir la bouche devant eux."
"Heureusement," dit une des jeunes dames, "que leur science ni leur philosophie n'ont pas grand chose à faire au bal, et je crois que pour la danse ils ne sont pas fort à redouter. On dit pourtant qu'il y a une beauté dans la famille."
"J'avoue que je suis curieuse de voir leur tournure," dit Georgiana. "Ne perdons pas de\setcounter{page}{129} de vue la porte du salon, car je ne voudrois pour rien au monde manquer le moment de leur arrivée. "
Cependant l'impatience et la curiosité de la cotterie des moqueurs fut trompée; la musique se fit entendre, et les Percy n'arrivoient point.
Le comte Altenberg parut. Mistriss Falconer s'avança à sa rencontre avec autant d'empressement que le permettoit le décorum. Miss Georgiana se para de toutes les grâces qu'on peut avoir à commandement. Elle se composa de jolis airs de tête. Le Comte vint la saluer. Elle lui répondit tout en jouant de l'éventail; et ses yeux se dirigèrent d'abord sur le cordon et le crachat qu'il portoit; puis sur sa mère comme pour l'assurer qu'elle étoit bien d'accord dans leur plan d'attaque. Georgiana se croyoit à-peu-près sûre de son fait; et ses bonnes amies la confirmoient dans cette espérance tout en se moquant de sa vanité.
Mad. F. pria le Comte d'ouvrir le bal avec lady Francis Alrington. Miss Georgiana lui succéda, et le sentiment d'envie qu'elle croyoit exciter parmi ses compagnes n'ajoutoit pas peu au plaisir de jouer un\setcounter{page}{130} rôle brillant. Le comte Altenberg ne paroissoit occupé qu'à faire valoir sa danseuse; et si la grâce et la noblesse de ses mouvemens le faisoient admirer, c'est que cela lui étoit naturel, et que tout en lui annonçoit l'homme d'un rang élevé. Georgiana se surpassa dans cette occasion, et la satisfaction de sa mère fut au comble lorsqu'elle entendit répéter, d'après l'éloge du Comte, qu'il n'avoit vu danser à ce point de perfection qu'à l'opéra de Paris. Dans ce moment d'un succès si flatteur, Georgiana, loin de craindre l'arrivée de Caroline Percy, auroit voulu l'avoir pour témoin de son triomphe ; mais sa prudente mère qui se faisoit moins d'illusion, étoit charmée de l'absence d'une rivale aussi redoutable. La soirée étoit fort avancée; l'heure du souper approchoit. Les Percy n'étoient point venus. Mad. exprimoit tout haut ses regrets, et commençoit au fond du cœur à
espérer que quelque incident les auroit retenus.
"Etes-vous sûre," dit le Commissaire à la femme, "de n'avoir point fait quelque erreur dans votre carte d'invitation?"
"Parfaitement sûre; mais il est clair à
présent qu'ils ne viendront pas."\setcounter{page}{131} "Ce qui me surprend," reprit Mr. Falconer, "c’est qu’ils aient gardé la voiture et les gens de milord Olborough, que j’ai vu partir il y a plus de trois heures." "Enfin, mon ami, ce n’est pas le cas de nous inquiéter de la chose. Le fait est que nous avons mis le procédé de notre côté. Milord ne peut pas nous en demander davantage; et quant au Comte, cela va fort bien." Sur ce dernier point, le mari et la femme ne s’entendoient pas tout-à-fait. Le ton significatif de Madame fit lever les épaules à Monsieur, tandis qu’elle s’éloignoit avec un geste d’impatience sur son incrédulité. Aussitôt que Georgiana fut revenue à sa place, ses amies la félicitèrent tout bas sur la conquête qu’elle venoit de faire. L’une avoit trouvé dans le premier abord du Comte, lorsqu’il avoit salué Georgiana, quelque chose de tout-à-fait particulier; l’autre avoit remarqué son émotion lorsqu’il lui adressoit la parole; une troisième l’avoit vu pâlir et rougir plusieurs fois pendant la contredanse. Toutes s’accordoient à conclure qu’il étoit décidément amoureux. Georgiana affectoit l’incrédulité, et encourageoit les assertions de ses amis, en exprimant ses propres doutes;\setcounter{page}{132} et elle ne manquoit pas de faire toutes les petites façons et toutes les exclamations requises en pareil cas. "Quelle idée ! quelle folie ! vous l'avez rêvé. Mais mon Dieu, que vous êtes absurdes avec vos suppositions ! Jamais cette pensée ne me seroit entrée dans la tête," etc., etc.
Tout en parlant, Georgiana suivoit des yeux le Comte, qui étoit fort loin de se douter du complot, et chez qui l'idée de Georgiana et celle de l'amour n'avoient rien de commun.
Il étoit observateur. L'Angleterre étoit un pays nouveau pour lui, et la société lui offroit des sujets de remarques intéressans. Il disparut du salon ; et bientôt après on entendit répéter que la famille Percy étoit arrivée, que le Comte avoit paru très-frappé de la beauté de Caroline ; et qu'il avoit dit à son ami le colonel Bremen : "elle ressemble au portrait que nous avons vu, mais il y a bien plus d'expression dans sa physionomie."
A cette nouvelle, Georgiana ne put cacher son dépit. Sa mère, qui avoit toute l'impossibilité apparente d'un diplomate consommé, se hâta d'aller recevoir ses parens, et de leur faire l'accueil le plus aimable. En arrivant dans l'antichambre, elle apprit qu'il y avoit eu un accident de voiture, et son inquiétude...\setcounter{page}{133} tude, ses questions, son empressement à offrir ses services, furent tels qu'on peut les imaginer.
Elle se calma cependant, car personne n'étoit blessé; seulement les dames avoient été obligées de faire un peu de chemin à pied, et cela les avoit retardées. Après que Mr. Percy eut présenté ses filles à Mad. Falconer et reçu les complimens les plus flatteurs sur le bonheur de faire leur connoissance, on s'occupa de réparer les petits dérangemens que l'incident de la route pouvoit avoir occasionné dans la toilette des dames. Mad. Falconer insistoit pour qu'elles passassent dans sa chambre et qu'elles changeassent de souliers; elles eurent beau assurer qu'elles n'avoient point trouvé de boue, il fallut céder à la tendre sollicitude de Mad. sur le danger de garder une chaussure humide. Rosamonde seule tint bon, mais Caroline lasse de résister, mit les souliers qu'Arabella Falconer lui offrit, quoique Rosamonde la poussât du coude, pour n'en rien faire, en lui disant à l'oreille qu'ils étoient trop grands et qu'elle ne pourroit pas danser.
Lorsqu'elles furent entrées dans la salle du bal, Mad. Falconer vit avec inquiétude les\setcounter{page}{134} regards du Comte se tourner fréquemment sur Caroline, comme vers un objet d'intérêt plus encore que de curiosité. "Qu'est-ce donc," pensoit-elle, "qui peut le frapper chez cette jeune personne. Si Georgiana n'a pas l'avantage sur elle pour la régularité des traits, du moins est-il sûr qu'elle a infiniment plus de grâce, elle est plus formée pour les manières; elle est beaucoup mieux mise; pour la danse il n'y aura sûrement aucune comparaison." — En effet ce n'étoit pas par sa beauté seule que Caroline faisoit tant d'impression sur le Comte. Il avoit été préparé par plusieurs circonstances à la voir avec intérêt, ainsi que sa famille; et quelques traits de son caractère qui étoient venus à sa connoissance, lui en avoient donné une opinion extrêmement favorable. Son extérieur confirmoit tout ce qu'il en avoit entendu dire. Son air de modestie et de simplicité surtout, le charmoit, car c'étoit un mérite qu'il avoit rarement trouvé réuni à la beauté. Il lui proposa de danser, mais elle refusa, et il s'assit à côté d'elle et commença une conversation, qui donna de terribles distractions à Mad. Falconer, malgré ses efforts pour paroître attentive à des détails de santé que sa\setcounter{page}{135} voisine lui donnoit avec complaisance, ne doutant point du vif intérêt qu'elle y mettoit. Enfin voyant que l'entretien se soutenoit, et que le Comte paroissoit y prendre beaucoup de plaisir, elle se souvint qu'il étoit temps de faire servir le souper, et envoya à son cuisinier message sur message ; mais celui-ci qui ne se doutoit point qu'un quart d'heure de retard pouvoit acheminer ou faire manquer un mariage, et qui attachoit sa gloire et sa réputation au succès du souper, s'obstina à ne pas servir que l'entremêt ne fut au point de perfection.
Mad. Falconer fut au supplice pendant une demi heure entière, mais elle dissimula à merveille. Georgiana fit retomber toute son humeur sur un de ses plus humbles soupirans, dont sa mère lui ménageoit toujours un certain nombre pour avoir plus de ressources. Une des choses qui les désoloient toutes deux, c'étoit de voir que les Percy recevoient les mêmes témoignages de considération et d'égards qu'avant la perte de leur fortune.
Enfin on annonça que le souper étoit servi. Mad. Falconer plaça le Comte auprès de lady Francis, et Georgiana saisit l'occasion de montrer son amitié pour celle-ci en déclarant\setcounter{page}{136} qu'elle ne voulait point se séparer d'elle. Le Comte fut fort aimable. Étoit-il animé par le désir de plaire aux dames qu'il avoit à ses côtés, ou s'occupoit-il davantage de Caroline qui étoit placée vis-à-vis de lui? c'est une question que sa parfaite politesse laissoit dans le doute, mais que la vanité des deux premières décida en leur faveur.
Lorsqu'on quitta la table pour retourner danser, Rosamonde saisit son moment, se glissa dans la chambre de Mad. Falconer, reprit les souliers de Caroline, s'arrêta dans un cabinet de passage et les lui fit remettre, en la conjurant de danser quand elle en seroit priée. Le Comte dansa deux fois avec Georgiana; puis il s'adressa de nouveau à Caroline, qui accepta. Georgiana qui avoit compris en voyant danser Rosamonde, que l'éducation des deux sœurs avoit été soignée sur ce point là comme sur les autres, malgré le séjour à la campagne, commençoit à trembler. Mad. Falconer comptoit sur l'incident des souliers et s'applaudissoit de cette petite manœuvre; mais Rosamonde l'avoit déjouée fort-à-propos, et Caroline se montra de la manière la plus agréable pour une femme comme il faut; c'est-à-dire, qu'elle n'avoit pas atteint le point de perfection qui ne s'acquiert que par\setcounter{page}{137} par un long travail, mais elle en donnoit l’idée: on sentoit qu’elle avoit la capacité, mais non l’ambition, d’exceller dans ce genre.
Après le comte Altenberg, Caroline eut pour danseur le colonel Spandrill, l’homme à la mode, et par conséquent l’objet particulier des prétentions rivales parmi les jeunes dames du bal. Aussi son choix excitat-il une envie générale. On ne s’en consola qu’en dénigrant Caroline sur tous les points qui pouvoient le moins du monde prêter à la critique, et en assurant qu’on avoit entendu Mad. Falconer demander au colonel de faire danser miss Percy. Il étoit curieux d’observer comment cette classe de jeunes personnes, qui étoient au second rang d’élégance et qui avoient vu sans jalousie Caroline attirer l’attention du Comte, s’acharnèrent à la rabaisser au moment où le Colonel lui adressa ses hommages. L’envie et l’ambition ne sont excitées que par les objets qu’elles peuvent atteindre: il faut un mélange d’espérance et de crainte pour les soutenir et les animer. Aucune de ces demoiselles n’avoit formé de projets sur le Comte, et toutes avoient quelques prétentions sur le Colonel.
Georgiana cependant faisoit auprès d’elle.
i Littérat. Vol. 57. N°. 1. Sept. 1814. \setcounter{page}{138} Comte l'essai de tout ce qu'elle croyoit être des moyens de séduction, mais le Comte n'étoit pas assez novice pour être dupe de ce petit manège de coquetterie. Le naturel seul avoit des charmes pour lui, et malgré sa politesse, Georgiana comprit bientôt qu'elle n'avoit pas réussi. Elle n'étoit point assez maîtresse d'elle-même pour que son dépit ne fût pas évident, quoique son humeur parût tomber sur la chaleur du sallon, sur le choix des contredanses, sur les musiciens; quoiqu'elle se plaignît de fatigue et de mal de tête. Elle s'aperçut de sa maladresse et sentit qu'elle étoit l'objet des railleries de celles mêmes parmi ses compagnes qui l'avoient flattée si complaisamment. Sa présence d'esprit l'abandonna; tout ce qui chez elle étoit factice fit place à une disposition naturellement hautaine, colère et jalouse; sa physionomie en fut altérée au point d'étonner ceux qui l'entouroient. Sa mère aperçut de loin le danger; elle appela le Comte, et l'entraîna sous quelque prétexte à l'autre bout du sallon. Les flatteries du colonel Spandrill ne paroissoient pas avoir plus d'effet sur Caroline, que la coquetterie de Georgiana sur le Comte; et Mad. Falconer, qui avoit espéré beaucoup de cette double diversion, vit avec un extrême\setcounter{page}{139} chagrin le peu de succès de ses manœuvres. Elle devint fort impatiente du départ de la famille Percy. "Mon ami," dit elle à son mari, "il y a long-temps que Mad. Percy a demandé sa voiture. Je suis vraiment inquiète pour leur retour. La nuit est fort obscure, les chemins sont mauvais; il y a loin d'ici chez eux. Voyez, je vous prie, si leurs gens sont prêts." Le Commissaire sortit, et revint bientôt avec une mine allongée qui effraya sa femme; elle l'emmena dans l'anti-chambre, et apprit de lui que lord Olborough avait envoyé une invitation pressante à la famille Percy pour passer la nuit à Clermont-Park, où il avait fait préparer des appartemens. Mr. Falconer ne pouvoit pas revenir de sa surprise. L'humeur de sa femme étoit au comble, et lorsqu'elle vit le Comte présenter la main aux dames Percy pour monter en voiture et les suivre chez lord Olborough, elle maudit l'idée qu'elle avait eue de donner cette fête.
\large{(La suite à un autre Cahier.)}
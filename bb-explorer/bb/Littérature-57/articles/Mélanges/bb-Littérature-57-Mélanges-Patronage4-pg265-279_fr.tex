\setcounter{page}{265}
\chapter{ROMANS.}
\section{CONSEILS D'UNE MÈRE. FRAGMENS du PATRONAGE, par MARIA EDGEWORTH. \large{(Quatrième extrait. Voy. p. 112 de ce vol.)}}
(MAD. FALCONER, !qui voit que le Comte a échappé à Georgiana, porte ailleurs ses vues, et afin de mieux dissimuler ce mécompte, et paroître exempte de toute jalousie de mère, elle écrit à Mad. Percy, pour proposer à Caroline de prendre un rôle dans les pièces qu'on se prépare à jouer chez elle. Mad. Falconer compte si bien sur le refus de Caroline, qu'elle n'en parle point à sa fille. Elle reçoit en effet une réponse négative; et va chercher Georgiana, qu'elle trouve en consultation avec sa femme-dechambre sur le costume de Zaïre dont elle s'est adjugé le rôle ).
"Je pense, ma chère," lui dit-elle, "qu'il ne vous en coûtera pas grand'chose pour l'habit de Zaïre."
"Comment donc, maman? il faut, au contraire, un costume très-riche."\setcounter{page}{266} "Alors, mon enfant, vous arrangerez cela du mieux que vous pourrez avec Lydie: je ne m'en mêle pas. Vous avez votre dernier habit de cour, qui est assurément fort bon pour cela avec quelques petits changemens pour lui donner le caractère asiatique."
"Impossible, maman! ce serait peine perdue, et je déteste la mesquinerie."
"Il est sûr, madame," dit Lydie, "que c'est le cas ou jamais, de montrer ce qu'on peut faire."
"Quant à moi," ajouta Georgiana, "je déclare que s'il faut marchander sur la dépense, j'aime mille fois mieux y renoncer."
"Mais cependant, ma chère, vous avez votre robe de satin bleu, celle de léventine rose, ou la verte; ou encore mieux, votre robe de satin blanc. Vous ne manquez pas de robes assurément. Voyons.— Lydie, montrez-les moi."
Lydie étala sur le lit toute la garderobe de miss Georgiana, mais il n'y eut pas une de ces parures sur lesquelles la maîtresse ou la femme-de-chambre n'eussent des objections insurmontables.
"Mais je ne vois pas là," dit Mad. Falconer, cette robe hortensia qui était si fraîche."\setcounter{page}{267} Ah, oui... cette robe hortensia... je sais ce que voulez dire, maman. La couleur ne m'en plaisoit pas, et je l'ai vendue à Lydie.
Quelque lecteur ignorant sur les usages que la mode autorise, croira peut-être que miss Georgiana se trompoit, en disant qu'elle avoit vendu et non pas donné une robe à sa femme-de-chambre. Mais il apprendra que rien n'est plus commun aujourd'hui que ces sortes de marchés, et que nos jeunes dames n'ont aucune répugnance à faire le métier de revendeuse à la toilette.
"Allons! c'est très-bien," dit Mad. Falconer, "je suis charmée de voir que vous avez des ressources, car il y a peu de chance que votre père veuille fournir à cette nouvelle fantaisie. Le dernier compte de Mad. Spark ne lui a pas fait plaisir du tout, je vous en avertis. Ainsi, puisque vous tenez décidément à avoir quelque chose de nouveau, faites votre affaire avec Lydie, et tirez vous-en comme vous pourrez. Pendant que vous faites votre marché, je m'en vais écrire un billet."
"Un marché!" dit Lydie, "s'il est question d'un marché, je n'y entends rien pour ma part. Personne au monde ne sait moins bien\setcounter{page}{268} faire son compte que moi ; mais si je puis obliger mademoiselle, elle n'a qu'à parler."
"Eh bien ! voyons donc, Lydie. Voulezvous me débarrasser de cette robe de satin blanc ?"
"Tout ce que mademoiselle voudra. Je ne regarde seulement pas la marchandise," répondit Lydie d'un air sentimental, tandis que la maîtresse étendait la robe sur un dossier de chaise. "Mademoiselle n'a qu'à dire son prix. La robe est certainement fort belle. Cependant, mademoiselle doit bien sentir qu'il y a une grande tare, c'est que les manches ne sont pas de même étoffe. On ne porte plus de manches de crêpe, et il n'y a pas un morceau pour en faire d'autres."
"A la bonne heure! nous ne portons plus de manches de crêpe ; mais vous autres, qu'estce que cela vous fait ?"
"Je conviens, mademoiselle, que cela est fort différent. Je n'oublie point la distance de vous à moi ; mais cependant, je prendrai la liberté de vous dire que... lorsque... lorsqu'on a à faire ensemble, il faut entendre la raison des deux côtés." Lydie s'arrêta, et voyant passer sur le front de sa maîtresse un nuage menaçant, elle baissa le ton, radoucit sa voix, et ajouta : "mademoiselle comprend\setcounter{page}{269}
comprend bien ce que je veux dire. Cette robe là est trop belle pour moi. Si je m'en charge, c'est pour la revendre; et moi, qui suis toujours si facile, je suis bien sûre que j'y perdrai; mais c'est égal, je répète que si je puis obliger mademoiselle, elle n'a qu'à faire ses conditions."
"Oh! je vous assure que je ne mets pas grande importance à cette affaire là; mais, soit dit en passant, vous en avez fait une assez bonne avec ma robe hortensia. Je ne l'avois mise que deux fois."
"Ah! par exemple, c'est bien là où mademoiselle se trompe. J'ai cru que jamais je ne pourrois m'en défaire à cause de cette tache de café; mademoiselle sait bien.... et je puis jurer sur ma conscience, que j'y ai perdu. On n'aime pas à revenir sur ces sortes de choses, mais je n'ai pas eu du bonheur avec ces dames l'hiver dernier. Mistriss Crag, la femme-de-chambre de lady Trent, et celles des dames Arlington peuvent le dire; elles savent comment tout cela est allé; aussi combien de fois ne m'ont-elles pas répété, que si mes maîtresses en étoient informées, elles ne permettraient certainement pas que je fusse en perte, mais comme je leur disois." "Mesdemoiselles," leur ai-je dit,
Littérature. Vol. 57. No. 2. Octob. 1814. \setcounter{page}{270} "vous comprenez bien que mes jeunes maîtresses..."
"Abrégeons, abrégeons, s'il vous plaît, Lydie," interrompit Georgiana, "il ne s'agit pas de savoir ce que vous avez répondu à mistriss Crag.—Voyons! combien voulez-vous me donner de ces trois robes?"
"Ah, bien oui! donner!—c'est bien dit cela... Comme les temps sont changés! autrefois c'étoit les maîtresses qui donnoient; à présent ce sont les femmes-de-chambre. Mais passons là-dessus.—Ces trois robes, dites vous.? Eh bien... si mademoiselle veut mettre celle de crêpe rose par dessus le marché, je crois que je ne risquerais pas trop de donner neuf guinées du tout."
"Vous me prenez donc pour une imbécille, avec vos neuf guinées.—La proposition est bonne!—comme si de ma vie je n'avois entendu parler du prix des choses!"
"Ah! je ne dis pas cela. Assurément, mademoiselle en sait plus que moi sur cet article. On ne trompera jamais mademoiselle d'un denier seulement; mais il faut penser que ce qui se vend de seconde main, perd au moins cinquante pour cent de sa valeur. Au reste, j'ai déjà dit à mademoiselle, que j'étois prête à tout faire pour l'obliger. Cependant,\setcounter{page}{271} j'ai dans l'esprit que personne ne pourrait me donner un meilleur conseil que miss Flore ou miss Pritchard. Je m'en vais les consulter. Si elles mettent un prix plus haut que moi, j'en passerai par où elles voudront, et je ne regarderai pas à une guinée de plus ou de moins, parce que je ne sais point lesiner dans les affaires. Ainsi, mademoiselle, vous allez avoir mon ultimatum.
En disant cela, Lydie emporta les robes sur son bras; laissant sa jeune maîtresse dans un combat fort pénible entre l'orgueil et l'intérêt.
Mad. Falconer rentra immédiatement après dans la chambre de sa fille. "Eh bien! ma chère, cette négociation où en est-elle? —Serez-vous en fonds pour vos projets de toilette?"
"Mon Dieu non, maman. Cette Lydie est une véritable juive. C'est bien la créature la plus intéressée! —une rusée petite personne, je vous en réponds,—et d'une impertinence!"
"Comme toutes les femmes-de-chambre, ma fille; mais on ne peut pas s'en passer."
"Il faut en changer du moins quand elles deviennent insupportables."
"Oui! et payer les gages, sans être sûre de gagner au change;—et se mettre à la merci d'une mauvaise langue pour tous les petits secrets de famille."\setcounter{page}{272} "C’est odieux! c’est abominable!" dit Georgiana, agitée par la colère, et se promenant dans sa chambre. "C’est une volerie ! - neuf guinécs pour deux robes de satin et une de tull! - Convenez, maman, que c’est une volerie."
"C’est payer un peu cher l’avantage d’avoir de l’argent comptant ; mais cependant, avec neuf guinées vous aurez un costume très-passable. Au fond, c’est là l’objet."
"Avec neuf guinées! c’est impossible, parfaitement impossible. Il n’y a qu’à laisser
tomber la chose et qu’il n’en soit plus question."
"Un moment, ma chère," dit Mad. Falconer froidement."Vous êtes bien maîtresse de renoncer au rôle de Zaïre, mais alors, je le proposerai à quelqu’un."
"Et à qui donc, maman?"
"Mais… par exemple… Une des dames Arlington ? Pour la figure, toutes deux feroient fort bien."
"Ah, bon! croyez-vous que lady Anne, avec sa nonchalance, vînt jamais à bout d’apprendre un rôle?"
"Eh bien, si ce n’est elle, ce sera lady Francis. Je suis sûre qu’elle ne demandera pas mieux."
"Peut-être qu’elle en aura la fantaisie an\setcounter{page}{273} premier moment, mais qui peut compter sur une tête comme la sienne? Elle ne tient pas deux minutes à la même idée. "
"J’ai encore une autre ressource ," dit Mad. Falconer, "et je suis presque sûre de réussir."
"Comment donc, maman?"
"Le comte Altenberg remarquait l’autre jour, que miss Caroline Percy avait tout-à-fait une figure de Zaïre, et je crois en effet qu’elle feroit fort bien pour ce rôle. J’ai envie de lui écrire."
La surprise et le dépit ôtèrent à Georgiana la possibilité de rien répondre.
"Je conçois votre étonnement.—Vous croyez que j’ai perdu la tête ce matin.—Plût à Dieu que vous sussiez toujours aussi bien ce que vous faites! vous n’auriez pas donné la comédie à vos dépens, comme vous le fîtes hier au bal. Encore une scène de ce genre, et vous vous attachez un ridicule pour la vie. Croyez-vous de bonne foi, mon enfant, que je pourrai vous procurer un établissement convenable, si vous détruisez ainsi dans un instant l’effet de toutes mes mesures? — J’avoue que je vous crois moins novice."
"Que voulez-vous, maman!—J’ai le malheur d’avoir une sensibilité excessive."
"C’est un malheur en effet , mais il\setcounter{page}{274} n'est pas sans remède, et si vous aviez un peu plus de déférence pour mes avis. ......... mais vous n'écoutez rien. "
" Gageons, Maman, que vous n'avez jamais eu d'amour! "
" Allez-vous me faire croire que vous avez une passion d'héroïne de roman, une passion invincible? —Parlons raison mon enfant. "
" A la bonne heure ! mais surtout ne parlons pas de Caroline Percy. "
" Point de conditions, mademoiselle.— Ecoutez-moi tranquillement. "
" Savez-vous, maman, ce qu'il faut faire! Il n'y a qu'à mettre dans l'affiche: Zaïre, par miss Caroline Percy, à la demande instante du comte Altenberg.—L'idée est bonne. —Je suis décidée à mettre cela dans l'affiche. "
" Vous ne ferez rien, mademoiselle, sans mon autorisation, et je ne veux point compromettre mes intérêts et ceux de toute ma famille pour entrer dans vos petits ressentimens. "
" Mon Dieu, maman ! comme vous êtes vive aujourd'hui! allons ma bonne petite maman, vous qui ne vous laissez jamais emporter, ayez un peu d'indulgence pour votre pauvre Georgiana.—Rasseyyez-vous, maman, " ajouta-t-elle en passant ses bras autour du cou de sa mère. " Me voilà prête à vous entendre. "\setcounter{page}{275} Mad. Falconer connoissoit trop bien le caractère irritable de sa fille pour lui dire toute la vérité. Elle ne lui laissa pas comprendre qu’elle n’avoit plus d’espérance à l’égard du Comte; elle supposa que Caroline pouvoit bien lui inspirer une fantaisie passagère, mais qu’avec de la suite et de la prudence on pourroit le ramener. "Surtout, mon enfant, n’allez pas donner prise sur vous, en laissant percer la moindre nuance de jalousie. Votre jeu, c’est de paroître calme et occupée de toute autre chose. Il ne faut point décourager vos autres admirateurs. S’il y a un moyen de regagner le Comte, c’est celui-là. — Encore une fois, il faut de la patience et de la circonspection. "
"A présent, maman, convenez que ce n’est pas sérieusement que vous parliez de m’ôter le rôle de Zaïre? "
"La proposition que j’ai faite, et tous les complimens que contenoit mon billet de ce matin étoient nécessaires pour réparer vos bévues d’hier. "
"Est-il possible, maman? vous avez réellement écrit? "
"J’étois sûre d’un refus. "
"Et si vous vous étiez trompée! " dit Georgiana, avec une agitation toujours croissante, et prête à pleurer de colère : "Si elle\setcounter{page}{276} avait accepté !..... Qui sait encore si avec son affectation de simplicité, elle n'en meurt pas d'envie?— Elle trouvera moyen d'en revenir de ce refus. Vous verrez qu'elle en reviendra. La vanité des parens sera flattée ; ils se cotiseront tous pour faciliter la chose : ils emprunteront plutôt....... Cette odieuse créature est née pour mon tourment !.....
"Doucement ! doucement ! comme cette petite tête se monte !......, Tenez ,, dit Mad. Falconer, en remettant à sa fille le billet de Mad. Percy ,, et une autre fois, fiez-vous en au jugement de votre mère. "
"Il faut avouer ,, dit Georgiana , après avoir lu , "que nous l'avons échappé belle. Je gage que cela a tenu à la dépense. "
"Que nous importe, mon enfant ? Voyons ce que nous avons à faire.—Vous comprenez, ma chère, que je mets beaucoup d'intérêt à vous voir bien mise dans une occasion comme celle là. L'auditoire sera nombreux. Nous aurons tout l'univers. — Voila les deux MM. Clay qui s'annoncent encore. Quant à l'argent, mon enfant , je ne connois qu'un moyen de toucher cette corde avec votre père sans le mettre de mauvaise humeur. "
"Un moyen, maman? dites! dites! "
\setcounter{page}{277} "Il ne faut pas rebuter Murray comme vous l'avez fait jusqu'ici."
"Oh, maman! Je l'ai en horreur."
"Voilà justement ce qu'il faut bien se garder de laisser voir. Il a parlé de vous très-sérieusement à votre père; et quand j'ai dit qu'il ne vous plaisoit pas, votre père a été fâché tout de bon. — Cela et le Mémoire de Mad. Spark, sont d'un mauvais effet, je vous en avertis. — Si cependant nous réussissions à mener la chose à bien, quelle seroit votre idée pour le costume?"
"Mon idée?..... Voyons..... J'aimerois un satin d'un bleu pâle, délicat, le plus délicat possible, avec une frange d'argent, pour le premier acte. Ensuite je voudrois...."
"Quoi, deux habits! Vous êtes folle !;
"Vous me demandez mon idée, maman; je vous la dis."
"Oui; mais il ne faut vouloir que ce qui est possible. Revenons à l'essentiel. Dirai-je à votre père que vous êtes bien disposée en faveur de Murray ?"
"Mon Dien! quelle persécution ! comment pouvoir souffrir cet homme là, après avoir vu le comte Altenberg ?"
"A quoi servent les comparaisons, mon enfant? mettez de côté toute idée romanesque."\setcounter{page}{278} que. Rappelez-vous que le général Murray a la goutte dans l'estomac, et qu'il laissera à son neveu sa terre de ***, cela vaut bien la peine qu'on y pense."
"Oui, mais le rang du Comte !"
"A la bonne heure, mon enfant, le titre est une jolie chose ; mais allons au plus pressé ! Vous ne pouvez pas avoir un costume de Zaïre sans faire quelque sacrifice.— Prenez un peu de mes avis, vous vous en trouverez bien. Je m'engage à obtenir de votre père ce qu'il vous faut. Autorisez-moi seulement à lui dire que si, d'ici à un an, il ne se présente rien de mieux, vous récompenserez la persévérance de Murray. Remarquez bien que le si, nous tirera toujours d'affaire."
"Est-ce bien sûr, maman ?-si je le croyois !... Peut-être en effet est-ce le seul moyen...."
"Voilà ce qui s'appelle être raisonnable ! Je savois bien moi que vous finiriez par là. - Allons ! je m'en vais, de ce pas, vers votre père."
C'est ainsi que Mad. Falconer trouva moyen de faire prévaloir la vanité sur les autres passions dans le cœur de sa fille pour l'amener à ses fins. Bien satisfaite du succès, elle alla exercer son adresse sur l'esprit de son mari, et parvint, non sans peine, à gagner sa cause,\setcounter{page}{279}
c'est-à-dire, à avoir carte blanche pour le satin bleu, les franges d'argent, les perles, etc Georgiana, charmée de pouvoir reprendre avec sa soubrette, le ton de hauteur dont son intérêt l'avoit obligée à descendre, sonna pour rappeler Lydie, dont elle traitait les offres avec tout le mépris convenable, en lui ordonnant de remettre les robes en place. Lydie comprit à-peu-près, la raison de ce changement ; cependant comme elle pensa qu'on auroit besoin d'elle dans la grande affaire de la toilette pour la tragédie, elle ne compromit point sa dignité en laissant soupçonner à sa maîtresse qu'elle regrettait le marché ; mais le fait est qu'elle ne vit pas plus tôt l'affaire manquée qu'elle la trouva superbe. Georgiana de son côté, pour qui Lydie étoit une personne indispensable, parce qu'elle réunissoit les talents d'une couturière et d'une marchande de modes, prit son parti de ne pas entendre quelques impertinences, dites à demi-voix, et remit sa femme-de-chambre de bonne humeur par le cadeau qu'elle lui fit de sa robe de satin blanc.
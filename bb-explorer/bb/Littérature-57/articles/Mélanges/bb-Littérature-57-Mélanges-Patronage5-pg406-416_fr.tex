\setcounter{page}{406}
\chapter{ROMANS.}
\section{FRAGMENT DU PATRONAGE, traduit de miss MARIA EDGEWORTH. \large{(Dernier extrait. Voy. p. 265 de ce vol.)}}
C'EST une pénible tâche que d'avoir à tracer les progrès du mal dans une ame naturellement droite et généreuse. Buckharst avait dans sa première jeunesse, un sentiment vif d'honneur, une ouverture de caractère, qui auraient pu le conduire à tout ce qui est noble et utile sans la folle vanité, et l'esprit d'intrigue de ses parens. Sa mère l'avait dégoûté de la profession du barreau pour le jeter dans le grand monde et les prétentions frivoles. Son père avait employé la ruse et la tyrannie pour lui faire embrasser une vocation à laquelle il n'étoit point propre. Ses premiers pas dans cette carrière furent entourés d'écueils. Il suivit le colonel Hanton:\setcounter{page}{407} comme chapelain à son régiment, dont les officiers étoient buveurs et joueurs. La fausse honte, la crainte de passer pour pédant, d'être évité comme un censeur, firent bientôt taire chez Buckhurst la voix de la raison et de la conscience. Il en vint même à se glorifier de ce qu'il appeloit libéralité de principes, absence de préjugés. Cependant, il ne
pouvoit s'étourdir complétement sur ses torts; et il y avoit des momens où il se jugeoit lui-même, et sentoit le besoin de regagner sa propre estime. Il fit un effort pour acquérir au moins une réputation d'éloquence. Il eut occasion de prêcher dans une chapelle de Londres très-fréquentée, et acquit une espèce de célébrité; mais bientôt l'opposition reconnue entre ses mœurs et sa doctrine, lui firent perdre dans l'opinion des gens sensés plus de suffrages qu'il n'en avoit gagnés comme orateur, parmi les gens superficiels. L'envie peut-être se joignit aux clameurs qui s'élevèrent contre lui; c'est ainsi du moins que Buckhurst se plut à le croire, et craignant qu'on ne portât de plus près la lumière de la critique sur sa vie privée, il céda la place à un concurrent très-inférieur
en talent, mais dont la conduite étoit d'accord avec ses principes.
Le commissaire Falconer, qui n'avoit qu'un objet en vue; la survivance du bénéfice\setcounter{page}{408} de Chipping Friars, lequel dépendait du colonel Hauton, ne fut point fâché de voir le terme de cette fantaisie de vaine gloire. L'ambition de Buckhurst, arrêtée de ce côté là, prit un essor moins relevé. Il n'aspira plus qu'aux succès de société, et se contenta même du misérable avantage de divertir sa cotterie par des contes et des bons mots. Pour obtenir les applaudissemens de ses auditeurs, il falloit se plier à leurs goûts, adopter leur langage. Le colonel et ses amis faisoient peu de cas du bel esprit: les nuances leur échappoient. Il leur falloit des anecdotes burlesques, de la bouffonnerie. Buckhurst avoit un talent remarquable dans ce genre ; et en particulier celui de contrefaire les gens ridicules, à un point de perfection rare. C'étoit un des amusemens favoris du colonel ; et il falloit que Buckhurst lui donnât au moins une fois par semaine quelque scène de ce genre. Non-seulement il imitoit les manières et l'accent des personnages qu'il vouloit représenter, mais encore leur tour d'esprit. Il les faisoit parler avec tant de naturel et de vrai comique, qu'on ne se lassoit point de l'entendre. Les deux frères Clay étoient souvent le sujet de cette espèce de plaisanterie pour leur bon ami le colonel, qui ne se doutoit pas qu'aussitôt qu'il auroit tourné le\setcounter{page}{409} le dos, il serviroit lui-même de modèle à l'acteur. C'est ce qui arriva un jour où Buckhurst étant encore à table avec ses convives, et ayant cru voir sortir le colonel, amusa parfaitement l'assemblée aux dépens de son patron. Celui-ci étoit resté derrière un paravent et vint se rasseoir en face de Buckhurst. Jamais apparition nocturne ne porta plus d'effroi dans l'ame d'un coupable qu'il n'en éprouva dans cette occasion. Il pâlit, bégaya quelques mots. Un moment de silence complet s'en suivit. Buckhurst eut cependant assez de présence d'esprit pour trouver un faux-fuyant. A l'aide de quelques signes du coin de l'œil et de quelques avertissements du bout du pied par dessous la table, il mit ses complices dans le secret ; il feignit d'avoir voulu imiter un certain colonel Hallerton. Il recommença même cette plaisanterie, dans laquelle les autres entrèrent de leur mieux. Le colonel Hanton ne laissa rien paroître sur sa physionomie qui pût confirmer ou dissiper les doutes et les inquiétudes des spectateurs. — Après son départ, bien et duement constaté par le bruit des roues de sa voiture, on tint conseil sur ce qu'il y avoit à faire. On s'accorda à tranquilliser Buckhurst sur la réussite de son stratagème; mais personne n'y mit plus d'intérêt\setcounter{page}{410} qu'un de ses confrères, chapelain d'un grand seigneur, et avec qui il avait fait récemment connoissance. Cependant il y avait quelque chose de doucereux et d'hypocrite dans le ton de Mr. Sloak, qui ne plaisait pas à Buckhurst, et lui laissa un peu de défiance.
Le lendemain, le colonel revit son protégé, et le traita comme à l'ordinaire. Au bout de quelques jours, Buckhurst oublia cette aventure. Un mois après, le Recteur de Chipping Friars mourut. Burckhurst reçut une invitation du colonel à un grand dîner; ses amis vinrent le féliciter. Son imagination travaillait déjà sur l'emploi des revenus de la dîme. Il se choisissait d'avance un suffragant, et lorsqu'il arriva chez le colonel, tout son plan était arrangé dans sa tête. Il trouva les convives disposés à la gaieté; lui-même était rayonnant de joie. On ne disait rien du bénéfice, mais c'était une chose entendue que Buckhurst allait entrer en possession.
Au milieu du repas, le colonel s'écria: "allons! messieurs! remplissez vos verres. — Buvons à la santé du nouveau Recteur." Chacun se prépara à répondre à l'invitation du colonel. Buckhurst seul, attendit modestement les salutations qui allaient s'adresser à lui.
"Mr. Sloak," dit le colonel, en élevant voix. "Je bois à votre santé et vous salue\setcounter{page}{411} comme Recteur de Chipping Friars. — Buckhurst!" ajouta-t-il avec un sourire malicieux: "vous ne remplissez pas votre verre?"
Buckhurst resta stupéfait. "Qu'est-ce donc, dit-il, "que cette plaisanterie?"
"Une plaisanterie! — parbleu non! ce n'en est pas une. Je sais ce que valent les plaisanteries et ceux qui les font."
"Mais que signifie donc cela?....."
"Cela signifie," répondit froidement le colonel, "que si vous avez cru me prendre pour dupe, vous vous êtes trompé, et que je ne me soucie pas de payer un bouffon pour me tourner en ridicule.— D'ailleurs, j'ai trop à cœur le bien de l'église pour faire un Recteur d'un homme qui n'a aucune rectitude: — Je sais aussi jouer sur le mot, moi.— Qu'en dites-vous?"
Les rieurs furent cette fois pour le colonel. Buckhurst se leva de table dans un accès de rage, en s'écriant: "hypocrisie! mensonge! ingratitude! lâcheté!—oh! si je n'étois retenu par l'habit que je porte!.... pourquoi me suis-je imposé de semblables chaînes!"
"Il est un peu tard pour s'en aviser," dit le colonel. "Il falloit y penser avant de s'exposer à être en scandale à l'église."
Buckhurst furieux, sortit en se frappant la tête avec son poing, et en maudissant sa destinée. Les huées des prétendus amis l'accom\setcounter{page}{412} pagnèrent. Il s'éloigna rapidement, et ne s'arrêta que lorsqu'il fut hors de portée. Alors s'appuyant contre un arbre, il donna l'essor à l'amertume de ses sentimens. "Oh! Mr. Percy!" s'écria-t-il." Vous futes une fois mon ami.—Un ami sincère, généreux! Pourquoi n'ai-je pas suivi vos conseils!—Vous me connoissiez mieux que je ne me connoissois moi-même.—Oh, Caroline! si vous pouviez me voir, vous auriez pitié de moi!... mais, hélas! vous me mépriseriez.—Mon père, oh, mon père! que n'avez-vous pas à vous reprocher?,"
À ces cruels retours sur le passé, succédèrent des craintes non moins pénibles sur l'avenir. Buckhurst avoit contracté des dettes pour des sommes considérables, et ne voyoit aucun moyen de les acquitter. Ces réflexions ramenèrent dans son cœur un sentiment de haine pour l'homme qui avoit pu le traiter avec cette malice froide et rafinée. Il passa le reste du jour, et la nuit qui le suivit, dans l'agitation d'une ame profondément ulcérée. Si le sommeil s'emparoit un moment de ses sens fatigués, c'étoit pour lui retracer plus vivement encore la scène de la journée; l'ironie amère du colonel, l'hypocrisie de Sloak, le sourire de ses lâches amis. Deux fois il se réveilla en sursaut, croyant tenir au collet les traîtres qui s'étoient joués si cruellement de lui.\setcounter{page}{413} Il s'attendoit à être assailli par ses créanciers, dès que cet évènement seroit connu, et il résolut, quoiqu'il lui en coûtât, d'aller chez son père, puisqu'il falloit bien qu'il fût instruit de la chose.
C'est épargner au lecteur un tableau révoltant, que de supprimer le récit des accusations mutuelles du père et du fils, s'attribuant l'un à l'autre les humiliations qu'ils éprouvoient. Tous deux enfin, rapprochés par un intérêt commun, en revinrent à chercher ce qu'il y avoit à faire. Mr. Falconer déclara qu'il étoit dans l'impossibilité de payer les dettes de son fils, et qu'il ne voyoit qu'un moyen de se tirer de ce mauvais pas, c'étoit de chercher un meilleur patron. A ce mot, Buckhurst s'écria que de sa vie il ne se remettroit dans une pareille dépendance.
" Eh bien, monsieur, " dit le commissaire, " préparez-vous à souffrir les conséquences de votre obstination, et voyez si la perspective de passer votre vie dans une prison, vous paroît moins effrayante que les moyens que je vous propose. "
Buckhurst soupira, garda le silence, puis il reprit: " Quel seroit donc le nouveau patron que vous auriez en vue? "
" Votre ancien ami l'évêque Clay. "
" Je n'ai aucun titre à sa bienveillance, d'ailleurs, il a déjà beaucoup fait pour moi. "\setcounter{page}{414} "C'est une raison pour qu'il fasse davantage..."
"Mais il ne payera pas mes dettes, et c'est la difficulté pressante. Il n'y auroit qu'une chose qui pût me tirer d'affaire, ce seroit un bon bénéfice: il ne peut point m'en procurer. Si le doyenné du Dr. Leicester étoit vacant..... c'est la Couronne qui en dispose.... mais le bon homme a l'air de vivre autant que moi. "
"Ecoutez-moi, jeune homme.- Vous ne voyez pas où aboutissent mes projets.- Tout peut encore être réparé, et cela sans qu'il nous en coûte rien.- L'évêque a une sœur. - Miss Tammy Clay......"
"Oui, oui, je sais; - avare et laide comme le diable. "
"Riche; très-riche; - et fort bien disposée en votre faveur, comme vous le savez. "
Buckhurst repoussa avec horreur l'idée d'un mariage qui ne seroit qu'un vil marché. Il aimoit mieux mille fois, disoit-il, mourir en prison. "
Le commissaire le laissa déclamer contre les mariages d'intérêt, bien sûr que la nécessité l'amèneroit à entrer dans ses vues. Il ne se trompoit pas. Les créanciers de Buckhurst devinrent pressans. Il falloit choisir entre la prison et miss Tammy. Buckhurst se retira à la campagne, chez l'évêque, pour y réfléchir plus tranquillement.- Miss Tam-\setcounter{page}{415} my lui laissoit entrevoir qu'il n'éprouveroit pas un refus. Elle avoit pris un empire absolu sur son frère, dont elle dirigeoit la table avec un talent distingué; et malgré la répugnance de l'évêque à se séparer d'une si bonne économe, il consentit au mariage.
En revanche, les neveux de miss Tammy, Clay l'Anglais et Clay le Français, ne furent pas de si bonne composition. Ils s'étoient accoutumés à regarder l'héritage de leur tante comme une propriété. L'idée d'en être dépouillés de cette manière ne s'étoit pas seulement présentée à eux; et Falconer étoit l'homme du monde qu'ils auroient le moins soupçonné de leur jouer un pareil tour: il s'en étoit moqué cent fois; et quoique Clay le Français eût attiré sur lui le blâme du public en épousant une chanteuse italienne, il clabauda plus fort que personne contre l'union désassortie que formoit sa tante. — Clay l'Anglais attachoit moins de prix à l'argent, mais il ne pouvoit supporter l'idée d'être dupe de l'intrigue des Falconer. C'étoit justement alors qu'il venoit de se décider à demander la main de miss Georgiana, par l'intercession de lady Trant. Sa tante lui avoit promis six mille livres quand il se marieroit. Elle rétracta cet engagement, et se vengea de quelques mots piquans qui lui étoient échappés sur son âge et sur sa figure,\setcounter{page}{416} en déclarant qu’elle n’avoit pas trop de fortune pour satisfaire aux goûts d’un époux qui aimoit à faire de la dépense. Buckhurst auroit volontiers abandonné la somme en question ; mais Clay ne voulut pas la recevoir à titre de grace. L’orgueil de tous deux fut irrité : ils se brouillèrent décidément. Clay revit lady Trant, lui reprocha de s’être trop hâtée d’exécuter sa commission, et la chargea d’un second message beaucoup plus important à ses yeux que le premier ; c’étoit d’informer Mad. Falconer, qu’une double alliance avec sa famille étoit un honneur auquel il n’avoit point aspiré, et qu’en un mot, il falloit renoncer à l’une des deux unions pour que l’autre pût avoir lieu. Lady Trant pouvoit donner à cette déclaration la forme qu’elle voudroit ; mais telle devoit en être la substance.
Il étoit difficile de rendre cette communication agréable. Mad. Falconer ne perdit cependant pas la tête. Elle écrivit à son fils pour l’engager à retirer sa demande. Il y avoit, disoit-elle, tant d’autres vieilles femmes riches à épouser. Comment pourroit-il persister à être un obstacle au bonheur de sa sœur?
Buckhurst répondit : "Ma chère maman, je suis marié depuis hier, et surement plus fâché que vous ne pouvez l’être."
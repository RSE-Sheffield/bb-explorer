\setcounter{page}{532}
\chapter{ROMANS}
\section{WAVERLEY, ou IL Y A SOIXANTE ANS. (Edinburgh 1814) 3 vol. in-12°}
De tous les dons de l'esprit, celui auquel nous applaudissons avec le plus de plaisir, est sans doute le don d'une imagination brillante. Aucun ne répand des jouissances plus vives, plus faciles, plus variées, et peut-être aucun ne désarme-t-il davantage la jalousie. L'on est d'ailleurs porté à craindre que ce don précieux ne devienne de jour en jour plus rare. Les mêmes causes qui favorisent les progrès des sciences chez les nations dès long-temps civilisées, nous paroissent menacer de stérilité la région des beaux-arts; et quand la nature libérale dissipe nos inquiétudes en créant des talens, nous voyons avec joie l'espèce humaine recouvrer ses droits contestés à une jeunesse éternelle.
C'est avec ce sentiment de satisfaction que nous pouvons contempler la Grande-Bretagne. Deux poètes distingués y fleurissent dans ce moment; ils s'y partagent tous les suffrages,\setcounter{page}{533} et y ont ranimé à un degré qui va jusqu'à l'enthousiasme, le goût pour l'art qu'ils cultivent. Tous deux sont pleins d'imagination, tous deux ont une veine abondante et facile, tous deux dédaignent de s'asservir aux règles que le sentiment du beau n'a pas imposées, mais d'ailleurs ils n'ont aucun rapport entr'eux. L'un, Lord Biron, répand sur ses ouvrages une teinte sombre. Il a parcouru l'orient et se plaît à faire contraster les bienfaits de la nature, dans les belles contrées de la Grèce et de l'Asie, avec les ravages du despotisme. De même encore, quand il peint les hommes, il aime à faire briller les traits d'une grandeur primitive dans une ame dévastée, désséchée même, par les passions les plus orageuses. C'est un penseur autant qu'un poète, et il est aussi remarquable par l'harmonie que par le sens profond de ses vers. Il y a beaucoup d'originalité, d'énergie et même d'éclat dans son talent, mais cet éclat a quelque chose de sinistre, et ses ouvrages laissent un sentiment de tristesse. L'autre poète que l'Ecosse a vu naître, Walter Scott, imprime au contraire à l'imagination le mouvement le plus rapide et le plus agréable. C'est à la fois un barde et un troubadour. Il trace le tableau des mœurs féodales et l'anime des plus vives couleurs. Tout vit,\setcounter{page}{534} tout se meut, dans ses poëmes. Il rend frappans de vérité les détails les plus bizarres, familiers, les usages les plus éloignés des nôtres. On a sous les yeux tous les personnages, les costumes, les chevaux, les bannières, les chasses, les combats; et le fond sur lequel ressortent une foule de figures brillantes, c'est toujours cette nature agreste, ces brouillards, ces sapins, ces cascades, ces rochers, ces lacs sauvages, qui forment les traits un peu rudes, mais beaux néanmoins, des climats glacés. Le talent de Walter Scott est éminemment pittoresque, et s'il peut s'abstenir de le délayer dans des compositions négligées, il méritera d'être appelé l'Arioste du Nord.
C'est au poëte de l'Ecosse qu'est universellement attribué le roman dont nous allons nous occuper; et assurément, il est facile de l'y reconnaître, soit à l'extrême vivacité des peintures, soit à deux pièces de vers très-remarquables. Comme roman, Waverley n'est pourtant pas un bon ouvrage. La fable en est foiblement ourdie, et le héros, dont le nom même indique le caractère vaillant, inspire peu d'intérêt. Quelques figures semblent tracées d'après nature, avec une force et une vérité extraordinaires, tandis que d'autres, qui paroîtroient devoir jouer un rôle\setcounter{page}{535} important, sont à peine esquissées, et que des personnages vulgaires occupent beaucoup trop de place. Malgré quelques belles scènes et la catastrophe qui est très frappante, la fiction est peu de chose dans cet ouvrage, c’est la vérité qui en fait le mérite et le charme. Si l’on y voit un cadre léger fait pour amener des tableaux d’histoire du plus grand intérêt, des peintures de la nature et des mœurs, originales autant que fidèles, l’on saisira le point de vue le plus juste et le plus favorable. Les Ecossais disent que c’est un monument précieux pour constater l’état de leur pays à l’époque choisie par l’auteur. Il servira sur-tout à fixer le souvenir des usages si singuliers de ces tribus appelées Clans, composées d’hommes qui tous portent le même nom et reconnaissent l’autorité, à la fois patriarchale et militaire, de celui qui est censé le chef de la famille. C’est donc comme histoire et comme voyage qu’il faut considérer ce roman, et c’est sous ce double aspect que nous nous attacherons à le présenter. Les évènemens nous fourniront un simple fil pour lier des scènes, en grande partie instructives, dont quelques-unes sont dignes de figurer dans les fastes britanniques. Nous ne pourrons malheureusement pas donner des échantillons de chaque genre de tableau.\setcounter{page}{536} La bigarrure des dialectes, qui ajoute du piquant et de la vérité à plusieurs morceaux, ne peut point passer d'une langue dans une autre ; et c'est ce qui rendra cet ouvrage impossible à traduire. La forme d'extrait est peut-être la seule sous laquelle on puisse le faire connaître en français, et cela du moins, donnera quelque prix à notre travail.
Edouard Waverley, le héros du roman, est issu d'une famille Anglaise, illustre et considérée, qui s'est depuis long-temps fait connaître par son attachement pour la maison de Stuart. Toutefois, le père de ce jeune homme, Richard Waverley, se trouvant réduit à la mince fortune d'un fils puîné, s'est vendu au ministère de la branche régnante de Hanovre, et a, en conséquence, été pourvu d'une place lucrative, ce qui a mis la division entre lui et un frère aîné, Sir Everard Waverley. Ce dernier, digne héritier de ses ayeux, et imbu de tous les préjugés religieux et politiques du parti jacobite, vivoit dans une terre magnifique, nommée Waverley-Honour, avec tout l'orgueil d'un noble et d'un mécontent. Sa sœur, Miss Rachel Waverley, faisoit les honneurs de sa table, et ni l'un ni l'autre ne voyoient jamais leur frère cadet, qui passoit une partie de l'année, non loin de leur demeure, dans une maison de\setcounter{page}{537} campagne appartenant à la femme qu'il avait épousée. Cependant, un jour que le fils issu de ce mariage, notre héros, alors âgé de cinq ans, vit, en se promenant avec sa gouvernante le carosse doré de son oncle, traîné majestueusement par six chevaux, le petit Edouard, à l'aspect des armes de sa famille (trois hermines passant) peintes sur la portière, crut, on ne sait comment, que le carosse lui appartenait, et poussa de tels cris pour y entrer, que Sir Everard, frappé de l'énergie avec laquelle ce bel enfant revendiquait ses droits héréditaires, ne put résister au désir de le caresser. Le jeune Waverley fut renvoyé dans ce même carosse chez ses parens avec un message poli, et il s'ensuivit une espèce de réconciliation entre les deux frères. Un commerce de froide politesse s'établit du moins entr'eux, et Edouard, devenu l'objet particulier de la tendresse de son oncle, fut élevé alternativement à Waverley-Honour et dans la maison de son père.
Cette éducation fut d'une nature très-incohérente. Non-seulement Edouard puisait des principes différens dans chacun de ses domiciles, mais on ne le soumettait à aucune discipline sévère. Le chapelain de Sir Everard, un savant docteur, nommé Mr. Pembroke, qui fut chargé de l'éducation de\setcounter{page}{538} ce jeune homme pendant ses séjours à Waverley-Honour, le laissoit passer suivant son caprice d’une étude à une autre, en sorte qu’il prit l’habitude de se livrer à ses diverses inspirations et de ne consulter que ses goûts. Une imagination très-vive et une intelligence rare lui faisoient aimer la lecture, et particulièrement celle des ouvrages de littérature; mais tout en dévorant d’innombrables volumes dans la vaste salle gothique qui servoit de bibliothèque à son oncle, il ne prit point d’aptitude pour un travail suivi et n’acquit pas même d’instruction régulière. L’histoire des Waverley, qui faisoit le fond de la conversation de Sir Everard, celle des guerres civiles dans lesquelles ils avoient pris parti, les traditions superstitieuses du vieux temps, que lui transmettoit Miss Rachel, la vue des restes de châteaux forts qui hérissoient alors les sommets de la plupart des collines dans une contrée romantique, tout conspiroit à favoriser, chez notre héros, une disposition enthousiaste et rêveuse, comme à l’éloigner de la vie réelle et de l’activité persévérante qui y fait jouer un rôle utile.
Les parens d’Edouard sentirent la nécessité de donner une direction différente à ses pensées; en conséquence, sir Edouard, après de mûres réflexions, résolut de l’envoyer\setcounter{page}{539} voyager avec Mr. Pembroke, et fit part de ce dessein au père du jeune homme; mais celui-ci, en ayant parlé chez le Ministre auquel il était dévoué, la cour prit de l'ombrage des principes politiques du Chapelain de Waverley-Honour, et fit offrir à Richard Waverley pour son fils, une compagnie de dragons. Cet homme ambitieux n'osa point refuser une offre pareille, quoiqu'il sentît la difficulté de réconcilier sir Édouard avec l'idée que son neveu et son héritier serviroit la maison d'Hanovre. Heureusement les préjugés de ce chef de la famille étoient déjà un peu affoiblis par le temps, et l'on releva tellement à ses yeux l'éclat de la profession des armes qu'avoient suivie ses ancêtres, qu'après avoir long-temps regardé son arbre généalogique, où ils étoient tous désignés par leurs grades militaires, il prit son parti de ce qui étoit déjà irrévocable.
Édouard, lui-même, fut très-satisfait de cette détermination. Son imagination prit un autre cours et lui promit de la gloire. Une pièce de vers charmante, dont il est censé être l'auteur, peint ce qui passe dans la tête d'un jeune homme, quand les visions aériennes au milieu desquelles il se plaisoit à vivre, font place aux perspectives mieux arrêtées d'une carrière honorable.\setcounter{page}{540} Sir Edouard acheva de se consoler du parti que prenoit son neveu, en faisant préparer pour lui un magnifique équipage de guerre, et, le jour du départ, après une exhortation qu'il vouloit rendre solennelle et qui ne fut que touchante, pour couper court à sa propre émotion, il mena Edouard dans ses écuries voir les beaux chevaux qu'il lui destinoit. — "Vous croirez, lui dit-il, partir avec une suite bien peu nombreuse, si vous vous comparez à sir Hildebrand Waverley, qui passoit en revue dans la cour de ce château, un corps de cavalerie plus considérable que tout votre régiment. J'aurois désiré que ces vingt jeunes gens de mes domaines, que j'ai fait enrôler dans votre troupe, vous eussent accompagné en Ecosse ; c'eût été du moins quelque chose. Mais on dit qu'un tel cortège paroîtroit extraordinaire, dans ces temps où l'on introduit mille usages insensés, qui tous tendent à détruire la dépendance naturelle des vassaux à l'égard de leurs seigneurs." — Sir Edouard, pour corriger le plus possible l'influence des mœurs nouvelles, avoit attaché à son neveu, par toutes sortes de bienfaits, les jeunes gens qui devoient le suivre à l'armée. Le moment de la séparation s'approchant, il fit entrer Edouard\setcounter{page}{541} dans sa bibliothèque, et lui remit avec grand appareil une lettre entourée, suivant l'ancien usage, d'un petit cordon de soie et soigneusement cachetée du sceau des Waverley, sur le couvert de laquelle étoient couchés fort au long tous les titres et qualifications, tant de celui qui l'avoit écrite, que de celui à qui elle étoit adressée, et de celui qui en étoit le porteur, lui disant, que comme son neveu et son héritier présomptif, il ne pouvoit visiter l'Ecosse sans porter à son respectable ami, le Baron de Bradwardine, cette lettre de recommandation.
Quant au Chapelain, Mr. Pembroke, il crut réparer tous les vides de l'éducation qu'il avoit donnée à Edouard, en lui remettant les copies, faites pour lui de sa propre main, de deux gros manuscrits de sa composition, sur l'état actuel de l'Eglise Anglicane, manuscrits pleins d'une controverse si abstruse et de principes politiques si réprouvés, qu'aucun libraire n'avoit voulu les imprimer. Mais Edouard, à l'aspect de toute cette écriture fine et serrée, fut saisi d'un tel respect, qu'il déposa soigneusement ces volumes au fond de sa male et ne les regarda plus.
Les adieux avec la tante Rachel furent courts et tendres. Elle prémunit son cher\setcounter{page}{542} Edouard contre la séduction des beautés Écossaises. "Il y a bien, dit-elle, dans le nord de l'isle, quelques familles anciennes, mais ce sont tous des Whighs et des Presbytériens, excepté cependant les Montagnards. Et chez ces derniers, ajouta-t-elle, on ne peut pas s'attendre qu'il y ait beaucoup de délicatesse parmi les femmes, lorsque le costume habituel des hommes est aussi singulier qu'on le prétend." A ces mots, lui ayant donné sa bénédiction, elle lui fit présent d'une belle bague de diamant, telle qu'on en portoit beaucoup alors, et d'une bourse remplie de grosses pièces d'or, lesquelles, dit l'Auteur en terminant un chapitre, étoient aussi plus communes il y a soixante ans en Angleterre qu'elles ne l'ont été dernièrement.
Waverley, après avoir commencé sa nouvelle carrière avec beaucoup de zèle, s'ennuia bientôt des détails minutieux qui tombent en partage aux officiers subalternes, et comme la petite ville d'Ecosse où il étoit en garnison n'offroit aucune ressource de société, il demanda un congé de quelques semaines pour aller remettre la lettre de recommandation que lui avoit donnée son oncle. Il partit à cheval accompagné d'un seul domestique, et s'approchant insensiblement des montagnes.\setcounter{page}{543} du Perthshire, après avoir vu leurs contours bleuâtres se dessiner légèrement à l'extrémité de l'horizon, il se trouva au pied de ces masses gigantesques dont le front sourcilleux semble menacer toute la plaine. C'était auprès de cette barrière formidable, mais encore dans le plat pays, qu'habitaient les Barons de Bradwardine, s'il en faut croire les vieilles chroniques, depuis les jours du gracieux roi Duncan.
Ici commence cette peinture si remarquable des montagnes d'Écosse et du pays qui les avoisine immédiatement. Et d'abord l'Auteur décrit avec la vivacité de couleurs qui lui est propre, le pauvre village de Tully Veolan. L'on voit les misérables huttes qui le composent dispersées sur un terrain inégal, et d'énormes tas de tourbe et de fumier déposés aux deux côtés de chaque porte. L'on entend les aboiements d'une multitude de chiens \footnote{Ces chiens sont tellement un trait caractéristique de ces villages, qu'au dire de l'auteur, un voyageur Anglais a imprimé que l'État entretenoit, dans chaque village Écossais, un relais de chiens, pour donner la chasse aux chevaux de poste qui étoient si épuisés et si affamés qu'ils ne pourroient avancer sans ce genre d'aiguillon.}qui quittent chacun leur fumier pour courir après les chevaux des voyageurs,\setcounter{page}{544} et les cris des petits enfans que viennent enlever de vieilles femmes en furie, pour qu'ils ne soient pas écrasés sous les pieds des chevaux. "Deux ou trois jeunes paysannes revenant du bord du ruisseau avec des seaux sur leurs têtes, offrent des objets plus agréables, et leur jupe courte, leurs bras et leurs jambes nues, leur front découvert et leurs cheveux tressés, rappellent les figures des paysages Italiens. Un amateur du pittoresque auroit admiré l'élégance de leur costume et la régularité de leurs traits; quoiqu'à dire le vrai, un bon Anglais qui priseroit sur-tout le bien être eût désiré des vêtemens plus simples, des jambes et des pieds mieux garantis de l'humidité, un teint mieux préservé du soleil, et un emploi plus abondant de l'eau de la source sur les habits et sur toute la personne. - Toute cette scène avoit quelque chose de triste, car elle prouvoit, au premier coup-d'œil, une stagnation d'industrie et peut-être de pensée. Cependant, la physionomie des habitans,examinée de plus près, n'indiquoit point l'indifférence de la stupidité; leurs traits étoient rudes mais animés, graves mais intelligents, et, parmi les femmes, plus d'unmodèle de Minerve se seroit offert à un artiste; les enfans eux-mêmes, dont la peau\setcounter{page}{545} étoit brûlée et les cheveux décolorés jusqu'au blanc par le soleil, avoient le regard perçant et une expression pleine de vie et d'intérêt. Il sembloit en tout que la pauvreté et sa compagne trop fréquente l'indolence, eussent en vain combiné leurs forces pour déprimer le génie naturel d'une race hardie, robuste et réfléchissante."
Le château de Tully Veolan, demeure des Barons de de Bradwardine, est dépeint avec un détail un peu prolixe, mais qui produit bien l'impression de la chose. En lisant cette description, qui tient près de deux chapitres, on se représente la fatigue qu'on éprouveroit à parcourir les longues avenues, les cours nombreuses, les édifices de tout genre de cette ancienne habitation. La maison avoit été bâtie à une époque où les châteaux forts n'étoient plus nécessaires, mais où les architectes Ecossais ne connoissoient pas encore l'art de tracer le plan d'une résidence paisible, et une multitude de fenêtres fort étroites, ainsi que le toit formant une infinité de projections et d'angles saillans, surmontés chacun d'une petite tour, lui donnoient un aspect très singulier. Quoique l'ensemble n'en fût pas militaire, divers moyens de défense avoient été mis en usage\setcounter{page}{546} contre les bandes errantes des Bohêmes, ou contre les incursions dévastatrices des Montagnards. Ainsi les murailles étoient la plupart crenelées, et munies de meurtrières pour y placer de la mousqueterie; et les écuries de toute espèce, étoient tellement fortifiées de barreaux de fer qu'elles en prenoient l'apparence de prisons. Dans un des coins du mur d'enclos, s'élevoit un immense pigeonnier de forme circulaire, ressource alors très-utile aux Lairds Écossais, dont les revenus étoient à tout moment diminués par les contributions que levoient les brigands des montagnes. Mais ce qui sur-tout frappoit les regards, des deux côtés du portail de toutes les avenues, au-dessus de toutes les fontaines, sur les corniches de toutes les fenêtres, à la base de toutes les tours, et terminoit les formes hyperboréennes de toutes les constructions, c'étoient des ours grands et petits, sculptés en haut ou bas-relief, avec la devise de la famille en caractères gothiques, *Beware the Bar, prenez garde à l'Ours*. Ces mêmes ours se retrouvoient à l'extrémité des murs dans toutes les terrasses, où des ifs soigneusement taillés les reproduisoient encore. L'ensemble de cette scène avoit quelque chose de vénérable, de tranquille et presque de monastique, qui plût singulièrement à\setcounter{page}{547} L'imagination de Waverley, et le temps prodigieux qu'il lui fallut pour se faire entendre d'aucune créature vivante, excepté des pigeons effrayés qui s'envolaient en foule de leur antique rotonde, transporta sa pensée dans les châteaux enchantés des poëtes et des romanciers.
La première figure humaine qu'il rencontra, fut une espèce de fou grotesquement vêtu, nommé David Gelatlie, personnage vivant des bontés du Baron, et qui, par sa niaiserie mêlée d'esprit et par la bizarrerie de ses chansons, rappelle assez les Clown de Shakespeare. Mais bientôt paroît sur la scène Cosmo Comyne de Bradwardine, véritable type des Barons de la basse Ecosse. il y a soixante ans. De ces nobles auxquels une longue résidence en France, à la suite de Jacques II, avoit donné une teinte de politesse cérémonieuse, qui prêtoit un voile léger à la pédanterie d'une éducation de barreau, alors à la mode en Ecosse, à tous les préjugés de rang et de naissance, et à des habitudes impératives que fortifioit sans cesse une autorité non disputée dans des domaines à demi-sauvages. A ces traits généraux, le Baron de Bradwardine joignoit des prétentions particulières à l'érudition, et une telle passion pour les classiques.\setcounter{page}{548} qu'ayant été fait prisonnier en 1715, à Preston où il combattoit pour les Stuart, après avoir réussi à s'échapper, il étoit revenu sur ses pas pour chercher son Tite Live, circonstance qui avoit aidé à lui faire obtenir sa grâce. D'ailleurs c'étoit le plus humain des seigneurs, le plus chaud des amis, et le tendre père d'une jeune et belle personne, Rose de Bradwardine, qui faisoit tout le bonheur de sa vie.
Aux premiers mots que Waverley adressa à Cosmo Comyne, la joie de voir le neveu de son ami dérangea un peu la gravité du Baron. Les yeux du vieillard se remplirent de larmes et, après avoir secoué la main d'Edouard à la manière Anglaise, il l'embrassa cordialement à la mode Française\footnote{Tous les mots italiques sont en français dans l'original.}. "Foi de gentil-homme, dit-il, je redeviens jeune en vous voyant, Mr.Waverley. Un digne rejeton de la vieille tige de Waverley - Honour — spes altera, comme dit Virgile, et vous avez tout le port de l'ancienne race! capitaine Waverley. — Pas tout à fait aussi solennel que mon vieil ami sir Everard, mais cela viendra avec le temps, comme disoit un Hollandais de mes amis, le baron\setcounter{page}{549} "Kikilbroeck, de la sagesse de Mde. son épouse. Et vous avez donc arboré la cocarde, fort bien, fort bien, quoique j'eusse mieux aimé une couleur différente, et c'est ainsi, je suppose, qu'eût pensé sir Everard. — Mais parlons d'autre chose, je suis vieux et les temps ont changé."
Il nous est difficile de rendre justice à ce caractère très-original du Baron, parce que des quatre langages dont se composent ses discours, l'anglais, le bas écossais, le latin et le français, nous ne pouvons faire ressortir que le latin, ce qui donne au personnage une teinte plus pédante qu'il ne l'a dans l'ouvrage même.
L'Auteur s'est moins arrêté sur le portrait de Rose Bradwardine. C'est tout simplement une beauté ingénue qui manifeste quelque goût pour la musique et la littérature : Édouard lui donne, dans ces deux genres, des leçons auxquelles elle prend un dangereux plaisir.
Le vieux Baron déploie en faveur du neveu de son ami, toute la profusion de l'hospitalité Écossaise. Et d'abord il lui fait préparer un grand repas, dans lequel assez de cérémonie ne nuit ni à la cordialité, ni à la joie bruyante, ni même finalement à l'intempérance. Là sont invités tous les originaux des environs, seigneurs voisins et principaux vassaux du Baron, et cette peinture est excellente.\setcounter{page}{550} A la fin du repas, le vin ayant depuis longtemps coulé à grands flots, le Baron fait un signe à Saunders Saunderon son sommelier, et celui-ci, qui sort de la chambre d’un air d’intelligence, y rentre bientôt après, avec un souris mystérieux et solennel, et place devant son maître une petite cassette d’un bois poli, garnie d’ornemens de cuivre très-singuliers. Le Baron, tirant alors de sa poche une petite clef, ouvre la cassette, en relève le couvercle, et met au jour un gobelet d’or d’une structure ancienne et curieuse, représentant un Ours rampant; figure que le possesseur regarda avec un tel mélange de respect, d’orgueil et de plaisir que Waverley ne put s’empêcher de rire. Le Baron s’étant alors tourné vers son hôte, le pria de bien examiner ce monument des temps anciens. "Ce gobelet représente, dit-il, le cimier favori de notre famille, un Ours, et un Ours rampant, comme vous pouvez l’observer; car tout adepte dans l’art Héraldique, donne à chaque animal sa plus noble posture. Ainsi un cheval est représenté se dressant; un lévrier courant, et un animal rapace comme l’Ours, in actus ferociori, au moment où il déchire, où il dévore sa proie. Or, Monsieur, nous tenons ces armes honorables de l’Empereur\setcounter{page}{551} Frédéric Barberousse... et quant à la coupe, capitaine Waverley, elle a été fabriquée par ordre de St. Duthac, abbé d'Aberbrothock, pour un Baron de la maison de Bradwardine, qui avoit vaillamment défendu le patrimoine de ce monastère contre l'avidité de certains nobles. On la nomme proprement l'Ours béni des Bradwardines (quoique le vieux docteur Doublet eût coutume de l'appeler en riant Ursa major) et elle passoit pour posséder certaines vertus mystiques et surnaturelles. Je ne donne assurément pas dans de tels anilia, mais il n'est pas moins certain que c'est un gage précieux, et une sorte d'appanage de notre maison. Nous ne la faisons paroître que dans les grandes fêtes, et telle est à mon avis l'arrivée de l'héritier de sir Everard dans mon château". Après ces mots, ayant porté solennellement la santé de l'illustre maison des Waverley, il vida la coupe d'un seul trait.
Edouard vit avec effroi l'Ours redoutable, que remplissoit à chaque fois le sommelier, passer de main en main et s'approcher de lui; et il se disoit beware the bar. Mais après qu'il eut vidé la coupe, on ne le tint pas quitte. L'on alla finir la soirée dans une auberge à L'enseigne de la Poule, et, suivant le mot de\setcounter{page}{552} L'auteur \footnote{Nous n'en garantissons assurément pas le bon goût non plus que de bien d'autres, mais il faut cependant risquer quelque chose. Nous ne connoîtrons jamais les étrangers, si nous repoussons toujours leur plaisanterie, car c'est dans ce genre que la physionomie nationale se prononce de la manière la plus marquée comme la plus amusante. (R)}, le peu de raison qui avoit échappé à l'Ours ayant été englouti par la Poule, l'on chanta, l'on se disputa, l'on se battit, et le Baron, qui fut ramené à grand peine chez lui par Wawerley, ne voulut pourtant jamais se mettre au lit sans avoir fait, en faveur de cette soirée, une docte apologie, dans laquelle, toutefois, il n'y avoit pas un mot d'intelligible, si ce n'est quelque chose sur les Centaures et les Lapithes.
Le divertissement de la chasse fut celui du lendemain. Le Baron, monté sur un cheval bien dressé, avoit une selle à la française et une housse assortie à sa livrée. Son habit brodé de couleur gaie, sa veste à raies d'argent, sa perruque à la brigadière, surmontée d'un petit chapeau retroussé à galon d'or, lui donnoient tout l'air d'un chasseur de la suite de Louis XIV. C'est ainsi qu'accompagné de deux domestiques armés de pistolets, il parcourut avec Wawerley les côteaux et les vallées, à la grande admiration de tous les fermiers du voisinage.\setcounter{page}{553} Bien des jours s'écoulèrent de la sorte et ils ne furent pas dénués de plaisir. Rien n'était, sans doute, plus opposé que les caractères d'Édouard et du Baron; l'un avait quelque chose de poétique, d'impétueux, d'inattendu dans ses idées, l'autre se plaisait à parcourir la vie, avec la gravité roide et empesée qui caractérisait sa promenade sur la terrasse de Tully Veolan. Toutefois leurs cœurs les mettaient d'accord, et Wawerley trouvait d'ailleurs une agréable variété dans la société de Mlle. Rose. Ce n'est pas que les charmes de cette jeune personne eussent fait sur lui une impression profonde. Belle et intéressante comme elle l'était, elle n'avait pas précisément le genre de beauté et de mérite qui, dans la première jeunesse, captive une imagination exaltée. "Comment était-il possible de trembler, de se prosterner, d'être en adoration devant la jeune fille naïve, qui tantôt demandait à Edouard de lui retailler sa plume, tantôt de lui débrouiller la construction d'une stance du Tasse, tantôt même de lui aider à épeler un mot bien long... bien long... dont elle ne savait que faire. Il vient un temps où tous ces détails ont leur charme, mais ce n'est pas à l'entrée de la vie, à cette époque où l'on aime à revêtir l'objet de ses affections d'une sorte de merveille.\setcounter{page}{554} veillleux, et à se sentir attiré dans une région élevée par un Etre supérieur. Ainsi, quoi qu'il n'y ait pas de règle pour une passion aussi capricieuse , le premier amour est souvent ambitieux dans son choix."
Mais ce qui mettait Edouard à l'abri du danger, étoit précisément ce qui y exposoit Rose. Elle admiroit, elle aimoit chaque jour davantage son jeune instituteur, et tout le voisinage croyoit déjà que le Baron ne fermoit point les yeux sur les suites naturelles de cette intimité, mais qu'il les ouvroit sur les avantages d'une aussi riche alliance. En cela, le voisinage se trompoit. Rien n'étoit plus loin de l'esprit du Baron que les projets de mariage et d'établissement. Il s'occupoit sur-tout si peu de la fortune de Rose, que son régisseur, le baillif Macwheeble, avoit inutilement voulu lui persuader de consulter un avocat, sur les moyens de laisser à sa fille une grande partie de ses possessions, substituée au plus proche héritier mâle. C'étoit même avec une satisfaction perverse, qu'il répétoit souvent que la baronie de Bradwardine étoit un fief mâle, soumis à une espèce de loi Salique. "Siéroit-il à une femme, s'écrioit-il, et sur-tout à une Bradwardine, d'être employée in servitio excuendi seu detrahendi calligas regis post\setcounter{page}{555} battaliam." C'est-à-dire, à tirer les bottes du Roi après le combat ; service féodal attaché à la baronnie de Bradwardine."A Dieu ne plaise, que je déroge aux intentions de mes pères, et que j'empiette sur les droits de mon parent, Malcolm de Bradwardine d'Inchgrabit, branche honorable, quoique pauvre, de ma propre famille."
Edouard était depuis six semaines à Tully Veolan, lorsqu'un matin, il voit de toutes parts des signes d'un trouble extraordinaire. Des laitières à jambes nues courent çà et là, des seaux vides à la main, en faisant des exclamations de surprise et de douleur. Le baillif Macwheeble arrive au grand trot de son petit cheval blanc, suivi d'une troupe de paysans hors d'haleine, et le Baron en personne arpente à grands pas sa terrasse, avec une physionomie tellement obscurcie par l'orgueil et l'indignation, qu'Edouard n'ose pas l'interroger. Enfin, étant entré dans le sallon, il y trouve sa jeune amie, Mlle. Rose, qui, sans avoir l'air furieux ni important comme les autres, était évidemment inquiète et pensive.
"Vous nous voyez dans un moment de trouble, Mr. Waverley. Un parti de Cathérans a fondu sur nous cette nuit, et nous a enlevé tout notre troupeau de vaches,..\setcounter{page}{556} "Un parti de Cathérans?", "Oui, les brigands des montagnes... Nous en avons été complétement délivrés tant que nous avons payé un tribut à Fergus Mac Ivor Vich Jan Vohr, mais mon père a jugé indigne de son rang de s'y soumettre plus long-temps. Et voilà pourquoi ce malheur est arrivé.— Ce n'est pas la perte du bétail, Mr. Waverley, qui me tourmente, c'est l'idée que mon père est horriblement blessé de cet affront, et il est si violent et si brave... Il voudra reconquérir son troupeau par la force, et alors s'il ne lui arrive rien à lui-même, on tuera peut-être quelqu'un de ces sauvages, et il n'y aura point de fin aux vengeances.— Et à présent que le gouvernement nous a enlevé toutes nos armes.— Ah! Monsieur, je ne sais ce que nous deviendrons!..." Ici la pauvre Rose perdit tout-à-fait courage et répandit un torrent de larmes.
Le Baron entra dans ce moment, et gronda sa fille avec plus d'âpreté que Waverley ne lui en 'avoit encore vu."N'est-il pas honteux, dit-il, qu'elle se montre devant vous comme une fermière de village qui pleure de misérables vaches. Non, capitaine Waverley, cechagrin ne doit venir que de ce qu'elle voit le domaine de son père exposé à de tels affronts, tandis qu'il ne nous a pas été laissé\setcounter{page}{557} une demi douzaine de mousquets pour nous défendre ou nous venger."
Là-dessus entre Macwheeble, qui rend compte de l'état de la place. L'on tient conseil, et différens avis sont débattus avec vivacité. On constate l'impossibilité d'attaquer, avec quelque avantage, douze Montagnards armés jusqu'aux dents à leur manière, et très-redoutables à tous égards. — Le Baron, qui pendant ce temps parcouroit la chambre dans une indignation silencieuse, s'arrête enfin devant le portrait rembruni d'un terrible guerrier à moustaches et s'écrie. "Mon ayeul que voilà, capitaine Waverley, avec deux cents chevaux, qu'il leva sur ses propres domaines, dispersa et mit en fuite ( lors des dissentions civiles, l'an de grace 1642 ) plus de cinq cents de ces Montagnards, qui ont toujours été pour leurs voisins petra scandali une pierre d'achoppement. Et à présent, moi, Monsieur, moi, son petit-fils, on me traite avec cette indignité !..." Il se tut, et il régna un silence imposant.
Le Baron veut absolument rassembler tous les gentilshommes d'alentour, afin de poursuivre les brigands, et de leur faire subir le sort de leur prédécesseur Cacus; mais Macwheeble lui prouve, qu'à l'heure qu'il est, ils ont déjà gagné leurs retraites inaccessibles\setcounter{page}{558}, et le conseil se sépare sans avoir rien conclu. Resté seul avec Rose, Waverley lui demande quelle est cette espèce de Vieux de la montagne, auquel on paie un tribut pour s'affranchir du pillage. Il apprend avec une grande surprise, que non-seulement c'est un noble d'ancienne race, mais un homme très-distingué par ses manières, dont la sœur, nommée Flora, jeune personne charmante, a été l'amie de Rose. Il est chef d'un puissant Clan des montagnes voisines, mais quelles sont ses relations avec les voleurs, c'est ce que l'on ne peut pas expliquer."Ce qu'il y a de certain, dit Rose, c'est que l'on ne vole pas un agneau à ses protégés. Au dernier rassemblement du Comté, cet homme, également fier de sa naissance et de son pouvoir, avait voulu prendre le pas sur tous les nobles de la plaine, et mon père seul s'y étant opposé, le chef de Clan lui avait dit avec hauteur, qu'il n'était que son tributaire, car, ajouta-t-elle, Maewheeble, qui connait la fierté de mon père, lui avait fait payer cet argent à son insu. Mon père donc, vivement irrité, nia qu'il payât aucun tribut, et la dispute qui s'ensuivit se serait terminée par un duel, si Fergus Mac Yvor n'avait pas dit, avec une\setcounter{page}{559} grace très-noble, que jamais il ne léveroit la main contre une tête aussi vénérable et aussi respectée. — Vous voyez cependant,"Monsieur, ce qui est résulté de là."
Ici Rose se tut, et après un moment de silence, elle reprit en ces termes :
"Je vous conjure, Monsieur, d'user de"votre pouvoir sur mon père pour tâcher"d'apaiser cette affaire. Jamais nous n'avons eu un moment de repos tant que nous avons été en mauvaise intelligence avec les"Montagnards. Je n'avois encore que dix ans, lorsqu'il y eut un combat entre un parti d'une vingtaine de ces hommes et mon père, à la tête de ses domestiques. Et les balles cassèrent nos carreaux de vitre au nord, tant ils étoient près... On tua trois Montagnards, et leurs camarades nous les apportèrent enveloppés dans leurs manteaux, et ils les couchèrent sur le pavé de pierre"du vestibule.— Et voilà que le lendemain arrivent leurs femmes et leurs filles, poussant des hurlemens affreux, et elles emportèrent les cadavres en chantant leur coronach ( chant de mort ) avec des gestes"épouvantables, les joueurs de cornemuse à leur tête.— Pendant plus de six semaines, je n'ai pu m'endormir sans me réveiller en sursaut, croyant entendre ces cris terribles,\setcounter{page}{560} et voir ces corps couchés sur les degrés, tout roides, et roulés dans leurs tartanes \footnote{Étoffe à carreaux des manteaux écossais qui là est prise pour le manteau même.} sanglantes." Waverley ne put s'empêcher de frémir à l'ouïe de cette histoire. Il crut voir réalisées les visions fantastiques au milieu desquelles il avait tant vécu. Devant lui étoit une jeune fille de seize à dix-sept ans, la douceur et la naïveté mêmes, et elle avoit de ses propres yeux été le témoin d'une scène, telle qu'il avoit coutume de les évoquer quand il vouloit se retracer les siècles les plus barbares. Il éprouvoit à la fois l'impulsion de la curiosité et ce léger sentiment de danger qui en exalte la vivacité.– Il pouvoit dire avec Malvolio, " je ne me prends pas moi-même pour dupe jusqu'à me laisser subjuguer par mon imagination. .... , – Me voici dans la terre des aventures guerrières et romanesques, et il ne me reste qu'à déterminer le rôle que j'y jouerai. . . . . (La suite au prochain Cahier.)
\setcounter{page}{522}
\chapter{Romans}
\section{PRIDE AND PREJUDICE. Orgueil et préjugé. Roman en 3 vol. Londres 1813. \large{Second extrait. Voy. p373 de ce vol.}}
(Mr. Collins rebuté par les rigueurs d'Elisabeth, porte ses vœux à Charlotte Lucas, qui l'épouse. Quelque temps après son établissement dans le presbytère, elle engage Elisabeth à venir y demeurer. Elisabeth part pour s'y rendre avec Mr. Lucas et sa fille cadette. )
Tous les objets, dans ce petit voyage, étaient nouveaux pour Elisabeth, et tout l'intéressait. Elle se sentait heureuse, car elle avait vu sa sœur si bien portante, qu'elle n'avait plus la moindre crainte pour sa santé. Quand la voiture quitta la grande route pour la plaine de Hunsford, Elisabeth et ses compagnons cherchèrent des yeux, avec empressement, l'habitation du Recteur. On ne voyait encore que les palissades du parc de Rosings ; et Elisabeth sourit en se rappelant tout ce qu'elle avait ouï dire à son cousin sur les habitans de ce château.
Enfin ils découvrirent le presbytère, dont\setcounter{page}{523} le jardin venoit en pente douce jusqu'à la route. La maison étoit dans le jardin même. La palissade peinte en vert, et la haie de laurier le firent reconnoître d'après la description qu'on en avoit faite. Mr. Collins et Charlotte se montrèrent devant la maison. La voiture s'arrêta vis-à-vis d'une petite porte. Les maîtres accoururent en faisant des signes de tête et des sourires. On descendit de voiture, on s'embrassa, on se félicita. Elisabeth s'applaudit de plus en plus de sa résolution en se voyant reçue si cordialement de son amie. Elle s'aperçut dès le premier instant, que les manières de son cousin n'étoient pas changées par son mariage : c'étoit toujours la même politesse empesée. Il les retint deux minutes à la grande porte pour s'informer en détail de toute la famille. Il eut soin de faire remarquer aux arrivans la propreté et la bonne apparence extérieure de son habitation. Lorsqu'ils furent dans le sallon, il recommença ses félicitations sur leur arrivée, et eut soin de répéter les offres de rafraîchissement que sa femme venoit de faire.
Elisabeth s'étoit préparée à voir son cousin dans sa gloire; et lorsqu'il lui fit remarquer les belles proportions de son sallon, elle se représenta qu'il comptoit exciter en\setcounter{page}{524} elle des regrets ; mais elle ne lui donna pas la satisfaction de la voir soupirer : elle éprouvoit plutôt de l'étonnement sur l'air heureux de Charlotte, après avoir épousé un homme comme celui-là. Toutes les fois que Mr. Collins disoit quelque chose de plat ou de gauche, elle ne pouvoit s'empêcher de regarder sa femme. Deux ou trois fois elle crut la voir rougir ; mais en général Charlotte avoit le bon esprit de ne pas entendre. Après qu'ils eurent été assis quelques momens, et qu'on eut passé en revue tous les meubles du sallon, Mr. Collins proposa un tour dans le jardin, à la culture duquel il présidoit ; il mettoit même quelquefois la main à l'œuvre : "c'étoit ,, disoit-il, "un de ses plaisirs les plus respectables." Charlotte fit bonne contenance, et loua les avantages de l'exercice en plein air, ajoutant qu'elle l'encourageoit de son mieux. Mr. Collins ne fit grace de rien dans la revue de son jardin. Il connoissoit la contenance de tous les champs des environs, le nombre de tous les arbres qui les bordoient ; mais, des divers points de vue de son jardin, celui qui le charmoit le plus, parce qu'il flattoit ses sentimens d'admiration respectueuse pour lady de Bourg, c'étoit une échappée qui lui laissoit apercevoir le château de Rosings.
\setcounter{page}{525}
Après avoir montré le jardin, Mr. Collins vouloit entreprendre le tour des prés; mais les dames n'ayant pas des souliers qui pussent braver l'humidité, revinrent sur leur pas, et sir William l'accompagna seul. Charlotte eut le temps de faire voir la maison en détail à son amie et à sa sœur. Elle fut probablement charmée de n'avoir pas son mari pour l'aider dans cette exhibition. La maison étoit petite, mais bien bâtie et commode: tout y étoit arrangé avec un ordre et une propreté qui faisoient honneur à la maîtresse de la maison. Abstraction faite de Mr. Collins, Élisabeth trouvoit que son amie pouvoit être assez heureuse; et à en juger par l'air de contentement de Charlotte, il paroisssoit qu'elle réussissoit assez bien à l'oublier.
On avoit déjà parlé de milady Catherine. Lorsqu'on fut à table, Mr. Collins reprit ce sujet et dit. "Oui, ma cousine, vous aurez la satisfaction de la voir dimanche prochain à l'église. Vous en serez enchantée, ma cousine. Vous verrez qu'elle est toute affabilité et condescendance, et peut-être vous adressera-t-elle la parole après le service divin. Je ne doute point qu'elle ne vous invite, ainsi que ma sœur Marie, avec nous, la première\setcounter{page}{526} fois qu'elle nous fera l'honneur de nous proposer de dîner à Rosings. Elle est admirable dans sa conduite avec ma chère Charlotte. Nous dînons deux fois la semaine à Rosings, et jamais on ne nous permet de revenir chez nous à pied. La voiture de milady est toujours à nos ordres. Je devrois dire une des voitures de milady, car elle en a plusieurs."
"C'est une personne d'un grand mérite, véritablement, ajouta Charlotte; et on ne peut pas être meilleure voisine qu'elle ne l'est. "
"Vous dites comme moi, ma chère Charlotte : c'est une femme qui mérite toutes sortes de respects et d'hommages. "
La soirée se passa à répéter ce qu'on avoit déjà écrit des nouvelles du Herefordshire. Elisabeth retirée dans son appartement, eut le temps de méditer sur la position de Charlotte, sur l'adresse, la bonne humeur, et le bon esprit, qu'elle savoit mettre à conduire et supporter son mari.
Le lendemain, vers midi, pendant qu'elle se préparoit à une petite promenade, elle entendit au rez-de-chaussée un bruit singulier; puis elle vit arriver Marie qui venoit toute essoufflée, et lui cria depuis la porte :" Venez vite, ma chère Liza! venez vite! vous\setcounter{page}{527} verrez quelque chose que je ne veux pas vous dire, mais venez tout de suite! "Elisa demanda en vain ce que ce' pouvoit être, et elle suivit Marie à la salle à manger, d'où elle virent deux femmes dans une calèche.
"Est-ce là tout?" s'écria Elisabeth. "Je ne vois que lady Catherine et sa fille." "Mais, ma chère, comment pouvez-vous prendre miss Jenkinson pour lady Catherine. L'autre est miss de Bourg. Mais regardez, je vous prie, quelle petite créature! Je n'aurois jamais pu croire qu'elle fût si petite et si mince."
"Elles ne sont pas trop polies ces dames, de tenir ainsi Charlotte sur ses jambes, et déhors de la maison par ce vent froid. Je ne comprends pas pourquoi elle n'entre pas."
Oh! miss de Bourg n'entre presque jamais. C'est une grande faveur lorsqu'elle veut bien entrer. "J'aime sa tournure," dit Elisabeth, qui pensoit à Darcy et au mariage projetté, "elle a l'air maussade et orgueilleuse, c'est bien une femme pour lui."
Mr. Collins et Charlotte étoient debout près de la calèche, en conversation avec ces dames; et ce qui amusoit sur-tout Elisabeth, étoit de voir Sir William placé à une\setcounter{page}{528} distance respectueuse, et toujours prêt à s'incliner lorsque miss de Bourg regardoit de son côté. Enfin la voiture partit; et les deux jeunes personnes allèrent à la rencontre de Mad. Collins. Son mari les félicita sur le bonheur qu'elles avoient d'être invitées à dîner chez lady Catherine pour le lendemain. C'étoit un véritable triomphe pour lui que cette invitation. Il se promettoit une grande jouissance de l'étonnement de ses hôtes quand ils verroient la magnificence du château de Rosings, et la politesse de lady de Bourg avec lui et sa femme. Il avoit une extrême reconnoissance pour milady, de ce qu'elle lui donnoit ainsi l'occasion de faire parade de la faveur dont elle l'honoroit.
"Je n'aurois point été surpris ," dit - il, si que lady Catherine nous eût proposé à tous de boire du thé chez elle un dimanche. J'avoue même que, connoissant son extrême affabilité', je m'y attendois ; mais comment aurois-je pu espérer une marque d'attention et de bonté aussi flatteuse ? Comment prévoir qu'immédiatement après votre arrivée , nous serions tous invités à dîner au château de Rosings ?"
"Je suis moins surpris que vous," dit Sir William, " parce que j'ai l'habitude de l'extrême politesse des grands. J'ai toujours remarqué\setcounter{page}{529} que les gens de qualité sont les plus polis. On apprend cela en vivant à la cour."
Il n'y eut presque point d'autre conversation pendant le reste du jour, et le lendemain matin, que ce qui avoit rapport à l'invitation de Rosings. Mr. Collins eut soin de représenter aux deux jeunes personnes, la grandeur des appartemens, la richesse des meubles, et le nombre des valets, afin qu'un coup-d'oeil si éblouissant ne leur fit pas perdre au premier moment, la présence d'esprit nécessaire. — Quand les dames se séparèrent pour aller faire leur toilette, Mr. Collins dit à Elisabeth. "Il ne faut pourtant pas, ma cousine, que vous ayez de l'embarras à l'égard de votre parure. Lady Catherine n'exige point chez des gens de notre classe, cette élégance qui convient à elle et à sa fille. Soyez mise aussi bien que le comportent vos moyens, c'est tout ce qu'il faut. Lady Catherine vous saura gré de votre simplicité : elle aime que la distinction des rangs soit observée en tout."
Tandis que les dames s'habilloient, Mr. Collins vint deux ou trois fois à leur porte, leur recommander d'être prêtes à l'heure, parce que milady tenoit beaucoup à ce que le dîner ne fût jamais retardé. La pauvre Marie Lucas commençoit à prendre une\setcounter{page}{530} peur terrible de cette milady, et la perspective du dîner lui donnait autant d'émotion qu'en avait eu son père lorsqu'il avait été présenté à la cour.
Le temps était beau; et ils s'acheminèrent à pied par les allées du parc. Elisabeth en admira la belle végétation et les points de vue, mais elle ne pouvait pas éprouver les mêmes transports que Mr. Collins, ni toute l'admiration qu'il chercha à lui inspirer, en lui racontant ce qu'il en avait coûté seulement pour les vitres, de toutes les croisées du château.
Lorsqu'ils montèrent le perron, et qu'ils entrèrent dans le vestibule, Marie sentoit croître ses alarmes. Sir William n'étoit pas très-calme. Pour Elisabeth, elle étoit à-peu-près dans son assiette ordinaire. Elle n'avoit rien entendu dire de lady Catherine qui la lui représentât comme très-imposante par l'esprit, les talens ou les vertus. Quant à l'avantage de la richesse, elle n'en étoit pas du tout éblouie.
Mr. Collins ne manqua pas de faire admirer les proportions du vestibule, et le bon goût des ornemens. Deux laquais les conduisirent par une antichambre, dans un salon, où étoient lady Catherine, sa fille, et Mad. Jenkinson. Milady eut la politesse de\setcounter{page}{531} se lever pour recevoir les dames ; et comme il avoit été arrangé entre Charlotte et son mari , que ce seroit elle qui seroit chargée de la présentation , cela se passa plus simplement que si Mr. Collins s'en étoit mêlé .
Quoique Sir William eût été , disoit-il , à St. James , il fut tellement abasourdi de l'air de grandeur de tout ce qui l'entouroit , qu'il ne lui resta que le courage de faire un profond salut , et de s'asseoir sans dire un mot .
Pour Marie , elle ne savoit plus où elle en étoit . Elle s'assit sur le bord d'une chaise , les yeux fixés à terre , sans oser regarder à droite ni à gauche . Quant à Elisabeth , elle supporta fort bien cette épreuve , et elle observa de sang-froid , les trois dames avec lesquelles elle faisoit connoissance . Lady Catherine étoit une grosse et grande femme . Elle avoit des traits marqués , et pouvoit avoir été belle . Elle manquoit de douceur dans le regard et dans le son de voix . Sa manière de recevoir les dames n'étoit pas propre à les mettre à leur aise . Ce n'étoit pas son silence qui étoit imposant ; mais quand elle parloit , c'étoit d'un ton d'autorité , et d'importance , qui rappeloit à Elisabeth tout ce que Mr. Wickham en avoit dit . Elle trouvoit aussi de certains rapports entre lady de Bourg et Mr. Darcy.
\setcounter{page}{532}
Elle se mit ensuite à examiner miss de Bourg. Celle-ci étoit pâle et maigre; elle avoit une physionomie sans expression; elle parloit peu; mais elle chuchotoit souvent à l'oreille de Mad. Jenkinson, qui étoit toute occupée de l'écouter, et de placer un écran, de manière que miss de Bourg ne fût pas incommodée du feu.
Après quelques minutes, milady dit aux jeunes demoiselles de regarder la vue; et elle eut soin de les avertir, qu'en été, elle étoit plus belle. Mr. Collins expliquoit, en détail, les beautés du paysage.
Le dîner fut splendide. L'abondance et la recherche des plats, le luxe de la vaisselle, le nombre des laquais, tout cela fut comme l'avoit promis Mr. Collins. Ainsi qu'il l'avoit annoncé, milady le fit gracieusement placer au haut de la table, et il avoit l'air de croire que le monde entier ne pouvoit lui offrir un succès plus glorieux. Il découpoit, il servoit, il mangeoit; il vantoit chaque plat avec une vivacité qui tenoit de l'enthousiasme. Sir William répétoit fidèlement les éloges donnés à chaque mets; car peu-à-peu il avoit repris une sorte de présence d'esprit, et il mangeoit comme un autre.
Lady Catherine paroissoit se plaire à l'admiration de ces messieurs, et sourioit avec\setcounter{page}{533} complaisance toutes les fois que la nouveauté d'un plat piquoit leur curiosité, et les faisoit redoubler de louanges pour l'habileté du cuisinier. La conversation n'étoit d'ailleurs pas nourrie. Elisabeth, placée entre Charlotte et miss de Bourg, auroit bien voulu causer; mais la première étoit uniquement occupée d'écouter milady, et miss de Bourg ne dit pas une parole de tout le repas. Mad. Jenkinson ne pensoit qu'à ce que mangeoit ou ne mangeoit pas miss de Bourg, et témoignoit sans cesse la crainte qu'elle ne fût indisposée, puisqu'elle n'avoit point d'appétit. Pour Marie, elle n'imaginoit pas qu'on fût à table pour causer.
On rentra dans le salon. Milady se mit alors à énoncer son opinion sur divers sujets, d'une manière si tranchée, qu'on voyoit bien qu'elle n'étoit pas accoutumée à la contradiction. Elle s'informa en détail des petits intérêts du ménage de Mad. Collins, et lui donna ses conseils sur chaque chose, en lui indiquant de quelle manière il convenoit de régler une maison avec un revenu modique. Elle parla même des soins de la laiterie et de la basse-cour. Elisabeth comprit que rien n'étoit indigne de l'attention de lady Catherine lorsqu'il s'agissoit de régenter ceux qui l'entouroient. Dans les intervalles\setcounter{page}{534} de sa conversation avec Mad. Collins, elle fit plusieurs questions à Marte et à Elisabeth, sur-tout à celle-ci, dont elle connoissoit moins la famille, et qui lui paroissoit ( dit-elle à demi voix à Charlotte ) une petite personne assez gentille. Elle lui demanda à plusieurs reprises combien elle avoit de sœurs; si elle étoit l’ainée ou la cadette; si elles étoient jolies; si elles avoient reçu une éducation soignée; si elles avoient l’espérance de se marier; quelle espèce d’équipage avoit son père; et quel étoit le nom de famille de sa mère. Elisabeth sentoit toute l’impertinence de ces questions, mais elle y répondoit tout comme si elles eussent été convenables.
"La propriété de Longbourn," reprit lady Catherine, "est substituée à Mr. Collins, je crois. J’en suis fort aise pour vous," ajouta-t-elle en se tournant du côté de Charlotte, "mais, en général, je ne vois pas pourquoi on substitue les biens aux mâles, dans les familles. Chantez-vous, miss Bennet? jouez-vous de quelqu’instrument?"
"Un peu, milady."
"En ce cas-là, nous serons bien aise de vous entendre une fois ou l’autre. Ma fille a un piano qui est assez remarquable : vous aurez du plaisir à le jouer. Vos sœurs chantent-elles?\setcounter{page}{535}  sont-elles musiciennes?"
"Il y en a une qui l'est."
"Une seulement? et pourquoi pas les autres? Il auroit fallu montrer à toutes. Les demoiselles Webs jouent toutes du piano, et cependant leur père n'a pas plus de revenu que le vôtre. Dessinez-vous?"
"Point du tout."
"Quoi! aucune des sœurs?"
"Aucune."
"C'est bizarre! point de maître à portée, je suppose? Votre mère auroit dû vous mener à Londres tous les printemps pour les leçons."
"Ma mère en avoit bien quelqu'envie, mais mon père n'aime pas Londres."
"Votre gouvernante vous a-t-elle quittées?"
"Nous n'avons pas eu de gouvernante."
"Ah! ah! quoi? pas de gouvernante du tout? Cinq filles élevées sans gouvernante, c'est un peu fort! votre mère étoit donc à la chaîne?"
Elisabeth eut peine à s'empêcher de sourire, et assura que sa mère n'avoit point été à la chaîne.
"Et qui donc vous donnoit des leçons? Il est impossible que vous n'ayez pas été négligées."
\setcounter{page}{536}
"Comparativement à certaines familles, cela peut être; mais je vous assure, milady, que celles d'entre nous qui ont voulu s'instruire, en ont eu suffisamment les moyens. Nous avons été encouragées à l'étude, et on ne nous a jamais refusé les maîtres nécessaires."
"Si j'avois connu votre mère, j'aurois pu lui rendre service. Je dis, qu'en éducation, on ne fait rien sans une instruction régulière et suivie; et il n'y a qu'une gouvernante qui puisse la donner. C'est incroyable combien j'ai placé de jeunes personnes pour gouvernantes, dans des maisons où elles font fort bien. J'ai procuré d'excellentes places aux quatre nièces de Mad. Jenkinson, et l'autre jour encore, je fis recevoir dans une maison une jeune fille qu'on m'avoit nommée par hasard, et qui réussit à merveilles. Vous ai-je dit, Mad. Collins, que lady Metcalfe est venue hier me remercier de cette miss Pope. "Mais c'est que vous m'avez donné un trésor, lady Catherine!" a-t-elle répété plusieurs fois. — Vos sœurs cadettes sont-elles déjà dans le monde, miss Bennet?"
"Oui, milady, toutes."
"Toutes! quoi? toutes cinq dans le monde déjà? et vous n'êtes que la seconde? Oh! c'est fort drôle! répétez-moi donc cela. Les\setcounter{page}{537} cadettes sont dans le monde, et les aînées ne sont pas encore mariées! Elles doivent être bien jeunes, vos cadettes."
"Oui, il y en a une qui n'a pas seize ans. Elle est un peu jeune, pour être beaucoup dans le monde; mais, il me semble qu'il seroit dur pour les cadettes, d'être privées des avantages de la société, parce que les aînées n'auroient pas eu l'occasion ou la volonté de se marier. Les cadettes ont, ce me semble, les mêmes droits que les aînées aux plaisirs de la jeunesse; et si on les en privoit par un motif de cette nature, ce seroit, je crois, un mauvais moyen de maintenir l'union entre les sœurs."
"Mais, mais, mais! vous donnez votre opinion d'une manière bien tranchée pour une jeune personne. Quel âge avez-vous, je vous prie?"
"Avec trois sœurs cadettes qui sont dans le monde," dit Elisabeth en souriant, "je dois avoir un peu de répugnance à dire mon âge."
Milady parut toute étonnée de ne pas recevoir une réponse directe à sa question; et Elisabeth se douta qu'elle étoit la première qui eût osé employer la plaisanterie devant une impertinence si imposante."
"Vous n'avez pas besoin de cacher votre\setcounter{page}{538} âge, car vous ne pouvez pas avoir plus de vingt ans, j'en suis sûre."
"Je n'ai pas tout-à-fait vingt-un ans."
Après le thé, on apporta des tables de jeu. Sir William, Mr. et Mad. Collins, firent la partie de lady Catherine. Miss de Bourg eut la fantaisie de faire un casino, et les deux jeunes demoiselles eurent l'honneur de jouer avec elle. Mad. Jenkinson fit la quatrième. La partie fut d'un ennui mortel. Il ne se dit pas un mot qui n'eût rapport au jeu, excepté les questions de Mad. Jenkinson à son élève pour s'assurer qu'elle n'avoit ni trop chaud ni trop froid; ni trop, ni trop peu de jour. A l'autre partie, il se dit beaucoup plus de paroles. Lady Catherine tenoit le dez, et racontoit toutes sortes d'anecdotes qui la concertoient, ou bien elle récapituloit les fautes que les trois autres faisoient au jeu. Mr. Collins approuvoit de la voix ou du sourire, tout ce que disoit milady, et s'inclinoit profondément toutes les fois qu'il lui gagnoit une fiche. Pour Sir William, il ne dit pas grand'chose : il étoit occupé de se meubler la mémoire de grands noms et d'anecdotes qu'il pût redire.
Lorsque lady Catherine fut lasse de jouer, on proposa la voiture à Mr. Collins. Il l'accepta humblement. On la demanda tout de\setcounter{page}{539} suite ; et en attendant qu'elle fût prête, milady leur apprit à tous quel temps il devoit faire le lendemain. On prit congé avec des remerciemens infinis de la part de Mr. Collins. Dès qu'on fut monté en voiture, il demanda à Elisabeth si elle n'étoit pas enchantée ; celle-ci , par égards pour son amie, en dit moins qu'elle ne pensoit sur le compte de milady ; mais Mr. Collins reprit bientôt la parole pour se charger lui-même de l'éloge de sa protectrice.
( Les deux neveux de lady Catherine, Mr. Darcy et le colonel Fitz-William sont venu demeurer au château de Rosings, pendant le séjour d'Elisabeth au presbytère où ils ont fait une visite. )
Les dames admirèrent le ton et les manières du colonel Fitz-William , et se promirent que sa présence au château rendroit les séances des dîners et des thés plus agréables. Mais il se passa huit jours entiers sans qu'aucune proposition parvînt au presbytère. Enfin le dimanche au sortir de l'église, on invita Mad. Collins et ses amies pour le soir même.
Milady les reçut avec politesse, mais il étoit facile de voir qu'elle n'avoit plus autant besoin qu'auparavant de la société de ces dames. Elle ne faisoit guère d'attention\setcounter{page}{540} qu'à ce que disaient ses neveux, surtout Darcy.
Le colonel, qui s'ennuyait ordinairement beaucoup à Rosings, était charmé de la diversion que faisaient les dames du voisinage; et la jolie petite amie de Mad. Collins, (comme il appelait Elisabeth) lui plaisait beaucoup. Il fut donc charmé de la voir arriver. Il s'assit à côté d'elle avec empressement, et ils se mirent à causer sur le Herefordshire, sur Kent, sur les voyages, sur la lecture et sur la musique. Elisabeth s'accommodait fort bien de cette conversation; elle répondait avec vivacité, et ne s'était jamais si bien amusée dans ce salon. Darcy, qui se tenait un peu loin, tournait souvent les yeux de leur côté, et paraissait s'étonner de la promptitude des reparties, et de l'abondance des matières, quoiqu'il n'entendit pas ce qui se disait. Lady Catherine fit attention à cette causerie animée, et dit à son neveu, en élevant la voix: "Fitz-William! qu'est-ce que vous dites donc vous autres, là-bas?" "Nous parlons musique, milady." "Eh bien, parlez haut, qu'on vous entende. C'est un de mes sujets favoris de conversation que la musique. Je ne crois pas qu'il y ait, en Angleterre, beaucoup de femmes qui aient plus de goût naturel que\setcounter{page}{541} je n'en avois pour la musique. Si j'avois travaillé, je serois devenue très-forte. Anne est dans le même cas. Si sa santé lui avoit permis d'étudier, elle auroit joué délicieusement. Comment va Georgiana sur le piano, Darcy ?"
"Je suis fort content de son application et de ses progrès."
"Je suis bien aise de cela. Dites-lui de ma part, je vous prie, que ce n'est qu'à force de travailler qu'on réussit."
"Je vous assure, milady, qu'elle n'a pas besoin d'être pressée : elle travaille beaucoup."
"Tant mieux! on ne sauroit trop travailler quand il s'agit de piano. La première fois que je lui écrirai, je l'encouragerai à ne pas se négliger. Je le répète souvent aux jeunes demoiselles : on ne devient très-habile qu'à force de jouer. Je l'ai dit à miss Bennet : elle ne jouera jamais bien, si elle ne travaille beaucoup. Mr. Collins n'a pas de piano; mais je lui ai dit qu'elle pouvoit venir tous les jours si elle vouloit, pour jouer sur le piano de ma fille chez Mad. Jenkinson. Elle n'embarrassera personne, parce que c'est dans l'appartement de la gouvernante.
\setcounter{page}{542}
Mr. Darcy rougit du défaut de tact et de l'impertinence de sa tante.
Quand on eut pris le thé, le colonel Fitzwilliam rappela à Elizabeth qu'elle lui avoit promis de lui jouer quelque chose. Elle se mit aussitôt au piano, et joua des variations. Il se plaça à côté d'elle, et parut prendre beaucoup de plaisir à l'entendre. Milady écouta quelques momens, puis se remit à parler à Darcy, comme si l'on n'eût point fait de musique. Mais bientôt son neveu lui échappa, et vint s'établir en face d'Elisabeth pour l'entendre et la fixer tout à son aise.
"Je comprends, lui dit-elle, en souriant avec un peu de malice," que vous vous établissez-là pour me faire peur. Mais je vous déclare que je suis décidée à ne pas avoir peur, quoique miss Georgiana joue mieux que moi. J'ai une espèce de courage qui s'obstine à ne point se laisser abattre à la volonté des autres; et je vous avertis que je ne suis jamais moins timide que quand on cherche à m'intimider."
"Je ne vous dirai pas, mademoiselle, que vous vous trompez; parce que vous ne pouvez pas croire que j'aie eu, en effet, l'idée de vous intimider. Je sais que vous vous amusez quelquefois à dire autrement que vous ne pensez."
\setcounter{page}{543}
"Bon!" dit-elle, en riant, "voilà une jolie recommandation. Votre cousin," ajouta-t-elle en s’adressant au colonel, "vous apprend, en passant, qu’il ne faut pas croire un mot de ce que je dis. C’est chanceux! de se voir faire une réputation de ce genre, quand on arrive dans un canton où on n’est pas encore connu. En vérité, Mr. Darcy, vous auriez bien dû être plus discret sur ce que vous avez appris dans le Herefordshire. C’eût été plus politique aussi; car si je m’avisois de recriminer, gare!"
"Je n’ai pas peur," dit Darcy en souriant.
"Voyons! voyons!," interrompit le colonel, "racontez-nous quelque chose de lui. Je voudrois savoir comment il se conduit quand il n’est pas sous les yeux de ses parens."
"Eh bien, je m’en vais vous le dire. Préparez-vous à entendre des choses effroyables. La première fois que j’ai rencontré Mr. Darcy, c’étoit au bal. Et que croyez-vous qu’il fit à ce bal? Il ne dansa que quatre fois: pas davantage! et cependant on manquoit d’hommes, et il y avoit plus d’une jeune personne qui auroit été charmée de danser avec lui. Ce sont des faits que vous ne pouvez nier, Mr. Darcy."
"Je n’avois point l’honneur de connoître ces dames."
\setcounter{page}{544}
"Ah, sans doute! et on ne trouvoit pas qu’il valût la peine de se faire présenter. Monsieur Fitz-William, je suis à vos ordres: que voulez-vous que je vous joue?"
"J’ai toujours peu de curiosité," dit Darcy, "pour faire de nouvelles connoissances; et je sens que je suis maussade avec les gens que je ne connois pas."
"Faut-il lui en demander à lui-même la raison dit-elle en s’adressant à Fitz-William."
"Je vous le dirai bien sans qu’il s’en mêle," répondit le colonel: "c’est qu’il ne veut pas se donner la peine d’être aimable."
"Que voulez-vous!" dit Darcy, "je n’ai pas le talent de certaines personnes pour me mettre d’abord au ton des étrangers; pour paroître prendre intérêt à ce qui les concerne, et pour m’amuser de ce qu’ils me disent."
"Je sens bien, dit Elisabeth que mes doigts ne sont ni aussi rapides ni aussi légers, sur le piano, que ceux de quelques personnes que j’ai entendues. Je ne crois pas pour cela que mes doigts ne pussent acquérir la même légèreté et la même vitesse, si je me donnois la peine d’étudier. Darcy sourit, et lui dit, "mademoiselle, vous avez fort bien fait de ne pas étudier davantage, car vous avez employé\setcounter{page}{545} votre temps beaucoup mieux. Personne, parmi ceux qui ont le bonheur de vous entendre ne peut trouver qu'il manque rien à votre jeu. Ni vous ni moi n'avons une exécution destinée aux étrangers."
Lady Catherine qui s'impatientoit de ce qu'on pouvoit faire la conversation sans elle, demanda ce qu'on disoit. Alors Elisabeth se remit à jouer du piano. Milady s'approcha; et après avoir écouté une minute, elle dit à Darcy : "Elle ne doigte point mal du tout; et je vous assure que si elle avoit été bien montrée, elle joueroit à faire plaisir. Pour le talent et le goût, je ne connois personne comme ma fille. Si sa santé lui avoit permis de travailler, elle auroit été d'une force très distinguée."
Elisabeth regarda Darcy, pour savoir s'il entroit dans cet éloge de sa cousine. Mais elle ne sut pas discerner le plus léger symptôme d'incli\-nation. Lady Catherine continua à parler piano et exécution, tranchant sur bien des choses qu'elle n'entendoit pas, et donnant des avis et des leçons à Elisabeth, qui les recevoit avec autant de politesse et de modestie qu'elle savoit le faire. Tous les intervalles d'un morceau à l'autre étoient remplis de cette manière, et toujours Mr. Darcy et le colonel redemandoient quelque\setcounter{page}{546} chose de nouveau. Enfin milady sonna, et fit mettre les chevaux à sa voiture pour renvoyer les dames au presbytère.
(Dans une conversation qu'Elisabeth a avec le colonel Fitz-William, à la promenade, celui-ci raconte que Bingley a de grandes obligations à Darcy pour l'avoir empêché de faire un mariage d'inclination peu convenable. Elle comprend qu'il est question de sa sœur aimée, dont elle a reçu une lettre fort triste ; et elle veut beaucoup de mal à Darcy de s'être mêlé de cette affaire. Elle prend la chose tellement à cœur qu'elle en est malade pendant deux ou trois jours.)
Le lendemain Elisabeth étoit occupée à écrire à sa sœur pendant que Mad. Collins et Marie étoient allées au château lorsqu'elle 'entendit sonner à la porte.
Sa première idée fut que c'étoit le colonel Fitz-William, qui déjà une fois étoit venu un peu tard dans la soirée, et qui peut-être passoit pour s'informer de sa santé. Mais elle fut bien étonnée en voyant entrer Mr. Darcy lui-même. Il lui demanda d'un air d'empressement et d'embarras des nouvelles de sa santé, et s'excusa de sa visite sur l'extrême désir qu'il avoit d'apprendre d'elle même qu'elle étoit mieux. Elle répondit avec une politesse froide. Il s'assit ; resta quelques\setcounter{page}{547} moments en silence ; puis il se leva, et se promena dans la chambre.
Elisabeth était de plus en plus étonnée, et ne disait pas un mot. Plusieurs minutes se passèrent ainsi. Enfin il s'approcha d'elle, et lui dit d'une voix émue : "J'ai fait tous mes efforts pour me taire ; et je ne le puis. Je ne puis plus contraindre l'expression de mes sentimens. Je suis comme forcé de vous dire combien je vous admire et vous aime."
L'étonnement d'Elisabeth ne peut pas se rendre. Elle rougit ; elle le regardait fixement ; et elle doutait si ce n'était point un rêve. Elle ne répondit pas. Il prit cela pour un encouragement suffisant, et il lui dit d'un ton passionné tout ce que depuis long-temps il avait éprouvé pour elle. Il fut très éloquent ; mais il y avait une partie de sa conduite qu'il lui était difficile d'expliquer sans blesser la fierté d'Elisabeth. Il avoit eu, dit- il, des préjugés à combattre, soit en lui-même, soit chez ses parens, pour se décider à une alliance qui ne réunissait pas tout ce à quoi il pouvait prétendre ; et il essayait de justifier ainsi le retard de cet aveu qui venait de lui échapper.
Quoiqu'Elisabeth eut pris une sorte d'aversion pour Darcy, elle ne put être tout-à-fait insensible au sentiment que lui manifestoit\setcounter{page}{548} un homme aussi distingué. Elle n'hésitoit pas sur ce qu'elle avoit à répondre ; mais elle se faisoit quelque peine de lui donner du chagrin.
Telle étoit sa disposition pendant la première partie du discours de Darcy ; mais lorsqu'il vint à en faire valoir la résolution qu'il prenoit de se mésallier en quelque sorte, en fixant son choix sur elle, Elisabeth éprouva un mouvement de véritable colère. Elle eut pourtant assez de présence d'esprit pour se préparer à répondre avec politesse. Les derniers mots de Darcy articulés d'un ton de confiance ; son attitude, son expression en terminant son discours, indiquoient assez qu'il ne craignoit pas un refus.
Au moment où il eut cessé de parler, Elisabeth sentit que la rougeur lui montoit au visage.
Je crois, monsieur, qu'il est d'usage en pareil cas de témoigner de la reconnoissance pour de tels sentiments, lors même que l'on ne peut point y répondre. Si je pouvois éprouver, en effet, cette reconnoissance, je vous la manifesterois; mais je ne le puis pas. Je n'ai jamais ambitionné de faire aucune impression sur vous, et je n'ai rien à me reprocher à cet égard. Je suis fâchée de faire\setcounter{page}{549} du chagrin à qui que ce soit; mais j'espère que le sentiment pénible que vous pourrez éprouver de mon refus, ne sera pas de longue durée. Ces mêmes préjugés qui vous ont fait hésiter si long-temps, pourront servir à votre consolation. "
Darcy étoit debout devant elle, appuyé contre la cheminée. Des sentimens confus de surprise, d'humiliation, de chagrin et sur-tout de dépit, le rendoient muet. Il étoit pâle de colère, et n'osoit dire un mot, de peur de trahir son émotion. Enfin, il dit avec un calme affecté: " est-ce là tout ce que je puis espérer de vous? J'aurois peut-être le droit de demander pourquoi je me vois ainsi rejeté sans qu'on cherche même à employer avec moi les égards de la politesse; mais cela est de fort peu d'importance. "
" J'aurois autant de droit pour vous demander, monsieur, pourquoi, avec l'intention évidente de blesser ma fierté, vous venez me dire que vous m'aimez contre votre volonté, votre raison, et même aux dépens de votre réputation? Si je n'ai pas été polie, vous m'en avez fourni vous-même l'excuse, mais j'ai d'autres raisons de mécontentement, et vous les connoissez. Si mon sentiment ne vous avoit pas été absolument contraire, si vous m'aviez été indifférent, ou que même\setcounter{page}{550} j'eusse eu quelque penchant pour vous, pen- sez-vous donc qu'aucune considération eût pu m'engager à accepter la main d'un hom- me qui a détruit, peut-être pour jamais, le bonheur d'une sœur chérie?"
Lorsqu'elle prononça ces mots, Mr. Darcy changea de couleur, et fit un mouvement , mais il n'essaya pas de l'interrompre; et elle continua avec véhémence.
"J'ai toutes les raisons du monde de pen- ser du mal de vous. Aucun motif quelcon- que ne peut excuser la conduite injuste et peu généreuse que vous avez eue dans cette affaire. Vous n'oseriez, vous ne pouvez pas nier que vous avez été le principal agent de leur séparation; que vous avez exposé l'un des deux à la censure du public comme un être capricieux et léger, et l'autre à l'espèce de ridicule qui accompagne un mécompte de ce genre; qu'enfin vous les avez rendus l'un et l'autre parfaitement malheureux."
Elle s'arrêta. Elle s'attendoit à une repli- que; et lorsqu'elle jeta les yeux sur Darcy, elle fut indignée de voir qu'il la regardoit avec un calme apparent, et sans témoigner le moindre remords. Elle crut même le voir sourire, et elle lui dit vivement: "Pouvez- vous nier que ce soit à vous qu'est due leur séparation?"
\setcounter{page}{551}
"Je n’ai pas la moindre envie de nier que j’ai fait pour mon ami ce que j’aurois dû faire pour moi-même."
Elisabeth ne fit pas semblant de comprendre ce que cette expression avoit de désobligeant pour elle ; mais elle n’en fut pas mieux disposée pour Darcy. "Ce n’est pas seulement sur cette affaire que je vous ai jugé," continua-t-elle avec la même vivacité. "Mon opinion sur votre compte étoit depuis long-temps arrêtée, d’après la connoissance que j’ai eue de votre conduite avec Mr. Wickham. Sur ce sujet, que pouvez-vous avoir à dire ? Est-ce aussi pour rendre service à quelqu’un que vous avez été injuste avec lui ?"
"Vous prenez un intérêt bien vif à ce monsieur là !" reprit Darcy en rougissant beaucoup.
"Il est impossible de ne pas s’intéresser à lui quand on connoît ses malheurs."
"Ses malheurs," dit Darcy, d’un ton de mépris.
"Oui, ses malheurs, dont vous êtes la cause. C’est vous qui l’avez réduit à la pauvreté, en lui retirant des avantages que vous saviez fort bien lui être destinés ; en le privant de ce qu’il avoit mérité ; et après vous être conduit comme vous l’avez fait, vous\setcounter{page}{552} osez parler de lui avec mépris! vous avez la dureté de jeter sur lui du ridicule!
"Ah! ah! voilà donc l'opinion que vous avez de moi. Voilà la dose d'estime que vous m'accordez. Certes! en effet, à votre compte mes torts sont grands. Cependant vous me les auriez peut-être pardonnés, si je n'avais pas blessé votre orgueil, en vous confessant les scrupules qui ont long-temps retenu mon aveu. Je vois bien que j'aurois dû vous cacher mes combats intérieurs, et ne vous parler que de mon amour. Mais je ne sais point déguiser mes sentimens; il n'y avoit rien d'ailleurs que de très-naturel dans ce que j'ai éprouvé. Auriez-vous voulu que je m'estimasse heureux de m'allier dans une famille inférieure à la mienne? Pouviez-vous croire que je n'eusse pas quelque répugnance à former des relations dans une société si différente de celle à laquelle je suis habitué?"
"Elisabeth sentoit croître à chaque instant sa colère; mais elle fit tous ses efforts pour paroître de sang-froid, en lui disant:" vous m'avez épargné, monsieur, l'espèce de chagrin que j'aurois eu en vous refusant, si vous vous étiez adressé à moi avec la délicatesse qui appartient à un homme comme il faut."
Il fit un mouvement d'indignation: au reste, monsieur, continua-t-elle, "quelle\setcounter{page}{553} qu'eût été la forme de votre demande, je n'aurois pas été tentée de l'accepter."
Il la regarda avec un étonnement mêlé de mortification et d'incrédulité.
"Depuis le premier instant où je vous ai connu," continua-t-elle, "je vous ai jugé fort disposé aux prétentions, à l'arrogance, et peu occupé des autres. Au bout d'un mois j'en savois suffisamment sur votre compte, pour être sûre que vous étiez l'homme qui me convenoit le moins."
"Vous en avez bien dit assez, mademoiselle, pour vous faire comprendre tout-à-fait, et pour me donner une véritable honte du sentiment que j'ai éprouvé. Je vous souhaite beaucoup de bonheur, et je vous salue." Il sortit en achevant ces mots.
Aussitôt qu'il fut parti, Elisabeth se mit à pleurer; et pendant une demi heure, elle ne sut pas se rendre compte de tout ce qu'elle éprouvoit. Elle croyoit avoir fait un songe. Il lui paroissoit incroyable que Mr. Darcy lui eût fait une proposition de mariage ; qu'il eût été amoureux d'elle depuis long-temps, et tellement amoureux, qu'il eût passé sur toutes les objections qu'il avoit lui-même présentées à son ami. Mais quand elle réfléchissoit à son abominable orgueil, à l'impudence avec laquelle il sembloit se\setcounter{page}{554} vanter de ce qu'il avoit fait contre miss Jane; enfin, à la dureté insultante avec laquelle il s'exprimoit sur Mr. Wickham après tout le mal qu'il lui avoit fait, elle n'étoit point disposée à le plaindre ni à l'espèce de reconnoissance qu'elle auroit eue de son attachement pour elle. Elle demeura plongée dans ces réflexions jusqu'au moment où le bruit de la voiture lui rappela qu'elle n'étoit pas en état de supporter les regards observateurs de Charlotte; et elle courut s'enfermer dans sa chambre.
La suite au Cahier prochain.
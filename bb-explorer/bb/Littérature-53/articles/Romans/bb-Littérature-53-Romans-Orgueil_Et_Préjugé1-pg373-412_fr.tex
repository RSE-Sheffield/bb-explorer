\setcounter{page}{373}
\chapter{Romans}
\section{PRIDE AND PREJUDICE. Orgueil et préjugé. Roman en 3 vol. Londres 1813.}
C’est une vérité reconnue, qu’un jeune homme qui a de la fortune doit chercher à se marier. On sait si bien cela dans toutes les familles, que sans s’informer des projets et de la façon de penser d’un gentilhomme qui arrive dans un canton, il est déjà regardé comme acquis à une des familles du voisinage.
"Savez-vous que Netherfield est loué ?" dit un jour Mad. Bennet à son mari.
"Ah, ah !"
"Oui, il est loué ; c’est Mad. Long qui vient de me raconter tout cela."
Mr. Bennet garda le silence.
"Mais n’êtes-vous donc pas curieux de savoir à qui le château est loué ?,"
"Non, mais vous avez bien envie de me le dire."
"Eh bien, Mad. Long m’a assuré que c’étoit à un jeune homme fort riche. Il est venu lundi dans une voiture à quatre chevaux,\setcounter{page}{374} pour voir la maison ; et il a été si enchanté de tout, qu'il a d'abord passé sa location avec Mr. Morris. Il doit entrer à la St. Michel, et ses gens arrivent la semaine prochaine.
"Comment l'appelez-vous ?",
"Bingley."
"Est-il marié ?",
"Marié ! et non, sans doute, il n'est pas marié ; quelle question ! mais savez-vous qu'on lui donne quatre mille livres sterling de rente ? Ce serait joli pour nos filles, cela."
"Quel rapport cela peut-il avoir avec nos filles ?"
"Mais, mais, mais ! vous êtes insupportable. On dirait que vous ne savez pas que je veux les marier."
"Est-ce que ce monsieur vient dans ce canton pour se marier ?"
"Quelle absurdité ! il n'y vient pas tout exprés ; mais pourquoi ne deviendrait-il pas amoureux d'une de mes filles ! Vous irez lui faire une visite, n'est-ce pas ?"
"Je n'en vois pas la nécessité. Vous pourrez bien y aller sans moi avec les enfants, ou bien vous pourrez les y envoyer toutes seules ; car s'il allait devenir amoureux de vous, cela dérangerait tous les projets."
\setcounter{page}{375}
"Mon cher ami, je sais que j’ai été belles, mais qu’à présent je n’ai plus rien de très remarquable. Quand on a cinq grandes filles, il ne faut plus penser à sa beauté. Or çà ! vous irez voir ce Mr. Bingley dès qu’il arrivera, n’est-il pas vrai ?"
"C’est plus que je ne puis vous promettre."
"Mais pensez donc à vos filles. Quelle superbe affaire cela ne feroit-il pas ! Sir William et lady Lucas sont décidés à aller lui faire visite, avec un but semblable ; car vous savez qu’en général, ils ne voient personne. Vous comprenez bien que je ne pourrai pas y aller si vous n’y faites pas une visite."
"Mais... : Je crois que vous êtes trop scrupuleuse : il seroit charmé de vous voir. Je pourrai lui écrire quelques mots pour lui dire que je ne mets point d’opposition à ce qu’il épouse une de mes filles. Je lui recommanderai de préférence ma petite Lizzy."
"N’allez pas faire cela ! vous êtes toujours bien prévenu pour Lizzy. Jeanne est pourtant beaucoup plus belle, et Lydie est bien plus douce."
"Elles n’ont rien de bien extraordinaire ni les unes ni les autres. Elles sont sottes et ignorantes comme toutes les jeunes filles ; mais Lizzy a un peu plus de vivacité que ses sœurs."
\setcounter{page}{376}
"Est-il possible de se divertir comme vous faites à dire du mal de ses enfants. Vous avez bien peu d’égards pour mes nerfs; car vous savez le mal que cela me fait."
"Vos nerfs? Tout au contraire, il y a vingt ans que je les connois et les respecte beaucoup."
"Ah! vous ne savez pas ce que je souffre."
"Il viendra d’autres jeunes gens riches s’établir dans le voisinage, et cela vous fera du bien."
"Quand il en viendroit vingt, à quoi cela serviroit-il; puisque vous ne voulez pas faire une seule visite."
"Quand il y en aura vingt, j’irai les voir tous, je vous le promets."
Mr. Bennet offroit un si singulier mélange d’esprit, de malice, de réserve et d’humeur capricieuse, qu’après vingt-trois ans de mariage, sa femme ne le connoissoit pas. Pour elle, elle étoit facile à connoître: elle n’avoit point d’esprit, point d’instruction et beaucoup de vanité. Si quelque chose la contrarioit, elle se croyoit malade. La grande affaire de sa vie étoit de marier ses filles: en attendant, elle ne s’occupoit que de visites et de caquets.
Mr. Bennet fut un des premiers à aller\setcounter{page}{377} voir Mr. Bingley. Il en avoit toujours eu le projet, quoiqu'il eût dit le contraire à sa femme. Le soir du jour où il avoit vu l'étranger, il dit tout-à-coup à Lizzy qui arrangeoit un chapeau : j'espère que Mr. Bingley, le trouvera joli.
"Comment pourrons-nous savoir ce qu'il trouvera ou ne trouvera pas," dit la mère avec aigreur, "puisque nous ne le verrons point."
"Mais, maman, dit Elisabeth, vous oubliez que nous le verrons toujours aux assemblées. D'ailleurs Mad. Long a promis de nous le présenter."
"Mad. Long est une hypocrite. Elle s'en gardera bien : n'a-t-elle pas des nièces?"
Il y eut un silence. Mad. Bennet étoit de mauvaise humeur. Elle se mit à gronder Kitty de ce qu'elle toussoit. "Au nom de Dieu ne toussez donc pas comme cela, Kitty, vous me faites horriblement mal aux nerfs."
"Votre rhume n'a point de discrétion," dit Mr. Bennet.
"Ce n'est assurément pas pour m'amuser que je tousse," dit la jeune fille d'un ton sec.
"A quel jour le premier bal, Lizzy," dit Mr. Bennet.
"De demain en quinze."
"Il sera impossible," reprit Mad. Bennet,\setcounter{page}{378} que Mad. Long nous le présente, car elle n’arrivera que la veille, et ne le connaîtra pas."
"Eh bien ce sera nous qui le lui présenterons."
"Belle idée, présente-t-on les gens que l’on ne connaît pas ?"
"Je conviens que quinze jours ne suffisent pas pour connaître un homme à fond ; mais si vous ne le présentez pas, ce sera moi qui le présenterai."
Les cinq filles regardèrent leur père fixement pour savoir ce qu’il vouloit dire.
"Vous dites des absurdités," s’écria Mad. Bennet en colère.
"Des absurdités ? Je ne suis pas de votre avis. Et vous, Marie, qu’en pensez-vous ? vous êtes une personne de réflexion ; vous lisez beaucoup ; et vous faites des extraits. Voyons ce que vous pensez de tout cela."
Marie auroit voulu répondre quelque chose de bien sensé, mais elle ne sut pas comment s’y prendre.
"Pendant qu’elle cherche sa réponse," dit le père, "revenons à ce Mr. Bingley."
"Je suis ennuyée de ce sujet-là," dit la mère sèchement.
"Tant pis ! mais parbleu ! vous auriez bien dû me dire cela auparavant ; je ne serois\setcounter{page}{379} pas allé le voir. A présent nous ne pouvons pas éviter de faire connoissance avec lui."
- L'étonnement de toutes ces dames fut extrême.
"Ah! je savois bien que vous finiriez par me croire! s'écria Mad. B., vous êtes aimable, il n'y a rien à dire. Mais pourtant, quelle malice [de n'en pas dire un mot jusqu'à présent!
"Toussez, Kitty, toussez à votre aise, ma fille," dit Mr. Bennet en se retirant sans ajouter un mot.
"Quel excellent père vous avez, mes enfans!" s'écria-t-elle quand il fut sorti. Ce n'est pas une chose agréable que de faire tous les jours des connoissances nouvelles à notre âge; mais on fait des sacrifices pour ses enfans. Lydie, quoique vous soyez la cadette, ce Mr. Bingley dansera peut-être avec vous."
"Oh! je ne suis pas inquiète. J'ai beau être la cadette, je suis la plus grande de toutes."
Le reste de la soirée se passa à conjecturer le moment où l'étranger rendroit la visité, et à discuter sur le jour où il conviendroit de lui donner à dîner.
(Mr. Bingley vient s'établir à Netherfield\setcounter{page}{380} avec ses deux sœurs, dont l'une est mariée à Mr. Hurst. Un ami de la famille, Mr. Darcy, demeure avec eux. )
Les dames de Longbourn allèrent bientôt faire visite aux dames de Netherfield, et la visite fut rendue en toute cérémonie. Mad. Hurst, et miss Bingley goûtèrent les manières et les agréments de miss Jane Bennet. Elles convinrent que la mère étoit insupportable, et que les quatre sœurs cadettes étoient nulles. Elles témoignèrent néanmoins le désir d'être en relation avec la famille. Miss Jane fut sensible à cette attention. Elisabeth en fut peu flattée. Elle trouvoit dans la politesse de ces dames un fond de hauteur qui ressembloit à de l'impertinence; mais elle crut voir que sa sœur étoit déjà tellement prévenue pour Mr. Bingley qu'elle avoit un peu d'illusion sur ses sœurs. Elle en parla à son amie, miss Charlotte Lucas, en ajoutant que comme sa sœur avoit beaucoup de réserve dans les manières, et une bienveillance générale, elle ne donneroit pas prise sur elle aux observations malignes.
"La réserve est bonne," dit Charlotte; "parce qu'il faut, en effet, que le public ne se doute de rien, mais trop de réserve nuit quelquefois. Si l'on cache soigneusement à un homme le goût qu'on a pris pour lui,\setcounter{page}{381} on peut manquer l'occasion de le fixer: c'est alors une triste consolation que de n'avoir pas été devinée par le public. Il y a toujours de la vanité et de la reconnaissance dans l'attachement des hommes: il n'est point sûr de laisser agir leur cœur tout seul. Je trouve qu'il faut toujours que nous commencions un peu. Une légère préférence est une chose à laquelle personne ne peut trouver à redire, et si on veut qu'un homme prenne de l'amour tout de bon, il faut l'encourager. J'ai pour principe qu'on ne risque jamais rien à montrer un peu plus de sensibilité qu'on n'en a. Bingley a du goût pour votre sœur! ce n'est pas douteux; mais si elle ne lui aide pas, il en restera-là: il ne prendra pas une passion pour elle.,
" Il me semble qu'elle lui aide autant qu'elle comporte son caractère. Si je m'aperçois qu'elle a de la disposition à l'aimer, il faudrait qu'il fût bien bête pour ne pas s'en douter. "
- "Mais songez donc qu'il ne connoît pas le caractère de votre sœur comme vous."
- "Lorsqu'une femme a une préférence pour un homme et qu'elle ne prend pas soin de la lui cacher, il est impossible qu'il ne s'en doute pas. "
- "Cela serait bon s'ils se voyaient très-souvent;\setcounter{page}{382} mais quand les occasions sont rares, il faut profiter de toutes les minutes. Quand une fois, elle sera sûre de lui, elle aura tout le temps de l'aimer."
"Votre plan seroit fort bon pour une personne qui n'auroit que le projet de se marier richement; mais ma sœur n'a point de projet. Il n'y a que quinze jours qu'elle connoit Mr. Bingley. Elle a dansé et dîné trois ou quatre fois avec lui, et voilà tout."
"C'est-à-dire, ils ont passé la soirée ensemble."
"Qu'est-ce que cela fait pour connoître les gens? Ma sœur a vu qu'il aimoit mieux jouer au vingt et un qu'au commerce, c'est tout ce qu'elle peut savoir de son caractère."
"Soyez sûre que c'est bien assez. Si elle l'épousoit demain, elle auroit tout aussi bonne chance que si elle l'eût étudié un an. Le bonheur, en fait de mariage, est une affaire de pur hasard. On croit se connoître d'avance, et cela n'empêche pas de faire des découvertes désagréables et de se tourmenter ensuite réciproquement. Ne vaut-il pas mieux en savoir aussi peu que possible sur les défauts de celui qu'on doit épouser?"
"Ma chère Charlotte, vous plaisantez trèsagréablement; mais convenez que vous ne raisonnez pas; et que s'il s'agissoit de vous-même,\setcounter{page}{383} vous ne vous conduirez pas d’après ce système."
Elisabeth, toute occupée de faire des observations sur sa sœur, ne se doutoit pas qu’elle devenoit un objet d’intérêt pour Mr. Darcy. Au premier bal, il n’avoit pas voulu convenir qu’elle fût jolie. Au second, il fit la critique de sa figure; mais il n’eut pas plutôt prouvé à ses amis qu’elle manquoit de régularité dans les traits, qu’il fut obligé de convenir que son regard avoit beaucoup d’expression, que sa tournure étoit légère et gracieuse, et que quoiqu’elle n’eût pas les manières du grand monde, elle avoit le charme de la gaîté et de l’aisance. Quant à Elisabeth, elle ne voyoit en Mr. Darcy qu’un homme qui ne faisoit rien pour se rendre agréable, et qui n’avoit pas daigné lui proposer de danser. Cependant à une assemblée chez Sir Williams Lucas, Mr. Darcy parut curieux d’entendre causer Elisabeth sans se mêler à la conversation.
"Comprenez-vous," dit-elle ensuite à Charlotte, "pourquoi ce Mr. Darcy est venu nous écouter quand nous causions avec le colonel Forster, et que je le pressois de nous donner un bal?,
"C’est une question à laquelle il répondroit mieux lui-même."
\setcounter{page}{384}
"Si cela lui arrive encore, je lui montrerai que je m'en aperçois. Il a quelque chose dans le regard qui est fort satirique; et si je ne m'affranchis pas par un peu d'impertinence, je sens que j'aurai peur de lui."
Un moment après, Mr. Darcy s'approcha, sans avoir l'air de vouloir entamer la conversation. Charlotte dit tout bas à son amie qu'elle la défioit de commencer. Elisabeth encouragée prit un grand parti: et se tournant vers lui, elle lui dit: "Avez-vous été content, monsieur, de la manière dont je plaidons la cause du bal tout-à-l'heure avec le colonel Forster?"
"Avec beaucoup d'énergie, mademoiselle; et c'est un sujet qui en inspire infiniment aux dames."
"Vous êtes un peu sévère avec nous."
"C'est votre tour, ma chère d'être persécutée," dit Charlotte Lucas à Elisabeth.
"Je m'en vais ouvrir le piano, et vous savez ce que cela veut dire."
Elisabeth fit un peu de façons; mais enfin elle chanta deux airs assez agréablement. Sa sœur Marie lui succéda avec empressement au piano. C'étoit la plus forte des sœurs; parce qu'elle s'étoit exercée avec beaucoup de persévérance; mais comme elle manquoit\setcounter{page}{385} de talent naturel et de goût, elle jouoit sans agrément. Après un long concerto, écouté avec ennui, elle se mit à jouer des petits airs de danse Ecossais et Irlandais, et deux ou trois officiers dansèrent avec les dames.
Mr. Darcy gardoit le silence et paroissoit indigné de ce que des gens raisonnables pouvoient préférer ainsi un amusement qui excluait la conversation, lorsque Sir William Lucas, qui se trouvoit près de lui, lui dit: "c'est une jolie chose que la danse; ne trouvez-vous pas Mr. Darcy. C'est la plus charmante récréation des peuples civilisés."
"Ils la partagent avec tous les peuples sauvages."
"Votre ami Bingley danse à merveille, et vous êtes vous-même un excellent danseur."
"Vous m'avez vu danser à Meryton, je suppose."
"Et avec un très-grand plaisir. Vous avez dansé à la cour, je pense?"
"Jamais, monsieur."
"Et pourquoi donc cela?"
"Parce que jamais je ne danse quand je puis faire autrement."
"Je suppose que vous avez une maison à Londres, monsieur?"
Mr. Darcy s'inclina sans répondre:
\setcounter{page}{386}
"Je veux aussi avoir une maison à Londres; mais ce qui me retient, c'est que je ne suis pas sûr que lady William s'y porte bien."
Il espérait une réponse, mais Mr. Darcy se tut. Dans ce moment-là, Elisabeth s'approcha d'eux. Sir William la prit par la main, et dit à Mr. Darcy : "Permettez-moi de vous présenter une charmante danseuse." Elisabeth rougit et retirant sa main, elle dit que son intention n'étoit pas de danser. Mr. Darcy un peu surpris des soins officieux de Sir William, avançoit cependant la main, en proposant en toute forme à Elisabeth de danser; mais elle répéta son refus. Sir William se mit à la presser. "Vous dansez si bien, miss Eliza," lui dit-il, "que c'est vraiment une cruauté de refuser obstinément; et monsieur se seroit prêté à la chose, quoiqu'en général, il ne se soucie pas de danser."
"Je sais que monsieur est fort poli,"
"Assurément," dit Mr. Lucas; "mais dans ce cas-ci, convenez qu'il n'y auroit pas grand mérite."
Mr. Darcy gardoit le silence, avec cet air ennuyé qu'il avoit habituellement, quand miss Bingley vint lui dire à l'oreille. "Je parie que je devine à quoi vous pensez."
\setcounter{page}{387}
"Et à quoi done."
"Vous réfléchissez combien il seroit insupportable d'avoir à passer souvent de pareilles soirées. Tant d'importance et tant de nullité dans tous ces gens-là! on a l'ennui, et le bruit par dessus. Je voudrois bien entendre vos observations."
"Eh bien, vous n'y êtes pas du tout. J'étois-là à réfléchir au plaisir qu'on a à voir de beaux yeux et un joli visage."
Miss Bingley le regarda, toute étonnée, en lui demandant s'il étoit permis de savoir quelle étoit la personne qui lui avoit fait faire cette rare découverte.
Mr. Darcy répondit hardiment que c'étoit miss Elisabeth Bennet.
"Ah! ah!" s'écria miss Bingley avec surprise. "Voilà qui est nouveau; et je vous prie, quand pourra-t-on vous faire compliment?"
"Voilà précisément la question que j'attendois de vous. Il n'y a rien de si rapide que l'imagination d'une femme. Elle saute de l'admiration à l'amour, et de l'amour au mariage en un clin-d'œil."
"Fort bien! A votre manière de répondre, je vois que la chose est décidée. Ma foi! je vous félicite de tout mon cœur, sur la belle-mère que vous aurez; elle est aimable.\setcounter{page}{388} Je pense que vous l'aurez beaucoup à demeure à Pomberley. Ce sera charmant pour les voisins."
Mr. Darcy l'écouta avec une profonde indifférence, et elle continua à plaisanter sur ce ton-là.
(Miss Jane Bennet est invitée à dîner chez les dames de Netherfield. Mad. Bennet, qui a toujours en tête son projet de mariage, invente d'envoyer sa fille à cheval, parce que le temps menace de pluie et qu'elle espère qu'on gardera miss Jane au château jusqu'au lendemain. La jeune personne essuie en effet un orage et prend une fièvre catharrale. — On reçoit un message. Elisabeth, inquiète s'achemine le lendemain matin à pied par le mauvais temps pour aller soigner sa sœur à Netherfield.)
A cinq heures les dames se retirèrent pour s'habiller, et à six heures et demie on avertit Elisabeth que le dîner étoit servi. Tout le monde à-la-fois lui demanda des nouvelles de sa sœur; mais Mr. Bingley le fit avec un intérêt marqué. Elle répondit que sa sœur n'étoit point bien; et qu'elle continuoit à avoir de la fièvre. Là-dessus les sœurs répétèrent qu'elles en étoient bien fâchées, que c'étoit une chose fort ennuyeuse, que d'être malade; puis elles changèrent de conversation.\setcounter{page}{389} Elisabeth fut blessée de l'indifférence que ces dames montroient. En revanche elle remarqua que Mr. Bingley avoit une véritable inquiétude sur sa soeur. Il étoit absorbé et distrait. Mr. Darcy étoit occupé de la conversation de miss Bingley, à côté de laquelle il étoit assis. Mr. Hurst qui étoit auprès d'Elisabeth étoit un homme tout matériel, qui ne savoit que manger, boire et jouer, ensorte qu'elle trouva le dîner long. En sortant de table, elle rentra immédiatement dans l'appartement de sa soeur; et miss Bingley se mit à en dire du mal. "Cette petite personne," dit-elle, "est un mélange d'orgueil et d'impertinence. Elle n'a ni beauté, ni manières, ni conversation. Mad. Hurst convint de tout cela, puis elle ajouta :"elle a pourtant une qualité : elle est bonne mancheuse. Mon Dieu quelle drôle de mine elle avoit quand elle est arrivée ce matin. Avez-vous vu son air effaré. Je proteste qu'elle avoit l'air d'une folle."
"Vous avez raison, Louise, exactement l'air d'une folle avec ses cheveux ébouriffés."
"Et sa jupe donc ! avec un demi pied de crotte, un bon demi pied, j'en suis sûre. Et, puis sa robe par dessus qui ne cachoit rien."
"Ma foi," dit Bingley, "vous avez fait là\setcounter{page}{390} d'étonnantes observations. Pour moi elle m'a paru charmante quand elle est arrivée,"
"Et vous, monsieur Darcy," dit miss Bingley. "Aimeriez-vous que votre sœur arrivât ainsi toute crottée dans une maison où elle ne connoît personne?"
"Assurément pas."
"Je trouve que faire ainsi une lieue dans la boue toute seule, montre une sorte de hardiesse villageoise, un mépris des convenances qui est choquant."
"Je trouve," dit Mr. Bingley, que cela montre une grande affection pour sa sœur, et que cela n'est qu'intéressant."
"J'ai peur, dit miss Bingley à l'oreille de Mr. Darcy, que cette aventure n'ait un peu diminué votre admiration pour ses beaux yeux."
"Non, du tout; et au contraire: son regard étoit plus animé par l'exercice."
Il y eut un silence après lequel Madame Hurst prit la parole. "Cette miss Jane est vraiment très-bien. Je voudrois qu’elle se mariât convenablement; mais le père et la mère sont un furieux obstacle."
"Il me semble," reprit miss Bingley, "qu’il y a un oncle qui est procureur à Meryton."
"Et un autre à Cheep-side."\setcounter{page}{391} Diantré! cela est respectable ! et elles se mirent à rire.
"Quand tout le quartier de Cheep-side seroit peuplé de leurs oncles," dit Mr. Bingley, "elles n'en seroient pas moins très agréables."
"A la bonne heure, mais cela leur ôteroit la chance d'épouser des gens comme il faut," répondit Mr. Darcy.
Les dames appuyèrent cette observation, et continuèrent à s'amuser aux dépens des parens de leur chère miss Jane Bennet. Après quoi elles passèrent dans sa chambre, et lui firent des tendresses, jusqu'au moment où on les appela pour le café. Elisabeth ne vouloit pas descendre; mais sa sœur s'étant endormie, elle trouva convenable de paroître aussi dans le sallon. Tout le monde étoit au jeu quand elle entra. On lui proposa d'en être; mais craignant qu'on ne jouât cher, elle s'en excusa, et prit un livre. Mr. Hurst la regarda avec de grands yeux; puis il lui dit: "préférez-vous la lecture au jeu? cela seroit singulier, à votre âge."
"Miss Elisa Bennet," dit miss Bingley, "méprise les cartes: elle ne trouve du plaisir qu'à lire."
"Je ne mérite," dit Elisabeth, "ni cet éloge ni cette censure. Je n'aime point la\setcounter{page}{392} lecture par dessus tout; et il y a beaucoup de choses auxquelles je trouve du plaisir." "Vous en trouvez sûrement," dit Mr. Bingley, "à soigner votre sœur." Elisabeth fut sensible à son compliment; et elle s'approcha d'une table où il y avait quelques livres, pour les regarder. "C'est vous," dit miss Bingley à Mr. Darcy, " qui avez une belle bibliothèque à Pemberley." "Cela doit être, parce que c'est l'ouvrage de plusieurs générations. D'ailleurs vous achetez des livres sans cesse. C'est un bel endroit que ce Pemberley. Mon frère, quand vous bâtirez, et planterez, il faut imiter Pemberley." Elisabeth vint s'asseoir entre Mr. Bingley et sa sœur aînée pour regarder jouer. "Est-ce que miss Darcy est grandie depuis ce printemps?" dit miss Bingley. "Mais oui. Elle est à-peu-près de la taille de miss Elisabeth Bennet." "Elle est charmante, votre sœur! Elle a une expression délicieuse, des manières nobles, et tout plein de talents. Elle joue du piano comme un ange." "C'est toujours une chose qui m'étonne," dit Mr. Bingley, que de voir toutes les jeunes demoiselles remplies de talents."
\setcounter{page}{393}
"Toutes les jeunes demoiselles, dites-vous?, "Mais oui assurément. Je n'entends jamais nommer une jeune personne pour la première fois, sans entendre aussi vanter ses talents. L'une fait des ouvrages superbes en broderie, une autre fait des peintures admirables pour les écrans, ou d'autres choses semblables."
"Vous dites fort bien, Bingley," reprit Darcy; "mais les vrais talents sont une chose très-rare, et il est encore plus rare d'y voir réunies l'instruction, la modestie et les graces."
Elisabeth observa qu'elle n'avoit pas encore rencontré une telle réunion. Les deux sœurs se récrierent et dirent que cela se voyoit cependant assez souvent. Comme elles s'animoient là-dessus, Mr. Hurst impatienté, les rappela à l'ordre, en observant qu'il étoit impossible de jouer avec toutes ces causeries.
Quand Elisabeth fut remontée vers sa sœur, miss Bingley dit en parlant d'elle : "C'est une manière de faire sa cour aux hommes que de dire du mal des femmes. On connoît cette petite finesse. "Qu'en dites-vous Mr. Darcy ?"
"Moi, mademoiselle, je pense que tout ce qui est finesse est mauvais."
\setcounter{page}{394}
Miss Bingley ne fut pas très-contente de cette réponse, et la conversation tomba.
Lorsqu'Élisabeth parut dans le salon en amenant sa sœur convalescente, les dames de la maison coururent à la rencontre de miss Jane, en la félicitant et lui faisant mille caresses. Pendant l'heure qui précéda l'arrivée des hommes, ces dames furent très-aimables. Elles racontoient fort agréablement les anecdotes de société, et se moquoient avec esprit de toutes leurs connoissances. Mais au moment où les hommes entrèrent, ce fut tout autre chose. Mr. Darcy fit un compliment poli à miss Bennet sur sa convalescence. Mr. Hurst lui dit aussi quelques mots; mais Mr. Bingley s'exprima avec une vivacité et un intérêt qui partoient du cœur. Il fut tout attention pour elle. Il craignoit que le salon ne fût froid. Il la fit placer loin de la porte; il s'assit auprès d'elle, et sa conversation l'occupa entièrement. Élisabeth à l'autre bout du salon jouissoit en silence des succès de sa sœur.
Après le thé, Mr. Hurst rappela à sa belle-sœur qu'elle n'arrangeoit pas le whist; mais elle avoit découvert que Mr. Darcy ne se soucioit pas de jouer; et elle répondit que personne n'en avoit envie. En effet, personne\setcounter{page}{395} ne réclama; et Mr. Hurst ne sachant que dire et que faire, ne tarda pas à s'endormir sur un sofa. Darcy prit un livre; miss Bingley en fit autant; et Mad. Hurst, dont la principale occupation étoit de jouer avec ses bracelets et ses bagues, se joignoit de temps en temps à la conversation de son frère et de miss Bennet.
Miss Bingley essaya inutilement de distraire Darcy de sa lecture. Il répondoit par monosyllabes aux questions qu'elle lui adressoit, et continuoit à lire. Enfin, se fatiguant de tenir elle-même un livre qu'elle ne lisoit pas, elle le ferma en bâillant, et s'écria: "ma foi! voilà une manière bien gaie de passer la soirée ensemble! Il faut avouer que la lecture est une chose délicieuse. Quand j'aurai une maison, je suis décidée à avoir une belle bibliothèque. "
Personne ne répondit. Elle bâilla encore une fois, et posa son livre. Ensuite elle regarda autour d'elle, comme pour chercher ce qui pourroit l'amuser. Enfin elle entendit que son frère prononçoit à miss Bennet le mot de bal; et elle s'écria: " Persistez-vous réellement à vouloir donner un bal? Vous devriez consulter autour de vous; car il y a des gens pour qui un bal est une chose décidément ennuyeuse."
\setcounter{page}{396}
"Si c'est de Darcy que vous parlez, je ne le gênerai pas : il pourra s'aller coucher de bonne heure ; mais je suis décidé à faire danser chez moi."
"J'aimerois bien le bal si on ne s'y réunissoit que pour causer."
"Cela seroit peut-être plus raisonnable ; mais cela ressembleroit peu à un bal."
Miss Bingley ne répondit point, et se mit à se promener dans le salon. Elle avoit une tournure élégante, et marchoit avec grâce ; mais Darcy étoit absorbé par sa lecture. Elle essaya de s'associer Elisabeth. "Miss Elisabeth, lui dit-elle, venez : faisons un tour de promenade ensemble, voulez-vous ? cela repose quand on a travaillé si long-temps."
Elisabeth fut surprise, mais elle se leva de bonne grâce pour la joindre. Darcy leva les yeux, et ferma son livre. Miss Bingley l'engagea à se promener aussi. "Non, dit-il, vous ne pouvez avoir que deux raisons pour arpenter ainsi le salon ; et dans les deux suppositions, j'aurois tort de me joindre à vous."
"Mais qu'est-ce qu'il veut donc dire ?" reprit miss Bingley, en parlant à Elisabeth : "le comprenez-vous?"
Non, du tout ; mais je suis sûre qu'il y a de la malice là-dessous ; et croyez-moi, le meilleur moyen de le punir est de ne pas le faire expliquer.
\setcounter{page}{397}
Miss Bingley ne put y tenir : elle voulut savoir ce que c'étoit que ces deux raisons pour se promener.
"Ou vous avez," dit-il, "des secrets à vous dire, ou bien vous sentez qu'en vous promenant ainsi vous faites un joli tableau pour ceux qui vous regardent. Je ne veux point déranger vos confidences; et pour vous admirer, je suis beaucoup mieux placé ici."
"Mais peut-on rien de plus pervers, de plus abominable que cet esprit-là?" dit miss Bingley à Elisabeth. "Comment pourrions-nous le punir?"
"Rien de plus facile, vous n'avez qu'à le plaisanter."
"La plaisanterie ne mord point sur lui, il a un sang-froid qui lui donne toujours l'avantage. Il n'y a point le mot pour rire avec Mr. Darcy."
"Tant pis! car j'aime à rire, moi."
"Sans être ridicule," dit Mr. Darcy, "je sens bien que je pourrois apprêter à rire à ceux qui savent se moquer de tout."
"Je ne suis pas de ces gens-là assurément," reprit Elisabeth."Dieu me garde de traiter légèrement ce qui est honnête et bon! Les caprices, les inconséquences m'amusent; au reste, c'est précisément ce que je ne trouverai pas chez vous, monsieur."
\setcounter{page}{398}
"Mais.... j'ai du moins cherché à éviter les faiblesses qui peuvent rendre un galant homme ridicule."
"Oui, comme, par exemple, la vanité et l'orgueil."
"La vanité est, en effet, une faiblesse, mais pour l'orgueil, lorsqu'il y a vraiment supériorité d'esprit, il y a de quoi le bien régler."
Elisabeth se détourna en souriant.
"Eh bien," lui dit Bingley, "votre examen est fini, je suppose : quel en est le résultat ?"
"La conviction que Mr. Darcy n'a aucun défaut. Il en convient lui-même."
"Je n'ai point dit cela," reprit Darcy. "J'ai mes faiblesses comme un autre. Je sens que j'ai certains défauts de caractère. Par exemple, je manque de liant; je ne suis point assez souple. Je ne sais pas pardonner aisément les folies et les travers. Je ne sais point glisser sur les torts; et lorsqu'une fois on a perdu mon estime, c'est pour toujours."
"Voilà assurément un défaut," dit Elisabeth: "mais il est fort bien choisi: cela ne prête point au ridicule."
"Je suis convaincu que chacun naît avec un défaut, qu'on peut appeler constitutionnel."
\setcounter{page}{399} 
"Le vôtre est de haïr cordialement."
"Et le vôtre, miss Elisa, c'est de n'être pas de bonne foi dans la discussion."
Miss Bingley rompit cette conversation qui ne l'amusoit point, en proposant un peu de musique.
(Miss Jane se rétablit, et retourne chez ses parens. Mr. Bennet annonce à sa famille la visite de leur cousin, Mr. Collins, qui a une cure dans le voisinage et desire passer quelques jours à Longbourn pour faire connoissance avec ses cousines.)
Mr. Collins étoit un homme sans esprit et sans instruction, auquel une fortune inattendue et l'habitude de vivre seul, avoient donné une haute idée de lui-même. Envers ses supérieurs, il avoit des manières humbles, qui constastoient avec l'orgueil qu'il montroit dans l'exercice de son autorité de recteur et d'homme d'église. Comme les biens de la famille de Longbourn, lui étoient substitués si Mr. Bennet mouroit sans enfans mâles, il avoit formé le projet d'épouser une des cinq sœurs, par voie de compensation, et par esprit d'équité.
La jolie figure de miss Jane Bennet le confirma dans ces bons sentimens; et dès\setcounter{page}{400} la première soirée, son choix fut fixé; mais le lendemain matin, Mad. Bennet dérangea tout son plan, en lui faisant entendre que sa fille aînée alloit probablement bientôt être engagée.
Mr. Collins dirigea immédiatement ses vues sur Elisabeth: cela fut fait pendant que Mad. Bennet arrangeoit le feu. Elle l'encouragea sur l'insinuation qu'il fit; et dès les premiers mots, elle vit déjà deux de ses filles mariées.
Lydie avoit le projet d'aller faire une promenade à Meryton. Ses sœurs se joignirent à elle; et Mr. Bennet inventa de prier Mr. Collins de les accompagner. Celui-ci l'avoit poursuivi dans sa bibliothèque après déjeuner. C'étoit, disoit quelquefois Mr. Bennet, le seul endroit de la maison où il trouvât du loisir et de la tranquillité. Il étoit donc de fort mauvaise humeur de ce que Mr. Collins s'y étoit établi, et faisoit semblant de lire. Pour s'en débarrasser, il le pressa d'accompagner ses filles à Meryton. (Dans la petite ville de Meryton, où les Dlles. Bennet alloient souvent, chez une sœur de leur mère, il y avoit un régiment en garnison. Le jour où elles y vont avec Mr. Collins, elles rencontrent à la promenade des officiers\setcounter{page}{401} de leur connaissance, et on leur présente Mr. Wickam, jeune homme nouvellement arrivé au corps, et qui se distinguoit autant par une figure très-remarquable, que par des manières nobles et polies. Pendant qu'ils font la conversation Bingley et Darcy surviennent. Celui-ci pâlit d'émotion en reconnoissant Wickam, lequel de son côté, rougit beaucoup, paroît embarrassé et salue gauchement Darcy, qui lui rend à peine le salut, et s'éloigne bientôt. Elisabeth observe tout cela avec surprise.
Dans une seconde visite à Meryton, chez sa tante Philips, elle questionne Mr. Wickam sur cet incident. Mr. Wickam lui fait un récit vraisemblable de ses relations et de sa brouillerie avec Mr. Darcy, qu'il peint des couleurs les plus noires, tout en paroissant le ménager. Il reste de tout cela à Elisabeth une impression très-favorable sur le compte de Mr. Wickam. — Il se donne ensuite un bal chez Mr. Bingley. )
Jusqu'au moment où Elisabeth chercha en vain Mr. Wickam parmi les officiers déjà arrivés pour le bal, elle n'avoit pas eu de doute qu'il ne fût invité. Elle avoit fait une toilette plus soignée encore qu'à l'ordinaire pour achever sa conquête. Lorsqu'elle fut\setcounter{page}{402} sûre qu'il n'y étoit pas, il lui vint à l'esprit que c'étoit à dessein que Mr. Bingley ne l'avoit pas invité, par ménagement pour Mr. Darcy. Ce n'étoit pas précisément cela, mais Mr. Denny apprit à Lydie que Mr. Wickam avoit été obligé d'aller à Londres le jour précédent et n'étoit pas de retour: Mr. Denny ajouta avec un sourire signifiant: "Je crois que ses affaires ne l'y auroient pas appelé précisément aujourd'hui, s'il ne lui eût convenu d'éviter un certain monsieur." Elisabeth qui entendit cette observation, en prit de l'humeur contre Darcy, et eut à peine assez de présence d'esprit pour lui répondre avec politesse quand il vint s'informer de sa santé. Toute attention, tout ménagement pour lui étoit un tort envers Mr. W. et elle forma le dessein de ne lui point cacher son sentiment. Mais il n'étoit pas dans le caractère d'Elisabeth de conserver de l'humeur ni de la rancune. Elle raconta tout à Charlotte Lucas; et quand elle se fut ainsi soulagée elle se sentit entrain de s'amuser aussi bien que si Mr. Wickam eût été présent. Elle fit pourtant amende honorable en débutant. Elle étoit engagée pour les deux premières contredanses avec Mr. Collins. Il étoit si gauche, si roide, et si\setcounter{page}{403} solennel, il se trompoit si souvent dans les figures qu'elle en eut tout le désagrément qu'on peut avoir d'un danseur ridicule. Elle dansa ensuite avec un officier, et eut le plaisir de parler de Mr. Wickam, et d'apprendre que celui-ci étoit généralement aimé dans son régiment. Lorsqu'elle fut revenue auprès de son amie Charlotte, Mr. Darcy lui proposa tout-à-coup de danser la contredanse suivante. Elle accepta faute de savoir trouver à l'instant une excuse ; et lorsqu'il se fut retiré, elle en témoigna son chagrin à Charlotte.
"Mais," lui dit miss Lucas, "je suis sûre que vous en serez fort contente."
"Dieu m'en garde ! car je suis décidée à le haïr."
Lorsque Darcy s'approcha pour prendre sa main, Charlotte lui dit à l'oreille : "n'allez pas faire la sottise d'être maussade, à cause de Wickam, aux yeux d'un homme qui est bien un autre parti."
Elle ne répondit pas, mais elle fut surprise de voir qu'on la regardoit avec envie et considération lorsque Mr. Darcy se fut placé vis-à-vis d'elle. Il garda long-temps le silence, et elle résolut d'abord de ne pas le rompre. Ensuite elle imagina que ce seroit\setcounter{page}{404} le punir que de l'obliger à répondre à des lieux communs, et elle lui dit : "elle est charmante cette contredanse." Il en convint et n'ajouta rien. Quelques momens après, elle reprit : "allons, Mr.Darcy; c'est à votre tour : faites-moi une observation aussi piquante que la mienne."
"Je suis prêt à dire tout ce qui pourra vous faire plaisir," répondit-il en souriant. - "Vous comprenez qu'il faut faire un peu de conversation pour n'avoir pas l'air ridicule ; mais ce sera aussi peu que vous le voudrez; car nous sommes assez silencieux tous les deux. " à Sans doute : nous avons de singuliers rapports."
"Moi, je garde le silence , parce que je ne sais que dire, et vous ; parce que vous aiguisez vos traits pour parler avec effet."
Il sourit et se tut. Quelques momens après , il lui demanda si elle alloit souvent promener à Meryton.
Elle ne sut pas résister à la tentation de lui donner un peu d'inquiétude, et elle lui dit : "Quand vous nous rencontrâtes l'autre jour, je venois de faire une nouvelle connoissance. "
` A ce mot Darcy prit un air froid et hautain. Il lui dit ensuite avec contrainte: "Mr.\setcounter{page}{405} Wickam a de la grâce, et se fait aisément des amis. Il n’est pas aussi heureux pour les conserver."
"On dit en effet qu’il a eu le malheur de perdre votre amitié."
Darcy ne répondit point, prit un air sérieux, et parut desirer de changer de sujet; mais la conversation ne se renoua pas; parce qu’Elisabeth un peu embarrassée de ce que son insinuation n’avoit pas réussi, devint silencieuse. Quand la contredanse fut finie, Darcy fit une profonde révérence à Elisabeth sans lui dire un mot. . . . . . . . .
Mr. Collins ayant résolu de faire sa déclaration en forme, pendant son séjour à Longbourn, entra après déjeuné dans le salon où étoit Mad. Bennet avec Elisabeth et une de ses sœurs cadettes, et il dit en entrant: "puis-je espérer, madame, que j’obtiendrai de votre crédit sur votre charmante fille Elisabeth, la faveur que je lui demande d’une audience particulière dans le cours de la matinée ?"
Avant qu’Elisabeth eût eu le temps de faire autre chose que de rougir de surprise, sa mère dit en se levant: "certainement monsieur, certainement! Il n’y a point de difficulté à cela. Allons, Kitty, montons, ma\setcounter{page}{406} fille." Elle prit son ouvrage et alloit sortir lors qu'Elisabeth lui cria, "mais, ma mère, je vous prie de ne point vous en aller. Mr. Collins n'a surement rien à dire que vous ne deviez entendre. Je m'en vais sortir, moi-même. "
"Quelle folie ! J'insiste absolument pour que vous restiez, Lizzy. Il faut entendre ce que votre cousin a à vous dire. "
Elisabeth comprit qu'elle ne pouvoit pas l'échapper. Elle s'assit, et tenant les yeux sur son ouvrage, elle tâcha de cacher son envie de rire, qui étoit mêlée d'impatience et de dépit.
" Croyez-moi, ma chère miss Elisabeth, " lui, dit-il, après s'être assis à côté d'elle, "votre modestie ne fait qu'ajouter à mes yeux à toutes vos autres perfections. Je vous affirme que vous auriez été moins aimable pour moi sans cette petite nuance de répugnance que vous me montrez ; mais permettez-moi de vous assurer que j'ai l'autorisation de votre mère. Mes attentions pour vous ont été trop marquées pour que le but vous en ait échappé, quoique votre délicatesse naturelle vous ait porter à dissimuler avec moi. A peine j'ai été admis dans votre maison que je vous avois choisie comme la compagne\setcounter{page}{407} future de mes jours. Mais avant de me laisser emporter par l'effusion de mes sentiments, il est peut-être convenable de vous dire mes raisons pour entrer dans l'état du mariage, et même les motifs que j'ai eu pour venir m'établir dans le comté de Hertford avec le projet d'y chercher une femme.
- L'idée que Mr. Collins pouvoit être entraîné par l'effusion de ses sentiments, parut si drôle à Elisabeth qu'elle eut toutes les peines du monde à s'empêcher de rire. Il continua ainsi :
" Mes raisons pour entrer dans l'état du mariage sont les suivantes : 1°. Je pense qu'il est convenable qu'un ecclésiastique, dans ma position, donne cet exemple à ses paroissiens, lorsqu'il est, comme c'est mon cas, dans des circonstances de fortune qui lui permettent d'élever honorablement une famille. 2°. Je suis convaincu que le mariage ajoutera sensiblement à la somme de mon bonheur. 3°. Enfin, j'ai une raison que j'aurois dû mentionner la première, c'est que la noble dame que j'ai l'honneur de nommer ma protectrice, m'a donné deux fois ce conseil sans que je le lui eusse demandé. "Mr. Collins, il faut vous marier," m'a-t-elle dit, "il est bon qu'un homme d'église soit marié. Prenez\setcounter{page}{408} une femme de bonne famille, qui soit active et sache s'employer à tout, qui ait ces principes d'économie avec lesquels on alonge un petit revenu. Trouvez, " me disoit-elle, " une femme de ce caractère et qui soit bien élevée; amenez la à Hunsford, et je ne ferai aucune difficulté pour la voir. " Je vous observe en passant, ma belle cousine, que la bienveillance de milady Catherine de Bourg n'est pas un des moins grands avantages qu'il est en mon pouvoir de vous offrir. Vous trouverez ses manières au-dessus de tout ce que je puis vous en dire; et je ne doute pas que votre esprit et votre vivacité ne lui soient agréables, sur-tout si vous les tempérez par un peu de silence, ainsi que le respect pour son rang l'exigera. "
" Voilà, mon aimable cousine, qui vous explique en général l'intention que j'ai de me marier. Il me reste à vous dire pourquoi j'ai été conduit à porter mes vues sur votre famille, au-lieu de me marier dans mon voisinage où il ne manque pas, je vous assure, de fort aimables personnes. Étant appellé à hériter de la terre de Longbourn, après la mort de votre respectable père ( lequel, au reste, peut vivre encore long-temps ), j'ai voulu choisir ma femme parmi ses filles, afin que\setcounter{page}{409} la perte à laquelle elles seront appelées fût aussi petite qu'il est possible, lorsque ce triste événement arrivera, chose qui, ainsi que je viens de le dire, peut être encore très-éloignée. J'espère, ma belle cousine, qu'un tel motif est de nature à augmenter l'estime que vous pouvez avoir conçue pour moi; et il ne me reste qu'à vous assurer dans les termes les plus passionnés de toute la violence de mon attachement pour vous. Quant à la fortune, c'est un objet qui m'est tout-à-fait indifférent. Je ne demanderai point de dot parce que je sais que votre père ne pourroit pas vous en donner une. Tout ce qui doit vous revenir, c'est mille livres sterling, dans les quatre pour cent, après la mort de votre mère. Je ne dirai donc mot sur le chapitre de la fortune; et vous pouvez être assurée qu'une fois que nous serons mariés, vous n'aurez à essuyer de ma part aucun reproche peu généreux."
Il devenoit nécessaire de l'interrompre: aussi Élisabeth, s'écria-t-elle: "vous oubliez, monsieur, que je ne vous ai pas encore répondu. Je vous remercie de tous vos beaux complimens, et de l'honneur que vous voulez bien me faire, mais il m'est impossible de ne pas le refuser."
\setcounter{page}{410}
"Ce n'est pas d'aujourd'hui," reprit gravement Mr. Collins, "que je sais qu'une jeune demoiselle rejette toujours la première demande d'un homme, qu'au fond du cœur elle se réserve d'accepter. Quelquefois même elle renouvelle son refus deux et trois fois. Je ne suis donc nullement découragé, et j'espère avant qu'il soit long-temps avoir, ma belle cousine, le plaisir de vous conduire à l'autel."
"C'est un peu fort," s'écria Élisabeth, après ce que je viens de vous dire si positivement. Je vous répète, monsieur, mon refus formel et positif. Vous ne pourriez pas me rendre heureuse ; et je suis la personne la moins propre à faire votre bonheur. Je suis convaincue que votre amie lady Catherine seroit la première à en juger ainsi."
"S'il étoit certain que la chose fût désapprouvée par milady Catherine... ..... Mais il est impossible que milady ait pris sur votre compte des notions erronées ; d'ailleurs, je m'engage, pour la première fois que j'aurai l'honneur de la voir, à lui parler fortement sur votre modestie, sur votre économie, et sur toutes vos aimables qualités."
"Je vous assure, monsieur, que tout cela sera fort inutile : il faut que vous me fassiez\setcounter{page}{411} l'honneur de croire que je vous dis la vérité. Je souhaite que vous soyez très-heureux et très-riche; et en refusant votre main, je fais tout ce que je puis pour que cela arrive. L'offre que vous avez bien voulu me faire aura satisfait vos scrupules sur la possession de Longbourn, et vous ne pourrez rien vous reprocher à cet égard."
En achevant ces mots, elle se leva et lui fit une profonde révérence.
"La première fois que j'aurai l'honneur de vous entretenir sur ce sujet, "reprit-il, "j'espère une réponse plus favorable que celle que vous venez de me faire. Mais je suis loin de vous accuser de cruauté. Je sais que c'est l'usage de refuser une première demande. Peut-être en avez-vous dit autant que le comportoit la délicatesse qui doit toujours caractériser votre sexe."
"En vérité, monsieur, vous m'embarrassez beaucoup. Si vous prenez tout ce que je vous ai dit pour des encouragemens, comment dois-je donc faire pour vous convaincre que ma réponse est un refus ?"
"Il faut, ma belle cousine, que vous me permettiez de me flatter que ce sont là des expressions d'usage. Voici, en peu de mots, mes raisons pour le croire. D'abord, je ne\setcounter{page}{412} trouve pas que ma main soit précisément une chose à dédaigner, si vous me permettez de vous le dire. Ma position, mes liaisons dans la famille de Bourg, sont des circonstances qui méritent considération. Je pourrois même vous faire observer, que malgré vos agrémens, ce n'est point une chose sûre que vous eussiez l'occasion de faire un second refus. Malheureusement, votre perspective de fortune est très-petite. Comment voulez-vous sérieusement, que je ne voie pas que tout ceci est un petit manège pour me rendre plus amoureux, comme cela se pratique chez les élégantes."
"Je n'ai pas la moindre prétention, je vous assure, monsieur, à cette espèce d'élégance. Je vous répète mes remercimens pour l'honneur que vous me faites. Mais veuillez me croire de bonne foi, quand je dis non."
Ma belle cousine, je demeure convaincu que le consentement de vos chers parens lévera les derniers obstacles."
Elisabeth impatientée, le quitta sans répondre, et bien décidée à prier son père de mettre fin à cette persécution.
(La suite à un autre Cahier.)
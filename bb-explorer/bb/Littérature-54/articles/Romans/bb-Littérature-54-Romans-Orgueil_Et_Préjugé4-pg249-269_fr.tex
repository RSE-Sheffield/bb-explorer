\setcounter{page}{249}
\chapter{Romans}
\section{PRIDE AND PREJUDICE. Orgueil et préjugé. Roman en 3 vol. Londres 1813. \large{(-Dernier extrait. Voy. p. 90 de ce vol.)}}
MAD. GARDINER explique dans sa réponse à Elisabeth, que Darcy a insisté auprès de Mr. Gardiner pour se charger seul des sacrifices d'argent qui pouvaient déterminer Wickham à épouser Lydie. Le motif qu'il donne pour se mettre en avant de cette manière, c'est qu'il veut réparer le tort qu'il avait eu de n'avoir pas dévoilé le caractère de Wickham; il se reproche de se trouver ainsi la cause du malheur de la famille Bennet.— Bingley revient à Netherfield et y amène Darcy. Le premier se reprend de passion pour miss Jane, et le mariage s'arrange. Darcy est toujours froid et réservé ).
Il y avait huit jours que le mariage de Mr. Bingley avec miss Jane était décidé; et il se trouvait le matin dans la salle à manger avec les dames de la famille, lorsqu'on vit arriver une voiture à quatre chevaux,\setcounter{page}{250} qui s'approchoit de la maison. C'étoit trop matin pour les visites, et d'ailleurs, l'équipage ne ressembloit à aucun de ceux du voisinage. La livrée n'étoit pas connue, et la voiture étoit attelée de chevaux de poste. Bingley engagea miss Jane à ne pas se laisser prendre par cette visite; et ils s'esquivèrent ensemble du côté du jardin. Mad. Bennet, Elisabeth et Kitty tinrent ferme, et elles virent paroître lady Catherine de Bourg. Leur étonnement fut grand. Elle entra sans saluer; elle répondit que par un signe de tête à la révérence que lui fit Elisabeth; et s'assit sans dire un mot; et sans demander d'être présentée à la maîtresse de la maison, mais Elisabeth avoit nommé lady Catherine à sa mère au moment où elle la vit entrer.
Mad. Bennet, toute contente d'avoir à recevoir une femme de qualité, s'évertua pour être polie. Après quelques instans de silence, milady dit à Elisabeth : "j'espère que vous êtes en bonne santé, miss Bennet. Cette dame là est votre mère, je suppose, et celle-ci une sœur cadette, n'est-ce pas?"
Mad. Bennet enchantée de parler à lady Catherine, se hâta de prendre la parole, et dit: "c'est une de mes cadettes; mais la plus jeune est mariée. L'aînée se promène dans le jardin avec un jeune homme qui ne tardera pas à être de la famille."
\setcounter{page}{251}
Lady Catherine parut distraite, puis elle dit un moment après : "votre parc est bien petit."
"Il est assurément bien petit en comparaison de Rosings, milady," répondit Mad. Bennet ; "mais je vous assure qu’il est beaucoup plus grand que celui de Sir William Lucas."
"Ce salon-ci," reprit lady Catherine, "doit être bien incommode pour le soir en été ; les croisées sont en plein couchant."
Mad. Bennet lui répondit qu’on ne s’y tenoit jamais après dîner, puis elle ajouta : "oserois-je, milady, vous demander comment se portent Mr. et Mad. Collins,"
"Je les ai laissés en bonne santé, c’est avant-hier que je les ai vus, je crois."
Élisabeth crut que lady Catherine alloit lui remettre une lettre de Charlotte : car c’étoit le seul motif qu’elle pût supposer à sa visite ; mais il n’en fut pas question et elle ne savoit plus qu’en penser.
Mad. Bennet offrit alors à lady Catherine de faire apporter quelques rafraîchissemens ; mais celle-ci refusa très-positivement et avec assez peu de politesse. Un moment après, elle se leva en disant à Élisabeth : "il me semble que je vois là-bas un assez joli petit bosquet qui borde votre prairie. Je ne serois\setcounter{page}{252} pas fâchée de m'y promener quelques momens, si vous voulez bien m'accompagner."
"Allez, ma chère," dit Mad. Bennet, "faites voir à milady les différents points de vue: je crois que l'hermitage lui plaira."
Elisabeth alla chercher son chapeau et ses gants, puis elle revint joindre milady, qui en traversant la chambre à manger et le vestibule, jeta autour d'elle un coup-d'œil d'approbation, et dit: "mais ceci n'est pas mal, en vérité."
Sa voiture était restée à la porte, et Elisabeth vit que la femme-de-chambre l'y attendait. Elles marchèrent en silence dans le sentier qui conduisait à la plantation. Elisabeth était déterminée à ne pas se mettre en frais de conversation pour une femme qui lui paraissait plus que jamais insolente et désagréable. Elle ne comprenait pas comment elle avait pu lui trouver une fois quelque rapport avec son neveu.
Lorsqu'elles furent dans le bosquet, lady Catherine prit la parole et lui dit: "il ne vous est pas difficile, miss Bennet, de deviner le motif de ma visite. Votre conscience vous dit certainement ce qui en est."
Elisabeth parut extrêmement surprise, et lui dit en la fixant: "en vérité, milady, vous\setcounter{page}{253} êtes dans l'erreur, je ne conçois point quelle est la raison qui nous procure l'honneur de votre visite."
"Miss Bennet," reprit-elle d'un son de voix ému par la colère. "Je vous avertis qu'on ne me joue pas; mais si vous n'êtes pas franche, je le serai, moi. Mon caractère a toujours été renommé pour la franchise ; et dans une affaire d'une si haute importance, on ne me verra pas tergiverser. On m'a fait ces jours derniers un singulier rapport. On m'a dit que non-seulement votre sœur étoit sur le point de se marier très-avantageusement, mais que vous-même, miss Elisabeth Bennet, vous ne tarderiez probablement pas à épouser mon propre neveu, Mr. Darcy. Quoique je prenne cela pour une absurdité et que je ne fasse point à mon neveu le tort de croire la chose possible, j'ai résolu de venir m'en expliquer directement avec vous."
Elisabeth rougit beaucoup, et répondit avec dépit : "puisque vous n'avez pas cru la chose possible, milady, je m'étonne que vous ayez pris la peine de venir si loin pour vous en informer."
"J'ai voulu avoir un désaveu formel, et pouvoir faire tomber ce bruit tout-à-fait."
\setcounter{page}{254}
"Votre présence ici, milady, pourrait fort bien l'accréditer, au contraire, en supposant que ce bruit ait couru."
"En supposant que ce bruit ait couru, dites-vous miss Bennet? C'est-à-dire, que vous ne le savez pas."
"Non, du tout: j'ignorois qu'il en fût question."
"Et pouvez-vous également affirmer qu'il n'y ait rien de vrai dans la chose?"
"Je ne prétends point à la même franchise que vous, milady: vous pourriez me faire des questions auxquelles il ne me conviendroit pas de répondre."
"Ceci est un peu fort! miss Bennet, j'insiste pour que vous répondiez directement à ma question: mon neveu vous a-t-il fait quelque proposition de mariage?"
"Vous venez vous-même de dire, milady, qu'il étoit absurde de le supposer..."
"Cela est absurde et impossible si mon neveu conserve l'usage de la raison; mais il peut y avoir eu de la séduction dans tout ceci: et votre petit manège peut lui avoir fait oublier un moment ce qu'il devoit à sa famille, et ce qu'il se devoit à lui-même."
"Si j'avois réussi de cette manière, je ne serois pas disposée à l'avouer."
"Miss Bennet, savez-vous à qui vous parlez?\setcounter{page}{255} Je suis sa plus proche parente, et j'ai des droits à son entière confiance."
"A la bonne heure, mais vous n'avez pas les mêmes droits à la mienne, et je vous avertis que ce n'est pas ainsi qu'on la gagne."
"Ecoutez, miss Bennet, il faut que vous me compreniez bien. Je vois que vous avez la présomption de prétendre à mon neveu: or, vous ne pouvez pas l'épouser, parce qu'il est engagé. Qu'avez-vous à répondre à cela?"
"J'ai à répondre que s'il est engagé, il ne peut pas m'avoir fait de proposition."
Lady Catherine hésita un moment, puis elle dit: "leur engagement est d'un genre particulier. Le projet de les unir date de leur première enfance. C'étoit le vœu de sa mère comme le mien. Seroit-il dit qu'au moment de voir accomplir un projet si doux, et à tous égards si convenable, il se trouvât dérangé par une inclination pour une petite personne sans naissance, et qui ne tient à rien de marquant? Le vœu de ses parens est-il donc nul à vos yeux, mademoiselle? Ne sentez-vous pas que vous devez respecter son engagement? Avez-vous donc perdu tout sentiment des convenances, et toute délicatesse? Ne m'avez-vous pas entendu dire souvent que dès son enfance il étoit destiné à sa cousine?"
\setcounter{page}{256}
"Oui, milady, et je l'avois déjà ouï dire auparavant; mais en quoi donc est-ce que cela me lie? S'il n'y a pas d'autres objections à mon mariage avec votre neveu, je ne serai assurément pas retenue par la seule raison que sa mère et sa tante avoient projeté de le marier avec sa cousine. Si Mr. Darcy n'a ni inclination ni engagement, qu'est-ce qui pourroit l'empêcher de faire un autre choix? Et si c'est moi qu'il choisit, pourquoi ne l'accepterois-je pas?,"
"Parce que l'honneur, la décence, la prudence, votre intérêt même s'y opposent. Oui, mademoiselle, votre intérêt. Ne vous attendez pas à être jamais reconnue par la famille si vous allez en avant dans ce projet insensé. Cette alliance seroit une tache pour notre maison, et jamais votre nom ne seroit même prononcé par aucun de nous.,"
"Voilà des menaces fort effrayantes; mais la femme de Mr. Darcy aura tant de moyens d'être heureuse, qu'elle pourra prendre son parti de bien des choses.,"
"Quelle tête! quelle obstination. J'ai véritablement honte pour vous.... Voilà donc la reconnoissance que vous montrez de mes bontés. Miss Bennet, croyez-vous ne me rien devoir? Ecoutez: asseyons-nous encore un moment. J'ai résolu de ne pas\setcounter{page}{257} vous quitter sans avoir obtenu ce que je desire. Je ne suis point accoutumée à céder aux caprices et à avoir le dessous dans les affaires."
"Cela pourra vous donner du chagrin, milady, mais cela n'influera point sur ma conduite."
"Je ne veux pas qu'on m'interrompe : écoutez-moi en silence. Mon neveu et ma fille sont faits l'un pour l'autre. Du côté maternel ils sont issus d'une noble race ; et du côté paternel, de familles anciennes, honorables et respectables, quoique non titrées. Leur fortune est également brillante. Ils sont destinés l'un à l'autre par la voix unanime des individus qui composent les deux familles. Et vous croyez qu'on souffrira que tout cela soit dérangé par les prétentions d'une jeune personne qui n'a ni parens ni fortune ? Non, assurément ! et si vous entendiez bien vos intérêts, vous ne songeriez pas à sortir de la sphère dans laquelle vous avez été élevée."
"Si j'épousois votre neveu, milady, je ne croirois point sortir de la sphère dans laquelle j'ai été élevée. Il est gentilhomme : je suis fille de gentilhomme : je ne vois pas en quoi il dérogeroit."
"Votre père est gentilhomme, à la bonne\setcounter{page}{258} heure; mais votre mère qu'est-elle? Que sont vos oncles et tantës? Me croyez-vous donc ignorante de tous ces détails?"
"Si votre neveu n'a point d'objection à faire contre mes parens, je ne vois pas, milady, pourquoi vous en auriez vous-même."
"Voyons! à présent. Etes-vous engagée à lui, oui ou non?,,Elisabeth hésita à répondre. Ensuite elle dit: "non, je ne suis pas engagée."
"Et me promettez-vous de ne pas vous engager?
"Non, assurément! je ne le promets pas."
"Miss Bennet, j'éprouve beaucoup d'étonnement, et je suis révoltée de votre conduite. Je m'attendais à trouver en vous une personne raisonnable; mais n'allez pas vous imaginer que je vous céderai. Je suis décidée à ne pas vous quitter sans avoir reçu votre parole."
"Et moi je suis décidée à ne la pas donner. D'ailleurs, milady, croyez-vous que le mariage de miss Debourg avec Mr. Darcy devînt plus probable, si je vous faisois la promesse que vous demandez? S'il a de l'attachement pour moi, sera-t-il plus disposé à donner sa main à votre fille quand je l'aurai refusé? Vous vous êtes beaucoup trompée, milady, si vous avec cru pouvoir rien\setcounter{page}{259} obtenir de moi par des menaces. Je ne sais pas jusqu'à quel point votre neveu permet que vous vous mêliez de ses affaires, mais ce qu'il y a de certain, c'est que vous n'avez rien à voir aux miennes. Je demande donc, s'il vous plaît, que nous en finissions.
"Doucement, doucement! Je suis bien loin d'avoir fini. Je suis informée de la fuite infâme de votre sœur cadette. Je sais que le mariage n'est qu'une manière de plâtrer l'affaire, et qu'il en a coûté de l'argent à votre père et à vos oncles. Croyez-vous tolérable de donner cela pour belle-sœur à mon neveu? Faudra-t-il qu'il ait pour beau-frère le fils de l'intendant de son père? Quelle honte! Verra-t-on le séjour de Pemberley ainsi profané!"
"Milady," lui dit Elisabeth en se levant, "vous ne pouvez plus rien avoir à me dire, car vous m'avez insultée de toutes les manières possibles. Je vous demande donc la permission de rentrer."
Lady Catherine la suivit en colère : "vous êtes," lui dit-elle une personne opiniâtre, égoïste, et dépourvue de toute délicatesse. Vous ne voulez pas voir que Darcy en vous épousant se brouille avec tout le monde."
"Je n'ai plus rien à répondre, milady."
"Vous êtes donc résolue à l'épouser?"
\setcounter{page}{260}
"Je n'ai pas dit cela du tout ; mais je suis résolue de faire ce que je croirai qui me conviendra, sans me laisser arrêter par des considérations qui me sont étrangères."
"Voilà donc votre dernier mot? Eh bien! je vous déclare qu'il n'en sera rien. J'étois venue pour vous sonder. J'espérois de vous trouver raisonnable: vous ne l'êtes point : votre ambition sera déçue: vous ne réussirez pas. ,, Elle continua à lui parler de ce ton-là jusqu'au moment où elles se trouvèrent près de la voiture. Elle ajouta alors brusquement. "Je ne prends point congé de vous, je ne vous charge de rien pour votre mère. Vous ne méritez point d'égard. J'emporte un sentiment de véritable indignation."
Elisabeth ne fit aucune réponse, et rentra tranquillement dans la maison. Sa mère courut à sa rencontre en lui demandant pourquoi milady n'étoit pas entrée.
"Elle étoit pressée," dit Elisabeth: "elle n'a pas voulu s'arrêter."
"Elle a un beau port cette femme! Bonne façon! Elle est prodigieusement polie! Car je suppose qu'elle est venue tout exprès pour nous donner des nouvelles des Collins. Elle n'avoit rien à nous dire de particulier, je pense?"
Elisabeth répondit un peu vaguement\setcounter{page}{261} parce qu’il n’y avait pas moyen de rendre compte de la conversation qui venoit d’avoir lieu. . . . . . . . . . . . . . . . . . . . . . . . . . . .
Quelques jours après la visite de Lady Catherine, Bingley et Darcy arrivèrent à Longbourn. Lorsqu’on eut fait la conversation quelques momens dans le sallon, on fit un tour de promenade. Mad. Bennet resta à ses affaires. Bingley et miss Jane se séparèrent bientôt en ralentissant leur marche. Kitty alla en avant pour chercher son amie Marie Lucas, et Elisabeth se trouva ainsi seule avec Darcy. C’étoit le moment d’exécuter la résolution qu’elle avoit prise, et pendant qu’elle sentoit en avoir le courage, elle commença ainsi :
"Mr. Darcy, je suis une créature extrêmement personnelle, et pour soulager mon sentiment, je veux courir le risque de blesser le vôtre. Je ne peux pas retenir plus long-temps l’expression de ma reconnoissance pour la bonté sans exemple que vous avez eue, à l’égard de ma pauvre sœur. Depuis le moment où je l’ai su, j’ai été tourmentée du desir de vous dire là-dessus tout ce que je sens. Si mes parens en étoient informés, ils se joindroient à moi pour vous offrir les témoignages de leur gratitude."
Darcy parut surpris et ému. "Je suis extrêmement\setcounter{page}{262} fâché, dit-il, "que vous ayez été informée, et probablement mal informée, de ce qui s'est fait. Je croyois Mad. Gardiner plus sûre pour la discrétion."
"Il n'y a point de la faute de ma tante. C'est l'étourderie de ma sœur qui a trahi votre secret; et quand j'en ai su un peu, vous comprenez que j'ai voulu savoir tout. Je ne puis assez vous dire à quel point je suis reconnoissante de cette compassion généreuse qui vous a porté à prendre tant de peines et à vous exposer à tant de désagréments. Je vous remercie encore au nom de toute ma famille."
"Ne me remerciez que pour vous-même, miss Bennet. Je ne nie point que l'espérance de faire une chose dont le résultat vous fut agréable n'ait été mon principal motif. Votre famille ne me doit rien, car il me semble que je n'ai pensé qu'à vous."
Un silence suivit. Elisabeth étoit émue et embarrassée.
Darcy ajouta après quelques momens, et d'un son de voix un peu altéré. "Vous avez trop de générosité, mademoiselle, pour vous jouer de mon bonheur. Dites-moi sans détour si vos sentimens à mon égard sont ce qu'ils étoient au mois d'avril dernier. Mon vœu est toujours le même, mais un mot\setcounter{page}{263} de vous m'imposera silence à jamais. L'émotion et l'embarras d'Elisabeth augmentèrent. Elle ne savoit comment répondre, et pourtant il falloit parler. En hésitant beaucoup, et en s'interrompant sans cesse; elle lui laissa pourtant voir qu'il s'étoit fait un grand changement en elle, et qu'elle recevoit avec plaisir cette assurance inattendue.
Darcy écouta cet aveu avec transport et exprima toute la violence de sa passion. Si Elisabeth eût osé lever les yeux, elle auroit vu combien sa physionomie étoit animée et son regard expressif. Il parloit avec une éloquence et une chaleur de sentiment qui la charmoit, et chacune de ses paroles lui faisoient mieux apprécier le bonheur d'avoir fixé le cœur d'un tel homme.
Elle apprit que ce qui avoit amené l'explication étoit une conversation qu'il avoit eue avec sa tante, et dans laquelle après avoir raconté à sa manière la visite qu'elle avoit faite à Longbourn, elle avoit tâché d'obtenir de lui la promesse formelle de renoncer à Elisabeth. C'est ce que m'a dit lady Catherine," ajouta-t-il, "qui m'a rendu l'espérance. Je vous connoissois assez pour être sûr que si vous aviez été irrévocablement décidée contre moi, vous l'auriez dit franchement. ,"
\setcounter{page}{264}
Elisabeth rougit et se mit à rire. "Oui, vous connaissez en effet ma franchise. Après vous avoir dit en face tant de mal de vous-même, je pouvais bien en dire autant à votre parente."
"Vous n'avez rien dit de moi que je n'eusse mérité, car ma conduite était bien étrange, impardonnable même."
"Il ne faut pas examiner lequel des deux eut plus de torts ce jour-là ; mais dès-lors nous sommes devenus bien plus polis tous les deux."
"Je ne puis vous exprimer," reprit Darcy, d'un ton sérieux, "combien je me suis fait de reproches à cette occasion. Vous me dites un mot qui me pénétra, et qui m'a horriblement tourmenté depuis." Si votre conduite, "me dites-vous," avait été celle d'un homme comme il faut....... ah ! que cette expression qui d'abord me révolta, m'a paru ensuite bien méritée !
"J'étais loin d'y attacher tant d'importance, et je n'imaginais pas vous faire une impression si pénible."
"Certes, je le crois. Vous me regardiez alors comme un homme sans délicatesse. Je suis sûr que c'était-là votre opinion de moi. Je n'oublierai jamais l'expression de votre physionomie quand vous me dites que rien\setcounter{page}{265} n'auroit pu vous engager à accepter ma main."
"Ah ! ne répétez pas, ne répétez pas mes paroles. J'en ai été si honteuse ! .........
Darcy rappela sa lettre, et demanda si Elisabeth avoit ajouté foi à sa justification. "Je me croyois de sangfroid en l'écrivant; continua-t-il, "mais j'ai bien senti depuis qu'elle avoit été dictée par un sentiment amer. "
"Il y avoit en effet un peu d'amertume au début, mais vous finissiez d'une manière toute charitable. Ne pensons plus à cette lettre. Nos sentimens à tous deux sont entièrement changés. Prenez un peu de ma philosophie : je tâche toujours de perdre le souvenir de ce qui n'est pas agréable. "
"Cela vous est très-facile, à vous qui n'avez jamais de reproches à vous faire ; mais moi j'ai de pénibles souvenirs quoique je fasse. J'étois fils unique et j'ai eu le malheur d'être gâté par le plus excellent père. J'étois devenu un être orgueilleux, personnel. J'avois été élevé à regarder avec dédain tout ce qui n'étoit pas dans mon cercle de famille. J'étois plein de vanité et de sottise, et je serois demeuré tel, si je n'avois été humilié par vous. Il ne m'étoit pas entré dans l'esprit d'avoir le moindre doute sur l'empressement avec lequel vous accepteriez ma main.\setcounter{page}{266} La leçon sévère que vous me donnâtes, fit sur moi une profonde impression.
"Vous me supposiez donc une disposition secrète à accueillir vos vœux? "
"Moquez-vous de moi, si vous le voulez; mais je me croyois sûr d'être écouté favorablement. "
"Je ne me moquerai point de vous, car cela prouve que j'ai quelque chose à me reprocher qui ressemble à des torts de coquetterie; mais je vous assure que c'étoit sans intention et sans projets. Dieu sait comme vous me haïssiez après cela! "
"Haïr n'est pas tout-à-fait le mot. Je fus fort en colère, mais ensuite cela prit une direction convenable."
"Avouez que vous fûtes disposé à me blâmer lorsque vous me trouvâtes à Pemberley."
"Non, je vous jure: je n'éprouvai que beaucoup de surprise. "
"Votre surprise ne fut pas plus grande qu'à la mienne, en vous voyant si obligeant et si poli: Je sentois que je ne le méritois guères. "
"Je voulus vous convaincre que je n'avois aucun ressentiment de ce qui s'étoit passé et que j'avois profité de vos leçons. Je n'avois dans ce moment-là aucune espérance. Je ne saurois dire combien il se passa de temps\setcounter{page}{267} avant qu'elle rentrât dans mon cœur : mais je crois bien que ce fut l'affaire d'une demi heure."
Darcy et Elisabeth oublioient que le temps passoit. Celle-ci prit tout-à-coup de l'inquiétude, en s'apercevant qu'ils étoient bien loin de la maison où ils seroient peut-être attendus. La conversation ne languit point pendant leur retour, et lorsqu'ils se séparèrent dans le vestibule, ils avoient encore tous deux bien des choses à se dire.
Lorsque Mr. Bennet se retira après dîner dans son cabinet, Mr. Darcy le suivit. Elisabeth s'en aperçut et en eut beaucoup d'émotion ; elle attendit avec un tremblement dont elle n'étoit point maîtresse, que Darcy revînt dans le salon. Lorsqu'il reparut, il avoit l'air serein. Il sourit en la regardant, et elle se rassura. Il s'approcha de la table où elle travailloit avec Kitty, et faisant semblant d'admirer son ouvrage en l'examinant de près, il lui dit tout bas : "Allez vers votre père : il vous attend." Elle passa en effet dans le cabinet de Mr. Bennet. Elle le trouva qui se promenoit d'un air soucieux. "Lizzy, mon enfant," lui dit-il, "à quoi pensez-vous, d'accepter la main de cet homme que vous ne pouvez pas souffrir."
Elisabeth regretta bien alors les expressions exagérées d'éloignement pour Mr. Darcy,\setcounter{page}{268} dont elle s'étoit souvent servie en présence de son père. Elle étoit embarrassée à expliquer ses sentimens, en contradiction avec ceux qu'elle avoit manifestés; mais elle fut obligée d'articuler que son inclination étoit d'accord avec le vœu de Mr. Darcy.
"Cela signifie, ma fille, que vous voulez vous marier. Il est riche, vous aurez des équipages et des diamans; mais êtes-vous bien sûre que cela vous rendra heureuse? "
"N'avez-vous, mon père, aucune objection à ce mariage que l'indifférence que vous me supposez? "
"Aucune absolument. Nous le connoissons tous pour un homme vain et d'un commerce désagréable; mais si vous aviez du goût pour lui, je n'aurois rien à dire. "
"J'en ai, mon père, j'ai réellement du goût pour lui," dit Elisabeth, les larmes aux yeux, "car il n'est point vain, je vous assure. Il est parfaitement aimable. Ne me donnez donc pas du chagrin, en parlant de lui comme vous venez de faire. "
"Lizzy, je lui ai donné mon consentement: ce n'est point un homme que je doive refuser, mais réfléchissez pourtant encore. Je vous connois assez pour savoir que vous ne pourrez pas être contente de votre propre conduite, et par conséquent, heureuse,\setcounter{page}{269} si vous ne regardez pas votre mari comme supérieur à vous. Votre esprit et vos talens rendent extrêmement dangereuse pour vous une association dans laquelle vous sentirez l'ascendant de votre côté. Je vous le répète mon enfant, réfléchissez encore : il faut qu'une femme distinguée, comme vous l'êtes, puisse respecter son mari, pour que les choses aillent bien."
Elisabeth s'étendit alors sur le mérite réel de Arug, qu'elle avoit bien appris à connoître. Elle expliqua de quelle manière il s'étoit mis en avant pour arranger la malheureuse affaire de Lydie, et avec quelle délicatesse il avoit désiré que son intervention ne fût point inconnue.
"Allons, allons ! lui dit Mr. Bennet, puisqu'il en est ainsi, je n'ai plus d'objections. Ah ! ce n'est donc pas votre oncle qui a avancé la somme? Voilà qui va me faire une grande économie. Les jeunes gens amoûreux jettent tout par les fenêtres. Je lui proposerai de le payer: je parie qu'il me refusera. Or çà, ma petite Lizzy, te voilà contente, et moi de même. Dis en sortant à tes sœurs Marie et Kitty, que si elles ont quelqu'épouseur, elles n'ont qu'à entrer pendant que j'ai le temps, et que je suis en train de dire oui."
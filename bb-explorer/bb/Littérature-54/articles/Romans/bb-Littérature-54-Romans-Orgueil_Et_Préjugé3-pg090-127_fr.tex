\setcounter{page}{90}
\chapter{Romans}
\section{PRIDE AND PREJUDICE. Orgueil et préjugé, Roman en 3 vol. Londres 1813. \large{(Troisième extrait. Voy. p. 522 du vol. précd.)}}
LE lendemain, Elisabeth se réveilla avec les mêmes idées que la veille. Elle ne revenoit point encore de sa surprise ; elle ne pouvoit ni penser à autre chose, ni s'occuper de rien. Elle prit le parti d'aller faire un tour de promenade après déjeûner. Elle se dirigea d'abord vers le parc ; mais réfléchissant qu'elle pourroit y rencontrer Mr. Darcy, elle prit d'un autre côté en suivant un sentier qui également la ramena jusqu'auprès du parc. Elle n'y entra point ; mais elle fit cinq ou six tours sur la pelouse qui le bordoit. Elle s'arrêta ensuite vis-à-vis d'une des barrières pour admirer la belle verdure du parc. Au moment où elle se retiroit, elle entrevit au travers des arbres un homme qui s'approchoit. Craignant que ce ne fùt Mr. Darcy, elle vouloit s'éloigner, lorsqu'elle s'entendit appeler par son nom, et qu'elle le vit,\setcounter{page}{91} (' car c'étoit lui-même ) s'avancer avec une lettre à la main. Sans penser à ce qu'elle faisoit, elle prit cette lettre qu'il lui présentoit.
"Mademoiselle," dit-il en la lui remettant, "je me suis promené dans le parc avec l'espérance de vous rencontrer. Veuillez me faire l'honneur de lire ceci." Il s'inclina légèrement et rentra dans le parc.
Elisabeth ouvrit la lettre sans se promettre aucun plaisir de sa lecture, mais avec une extrême curiosité. La date étoit de Rosings, et de la nuit même. Elisabeth lut ce qui suit.
"Ne craignez point, mademoiselle, que je vous répète l'expression des sentimens, ni que je vous renouvelle des offres que vous avez rejetés avec dédain. Je n'ai point l'intention de vous faire du chagrin, ni de m'humilier devant vous en persistant dans un vœu qui, pour le bonheur de tous deux, ne sauroit être trop tôt oublié. Je me serois épargné l'effort d'écrire cette lettre, et à vous, mademoiselle, celui de la lire, si ma réputation ne se trouvoit compromise. Je demande donc de votre justice un peu d'attention à ce que je vais vous dire."
"Vous m'avez accusé de deux torts: l'un\setcounter{page}{92} d'avoir travaillé à détacher Mr. Bingley de votre sœur; l'autre, d'avoir manqué aux devoirs de l'honneur et de l'humanité envers Mr. Wickham. Cette dernière accusation est tout autrement grave. Séparer deux jeunes gens dont l'affection ne peut avoir duré que quelques semaines, est une bagatelle auprès de l'injustice et de la cruauté qu'il y auroit à perdre le camarade de mon enfance, le favori de mon père, celui qui auroit été élevé à tout espérer de moi. Il ne me fallut pas beaucoup de temps pour voir que mon ami Bingley avoit une préférence décidée pour miss Jane votre sœur. Comme je l'ai vu plusieurs fois amoureux, je n'en avois pas pris d'inquiétude, mais à ce même bal où j'eus l'honneur de danser avec vous, j'appris de Sir W. Lucas, qu'on parloit déjà du mariage de Bingley comme d'une affaire arrêtée. J'observai alors mon ami avec attention. Il me parut plus fortement occupé de miss Bennet que je ne l'eusse encore vu d'aucune femme. Je m'attachai à observer aussi miss Jane. Je vis qu'elle recevoit la cour de Bingley avec plaisir; mais qu'il n'y avoit rien qui annonçât en elle un goût décidé. Son calme et sa gaîté sereine m'indiquoient un naturel heureux, mais un cœur fort tranquille. En me\setcounter{page}{93} mettant à la place de mon ami, je trouvois plusieurs objections à ce mariage. Je dois vous les dire, pour ma justification. Indépendamment de ce que cette union ne lui donnoit aucune fortune, aucune relation utile à sa situation dans le monde, elle l'exposoit à des désagréments et à des mortifications de tous les jours. Il y avoit des objections à faire sur les parens de madame votre mère; mais ce n'étoit rien auprès de ce ton vulgaire, ou peu convenable, qu'elle a presque toujours, que vos sœurs cadettes ont quelquefois, et dont Mr. Bennet lui-même n'est pas tout-à-fait exempt. J'ai un extrême regret de blesser votre sentiment sur ce point délicat; mais je suis forcé, pour ma justification, de vous dire la vérité, tout en rendant hommage au tact parfait et à l'excellent ton, qui vous distinguent, ainsi que miss Jane au milieu de votre famille. Les sœurs de mon ami partageoient mes craintes sur les dispositions de leur frère. Nous nous entendîmes. Pendant notre voyage à Londres, je fus chargé de lui en parler; et je l'assurai que la préférence que votre sœur lui accordoit étoit un sentiment si foible, qu'il ne risqueroit point de lui donner un chagrin profond en l'oubliant. Bingley a beaucoup de confiance en moi. J'obtins de\setcounter{page}{94} jui qu'il renonceroit à voir Miss Bennet. Je n'ai qu'une chose à me reprocher là-dedans; c'est de lui avoir caché que votre sœur étoit à Londres en même temps que lui, et qu'il auroit pu la rencontrer. Je déteste les voies détournées; et je lui aurois dit la vérité sur ce point là, comme sur le reste, si je n'avois redouté la foiblesse de Bingley. A cet égard, j'ai agi pour le mieux; si j'ai eu tort; je m'en accuse; mais sur le fond de la chose, je n'ai rien fait que je ne fusse prêt à faire encore.
Relativement à Mr. Wickham, je ne puis me justifier qu'en vous faisant l'histoire de ses relations avec ma famille. Je ne sais de quoi il m'a accusé auprès de vous; mais il m'est facile de donner des preuves de tout ce que je vais vous dire. Mr. Wickham est fils d'un homme fort respectable, qui a eu pendant plusieurs années la conduite des terres de Pemberley. Mon père en faisoit grand cas. Il étoit parrain de George Wickham. Il fournit à ses dépenses à l'école et à l'université, chose que son père n'auroit point pu faire, parce qu'il avoit une femme qui dissipoit promptement tout ce qu'il gagnoit. Le jeune homme étoit d'une figure fort agréable; ses manières étoient séduisantes; mon père lui étoit singulièrement\setcounter{page}{95} attaché; il en avoit très-bonne opinion; et il projetoit de le destiner à l'église. Quant à moi, j'avois appris de bonne heure à le juger différemment, parce que je le voyois de plus près. Il avoit l'art de cacher à mon père ses inclinations vicieuses, et ses mauvais principes: il ne pouvoit y réussir avec moi, parce que nous étions camarades, et à-peu-près de même âge.
Il y a cinq ans que mon excellent père mourut, en léguant à George Wickham mille livres sterling, et en me recommandant particulièrement dans son testament, de servir son avancement de mon mieux, et dans le cas où il voudroit suivre la carrière de l'église, de lui procurer un des meilleurs bénéfices qui sont à la disposition de ma famille. Son père ne survécut que de fort peu de temps au mien; et six mois après cet événement, je reçus une lettre de Mr. Wickham. Il me représentoit qu'il n'avoit point de vocation pour l'église. Il espéroit que je n'aurois pas d'objection à compenser par une augmentation de son legs, les avantages que mon père avoit voulu lui faire, dans la carrière ecclésiastique, vu qu'il s'étoit décidé à étudier pour le barreau. Sans croire tout-à-fait à cette résolution, je consentis à entrer dans ses vues; et je convertis en une somme\setcounter{page}{96} de trois mille livres sterling, celle qui lui avait été léguée par mon père. Nous fûmes dès lors complètement séparés; j'avais trop mauvaise opinion de lui pour l'inviter à Pemberley, ou pour le recevoir dans ma maison à Londres, où il vivoit, sous prétexte d'étudier le droit, mais en effet pour y mener une vie d'oisiveté et de dissipation.
Pendant trois ans, je n'entendis presque point parler de lui; mais à la mort du Recteur, dont le bénéfice lui avoit d'abord été destiné, il m'écrivit pour me dire que sa fortune étoit fort réduite, que l'étude du barreau étoit ingrate, et que si je voulois le présenter pour le bénéfice en question, il étoit décidé à prendre les ordres. Il ne doutoit pas, ajoutoit-il, de mes intentions à son égard, vû les dispositions testamentaires de mon père. Vous ne me blâmerez pas assurément, mademoiselle, de m'être refusé à cette demande, qui fut répétée à plusieurs reprises. Il en fut de fort mauvaise humeur, et je suppose qu'il ne m'a pas ménagé à cette occasion. De ce moment-là, je le perdis complétement de vue jusqu'à l'été dernier, que j'eus le malheur de le rencontrer comme vous savez.
Je suis forcé de dire ici une chose que je n'ai dite à personne jusqu'à ce moment,\setcounter{page}{97} et du secret de laquelle vous sentirez toute l'importance. Ma sœur Georgiana, qui a dix ans de moins que moi, étoit sous la tutelle d'un neveu de ma mère, le colonel Fitz-William, et sous la mienne. Il y a un an que nous la retirâmes de l'école, pour l'établir à Londres avec sa gouvernante Mad. Younge. L'été dernier, celle-ci conduisit ma sœur à Ramsgate. Mr. Wickham s'y rendit aussi: il paroissoit que cela étoit arrangé d'avance, parce que nous découvrimes ensuite qu'il avoit des relations très-particulières avec Mad. Younge, sur le compte de laquelle nous avions été grossièrement trompés. Georgiana, qui avoit quinze ans, et qui avoit conservé de la reconnoissance des soins que Mr. Wickham avoit eus d'elle lorsqu'elle étoit enfant, se laissa persuader qu'elle avoit de l'amour, et consentit à s'enfuir avec lui. Le hasard fit que j'allai la voir la veille du jour fixé pour l'enlèvement. La pauvre enfant, qui me regardoit comme un père, et avoit en moi une entière confiance, me fit l'aveu d'une résolution qu'on avoit arrachée d'elle, et qui la tourmentoit. Jugez de ce que j'éprouvai. Tout éclat auroit eu les plus graves inconvéniens. Mais j'écrivis immédiatement à Mr. Wickham, qui partit le même\setcounter{page}{98} jour, et je remplaçai la gouvernante. Il est évident que Mr. Wickham en vouloit surtout à la fortune de ma sœur, qui est de trente mille livres sterling; mais je ne puis m'empêcher de croire que le désir de se venger de moi étoit aussi pour lui un mobile puissant. Il est bien vrai qu'il se seroit vengé d'une manière cruelle.
Voilà, mademoiselle, le récit exact de ce qui s'est passé entre nous; et si vous ne le rejetez pas complétement comme faux, vous ne m'accuserez plus, à l'avenir, de cruauté et d'injustice envers Mr. Wickham. J'ignore de quelle manière il m'a calomnié auprès de vous; et je ne dois pas m'étonner qu'il ait réussi, puisque vous n'aviez aucune connoissance des faits. Vous serez surprise, peut-être, que je ne vous aie pas dit tout cela hier au soir; mais je ne me sentois pas assez maître de moi-même pour être sûr de ne dire que ce que je voudrois. J'en appelle au colonel Fitz William sur la vérité de ce que je viens de vous exposer: comme exécuteur-testamentaire de mon père, et comme mon ami, il n'a ignoré aucun détail. Si l'aversion que je vous ai inspirée ne vous permet pas d'ajouter foi à ce que je vous dis, vous n'aurez aucune raison de douter de ce que vous dira le colonel; et pour que vous\setcounter{page}{99} puissiez le consulter, je tâcherai de vous rencontrer à votre preomenade du matin et vous remettrai ceci. Agréez tous med vœux.
FITZ WILLIAM DARCY.
(Elisabeth ne conserve aucun doute sur la vérité des faits contenus dans la lettre de Darcy. Elle se reproche ses préventions, et la vanité qui l'a induite en erreur. Darcy quitte Rosings sans revoir Elisabeth, qui retourne chez ses parens. Lydie est invitée par une amie à aller passer quelque temps aux eaux de Brighton, où le régiment de Wickham est en garnison. Ses parens y consentent, malgré les représentations d'Elisabeth à son père sur les inconvéniens d'un tel séjour, pour une jeune personne fort étourdie. Mr. Gardiner, frère de Mad. Bennet, et sa femme, proposent à leur nièce Elisabeth une tournée dans le Derbyshire, où Mad. Gardiner a été élevée, et où se trouve la terre de Pemberley, qui appartient à Mr. Darcy. Arrivés dans le voisinage de ce lieu, ils apprennent que le propriétaire en est absent, et ont la curiosité de voir la maison et les jardins).
Quand la voiture approcha des bois de Pemberley, Elisabeth éprouva quelqu'émotion ;\setcounter{page}{100} et lorsqu'ils entrèrent par la porte de la loge du parc, elle se sentit agitée. Le parc étoit très-vaste ; et ils traversèrent d'abord de très-beaux bois, en montant un peu pendant environ un quart d'heure. En sortant du bois, ils se trouvèrent sur une éminence, d'où ils eurent la vue du château, de l'autre côté de la vallée. On voyoit serpenter la route qui y conduisoit. Ce château étoit un bel édifice, d'une noble architecture, fort bien situé, et derrière lequel s'étendoit une forêt sur le penchant d'un côteau. Dans la vallée, couloit une jolie petite rivière, dont les bords offroient une riche végétation. Elisabeth étoit enchantée. Elle n'avoit jamais vu un endroit pour lequel la nature eût fait davantage, et que le mauvais goût eût moins gâté. Son oncle et sa tante n'étoient pas moins charmés qu'elle de la beauté de cet ensemble. Dans ce moment là, Elisabeth se représenta que c'auroit été pourtant une chose agréable que d'être maîtresse de Pemberley. Ils passèrent le pont, et s'approchèrent du château. Alors les craintes d'Elisabeth revinrent. Elle s'imagina que peut-être les domestiques de l'auberge s'étoient trompés, et que Mr. Darcy seroit chez lui. Ils demandèrent à voir le château ; et pendant qu'on appeloit la femme de l'intendant,\setcounter{page}{101} Elisabeth eut le temps de réfléchir à la singularité de cette visite.
La gouvernante arriva. Elle était beaucoup plus polie et moins grande dame qu'Elisabeth ne se l'était représenté. Ils la suivirent dans la salle à manger. Elle était grande, ornée, et meublée avec goût. Elisabeth s'approcha d'une des croisées pour admirer la vue, qui s'étendait au loin sur les détours de la rivière et de la vallée. L'effet en était enchanteur. A mesure qu'ils parcouraient les divers appartemens, Elisabeth en admirait la noblesse. Partout les ameublemens, les sculptures et les divers ornemens étaient riches et de bon goût, mais sans recherche : il y avait moins de luxe, mais plus de véritable élégance qu'à Rosings. Elle se répétait à elle-même, qu'elle aurait pu être maîtresse de tout cela; qu'elle aurait pu y avoir ses amis à demeure, y inviter son oncle et sa tante. "Ah, non!" se dit-elle ensuite, "on n'aurait pas permis de les inviter, et ils auraient été perdus pour moi. "Cette réflexion prévint un sentiment qui commençait à tenir du regret.
Elle avait envie de demander à la gouvernante si Mr. Darcy était bien véritablement absent; mais elle n'osa pas. Son oncle fit cette question, et ils apprirent que\setcounter{page}{102} Mr. Darcy étoit attendu le lendemain avec plusieurs amis. Elisabeth se félicita vivement de ce que leur partie n'avoit pas été renvoyée d'un jour.
Sa tante l'appela pour voir un portrait en miniature : c'étoit celui de Mr. Wickham.
"C'est," dit la gouvernante, "le portrait d'un jeune homme qui a été élevé dans la maison, et qui à présent est officier. Mais on dit qu'il tourne mal."
Mad. Gardiner regarda sa nièce en souriant; mais Elisabeth n'ent pas la force de lui rendre ce sourire.
"Voici le portrait de mon maître," ajouta la gouvernante, "et il est parfaitement ressemblant. Il y a à présent huit ans qu'il a été peint."
"J'ai ouï dire que votre maître étoit un fort bel homme," dit Mr. Gardiner, "voilà assurément une physionomie distinguée. Lizzy peut nous dire à qui cela ressemble."
La considération de la gouvernante pour Elisabeth augmenta beaucoup, quand elle comprit que son maître en étoit connu.
"Cette demoiselle connoît-elle Mr. Darcy?," dit-elle.
Elisabeth rougit beaucoup et répondit : "oui, un peu."
"Et ne trouvez-vous pas, mademoiselle, que c'est un bien bel homme?"
\setcounter{page}{103}
"Mais.... certainement."
"Quant à moi, je ne crois pas d'avoir jamais vu de plus bel homme de ma vie. Vous verrez son portrait en pied dans la galerie. Cette chambre-ci étoit celle de mon maître défunt, et on a laissé ces miniatures comme il les avoit placées."
Cela expliqua à Elisabeth pourquoi le portrait de Mr. Wickham étoit là.
La gouvernante lui montra ensuite un portrait de miss Darcy, fait à l'âge de six ans.
"Est-elle d'une aussi jolie figure que son frère?" dit Mr. Gardiner.
"Ah! oui, elle est charmante; et elle a des talens infinis. Il y a dans l'autre chambre un piano qui vient d'arriver pour elle. Elle sera ici demain avec mon maître."
Mr. Gardiner, qui étoit communicatif, encouragea le babil de la gouvernante, en lui faisant des questions. "Votre maître est-il souvent à Pemberley?" lui dit-il.
"Pas autant que je le voudrois. Il y passe environ six mois de l'année, et miss Darcy est ici dans les mois d'été. " Excepté, " pensa Elisabeth, " lorsqu'elle est à Ramsgate."
"Si votre maître se marioit," reprit Mr. Gardiner, " vous le verriez davantage."
"Assurément, mais je doute qu'il trouve une femme digne de lui, " Mr. et Mad.\setcounter{page}{104} Gardiner sourirent, et Elisabeth observa que ce que la gouvernante disoit-là était bien honorable à son maître.
"C'est l'opinion de tous ceux qui le connoissent." Elisabeth trouva que c'était dire beaucoup, et son étonnement s'accrut quand la gouvernante ajouta : je ne lui ai jamais vu un instant de mauvaise humeur, et je l'ai eu sous les yeux depuis l'âge de quatre ans."
Cet éloge était complètement opposé à l'idée qu'elle s'était faite du caractère de Mr. Darcy. Son attention fut vivement excitée, et elle aurait voulu en entendre davantage.
"Il y a fort peu d'hommes dont on puisse faire le même éloge, dit Mr. Gardiner. Vous êtes bien heureuse d'avoir un tel maître."
"Je ferais bien le tour du monde sans en trouver un pareil ! mais j'ai toujours remarqué que ceux qui sont bons, étant enfants, deviennent des hommes excellens, et c'était bien le meilleur caractère d'enfant qu'il fût possible d'imaginer."
"Son père était un homme très-respectable," dit Mad. Gardiner.
"Oh, oui, madame, assurément ! Mais son fils ne lui cède en rien." Elle continua à s'étendre sur les qualités et les vertus de son\setcounter{page}{105} maître; et elle ajouta: "il y a des gens qui le trouvent trop fier; mais c'est apparemment parce qu'il est silencieux et ne parle pas au hasard comme les jeunes gens d'aujourd'hui."
"Voilà un éloge," dit Mad. Gardiner à l'oreille d'Elisabeth, "qui ne ressemble pas à sa conduite avec notre ami."
"Nous pourrions avoir été trompés," dit Elisabeth.
On leur montra un joli appartement meublé encore avec plus d'élégance, que les autres, et qu'on destinoit à miss Darcy. La gouvernante parla à cette occasion de l'amitié tendre et délicate de son maître pour miss Darcy.
Ils passèrent dans la galerie de tableaux, où il y avoit une longue suite de portraits de famille. Elisabeth chercha d'abord des yeux le seul portrait qui l'intéressât. Il étoit en pied, de grandeur naturelle, et très-ressemblant. Elle le fixa long-temps, et y revint après l'avoir quitté quelques momens. Les éloges que la gouvernante prodiguoit à son maître l'avoient disposée plus favorablement pour lui qu'elle ne l'eût encore été. Elle se disoit qu'un homme qui est bon maître et bon frère, devoit aussi être bon mari; et tout en fixant ce portrait qui lui\setcounter{page}{106} rappeloit son expression au moment où il lui avoit offert sa main, elle se sentit une véritable reconnaissance de cet hommage.
Après avoir parcouru toute la maison, ils descendirent au jardin, où le jardinier les attendoit pour leur montrer les plantations. Quand ils eurent fait cinquante pas sur l’esplanade devant la maison, Elisabeth se retourna pour voir l’effet de la façade. Mr. Gardiner faisoit ses conjectures sur l’époque où la maison avoit été bâtie, lorsque tout-à-coup Mr. Darcy se montra à vingt pas d’eux, un fouët à la main et en bottes. Il revenoit des écuries, où il avoit mis pied à terre.
Il étoit impossible de l’éviter. Il rougit autant qu’Elisabeth en l’apercevant. Il resta un moment immobile, avec les signes d’une extrême surprise. Il se remit cependant assez promptement; et s’avançant vers Elisabeth, il lui parla avec beaucoup de politesse, mais non sans émotion. Elle avoit voulu s’éloigner; et elle reçut son compliment avec un embarras évident. Son oncle et sa tante en voyant l’étonnement du jardinier et son air de respect, comprirent que c’étoit Mr. Darcy. Ils se tinrent à une certaine distance, pendant qu’il faisoit la conversation avec leur nièce, qui dans son trouble, ne savoit ce\setcounter{page}{107} qu'elle répondoit aux questions qu'il lui adressoit sur la santé de ses parens. Elle étoit singulièrement frappée du changement de son ton, depuis leur dernière conversation. L'idée d'avoir blessé les convenances, en se trouvant ainsi chez lui en son absence, ajoutoit à son embarras. Lui-même n'étoit pas plus à son aise ; et il répéta deux ou trois fois les mêmes questions, d'une voix altérée, sans bien savoir ce qu'il disoit. Enfin après être resté un instant en silence, il la salua et se retira.
Mr. et Mad. Gardiner s'approchèrent alors, firent l'éloge de la figure de Mr. Darcy; mais Elisabeth ne les entendit point. Le dépit et la honte lui ôtoient toute présence d'esprit. Elle se reprochoit la faute qu'elle avoit faite de venir à Pemberley. Qu'alloit-il dire de cet empressement à se trouver sur son chemin, lui qui étoit rempli de vanité? Ah pourquoi étoit-elle venue! Pourquoi étoit-il arrivé un jour plus tôt qu'on ne l'attendoit. Quelle fatalité avoit retardé leur départ de quelques minutes! Elle ne pouvoit pas s'en consoler. Après cela, que signifioit son changement de ton? N'étoit-il pas étrange qu'il lui eût même adressé la parole? Mais la politesse qu'il avoit montrée l'étonnoit bien plus encore. Il s'étoit informé de toute sa\setcounter{page}{108} famille. Jamais elle ne l'avoit vu si attentif et si bon. Elle ne savoit comment s'expliquer tout cela.
Elle suivoit son oncle et sa tante dans une promenade le long de la rivière, où les beaux points de vue se succédoient : elle n'en distinguoit aucun, quoiqu'elle eût l'air de regarder le paysage. Sa pensée étoit toute entière au château de Pemberley ; et elle cherchoit à se représenter ce qui se passoit alors dans l'esprit de Mr. Darcy. Après ce qui étoit arrivé, pouvoit-il conserver pour elle un sentiment tendre ? et n'étoit-il pas évident que puisqu'il lui avoit parlé, c'étoit parce qu'il avoit le cœur libre ? Que signifioit cependant l'émotion qu'elle avoit remarquée dans le son de sa voix, et le peu de suite de ses discours ? Avoit-il éprouvé plus de chagrin que de plaisir en la voyant ? C'est ce qu'elle ne pouvoit décider ; mais certainement il ne l'avoit pas revue de sang froid. Les observations de son oncle et de sa tante, lui rendirent enfin sa présence d'esprit. Ils entrèrent dans le bois, en s'éloignant un peu de la rivière. Mr. Gardiner vouloit, disoit-il, faire le tour du parc, et il demanda au jardinier si c'étoit une longue promenade. "C'est une affaire de dix milles," dit le jardinier en souriant d'un air de triomphe. Cela leur ôta la tentation d'en\setcounter{page}{109} treprendre ce tour; mais en suivant les sentiers pratiqués au bord de la rivière, ils allongèrent si bien la promenade, que Mad. G. fatiguée, demanda à revenir vers la maison pour gagner leur voiture. Quant à Mr. G., il étoit si occupé à regarder les truites dans la rivière, qu'il restoit toujours le dernier. Quelle fut la surprise d'Elisabeth en voyant Mr. Darcy qui venoit à eux, et étoit déjà tout près, sans qu'on eût pu l'apercevoir plus tôt, parce que le bois le cachoit! Elle étoit cependant un peu mieux préparée que la première fois, et elle résolut de faire tous ses efforts pour paroître calme. Elle eut encore l'espérance qu'il alloit prendre un autre sentier; mais point du tout, il vint à elle, et la salua avec politesse. Elle voulut essayer, pour avoir l'air à son aise, de lui parler la première de la beauté de son parc; puis tout-à-coup, elle réfléchit qu'elle auroit l'air de regretter cette propriété qui auroit pu devenir la sienne. Elle hésita, elle rougit, et se tut. Mad. Gardiner étoit à quelque distance derrière. Mr. Darcy demanda à Elisabeth si elle vouloit bien lui faire l'honneur de le présenter à ses parens. Elle n'étoit guères préparée à cette attention de sa part; et elle put à peine s'empêcher de sourire de le voir demander d'être présenté à des gens contre\setcounter{page}{110} lesquels son orgueil l'avoit tant prévenu. Il sera bien surpris, pensoit-elle, quand il saura qui ils sont. Tout en les lui nommant, elle jeta sur lui un coup-d'œil à la dérobée, pour juger de l'effet de cette connoissance. Il fut évidemment surpris ; mais au lieu de s'en aller, comme Elisabeth s'y étoit attendue, il entra immédiatement en conversation avec Mr. Gardiner. Elle écouta avec plaisir son oncle qui étoit un homme d'esprit et de sens, et se glorifia d'avoir en lui un parent qui lui faisoit honneur auprès de Mr. Darcy. Ils se mirent à marcher en avant, tandis qu'elle demeuroit un peu en arrière avec sa tante, mais de manière à entendre ce qui se disoit. Ils parlèrent de pêche ; et Mr. Darcy indiquant les endroits où il y avoit le plus de truites, offrit à Mr. des lignes pour pêcher. Elisabeth se disoit à elle-même que sans doute ce qu'il en faisoit étoit pour elle ; cependant il lui sembloit impossible qu'il eût conservé de l'inclination après tout ce qui s'étoit passé. Quelques momens après, Mad. se trouvant trop fatiguée pour que le bras de sa nièce lui suffît, pria son mari de lui donner le sien. Alors, tout naturellement, Mr. Darcy fut appelé à offrir son bras à Elisabeth. Le silence régna pendant\setcounter{page}{111} quelques momens. Elisabeth parla la première. Elle eut soin de lui dire qu'avant de venir à Pemberley elle avoit appris qu'il n'y étoit point; "et la femme de votre intendant nous avoit encore assuré," dit-elle, "que vous ne deviez arriver que demain."
Il répondit qu'ayant des affaires à régler avec son intendant, il avoit voulu précéder de quelques heures les personnes qui venoient demeurer chez lui. "Ce sont de vos amis," ajouta-t-il : "C'est Mr. Bingley et ses sœurs."
Ce nom rappela à Elisabeth tout ce qu'elle avoit éprouvé la dernière fois qu'il avoit été articulé entr'eux. Darcy avoit rougi en le prononçant, et avoit sans doute le même souvenir. Il y eut un silence, puis il lui dit: "Parmi les arrivans, il y a une personne qui demandera la faveur de vous être présentée pendant votre séjour à Lambton : c'est ma sœur." Elisabeth fut très-étonnée; mais évidemment le desir de miss Darcy ne pouvoit lui venir que de son frère, elle en eut un sentiment agréable, qu'elle n'exprima pas par des paroles. Ils continuèrent à marcher en silence, et les réflexions d'Elisabeth étoient assez douces. Lorsqu'ils atteignirent la voiture, ils se trouvoient de cent pas en avant de Mr. et Mad. Gardiner. Mr. Darcy lui proposa d'entrer pour se reposer un moment;\setcounter{page}{112} mais elle dit qu'elle n'étoit pas fatiguée, et ils restèrent debout sur la pelouse. Dans une telle situation, rien n'étoit plus gauche que le silence. Elle le sentoit, et auroit voulu trouver quelque chose à dire; mais il sembloit y avoir un embargo sur tous les sujets. Enfin elle se rappela son voyage; et ils se mirent à parler Matlock et Dove-Dale. Elle trouva le temps très long; et tous ses moyens de conversation étoient épuisés, lorsqu'enfin son oncle et sa tante vinrent rompre le tête-à-tête. Mr. Darcy les pressa tous trois d'entrer au château pour prendre quelques rafraîchissements; mais ils s'y refusèrent avec politesse, et prirent congé. Mr. Darcy donna la main aux dames pour monter en voiture; et Elisabeth en s'éloignant le vit qui s'acheminoit vers la maison à pas lents.
Alors commencèrent les observations sur Mr. Darcy. "Il est impossible d'être plus poli et plus modeste," dit Mr. Gardiner.
"Il a quelque chose de fier dans l'extérieur," dit Mad. G.; mais à présent je vois
que c'est dans son air seulement, et je comprends l'éloge qu'en faisoit la femme de l'intendant."
"Je ne puis dire à quel point j'ai été surpris de sa manière d'être avec nous," reprit Mr. "Il a été plus que poli, il a été très-attentif;\setcounter{page}{113} et cela est d'autant plus remarquable, qu'à peine il connoissoit Elisabeth.
"Mais dites-moi, Lizzy," continua Mad. G., "où avez-vous donc pris qu'il a une figure désagréable ?"
Elisabeth s'excusa comme elle put, et convint qu'elle ne l'avoit jamais trouvé si bien.
"Au reste," dit son oncle, "il se pourroit qu'il fût capricieux dans sa politesse : cela arrive souvent aux gens de qualité, et je ne voudrois pas jurer qu'un autre jour il me fît comme aujourd'hui les honneurs de ses truites."
"Je trouve qu'il a dans le regard," poursuivit Mad. G., "quelque chose d'extrêmement noble, et on ne diroit pas que c'est le même homme qui s'est si mal conduit avec Mr. Wickham."
Elisabeth se crut obligée de justifier sur ce point, la conduite de Darcy, et expliqua que ce qu'elle avoit ouï dire en Kent présentoit son différend avec Wickam sous un tout autre point de vue.
(Mr. Darcy fait une visite à Elisabeth, et lui amène sa sœur à Lambton. Celle-ci paroît prévenue en faveur d'Elisabeth et lui plaît beaucoup. Bingley qui les accompagne, s'informe avec intérêt de la famille Bennet, \setcounter{page}{114} et trahit un sentiment pour miss Jane, qui redonne de l'espérance à sa sœur. Les Gardiner rendent la visite à Pemberley. Ils y trouvent miss Bingley et sa sœur. Celles-ci sont toujours les mêmes pour Elisabeth et laissent percer en toute occasion, de l'envie et de la malice. Elisabeth, de retour à Lambton, reçoit deux lettres de miss Jane, dont l'une a été fort retardée. Elles lui apprennent la fuite de Lydie avec Wickham. On ne sait point ce qu'ils sont devenus, et la famille, éclairée sur le caractère du personnage, est dans les plus vives alarmes. Ils demandent instamment à Mr. Gardiner de venir leur donner conseil et secours. En achevant la lecture de la lettre, Elisabeth se leva avec précipitation en s'écriant : "il n'y a pas un moment à perdre : il faut chercher mon oncle." Elle s'élança vers la porte, qu'un domestique ouvroit en annonçant Mr. Darcy. Il fut frappé de sa pâleur et de l'altération de ses traits. Sans lui donner le temps de dire un mot, elle s'écria : je vous en demande pardon, mais je ne peux pas vous recevoir. Il faut que je joigne Mr. Gardiner et je n'ai pas un instant à perdre. "Ah Dieu ! qu'est-il donc arrivé," s'écria Darcy, "permettez-moi d'envoyer cher-cher\setcounter{page}{115} Mr. Gardiner: vous n'êtes pas assez bien pour aller vous-même.
Elisabeth hésita; mais elle sentoit ses genoux trembler et elle s'assit, puis elle donna commission au domestique de chercher son oncle et sa tante le plus vite possible. Elle étoit si troublée qu'elle pouvoit à peine parler.
Darcy fort inquiet, voulut appeler quelqu'un. Elle s'y opposa absolument. Je n'ai besoin de personne, "s'écria-elle, " mais je viens de recevoir de Longbourn des nouvelles affreuses. En achevant ces mots elle se mit à pleurer, et fut pendant quelques instans incapable de répondre aux questions de Darcy. Enfin elle lui dit: "il est inutile de vouloir cacher ce qui ne sauroit l'être : ma sœur cadette a quitté la maison de mes parens : elle est partie de Brighton avec Mr. Wickham. "
Darcy fut comme pétrifié.
"Et quand je pense que j'aurois pu prévenir la chose, " s'écria-t-elle avec un sentiment d'amertume. " Je le connoissois, j'aurois dû dire à mes parens ce que je savois de lui. "
" A-t- on pris quelques mesures?" dit Darcy.
"Mon père est parti pour Londres. Ma sœur aînée a écrit pour demander conseil\setcounter{page}{116} à mon oncle, et dans une demi-heure nous serons partis; mais j'ai bien peu d'espérance, que peut-on attendre d'un homme de cette espèce? qui sait même si on parviendra à les découvrir!...... Ah, si j'avais agi dès le premier instant où j'ai connu ce caractère!" Darcy ne répondit point; il se promenait dans la chambre d'un air sombre et les yeux fixés en terre. Élisabeth sentoit avec amertume tout ce qu'un pareil événement avoit d'humiliant pour sa famille. Elle croyoit voir dans l'expression de la physionomie de Darcy, qu'il s'applaudissoit d'avoir été rejeté par elle. Elle porta son mouchoir sur ses yeux, et garda un silence interrompu par de profonds soupirs.
"Je sens," lui dit Darcy d'une voix émue, "que ma présence vous gêne; mais je n'ai pas le courage de m'éloigner de vous au moment où vous éprouvez un si vif chagrin. Plut à Dieu que j'y pusse quelque chose! Cette malheureuse affaire empêchera que ma sœur n'ait le plaisir de vous voir aujourd'hui à Pemberley."
"Ah! je vous en supplie excusez-moi auprès d'elle. Dites-lui qu'une affaire pressante me force à partir; mais cachez-lui la vérité aussi long-temps qu'il sera possible. Hélas! je sens bien que cet événement ne tardera pas à être public!"\setcounter{page}{117} Il l'assura qu'il en garderoit le secret, lui exprima ses vœux, et la chargea de dire à ses parens tout l'intérêt qu'il prenoit à eux dans cette occasion. Il parut profondément pénétré en lui disant adieu ; et Elisabeth en le voyant partir se dit à elle-même, que probablement jamais elle ne se retrouveroit avec lui dans une relation aussi amicale que celle qui les avoit réunis en Derbyshire. ( Les Gardiner et leur nièce retournent immédiatement à Longbourn. Ils trouvent toute la famille dans une grande inquiétude. Mr. Bennet est parti pour Londres, où l'on suppose que les fugitifs sont allés ; mais on n'en a aucune nouvelle. Mr. Gardiner, après avoir tenu conseil avec sa sœur et ses nièces, va à Londres joindre son beau-frère. Il se passe quelques jours sans que leurs perquisitions aient aucun succès. Mr. Collins écrit à Mr. Bennet la lettre suivante ) .
Mon cher monsieur,
Je me sens appelé par nos relations de parenté, et par mon état, à vous faire un compliment de condoléance sur l'évènement dont j'ai été informé par les lettres du Herefordshire. Ne doutez pas, mon cher monsieur, que Mad. Collins et moi ne sympathisions avec vous et votre respectable famille\setcounter{page}{118} dans cette occasion d'un chagrin du genre le plus amer, puisque le temps ne sauroit jamais l'adoucir. Soyez convaincu, monsieur, que de mon côté, je n'épargnerai aucun raisonnement qui puisse tendre à mitiger une si grande douleur; et apporter quelque consolation dans l'ame déchirée d'un père. La mort de votre pauvre fille auroit été un bonheur en comparaison de ce qui est arrivé, chose d'autant plus à déplorer, que sur ce que ma chère Charlotte m'a dit, il paroît qu'il y a eu de la faute des parens dans l'extrême indulgence qu'ils ont eue pour leur fille. Cependant, pour votre consolation ainsi que celle de Mad. Bennet, je vous dirai que je suis persuadé qu'il falloit que cette jeune fille eût naturellement de très mauvaises dispositions pour avoir commis une faute si énorme dans un âge si tendre. Quoiqu'il en soit, vous êtes certainement fort à plaindre. Non seulement ma femme, mais lady Catherine et sa fille à qui j'ai communiqué la chose, vous plaignent tout comme moi; et elles sont convaincues que cette faute d'une de vos fille ne manquera pas de nuire à toutes les autres. Lady Catherine a eu la bonté de s'en affliger en observant que personne ne voudroit s'allier à une famille qui a éprouvé un tel malheur.\setcounter{page}{119} Cette réflexion me conduit à me rappeler avec satisfaction un certain événement du mois de novembre dernier; car à quoi a-t-il tenu que je ne fusse enveloppé moi-même dans les suites de la faute que vous déplorez! Je ne puis que vous exhorter, mon cher monsieur, à tâcher de vous consoler, et de rejeter de votre cœur pour toujours cette indigne enfant, lui laissant savourer les fruits amers de sa faute. J'ai l'honneur d'être, etc.
(Les informations prises par Mr. Gardiner ne sont point de nature à tranquilliser la famille Bennet. On ne sait rien de Mr. Wickham, si ce n'est qu'il a fait des dettes de jeu et autres à Brighton pour plus de mille livres sterling, et que c'est encore une raison pour lui de se cacher. Mr. Bennet, découragé par l'inutilité de ses recherches, revient à Longbourn, en laissant à son beau-frère le soin de continuer les perquisitions. Bientôt après, on reçoit une lettre de Mr. Gardiner. Il a enfin découvert Lydie. Elle n'est point encore mariée, mais Mr. Gardiner s'est chargé de communiquer à Mr. Wickham les conditions au moyen desquelles Mr. Wickham s'engage à épouser Lydie. Il demande\setcounter{page}{120} que son père lui fasse une pension de 100 liv. sterl. par an, et lui en assure cinq mille par son testament. Mr Bennet se trouve trop heureux de sauver à ce prix la réputation de sa fille. Il soupçonne que son beau-frère a fait un sacrifice d’argent pour obtenir ce résultat, et se promet d’acquitter cette dette dès que cela lui sera possible. En attendant que le mariage se fasse, Mr. Gardiner retire sa nièce chez lui. Wickham satisfaisait ses créanciers et obtient de l’avancement dans un autre corps destiné au nord de l’Angleterre, sans qu’on sache la source de ses moyens.
Le jour fixé pour le mariage arriva. Miss Jane et Elisabeth étoient l’une et l’autre fort ébranlées, lorsque le moment approcha où l’on attendoit le retour de la voiture qui étoit allé chercher les époux; elles supposoient à leur sœur le même sentiment qu’elles auroient eu à sa place. Mad. Bennet parut transportée de joie lorsqu’on entendit la voiture, son mari avoit, au contraire, l’expression la plus sombre.
La voix de Lydie se fit entendre dans le vestibule. La porte s’ouvrit, et elle s’élança dans le salon d’un air gai et assuré. Sa mère s’avança à sa rencontre, et l’embrassa tendrement. Elle tendit la main à Mr. Wickham en lui souriant affectueusement, et les félicita\setcounter{page}{121} tous deux sur leur union d'une manière qui prouvoit qu'elle n'avoit aucun doute sur leur bonheur futur.
La réception de Mr. Bennet, ne fut pas si cordiale. Son air étoit sévère, et il ouvrit à peine la bouche. L'assurance parfaite de Lydie et de son époux étoit en effet d'une impudence choquante. Elisabeth en étoit indignée, et même l'indulgente miss Bennet ne trouvoit point d'excuse pour sa sœur. Lydie étoit toujours la même ; toujours étourdie, bruyante, pleine de confiance en elle-même; sans jugement, sans tact et sans timidité. Elle s'adressoit successivement à chacune de ses sœurs pour recevoir des complimens de félicitations, et lorsqu'enfin tout le monde fut assis elle regarda autour d'elle, et dit en riant, qu'il y avoit bien long-temps qu'elle avoit quitté la maison.
Wickham n'étoit pas plus embarrassé qu'elle, et il avoit tant de grace dans les manières, et tant de talent pour se faire bien vouloir de tout le monde, que s'il avoit été possible d'oublier son caractère et les circonstances qui avoient précédé le mariage, la famille auroit été contente de lui.
Elisabeth, qui savoit mieux que personne ce que cachait cet extérieur séduisant, ne pouvoit concevoir un tel degré d'effronterie.\setcounter{page}{122} Elle et miss Jane rougirent souvent pour Lydie et pour son mari, mais ceux-ci ne paroissoient s'apercevoir de rien qui pût le moins du monde les humilier.
Lydie et sa mère causèrent beaucoup, et Wickham, qui s'étoit assis auprès d'Elisabeth, se mit à lui demander des nouvelles de toutes ses connoissances, d'un ton si dégagé, qu'Elisabeth put à peine prendre sur elle de lui répondre. Wickham et son épouse mirent la conversation sur des sujets auxquels pour rien au monde Elisabeth et sa sœur n'avoient voulu toucher.
"Peut-on comprendre qu'il y ait déjà trois mois que je suis partie d'ici ?" s'écria Lydie, "je jurerois qu'il n'y a que quinze jours! et pourtant, que de choses se sont passées dès lors! Ma foi! celui qui m'auroit dit que je reviendrois mariée m'auroit bien surprise! C'est pourtant une drôle de chose!"
Mr. Bennet leva les yeux au ciel; miss Jane souffroit le martyre. Elisabeth jeta à Lydie un coup-d'œil que toute autre auroit compris; mais celle-ci, qui ne voyoit et n'entendoit que ce qu'il lui convenoit d'entendre et de voir, n'y fit aucune attention et continua son babil. "Oh! maman, les gens du voisinage savent-ils que je suis mariée d'aujourd'hui ? Voici ce que j'ai fait pour\setcounter{page}{123} qu'on le sût bien vite. Nous avons dépassé le carricle de William Goulding, et moi, j'ai baissé la glace de son côté; j'ai ôté mon gant et j'ai posé ma main sur la portière afin qu'il vît mon anneau. Ensuite j'ai salué et j'ai souri d'un air qui expliquait tout.
Elisabeth ne put pas y tenir plus long-temps. Elle sortit de la chambre, et ne rentra que lorsqu'on se mit à table. Elle arriva tout à point pour entendre Lydie qui disait à sa sœur aînée en se plaignant à la droite de sa mère. "Ah! ah! miss Jane, vous aurez la bonté de descendre d'une place à présent que je suis mariée."
Lydie était de plus en plus animée et bruyante. Elle se réjouissait de voir sa tante Philips, les Lucas, et tous les gens du voisinage pour s'entendre appeler Mad. Wickham, et en attendant, elle courut après dîner chercher la gouvernante et les femmes-de-chambre pour leur montrer sa bague de noces.
"Eh bien, maman," dit-elle, "quand la famille fut de nouveau rassemblée, "comment trouvez-vous mon mari? n'est-ce pas qu'il est charmant? Je gage que mes sœurs me l'envient bien! Je leur souhaite d'attraper seulement la moitié aussi bien que moi. Il faut\setcounter{page}{124} les envoyer à Brighton. C'est un endroit parfait pour accrocher des maris. Quel dommage, maman, que nous n'y soyons pas toutes allées!"
C'est ce que j'ai toujours dit, mais personne ne m'écoute ici," dit Mad. Bennet. Cependant, ma chère amie, je ne puis pas prendre mon parti de vous voir établie si loin de nous. Cela ne peut-il donc pas être autrement?
"Oh, mon Dieu, non! maman; mais qu'est-ce que cela fait? Je me réjouis bien d'être à Newcastle, vous viendrez me voir, avec papa et mes sœurs. Laissez-moi faire: je leur procurerai des danseurs convenables quand il y aura des bals."
"Cela serait bien agréable," dit la mère. "Eh puis, quand vous partirez, vous me laisserez une ou deux de mes sœurs, et je parie qu'avant que l'hiver soit passé, je vous les rends mariées?"
"Je vous remercie pour ma part," dit Elisabeth, "mais je n'aime pas beaucoup votre manière de trouver des maris."
Le lendemain matin, Lydie étant seule avec ses deux sœurs aînées, leur dit: "vous n'étiez pas là quand j'ai raconté le mariage à maman et aux autres: il faut pourtant que je vous dise comment cela s'est passé, n'en êtes-vous pas bien curieuses?"
\setcounter{page}{125}
"Non, en vérité," répondit Elisabeth, moins on en parlera mieux ce sera. "
"Mon Dieu, que vous êtes bizarre ! mais c'est égal : je m'en vais vous conter tout cela. Vous savez que nous avons été mariés à l'église de St. Clément, parce que c'est dans cette paroisse que Wickham logeoit. Le rendezvous étoit pour onze heures. Je devois aller avec mon oncle et ma tante. Lundi matin arrive. J'étois toute bouleversée. Je mourois de peur qu'il ne survint quelque chose qui fit tout renvoyer; car cela m'auroit mis au désespoir. Et par là-dessus voilà ma tante, qui pendant que je m'habillois se mit à me faire des sermons sans fin. Ah! c'étoit d'un ennui ! il est vrai que je n'écoutois guères, car vous pouvez croire que je pensois à mon cher Wickham. Ce qui m'inquiétoit, c'étoit de savoir s'il mettroit son habit bleu pour se marier. Enfin nous déjeunâmes à dix heures comme à l'ordinaire; et je trouvai le temps horriblement long; car, par parenthèse, mon oncle et ma tante ont été odieux pendant que j'ai été chez eux: il n'est pas possible d'être plus désagréable. Imaginez que pendant quinze jours que j'ai été dans leur maison, ils ne m'ont pas laissé mettre le pied à la rue. Il n'y avoit pas grand monde à Londres, à la bonne heure;\setcounter{page}{126} mais enfin il y a toujours les spectacles. Quand la voiture arriva, Mr. Stones fit demander mon oncle. Ce malheureux Stones est l'homme le plus ennuyeux de la terre ; et quand il tenoit mon oncle, il en avoit toujours pour deux heures. J'étois dans l'huile bouillante, parce que je savois que si nous manquions l'heure, c'étoit fini, nous ne pouvions pas être mariés ce jour-là ; mais heureusement, mon oncle rentra à temps, et nous montames en voiture. Au reste, j'avois tort de m'inquiéter ; parce que s'il n'étoit pas venu, Mr. Darcy l'auroit remplacé.
"Mr. Darcy !" interrompit Elisabeth.
"Ah ! que je suis bête, moi ! Je me souviens à présent que j'avois promis de ne pas dire qu'il étoit là."
"Eh bien, Lydie, si vous l'avez promis," dit sa sœur aînée, "il ne faut pas dire un mot de plus."
"Non, certainement !" ajouta Elisabeth, qui pourtant mouroit de curiosité d'en savoir davantage.
"Oh ! je vous aurois tout dit, sans conséquence, si vous aviez voulu le savoir."
Dans la crainte d'être entraînée par la curiosité, Elisabeth prit le parti de sortir précipitamment. Ce qui venoit d'échapper à sa sœur lui donna beaucoup à penser. Mr.\setcounter{page}{127} Darcy avait assisté au mariage; et cependant il devait avoir une grande répugnance à se trouver là. Les suppositions les plus flatteuses, les plus honorables pour son caractère, se présentoient à l'esprit d'Elisabeth; et pour mettre fin à l'incertitude dont elle étoit tourmentée, elle écrivit à sa tante, et lui demanda l'explication du mot de Lydie.
(La suite au Cahier prochain.)

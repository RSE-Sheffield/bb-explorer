\setcounter{page}{394}
\chapter{Melanges}
\section{NOTICE sur l'inoculation de la VACCINE; par le Prof. ODIER.}
Nos inoculations de vaccine se sont singulièrement multipliées dans le courant de ce mois. L'épidémie de petite-vérole qui règne ici depuis le printemps, et qui est très-meurtrière, a engagé les pères et mères à avoir recours à l'inoculation pour en garantir leurs enfants. Quant au choix du virus, les uns ont préféré s'en tenir à l'ancienne méthode, et employer le virus variolique. Le plus grand nombre s'est déterminé pour la vaccine, surtout depuis que l'expérience a démontré ses avantages. Car en général, on ne peut disconvenir que ce ne soit une maladie incomparablement plus bénigne. Sur près de 250\setcounter{page}{395} personnes auxquelles on l'a inoculée à Genève depuis deux mois; il n'y en a pas eu une seule qui ait eu des convulsions, ni aucun autre symptôme le moins du monde alarmant. La maladie s'est très-uniformément manifestée dans tous les malades, à-peu-près telle que je l'ai décrite dans le dernier Numéro de ce Journal, page 286. Dans trois de nos inoculés, un érysipèle assez étendu s'est manifesté sur tout le bras. Le Dr. Woodville avait vu cet accident, qui ne paroît au reste d'aucune conséquence, survenir cinq fois sur 200. (Voyez Bibl. Brit. vol. 12. Sc. et A. p. 284.) Dans l'un de nos malades, cet érysipèle s'est non-seulement étendu sur la totalité du bras et de l'avant-bras, mais encore il a gagné le col et le visage, au point de fermer l'œil et de produire assez de fièvre.
Quelques-uns de nos inoculés ont eu les taches rouges dont parle le Dr. Pearson; mais il n'a pas paru qu'elles fussent dures et élevées. J'en ai vu qui avoient jusqu'à deux pouces de longueur sur un pouce de largeur. Mais ces taches ont toujours été fugitives et sans aucune conséquence. Plusieurs enfants ont été inoculés d'après ceux qui avoient de pareilles taches. Elles ne se sont point transmises, et les effets de la vaccine n'ont point été différens dans ces enfants, de ceux qu'elle a produits dans les autres.
\setcounter{page}{396}
Quant aux boutons, quelques-uns de nos inoculés en ont eu, et même assez abondamment. Mais il a paru évidemment qu’ils tenoient à l’épidémie de petite-vérole. Car ils ressembloient parfaitement aux boutons varioliques, et leur cours étoit le même. Ils sont survenus deux fois, dès le 3e ou 4e jour à dater de l’inoculation, et avant qu’il y eut à peine aucun signe d’infection locale. Ils n’ont point empêché dans ces cas-là le développement de la vaccine ; mais il n’y a eu aucune aréole érysipélateuse autour de l’incision, et la petite-vérole a eu d’ailleurs toute l’apparence et tous les caractères de la petite vérole naturelle. L’un de ces enfans en est mort. C’est celui dont nous avons parlé, vol. 14, pag. 199. L’autre s’est guéri. Deux ou trois autres inoculés ont eu aussi des boutons, mais seulement après la formation de l’aréole érysipélateuse autour de l’incision vaccinique. Dans ceux-ci les boutons ont eu la même apparence que dans la petite-vérole inoculée. Il a donc paru clairement que si l’on inocule la vaccine à un enfant qui ait déjà le germe de la petite-vérole, c’est un accident que l’on n’a aucun moyen de prévoir, et qui doit naturellement se présenter quelquefois dans le cours d’une épidémie forte; et si celle-ci se manifeste avant celle-là, la\setcounter{page}{397} vaccine ne la modifie pas; elle n'en est ni plus heureuse, ni moins abondante; elle de meure ce qu'elle auroit été indépendamment de la vaccine; mais si la petite-vérole ne se manifeste qu'après le développement de la vaccine, surtout après l'affection générale qu'elle produit, cette petite-vérole est modifiée par la vaccine, comme elle l'auroit été par l'inoculation; elle est incomparablement plus bénigne que la petite-vérole naturelle, et les boutons se sèchent plus promptement.
Enfin, quant à la crainte de prendre la petite-vérole après avoir eu la vaccine, cette crainte paroît certainement mal fondée; et nous croyons pouvoir affirmer, dès à-présent, par notre propre expérience, que la vaccine garantit très-sûrement de la petite-vérole. Car on a déjà inoculé celle-ci sans aucun succès à trois enfans qui avoient eu la vaccine un mois ou six semaines auparavant; et quoi que cette inoculation variolique ait été faite de bras à bras et avec tout le soin possible, elle n'a eu absolument aucun effet. De plus, il est moralement impossible que sur un aussi grand nombre d'inoculés vaccins, ils eussent tous échappé à la contagion, dans le cours d'une épidémie aussi générale que celle qui règne actuellement dans nos murs, s'ils en\setcounter{page}{398} étoient encore susceptibles. On ne peut douter qu'ils ne doivent cet avantage à la vaccine même.
Genève 9 Fructidor an VIII.
ODIER, Prof.

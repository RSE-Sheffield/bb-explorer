\setcounter{page}{199}
\chapter{Melanges}
\section{AVIS SUR LA VACCINE.}
J'ai reçu des Drs. Pearson et Jenner des fils imprégnés d'un virus vaccin, bien choisi à Londres, et dans le Gloucester-shire. Celui que m'a envoyé le Dr. P. étoit renfermé sans autre précaution dans une lettre qui est venue par Hambourg, et est restée un mois en route. Malgré cela il a fort bien réussi. Mes confrères, les Drs. Dunant et Butini en ont inoculé 3 enfans. Le résultat de l'inoculation a été parfaitement conforme à ce que nous avoient annoncé les Anglais. Les signes d'infection locale ne se sont manifestés qu'au bout de 3 jours; il n'y a eu de la fièvre qu'au 8me. jour. Malheureusement un de ces enfans s'est trouvé avoir le germe de la petite-vérole naturelle au moment où il a été inoculé. La fièvre variolique s'est déclarée avec des symptômes alarmans au 5me. jour de l'inoculation de la vaccine. Il s'est manifesté une éruption confluente et pétéchiale d'une mauvaise nature et l'enfant est mort. La dessication avoit commencé plus promptement qu'à l'ordinaire et sans orage. La petite-vérole n'a pas empêché les progrès de la pustule vaccine. Mais elle n'a point été entourée de l'aréole erysipelateuse qui a été grande et bien marquée dans les autres enfans.
\setcounter{page}{200}
J'ai inoculé hier de bras à bras, d'après l'un de ces derniers, un autre enfant de 22 mois, qui avoit déjà été inoculé ci-devant avec la fausse vaccine; et que des circonstances domestiques avoient empêché de soumettre à l'épreuve de la petite-vérole; mes confrères en ont inoculé d'autres. Je ne doute pas que ces inoculations ne réussissent fort bien.
Nous reviendrons sur ces détails. Nous parlerons aussi de nouvelles pièces intéressantes que nous avons reçues de Vienne et de Londres sur la vaccine. Le nombre des inoculés est déjà en Angleterre de plus de 6000 personnes de tout âge. Le succès de cette pratique y a été aussi satisfaisant qu'on pouvoit le desirer. Il n'est mort qu'UN SEUL de ces inoculés. C'est celui dont nous avons parlé en rendant compte de l'ouvrage du Dr. Woodville.
Aucun des autres n'a eu les suites qu'on voit fréquemment après la petite-vérole naturelle, et quelquefois après la petite-vérole inoculée. Tout annonce, que, comme le dit le Dr. Pearson, l'introduction de la vaccine inoculée, fera époque dans l'Histoire de la Médecine. (O)
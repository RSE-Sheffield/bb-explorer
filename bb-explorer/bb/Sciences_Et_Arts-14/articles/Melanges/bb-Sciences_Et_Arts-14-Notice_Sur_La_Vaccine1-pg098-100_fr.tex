\setcounter{page}{98}
\chapter{Melanges}
\section{AVIS DU PROF. ODIER SUR LA VACCINE.}
Nous avons continué à inoculer la petite vérole aux enfans auxquels nous avions précédemment inoculé la vaccine prise sur les bras du Comte M. Le virus variolique a eu chez tous son effet. Le Dr. Allamand de Colombiers près de Neuchâtel, à qui nous avions envoyé des fils imprégnés du même\setcounter{page}{99} virus vaccin que les nôtres, s'en est servi comme nous, pour inoculer successivement une vingtaine d'enfants. Ils ont tous eu une maladie bien caractérisée et semblable à celle que nous avions observée ici.
Mais cette maladie s'est trouvée comme chez nous insuffisante pour les garantir de la petite-vérole. Six l'ont eue par inoculation. Trois l'ont prise naturellement, et l'on ne peut raisonnablement se flatter qu'aucun des autres, qui n'ont pas encore été exposés à la petite vérole, ait été préservé par-là de la possibilité de la prendre.
Or, comme dans toutes ces inoculations de vaccine, le virus inséré dérivoit de la même source, il paroit bien prouvé par l'identité du résultat sur ces 40 enfants, que c'est à cette source; c'est-à-dire, dans les circonstances qui ont accompagné l'inoculation du Comte et spécialement dans la petite-vérole qu'il avoit eue dans son enfance qu'il faut chercher la cause qui a fait dégénérer la vaccine sur son corps, au point de lui ôter la propriété de préserver de la petite-vérole, sans lui ôter celle de se multiplier successivement et\setcounter{page}{100} de se transmettre d'un individu à l'autre avec les mêmes caracteres de bâtardise. On peut donc légitimement conclure de nos observations et de celles du Dr. Allamand, comparées à celles des Anglais, que la vaccine ne garantit sûrement de la petite-vérole, que lorsque dans le nombre des individus au travers desquels on l'a transmise, il ne s'en trouve aucun qui ait déjà eu cette maladie. Il faudra voir si lorsque nous aurons reçu de Londres ou de Vienne, des fils dont nous puissions être bien assurés sous ce rapport, et qui n'aient pas d'ailleurs perdu leur efficacité, l'expérience confirmera cette conjecture.

\setcounter{page}{254}
\chapter{MÉDECINE}
\section{OBSERVATIONS CONCERNING THE ERUPTIONS, etc. Observations sur les éruptions semblables à celle de la Petite-vérole, qui surviennent quelquefois dans la Vaccine inoculée; par George PEARSON, Médecin de l'Hôpital de St. George, Membre de la Soc. Roy. etc. Londres 1800.}
Le danger de la petite-vérole est presque toujours proportionné à l'abondance des boutons; et c'est surtout à cet égard que l'inoculation de cette maladie en diminue tellement la mortalité, que tandis que celle de la petite-vérole naturelle est d'un sur dix, celle de la petite-vérole inoculée est tout-au-plus d'un sur deux cent. Il faut convenir cependant, que même dans la petite-vérole inoculée, le nombre des boutons est quelquefois\setcounter{page}{255} assez considérable pour laisser des marques, pour donner de grandes inquiétudes pendant toute la durée de la maladie, et pour la rendre plus ou moins dangereuse.
Or, le principal avantage qu'on attribuait à la vaccine inoculée était de garantir toujours de la petite vérole aussi efficacement qu'on le ferait par l'inoculation du virus variolique, mais sans produire aucune éruption. Si cet avantage était bien constaté, il serait sans doute immense; car, indépendamment de ce que cette circonstance diminuerait infiniment le danger de la maladie en elle-même, elle conduirait probablement, tôt ou tard à la destruction totale de la petite-vérole en Europe, destruction dont on ne peut se flatter par l'inoculation de la petite-vérole même; parce qu'elle entretient, perpétue et multiplie toujours jusqu'à un certain point les foyers de contagion; beaucoup moins à la vérité que la petite-vérole naturelle, mais incomparablement plus qu'une maladie qui ne pourrait jamais affecter que l'individu auquel on la donne.
C'est donc un point important à bien déterminer dans l'histoire de la vaccine, que cette différence entre elle, et la petite-vérole. En rendant compte de l'ouvrage du Dr. Woodville, nous avons vu que sur 510 inoculés\setcounter{page}{256} vaccins, il s'en étoit trouvé 294 qui avoient eu des boutons, et quelques-uns en très-grand nombre. Mais nous avons vu aussi qu'il y avoit lieu de croire que ces boutons avoient été produits soit par le virus de la petite-vérole qui leur avoit été inoculée presque en même temps que la vaccine; soit par les émanations varioliques auxquelles les malades avoient été exposés pendant l'inoculation. Cette conjecture paroissoit d'autant mieux fondée qu'un très-grand nombre de praticiens affirmoient au contraire, avoir inoculé la vaccine à plusieurs centaines d'individus sans l'avoir jamais vu produire des boutons. Mais le Dr. Woodville insistoit, et citoit en preuve de son opinion, des inoculations faites à la campagne, loin de Londres, et avec toutes les précautions nécessaires pour soustraire les inoculés vaccins à l'influence de toute émanation variolique; inoculations dans lesquelles il y avoit cependant eu une éruption plus ou moins abondante. La vaccine peut-elle donc produire des boutons dans d'autres places du corps qu'à l'endroit de l'insertion?
C'est sur cette question que roule un petit Mémoire que le Dr. Pearson nous a envoyé, et qu'il a inséré dans le Magazin Philosophique de Londres pour le mois de Janvier 1800. En voici l'extrait.
\setcounter{page}{257}
"Quoique les avantages de l'inoculation de la vaccine aient été démontrés dans le cours de cette année (1799), par un assez grand nombre d'observations, pour que nous n'ayions plus à craindre de voir tomber cette nouvelle pratique; cependant, l'apparence inattendue d'une éruption dans un grand nombre de cas a un peu ébranlé l'opinion de quelques personnes, qui dès lors ont cru ces avantages jusqu'à un certain point problématiques. Les partisans zélés de la nouvelle inoculation n'ont pas craint d'affirmer en réponse, que toutes les fois que dans le cours d'une vaccine inoculée il survient des boutons ailleurs qu'à l'endroit de l'insertion, on doit les attribuer, ou à ce que par inadvertence l'on a inoculé la petite-vérole au lieu de la vaccine, ou à ce que le malade avoit fortuitement pris la petite-vérole naturelle par contagion, avant que d'être inoculé. Comme cette assertion est loin d'être prouvée, et que j'ai déjà acquis sur la vaccine une grande expérience, je crois de mon devoir de publier le résultat de mes observations à cet égard."
"Dans le cours de ma pratique, à la fin de Février et au commencement de Mars', j'ai vu quatre inoculés vaccins avoir des boutons qui me parurent au premier coup-d'œil ressembler si parfaitement à ceux de la petite-vérole,\setcounter{page}{258} que je les aurois pris pour de véritables boutons varioliques, si je n'avois été bien sûr de la qualité du virus que j'avois employé. Je remarquai cependant alors quelques différences entre ces boutons et ceux de la petite-vérole. Ceux-là avortèrent presque tous sans suppurer, et se terminent tous par des croûtes lisses, luisantes, et d'un brun rouge ou noirâtre. J'inoculai deux autres malades avec le virus pris sur le bras d'un de ces quatre premiers. Ils eurent aussi des boutons semblables, ainsi que toutes les personnes inoculées avec ce même virus par deux ou trois de mes correspondans à qui j'en avois envoyé. Cela me détermina à n'employer à l'avenir que du virus pris sur le bras d'inoculés qui n'eussent point eu de boutons. Dès lors je n'en ai point vu qui eussent aucun rapport avec ceux de la petite-vérole. Mais ce que j'ai très-certainement vu assez fréquemment, peut-être une fois sur 20 ou 30, c'est une éruption de gros boutons rouges et durs, mais peu élevés, et qui ne contenoient ni pus, ni sérosités. Ces boutons, ou pour mieux dire, ces taches, n'étoient accompagnées d'aucune incommodité, et ne duroient que fort peu de temps. J'ai vu encore une rougeur générale et semblable à celle de la scarlatine survenir quelquefois au quatorzième\setcounter{page}{259} jour de l'inoculation, mais tout-à-fait fugitive et sans conséquence."
"Si donc je n'en jugeois que d'après ma propre expérience, je pourrois être porté à croire que puisqu'en évitant d'inoculer avec le virus d'une personne ayant des boutons semblables à ceux de la petite-vérole, je n'en ai plus apperçu, (observation qui coïncide avec celles du Dr. Woodville) le virus variolique a pu, de manière ou d'autre, et malgré toutes mes précautions, s'insinuer au lieu du virus vaccin dans le corps de ceux de mes premiers inoculés qui en ont eu. Mais le nombre des cas dans lesquels d'autres inoculateurs ont vu des boutons parfaitement semblables à la suite de l'inoculation de la vaccine bien choisie, a été trop considérable cet autonue pour me permettre de révoquer en doute la possibilité d'une pareille éruption produite par le virus vaccin seul.
"Au mois d'octobre dernier j'inoculai la vaccine à un enfant de deux ans. J'avois, moi-même pris originairement le virus sur une vache au mois de Mars; mais il avoit dès lors passé successivement au travers du corps de plusieurs malades. Celui-ci eut l'affection locale ordinaire. Elle fut suivie d'une légère fièvre, et deux ou trois jours après d'une éruption peu abondante de taches rouges\setcounter{page}{260} et dures, très-différentes des boutons de petite-vérole. Mr. Keate prit du virus sur le bras de cet enfant, et le porta à Brighelstone, où Mr. Barrett s'en servit pour inoculer deux enfans, d'après lesquels Mr. Keate en inocula ensuite trois autres; et enfin, Mr. André de Petworth, à qui l'on avoit envoyé du virus de ceux-ci en inocula 14 individus. Ces 19 inoculés eurent tous des boutons semblables à ceux de la petite-vérole, et quelques-uns en très-grand nombre. Il ne paroît pas cependant, qu'aucun d'eux ait été le moins du monde en danger. L'affection locale dans un de ceux de Mr. Keate ressembloit à celle de la vaccine. Dans ceux de Mr. Barrett, à celle de la petite-vérole. Dans ceux de Mr. André il n'en est pas question. Mais, ce qui paroît encore plus positif, c'est une observation que m'a envoyée en dernier lieu le Dr. Thornton. Il avoit inoculé un enfant avec du virus que j'avois pris moi-même sur une vache. L'enfant eut une éruption variolique.
"Il paroît donc bien prouvé; 1°. que dans certaines circonstances, ou plutôt à l'aide de certains co-agents simultanées, dont la nature ne nous est pas bien connue, le virus vaccin produit une maladie semblable à celle de la petite-vérole, soit par l'apparence du bouton à l'endroit de l'insertion, soit par l'éruption\setcounter{page}{261} subséquente. 2. Que dans certains cas l'apparence du bouton à l'endroit de l'insertion a été parfaitement semblable à celle de la vaccine, quoique l'éruption subséquente ressemblât à la petite-vérole. 3. Que le virus pris sur des individus dans lesquels la vaccine a développé cette maladie variolique, soit qu'il ait été pris sur le bras à l'endroit de l'insertion, ou dans d'autres parties du corps, produit universellement, ou du moins très-généralement une éruption semblable, sans jamais avoir été vu, que je sache, rétrograder, en passant successivement au travers du corps de plusieurs individus et revenir à son état primitif de vraie vaccine. 4. Qu'enfin, la vraie vaccine produit quelquefois d'autres éruptions particulières, très-différentes de celle de la petite-vérole."
"Le virus vaccin est-il donc susceptible de se changer en se décomposant ou en se combinant avec d'autres agens, en vrai virus variolique? C'est ce que nous ignorons. Mais, il n'en est pas moins incontestable que les apparences produites par l'un de ces deux virus étant très-différentes de celles que produit l'autre, ce sont deux virus très-distincts et spécifiquement dissemblables, quoique le premier soit susceptible d'acquérir tous les caractères du second. C'est ainsi que quoique\setcounter{page}{262} la magnésie soit susceptible de se changer en sulfate de magnésie par la seule addition de l'acide sulfurique, il ne viendra dans l'esprit de personne de confondre ces deux substances l'une avec l'autre. Il doit en être de même du virus vaccin et du virus variolique. Ils ne diffèrent peut-être l'un de l'autre que par l'addition d'une substance jusqu'à présent inconnue; mais c'en est assez pour en faire deux agens très-distincts qu'on ne doit pas confondre sous le même nom. Aussi le Dr Odier a-t-il eu bien raison de rejeter l'absurde nom de petite-vérole des vaches que nous donnions à la maladie produite par le premier, pour l'appeler la vaccine."
"Mais puisque cette vaccine peut, en certaines circonstances, se changer en petite-vérole, a-t-elle encore quelques droits à être préférée à celle-ci pour l'inoculation? Il faut en convenir; cette circonstance affoiblit un peu l'idée qu'on s'étoit d'abord formée de ses avantages; mais il lui en reste cependant encore assez pour lui mériter hautement la préférence sur la petite-vérole. Car 1. en évitant soigneusement d'inoculer avec du virus pris sur des individus qui aient des boutons, il n'arrivera pas à plus d'un inoculé vaccin sur 200 d'avoir une éruption semblable à celle de la petite-vérole. 2. Même dans ces cas\setcounter{page}{263} là, l'éruption ne sera pas plus facheuse que celle de la petite vérole inoculée."
"On a donc beaucoup à gagner et rien à perdre en substituant la vaccine à la petite vérole dans l'art de l'inoculation; et le nombre des observations favorables à cette nouvelle pratique s'est aujourd'hui tellement multiplié qu'il y a lieu de croire qu'elle fera une époque mémorable dans l'histoire de la médecine."
C'est par cette prophétie que le Dr. P. termine son Mémoire; et l'événement semble déjà la justifier jusqu'à un certain point, puisque de toutes parts, en Europe, on songe à faire des établissemens pour répandre cette nouvelle manière d'inoculer. Tous nos lecteurs connoissent celui de Paris; et le Moniteur a rendu compte de ses premiers succès.
Voici quels sont les motifs et les bases de celui qu'on a fondé à Londres. C'est du Dr. Pearson lui-même que nous en tenons le Prospectus.
Il mérite d'être connu; ne fut-ce qu'à cause du préambule, qui est une excellente récapitulation de tous les avantages que la vaccine a sur la petite-vérole, pour l'inoculation\footnote{Je me permettrai pourtant d'en faire un peu la critique, parce que j'y trouve quelques exagérations et quelques inexactitudes. D'ailleurs je l'avouerai. Tous ces établissemens me paroissent beaucoup trop fastueux pour un objet aussi simple que l'inoculation de malades qu'on va visiter chez eux et dont la maladie doit être si légère qu'elle n'exige aucun remède. Les Médecins et les Chirurgiens que l'expérience a convaincus de ses avantages ne peuvent-ils pas l'inoculer gratuitement sans tout cet étalage qui a toujours quelque chose d'avilissant pour la classe indigente. A Genève, nous nous sommes fort bien passés de tout cela. Depuis près de 50 ans l'inoculation y est peut-être plus répandue, plus généralement adoptée qu'en Angleterre, et que partout ailleurs en Europe. Or jamais nous n'avons eu d'établissement public d'inoculation.}.
\setcounter{page}{264}
Plan de l'Etablissement fondé à Londres le 2 décembre 1799, pour l'inoculation de la vaccine, au No. 36, Warwick-Street, (Golden square).
\subsection{Avant-propos}
Ceux qui ne connoissent qu'en partie l'histoire de la petite-vérole, sont naturellement portés à croire que la petite-vérole inoculée' étant incomparablement plus bénigne que la naturelle, et garantissant de la mort un très-grand nombre d'individus, il est difficile d'aller plus loin pour diminuer le danger de cette maladie, et inutile de tenter. Mais ceux qui\setcounter{page}{265} sont plus profondément versés dans ce sujet savent très-bien que, malgré les avantages immenses de l'inoculation, la petite-vérole fait encore beaucoup de mal au genre humain. Car
1. Quelque bien dirigé que soit le traitement de la petite-vérole inoculée, elle n'est pas exempte de tout danger; et quoique le nombre de ceux qui en meurent ne s'élève probablement pas à plus de 5 sur 1000 \footnote{A Genève, je doute qu'il soit mort plus de 5 inoculés sur 1000. C'est à peu près la proportion que donnent aussi les registres des hôpitaux d'inoculation, soit en Angleterre, soit en Russie; en calculant pendant un grand nombre d'années de suite.}, ces accidens sont incomparablement plus affreux pour les parens que si la mort avoit été le résultat d'une maladie accidentelle. Quelque bénigne que soit la petite-vérole inoculée en général, il y auroit donc beaucoup à gagner à pouvoir lui substituer une maladie beaucoup plus légère et bien moins dangereuse encore.
2. On peut, sans exagération, affirmer que s'il ne meurt que 5 inoculés sur 1000, il y en a au moins 40 pour lesquels la petite-vérole, quoiqu'inoculée, est une véritable maladie;\setcounter{page}{266} un état pénible, douloureux et jusqu'à un certain point alarmant \footnote{Quand on inocule dans le cours d'une épidémie mauvaise, le nombre des inoculés qui ont beaucoup de boutons, et sont indisposés d'une manière assez grave, m'a paru beaucoup plus considérable que ne le suppose ici le Dr. P. mais il l'est peut-être un peu moins dans les intervalles des épidémies. Dans les grandes villes, où la petite-vérole est pour ainsi dire en permanence, je suis loin de croire la supposition de notre auteur exagérée. (O)}.
3. Les nombreux foyers de contagion que laisse la petite-vérole, après elle, ne permettent pas d'espérer qu'elle puisse être universellement détruite; et à moins que l'inoculation ne devienne beaucoup plus générale qu'elle ne l'est actuellement, il y a lieu de croire qu'en disséminant davantage la contagion, elle contribue plutôt à augmenter la mortalité qu'à la diminuer \footnote{C'est ce que je ne crois point. J'ai amplement démontré le contraire. (Voyez le IXᵐᵉ. vol. de la Bib. Brit. Sc. et Arts p. 262.) L'inoculation diminue tellement les foyers de contagion, que quoique la petite-vérole soit contagieuse, elle ne sauroit augmenter la mortalité générale. Mais il n'en est pas moins vrai qu'un des grands avantages de la vaccine sur la petite-vérole est de ne se communiquer que par artifice, ou en cas de coupure, écorchure, ou autre dénudation de la peau. (O)}.
4. L'inoculateur le plus habile et le plus\setcounter{page}{267} heureux ne peut jamais répondre que ses inoculés soient tous à l'abri des marques, des cicatrices, des difformités que la petite vérole laisse si souvent après elle; ou complettement garantis des maladies constitutionnelles qu'elle réveille fréquemment à sa suite.
5. Il y a certaines familles, certains tempéramens, certaines positions, telles par exemple, que la grossesse, dans lesquelles la petite-vérole, même inoculée, est presque toujours une maladie très-dangereuse.
Or, toutes les observations qu'on a recueillies dans le courant de cette année, sur la vaccine, et particulièrement les nombreuses expériences qu'on a faites pour bien déterminer les effets de cette maladie, communiquée par inoculation, ont démontré clairement qu'on peut obvier à tous ces hasards de la petite-vérole inoculée, en inoculant la vaccine à sa place. Car,
1. Sur plus de 4000 personnes auxquelles on a inoculé la vaccine, il n'en est mort qu'une; et il y a tout lieu de croire que la mortalité de cette maladie sera à l'avenir beaucoup plus foible encore.
2. Quand on a eu la vaccine, soit naturelle, soit inoculée, il n'y a pas un seul exemple avéré qu'on ait été susceptible ensuite de prendre la petite - vérole. C'est une vérité qui\setcounter{page}{268} s'est transmise par tradition depuis un temps immémorial dans les pays où la vaccine naturelle est connue \footnote{Ce n'est pas seulement dans le Glocestershire que subsiste cette tradition. Voici ce que le Dr. De Carro écrivoit il y a quelque temps de Vienne au Dr. Peschier. "Je vous ai parlé dans ma précédente de la connoissance que les Médecins du Duché de Holstein ont de la vaccine; j'ai eu dernièrement l'occasion de vérifier jusqu'à un certain point cette nouvelle de MM. Ballhorn et Strohmeyer, par le témoignage d'un domestique Allemand, homme qui m'a paru fort intelligent, appartenant à un gentilhomme Américain, et qui a vécu 3 ans dans les environs de Kiel, dans le Duché de Holstein. Il m'a dit, qu'il y avoit très-souvent entendu parler d'une maladie des vaches, appelée dans le pays finnen (fî nne signifie bouton) et qu'il en avoit souvent vu les vaches attaquées. Que sa propriété de préserver de la petite-vérole étoit connue des paysans et des Médecins du Holstein. Que dans la ville même de Kiel, on inoculoit quelquefois des enfans, dans l'intention de leur conserver la beauté. Qu'il a souvent entendu parler de cette maladie, en servant à table à plusieurs Messieurs de ce pays, parmi lesquels se trouvoit le Dr. Ackerman. Il décrit la maladie des vaches comme un bouton entre chair et cuir ; et dit que pendant que la vache a cette maladie, elle perd son lait et maigrit beaucoup. Que le bouton produit par l'inoculation n'est jamais accompagné d'aucune éruption sur le reste du corps, et qu'il est de la grosseur d'un pois."
"Il faut observer que le domestique ne savoit pas un mot de la découverte de Jenner, et a vécu en Holstein long-temps avant qu'elle fût connue sur le Continent."}. Des 4000 inoculés vaccins dont nous venons de parler, on a inoculé la petite-vérole à plus de 2000; la plupart ont été plusieurs fois exposés depuis aux émanations varioliques, sans qu'aucun d'eux en ait jamais été attaqué.
3. On peut affirmer que, généralement parlant, la vaccine inoculée est une maladie beaucoup plus légère et plus bénigne que la petite-vérole inoculée; tellement que pour 10 inoculés de la petite-vérole qui en sont\setcounter{page}{269} 4. Il ne paraît pas que la vraie vaccine puisse se communiquer, comme la petite-vérole, par les émanations des malades ; ensorte qu'il y a lieu de croire que si jamais on l'inocule généralement, au lieu de la petite-vérole, celle-ci disparaîtra finalement de la Grande-Bretagne, comme en ont disparu la peste, la suette, et certaines espèces de lèpre, qui n'y sont plus connues que de nom.
5. Il ne paraît pas non plus que le virus vaccin puisse comme celui de la petite-vérole, transmettre indirectement la maladie par l'attouchement\setcounter{page}{270} des habits, du linge et des meubles qui ont servi aux malades; ensorte qu'on ne court point le danger de la propager de cette manière en l'inoculant généralement.
6. Il a été démontré que quand la constitution a été une fois manifestement atteinte par la vaccine, on n'est plus susceptible à l'avenir de la prendre; ensorte qu'on ne doit plus appréhender, comme on le craignoit il y a quelque temps, de substituer à la petite-vérole une nouvelle maladie éruptive, à laquelle on pourroit être sujet plusieurs fois dans la vie \footnote{Le Dr. Pearson fait allusion dans ce paragraphe et dans le suivant aux expériences négatives par lesquelles il a refuté l'opinion du Dr. Jenner, qui soutenoit que quoiqu'il ne fût pas bien ordinaire de prendre la vaccine plusieurs fois, ou après avoir eu la petite-vérole, cela n'étoit pas impossible, puisqu'il en avoit vu plusieurs exemples. Mais il me semble que jusqu'à ce que le Dr. P. ait bien clairement expliqué d'où est venue l'illusion dans les cas que cite le Dr. J.; des expériences négatives ne peuvent anéantir des observations positives; d'autant plus que le Dr. Woodville en a cité d'analogues, tirées de sa propre pratique. (Voy. la Bibl. Brit. vol. XII. Sc. et Arts, p. 161.) Il n'en est pas moins vrai que la crainte dont on parle ici seroit bien chimérique, puisque la vaccine n'est pas contagieuse.}.
7. Il a été de même démontré qu'on n'en\setcounter{page}{271} est pas susceptible lorsqu'on a eu la petite vérole; en sorte que les personnes qui ont déjà eu celle-ci, soit naturellement, soit par inoculation, n'ont rien à craindre de l'introduction de la vaccine, comme elles pouvoient l'appréhender il y a quelque temps\footnote{L'exemple du Comte M. de Vienne, et celui des 40 personnes qui ont été successivement inoculées d'après lui, tant à Genève qu'à Colombier, montre au moins, que si la petite-vérole garantit efficacement de la vraie vaccine, (ce dont on peut encore douter d'après les observations de W.; malgré les expériences négatives de notre auteur) elle ne garantit pas d'une vaccine bâtarde, susceptible de se propager par inoculation, et qui quoique très-rapide n'en a pas moins les caractères d'une indisposition générale. Mais encore une fois qu'importe, si cette maladie n'est pas contagieuse? Ceux qui ont eu la petite-vérole n'ont pas besoin de se faire inoculer la vaccine. (O)}.
8. L'expérience a démontré qu'on ne court aucune chance de difformité par l'inoculation de la vaccine.
9. Il n'a pas paru dans les nombreuses observations faites jusqu'à présent que la vaccine, soit naturelle, soit inoculée, ait jamais excité après elle aucune autre maladie qui pût à juste titre être regardée comme en étant la suite.
En voilà assez pour démontrer que c'est\setcounter{page}{272} ici un objet très-important d'intérêt public ; que tous les habitants de la Grande-Bretagne doivent s'empresser d'adopter la substitution de la vaccine à la petite-vérole dans l'inoculation ; et que l'établissement actuel, qui a pour but de faire jouir de cette heureuse découverte les individus les plus indigens, mérite hautement la bienveillance et la protection de toutes les personnes bienfaisantes. Il n'est aucune institution de charité par laquelle on puisse espérer de faire autant de bien, à aussi peu de frais.
Ajoutons, que cet établissement est peut-être le meilleur moyen de faire connoître les avantages de la vaccine inoculée à ceux qui n'en ont pas encore entendu parler ; de décider toutes les questions qui y sont relatives, et qui paroissent encore douteuses ; de découvrir toutes les nouvelles sources d'erreur qu'on n'a point encore apperçues, puisque tous les cas seront enregistrés, toutes les expériences, faites sous la direction des médecins attachés à l'établissement ; et tous les résultats, communiqués aux souscripteurs.
Comme on s'addresse souvent à l'établissement pour avoir du virus vaccin, le public est averti qu'on ne peut compter que sur celui qui portera le sceau de l'établissement, c'est-à-dire, l'empreinte d'une vache, avec cette\setcounter{page}{273} devise. Feliciores inserit. On y trouvera aussi des lancettes bien imprégnées deux desquelles coûteront une demi-guinée.
\subsection{PLAN DE L'ÉTABLISSEMENT.}
ART. I. Tous les mardis et les jeudis, un Médecin et un Chirurgien se trouveront à une heure après midi à l'Etablissement, pour y examiner, inoculer, et soigner les malades qui s'y rendront aux jours et heures qui leur seront indiqués.
2. Un Apothicaire s'y trouvera aussi pour exécuter les ordonnances et s'acquitter des autres fonctions qui le concernent.
3. On admettra tous les individus porteurs d'une lettre de recommandation de l'un des Directeurs.
4. Les remèdes nécessaires seront aux frais de l'Etablissement ; et quand cela sera jugé convenable les Officiers de Santé qui y sont attachés visiteront les malades chez eux.
5. Tous ceux qui payeront une guinée de souscription annuelle, ou dix guinées en un seul paiement, auront droit à avoir toujours deux malades à la fois, successivement admis et enregistrés sous leur nom. Ceux qui souscriront pour de plus fortes sommes pourront en faire admettre un plus grand nombre, dans la même proportion.
\setcounter{page}{274}
6. Les souscrivans seront les Directeurs de l'Etablissement. Ils feront entr'eux tous les Réglemens qu'ils jugeront convenables pour sa réussite.
7. Les souscriptions seront employées à couvrir les frais de l'Etablissement.
8. L'Etablissement aura un Président, six Vice-Présidens et un Trésorier, outre les Officiers de Santé, dont le service sera gratuit et qui seront pris parmis les Directeurs.
9. Ces Officiers de Santé seront deux médecins, deux Chirurgiens consultans, deux Chirurgiens ordinaires, et trois Apothicaires, lesquels seront appelés à visiter les malades chez eux \footnote{C'est l'usage en Angleterre. Les apothicaires visitent et soignent la grande majorité des malades. Les Médecins ne sont guères appelés que dans les cas graves. Aussi les paie-t-on en conséquence. Il n'y a que les grands seigneurs, les gens très-riches et les malades des hôpitaux ou des dispensaires qui puissent avoir un Médecin en titre dès le commencement et jusqu'à la fin de leurs maladies.}.
Il y aura à demeure dans la maison destinée à l'Etablissement, un Apothicaire pour préparer et distribuer les remèdes; un Receveur, et un Portier, outre les autres Officiers subalternes qui seront trouvés nécessaires.
\setcounter{page}{275}
Suit la liste nominative des Officiers de l'Établissement, et la liste des banquiers chez qui l'on souscrit.
\subsection{Forme du Registre des malades.}
Ce Registre doit avoir 16 colonnes destinées à enregistrer.
1. Le nom et l'âge de chaque malade, avec la date de son admission.
2. Le genre des alimens et des boissons qu'on lui prescrit.
3. L'origine du virus vaccin servant à son inoculation.
4. Le nombre et l'espèce des piqûres ou incisions, qu'on lui fait en l'inoculant.
5. L'état de santé avant l'inoculation.
6. L'état des bras jusqu'au 4e. jour.
7. L'état de la santé pendant ces 4 premiers jours.
8. L'état des bras depuis le 5e. jusqu'au 8e. jour.
9. L'état de la santé pendant ces 4 jours.
10. L'état des bras depuis le 9e. jusqu'au 11e. jour.
11. L'état de la santé pendant ces 3 jours.
12. L'état des bras depuis le 12e. jusqu'au 15e. jour.
13. L'état de la santé pendant ces 4 jours.
\setcounter{page}{276}
14. Les remèdes qui lui sont administrés.
15. L'époque où on lui inocule la petite-vérole, et le résultat de cette inoculation.
16. Les observations nécessaires auxquelles la maladie peut donner lieu.

Nous terminerons cet Extrait par quelques fragments de notre correspondance avec les Drs. Jenner, Pearson, Aubert et De Carro.
Premier Fragment. Le Dr. JENNER au Prof. ODIER, en réponse à une lettre du 26 mars par laquelle on lui témoignoit quelques doutes sur la nature du virus vaccin employé à Genève dans le courant de l'hiver dernier; vu l'extrême rapidité de son développement. Chattenham en Gloucestershire 3me. juin.
"Je crains que le virus que vous avez employé ne soit pas bon. Dans le cours de ma pratique j'ai vu quelques cas où l'efflorescence a suivi de très-près l'inoculation. Mais ces cas ont été extrêmement rares, et alors je n'ai jamais considéré mon malade comme étant à l'abri de la petite-vérole. J'ai réitéré l'inoculation à l'autre bras, en en variant le mode; et cette seconde inoculation\setcounter{page}{277} a réussi, pour l'ordinaire, à produire un vrai bouton vaccin. J'entends par là un bouton qui parcourt régulièrement ses périodes d'inflammation, de maturation et de dessiccation, à peu-près dans le même espace de temps que dans l'inoculation de la petite-vérole."
"L'expérience m'a appris qu'il ne faut pas délayer le virus vaccin avec de l'eau chaude, mais avec de l'eau froide. Quand on l'expose au feu pour le faire sécher, il se décompose. C'est pourquoi je crains que la vapeur de l'eau bouillante avec laquelle quelques inoculateurs humectent une lancette imprégnée n'ait le même effet."
"La constitution de l'homme semble être toujours en guerre avec les poisons. S'ils s'introduisent dans le corps, ou si on les insère artificiellement dans la peau, ils produisent un choc, une action nuisible; et tous les pouvoirs de la machine humaine se mettent en activité pour les expulser. Si le poison est foible, ils y réussissent promptement. S'il est fort, leur victoire est lente, ou ils sont eux-mêmes subjugués \footnote{Cette théorie me paroît bien hasardée. Il est vrai que l'un des poisons les plus indomptables par les seules forces de la nature, celui de la rage, agit très-lentement; mais l'arsenic, l'opium, la plupart des poisons intérieurs agissent très-promptement et sont pas moins formidables. Quant aux poisons extérieurs, celui de la vipère, celui du serpent à sonnettes, agissent presque instantanément ; et cependant ce ne sont pas des poisons foibles. Le sublimé corrosif appliqué sur la peau, est bien plus actif que les cantharides, et cependant il occasionne une réaction beaucoup plus prompte. La peste n'est jamais plus terrible que lorsqu'elle agit promptement, etc. Il ne m'a point paru que le poison de la vaccine bâtarde fût un poison foible. Je l'ai vu produire, à la 3me génération, un érysipèle de 4 pouces de diamètre avec engorgement de tout le tissu cellulaire, au centre duquel étoit un ulcère fistuleux et très-profond, qui a duré plus de trois semaines en pleine suppuration. Cependant il n'a point garanti l'enfant de la petite vérole. Je la lui ai inoculée depuis. Il l'a eue heureuse, mais abondante. }. La promptitude de la\setcounter{page}{278} réaction, dans les cas que vous me citez, me fait craindre que le virus n'ait été, de manière ou d'autre trop affoibli pour pouvoir affecter la constitution au point nécessaire pour garantir de la petite vérole. Vos inoculés la prendront, ou je suis bien trompé. L'indisposition générale qu'ils ont éprouvée ne me paroît que symptomatique de l'affection locale. Je suis impatient d'apprendre le résultat de cette épreuve ; quelqu'il soit, il sera fort curieux."
"Je n'ai point encore lu vos remarques sur la vaccine. Je n'en connois que ce que\setcounter{page}{279} le Dr. Pearson en a publié dans le Magasin Philosophique. — Nous aurons besoin d'inventer un nouveau nom pour cette maladie. Car à présent, il ne me reste aucun doute sur la vérité de ma première conjecture, que la vaccine vient originairement du cheval. C'est ce que je viens de démontrer complètement, etc. \footnote{J'ai regret que le Dr. Jenner ne m'ait pas communiqué ses preuves. On paroît douter assez généralement en Angleterre de cette origine de la vaccine. Mais fût-elle bien démontrée, ce ne seroit pas une raison suffisante d'appeler la maladie Equine, au lieu de vaccine. Car d'après le Dr. J. lui-même, c'est dans la vache que le virus s'élabore et acquiert toutes ses propriétés. Quand on le prend sur le cheval même, on ne peut compter qu'imparfaitement sur ses effets, relativement à la petite-vérole. (O)}.

Second Fragment. Le Dr. PEARSON au Prof. ODIER, en réponse à une lettre du 4 avril, par laquelle on lui annonçoit que les deux premiers enfans inoculés à Genève avec le virus du Comte M. de Vienne avoient pris ensuite la petite-vérole. Londres. Leicester Square. 6me. mai.
"Il n'est pas aisé de résoudre, de prime abord, et sans de nouvelles expériences toutes\setcounter{page}{280} les difficultés qui se présentent dans l'histoire de la vaccine. Il est vraisemblable cependant, que le Comte n'avoit eu qu'une affection locale, et non la vraie vaccine. Car sa maladie a été trop rapide. Ce n'est qu'au bout de huit jours que le vrai bouton vaccin, semblable aux vésicules qui caractérisent le pemphigus est complètement développé. Il survient alors des douleurs sous l'aisselle et une légère fièvre éphémère, après laquelle le bouton sèche et se convertit en une croûte circulaire, noire, ou d'un beau brun, semblable à un petit bouton de corne, qui ne tombe qu'au bout de 15 jours ou plus tard, et qui laisse alors un creux, plus profond que celui de la petite vérole inoculée. Il arrive très-souvent que ce bouton n'est accompagné ni de douleurs subaxillaires, ni de fièvre, ni d'aucune indisposition constitutionnelle. Cependant il n'y a jusqu'à présent aucun exemple que, ni dans l'un ni dans l'autre cas, on ait pris ensuite la petite vérole. Toute la difficulté git donc à bien distinguer dans le bouton les caractères d'une maladie constitutionnelle; et c'est ce qui ne peut s'apprendre que par l'expérience. Il faut avoir vu au moins sept à huit cas de vraie vaccine pour être en état de la bien reconnaître. On ne peut jamais la prendre quand on a eu\setcounter{page}{281} la petite-vérole ou quand on a eu la vaccine elle-même\footnote{Le Dr. P. paroît tenir beaucoup à cette assertion. Il la répète dans toutes ses brochures et dans toutes ses lettres. Mais malgré ses expériences négatives, il est difficile d'expliquer, dans cette supposition, les observations positives des Drs. Jenner, Woodville et De Carro. L'effet de la vaccine sur le Comte fut rapide, mais non fugitif. L'érysipèle fut très-étendu; les plaies suppurèrent abondamment, et les croûtes ne tombèrent qu'au 29me. jour. Dans les enfans que nous avons inoculés d'après lui jusqu'à la 21me. génération, nous avons observé la même chose. (O)}; mais on peut avoir, et cela aussi souvent qu'on en fera l'expérience sur le même individu, un petit bouton plus fugitif, qui produit même une petite croûte noire, mais qui laisse à peine aucun creux après lui, et qui ne peut en imposer à un Médecin expérimenté. J'ai vu des cas analogues dans ce pays. J'avois inoculé près de 150 enfans aux environs de Londres. L'apothicaire qui les voyoit m'écrivoit qu'un d'entr'eux avoit pris la petite-vérole naturelle; j'y fus, et je trouvai qu'à l'endroit de l'inoculation il n'y avoit point le vrai bouton vaccin, mais un phlegmon ordinaire et suppurant. Il n'étoit donc pas étonnant qu'il eût pris la petite-vérole, ect."
\setcounter{page}{282}

Troisième Fragment. Lettre du Dr. ALBERT (le traducteur de Woodville) au Prof. ODIER. Londres 9me. mai.
"Je suis à présent convaincu de la vérité des assertions des Drs. Jenner, et Woodville. La vaccine est un préservatif sûr contre la petite-vérole. Elle n'est pas contagieuse. Elle n'a aucun symptôme dangereux, ou seulement grave. — J'ai eu le bonheur de trouver un ami dans la personne de Mr. Woodville. Le nombre de ses inoculations est déjà, à présent, tout près de 4000. Il m'a instruit, il m'a fait part de ses lumières, avec une bonté et une philanthropie rares. Il m'a placé dans son hôpital, me confiant le soin d'inoculer moi-même et d'observer par mes yeux. Ainsi j'ai vu dans un espace de temps très-court un nombre de faits que je n'eusse pas rencontré dans une pratique particulière de plusieurs années. Vous savez que dans ce pays-ci tout se fait sur une grande échelle. Outre 70 malades que nous avons dans la maison, il y en a constamment 6 ou 700 qui, après s'y être fait inoculer, retournent chez eux et ne viennent passer en revue que tous les deux jours. La saison commence à s'avancer; mais il y a des petites-véroles dans\setcounter{page}{283} la ville; et la peur nous amène encore beaucoup de sujets. Nous en eûmes 95 à inoculer il y a huit jours. Il est vrai que c'étoit un lundi; c'est le grand jour; mais depuis que je suis dans cette maison il est très-rare que je n'aie pas eu une vingtaine d'individus à inoculer chaque jour. - Je vous ferai parvenir des fils. Je soupçonne que c'est le meilleur mode d'inoculation, lorsqu'on ne peut pas inoculer d'un sujet à l'autre. Cependant on a souvent réussi à Londres avec des lancettes imprégnées six semaines ou deux mois auparavant. Si le Dr. C. a échoué à Paris, je présume que c'est parce qu'il n'a pas tenu assez long-temps à la vapeur de l'eau bouillante les lancettes que je lui avois remises, et qui n'avoient pas 15 jours, lorsqu'il les employa\footnote{L'observation du Dr. Jenner prouve que c'est plutôt la chaleur qu'il faut accuser de ce non succès. C'est peut-être aussi pour la même raison que nous n'avons pas réussi avec les premiers fils que nous avons essayés, parce que nous les trempions dans de l'eau tiède. (O)}. Le virus vaccin est très-visqueux; il perd sa limpidité presqu'au moment où il est en contact avec l'air; placé sur une lancette il s'y colle, avec un aspect si luisant et si lisse que je croirois presque qu'on pourroit faire une piqûre avec une pareille lancette,\setcounter{page}{284} sans qu'aucune parcelle de la matière demeurât attachée aux bords de la piqûre; en inoculant avec le fil, on n'a pas besoin d'humecter ou de délayer la matière. La sérosité, légèrement teinte de sang, qui sort de l'incision suffit."
"Tous les symptômes caractéristiques de la vaccine sont contenus dans le cours et l'aspect de la tumeur ; tellement que si vos inoculations futures vous donnoient des tumeurs semblables à celles que vous avez obtenues jusqu'ici, et à celle de l'enfant qui prit la petite-vérole, après avoir eu ce qu'on croyoit être la vaccine, vous seriez autorisés à affirmer que cet enfant avoit réellement eu la vaccine ; et alors on ne pourroit sauver la nouvelle inoculation du tort que cet exemple lui feroit, qu'autant qu'on démontreroit que l'inoculation subséquente de la petite-vérole n'avoit produit qu'une affection purement locale. Ce que je vois ici ne me permet pas de croire que ni cet enfant, ni aucun de vos premiers inoculés ait réellement eu la vaccine. Je désire vivement que vous réussissiez et que vous puissiez m'envoyer une comparaison de vos derniers résultats avec les premiers, surtout relativement au bouton, où à la tumeur, résultant de l'inoculation, etc. \footnote{Il n'y a en effet aucune ressemblance entre les résultats de nos premières inoculations, c'est-à-dire, celles que nous avions faites avec un virus originaire du Comte M., et de celles que nous avons faites depuis avec le virus qui nous a été envoyé directement d'Angleterre. Les premières avoient constamment produit à l'incision une rougeur très-prompte avec une apparence très-remarquable dans les bords, qui devenoient durs et gris, de manière à la faire ressembler exactement à un grain de café crud. L'efflorescence érysipélateuse se formoit tout autour, au 2\textsuperscript{d}. ou 3\textsuperscript{e}. jour au plus tard. Elle affectoit plus ou moins le tissu cellulaire qu'on sentoit engorgé au-dessous. Il n'y avoit à proprement parler, ni vésicule, ni bouton; mais il suintoit de l'incision et des bords un fluide limpide et jaunâtre, qui se coaguloit très-promptement à l'air et formoit une croûte épaisse, sous laquelle on trouvoit un vrai pus jusqu'au 8e. ou 9e. jour; et ce pus employé dans les inoculations subséquentes produisoit le même effet. On voit que cette maladie n'avoit aucun rapport avec la description que les Anglais nous donnent de la véritable vaccine. C'étoit pourtant, non l'effet fugitif d'un virus affoibli, mais une maladie sui generis, spécifiquement semblable à elle-même ; susceptible de se transmettre avec les mêmes caractères. Etoit-ce la vaccine dégénérée en conséquence de la petite-vérole que le Comte avoit eue dans son enfance ? Mais pourquoi les personnes que Woodville a inoculées avec le virus pris sur les bras de Sara Rice ou de Françoise Jerdel (Voyez la Bibl. Brit. vol. XII. Sc. et Arts p. 161) qui toutes deux avoient eu aussi la petite-vérole, n'ont-elles pas eu une maladie semblable à celle que nous avons observée? Pourquoi aucun de ces inoculés ne s'est-il, comme les nôtres, trouvé susceptible ensuite de prendre la petite-vérole? Quoiqu'il en soit, nos inoculés actuels nous présentent de tout autres phénomènes. Il n'y a aucune rougeur à l'incision pendant les trois ou quatre premiers jours. Dès qu'il y paroît quelque chose, on distingue l'apparence vésiculaire, qui va en croissant comme un bouton de la grosseur d'un pois, jusqu'au 11\textsuperscript{e}. jour, et s'entoure alors d'une auréole érysipélateuse d'un à trois pouces de diamètre. C'est le moment où la fièvre cesse. Cette fièvre est légère, elle ne se manifeste guères que par la fréquence du pouls. Elle survient au 8\textsuperscript{e}. jour. La vésicule est remplie d'un fluide limpide et visqueux qui ne contient point en pus. Au 11\textsuperscript{e}. ou 12\textsuperscript{e}. jour, elle sèche et se transforme ensuite en une croûte brune ou noirâtre, ronde et tenace. Toutes ces apparences indiquent clairement la vraie vaccine. On va commencer à en faire la preuve par l'inoculation de la petite-vérole}."
\setcounter{page}{285}

Quatrième Fragment. Lettre du Dr. PEARSON au Dr. DE CARRO. Londres 10me. avril.
"Nous ne sommes pas plus exempts d'intrigues au sujet de la vaccine que vous ne\setcounter{page}{286} fêtes à Vienne. Mais les nôtres sont d'un genre bien différent de celles dont vous vous plaignez. Chez nous elles partent de quelques-uns des plus instruits et des plus zélés participants de la nouvelle inoculation, qui se permettent d'insinuer que, toutes les fois que la vaccine inoculée est accompagnée d'éruption,\setcounter{page}{287} c'est parce que, par méprise, on a inoculé le virus variolique au lieu du vrai virus vaccin. Certainement, lorsque les éruptions ressemblent parfaitement, et en apparence et en durée, aux boutons de petite-vérole, je suis porté à croire qu'on doit les considérer comme vraiment varioliques; et le Dr. Woodville reconnaît lui-même aujourd'hui que le grand nombre de cas semblables qu'il a observés dans son hôpital, comparativement à leur rareté dans la pratique particulière, doit, jusqu'à un certain point, être attribué à l'influence de l'atmosphère variolique de cet hôpital. Mais prenez toutes les précautions qu'ils vous plaira pour écarter toute influence semblable, vous aurez encore, de temps à autre, un cas de vaccine inoculée, avec des boutons semblables à ceux de la petite-vérole. Est-ce le mélange inaperçu du virus variolique avec le virus vaccin qui les produit, ou bien le virus vaccin est-il susceptible de changer de nature dans certains tempéramens? C'est ce qui est encore douteux. Il seroit bien singulier que la vaccine pût mettre en activité un germe dormant de petite-vérole dans certains individus \footnote{Puisque les émanations varioliques peuvent avoir de l'influence sur la vaccine, pourquoi ne considérerait-on pas les inoculés vaccins qui ont eu des boutons semblables à ceux de la petite-vérole, comme ayant déjà pris cette dernière maladie par contagion avant l’inoculation? Les expériences de Woodville prouvent, que si l’on inocule la petite-vérole en même-temps que la vaccine, celle-ci n’empêche pas celle-là de se développer. Or, dans une ville comme Londres où la petitevérole est toujours épidémique dans un quartier ou dans l’autre, comment peut-on être assuré que ceux auxquels on inocule la vaccine n’ont pas déjà le germe d’une petite-vérole naturelle? C’est ce qui est certainement arrivé au premier enfant auquel on a inoculé ici la vraie vaccine. Il sortoit d’un quartier où la petite-vérole étoit épidémique, lorsqu’on l’a transporté à l’hôpital pour l’inoculer. Les bras n’ont commencé à marquer qu’au 4°. jour, et dès le 5°, long-temps avant que l’effet de la vaccine fût assez complet pour le garantir, la petite-vérole confluente dont il est mort, s’est manifestée. (O)}.
\setcounter{page}{288} " La principale difficulté relativement à la vaccine, est de bien distinguer quand elle affecte réellement la constitution. Car si son action est purement locale, elle ne garantit pas plus de la petite-vérole que l’inoculation de la petite-vérole elle-même, lorsqu’elle ne produit qu’une affection de ce genre \footnote{N’employons pas d’expressions obscures. Si l’action locale est complète, dans l’un et l’autre cas elle est suffisante. Si elle est incomplète, elle ne suffit pas. Qu’est-ce qu’une affection constitutionnelle, qui ne se manifeste que par des effets locaux? (O)}
\setcounter{page}{289}
La confiance du public dans l'inoculation de la vaccine a beaucoup augmenté ce printemps. Elle s'étend rapidement, surtout depuis que j'ai prouvé qu'on ne peut l'avoir constitutionnellement, ni deux fois, ni après avoir eu la petite-vérole\footnote{C'est ce dont je doute beaucoup. Car la vaccine n'étant pas contagieuse, peu importe qu'elle puisse revenir deux fois, ou qu'on puisse la prendre après avoir eu la petite-vérole. L'essentiel est qu'elle garantisse bien sûrement de cette dernière maladie. Quelque importante que soit, aux yeux des Médecins, la découverte du Dr. Pearson, si elle est bien constatée, elle n'est pas de nature à intéresser beaucoup le peuple. (O)}. On voit dans les campagnes les paysans se l'inoculer les uns aux autres avec la pointe d'un canif, ou d'une alène de cordonnier; tant il est vrai qu'on n'a aucune appréhension de danger. Et de fait, la vaccine est une maladie si peu dangereuse que sur plus de 6000 personnes, auxquelles on l'a inoculée jusqu'à présent, il n'en est mort qu'une seule; ce qui semblerait prouver qu'au contraire elle augmente la probabilité de vie\footnote{Dans l'âge où la probabilité de vie est la plus grande, il meurt une personne sur 50 dans l'espace d'un an; par conséquent une sur 600 dans l'espace d'un mois, et une sur 1200 dans l'espace de 15 jours. Puis donc qu'il ne meurt qu'un inoculé vaccin sur 6000 dans le même espace de temps, ceux-ci semblent avoir, par le seul fait de leur inoculation, une chance de vie 5 fois plus grande que ceux qu'on n'inocule pas. Mais pour pouvoir adopter cette conclusion qui prouveroit que la vaccine écarte d'autres maladies et qui feroit desirer qu'on pût l'inoculer plus d'une fois, il faudroit d'abord que l'on eût par-devers soi un grand nombre d'années d'expérience pour pouvoir calculer la moyenne de la probabilité de vie des inoculés vaccins, sur une base aussi étendue que celle sur laquelle on calcule la moyenne de la probabilité de vie en général. Il faudroit encore que cette dernière n'eût été calculée que sur des enfans aussi bien portans que ceux auxquels on inocule la vaccine; et c'est ce à quoi on n'a point eu égard dans les Tables générales de probabilité de vie. (O)}. Notre Etablissement va trèsbien,\setcounter{page}{290} malgré les basses intrigues par lesquelles on a d'abord cherché à le discréditer. Nous n'y voyons que très-rarement les inoculés avoir des boutons.—Il n'y a jusqu'à présent aucune preuve que la vaccine tire son origine du javart, etc."

Cinquième Fragment. Lettre du Dr. DE CARBO aux Rédacteurs de la Bibl. Brit; en leur envoyant copie de la précédente. Vienne 31 Mai.
"Je ne me dissimule point que ce que le Dr. P. appelle l'unique difficulté, c'est-à-dire, de distinguer les cas où la vaccine a agi constitutionnellement ( suivant son expression )\setcounter{page}{291} n'en soit une fort grande. Car il est certain que les cas que nous lisons dans l'ouvrage de Woodville, où il n'a aperçu aucune indisposition quelconque, sont trop nombreux pour pouvoir douter que l'effet vaccin ne soit suffisant pour garantir de la petite-vérole, même dans ces circonstances. Et sur 10 malades à qui j'ai inoculé la vaccine cette année avec des fils du Dr. ou avec de la matière provenant de ces fils, j'en ai eu trois ou quatre chez qui je n'ai pas pu observer la moindre indisposition, quoique les piqûres eussent toutes pris, et que les pustules aient été caractéristiques, c'est-à-dire, semblables à la description et aux gravures du Dr. Jenner."
"Je crois bien que dans les cas où il n'y a pas de fièvre, cette opération par laquelle la disposition à la petite-vérole se détruit, a lieu imperceptiblement; mais comment s'en assurer, et comment prononcer aux parens de l'enfant qu'il est bien à l'abri de l'infection variolique? En supposant même qu'on trouvât quelque moyen de distinguer l'effet local de l'effet général, comment engager des parens à laisser répéter une seconde ou troisième inoculation?"
"Parmi mes inoculés de cette année, un seul n'a pas pu prendre la vaccine que je lui ai inoculée à deux reprises. — Le premier\setcounter{page}{292} des 10 autres a eu une vaccine des mieux marquées et de la fièvre. Elle provenoit de ce fil de Pearson qui vous avoit toujours paru inerte. Les autres neuf ont tous été inoculés d'après lui, à la première ou à la seconde génération du virus. Aucun n'a eu la moindre apparence d'éruption. La fièvre a été des plus légères, et n'a jamais duré qu'une ou deux heures par jour. Aucune médecine n'a été nécessaire; aucune piqûre n'a manqué de produire son effet. Chez trois d'entr'eux, les seuls symptômes fébriles que j'aie pu observer ont paru le lendemain et sur-lendemain. Il ne s'est manifesté, ainsi que l'année passée, aucun engagement des glandes nécessaires, mais quelquefois de la douleur dans les bras. Un d'eux la ressentoit fortement jusqu'au bout des doigts. Aucune tendance à l'ulcération, la croûte restant toujours ronde, etc.

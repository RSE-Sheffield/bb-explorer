\setcounter{page}{146}
\chapter{MÉDECINE}
\section{RAPPORT SUR LA VACCINE, & fur l'inoculation de cette maladie, confidérée comme pouvant être fubftituée à la petite-vérole, par W. WOODVILLE, Médecin de l'hôpital des inoculés, à Londres. — Londres 16 mai 1799.}
CET ouvrage que nous avions annoncé, (vol. XI. Sc. & Arts, pag. 313) a paru à Londres il y a quelques mois, & a été auffitôt traduit à Paris par notre compatriote le Dr AUBERT, ce qui nous difpenfe de fuivre l'auteur pas à pas dans le récit de fes nombreufes & trop uniformes expériences. C'eft l'hiftoire de 200 inoculations faites avec le virus de la vaccine pris ou fur les boutons d'une vache atteinte de cette maladie, ou fur les bras des inoculés. Suit un tableau qui présente d'une manière plus rapprochée les noms des malades, leur âge, la filiation, la durée & le plus ou moins de gravité de la maladie &c. À ce premier tableau, l'auteur en ajoute un second qui contient de même les noms & l'âge de 310 perfonnes qui ont été inoculées depuis, avec le réfultat sommaire de ces inoculations,\setcounter{page}{147} dont l'auteur n'a pas eu le temps de donner l'histoire en détail. L'ouvrage est terminé par quelques réflexions sur la nature de la vaccine, sur les caractères qui la distinguent de la petite vérole, & sur la question de savoir s'il y aurait de l'avantage à substituer le virus vaccin au virus variolique pour préserver par l'inoculation de ce virus, l'individu qu'on inocule, de toute crainte de prendre à l'avenir la petite-vérole. Cette question est incontestablement de la plus haute importance. Car enfin la petite-vérole, on ne saurait trop le répéter, est le fléau le plus destructeur du genre humain. Tous les autres, la guerre, la peste, la famine, ne se manifestent que dans certains temps & dans certains lieux. De longs intervalles séparent leur retour. De vastes contrées en sont longtemps affranchies. La petite-vérole au contraire ne cesse depuis plusieurs siecles de faire, dans toutes les parties du monde à la fois, des ravages d'autant plus déplorables que malgré l'expérience acquise depuis plus de 80 ans de l'heureuse influence de l'inoculation pour dompter la malignité de ce poison; malgré les recommandations & les écrits multipliés des médecins les plus instruits, des savants & des hommes de lettres dont l'autorité a le plus de poids; malgré l'exemple qu'ils donnent presque tous en inoculant leurs enfants, cette belle découverte\setcounter{page}{148} n'a jusqu'à présent que peu contribué à diminuer la mortalité générale de la petite-vérole \footnote{C'est surtout en France qu'on a lieu de s'étonner du peu de progrès de l'inoculation: en France où Voltaire, La Condamine & tant d'autres auteurs du plus grand nom en ont depuis si long-temps fait victorieusement ressortir les avantages. Comment arrive-t-il, par exemple, que dans les Hôpitaux d'enfans trouvés on n'en inocule pas un seul; que les particuliers même les plus instruits & les plus aisés songent à peine à préserver leurs enfans de la petite-vérole autrement que par la ressource momentanée & précaire du séquestre; que dans la seule ville de Paris, on ait vu l'année dernière jusqu'à 15,000 personnes mourir de cette horrible maladie, & qu'enfin on y soit encore réduit à faire sonner si haut dans les Gazettes le succès de l'inoculation d'une vingtaine d'enfans au Prytanée, comme si c'étoit une découverte toute récente ? J'aime à remarquer à l'honneur de Genève que nous avons à cet égard devancé les Parisiens de 50 ans, puisque dès l'an 1751 on n'a cessé d'inoculer toutes les années dans cette Commune un très-grand nombre d'enfans avec un succès si marqué que l'inoculation est peut-être plus répandue chez nous qu'en Angleterre même. J'aime à rappeler encore que dans le petit Canton de Gex, un seul Médecin, le Cit. Girod a plus fait peut-être pour les progrès que tous les Médecins de Paris ensemble. (O)}. Il s'en faut bien que l'inoculation ait été adoptée, même en Angleterre, aussi universellement qu'il le faudroit pour avoir à cet égard un effet sensible. Malgré le généreux,\setcounter{page}{149} établissement d'un hôpital d'inoculation dans lequel les succès de cette pratique se soutiennent au point que sur plus de 3000 personnes inoculées en dernier lieu, le Dr. W. médecin de cet hôpital, nous apprend qu'il n'en est mort que 8 ou 9 : les registres mortuaires de Londres annoncent encore toutes les années un nombre effrayant de morts de petite-vérole\footnote{Je n'ai pas sous la main les derniers registres ; mais de 1766 à 1776, dans le temps où l'inoculation était le plus répandue, le nombre annuel et moyen des morts de petite-vérole était de 2273 sur 22046, nombre annuel et moyen de la totalité des morts. Voyez J. Watkinson's *Examination of a charge brought against Inoculation*. Lond. 1777. Nous avons de plus dans l'ouvrage même du Dr. Woodville une preuve frappante du peu de progrès que l'inoculation a fait jusqu'à présent à Londres, puisque sur 200 personnes dont une seule avait eu la petite-vérole, il s'en trouve plus du quart au-dessus de l'âge de 10 ans, et 36 qui en avaient plus de 15. On ne la leur avait donc pas inoculée dans leur enfance. Il y a 26 ans que j'inocule toutes les années à Genève ; mais il n'y a pas eu un de mes inoculés sur 50 qui fût d'un âge aussi avancé. (O)}. Les particuliers ne sont donc pas encore pleinement persuadés des avantages de l'inoculation. Les Gouvernements paraissent l'être encore moins, puisque loin de la favoriser et de chercher à la rendre générale, ils ont presque partout mis plus ou moins d'entraves\setcounter{page}{150} à ses progrès. Nous avons déjà remarqué ailleurs qu'on ne peut attribuer cette insouciance des uns & des autres qu'à deux causes. 1°. La crainte d'une éruption confluente qui quoiqu'infiniment rare dans la petite-vérole inoculée, arrive cependant quelquefois. 2°. La crainte de répandre la contagion. Or, les premieres observations que les Drs. Jenner & Pearson nous ont données sur la vaccine semblent prouver d'une part, que tout en garantissant très-sûrement de la petite-vérole, cette maladie est toujours très-légere, & exempte de boutons; & de l'autre qu'elle n'est jamais contagieuse que par un contact immédiat sur la peau mise à nud, & dépouillée de son épiderme. On avoit donc toutes fortes de raisons d'espérer que la substitution de ce virus à celui de la petite-vérole dans l'inoculation pourroit non-seulement dissiper les appréhensions du public & les craintes des Gouvernemens, mais encore fournir à la longue un moyen sûr de détruire enfin ou de rendre absolument inertes tous les foyers de contagion.
Ces avantages frapperent le Dr. Woodville à la premiere lecture de l'ouvrage du Dr. Jenner." Je crus, "dit-il," que dans la place" que j'occupois à l'hôpital d'inoculation, mon" devoir m'imposoit, pour le bien public, de" saisir la premiere occasion de mettre à exécu-" tion le plan des expériences que j'avois con-" çues pour les vérifier."
\setcounter{page}{151}
Il semble que les trois questions capitales à examiner étoient: 1. Si la vaccine garantit sûrement & pour toujours du danger de prendre la petite-vérole. 2. Si elle est exempte de boutons. 3. Si elle est contagieuse autrement que par inoculation. Il ne paroît cependant pas que notre auteur se soit proposé directement ces questions. Ses expériences nous semblent avoir été faites un peu au hasard & plutôt dans un but de curiosité & de théorie que de vraie utilité & de pratique, pour déterminer les rapports du virus vaccin avec le virus variolique, plutôt que pour examiner s'il mérite de lui être substitué dans l'inoculation. Voyons néanmoins jusqu'à quel point elles sont propres à décider les questions que nous venons d'énoncer. Cette manière de rendre compte de l'ouvrage fera moins fastidieuse pour nos lecteurs, & peut-être ne fera-t-elle pas sans intérêt pour ceux même qui possèdent l'original ou la traduction.
I. Toutes les personnes qui ont eu la vaccine, soit naturellement pour avoir trait des vaches atteintes de cette maladie, soit par inoculation, sont dès-lors incapables de prendre la petite-vérole, de quelque manière qu'elles y soient exposées. Tel est le premier fait avancé par les Drs. Jenner & Pearson. "Les expériences qui le démontrent, sont" dit notre auteur, "authentiques, décisives & en assez grand\setcounter{page}{152} nombre pour l'établir d'une manière satisfaisante. J'ai inoculé la petite-vérole à plus de 400 malades qui avoient eu la vaccine. Aucun d'eux ne l'a prise. Cependant il est à remarquer que le quart d'entr'eux avoit été si légèrement affecté par le virus vaccin qu'il n'avoit produit aucune indisposition sensible. Nous sommes loin de contester le fait; Il nous paraît bien démontré par les expériences des Drs. Jenner & Pearson; dont nous avons parlé ailleurs. Mais qu'il nous soit permis de remarquer que celles du Dr. W. ne sont point aussi concluantes. Sur les 200 personnes auxquelles il a inoculé la vaccine & dont il a publié l'histoire en détail, il y en a 166 dont il affirme en général qu'on leur a inoculé depuis la petite-vérole, sans qu'elle produisit aucun effet. Il ne dit point combien de temps après la guérison de la vaccine s'est fait cette inoculation. Mais à en juger d'après les premiers cas dont il rapporte l'histoire plus en détail; & dans lesquels il spécifie précisément le jour où il avoit soumis les malades à cette épreuve; il paraît s'être trop hâté de la faire. On fait que par une loi assez générale dans le corps humain, il est rare qu'il puisse être affecté tout-à-la-fois de deux maladies existant ensemble. On fait en particulier que la petite-vérole a été fréquemment suspendue, ou étouffée par la rougeole, la fièvre rouge, ou d'autres \setcounter{page}{153} très maladies\footnote{La petite-vérole paroît avoir été assez régulièrement épidémique à Genève tous les 4 ou 5 ans. Elle auroit dû l'être en 1615 ou 1616 ; mais cette année là il y eut une grande épidémie de peste. Aussi nos registres mortuaires ne présentent pas une seule mort de petite-vérole depuis 1611 jusqu'en 1620. Dans la grande épidémie de peste qui désola la ville de Londres en 1666, le nombre des morts de petite-vérole fut aussi réduit à 38 ; ce qui est infiniment peu, comparativement au nombre annuel & moyen, qui étoit alors de 995. (O)}. Le Dr. John Hunter présume que c'est-là une des raisons pour lesquelles il arrive que l'inoculation ne réussisse point dans certains cas, & dans d'autres produit son effet beaucoup plus tard qu'à l'époque ordinaire\footnote{Voyez John Hunter's Treatise on the venereal disease, in. 4°. Lond. 1786, pag. 3. En admettant le principe, je suis loin d'adopter l'application qu'en fait dans ce passage le Dr. H. pour expliquer l'effet tardif ou nul de l'inoculation. J'ai vu maintes & maintes fois les incisions varioliques ne produire leur effet local que de 6 à 12 jours plus tard que l'époque ordinaire, sans qu'on apperçût aucun symptôme de maladie auquel on pût attribuer cette suspension ; & de ce que la rougeole ou la fièvre rouge produisent quelquefois un semblable retard, on ne peut pas conclure que toutes les fois qu'il a lieu, il est produit par une maladie. (O)}. Or ne peut-on pas soupçonner que si la petite-vérole inoculée quelques jours seulement après la vaccine ne se manifeste point,\setcounter{page}{154} ce n'est pas que la vaccine ait rendu pour toujours le corps incapable de la prendre, mais parce qu'elle a suspendu pour un temps seulement cette susceptibilité. Pour avoir à cet égard une certitude complète, il faudrait donc différer l'inoculation de la petite-vérole jusqu'à-ce que la vaccine eût produit tout son effet, & qu'il n'en restât plus aucune trace. On sait, par expérience que l'action de la petite-vérole se prolonge jusqu'à 6 ou 7 décades ; celle de la fièvre rouge jusqu'à 4 ou 5 depuis le moment de l'invasion \footnote{Voyez l'excellent Mémoire sur la Fièvre rouge, que mon savant confrère le Dr. Vieuffeux vient de publier dans le Journal de Médecine de Sedillot. Je ne sais pourquoi les Rédacteurs de ce Journal se sont permis d'altérer toutes les pages de ce Mémoire d'une manière fort étrange & fort décourageante. Mais tel qu'il est, il contient encore beaucoup de bonnes & solides instructions. Il y a 15 ans que l'importance de l'objet m'engagea à envoyer à la Société de Médecine un Mémoire sur le même sujet avec un autre sur l'Hydrocéphale interne. On m'accusa la réception des deux Mémoires ; mais je ne sais pourquoi on n'imprima que le dernier. Le premier aurait pourtant été d'un intérêt bien plus général & plus analogue à l'institution de la Société, qui avait spécialement été établie pour recueillir tous les faits qui concernent les maladies épidémiques. (O)}. Il aurait donc été prudent de mettre un intervalle de deux mois au moins entre l'inoculation de la vaccine & celle de la petite-vérole. C'est ce que ne paraît\setcounter{page}{155} point avoir observé le Dr. Woodville. Voici ce qu'il rapporte là-dessus des 34 malades restans dont il donne l'histoire détaillée.
4 n'ont point été soumis à l'inoculation de la petite-vérole après celle de la vaccine; mais ils ont plusieurs fois depuis été exposés à la contagion naturelle.
1 avoit eu la petite-vérole inoculée plusieurs années auparavant.
4 ont été inoculés avec du pus variolique la veille du jour où on leur a inoculé la vaccine.
2 en même temps.

\comment{figure}
1 . . . 3
5 . . . 5
1 . . . 7
1 . . . 9
1 . . . 10
1 . . . 11
1 . . . 13
1 . . . 14
2 . . . 15
1 . . . 16
1 . . . 18
2 . . . 20
2 . . . 22
1 . . . 23
1 . . . 25 jours après.

Il n'a donc gueres mis plus de deux décades entre les deux inoculations; & par une expression qui lui échappe à la fin de sa brochure, il paroitroit qu'il a pris pour règle générale, d'inoculer la petite-vérole cinq jours seulement après avoir inoculé la vaccine. Cet intervalle est certainement beaucoup trop court.
\setcounter{page}{156}
Ce qui sembleroit d'abord infirmer un peu la critique que nous venons de faire, c'est l'effet singulier de ces inoculations rapprochées l'une de l'autre. Car lorsqu'il n'y a pas eu plus de cinq jours d'intervalle de l'une à l'autre, les deux virus ont produit simultanément leur effet local complet. L'incision variolique s'est enflammée aussi bien que la vaccine. Elle a été entourée d'une efflorescence. Le bouton s'est transformé en une vésicule. Le fluide qu'il contenoit d'abord clair & limpide, a ensuite blanchi. Enfin il s'est desséché, & le bouton s'est converti en une croûte épaisse qui n'est tombée qu'au bout de quelques jours. En un mot, la marche de l'incision variolique n'a point paru influencée par le virus vaccin. Et même, lorsqu'on a inoculé en même temps ou à-peu-près, l'une des deux maladies au bras droit, & l'autre au bras gauche, le malade s'est plaint de douleurs & de dégonflement sous les aisselles des deux bras.
On a bien pû dans ces cas là apercevoir la différence qui existe entre l'inflammation locale produite par l'inoculation de la vaccine, & celle qui est produite par la petite vérole. Dans la première, la tumeur est presque toujours entourée d'une efflorescence plus étendue ; elle est plus égale, mieux circonscrite ; le fluide qu'elle contient ne perd point sa transparence avant de se sécher. Il ne se convertit\setcounter{page}{157} point en pus. Sa désiccation est plus prompte; la croûte qu'elle produit est moins raboteuse, plus douce au toucher & d'une couleur moins foncée que celle de la petite-vérole. Elle ressemble par sa teinte au bois de mahogany. La tumeur produite par l'inoculation de la petite-vérole est au contraire plus inégale & plus angulaire. Le fluide qu'elle contient blanchit & se convertit en pus. La croûte qu'elle forme est plus raboteuse & d'un brun plus foncé. En un mot, les deux tumeurs conservent leurs caractères distinctifs & leur aspect pathognomonique reste le même pendant tout le cours de la maladie. On ne peut jamais les prendre l'une pour l'autre. Quant à l'affection générale qui en résulte, il est très-difficile de distinguer celle de la vaccine de celle de la petite-vérole inoculée. Dans quelques malades elle a été nulle ou très-légère & presque imperceptible. Dans d'autres la fièvre a été plus forte & accompagnée d'une éruption assez considérable.
Le résultat de ces expériences engagea le Dr. W. à les varier. Il avoit vu qu'il ne résulte jamais de deux inoculations simultanées avec les deux virus différens, une maladie hybride, ou participant à la nature de l'une & de l'autre, au moins quant à l'affection locale; mais que la tumeur produite par l'inoculation suit toujours la marche propre à la matière avec laquelle\setcounter{page}{158} on inocule. Il fut curieux de voir si une seule inoculation faite avec du virus mélangé ne produiroit pas une affection moyenne & différente de celle que produit chacun des virus séparément. Pour cet effet, il inocula le même jour 28 personnes avec le virus de la vaccine, & avec celui de la petite-vérole mêlés ensemble par parties égales. Le résultat de cette expérience fut que chez plus de la moitié des personnes inoculées de cette manière, l'affection locale prit les caractères distinctifs de la vaccine; chez les autres elle parut avoir ceux de la petite-vérole. Mais les uns & les autres n'eurent qu'une indisposition très-légère, & qu'un très-petit nombre de boutons.
Il paroit donc que les deux affections locales peuvent exister ensemble dans différentes parties du corps, mais que dans la même partie elles s'excluent mutuellement. L'une ou l'autre prédomine, & jamais il n'en résulte une tumeur d'un genre mixte.
Quant à l'affection générale, c'est autre chose. Il n'y a aucun caractère distinctif entre celle que produit la vaccine & celle que produit la petite-vérole. Le Dr. Jenner avoit cru que jamais la vaccine ne produit de boutons, ou du moins de boutons suppurans, & de plus qu'elle n'est jamais contagieuse autrement que par inoculation. Dans une 2de. brochure, postérieure à celle du Dr. W. il persiste dans cette opinion. Le Dr. W. a vu la\setcounter{page}{159} chose différemment. Reste à savoir si la maladie qu'a vue ce dernier n'est pas une maladie hybride. Cela semblerait assez naturel dans les cas où les deux inoculations ont été assez rapprochées pour produire l'une et l'autre leur effet local. Il est assez probable que dans ces cas-là les deux affections générales ont pu par leur coïncidence, se confondre l'une avec l'autre. Mais dans les cas où l'inoculation de la petite-vérole plus retardée n'a produit aucun effet local, il semble au premier coup-d'œil que le soupçon d'une maladie hybride serait bien hypothétique. C'est ce que nous examinerons bientôt.
Quant à présent il nous suffit de faire remarquer que bien que les deux maladies, tant qu'elles ne font que locales, puissent exister simultanément dans différentes parties du corps, il ne s'ensuit pas qu'il en soit de même de l'affection générale; et si l'on n'avait d'autres preuves de la propriété qu'a le virus vaccin de préserver pour toujours de la petite-vérole que les expériences du Dr. W.; on pourrait légitimement soupçonner quelque illusion dans leur résultat; on pourrait croire que si l'inoculation de la petite-vérole n'a eu aucun effet sur les malades auxquels il avait antérieurement donné la vaccine, c'est parce qu'il les a inoculés avant que l'action générale de cette dernière maladie fût entièrement terminée, et qu'il n'en eût peut.\setcounter{page}{160} être pas été de même, s'il les avoit inoculés plus tard.
Mais cette propriété du virus vaccin repose déjà sur un si grand nombre de faits authentiques qu'elle n'a pas besoin de preuves ultérieures. Jusqu'à présent elle n'a été contestée de bonne foi par personne; & quoique les expériences directes du Dr. n'ajoûtent que peu de poids aux preuves qu'en ont données les Drs. Jenner & Pearson, on trouve cependant dans son ouvrage un cas remarquable qui tendroit à prouver que même au milieu des miasmes varioliques, & lorsque le malade en est entouré de toutes parts, l'inoculation de la vaccine peut empêcher, suspendre, détruire ou fingulièrement modifier leur effet. C'est l'histoire d'un enfant de 4 mois, Charlotte Fisk, qui fut inoculée le 13 février dernier, ( 24 pluviôse ) dans un moment où sa mere étoit malade de la petite-vérole naturelle, & couverte de boutons purulens. Celle-ci ne cessa pas d'allaiter son enfant pendant sa maladie; & l'on vit souvent la petite couverte du pus des boutons de sa mere. Cependant la marche locale de la vaccine fut très-réguliere. L'enfant fut incommodé au 8e. jour, & continua à l'être pendant 3 ou 4 jours. Alors il sortit 14 boutons, mais dont la plus grande partie ne vint point à suppuration. Enfin, le Dr. remarque que plusieurs de ses inoculés se sont\setcounter{page}{161} trouvés dès le moment de l'inoculation placés de manière à être constamment exposés à la contagion de la petite vérole sans la prendre.
Ajoutons ici que, comme le Dr Jenner l'avoit déjà observé, cette propriété du virus vaccin n'est pas réciproque, & qu'il ne suffit pas d'avoir eu la petite vérole pour être à l'abri de la vaccine. Car dans les 200 premieres personnes auxquelles le Dr a inoculé cette dernière maladie, on voit une jeune femme (Françoise Jewel) âgée de 20 ans, qui avoit eu dans son enfance la petite vérole inoculée, ce qui n'empêcha pas que l'inoculation de la vaccine ne réussît fort bien. L'affection générale fut à la vérité assez légère, & bornée à deux jours de fièvre, accompagnée de maux de tête & de reins, mais d'ailleurs le succès de l'inoculation fut complet. De plus, une des sources naturelles où l'auteur a puisé le virus vaccin pour l'inoculer étoit une laitière, nommée Sara Rice, qui quoiqu'elle eût eu la petite vérole dans son enfance, prit la vaccine en trayant des vaches qui en étoient atteintes, & l'eût, légère à la vérité quant à l'affection générale, mais bien caractérisée. "Cependant j'ai quelque raison de croire" dit le Dr. "que la vaccine n'attaque pas les personnes qui ont eu la petite vérole, aussi aisément qu'on l'a prétendu. J'ai effayé plusieurs fois d'inoculer la vaccine à des malades encore convalescens\setcounter{page}{162} de la petite vérole naturelle, & qui avoient eu une éruption complete. Jamais je n'ai vu s'élever de tumeur à la fuite de ces inoculations. L'argument n'eft pas péremptoire. On pourroit facilement la rétorquer contre l'auteur, & demander fi de même que la petite vérole fufpend la fufceptibilité de prendre la vaccine fans la détruire complettement, il n'a pas pû arriver dans fes expériences fur l'inverfe de ce fait, que la vaccine n'ait fufpendu que momentanément la fufceptibilité de prendre la petite vérole.
Mais en voilà affez fur cette premiere queftion. Dans un autre Extrait, nous examinerons les faits relatifs à la feconde & à la troifieme. En attendant, nous nous empreffons de publier quelques fragmens de lettres que nous avons reçues fur cet objet, de notre compatriote le Dr. De Carro. Ils intérefferont peut-être nos lecteurs & les prépareront à ce que nous avons à dire encore de l'ouvrage du Dr. W.

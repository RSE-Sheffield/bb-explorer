\setcounter{page}{271}
\chapter{MÉDECINE.}
\section{RAPPORT sur la VACCINE & sur l'inoculation de cette maladie, considérée comme pouvant être substituée à la petite-vérole; par W. WOODVILLE, Médecin de l'Hôpital des inoculés à Londres. \large{Second Extrait. (Voyez pag. 146 de ce Vol. )}}
LA premiere question à examiner pour pouvoir avec sécurité substituer l'inoculation de la vaccine à celle de la petite-vérole, est, de savoir si la premiere de ces deux maladies, garantit bien sûrement & pour toujours du danger de prendre la seconde. Nous avons vu jusqu'à quel point les observations du Dr. W. confirment à cet égard celles qui ont été publiées par les Drs. Jenner, Pearfon, &c. Voyons maintenant, s'il y auroit quelque avantage réel dans cette substitution.
Il est évident que pour l'individu qu'on inocule, cette question se réduit à celle-ci: la vaccine inoculée, est-elle une maladie plus bénigne, moins dangereuse & moins meurtriere que la petite-vérole, transmise de la même maniere?
Le Dr. W. a publié deux tableaux propres\setcounter{page}{272} à décider cette question, & qui indiquent, l'un, l'âge des 200 premiers malades auxquels il a inoculé la vaccine; le nombre de jours pendant lesquels ils ont eu de la fièvre; le nombre de boutons qui se sont manifestés ailleurs qu'au bras, dans le cours de la maladie, & l'origine du virus dont on s'est servi pour l'inoculation; le second, le résultat sommaire de 310 autres inoculations semblables & postérieures. Mais comme nous avons remarqué plusieurs inexactitudes dans le premier de ces deux tableaux, inexactitudes qui ne peuvent être rectifiées que par l'histoire détaillée qui l'accompagne, nous avons lieu de soupçonner de même des erreurs dans le second, sans avoir aucun moyen de les rectifier, puisque l'auteur n'a pas eu le temps de publier en détail l'histoire des individus, inoculés, dont il contient la liste, comme il l'avoit fait pour le premier. Nous nous en tenons donc à celui-ci; & afin que nos lecteurs puissent s'en faire une idée nette, nous le leur présentons, non pas tel qu'il est, mais tel qu'il faut qu'il soit pour y voir, d'un coup-d'œil & dans différentes colonnes, toutes les conséquences qu'on peut en tirer. Le tableau du Dr. ressemble à ces suites d'observations météorologiques, dont il n'y a que les lecteurs, affez patiens, pour en calculer les moyennes & les comparer entr'elles, qui puissent en tirer parti. Le nôtre est le résultat\setcounter{page}{273} même de ces ennuyeux calculs. C'est une table de comparaisons toutes faites. Ce tableau est divisé par colonnes horizontales, qui indiquent l'âge des inoculés, & par colonnes verticales qui indiquent leur sexe, le nombre approximatif, total & moyen, des boutons qu'ils ont eus; & enfin, celui des jours de fièvre qu'ils ont éprouvés dans le cours de la maladie. Pour calculer le nombre moyen des boutons, nous ne divisons pas leur nombre total par celui des individus. Il est évident que cette moyenne ainsi calculée se trouveroit beaucoup trop forte, puisque le nombre total des boutons qu'ont eu ces 200 inoculés entr'eux tous étant de 8857, il en résulteroit une moyenne de plus de 44 boutons par individu, tandis qu'il n'y en a eu que 39, qui en ayent eu au-delà de 40. La véritable moyenne, se trouve donc en rangeant les inoculés, d'après le nombre de leurs boutons, & en indiquant le nombre qu'en a eu celui d'entr'eux qui se trouve de cette manière placé au milieu ou à une égale distance de ceux qui en ont eu le plus, & de ceux qui en ont eu le moins \footnote{C'est plutôt là le nombre probable de boutons qu'auront les inoculés que le nombre moyen de ces boutons. Il y a entre ces deux nombres la même différence que celle qui existe entre la probabilité de vie & la vie moyenne. (Voyez le vol. V de ce Journal Sc. & Arts, p. 308. Car, comme la probabilité qu'on parviendra à tel ou tel âge s'estime par le rapport du nombre d'individus qui l'atteignent avec le nombre de ceux qui le dépassent, de même, si l'on veut savoir la probabilité que tel inoculé aura tant ou tant de boutons, il faut chercher combien d'inoculés en ont eu davantage & combien d'autres en ont eu moins. Le rapport de ces deux nombres est précisément la probabilité qu'on cherche. (O)}. Cette\setcounter{page}{274} moyenne se trouve, par ce calcul, réduite pour la totalité des 200 inoculés à deux boutons tout au plus, par individu, c'est-à-dire, qu'il n'y en a pas eu la moitié qui ayent eu au-delà de 2 boutons. Mais en se bornant à des individus d'un certain âge, elle varie suivant les âges, & se trouve, par exemple, beaucoup plus forte au-dessus de l'âge de 10 ans qu'au-dessous.
Quant au nombre moyen des jours de fièvre, comme les différences d'un individu à l'autre, ont été beaucoup moins considérables, il peut se calculer de la manière ordinaire, en divisant le nombre total des jours de fièvre qu'ont eu entr'eux tous, les individus inoculés par le nombre de ces individus. C'est ainsi que nous l'avons indiqué, en poussant le calcul jusqu'aux millièmes, non que l'auteur ait indiqué le plus ou moins de fièvre de ses inoculés en fractions de jours, ni même que cela soit jamais praticable; mais pour mieux faire ressortir les différences\setcounter{page}{275} qu'il y a eu à cet égard, entre les inoculés de tel ou tel âge. Ainsi, les 200 inoculés ayant eu entr'eux tous, 423 jours de fièvre, on peut dire que l'un dans l'autre, chacun d'eux a eu $\frac{423}{200} = 2,115$ jours de fièvre, ou pour parler plus exactement $\frac{423}{199} = 2,125$; parce qu'il n'y en a eu que 199 sur lesquels l'inoculation ait produit son effet. Mais les garçons, au nombre de 98, ayant eu entr'eux tous 201 jours de fièvre, & les filles au nombre de 101, en ayant eu 222, on peut dire que le nombre moyen des jours de fièvre pour les uns a été $\frac{201}{98}=2,051$ & pour les autres $\frac{222}{101}=2,198$; ce qui montre qu'en général les filles ont eu un peu plus de fièvre que les garçons (1)\footnote{Ces moyennes sont bien différentes de celles qu'on obtiendrait si on les calculait comme la probabilité de vie. Car puisque sur 199 inoculés il y en a eu 85 qui n'ont point eu de fièvre, ou qui n'en ont eu qu'un jour, la probabilité que l'inoculation de la vaccine, si elle a son effet, ne produira pas plus d'un jour de fièvre, est de 85 contre 114, ou 1000 contre 1341; & puisqu'il y a eu 121 inoculés qui n'en ont pas eu plus de 2 jours, la probabilité que l'inoculation efficace de la vaccine ne produira pas plus de deux jours de fièvre est de 121 contre 78 ou 1551 contre 1000. Par un calcul analogue, on trouvera que relativement aux garçons, il y a 1000 à parier contre 1450 qu'ils n'auront pas plus d'un jour de fièvre, & 1578 contre 1000 qu'ils n'en auront pas plus de deux, au lieu que relativement aux filles, il y a 1000 à parier contre 1244 qu'elles n'en auront pas plus d'un jour, & 1525 contre 1000 qu'elles n'en auront pas plus de deux. On peut conclure de ce calcul qu'il est probable que le nombre des filles inoculées qui auront peu de fievre fera plus grand que celui des garçons dans la même catégorie; mais que celles qui en auront plus d'un jour, en auront davantage que ceux des garçons qui se trouveront dans le même cas.}.\setcounter{page}{276} Ou peut de cette maniere comparer sur le tableau l'influence de l'âge & du sexe dans toutes les époques de la vie.
Le premier fait remarquable, c'est le grand nombre d'inoculés de tout âge, chez lesquels on n'a observé aucun mal-aise général, aucun symptôme fébrile. Il y en a eu 53; savoir, 23 garçons & 30 filles. Cependant, aucun d'eux n'a pu prendre ensuite la petite-vérole, par l'inoculation, excepté une jeune fille âgée de 15 mois qui fut la seule des 200 personnes auxquelles on avoit inoculé la vaccine, sur laquelle elle ne prit, ni localement, ni d'une maniere générale. Tous les autres eurent la vaccine locale bien caractérisée; & cette affection quoique purement locale, fut suffisante pour les garantir de la petite-vérole. Il résulte delà ou que l'affection générale n'est pas nécessaire pour mettre à l'abri de la petite-vérole, ou qu'elle peut avoir lieu sans se manifester par la fievre. C'est probablement à ce dernier\setcounter{page}{277} résultat qu'il faut s'arrêter. Car, parmi ces 52 inoculés chez lesquels la vaccine ne parut exciter aucun symptôme fébrile, il y en eut 8 qui eurent des boutons. L'un d'eux, jeune homme de 21 ans, en eut même jusqu'à 300. Nous examinerons bientôt, si ces boutons appartenoient à la vaccine, ou à une autre maladie; mais il n'en font pas moins la preuve d'une affection générale, développée par la vaccine, & indépendante de la fièvre. Une affection générale peut donc avoir lieu sans se manifester par aucun symptôme fébrile, & il y a lieu de croire que cette affection existe toujours lorsque l'inflammation locale est complète\footnote{J'ai souvent observé que des inoculés qui n'avoient que peu ou point de boutons, & qui conséquemment, après la cessation de la fièvre sembloient n'être plus sous l'influence d'aucune affection générale, avoient cependant presque autant de malingrerie & de mauvaise humeur durant les cinq ou six jours pendant lesquels la suppuration auroit eu lieu, s'ils avoient eu beaucoup de boutons, que s'ils avoient réellement eu encore beaucoup de fièvre & une éruption abondante. C'est ce qu'on voit surtout chez les très-petits enfans. Il est incontestable que certaines affections locales produisent souvent une action générale qui ne se manifeste par aucun symptôme apparent. On voit quelquefois un simple furoncle produire une diathèse inflammatoire qui ne donne ni fièvre, ni mal-aise général, mais qui influe sur la totalité des vaisseaux artériels, au point de rendre le pouls dur & tendu, & de ralentir la coagulabilité de toute la masse du sang, ralentissement auquel tient comme l'on fait, la couenne jaune qui se trouve sur la surface du sang veineux qu'on tire par la saignée.}.\setcounter{page}{278} Seulement on doit la considérer comme beaucoup plus bénigne, lorsqu'elle ne produit aucun symptôme apparent; & comme cela n'arrive pas à beaucoup près aussi fréquemment dans la petite-vérole inoculée que dans la vaccine, il en résulte déjà qu'en thèse générale, la vaccine est une maladie plus bénigne que la petite-vérole transmise de la même manière.
C'est ce qui nous semble résulter encore de la considération que les 147 inoculés qui ont eu de la fièvre, n'en ont eu entr'eux tous que 423 jours, ce qui fait pour chacun d'eux, l'un dans l'autre 2,877, c'est-à-dire, moins de 3 jours. Or, ce n'est guères qu'au 4e. jour que la fièvre de la petite-vérole inoculée cesse, même dans les cas où la maladie est la plus bénigne. Car ce n'est qu'alors pour l'ordinaire que se manifeste l'aréole érysipélateuse qui la termine. Enfin, sur les 199 inoculés, il n'y en a eu que 21 qui aient eu plus de 4 jours de fièvre, c'est-à-dire, à peu-près 1 sur 10, ou plus exactement 105 sur 1000. C'est peut-être un peu plus que dans la petite-vérole inoculée. Cependant, celle-ci présente aussi bien des cas où si la fièvre est interrompue par la première apparence \setcounter{page}{279} de l'éruption, elle recommence ensuite, & dure trois ou quatre jours pendant la suppuration, ce qui met les deux maladies à cet égard pour le moins au niveau l'une de l'autre.
Remarquons ensuite qu'une différence essentielle paroît se trouver ici entre la vaccine & la petite-vérole inoculée; c'est que la première de ces deux maladies est beaucoup plus bénigne que la seconde dans les trois premières années de la vie, & beaucoup moins au-dessus de l'âge de 10 ans. Car sur les 103 enfans au-dessous de l'âge de 3 ans auxquels on a inoculé la vaccine, il y en a eu 31 qui n'ont point eu de fièvre & les 72 autres n'en ont eu entr'eux tous que 172 jours, ce qui fait pour chacun d'eux, l'un dans l'autre, 2,388, c'est-à-dire, moins de 2 jours & demi. Or, il est reconnu, que si dans la petite-vérole inoculée on a beaucoup de fièvre ou une éruption confluente à craindre, c'est surtout au-dessous de l'âge de trois ans. Il est même des inoculateurs qui ayant plus à cœur leur propre réputation que le bien général qui peut résulter de l'inoculation, refusent d'inoculer les enfans au-dessous de l'âge de trois ans, & aiment mieux les abandonner au risque incomparablement plus grand de prendre la petite-vérole naturelle & d'y succomber, que de courir eux-mêmes celui d'un succès tant soit peu moins assuré. Il est prouvé cependant que si leur opinion devenoit générale,\setcounter{page}{280} la petite-vérole exerceroit encore tous ses ravages malgré l'utilité de leur art sur plus de la moitié des individus qu'elle moissonne. Les avantages de l'inoculation pour le public feroient diminués de près des trois-cinquièmes \footnote{Dans l'espace de 180 ans, il est mort à Genève 6792 personnes de la petite-vérole. Voici leur âge:

\comment{table}
au-dessous de 3 mois . . . . . 140
entre 3 & 6 mois . . . . . . . 390
entre 6 & 9 mois . . . . . . . 430
entre 9 mois & 1 an . . . . . . 416
entre 1 & 2 ans . . . . . . . 1300
entre 2 & 3 ans . . . . . . . 1290
................... 3966
entre 3 & 4 ans . . . . . . . 898
entre 4 & 5 ans . . . . . . . 603
entre 5 & 6 ans . . . . . . . 381
entre 6 & 7 ans . . . . . . . 301
entre 7 & 8 ans . . . . . . . 189
entre 8 & 9 ans . . . . . . . 109
entre 9 & 10 ans . . . . . . . 78
au-dessus de 10 ans . . . . . 267
Total . . . . . . . 6792

Si donc on n'inoculoit aucun enfant avant l'âge de 3 ans, le bénéfice de l'inoculation feroit perdu pour les $\frac{3966}{6792}$ = 0,583 des hommes. Remarquons de plus, que le calcul précédent est fondé sur l'extrait mortuaire d'une ville où la petite-vérole n'a jamais été continuelle, mais assujettie à des retours épidémiques qui ont eu lieu assez régulièrement tous les cinq ans ; & dans les intervalles desquels il n'y a eu que peu ou point de foyers de contagion. Mais dans une grande ville comme Paris, Londres, Vienne, Berlin, & où les petits enfans sont continuellement exposés, le nombre des morts de petite-vérole au-dessous de l'âge de 3 ans doit être bien plus considérable, comparativement au nombre total. (O)}.\setcounter{page}{281} Il est donc extrêmement important de garantir de très-bonne heure les enfans de la petite-vérole, puisque c'est surtout sur les plus jeunes d'entr'eux que porte sa grande mortalité; & la vaccine paroissant avoir un grand avantage sur la petite-vérole inoculée à cette époque de la vie, c'est une grande & puissante raison de préférence en sa faveur, dans le choix du préservatif.
Il n'en est pas de même au-dessus de l'âge de dix ans. On sait qu'à cette époque la petite-vérole inoculée est en général une maladie très-bénigne; que la plupart des adultes n'ont que très-peu de fièvre & de boutons; que la plupart même n'en ont point du tout, & que la maladie n'est chez eux qu'une maladie locale de très-peu d'importance. La vaccine, au contraire, semble évidemment être beaucoup plus grave à cet âge qu'à tout autre. Sur soixante-sept personnes auxquelles le Dr. l'a inoculée, cinq seulement ont été exemptes de fièvre, & les soixante-deux autres en ont eu entr'eux tous 224 jours, ce qui fait pour chacun, l'un dans l'autre, 3,612, c'est-à-dire, plus de trois jours & demi.
\setcounter{page}{282}
L'âge le plus heureux pour l'inoculation de la vaccine paraît être de trois ans jusqu'à dix. Sur trente inoculés de cet âge, dix-sept n'ont point eu de fièvre, & les treize autres n'en ont eu entr'eux tous que 27 jours, c'est-à-dire, à-peu-près deux jours chacun, l'un dans l'autre. La petite-vérole inoculée, même à cette époque, qui est celle de toutes que préfèrent les Inoculateurs scrupuleux, n'atteint pas en général à ce degré de bénignité. Le nombre des malades exempts de fièvre, est bien moins considérable, & chez ceux qui en ont, elle est pour l'ordinaire d'une plus longue durée.
Mais ce n'est pas uniquement par le nombre des jours de fièvre qu'il faut comparer la bénignité des deux maladies. C'est encore par son intensité & par le plus ou moins de gravité des symptômes qui l'accompagnent. Nous n'avons aucune donnée précise sur l'intensité de la fièvre qu'ont eue les inoculés du Dr. Il n'en parle que d'une manière très-vague. L'histoire de la plus grande partie d'entr'eux se borne à une ou deux lignes. Et quant à la nature des symptômes, il n'y a que soixante-un malades dont il spécifie succinctement ceux qu'ils ont éprouvés. Voici ceux dont il fait mention. D'abord nous trouvons trente-quatre observations dans lesquelles il est question du mal de tête. C'est le symptôme dont il parle le plus fréquemment; vingt-trois malades se\setcounter{page}{283} font plaints d'un engorgement douloureux sous l'aisselle ; treize de maux de gorge ; onze de maux de reins ; dix d'agitation, d'angoisse ou d'inquiétude ; sept de vomissemens ; sept de convulsions & de spasmes ; quatre de douleurs dans les membres ; quatre de douleurs à l'épaule ou à la nuque ; trois de roideurs dans le bras ou à la nuque ; deux de diarrhée ; un de douleurs dans les entrailles ; un de pesanteur ; un d'affoupissement ; un de dégoût ; un d'altérations un de délire ; un de boutons dans la gorge ; un de maux d'yeux ; un d'enflure au visage ; & un de toux.
Il n'y a aucun de ces symptômes qu'on ne voie fréquemment dans la petite-vérole inoculée. L'engorgement douloureux des aisselles, ainsi que la roideur, & les douleurs du bras, de l'épaule ou de la nuque, paroissent appartenir à l'affection locale de l'inoculation, & dépendre de l'irritation des vaisseaux & des glandes lymphatiques dans le voisinage de l'incision. Ces symptômes sembleroient avoir eu plus d'intensité dans la vaccine qu'ils n'en ont dans la petite-vérole. L'affection locale elle-même semble aussi en avoir davantage. Car, indépendamment de l'aréole érysipélateuse qui entoure également l'incision dans les deux maladies, l'inoculation de la vaccine produit de plus fréquemment une grande rougeur, toujours érysipélateuse, & quelquefois plus ou moins livide:\setcounter{page}{284} qui s'étend à plusieurs pouces de distance autour de la piqûre. Sur les deux cents inoculés du Dr. il y en a cinq qui ont présenté ce symptôme, qu'on voit bien quelquefois dans la petite-vérole inoculée, mais beaucoup plus rarement\footnote{On voit fréquemment la fièvre éruptive de la petite-vérole inoculée produire fur tout le corps : une rougeur générale & fugitive, rougeur que Dimsdale appelle rash, & qui est pour l'ordinaire sans conséquence quoique parfaitement semblable à celle de la scarlatine. Je l'ai vue survenir deux fois dans le cours même de la suppuration, trois ou quatre jours après la cessation de la fièvre éruptive, & dans un de ces deux cas, elle reparut ensuite plusieurs jours après la déficcation. Mais il est infiniment rare que l'inflammation locale produite par l'inoculation prenne l'apparence d'un érysipele étendu. J'ai vu un enfant qui, le soir même du jour où je l'avais inoculé eut une rougeur très-vive de ce genre qui occupait tout le bras autour de la piqûre. L'enfant avait en même temps beaucoup de fièvre & quelques mouvements spasmodiques. Mais ces symptômes se dissipèrent complétement au bout de 24 heures, & l'inflammation locale se manifesta ensuite au cinquième jour, comme à l'ordinaire. La petite-vérole fut complète, mais très-bénigne, très-régulière & très-heureuse. (O)}. Au surplus l'auteur affirme que cette inflammation, quelqu'étendue qu'elle fût, a toujours été assez légère, soit dans ces deux cents premiers malades, soit dans tous ceux qui ont été inoculés depuis, pour ne donner aucune appréhension, & n'exiger aucun remède.\setcounter{page}{285} "J'en excepte, " dit-il, " un seul cas, où elle fut bientôt dissipée par l'application d'un peu d'acétite de plomb." Quant aux ulcères phagédéniques dont parle le Dr. Jenner, comme fréquens dans la vaccine naturelle, il paroît qu'ils sont fort rares dans l'inoculée. C'est ici la place de citer une observation assez intéressante qui explique peut-être cette différence. Le Dr. W., curieux de voir si l'exposition à l'air augmenteroit ou diminueroit l'effet du virus, & si la plaie prendroit un autre aspect lorsqu'elle ne feroit pas couverte par les habits, avoit inoculé en même temps au bras & à la main un jeune homme de dix-neuf ans. La différence fut en effet très-sensible. La tumeur de la main fut beaucoup plus étendue, d'une couleur beaucoup plus livide, & accompagnée de beaucoup plus d'inflammation que celle du bras.
Pour en revenir aux symptômes d'affection générale, le mal de tête, l'agitation, l'angoisse, l'inquiétude, les douleurs vagues, la pesanteur, l'assoupissement, le dégoût & l'altération, sont des symptômes de fièvre communs à toutes les maladies fébriles & qui ne paroissent point avoir été plus graves dans la vaccine qu'ils ne le sont fréquemment dans la petite-vérole inoculée.
Les maux de gorge sont un symptôme assez commun dans les fièvres éruptives, mais qu'on voit rarement dans la petite-vérole, & surtout dans la petite-vérole inoculée. Les observations\setcounter{page}{286} de notre auteur montrent qu'il est beaucoup plus fréquent dans la vaccine, sans augmenter cependant beaucoup la gravité de la maladie, puisqu'il ne paroît pas que dans aucun cas il ait exigé des remèdes, ou donné la plus légère inquiétude.
On peut en dire autant des maux de reins qui, dans la petite-vérole, sont toujours de mauvais augure, & annoncent pour l'ordinaire une éruption confluente, mais qui dans la vaccine ne paroissent avoir été d'aucune conséquence.
Les vomissements sont un symptôme très-ordinaire de toutes les maladies éruptives. Ils n'ont été ni plus fréquens ni plus graves dans la vaccine qu'ils ne le sont dans la petite-vérole inoculée.
Les spasmes & les convulsions sont des symptômes assez communs dans celle-ci. Quelquefois même ils ont un degré très-alarmant d'intensité, soit par leur violence, soit par leur durée, & l'on a vu des cas où ils sont devenus mortels, même avant l'éruption \footnote{Les convulsions de la petite-vérole passent communément pour n'être pas de mauvais augure. Cependant elles sont peut-être plus fréquemment mortelles qu'on ne l'imagine. Il y a quelque temps que je fus appelé auprès d'un enfant de 4 ans qui était depuis plusieurs heures sans aucune connaissance & dans une attaque très-forte de convulsions dont on ignoroit la cause, si ce n'est qu'il avoit depuis deux ou trois jours un peu de fièvre. Un Pharmacien du voisinage lui avoit donné sans succès quelques remèdes. Je lui en donnai d'autres plus actifs, & je lui fis sur le champ appliquer des vésicatoires aux jambes. Tout fut inutile. Il mourut dans la nuit. Deux ou trois jours après, un autre enfant du même âge & qui demeuroit dans la même maison, mais à l'étage au-dessus, prit tout d'un coup des convulsions du même genre, précédées de même d'une petite fièvre qui ne paroïssoit pas avoir rien de grave. Je fus appelé au premier moment des accidens. Je réussis à lui rendre la connoissance & à faire cesser les convulsions. Mais dès le lendemain il se manifesta une éruption de petite-vérole confluente & pétéchiale dont il mourut au neuvieme jour. Sans cet accident du second enfant, la nature de la maladie du premier auroit probablement été complètement ignorée; & il est possible qu'il meure ainsi beaucoup d'enfans avec des symptomes convulsifs qu'on n'attribue point à la petite-vérole, parce qu'ils deviennent mortels, avant qu'il y ait aucune apparence d'éruption. J'ai vu quelquefois la petite-vérole inoculée être précédée de convulsions terribles. Mais je ne les ai jamais vues devenir mortelles. Je fais cependant qu'elles l'ont été dans un petit nombre de cas; il y a quelques années que nous en eumes ici un exemple. Un enfant inoculé en mourut pendant la fièvre éruptive & avant l'éruption. Dans un autre cas plus malheureux encore, & qui fit beaucoup de bruit, les convulsions eurent lieu pendant la suppuration, & emportèrent la malade au bout de quelques heures. Mais dans ce dernier cas, on avoit imprudemment purgé l'enfant avant la défécation, & ce fut cette purgation qui parût provoquer les accidens. (O)}. Ces cas\setcounter{page}{287} malheureux font à la vérité infiniment rares mais on en a des exemples. Il ne paroît pas que la vaccine mette parfaitement à l'abri de pareils accidens, puisque sur les deux cents inoculés du Dr. sept en ont été atteints, & \setcounter{page}{288} qu'il ajoute que depuis l'impression de ses Tables, un enfant à la mammelle en est mort au onzieme jour de l'inoculation. Cependant ce symptôme ne paroit avoir été que léger dans ceux dont il nous donne l'histoire détaillée; & l'on ne peut pas nier que sur deux cents inoculés de la petite-vérole, pris au hazard, sur-tout s'il y en avoit comme ici plus du tiers au-dessous de l'âge d'un an, & plus de la moitié au-dessous de l'âge de trois ans, il n'y en eût au moins un aussi grand nombre qui en feroient atteints, & d'une maniere pour le moins aussi grave.
La diarrhée paroit moins ordinaire dans la vaccine que dans la petite-vérole, où elle termine très-fréquemment la maladie.
Enfin les douleurs dans les entrailles, les boutons dans la gorge, les maux d'yeux, l'enflure au visage, la toux & le délire, qui ne se sont manifestés que très-rarement dans la vaccine se voient bien pour le moins aussi fréquemment dans la petite-vérole inoculée.
En général il n'y a donc eu, ni dans la durée\setcounter{page}{289} des fymptômes fébriles qui accompagnent la vaccine inoculée, ni dans leur nature, ni dans leur intensité, aucune raifon de lui préférer la petite-vérole, dans le choix du virus à employer pour l'inoculation. Un examen attentif & détaillé des observations du Dr. femble, au contraire, devoir décider la queftion en faveur de la premiere de ces deux maladies, quoique moins bénigne peut-être qu'on ne l'avoit d'abord annoncée ; pour en bien juger cependant, l'auteur remarque avec raifon qu'il faudroit raffembler un nombre de cas de petite-vérole inoculée pris au hazard, égal à celui des inoculations de vaccine qu'il vient de décrire ; en faire des tables détaillées, femblables aux fiennes, & les comparer. C'eft ce qu'on n'a point encore fait.
Mais le point le plus effentiel à déterminer avec foin feroit de favoir s'il eft vrai, comme on l'a avancé, que la vaccine eft pour l'ordinaire exempte de boutons, & que ceux qu'a observés le Dr. dépendoient de caufes étrangeres à la maladie. C'eft furtout en celà que cette maladie auroit fur la petite-vérole inoculée un grand avantage, puifqu'enfin c'eft l'abondance de l'éruption qui eft l'accident le plus redoutable de la petite-vérole, tant inoculée que naturelle, & que fi l'on avoit un moyen sûr d'en garantir les malades, on ne la verroit prefque jamais fe terminer par la mort, ou avoir\setcounter{page}{290} des suites fâcheuses. Il est vrai que de pareils accidens font extrêmement rares dans la petite-vérole inoculée, comparativement à la petite-vérole naturelle. Mais il n'en est pas moins évident que ce feroit rendre un service éclatant à l'humanité que de trouver un moyen sûr, ou presque sûr, d'écarter de l'inoculation la possibilité d'une éruption abondante.
C'est ce que nous faisoit espérer le Dr. Jenner dans sa premiere brochure. C'est ce que confirment ses opérations postérieures. C'est ce qu'attestent tous les autres Médecins qui ont publié le résultat des inoculations nombreuses qu'ils ont faites en différens Comtés de la Grande-Bretagne. Tous s'accordent à dire que la vaccine est toujours ou presque toujours exempte de boutons ; & nous avons vu qu'elle a eu le même fort à Vienne, où le Dr. De Carro vient de l'introduire. Comment donc arrive-t-il que les observations du Dr. démentent cette assertion au point de présenter dans son premier Tableau 110 inoculés sur 200, & dans le second 194 sur 310, en tout, 294 sur 510, qui ont eu ailleurs qu'au bras, une éruption plus ou moins abondante.
Cette différence peut s'expliquer peut-être 1º. par le peu d'intervalle que le Dr. a mis pour l'ordinaire entre l'inoculation de la vaccine & celle de la petite-vérole. Nous en avons parlé suffisamment dans notre premier\setcounter{page}{291} Extrait. 29. Par le séjour que la plupart de ces inoculés ont fait dans l'Hôpital d'Inoculation où ils ont été constamment exposés à la contagion de la petite-vérole, soit par le contact immédiat des malades, soit par les miasmes, le pus desséché adhérent aux linges & aux meubles de la maison, ou autres foyers de contagion nécessairement plus accumulés dans cet endroit que dans tout autre. Voyez sur ce point les conjectures très-raisonnables du Dr. De Carro. 3°. Enfin, par le peu de précautions qu'a probablement prises l'Inoculateur pour se dépouiller lui-même des foyers semblables de contagion, qui ont pu, sans qu'il s'en doutât, s'attacher à ses habits & à sa personne. Car il est remarquable que pour réfuter une opinion du Dr. Jenner qui attribuoit à l'air de Londres l'éruption observée par le Dr. W., celui-ci affirme avoir vu presqu'aussi fréquemment des boutons sur les individus qu'il a inoculés à la campagne, jusqu'à la distance de 20 milles de Londres.
Ces boutons n'ont pas toujours suppuré; Plusieurs ont avorté. Mais pour l'ordinaire ils ont eu la plus grande ressemblance avec ceux de la petite-vérole inoculée, dans laquelle on remarque bien aussi que plusieurs des boutons qui surviennent avortent fréquemment, & ils ont fait exactement le même cours.
Sont-ce donc des boutons varioliques? Nullement.\setcounter{page}{292} Car le pus qu'ils ont fourni, effayé à plusieurs reprises tant par le Dr. que par le Dr. Jenner à qui il en avoit envoyé, a presque constamment produit l'affection locale propre à la vaccine, & très rarement celle que produit le pus variolique. Il semble donc qu'on doit les considérer comme une maladie hybride; car si cela étoit ainsi, cette maladie se transmettroit avec un caractère mixte; mais comme une simple modification de la vaccine produite par des exhalaisons, ou des foyers de contagion varioliques. Mais quoique la maladie qui en résultait pour l'ordinaire, autant qu'on en peut juger par l'affection locale, tous les caractères de la vaccine, le Dr. avoue cependant que dans certains cas, les phénomènes qui accompagnent l'inflammation locale de la vaccine, & qui la distinguent de celle de la petite vérole inoculée, tels que la limpidité permanente du fluide contenu dans la tumeur, la texture de la croûte par laquelle elle se termine, l'inégalité de sa circonférence, le poli de sa surface, sa dureté, sa couleur &c. se rapprochoient tellement de ceux qui accompagnent l'affection locale de la petite vérole, qu'on ne pouvoit pas distinguer ces deux affections l'une de l'autre; & il a observé que "lorsque la maladie perdoit ainsi" l'aspect qu'elle offre ordinairement dans la" place inoculée, ses effets sur la constitution" étoient beaucoup plus graves qu'ils ne le sont\setcounter{page}{293} lorfque la tumeur conferve bien tous fes caracteres diftinctifs. Cet aveu important femble prouver que le virus, foi-difant vaccin, s'eft pourtant quelquefois trouvé être purement variolique.
Un autre fait très-extraordinaire & qui prouve que la circonftance des boutons eft une circonftance acceffoire, indépendante de la nature vaccine du virus avec lequel on inocule, une fimple modification dans les effets de ce virus produite par les exhalaisons ou les foyers de contagion variolique, c'eft que le même virus, qui, employé par le Dr. avoit produit des boutons fur plus de la moitié de fes inoculés n'en produifit point entre les mains du Dr. Jenner , quoiqu'il s'en servit avec succès pour inoculer 20 perfonnes, non plus qu'entre celles d'un autre praticien de la campagne, pour l'inoculation de 140 individus qui tous eurent la vaccine fans aucune éruption, tandis que d'un autre côté celui que le Dr. Jenner envoya de Berkley au Dr. W., produifit des boutons quand celui ci voulut l'employer, quoiqu'il n'en eût jamais produit entre les mains de celui-là. Il semble donc que le Dr. accoutumé depuis long-temps à manier tous les jours le pus variolique portoit fur fa perfonne & communiquoit à fes inoculés vaccins , le germe des boutons, tandis que ceux qui étoient inoculés par d'autres praticiens n'en avoient point.
\setcounter{page}{294}
Quoiqu'il en foit, on peut remarquer fur le tableau qui indique le nombre qu'on ont eus les inoculés des deux sexes dans tous les âges de la vie:
1. Que les enfans de 3 à 10 ans ont été ceux qui ont paru le moins fusceptibles de cette modification, puisque sur 30 inoculés de cet âge, il y en a eu 19 qui n'ont point eu de boutons; qu'après eux, les petits enfans au-dessous d'un an en ont été les plus exempts dans la proportion de 41 sur 68; ensuite ceux de 1 à 2 ans dans la proportion de 10 sur 19; puis ceux de 2 à 3 ans dans la proportion de 7 sur 16; & qu'enfin ceux de 10 ans & au-dessus en ont été incomparablement plus affectés que les autres, puisqu'il n'y en a eu que 13 sur 67 qui n'ayent point eu de boutons. En réduisant tous ces rapports en figures décimales, & en ayant égard dans ce calcul à la différence des sexes, on peut exprimer la non-fusceptibilité de boutons dans les différens âges comme fuit:

\comment{table}
\begin{tabular}{llll}
 & Mâles. & Femelles. & Total. \\
De 3 à 10 ans & 0,769 & 0,529 & 0,633 \\
Au dessous d'un an & 0,600 & 0,605 & 0,602 \\
Entre 1 & 2 ans & 0,500 & 0,538 & 0,526 \\
Entre 2 & 3 ans & 0,375 & 0,500 & 0,427 \\
Au-dessous de 10 ans & 0,170 & 0,230 & 0,194 \\
\end{tabular}

On voit par là que l'avantage n'a pas été en faveur des filles, de l'âge de 3 à 10 ans;\setcounter{page}{295} mais qu'à cela près elles ont suivi presque la même progression que les garçons. Or en cela, la petite-vérole paraît différer beaucoup de la petite-vérole inoculée, dans laquelle le plus grand nombre des individus sans boutons, se trouve pour l'ordinaire au-dessus de l'âge de 10 ans, & le plus petit au-dessous de 3 ans\footnote{Ceci n'est pas parfaitement exact. Il m'a paru que parmi les inoculés au-dessous de l'âge de 3 ans, & particulièrement parmi ceux qui font encore à la mamelle, on voit un plus grand nombre de cas extrêmes, qu'à toute autre époque de la vie. Je suppose, par exemple, que sur cent inoculés de l'âge de 3 à 10 ans, il y en ait 15 qui n'aient point de boutons, 25 qui en aient plus de 100, & 60 qui en aient entre 1 & 100; on en verra peut-être sur un nombre égal en très-bas âge, 15 qui n'auront point de boutons, 35 qui en auront un très-grand nombre, & 40 seulement qui en auront peu. En compensant le danger de prendre la petite-vérole naturelle avec celui d'avoir une petite-vérole inoculée abondante, l'âge qui m'a paru préférable pour l'inoculation des petits enfans est celui de cinq à six mois. Si on les inocule plus jeunes, la petite-vérole est sujette à manquer, ou à être confluente. Plus tard, elle se complique avec le travail de la dentition & devient plus convulsive. (O)}. Mais il n'y a aucune époque de la vie où il y ait dans la petite-vérole inoculée, un aussi grand nombre d'individus, exempts de boutons, comparativement à ceux qui en ont. Dans l'âge le plus favorable à cet égard, à peine peut-on compter\setcounter{page}{296} le quart des inoculés qui en soient exempts.
2. En comparant de même les différens âges pour le nombre des boutons, nous trouvons que, sur 57 garçons qui ont eu des boutons, 20 en ont eu moins de dix; 22 entre dix & cent; & 15, entre cent & mille; & de ces 15 derniers, 6 avoient moins de 3 ans, & 9 plus de 10. Sur 53 filles qui ont été dans le même cas, 19 en ont eu moins de dix, 20 de dix à cent & 14 de cent à mille. Sur ces 14, 4 avoient moins de 3 ans, 2 entre 3 & 10, & 8 plus de 10 ans. En voici le tableau:

\comment{table}
D'un à 10 boutons | De 10 à 100 boutons | De 100 à 1000 boutons
M. F. | M. F. | M. F.
Au-dessous de 3 ans | 7. 10 | 7. 11 | 6. 4
De 3 à 10 ans | 2. 5 | 1. 1 | 0. 2
Au dessus de 10 ans | 11. 4 | 14. 8 | 9. 8
Total'. | 20. 19 | 22. 20 | 15. 14
| 39 | 42 | 29.

Il résulte encore delà relativement au sexe, que les filles au-dessus de 3 ans ont eu un peu plus de boutons que les garçons; & relativement à l'âge, que celui auquel les uns & les autres en ont eu le moins a été de 3 à 10 ans, & celui auquel ils en ont eu le plus, au-dessus de l'âge de 10 ans. Cette dernière différence peut être exprimée en décimales par les nombres, 222\setcounter{page}{297} pour les trois premières années de la vie, 0,181 pour les sept suivantes, & 0,314 pour les années au-dessus.
3. Quoique le nombre des boutons ait été généralement plus considérable, lorsque la fièvre éruptive a duré plus long-temps; cependant, il ne lui a pas été exactement proportionné. Car, comme nous l'avons vu, sur 52 inoculés qui n'ont point eu de fièvre; il n'y en a eu que 43 qui n'aient point eu de boutons. L'un des 9 autres en a eu trois cent. C'est ce qu'on ne voit jamais dans la petite-vérole inoculée, où l'éruption, si elle a lieu, est constamment précédée de fièvre. D'un autre côté, sur 89 inoculés qui n'ont point eu de boutons, il en est qui ont eu jusqu'à 4 & 5 jours de fièvre. Voici le tableau des rapports qu'il y a eu entre les jours de fièvre & le nombre des boutons.

\comment{table}
Point de fièvre Point de bout. 1-9 10-99 100-1000 Total
1 jour 43 8 6 1 52
2 . . 18 8 7 2 35
3 . . 16 11 7 2 36
4 . . 5 5 14 6 30
5 . . 4 6 9 8 27
6 . . 3 1 4 3 11
7 . . — — — 5 5
8 . . — — 1 — 1
Total 89 39 42 29 199

4. Une question bien plus essentielle à examiner,\setcounter{page}{298} feroit de favoriser, si le choix du virus est de quelque importance; si, lorsqu'on le prend sur un inoculé qui ait peu de fievre & peu de boutons, on obtient en général une maladie plus douce & plus légere que lorsqu'on le prend sur un inoculé plus griévement malade. On fait que dans la petite-vérole inoculée, ce choix est tout-à-fait indifférent; que le virus d'une petite-vérole confluente ne produit pas une maladie plus grave que celui d'une petite-vérole très-bénigne, & réciproquement qu'une petite-vérole confluente & très-mauvaise, doit souvent son origine à un virus pris d'un malade très-légèrement affecté. Il feroit fort extraordinaire qu'il en fût autrement de la vaccine. C'est pourtant l'opinion du Dr. "Je suis," dit-il," porté à croire, que si l'on choisit bien la" matiere de la vaccine, & qu'on ne l'inocule" que d'après des malades chez lesquels la maladie se présente sous un aspect plus benin," les résultats de l'inoculation feront beaucoup" plus favorables que ceux que j'ai obtenus." Car la matiere prise d'un malade qui n'avoit" eu ni fievre ni éruption, a toujours donné" une maladie plus bénigne que celle que causoit le pus pris sur un malade affecté d'une" maniere plus grave. C'est ce qu'on peut voir" en examinant les tables avec attention."
Nous avons été curieux de vérifier cette assertion. Nous avons pour cet effet dressé le tableau\setcounter{page}{299}bleau suivant, dont la premiere colonne indique l'origine du virus, employé pour chacun des inoculés. A gauche est indiqué le nombre de jours de fievre, et à droite le nombre des boutons qu'avoit eus le malade sur lequel on l'avoit pris, en observant que 0 veut dire, point de boutons, 1. d'un à 9; 10, de 10 à 99; et 100, de 100 à 1000. La seconde colonne indique le nombre d'individus inoculés avec ce virus. La troisieme, à gauche, le nombre de ceux qui ont eu de la fievre, et à droite le nombre total des jours de fievre qu'ils ont eus entr'eux tous. La quatrieme, à gauche, le nombre de ceux qui ont eu des boutons, et à droite le nombre total de boutons qu'ils ont entr'eux tous. Il faut observer de plus, que des 199 inoculés (car nous ne comptons pas la jeune fille sur laquelle l'inoculation a manqué) 7 ont été inoculés directement d'après une vache atteinte de la maladie, 3 d'après une seconde vache qui avoit été elle-même inoculée, et 5 d'après une laitiere qui avoit pris la maladie naturellement d'une troisieme vache.
\setcounter{page}{300}
\comment{table}
I.	II.	III.	IV.
ire. vache	5.	5. 17	4. 203
. de vache	3.	3. 15	3. 755
Laitiere.	5.	3. 9	4. 703
0. 0	15.	11. 28	3. 241
1. 0	3.	3. 10	3. 293
2. 0	17.	14. 46	10. 786
3. 0	8.	5. 6	4. 89
1. 1	22.	14. 51	10. 690
2. 10	1.	1. 5	1. 4
2. 1000	1.	0. 0	0. 0
3. 10	5.	3. 10	4. 167
3. 1	13.	6. 15	7. 158
4. 10	19.	15. 45	13. 1288
4. 100	17.	14. 40	11. 850
5. 10	10.	7. 26	6. 110
5. 100	2.	0. 0	0. 0
6. 100	34.	28. 64	18. 1852
7. 10	4.	4. 15	2. 166
8. 1000	12.	11. 21	6. 498
Total.	199	147. 423	110. 8857

On voit clairement par ce Tableau que l'opinion de l'auteur est mal fondée, & qu'il n'y a eu aucun rapport constant de bénignité entre la maladie de celui qui donnoit le virus, & la maladie de celui qui le recevoit, ni relativement à la fièvre, ni relativement aux boutons, puisque le virus pris dans son plus grand degré de bénignité & sur un malade qui n'avoit ni fièvre, ni boutons, a pourtant produit 28 jours de fièvre, & 241 boutons répartis entre 15 inosulés,\setcounter{page}{301} tandis que, pris sur un malade qui avoit eu 8 jours de fièvre & 1000 boutons, il n'a produit que 21 jours de fièvre & 498 boutons entre 12 inoculés; & puisque dans tous les degrés intermédiaires on remarque la même inégalité.
Il en est de même d'une autre assertion de l'auteur; c'est que le pus pris sur les boutons, produit constamment une maladie plus grave que si on le prenoit au bras. Cette assertion est erronée; car une personne inoculée avec le pus des boutons d'un malade qui avoit eu 6 jours de fièvre & 530 boutons, n'eut qu'un jour de fièvre & point de boutons; une autre inoculée des boutons d'un malade qui avoit eu 2 jours de fièvre & 1000 boutons, n'eut ni fièvre ni boutons. Il est vrai que sur 17 personnes inoculées de cette manière, nous en trouvons 16 qui ont eu de la fièvre, & 14 qui ont eu des boutons; & le Dr. affirme à cette occasion, d'après des observations postérieures, que sur 62 personnes inoculées avec le pus d'un bouton, 57 eurent une éruption; & que ceux qui reçurent ensuite la maladie de ces 57 malades eurent aussi des boutons dans la même proportion. Mais outre que, comme on vient de le voir, le fait est loin d'être constant, ne peut-on pas soupçonner delà que c'est peut-être de son origine variolique que le pus des boutons tient cette propriété de produire plus constamment de la fièvre & une éruption?
\setcounter{page}{302}
D'après cet exposé des observations du Dr. relativement aux effets du virus vaccin, il paroît évidemment, qu'en somme, & malgré les imprudences de l'auteur, il produit une indifpofition plus bénigne & beaucoup moins grave que la petite-vérole inoculée. L'auteur adopte lui-même cette conclusion. Cependant, ajoute-t-il," comme 3 ou 4 malades fur 500 ont" été réellement en danger, & qu'il en eft mort" un, je fuppofe que par la fuite il fe trouvat" que fur 500 inoculés de la vaccine il en meurt" un, affurément je ne voudrois point introduire dans mon hôpital cette nouvelle manière" d'inoculer. Car parmi les 5000 personnes qui" ont été inoculées de la petite-vérole dans ces" derniers tems, il n'en eft mort qu'un fur 600.
Nous laiffons à nos lecteurs le foin d'apprécier la validité de ce motif. Nous remarquerons feulement que ce n'eft pas fur 500, mais fur 510 inoculés de la vaccine, qu'il en eft mort un entre les mains du Dr. (& c'eft le feul cas mortel\footnote{Dans le difcours préliminaire que le Dr. Aubert a ajouté à fa traduction , il dit qu'on lui a écrit de Londres que depuis la publication de l'ouvrage du Dr. c'est-à-dire, depuis le 19 mai 1799, il est mort à Londres une autre personne de la vaccine. Mais on ne lui a transmis aucun détail sur cet accident, & nous avons lieu de douter de sa réalité, parce que par des lettres de très-fraiche date, un correspondant digne de foi nous assure qu'il n'est mort en tout jusqu'à présent qu'un seul inoculé de la vaccine. (R)} qui foit parvenu à notre connoissance\setcounter{page}{303} sur plus de 2000 inoculations faites avec ce virus tant à Londres que dans tout le reste de la Grande-Bretagne). Les deux maladies se trouvant donc à cet égard pour le moins au niveau l'une de l'autre, peut-être pensera-t-on qu'il vaut la peine de prendre en considération dans le choix du virus destiné à l'inoculation, l'avantage d'éteindre à la longue le terrible fléau de la petite-vérole, en donnant pour cet objet la préférence à celle des deux maladies qui n'est pas contagieuse.
3. Mais cet avantage de la vaccine est-il bien avéré? C'est la 3e question que nous nous sommes proposés d'examiner. Nous ferons briefs sur ce point, puisque dans tout l'ouvrage du Dr. on ne trouve que le paragraphe suivant qui soit relatif à cette question. Un des avantages majeurs qu'on attribuoit à la vaccine étoit celui-ci: on prétendoit qu'elle n'étoit pas contagieuse, & que les émanations ou effluves des personnes qui en étoient attaquées ne la communiquoient pas à d'autres personnes: cela est vrai lorsque la maladie ne dépasse pas les bornes de la partie inoculée; mais lorsqu'elle produit de nombreux boutons sur toute la surface du corps, les exhalaisons qui en émanent infectent les personnes qui entourent le malade, & leur communiquent la vaccine. J'ai eu dernièrement occasion d'observer deux cas semblables; dans l'un les\setcounter{page}{304} symptômes furent graves, l'éruption fut confluente; dans l'autre la maladie fut très-modérée, & il n'y eut que très-peu de boutons. Il est aisé de voir que cette objection tombe d'elle-même, s'il est vrai que les boutons observés par le Dr. ont eu une origine variolique, & que d'autres Inoculateurs de vaccine n'en aient jamais vu.
Nous terminerons cet Extrait par un petit fait assez singulier rapporté par l'auteur, & qui a échappé à l'exactitude du traducteur. Après avoir parlé de quelques expériences du Professeur Coleman pour donner la vaccine à des vaches en leur inoculant le javart, expériences qui n'eurent aucun succès, le Dr. ajoute en note qu'on essaya d'inoculer à une vache le virus vaccin pris sur une autre vache, ainsi que du virus variolique. Ni l'une ni l'autre de ces inoculations ne prit; mais la même vache prit fort bien la vaccine lorsqu'on l'inocula avec le premier de ces deux virus régénéré dans l'espèce humaine\footnote{Comme ce fait est raconté d'une manière un peu obscure, voici les propres expressions de l'auteur en anglais: Mr. Coleman caused one of his cows to be inoculated in its teats with Cow-Pox matter, and with that taken from a variolous pustle without effect; but the former, after being regenerated in the human subject, produced the disease in the Cow. (R)}.
\setcounter{page}{305}
Avant d'abandonner l'ouvrage du Dr. qu'il nous soit permis de dire encore un mot d'un excellent Discours préliminaire dont le Dr. Aubert a enrichi sa traduction. Il y retrace d'une manière très-nette l'histoire de la vaccine & des opinions successivement admises & rejetées par les Médecins Anglais sur l'origine, la nature & les effets de cette singulière maladie. Il fait lui-même sur les principaux points de cette recherche des observations judicieuses, & raconte en détail l'histoire de trois inoculations de vaccine faites sans succès à Paris. Nous regrettons seulement qu'il semble donner quelque créance à la possibilité d'avoir deux fois la petite-vérole. L'exemple qu'il en cite parait trop suspect pour pouvoir être mis en opposition à un fait aussi universellement reconnu que celui sur lequel repose depuis tant d'années l'utilité de l'inoculation ; & si quelqu'Inoculateur imprudent, après avoir fait usage du pus de la petite-vérole volante au lieu de celui de la vraie petite-vérole, n'a pas reconnu son erreur à l'apparition & à la marche de l'éruption qui en résulte, il suffit que les parents eux-mêmes l'aient soupçonné, pour rendre le fait de la vraie petite-vérole inoculée ensuite avec succès, très-insignifiant\footnote{J'ai eu connaissance de plusieurs cas dans lesquels une pareille méprise a failli avoir des suites funestes. J'en citerai deux, 1°. Le Dr. Sylvestre avoit}.
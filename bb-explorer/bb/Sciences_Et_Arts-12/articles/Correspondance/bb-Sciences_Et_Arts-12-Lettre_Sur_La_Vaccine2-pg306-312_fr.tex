\setcounter{page}{306}
\chapter{Correspondance.}
\section{AUX AUTEURS DE LA BIBLIOTHÈQUE BRITANNIQUE.}
\subsection{Vienne le 19 octobre 1799.}
J'ai appris avec peine, par une voie indirecte, que les premiers essais d'inoculation vaccine faits par Mr. le Dr. Odier, n'aient pas réussi, & je m'applaudis d'autant plus de n'en avoir pas attendu le résultat, pour vous envoyer une seconde provision de matière, que vous devez\setcounter{page}{307} avoir reçue depuis long-temps, incluse dans ma seconde lettre sur la vaccine. — Je ne doute point, ainsi que je vous le disois dans mes lettres précédentes, que son manque de succès ne soit entièrement dû à la méthode dont il a fait usage, de délayer la matiere vaccine avec de l'eau, & vous aurez vu certainement que celle de l'insertion d'un fil, quoique plus pénible, est beaucoup plus sûre.
J'ai eu dernièrement une occasion rare de vérifier un fait avancé par le Dr. Jenner & supporté par quelques observations, savoir, qu'une\setcounter{page}{308} personne qui a eu la petite-vérole, est encore susceptible de la vaccine. Le Dr. Pearson en a recueilli quelques-unes de semblables; & parmi les praticiens qui connaissoient la vaccine dans différens Comtés, les uns ont répondu à ses questions sur ce sujet, par l'affirmative, d'autres, également expérimentés, par la négative. Le Dr. Woodville en donne un exemple positif dans sa 27°. observation.
Quelque suffisans que paroissent ces témoignages, la chose en elle-même est si extraordinaire, qu'elle méritoit bien d'être vérifiée ailleurs qu'en Angleterre.
Mr. le Comte Mottet, âgé de 51 ans, a eu la petite-vérole naturelle à l'âge de 6 ans; sa mere qui vit encore, m'a raconté l'histoire de sa maladie de manière à ne laisser aucun doute sur la vérité de ce fait, & ajoute que la sœur du Comte, qui eut la petite vérole en même temps que son frere, en est encore marquée. Par amour pour la science & pour décider une question qui lui paroissoit intéressante, il me pria de l'inoculer. Un autre motif l'y encourageoit; étant décidé à faire inoculer sa fille au printemps prochain, il vouloit s'assurer par lui-même des sensations qu'on éprouve pendant cette maladie, si tant est qu'on puisse lui donner ce nom.
C'est un homme d'une santé excellente. La matière vaccine fut prise le 2 octobre du bras de Mr. le Comte de W., dont je vous ai parlé\setcounter{page}{309} dans ma lettre précédente, & inoculée de la maniere ordinaire; le neuvieme jour de la vaccine, une croûte très-ferme étoit déjà formée, & néanmoins la matiere étoit encore limpide, mais en si petite quantité que la pointe de ma lancette en étoit à peine humectée. Ne croyant pas qu'une quantité si petite pût produire la vaccine dans une personne qui se trouvoit dans des circonstances, en apparence, aussi, peu favorables à sa production, je l'inoculai par trois piqûres. Voici le journal de la maladie:
5 Octobre. Les 3 piqûres sont couvertes d'une matiere épaisse, les bords sont très-durs au toucher & les aréoles érysipélateuses ont au moins deux pouces de diametre. Il rapporte, que le soir même de l'inoculation, qui se fit à une heure après midi, il s'étoit déjà apperçu d'une tension au bras & de beaucoup d'inflammation à l'entour des piqûres. D'ailleurs il se porte bien.
6° Les pustules augmentent. La circonférence de l'effloreScence est beaucoup moindre que hier; mais il a observé chaque jour que l'inflammation diminuoit pendant la nuit & augmentoit le jour, surtout vers le soir. L'effloreScence est plutôt pourpre que rouge.
7° Les pustules augmentent beaucoup; une croûte commence à se former au milieu de deux d'entr'elles; l'inflammation est considérable. La chemise est très-abondamment imprégnée de\setcounter{page}{310} matière limpide. Il ressent de la douleur au-dessus des épaules. Les bords sont très-durs au toucher.
10°. Je ne l'ai pas vu pendant 3 jours. Il me croit pas avoir eu de la fièvre, mais il a éprouvé constamment une tension douloureuse au bras où sont les deux pustules, & un certain état de mal-aise auquel il n'est point sujet. Les aréoles ont diminué; les croûtes s'épaississent & il y a beaucoup de matière limpide.
14°. Les croûtes ont au moins un pouce de diamètre. Elles conservent toujours leur rondeur; elles donnent beaucoup de matière; la tension du bras continue, d'ailleurs il se porte bien. Il n'a paru aucune pustule sur le reste du corps.
19°. Les croûtes sont sèches; l'inflammation a disparu, la tension a cessé.
Voilà donc une observation aussi bien constatée que l'on puisse la désirer, qui prouve la possibilité d'avoir la vaccine après avoir eu la petite-vérole.
L'expérience nous démontrera par la suite, (si l'on trouve des personnes qui consentent à laisser inoculer, elles-mêmes, ou leurs enfants) si la vaccine est plus bénigne après avoir eu la petite-vérole que pour la première fois. Il est vrai que la vaccine inoculée est tellement douce que l'on ne conçoit pas comment\setcounter{page}{311} elle pourroit l'être davantage; cette différence dans la force des symptômes ne peut gueres s'observer que dans les pays de laiteries, où la vaccine se communique par le contact immédiat de la vache & de celui qui la traite. Dans le cas du Comte M. je serois plutôt tenté d'en tirer une conclusion contraire, car l'inflammation, la grosseur des pustules, l'étendue de croûtes, l'abondance de la matiere, étoient au moins le double de celles du Comte de W. d'où la matiere a été prise. Dans leur état de pustules, la ressemblance avec la plus grosse de celles qui sont représentées dans la premiere gravure du Dr. Jenner, étoit singulierement frappante.
Les observations des Médecins Anglois & celle-ci me paroissent suffisantes pour encourager des expériences d'inoculation vaccine, dans les cas où l'on auroit des raisons d'espérer qu'une fievre légere pourroit faire dans l'état d'un malade, quelque changement avantageux, ainsi que l'a suggéré le Dr. Jenner, pag. 70.
La vaccine n'étant jamais contagieuse que par inoculation, il est évident que la possibilité de l'avoir une seconde fois, & après la petite vérole, ne doit point être un obstacle à l'introduction de cette méthode.
Comme il est plus difficile de recueillir de la matiere vaccine avec la pointe d'une lancette, que cela ne le paroist d'abord ; je fais coudre un morceau\setcounter{page}{312} de linge sous la partie de la chemise, qui est en contact avec la pustule; de cette manière, le linge en est fortement imprégné.
Quoique la matière, dont le morceau de linge ci-joint est imprégné, paroisse épaisse & jaune, elle n'acquiert, cependant, cette apparence que lorsque le linge en est saturé, & qu'elle s'est desséchée. Si on l'eût observée sur la pointe d'une lancette dans son état de liquidité, elle auroit été très-claire.
Le bras du Comte Mottet, m'ayant fourni une grande quantité de matière vaccine très récente, je vous en envoie, ci-joint: c'étoit d'ailleurs la vaccine la mieux caractérisée, de toutes celles que j'ai été dans le cas de traiter\footnote{J'ai fait usage des quatre envois de matière vaccine dont il est ici question. Je n'ai point eu encore le bonheur de réussir, quoiqu'à l'exception de la première fois, je me fois exactement conformé à la manière prescrite par le Dr. D. Ni la dilution du virus, ni l'inoculation avec le fil trempé dans l'eau, le fil sec, ni avec le vésicatoire, ni avec la lancette, ni en plaçant très-soigneusement le fil dans l'incision, & en l'y assujettissant pendant quatre ou cinq jours par un sparadrap & un bandage, aucune de ces méthodes, dis-je, n'a eu le moindre succès. Le dernier fil employé avoit été imprégné à Vienne le 23 vendémiaire. Le 17 brumaire j'ai inoculé avec ce fil un enfant de cinq mois, \& ensuite le 23 brumaire un homme de 45 ans, qui par des motifs semblables à ceux du Comte voulut qu'on en fît un second essai sur lui-même, quoiqu'il eût eu la petite-vérole dans son enfance. Je les inoculai en même temps l'un \& l'autre avec un fil bien imprégné à Londres le 17 messidor dernier par le Dr. Pearfon; c'étoit un bout du même fil employé à Vienne avec succès pour le Comte Je ne puis pas comprendre à quoi tient la non réussite de toutes ces tentatives. Je vais encore effayer d'autres moyens. (O)}. J'ai l'honneur d'être, &c.
J. DECARRO, D.

\setcounter{page}{163}
\chapter{Correspondance}
\section{Aux Auteurs De La Biblothéque Britannique}
\subsection{Vienne le 11 septembre 1799.}
Quoique je n'aie pas encore reçu l'ouvrage du Dr. Pearson, ni celui du Dr. Woodville, je fais par une lettre du premier & par plusieurs autres du Dr. Marcet, notre ami & compatriote, qu'un des faits qui excitent le plus l'étonnement des médecins & du public Anglais est, que malgré que dans toutes les inoculations de vaccine faites à Berkeley par le Dr. Jenner, il n'ait jamais paru de pustules qu'à la partie inoculée, les Drs. Pearson & Woodville ont remarqué que le virus envoyé à Londres par le Dr. Jenner y produisait fort souvent des pustules sur le reste du corps; & que dans les cas où il y avait des pustules, la maladie était contagieuse ; tandis que dans ceux où elle suivait le cours décrit par le Dr. Jenner, c'est-à-dire, où elle ne produisait de pustules qu'à l'endroit de l'insertion, elle ne l'était pas.
Vous savez sans doute, que les expériences\setcounter{page}{164} très-nombreuses de ces deux médecins de Londres ont été faites dans l'hôpital appelé Inoculation & small Pox hospital. Le nom seul de cet établissement me paroît résoudre ce problème, & il me semble, au moins, que ce n'étoit pas le lieu que l'on auroit dû choisir pour faire des expériences qui devoient contribuer à fixer l'opinion publique sur cette nouvelle méthode ; puisqu'il est impossible de ne pas croire que les salles, les lits, les meubles & l'air même de cet hôpital ne fussent suffisamment infectés de miasmes varioliques, pour contrarier l'effet du virus vaccin, ou peut-être pour faire de cette maladie, sinon une vraie petite-vérole, du moins une maladie mixte qui ne ressemblât plus à la vaccine originale du Dr. Jenner.
Ces soupçons ont été presque changés en certitude par une lettre du Dr. Marcet qui me dit : "J'ai vu à l'hôpital d'inoculation des malades de vaccine & de petite-vérole ordinaire, dont les pustules vües & comparées à côté les unes des autres étoient tellement semblables que personne n'eût pû se vanter de les distinguer à coup sûr." — Puisque ces différens malades étoient à côté les uns des autres, pourquoi s'étonner que la maladie ne foit plus la vaccine du Dr. Jenner ? Cette remarque me paroît de la plus grande importance 1°. parce que si en effet la vaccine\setcounter{page}{165} étoit accompagnée de pustules sur le reste du corps comme la vraie petite-vérole, & que ces pustules fussent contagieuses, l'avantage de cette découverte se réduiroit à fort peu de chose.
2°. Comme il est vraisemblable que ces médecins auront envoyé dans plusieurs endroits de l'Angleterre & même de l'Europe, des fils imprégnés dans cet hôpital, il me semble qu'avant d'admettre implicitement les rapports qui pourront nous venir de différens côtés, il conviendra de prendre des informations exactes sur la généalogie de la matiere dont on se fera servi.
Une autre circonfiance moins facile à expliquer, est que les fils imprégnés à Londres & renvoyés à Berkeley y ont constamment produit une vaccine sans pustules sur le reste du corps. Mais ces fils étoient-ils pris des malades de Londres qui n'avoient eu de pustules qu'aux incisions, ou de ceux qui en avoient sur le reste du corps ? C'est ce que les mots du Dr. Marcet ne me semblent pas expliquer clairement; les voici ; "pourquoi la matiere qui produisoit" uniformément à Londres un grand nombre" de pustules, n'en a-t-elle plus produit en" Gloucestershire entre les mains du Dr. Jenner."
Du reste, l'on me mande de Londres que la proportion des cas où il n'y a point eu de pustules, est de beaucoup plus grande que celle des cas où il y en a eu.
\setcounter{page}{166}
Quoiqu'il en soit, c'est une raison de plus pour continuer avec zèle & persévérance des expériences aussi intéressantes; surtout en prenant toutes les précautions possibles pour s'assurer des circonstances qui ont accompagné les inoculations dont est résulté la matière qu'on employera. Or comme celle dont je me suis servi, n'a produit dans 5 inoculations d'autres pustules que celle de l'incision, je continuerai certainement à me servir de celle qui en est résulté, & c'est de celle-là dont je vous envoie par la présente \footnote{J'ai fait usage de ce virus le jour même où je l'ai reçu. J'ai inoculé pour la troisieme fois l'enfant de 8 ans que j'avois déjà inoculé deux fois avec le premier virus. Mais cette troisieme inoculation a manqué comme les deux premieres; & les parens impatients, ont desiré que je lui inoculasse enfin la petite-vérole. Il l'a eue fort heureuse. Il est vrai que s'il n'a pas pris la vaccine, cela vient peut-être de ce que ne soupçonnant point alors qu'il y eût aucune raison de se défier de ma méthode ordinaire, je l'avois inoculé par l'incision, & en délayant le virus. (O)}. J'accepte avec reconnaissance l'offre que vous me faites de m'en envoyer les résultats.
J'ai la plus grande espérance d'obtenir la permission d'inoculer une douzaine de personnes à l'Hôpital général de cette ville; Mr. le Conseiller Frank, qui en est le Médecin directeur,\setcounter{page}{167} m'a promis de faire tout ce qui dépendroit de lui pour encourager des expériences auffi utiles à l'humanité.
En attendant, que je vous communique un plus grand nombre d'inoculations, je vous dirai en peu de mots ce que j'ai fait depuis celles dont je vous ai fait part dans ma premiere lettre à laquelle vous avez bien voulu faire un accueil fi favorable.
J'ai inoculé 4 autres enfans, dont deux feulement ont pris la vaccine & l'ont eue, à très-peu de variations près, de la même maniere que mes enfans. Malheureusement pour le bien de la chofe en général, les parens de ces deux enfans font perfuadés qu'ils font parfaitement à l'abri de la contagion variolique; par conféquent, je n'efpere pas pouvoir faire l'Expérience d'une feconde inoculation.
Les deux autres enfans ont été inoculés avec de la matiere délayée, & ne l'ont pas prife. Cependant les parens fachant que l'inoculation de la petite-vérole ordinaire ne réüffit pas toujours la premiere fois, n'y ont rien vu de décourageant & laifferont inoculer encore une fois leurs enfans avec la matiere fraîche & liquide, qui proviendra, je l'efpere, de deux inoculations que j'ai faites le 9 de ce mois. — C'eft à caufe du défaut de réüffite dans ces deux cas, que je regarde comme fort important, d'entretenir autant que poffible une fourçe de\setcounter{page}{168} vaccine à laquelle on puisse puiser plus sûrement; c'est-à-dire que si l'on a plusieurs enfans à inoculer, on devroit les inoculer les uns après les autres, plutôt que tous à la fois, afin d'avoir toujours une pustule fraîche.
J'aurois attendu pour vous écrire, le résultat de ces deux inoculations qui feront suivies de plusieurs autres, si la crainte que vos fils ayent manqué, ne m'avoit engagé à ne point perdre de temps pour vous en envoyer d'autres. D'ailleurs comme la saison où il n'est gueres d'usage d'inoculer, s'avance à grands pas, je craindrois que vous ne trouvassiez pas si facilement des sujets d'expériences.
Il vient de paroître à Vienne une traduction latine des deux ouvrages du Dr. Jenner. L'auteur a pris la liberté d'y ajouter l'histoire de l'inoculation des enfans du médecin de Vienne dont je vous ai parlé dans ma premiere lettre, & des miens.
Je prens cette occasion de vous assurer que c'est absolument à notre insçu & même contre ma défense formelle. Outre les raisons qui me font trouver son procédé extraordinaire, je m'étois fait un plaisir de vous faire hommage de mes premiers essais, persuadé que mon exemple ne pourroit qu'être suivi à Genève, & que je ne pouvois mettre en meilleures mains les moyens de vérifier & de constater cette intéressante découverte.
\setcounter{page}{169}
P. S. Au départ de cette Lettre les bras des deux enfans inoculés, le 9 du courant, paroissent annoncer que la vaccine a pris.

\subsection{Vienne le 18 sept. (2e. jour complém. an VII.)}
.....Le résultat que j'ai obtenu dans trois inoculations où j'ai délayé cette matiere avec de l'eau, a été tel que je m'empresse de vous le communiquer afin de vous engager à ne pas employer cette méthode, mais à inoculer au moyen de fils pris de ces morceaux de linge dans la crainte que l'introduction de cette nouvelle inoculation dans Genève, ne souffre, par un manque de réussite dans les premiers essais.
Le 23 juillet, j'inoculai avec des fils pris des chemises de mes enfans, une petite fille âgée de 3 ans, chez qui ils produisirent une vaccine très-réguliere, avec une fievre suffisante pour ne pas douter que tout le système n'ait reffenti l'effet de l'inoculation.
Le 2 septembre, j'inoculai avec la même matiere délayée avec la quantité d'eau chaude absolument nécessaire pour la rendre liquide & pouvoir humecter ma lancette, une autre enfant âgé de 3 ans. Quoique l'inoculation fût faite à chaque bras avec tout le soin possible, elle ne produisit aucun effet.
\setcounter{page}{170}
Le 9 septembre j'inoculai deux jumeaux, âgés de 3 ans; & pour comparer les deux méthodes, chaque enfant fut inoculé par deux piqûres, avec de la matiere délayée à un bras, & par un fil imprégné de la même matiere à l'autre bras.
Les quatre piqûres n'ont encore cette fois produit aucun effet quelconque, tandis que les deux incisions où le fil a été inféré, ont produit deux pustules vaccines qui font leur cours très-régulierement. Ce n'est donc pas la matiere que l'on doit accuser dans les inoculations qui n'ont pas réussi; il paroît plus vraisemblable que la matiere vaccine offre encore cette différence de la variolique, qu'elle n'est pas susceptible d'être délayée après avoir été séchée pendant un si long temps.
Je vous invite donc à inoculer avec les fils, d'autant plus que je me suis convaincu par l'expérience, que rien n'est plus facile que de faire tenir ces fils dans l'incision, au moyen d'un emplâtre adhésif ( dyachil. cum gumm. )
Dans ma derniere lettre, je vous proposai l'explication de l'apparition des pustules vaccines sur les malades de l'hôpital de Londres, tandis que rien de semblable n'avoit été observé à Berkeley, ni dans les autres provinces d'Angleterre.
L'ouvrage même de Mr. le Dr. Woodville,\setcounter{page}{171} que j'ai reçu il y a peu de jours de Londres, avec celui du Dr. Pearson & le second du Dr. Jenner. a confirmé mes soupçons, & je m'en remets à vous pour en juger, & m'expliquer quelles peuvent avoir été les vues du Dr. en exposant les malades non-seulement à la contagion de leurs maladies respectives, mais de plus en inoculant avec de la matiere variolique, un très-grand nombre des personnes qu'il avoit inoculées avec de la vaccine 5 jours auparavant, pour faire cheminer les deux maladies ensemble. Je conçois bien que ce mélange pouvoit paroître intéressant, mais il ne change rien à l'objet important de cette découverte, savoir; que la vaccine ne produit de pustules qu'à l'endroit de l'infection. J'avoue de plus, que je n'ai pas saisi la raison qu'il en donne.
N'avez-vous encore fait aucune recherche pour savoir si la vaccine est connue en Suisse ou en France?
Ne pourriez-vous pas inviter les personnes qui font dans le cas de soigner les Bestiaux, à vous faire parvenir leurs observations? Votre Journal vous donne cette facilité, en y insérant quelques lignes d'invitation\footnote{Il y a quelque temps que je fus consulté pour une femme de campagne qui avoit, me disoit-on, pris des boutons érysipélateux à la main pour avoir trait une vache malade. J'eus quelque espérance que ce pouvoit être la vaccine. Je fis venir cette femme. Tout le poignet, le métacarpe & l'avant bras étoient fort enflés, fort rouges, livides même, fort douloureux & garnis de boutons cohérens, tubéreuleux & durs; mais dont aucun ne me parut contenir un fluide. Je la guéris par un cataplasme de charbon, qui calma sur le champ les douleurs, & dissipa graduellement tous les symptômes du mal. Il est évident que ce n'étoit pas la vaccine, quoiqu'elle l'eût bien prise de sa vache qui avoit eu, me dit-elle, les pis malades de la même manière. Je n'ai d'ailleurs rien appris qui pût me faire croire à l'existence de la vaccine dans notre Département.}.
\setcounter{page}{172}
\subsection{Vienne 2 octobre ( 10 vendémiaire an VIII. )}
....L'explication que je vous ai proposée pour rendre raison de l'éruption que le Dr. a fréquemment observée dans les individus auxquels il avoit inoculé la vaccine, me paroit recevoir un grand degré de plausibilité par une lettre qu'un inoculateur de Londres a écrite le 19 juin dernier à Mr. Coleman Professeur de l'Ecole vétérinaire, & dont celui-ci a bien voulu m'envoyer la copie avec un tuyau de plume sur lequel on avoit laissé sécher du virus vaccin. Il lui donne dans cette lettre la généalogie de ce virus, qui pris originairement sur la vache de Mr. Coleman a passé successivement par le corps de 8 enfans inoculés les uns des autres, dans Londres même, & qui tous ont eu la maladie sans aucune éruption.
Mr. Coleman\setcounter{page}{173} ajoute qu'il a envoyé du même virus à Deptford & à Lincoln & que toutes les personnes à l'inoculation desquelles il a servi ont eu la maladie d'une manière très-bénigne & sans aucune éruption.
Les personnes que le Dr. a inoculé à son hôpital immédiatement de la vache de Mr. Coleman, ont eu des pustules, même en grande abondance; les 3 premieres, par exemple, en ont eu 300, 105 & 350.
Comme dans tout l'ouvrage, il n'est pas question d'autre vache de M. Coleman que celle-ci, il y a toute raison de croire que celle dont parle Mr. Paytherns est bien la même dont le Dr. s'est servi. Nous avons vu, cependant que, cette même matiere inoculée dans divers quartiers de la ville de Londres, à Lincoln & à Deptford, n'ont jamais produit de pustules qu'à la partie inoculée. Où donc en chercher la cause, sinon dans les miasmes varioliques dont le dit Hôpital est nécessairement infecté ?
Aussitôt que j'eus reçu cette matiere séchée sur un tuyau de plume, je la délayai aussi peu que possible, & en inoculai un enfant de cinq mois; elle ne produisit aucun effet. L'enfant étoit, cependant, susceptible d'infection vaccine, car inoculé quelque temps après avec de la matiere prise du bras d'un enfant qui l'avoit été avec des fils pris d'une chemise de mes enfans, il\setcounter{page}{174} a eu une vaccine très-régulière, accompagnée d'une fièvre très-légère le 6e. & 7e. jour. Il est bien remarquable que cette matière qui perd son activité, en la délayant avec la plus petite quantité d'eau possible, conserve toute sa force, après un très-long espace de temps.
La lettre qui contenoit les premiers fils imprégnés dont je me suis servi, est datée de Londres, 20 mars 1799; aussitôt qu'elle eut produit l'effet désiré, je ne songeai plus à m'en servir, trouvant préférable de faire usage de la matière la plus récente que me fournissait chaque inoculation; par conséquent, je ne soignai point ce qu'il restait de ces fils & je n'ai cessé de porter constamment dans ma poche la lettre & les fils renfermés dans un portefeuille, afin de pouvoir satisfaire la curiosité des personnes qui désiraient savoir comment l'on faisait venir d'Angleterre, la matière de cette nouvelle maladie.
Mr. le comte de âgé de 40 ans, s'étant adressé à moi pour se faire inoculer, y mit la condition que ce serait avec les fils même d'Angleterre; je lui témoignai mes craintes sur la réussite de son inoculation avec des fils qui, pendant tout l'été avaient été exposés à la chaleur de la saison & de ma poche. Il insista, & l'inoculation se fit le 23 septembre. Elle a réussi on ne peut pas mieux & a produit une\setcounter{page}{175} pustule des mieux caractérisées, & à très-peu de chose semblable à la seconde gravure du Dr. Jenner. Depuis que j'ai inoculé Mr. le Comte de W., le Dr. Pearson de Londres a eu la bonté de m'envoyer un bout de fil imprégné d'environ 5 pouces de longueur. Si vous mettez quelque prix à en avoir d'une aussi bonne source, vous en trouverez un petit morceau ci-inclus.
J. DE CARRO, D. M.

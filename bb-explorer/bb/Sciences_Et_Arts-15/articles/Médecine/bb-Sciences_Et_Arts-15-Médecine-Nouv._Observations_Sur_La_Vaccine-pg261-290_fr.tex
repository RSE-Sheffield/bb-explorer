\setcounter{page}{261}
\chapter{MÉDECINE.}
\section{FURTHER OBSERVATIONS ON THE VARIOLE VACCINAE. Nouvelles observations sur la Vaccine, par EDOUARD JENNER Dr. M. de la Soc. Roy. etc. Londres 1799. 4to. pp. 64.}
Si les expériences nombreuses qu'on a faites, en Allemagne et en France, depuis que nous avons fait connoître la mémorable découverte du Dr. Jenner, n'en avoient pas amplement confirmé les résultats; si les conséquences de cette découverte n'étoient pas de la plus haute importance pour l'art de conserver les hommes, nous nous ferions quelque scrupule d'en entretenir aussi fréquemment nos lecteurs, et nous passerions sous silence les divers écrits auxquels elle a donné lieu, à moins qu'ils ne continssent des faits bien saillans par leur nouveauté.
Mais il s'agit ici d'un moyen d'expulser totalement de l'Europe un fléau terrible qui la ravage depuis plus de dix siècles, et qui bien plus funeste que les guerres les plus meurtrières, fait périr la seizième partie du\setcounter{page}{262} genre humain. Ce moyen a été éprouvé avec le plus grand succès dans différens pays\footnote{Le Dr. De Carro qui a été l'un des premiers à inoculer la vaccine sur le Continent, m'écrit de Vienne en Autriche, qu'il continue à le faire avec le plus grand succès. Une épidémie formidable de petite-vérole qui s'y est manifestée cet été, et qui emportoit tous les jours 30 à 40 individus, a engagé un très-grand nombre de personnes éclairées, tant parmi la Noblesse que parmi les Gens de Lettres, à faire vacciner leurs enfans. --- A Gênes, où j'ai envoyé des fils, le Dr. Scassi y a aussi introduit ce préservatif. --- Ceux que j'ai fournis au Dr. Des Granges de Morges, dans le Pays-de-Vaud, y ont également réussi. --- A Hanovre, à Gotha, dans le Holstein, partout où la vaccine a pénétré, elle a eu les plus heureux résultats. (O)}.
Partout, la vaccine s'est trouvée mériter à peine la qualification d'une indisposition légère; et cependant, tout annonce qu'elle garantit efficacement et pour toujours de la possibilité de prendre la petite-vérole. Tous les essais qu'on en a faits, jusqu'à présent, tendent à démontrer que lorsque la vaccination\footnote{Le mot de Vaccine, par lequel nous avons les premiers hasardé de traduire le nom Anglais de la maladie dont il est ici question, (cow-Pox) a été généralement adopté, même en Angleterre, où l'on en a étendu l'usage jusqu'à en faire un autre substantif ( vaccination) qui exprime très-heureusement l'inoculation de cette maladie. Celui-ci suppose encore, au moins dans notre langue le verbe vacciner et ses participes. Nous n'hésitons pas à les employer les uns et les autres, comme beaucoup plus commodes que les périphrases par lesquels il faudroit les remplacer. S'il est un Néologisme pardonnable, c'est celui qui dérive de la nature des choses; et quand on se trouve forcé de forger un nouveau mot pour exprimer une chose nouvelle, c'est suivre le génie de la langue que d'adopter avec le mot ses dérivés, comme on l'a fait dans le temps pour les mots: inoculation, inoculer, inoculateur, etc. innovation qui fut adoptée par l'Académie Française. (R)} sera généralement adoptée, lorsqu'on\setcounter{page}{263} aura enfin persuadé aux pères et mères d'y soumettre de très-bonne heure tous leurs enfants sans exception, la petite-vérole ne se développera plus en Europe. Alors il sera facile d'en préserver les générations futures, en renonçant même à la vaccine, qui n'étant pas contagieuse, s'éteindra d'elle-même. Il suffira, d'empêcher soigneusement à l'avenir, l'introduction de tout nouveau foyer de contagion, par les mêmes moyens qui ont réussi à écarter de nos climats la peste et la lèpre, fléaux que l'Orient vomissoit fréquemment, autrefois sur nos Continens, mais que les lazarets et les quarantaines en ont enfin repoussés.
Cette considération nous engage à ne négliger aucun des détails qui peuvent servir à apprécier, si ce n'est à réaliser cette espérance.\setcounter{page}{264} Ce n'est point ici un objet de pure curiosité, dans lequel de nouveaux faits, peuvent seuls satisfaire l'esprit. C'est un objet d'utilité publique, dans lequel un grand nombre d'observations de meme genre, est nécessaire pour écarter les illusions, et pour établir enfin la vérité sur une base solide. Nous allons donc continuer à rendre un compte sommaire des principales brochures qui nous sont parvenues d'Angleterre, sur ce sujet. Mais, comme elles sont écrites avec une espèce de désordre, assez familier aux Anglais, dans le grand nombre de productions éphémères qui sortent de leur plume, désordre qui ne peut que nuire à leur effet sur des lecteurs étrangers, nous croyons classer plus avantageusement les observations qu'elles contiennent, en les rangeant sous différens chefs comme on l'a fait pour le Mémoire inséré dans notre 114me. Numéro. La comparaison en sera plus distincte et plus facile.
La Brochure que nous annonçons aujourd'hui, est fort arriérée. Elle est du mois d'avril 1799. Mais elle sert de suite aux premières recherches du Dr. Jenner, dont nous avons rendu compte en détail; et l'on y trouvera quelques faits intéressans, qu'il ne sera pas inutile de rappeler aujourd'hui, puisque quelques-uns d'entr'eux peuvent servir de\setcounter{page}{265} réponse aux objections plus qu'indiscrètes\footnote{Voyez dans le Moniteur les réflexions des Cit. Goëtz, Vaume, Salmade, etc. Un de nos Concitoyens âgé de 30 ans, qui n'avoit pas eu la petite-vérole et qui frappé de la bénignité de la vaccine, étoit sur le point de se la faire inoculer, en fut détourné, dit-on, par la lecture d'une des feuilles où ils les ont fait insérer. Il vient de prendre la petite-vérole, et il en est mort, tellement regretté que plus de 2000 personnes ont honoré son convoi funèbre de leur présence. On n'a pas encore eu de pareils faits à reprocher aux apologistes de la vaccine. Le bien qu'ils ont fait jusqu'à présent est de toute évidence. Le mal qui est résulté de leurs recherches est nul. Il semble que cette considération devroit fermer la bouche à ceux qui n'ont rien de mieux à objecter si ce n'est qu'ils ne sont pas encore convaincus de l'utilité de la vaccine. Mais ils n'ont pas encore articulé un seul de ses désavantages. Que ne se contentent-ils donc d'exhorter les vaccinés à ne pas s'exposer à la petite-vérole, avant de s'être soumis à l'épreuve de l'inoculation variolique? Nous nous réunirions tous à eux de grand cœur. Voyez plus bas ce qu'en dit le Dr. Jenner lui-même.} qu'on a faites à Paris, contre le Rapport du Comité Médical, qui avoit été chargé de suivre les progrès de la découverte dont il s'agit.
\subsection{I. Origine de la Vaccine.}
L'on se rappelle que dans sa première publication, le Dr. Jenner avoit attribué la vac cine,\setcounter{page}{266} cette maladie, pour ainsi dire endémique dans le Comté de Gloucester, et particulière aux vaches de ce pays-là, à ce que les domestiques qui sont employés à les traire, sont aussi fréquemment appelés à panser en même temps les ulcères des chevaux atteints du Javart. Cette hypothèse a presque généralement été rejetée en Angleterre, parce qu'on a fréquemment essayé d'inoculer directement le Javart, soit à des hommes, soit à des vaches; sans pouvoir réussir à produire la vaccine; et parce que cette dernière maladie s'est quelquefois manifestée dans de grandes laiteries, près de Londres et ailleurs, dans lesquelles il n'y avait eu aucun cheval malade. Cependant, le Dr. J. persiste dans son opinion, 1°. parce que c'est une opinion populaire, et universellement admise par tous les fermiers du Comté de Gloucester, et particulièrement par ceux qui connoissent le mieux les maladies des bestiaux. 2°. Parce que la vaccine, lorsqu'elle n'a pas été apportée dans une ferme, par une vache, ou une personne déjà infectée, y a toujours été immédiatement précédée d'un javart. 3°. Parce que la vaccine est inconnue en Ecosse et en Irlande, où l'on n'emploie point d'hommes dans les laiteries. 4°. Parce que si le virus du javart, ne paroît pas susceptible de se développer\setcounter{page}{267} par l'inoculation sur une peau saine, il n'en est pas de même, lorsque ce virus se trouve en contact avec une peau déjà effleurée, ou affectée de petits accidens pareils, très-communs à la campagne. Il paroît certain qu'il produit souvent en pareil cas sur les hommes, une maladie parfaitement semblable à la vaccine, et qui dans le pays ne porte pas d'autre nom. Il en cite deux exemples qui lui ont été communiqués depuis; l'un par le Rev. Mr. Moore, dont le cheval, la vache et le domestique, furent successivement atteints en 1797 du javart, de la vaccine, et d'une maladie éruptive, semblable à la petite-vérole, mais non-contagieuse; l'autre par Mr. Fewster, chirurgien de Thornbury, qui fut appelé en avril 1798 à soigner un domestique nommé William Morris, qui, pour avoir pansé un cheval atteint du javart, eut une maladie parfaitement semblable à la vaccine. 5o. Parce que cette maladie a paru jusqu'à un certain point capable, comme la vaccine, de préserver de la petite-vérole, mais moins sûrement qu'après avoir passé par le corps d'une vache. 6o. Enfin, parce que l'enfant inoculé d'après un malade de cette espèce (voy. la Bib. Brit. Sc. et Arts, vol. IX, p. 369) eut toutes les apparences de la vaccine inoculée.
\setcounter{page}{268}
\subsection{2. Vaccine bâtarde ou dégénérée.}
Le choix du virus paroît à l'auteur de la plus grande importance dans la vaccination. Il craint qu'on ne s'y trompe souvent, qu'on ne produise en l'inoculant une vaccine bâtarde ; et que le retour de la petite-vérole après une pareille maladie, ne fasse révoquer en doute l'avantage principal de la vaccination. C'est ce qui est arrivé déjà plus d'une fois. Peu de temps après sa première publication, le Dr. reçut une lettre du Dr. Ingenhousz, par laquelle il lui marquoit ; qu'un fermier du Comté de Wilts avoit pris la petite - vérole quoiqu'il eût eu quelque temps auparavant la vaccine bien caractérisée. Il l'avoit, disoit-il, prise d'une de ses vaches, tellement infectée que son pis ulcéré exhaloit une odeur très - fétide. Plusieurs autres informations du même genre parvinrent en même temps à l'auteur. Mais toutes se trouvoient accompagnées de circonstances, qui prouvoient que la prétendue vaccine, dont il y étoit question étoit une maladie bâtarde ou dégénérée.
Le Dr. entre dans d'assez grands détails pour montrer les différentes sources de bâtardise ou de dégénération qui peuvent donner lieu à une pareille illusion. Indépendamment\setcounter{page}{269} de la possibilité que quelques-uns de ceux qui ont pris, dit-on, la petite-vérole après avoir eu la vaccine, n'aient eu que celle qui provient directement du javart, puisque dans le pays on la confond souvent avec celle qui vient des vaches, quoiqu'elle n'ait pas aussi constamment la propriété de préserver de la petite-vérole, le Dr. Jenner observe.
1°. Que lorsque les vaches changent de nourriture, surtout au printemps, elles sont fort sujettes à avoir sur leur pis des boutons qu'il faut bien distinguer de la vraie vaccine. J'ai vu, dit-il, ces boutons se communiquer à des laitières, produire même quelquefois une maladie assez grave, mais qui jamais ne les a mises à l'abri de la petite-vérole. Sarah Merlin, laitière chez Mr. Clark à Stonehouse, occupée avec deux de ses compagnes à traire les vaches de son maître, prit à l'âge de 13 à 14 ans une maladie qu'on crut être la vaccine. Trois des vaches avoient de grandes ampoules blanches sur leurs pis. La jeune laitière en eut les doigts couverts. Ses mains et ses bras s'enflammèrent et se tuméfièrent. On pansa les ampoules avec un onguent domestique, et la malade se guérit sans ulcération, et sans s'apercevoir d'aucune indisposition générale. La maladie ne se communiqua ni aux deux autres laitières, ni\setcounter{page}{270} à aucune des autres vaches. Sarah Merlin, se crut cependant à l'abri de la petite-vérole; mais quelques années après, elle la prit, et eut une éruption abondante. - Or, il est évident qu'on auroit pu prévoir cet accident. Car la maladie qu'elle avoit eue, différoit du tout au tout de la vraie vaccine, qui se communique pour l'ordinaire avec beaucoup de facilité à toutes les vaches et à toutes les laitières de la ferme; qui ne se manifeste pas par de grandes et nombreuses ampoules blanches, mais par une ou deux vésicules bleuâtres, lesquelles dégénèrent toujours, du plus au moins en ulcères, et qui produit toujours, surtout lorsqu'il y a plus d'un bouton, des symptômes d'affection générale.
2°. Le vrai virus vaccin, peut aussi lui-même dégénérer au point de ne produire qu'une vaccine bâtarde. Des circonstances particulières peuvent, par exemple, le décomposer par la putréfaction. Et c'est probablement ce qui étoit arrivé dans le cas rapporté par le Dr. Ingenhousz, puisque dans ce cas là le pis ulcéré des vaches qui avoient communiqué la maladie exhaloit une odeur très-fétide, circonstance que je n'ai jamais remarquée dans la vraie vaccine, dit l'auteur, quoique je sois fréquemment entré dans des étables dont toutes les vaches en étoient atseintes.
\setcounter{page}{271}
Peut-être encore, le pus qui se forme dans les ulcères sur la fin de la maladie altérée, est-il le vrai virus qui n'est jamais plus efficace pour produire une vaccine régulière que lorsqu'il est bien limpide. Enfin il peut encore arriver que même dans son état de limpidité, ce virus dégénère et se décompose par quelque cause inconnue, au point de ne pas produire la vraie vaccine, mais seulement une irritation irrégulière, qui n'en aura que l'apparence, et qui se trouvera insuffisante pour garantir de la petite-vérole. Cette fausse vaccine diffère pourtant toujours de la vraie à certains égards, de manière à ne point en imposer à un praticien expérimenté, prudent et attentif. Mais à moins d'y regarder de près, il est facile de s'y tromper; "et" c'est probablement par quelque erreur de" de ce genre que j'ai cru, "dit l'auteur," que quoique la vaccine garantisse efficacement de la petite-vérole, elle n'a point" la même propriété relativement à elle-même." J'en reviens aujourd'hui, et je suis disposé" à croire, avec le Dr. Pearson, qu'on ne" peut plus être atteint de la vaccine, quand" on a eu la petite vérole\footnote{Il ne paroît pas qu'à l'époque où le Dr. faisoit cet important aveu, qui indique bien sa candeur et son zèle pour la vérité, il eût encore eu connoissance de la vaccine bâtarde, que nous avions obtenue en inoculant avec des fils pris sur le bras d'un vacciné qui avoit eu la petite-vérole dans son enfance. Seroit-ce cette circonstance qui a fait dégénérer le virus sur sa peau? Ce virus ne pouvant plus produire sur elle ses effets primitifs; n'y a-t-il produit que ses effets secondaires? Mais il est bien extraordinaire que cette dégénération se soit transmise jusqu'à la onzième génération sans aucune déviation, et sans que la vraie vaccine ait jamais été la conséquence de ces essais. Lorsqu'on inocule la petite-vérole à une personne qui l'a déjà eue, cette inoculation produit quelquefois des symptômes d'irritation locale, de vrais boutons vésiculaires. Mais le fluide qui y est contenu, quoiqu'incapable d'affecter constitutionnellement l'inoculé, peut servir à donner une petite-vérole complète aux individus qui en sont susceptibles, et l'on n'observe point en pareil cas dans le virus variolique de dégénération transmissible. D'ailleurs nous avons vu une vaccine bâtarde exactement semblable à celles que nous avions observées l'hiver passé, être produite sur un enfant qui n'avoit pas eu la petite-vérole, et pour la vaccination duquel on avoit employé le vrai virus vaccin, mais dans un état purulent. (O)}."
\setcounter{page}{272}
Quoiqu'il en soit, jusqu'à-ce qu'on connoisse parfaitement toutes les irrégularités dont la vaccine est susceptible, la prudence exige que pour plus de sécurité on inocule la petite-vérole à tous les vaccinés, quelque temps après la vaccination.
Au surplus, ces réflexions sont applicables\setcounter{page}{273} au virus variolique même. Car on a dans la petite vérole inoculée un grand nombre d'exemples d'irrégularités et de demi-succès qui annoncent dans le virus qui l'a produite une dégénération dont les causes sont souvent inconnues, et qui exigent, de la part de l'inoculateur, la plus grande attention. Mr. Kite en a cité dans le 4me. vol. des Mémoires de la Société Médicale de Londres p. 114. Notre auteur en cite d'autres. Mr. Earle de Frampton fit en 1784 une inoculation générale à Arlingham. Tous ses premiers inoculés eurent la petite-vérole heureuse, complète et régulière. Mais sur la fin de cette inoculation, n'ayant plus de virus tel qu'il l'auroit désiré, il inocula cinq autres individus avec du pus pris sur un vieux bouton flétri (a stale pustule) unique reste d'une petite-vérole très-abondante et déjà sèche. Ces cinq inoculés eurent tous beaucoup d'inflammation aux bras, plus même qu'à l'ordinaire; des douleurs subaxillaires, de la fièvre au 9e. jour et des boutons sur le corps; mais ces boutons disparurent beaucoup plus tôt qu'ils ne l'auroient dû; et l'inoculateur concevant quelque doute sur la nature de la maladie, fit voir un de ses malades à un vieux praticien, qui prononça qu'ils étoient à l'abri.\setcounter{page}{274} Cependant quelque temps après quatre d'entre eux prirent la petite-vérole naturelle fort abondante, et l'un des quatre en mourut. Le cinquième étant averti évita la contagion, et mourut ensuite d'une autre maladie. En 1789, le même Mr. Earle inocula trois enfans avec du virus que lui avoit procuré un de ses amis. Le développement de ce virus suivit la même marche que chez les inoculés d'Arlingham. Les boutons disparurent au bout de deux jours, ce qui, joint a la ressemblance de l'inflammation locale, donna de la défiance à Mr. Earle. Il réinocula cés enfans avec du pus bien choisi, et ils eurent tous trois une petite-vérole heureuse et régulière.
Il est donc certain que le virus variolique peut subir des changemens qui rendent l'inoculation inefficace comme préservatif, quoiqu'elle produise des apparences assez semblables à celles de la bonne petite-vérole inoculée. Or, ce qui est vrai de la petite-vérole, peut être vrai aussi de la vaccine ; et l'on conçoit que jusqu'à-ce qu'on ait acquis une grande expérience de cette maladie, on peut facilement y être trompé \footnote{Cela seroit actuellement difficile. Car la marche de la vaccine inoculée est devenue si uniforme, si régulière, que sur plusieurs centaines de Vaccinés que nous avons été à portée d'observer, il ne s'en est trouvé qu'un très-petit nombre qui présentât des différences sensibles. C'est là sans doute une nouvelle et forte raison de nous défier des moindres déviations. Elles sont incomparablement plus fréquentes et plus variées dans la petite-vérole inoculée. Au surplus, cette uniformité de la vaccine est, suivant moi, un des grands avantages de la vaccination, une grande source d'inquiétude de moins. Il est pénible de n'avoir devant soi en inoculant la petite-vérole, qu'une perspective vague et incertaine des avantages de cette inoculation, sans aucune possibilité de prévoir d'avance avec certitude, ni le moment, ni la manière, ni le degré de la maladie. (O)}.
\setcounter{page}{275}
\subsection{3. Accidens propres à la Vaccine.}
Lorsque la vaccine se borne à un bouton vésiculaire dans lequel on ne trouve qu'un fluide limpide comme de l'eau, et que ce bouton se convertit en croûte; c'est toujours une maladie infiniment bénigne et légère. Tel est pour l'ordinaire son cours dans les petits enfants, qui paraissent l'avoir d'autant plus heureuse qu'ils sont plus jeunes. Le Dr. a fait vacciner un enfant 20 heures après sa naissance. Cette opération eut beaucoup de succès. La maladie se développa complètement, sans aucune indisposition\setcounter{page}{276} apparente\footnote{C'est aussi ce que nous avons observé. Nous avons vacciné plusieurs enfants peu de jours après leur naissance. Ils ont tous eu une vaccine des plus régulières et des plus heureuses. On peut donc préserver ainsi tous les enfants de la petite-vérole, avant qu'ils aient aucune chance de la prendre; avant même de les envoyer en nourrice. Si cet usage que nous travaillons à introduire à Genève en priant les Pasteurs de distribuer aux parens de tous les enfans qu'on leur présentera à baptiser un avis imprimé propre à les éclairer sur les avantages de la vaccination à cette époque, si dis-je, cet usage devient général, comme nous l'espérons, la vaccine aura bientôt expulsé de l'Europe la petite-vérole. (O)}, et l'inoculation subséquente de la petite-vérole ne produisit aucun effet. — Mais dans les individus plus âgés, il arrive souvent, surtout dans la vaccine naturelle, que le bouton se creuse et se convertit en un ulcère rongeant, dont l'irritation produit beaucoup d'inflammation, et quelquefois des symptômes graves.
Il paroît que le pus qui se forme dans ces ulcères ne communique la vaccine qu'autant qu'il est mêlé avec une certaine quantité de la sérosité limpide qui se forme dans la vésicule. Ces deux fluides sont probablement le résultat de deux sécrétions différentes; et le pus pur est par lui-même incapable de communiquer la maladie. —\setcounter{page}{277} Quatre domestiques furent inoculés d'une vache dans le moment où le bouton étoit en suppuration. L'inoculation manqua complettement. Un mois après, ils prirent la maladie, pour avoir trait des vaches infectées. Mais si le pus qui se forme dans un ulcère vaccin ne communique pas toujours la vraie vaccine, en revanche il peut produire des symptômes d'irritation plus ou moins graves; et c'est une nouvelle raison pour l'éviter soigneusement dans la vaccination. Il paroît qu'il en est jusqu'à un certain point de même dans la petite-vérole, et que dans l'une et dans l'autre maladie il faut distinguer l'affection primitive de l'affection secondaire. Il n'y a que le fluide limpide produit par la première qui soit propre à l'inoculation. Le pus qui est le résultat de la seconde n'est efficace que par son mélange avec la sérosité claire; et ce mélange produit quelquefois des symptômes subséquens d'irritation. Mr. Trye, Chirurgien de l'hôpital de Glou-cester, inocula 10 enfans avec du virus pris d'une petite-vérole discrète après le commencement de la dessication. Tous eurent une petite-vérole heureuse et régulière; mais sur la fin de la maladie, deux eurent un érysipèle qui s'étendit sur tout le bras, un autre\setcounter{page}{278} eut des furoncles autour des incisions, et cinq ou six autres eurent des abcès subaxillaires. On doit donc toujours se défier d'un pus trop épais. Plus le virus est limpide, et plus son action est régulière. Or cette attention semble aussi particulièrement nécessaire pour le virus vaccin, surtout si l'on inocule directement d'après une vache, sur le pis de laquelle il se forme presque toujours des ulcères, dont le pus produirait vraisemblablement plus d'irritation locale qu'il ne le faut pour que la vaccine soit régulière et vraiment préservatrice. Mais quelque attention que l'on apporte au choix du virus, il arrive souvent que la vaccination ne produit aucun effet\footnote{J'avois vacciné une jeune fille d'un an qui eut une des vaccines les plus bénignes et les plus régulières que j'aie vues. Je vaccinai d'après elle et de bras à bras trois autres enfans. Le virus étoit abondant et parfaitement limpide. Un de mes collègues en vaccina trois autres. De plus j'imprégnai des fils sur le bras de ce même enfant, je les envoyai en Dauphiné où l'on s'en servit pour vacciner encore trois autres individus. Aucune de ces neuf inoculations n'eut son effet. Cependant la limpidité du virus, la régularité du bouton et de l'efflorescence, la fièvre bien sensible dont étoit atteint l'enfant au moment où se firent ces vaccinations, ne me laissoient aucun doute sur la bonté du choix. La seule différence que je pusse appercevoir entre cette vaccine et d'autres c’est que l’efflorescence étoit d’un rouge un peu plus foncé qu’à l’ordinaire, qu’elle fut plus permanente, et que dans toute son étendue il y eut une légère desquamation de l’épiderme. Y auroit-il donc des enfans dont la vaccine n’est pas susceptible de se transmettre, même par inoculation ? Cette conjecture est encore bien hasardée ; mais il m’a semblé que lorsque j’ai vacciné plusieurs enfans à-la-fois, dans le même moment, et du même virus, il a réussi ou manqué sur tous également. Cependant j’ai vu un grand nombre de vaccinés, sur lesquels il n’y a eu qu’une des deux incisions qui ait bien pris ; et en général le nombre de ceux chez lesquels il a manqué après une première opération, a été à Genève beaucoup plus grand qu’à Londres, à moins que le Dr. Woodville ne nous ait pas fait connoître dans son Rapport tous ceux qu’il a fallu vacciner deux fois. (O)}.\setcounter{page}{279} Quelquefois elle excite une légère inflammation qui se dissipe promptement sans devenir vésiculaire. D’autres fois, elle va jusqu’à produire un fluide ichoreux, sans affecter le systême. Il est assez singulier que ce manque de succès soit beaucoup plus commun à la campagne qu’à Londres, où d’ailleurs le virus paroît produire aussi moins d’inflammation locale et un bouton plus relevé que dans le Gloucestershire. A quoi tiennent ces différences ? à quoi tient surtout la fréquence de ces éruptions générales semblables à celle de la petite-vérole, que le Dr. Woodville\setcounter{page}{280} a observées dans un aussi grand nombre de cas a la suite de la vaccine inoculée; éruptions qu'on n'apperçoit jamais à la campagne? Le Dr. paroît dans cette Brochure disposé à en accuser l'air de Londres, qui modifie, dit-il, évidemment toutes les inflammations érysipilateuses.
Enfin, une autre particularité de la vaccine, c'est qu'elle est susceptible de se compliquer avec d'autres maladies éruptives\footnote{J'ai parlé amplement dans mon Mémoire, de la manière dont nous avons vu à Genève la vaccine se compliquer avec la petite-vérole. Celle des deux maladies qui se développe la première, annulle ou modifie l'autre. Cependant le Dr. De Carro m'a communiqué un cas de ce genre, qui offre une particularité remarquable. C'étoit un enfant de six mois, qui ayant été vacciné le 13e. septembre, prit le même soir la fièvre de la petite-vérole, dont l'éruption fort peu abondante se fit le surlendemain; ce qui n'empêcha pas le bouton vaccin de se manifester au 4e. jour comme à l'ordinaire, et de suivre son cours d'une manière régulière et complète. La vaccine et la petite-vérole furent l'une et l'autre des plus heureuses. Peut-on dans ce cas là attribuer la singulière bénignité de la petite-vérole au développement de la vaccine, quoique postérieur à celui de la petite vérole? C'est ce qui, à en juger d'après nos observations, me paroît fort douteux.
J'ai vu un autre complication de vaccine assez singulière. Il y a environ un mois que j'avois vacciné un enfant de 4 ans. Dès le lendemain cet enfant prit de la fièvre, des maux de tête et des vomissemens. Je craignis la petite vérole qui étoit épidémique dans son quartier. Mais heureusement ce ne fut qu'une fièvre bilieuse assez bénigne et très-régulière, qui dura près de trois semaines, ce qui n'empêcha pas la vaccine de se développer comme à l'ordinaire, avec un gros bouton vésiculaire à chaque incision, et une belle efflorescence tout autour, mais sans aucune aggravation de fièvre au 8e. jour, et sans aucun accident. (O)}. On sait\setcounter{page}{281} que la petite-vérole et la rougeole s'excluent mutuellement. Lorsqu'on inocule la première à un enfant déjà atteint, sans qu'on s'en doute, par la contagion de la seconde, celle-ci, au moment de son développement suspend toujours celui de la petite-vérole qui ne se manifeste jamais qu'après la dessication de la rougeole. Il n'en est pas de même de la vaccine. Le Dr. l'avoit inoculée à un enfant qui la veille avoit pris la rougeole. Les deux maladies se développèrent en même temps sans aucune interruption ni aggravation de l'une ni de l'autre. Mais le bouton vaccin, quoique vésiculaire et de la grosseur d'un pois partagé ne fut entouré d'aucune efflorescence. L'enfant eut cependant de la fièvre au huitième jour.
\setcounter{page}{282}
\subsection{4. Traitement.}
Lorsque la vaccine est régulière, et que le bouton vésiculaire se convertit en croûte, sans aucun symptôme considérable d'irritation locale, il n'y a rien à faire; aucun remède, aucune application n'est nécessaire; mais lorsque le bouton paroît disposé à s'ulcérer, ou seulement lorsque l'inflammation érysipélateuse qui entoure l'incision est plus considérable qu'elle ne doit l'être, le Dr. recommande fortement de détruire le virus par des applications mercurielles, ou même par des escharotiques; non qu'il ne soit persuadé que de simples lotions avec l'eau de Goulard ou quelque autre solution minérale légèrement astringente réussiroient bien, mais parce qu'il a constamment éprouvé dans ces cas-là de bons effets du mercure et des caustiques, et qu'il n'en a jamais vu d'inconvéniens.
Susanne Phipps, âgée de 7 ans, avoit été vaccinée par le virus d'une vache infectée. La maladie fut régulière. L'efflorescence se manifesta au 13e. jour; mais 3 jours après, le bouton qui paroissoit d'abord disposé à se convertir en croûte, s'ulcéra et produisit de la fièvre et des douleurs subaxillaires. Ces\setcounter{page}{283} symptômes durèrent une semaine. L'ulcère s'étendit et devint de la grandeur d'un shelling. Alors il suppura abondamment; il s'y formades granulations; et enfin il se cicatrisa spontanément, mais après avoir duré long-temps. — Mary Hearn, âgée de 12 ans, fut vaccinée de Susanne Phipps. La maladie fut légère; L'efflorescence se fit au 14e. jour. Le bouton paroissant alors disposé à s'étendre, on y appliqua un onguent composé de cire, de blanc de baleine, d'huile d'olives et d'un peu de précipité rouge. L'efflorescence fut couverte d'onguent mercuriel fort. Au bout de 6 heures, l'efflorescence avoit disparu. Trois jours après, on pansa le bouton avec l'onguent citrin qui réussit. Ces deux enfans furent beaucoup plus malades par l'affection secondaire que par la maladie primitive; d'où l'auteur conclut qu'il est toujours prudent d'arrêter ces accidens plutôt que d'en courir la chance \footnote{Ces accidens qui, au rapport de Woodville, n'ont pas été observés à Londres, m'ont paru aussi à Genève incomparablement plus rares que l'auteur ne les suppose. Ils sont vraisemblablement beaucoup plus fréquens dans la vaccine naturelle ou dans celle qu'on inocule directement d'après une vache infectée que dans celle qui a déjà passé plusieurs fois par le corps humain. J'ai bien vu quelques enfans dan lesquels la chûte prématurée de la croute a été suivie d'une légére ulcération plus out moins longue, mais une simple application de pommade de Goulard a suffi pour l'arrêrer assez prompotement; et l'inflmattion érysipélateuse, qu'on voit aussi quelquesois autour de l'incision, m'a paru céder de méme tres-facilement á l'eau de Goulard. (O)}. Il veut\setcounter{page}{284} cependant qu'on attende que la vaccin produise tout son effet.
Qu'arriveroit-il donc, si l'on détruisoit le bouton aussitôt qu'il y auroit un peu de fièvre, sans attendre l'efflorescence? L'auteur n'ose pas l'affirmer; mais il présume que sans éteindre entièrement par là la susceptibilité pour la petite-vérole, on la diminueroit au point que ses effets sur le corps humain seroient beaucoup plus doux, et qu'elle pourroit se changer ainsi en une maladie très-légère, susceptible de se transmettre avec le même caractère de bénignité. Cette conclusion probablement trop hasardée, il la tire des observations suivantes:
En avril 1798, Mary James fut vaccinée avec trois autres enfants. On détruisit le virus aussitôt qu'il parut produire un effet général. Au mois de décembre suivant, on inocula la petite-vérole à l'enfant. L'incision s'enflamma,\setcounter{page}{285} suppura, et se convertit en croûte. Au 9e. jour l'enfant eut un peu de chaleur et une légère rougeur aux poignets, mais ces symptômes se dissipèrent très-promptement. — Sa mère, âgée de 50 ans, et son frère âgé de 6 ans furent inoculés d'après elle. L'enfant eut un peu de fièvre au bout de huit jours et une éruption semblable à celle de la rougeole; ensuite il eut quelques boutons, mais qui pour la plupart avortèrent. La mère eut aussi un peu de fièvre, mais ni rougeurs, ni boutons. Une garde qui les servoit et qui n'avoit pas eu la petite-vérole; quoiqu'elle y eût été exposée plusieurs fois pendant son enfance, eut aussi un peu de fièvre; et quelques boutons dont deux ou trois seulement suppurèrent.
\subsection{5. Certitude du préservatif.}
A l'époque où l'auteur publia cet écrit, les inoculations de vaccine, quoique déjà bien multipliées, ne l'étoient pas cependant assez pour que la propriété qu'a cette maladie de préserver de la petite-vérole ne fût encore un problème aux yeux de bien des gens. Il croit devoir la confirmer par quelques observations nouvelles qui portent principalement\setcounter{page}{286} sur la vaccine naturelle. Il raconte l'histoire d'une cinquantaine d'individus qui l'avoient eue dans leur enfance, et auxquels diversChirurgiens qu'il nomme avoient ensuite, et à plusieurs reprises, inoculé la petite vérole, sans pouvoir la leur communiquer. Un soldat qui disoit aussi avoir eu la vaccine, fut inoculé par Mr. Tierney Chirurgien du régiment de Gloucester - Sud, avec plusieurs de ses camarades qui étoient dans le même cas. L'inoculation ne produisit aucun effet sur aucun de ces derniers; mais le premier ayant eu une petite-vérole bien développée et bien complette, on prit des informations sur la vaccine qu'il disoit avoir eue, et il se trouva que c'étoit un conte, un mensonge fait à plaisir.
La vaccine naturelle préserve donc de la petite-vérole. C'est un fait bien prouvé. La vaccine inoculée aura-t-elle le même avantage? C'est ce dont l'auteur ne doute pas. "Mais je ne suis pas," dit-il, "assez attaché à" mon opinion pour ne pas l'abandonner aussitôt" qu'on m'aura démontré mon erreur par des" faits bien constatés\footnote{Ce ne sera pas par des lettres anonymes telles que celle que le Cit. Vaume prétend avoir reçue de Genève, et dans laquelle on lui marque, dit-il, "que" les vaccinés n’y sont pas plus à l’abri de la petite-vérole que d’autres individus." Que la lettre existe ou non, et quelque soit son auteur, je certifie la fausseté du fait. Depuis six mois nous avons vacciné huit à neuf cents personnes qui toutes ont été depuis, plus d’une fois, exposées directement ou indirectement à la contagion de la petite-vérole; aucune n’en a été atteinte, et nous n’avons pas eu le moindre accident. (O)}. Seulement, "ajoute-\setcounter{page}{287} t-il," quand on fera l’épreuve de l’inoculation variolique sur les vaccinés, rappelons-nous que la petite-vérole elle-même n’éteint pas entièrement la susceptibilité variolique; et que quand même on l’a eue, naturelle ou inoculée, on n’est pas complètement à l’abri d’en avoir encore quelques symptômes." Les nourrices qui l’ont eue prennent souvent des boutons de leurs nourrissons, et cette éruption est quelquefois précédée de fièvre et de malaise. L’auteur a vu un homme qui avoit eu la petite-vérole inoculée, bien caractérisée par l’inflammation locale, mais avec très-peu de fièvre, et sans boutons. Pour plus de sécurité, il se fit réinoculer à plusieurs reprises, et à chaque fois l’inoculation produisoit un bouton vésiculaire, des gonflemens sous l’aisselle, et une légère indisposition générale. Le Dr. a de même\setcounter{page}{288} produit par l'inoculation variolique sur le bras d'un vacciné un bouton vésiculaire, d'après lequel il a inoculé avec succès une jeune femme qui eut une petite-vérole heureuse, mais complète. On voit assez fréquemment des cas semblables après la petite-vérole même\footnote{Le Dr. Coindet qui avoit soigné le jeune homme dont j'ai parlé plus haut, a eu à la suite de ses communications fréquentes et directes avec lui, un gros bouton variolique sur la main, accompagné d'une grande rougeur tout autour, de vives douleurs sur tout le bras, de gonflemens suba-xillaires etc. Cependant il avoit eu la petite-vérole. L'inoculation variolique a produit de pareils accidens sur quelques-uns des vaccinés de Paris. Ne diroit-on pas en lisant les réflexions des Cit. Goëtz et Vaume sur ces accidens, qu'ils n'ont jamais rien vu de semblable? Si cela est, on ne peut pas plus comp-ter sur leur expérience que sur leurs correspondances. Qu'ils essaient d'inoculer la petite-vérole à un grand nombre d'individus qui l'aient déjà eue. Ils seront bientôt détrompés. (O)}. Mr. Fewster a vu une servante qui très-certainement l'avoit eue, en étoit même très-marquée, et qui pour avoir porté sur son col un petit inoculé dont le visage étoit perpétuellement en contact avec le sien, eut d'abord un peu de fièvre, et ensuite une éruption abondante sur la joue gauche, et pas ailleurs; éruption d'après laquelle deux\setcounter{page}{289} autres enfans furent inoculés avec succès. Pour s'en convaincre, on les mena auprès d'un malade qui avoit la petite-vérole confluente, d'après lequel on les inocula aux deux bras, mais sans aucun effet.
Il faut donc de la sagacité pour bien distinguer, dans les effets de l'inoculation variolique sur des vaccinés, ces demi-succès du succès entier et complet qu'elle a pour l'ordinaire sur ceux qui sont encore susceptibles de prendre la petite-vérole. Il faut surtout de la bonne foi, pour ne pas en tirer précipitamment des conclusions défavorables à la vaccine. Certes, si elle garantit aussi bien de la possibilité de prendre la petite-vérole que la petite-vérole même, qui ne voit qu'elle sera infiniment préférable, soit parce qu'elle n'est pas contagieuse, soit parce qu'elle est incomparablement plus bénigne; soit enfin, parce qu'elle n'excite jamais aucune autre maladie après elle. Combien de fois au contraire, n'avons-nous pas gémi de la funeste propriété qu'a la petite-vérole, même inoculée, de développer la disposition aux scrofules; d'entraîner à sa suite des cloux, des furoncles, des maux d'yeux\footnote{Je connois une femme qui a beaucoup de réputation et encore plus d'esprit, qui tout en convenant des avantages de la vaccination, a fait en dernier lieu inoculer la petite-vérole à sa fille. Elle l'a eue heureuse; mais il lui est survenu peu de temps après une ophthalmie, à la suite de laquelle il s'est formé une tache sur la cornée, actuellement guérie à la vérité, mais bien réelle. Mon ami Maunoir qui a été consulté pour elle, me racontoit d'un autre côté, qu'ayant été appelé en même temps à voir un enfant qui avoit depuis quelques mois des maux d'yeux très-rebelles, et qui n'avoit pas eu la petite-vérole, il l'avoit vacciné, ce qui avoit si bien réussi que l'enfant avoit été guéri de ses maux d'yeux par la vaccination seule. (O)}, des maux d'oreilles, et des dépôts?\setcounter{page}{290}

DEPUIS l'impression de cet Extrait, j'ai reçu une Lettre du Dr. Thourer, Directeur de l'Ecole de Médecine de Paris, en date du 24 Brumaire, an 9, qui m'apprend que les inoculations de Vaccine s'y multiplient de plus en plus chaque jour, et dans le sein des familles; que plusieurs personnes d'âge adulte s'offrent pour ces essais; que des mères se font inoculer avec leurs enfants; que des personnages de poids et d'une grande réputation, y ont soumis les leurs etc. — Il m'annonce de plus qu'à Rheims on est parvenu à inoculer la vaccine à une vache, et que la matière qu'elle a procurée a servi à inoculer huit personnes; — que le Dr. Howell, Médecin à Boulogne-sur-Mer, en a vacciné 102, dont 7 ont été soumis ensuite à l'inoculation de la Petite-vérole, sans la contracter; qu'un autre de ces sujets a couché avec un enfant varioleux, sans éprouver la moindre infection etc. (O)
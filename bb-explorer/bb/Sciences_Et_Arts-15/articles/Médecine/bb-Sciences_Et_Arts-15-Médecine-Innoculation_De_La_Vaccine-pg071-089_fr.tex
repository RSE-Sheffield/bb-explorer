\setcounter{page}{71}
\chapter{MÉDECINE}
\section{MÉMOIRE sur l'Inoculation de la VACCINE à GENÈVE.\footnote{Quoique la plupart des faits contenus dans ce Mémoire n'aient plus un caractère de nouveauté, nous avons invité le Prof. Odier qui l'a rédigé à la suite d'une demande particulière, à permettre qu'il fût inséré en entier dans notre Recueil. Cet écrit a entr'autres mérites, celui d'offrir un résumé de tout ce qu'il faut savoir pour tirer partir de la découverte du Dr. Jenner, découverte dont les faits observés à Genève donnent la confirmation la plus satisfaisante; et il servira en même temps de réponse aux questions qui nous sont très-fréquemment adressées tant de bouche que par écrit sur l'inoculation de la vaccine. Nous nous félicitons d'avoir prévu l'importance de cette découverte lorsque, dans le voyage que nous fîmes en Angleterre pour y assurer le choix et l'exportation des matériaux de notre Recueil, nous en rapportâmes, il y a bientôt deux ans, l'ouvrage original de cet ingénieux auteur, au moment même où il venoit de paroître. Nous nous empressâmes de le faire connoître au public avant qu'aucun Journal en eût encore parlé, même en Angleterre. Nous nous estimons fort heureux de voir nos compatriotes en recueillir si promptement les avantages, et d'y avoir contribué par le soin que nous avons mis à insérer à mesure tout ce que nous avons appris à cet égard, soit des étrangers soit des Médecins Genevois, qui avec un zèle prudent, éclairé, et très-louable, ont rapidement accrédité cette pratique à laquelle un grand nombre d'enfans doivent déjà d'avoir échappé à l'épidémie de la petite-vérole, très-meurtrière depuis quelque temps. Nous continuerons à informer régulièrement nos lecteurs de tous les faits nouveaux qui parviendront à notre connoissance sur cet objet. (R)}}
Il y a 21 mois que je commençai à rendre compte, dans la Bibliothèque Britannique (Sciences et Arts, Vol. 9.), des Observations du Dr. Jenner sur la vaccine: c'est lui qui, le premier, nous a appris que cette maladie, particulière aux vaches du Comté de Gloucester\setcounter{page}{72} en Angleterre, se transmet fréquemment aux personnes qui sont employées à les traire, et les préserve pour toujours de la possibilité de prendre la petite-vérole.
Or, comme la vaccine est toujours elle-même une maladie bénigne, exempte d'éruption, sans aucun danger pour l'individu qui en est atteint, et jamais contagieuse, le Dr. Jenner avoit imaginé de l'inoculer au lieu de\setcounter{page}{73} la petite-vérole, dans l'espérance d'obtenir ainsi tous les avantages de la petite-vérole inoculée, sans aucun de ses inconvénients, et spécialement sans courir aucun risque de la répandre en multipliant les foyers de contagion. Ces espérances furent bientôt réalisées. Un grand nombre de personnes de tout âge furent soumises à cette épreuve ; elles eurent toutes une maladie extrêmement légère, exempte d'éruption, non contagieuse, et qui se trouva suffisante pour les mettre complètement à l'abri de la petite-vérole, comme on s'en convainquit en leur inoculant cette dernière maladie ; inoculation qui ne produisit jamais qu'une légère efflorescence locale et fugitive.
Les expériences du Dr. Jenner furent bientôt répétées en différens endroits de l'Angleterre, et toujours avec le même succès. Le Dr. Woodville, Médecin de l'hôpital de petite-vérole naturelle et inoculée, à Londres, crut aussi devoir essayer l'inoculation de la vaccine dans son hôpital. Il publia l'année dernière le résultat de ses premiers essais. Sur 600 personnes inoculées avec le virus vaccin, il n'en mourut qu'une ; c'était un enfant d'un an, qui, au 12e jour de l'inoculation, fut inopinément atteint d'une attaque de convulsions qui ne parut avoir aucun\setcounter{page}{74} rapport avec la vaccine. Tous les autres inoculés l'eurent plus ou moins heureuse; et tous furent ensuite exposés, à la petite-vérole par l'inoculation, sans pouvoir la prendre.
Ces faits et une multitude d'autres du même genre, dont nous eûmes connaissance l'année dernière, ne pouvoient qu'intéresser vivement les Médecins de Genève qui, depuis 50 ans, inoculent toutes les années la petite-vérole avec un succès tel, qu'on ne voit presque plus aucun habitant de cette Commune qui en soit marqué. Mais, malgré la bénignité ordinaire de la petite-vérole inoculée, nous savons cependant qu'elle est quelquefois accompagnée d'accidens effrayans, de convulsions graves, de beaucoup de fièvre, d'une éruption abondante, de temps en temps même confluente, et mortelle au moins trois fois sur mille.
Ces accidens, quoique rares en comparaison de ceux que produit la petite-vérole naturelle, nous donnoient fréquemment de l'inquiétude, et nous firent desirer de vérifier chez nous les expériences des Anglais, dont nous avions tous les jours des rapports de plus en plus satisfaisans. Un de nos compatriotes établi à Vienne, le Dr. De Carro, nous écrivit qu'il avoit reçu de Londres des fils imprégnés du virus vaccin, dont il s'étoit servi avec succès pour inoculer tant ses propres\setcounter{page}{75} enfans qu’un assez grand nombre d’autres individus. Il nous envoya quelques-uns de ces fils. Nous les essayames. Ils ne réussirent point. Il nous en envoya d’autres pris sur le bras d’un homme de 51 ans, qui, quoiqu’il eût eu la petite-vérole dans son enfance, avoit voulu se faire inoculer la vaccine pour décider une question qui s’étoit élevée à Londres sur la possibilité de prendre cette dernière maladie, après avoir eu la première. L’incision s’étoit enflammée rapidement, et avoit abondamment suppuré. Il avoit eu trois jours de fièvre, des douleurs subaxillaires, et tous les symptômes qui sembloient annoncer la vraie vaccine, quoique très-précoce. Nous essayames ces fils, dans le courant de l’automne dernier. Ils réussirent en apparence. Ils produisirent sur une vingtaine d’enfans inoculés successivement avec ce virus, une maladie singulière, qui se développoit avec une telle rapidité que, dans l’espace de sept à huit heures, le bras s’enflammoit, l’incision s’entouroit d’une large efflorescence, il survenoit de la fièvre, quelquefois même des vomissemens. Mais dans 48 heures, tout étoit fini. La rapidité de cette marche nous donna des doutes; d’autant plus que, quoique le bras de nos inoculés suppurât abondamment, ce n’étoit que par un suintement qui formoit une croûte\setcounter{page}{76} épaisse, sous laquelle se trouvoit le pus; et jaunais, comme l'annonçoient les Anglais, par une vésicule bien circonscrite, et remplie d'un fluide limpide. J'écrivis en Angleterre aux Drs. Jenner et Pearson ce qui nous arrivoit, en leur demandant d'autres fils. Ils m'en envoyèrent sur la fin du mois de floréal dernier, en m'assurant qu'ils étoient convaincus que les prétendues vaccines, que nous avions observées, ne pouvoient en aucune manière préserver de la petite-vérole.
Effectivement une nouvelle inoculation avec le virus variolique a produit sur tous nos inoculés de ce temps-là un effet complet, comme s'ils n'avoient point été inoculés avec le virus vaccin.
Mais les nouveaux fils que m'avoit envoyés le Dr. Pearson ont bien réussi; ils ont produit une maladie parfaitement semblable à celle que décrivent les Anglais; et l'extrême bénignité de cette maladie a frappé le public qui, plus éclairé chez nous que partout ailleurs, et accoutumé depuis long-temps à recevoir l'inoculation comme un bienfait, sans en méconnoître les inconvéniens, s'est promptement pénétré des avantages de cette nouvelle manière d'inoculer.
On nous a présenté de toutes parts un grand nombre d'enfans pour les y soumettre.\setcounter{page}{77} L'épidémie de la petite-vérole actuellement régnante parmi nous, jointe à l'excessive chaleur de la saison qui ne permettoit guères l'inoculation de la petite-vérole a contribué à cet empressement; et depuis quatre mois tous mes collègues et moi nous avons inoculé la vaccine à près de 400 enfants, sur lesquels nous avons été tous à portée de bien observer et les avantages de cette nouvelle méthode; et la marche de la maladie; et la nature des accidens qui la troublent quelquefois. Voici le résultat de nos observations.
\subsection{Manière d'inoculer.}
1o. Lorsque nous avons inoculé avec le fil, nous avons fait au milieu de chaque bras une incision de la longueur de deux à trois millimètres (une ligne à une ligne et demie) et tellement superficielle qu'il n'en sortit point de sang. Nous avons écarté les bords de la plaie avec le pouce et le troisième doigt; et nous y avons placé un petit bout du fil vaccin de la longueur de deux millimètres (une ligne) de manière à le loger tout-à-fait dans l'incision. Le virus vaccin se sèche sur le fil comme un vernis et devient très-cassant. C'est pourquoi il faut avoir soin que le virus ne s'en sépare pas en éclats; et pour cet effet il faut couper le fil avec un canif ou autre instrument bien tranchant; plutôt qu'avec\setcounter{page}{78} des ciseaux. Quand il est dans l'incision, on le recouvre d'une petite compresse de linge qu'on assujetit par un petit bandage. On ne lève l'appareil qu'au bout de deux ou trois jours. Nous avons cru voir que le contact des corps gras empêche l'action du virus. C'est par cette raison que nous n'appliquons point de sparadrap sur l'incision.
Nous avons aussi inoculé avec du virus desséché sur du verre, en le délayant bien avec une lancette trempée dans de l'eau froide (car le Dr. Jenner nous a avertis que la moindre chaleur détruit son activité.) Avec cette lancette bien humectée du virus délayé, on fait comme ci-dessus, une petite incision, sur laquelle on essuie bien la lancette des deux côtés et à plusieurs reprises, en écartant avec soin les bords de la plaie, de cette manière aucun appareil n'est nécessaire.
Mais, quelques précautions que l'on prenne, l'inoculation faite avec du virus vaccin desséché manque beaucoup plus fréquemment que celles qu'on fait de la même manière avec du virus variolique ; c'est pourquoi nous avons préféré, autant qu'il a été possible, d'inoculer de bras à bras, avec du virus frais, et non délayé. L'inoculation faite ainsi a presque toujours réussi, cependant elle a manqué quelquefois, et il est difficile de dire à quoi cela tient.
\setcounter{page}{79}
Le choix du moment où le virus doit être pris ne nous a point paru indifférent. Nous avons trouvé que le moment préférable est celui où l'aréole est bien formée autour de l'incision. En plongeant alors dans le bouton la pointe d'une lancette, on l'en retire sèche. On croirait d'abord, qu'il n'y a rien. Mais un instant, après une goutte d'un fluide très-limpide comme de l'eau, sort de l'ouverture. On en humecte la lancette, et l'on fait aussitôt l'incision. Car si l'on tarde, le virus se sèche très-promptement.
Lorsque nous avons inoculé avec le pus opaque et plus épais qui se trouve sous la croûte déjà formée, nous avons eu des symptômes très-précoces d'irritation locale qui dans un cas en particulier, ont eu la plus grande ressemblance avec la vaccine bâtarde dont j'ai parlé plus haut; et ont produit dans l'espace de quelques heures de la fièvre, une grande aréole autour de l'incision, et un suintement abondant.

\subsection{Marche de la maladie.}
2. Mais quand nous avons inoculé avec le fluide bien limpide qui sort du bouton dans son état vésiculaire, avant la dessication, et sans négliger aucune des précautions suggérées ci-dessus, voici quelle a été très-uniformément\setcounter{page}{80} la marche de la maladie. Pendant les quatre premiers jours, l'incision ne manifeste pour l'ordinaire aucun signe d'infection ou presque aucun. Au 5me. jour on y apperçoit un peu de rougeur et d'élévation, semblable à celle que présente ordinairement à la même époque la petite-vérole inoculée, mais plus luisante et avec une apparence vésiculaire mieux prononcée. Cette petite tumeur augmente insensiblement jusqu'au 8me. jour; et jusqu'à cette époque, elle ressemble assez à celle de la petite-vérole inoculée; mais alors il survient de la fièvre; et dès ce moment la tumeur vaccine prend le caractère qui lui est propre, c'est-à-dire, qu'elle devient mieux circonscrite, plus circulaire, plus élevée que celle de la petite-vérole inoculée; d'un jaune pâle, et à demi transparente.
La fièvre ne se manifeste guères que par l'accélération du pouls; et le malade n'est ni moins gai, ni moins actif. Il arrive cependant quelquefois que l'invasion, ou plus fréquemment la fin de la fièvre est accompagnée de mal-aise, de nausées et de vomissemens. Mais ces symptômes sont toujours fugitifs et légers. Le symptôme accessoire le plus ordinaire chez les inoculés qui ont plus de trois ans, est la douleur sous les bras,\setcounter{page}{81} douleur qu'on observe aussi dans la petite-vérole inoculée, et qui communément précède la fièvre. Mais ce symptôme est fort rare, ou du moins on ne s'en aperçoit pas au-dessous de l'âge de 3 ans. En général les très - petits enfans paroissent moins incommodés de cette maladie que ceux qui sont plus âgés. Nous n'avons vu aucun malade qui eût des convulsions, et très-peu qui eussent des soubresauts. C'est-là une des plus importantes différences qui se trouvent entre la vaccine et la petite-vérole inoculée.
Au 10me. jour, la fièvre cesse, et la tumeur s'entoure d'une belle efflorescence d'un rouge pâle, d'un à deux pouces de diamètre (de 3 à 6 centimètres), qui dure deux jours et qui quelquefois disparoit dans le centre plus promptement qu'à la circonférence. Dès que l'efflorescence est bien formée, le bouton sèche du centre à la circonférence, et se convertit en une croûte dure, épaisse, brune, ou noire, qui ne tombe qu'au bout de 20. à 30 jours, et laisse après elle un creux assez profond. Tel est le cours ordinaire de la maladie; et j'observe qu'il est beaucoup plus invariable que celui de la petite-vérole inoculée, où l'on voit fréquemment de grandes disparates dans le temps et la manière dont elle se développe.
\setcounter{page}{82}
\subsection{Inflammation Érysipélateuse.}
3. Cependant il arrive, peut-être deux ou trois fois sur 100, un accident qui n’a pas échappé aux Anglais, et que nous avons eu occasion d’observer dans sept à huit de nos malades. C’est une inflammation érysipélateuse qui s’étend promptement à plusieurs pouces de distance de l’incision, et quelquefois même sur la totalité du bras et de l’avant bras. Je n’ai point vu d’accident pareil dans ma pratique particulière. S’il s’en fût présenté à moi, j’aurois probablement mis en usage les moyens prescrits par les Anglais pour arrêter l’érysipèle, savoir, des compresses trempées dans l’eau de Goulard, ou simplement, dans de l’eau et du vinaigre. Ici, ceux de nos inoculateurs qui ont observé ces accidents n’y ont attaché aucune importance, et l’événement a justifié leur sécurité; ils n’y ont fait aucune application. Il en est résulté que dans un ou deux cas, elle s’est répandue sur tout le corps, mais sans aucune conséquence alarmante pour l’enfant. Dans le cas le plus grave de cette espèce, dont j’aie entendu parler, la vaccine n’a produit aucun autre effet. L’incision ne s’est point enflammée, et l’érysipèle qui s’étoit manifesté dès le premier jour de l’inoculation, n’a commencé qu’à trois centimètres\setcounter{page}{83} de distance. Dans d'autres, l'érysipèle, quoique très-précoce, n'a point empêché l'infection locale et générale. Dans d'autres, enfin, l'érysipèle n'est survenu qu'à la suite de l'efflorescence qui forme le dernier période de la vaccine régulière. Dans un ou deux cas nous avons lieu de soupçonner que cet accident étoit dû à quelque saleté de la lancette qui avoit été récemment aiguisée et qui étoit encore grasse.
\subsection{Eruption de taches rouges.}
4. Un autre accident qui a été fréquemment observé par le Dr. Pearson et que nous n'avons vu cependant, que deux ou trois fois, a été l'éruption de taches rouges sur différentes parties du corps. Ces taches, semblables à celles de la Fièvre ourtilière, mais sans ampoulles, ne se sont manifestées qu'après l'efflorescence. Elles ont été tout-à-fait fugitives ; n'ont été accompagnées d'aucun mal-aise, et ne se sont point transmises aux enfans inoculés d'après ceux qui les avoient eues.
\subsection{Éruptions semblables à celle de la petite-vérole.}
5. Quant aux éruptions semblables à celle de la petite-vérole, nous les avons observées dans quelques-uns de nos inoculés, àpeu-près\setcounter{page}{84} deux ou trois fois sur 100; et dans un ou deux cas, cette éruption a été fort abondante. Mais il nous a paru clairement qu'elle tenoit à l'épidémie de petite-vérole. Le Dr. Woodville avoit déjà remarqué que lorsqu'on inocule en même temps à un enfant la vaccine et la petite-vérole, les deux maladies se développent simultanément, l'une n'arrêtant point le progrès de l'autre; d'où il suit que si on inocule la vaccine à un enfant qui ait déjà le germe de la petite-vérole, celle-ci se développera, ou avant la vaccine, ou après, selon que l'enfant aura été inoculé à une époque plus ou moins éloignée du moment où il a été exposé à la contagion de la petite-vérole. Dans le premier cas, c'est-à-dire, celui où le développement de la petite-vérole précède celui de la vaccine, la maladie se manifeste exactement telle qu'elle auroit été, si l'on n'avoit point inoculé la vaccine; elle n'est point modifiée par l'inoculation de cette dernière; elle conserve le caractère qu'elle auroit eu; elle est discrète ou confluente, pétéchiale, mortelle même, suivant la disposition de l'enfant, à laquelle l'inoculation subsequente de la vaccine n'apporte aucun changement. Et dans ce cas, le développement ultérieur de la vaccine n'a pas lieu. Il ne survient point d'aréole autour de\setcounter{page}{85} l'incision. Les boutons ont d'ailleurs tous les caractères de la petite-vérole naturelle. Ils durent 9 jours. Ils ont de l'odeur. Ils répandent facilement la contagion. Nous avons eu 4 cas de cette espèce, dans lesquels la petite-vérole s'est manifestée au 4me. ou 5me. jour de l'inoculation de la vaccine, qui par cet accident est devenue inutile. Deux de ces 4 enfans en sont morts; les deux autres se sont guéris, sans qu'on pût soupçonner aucune différence entre leur petite-vérole et la petite-vérole naturelle. Si, au contraire, le développement de la vaccine précède celui de la petite-vérole, la première de ces deux maladies modifie la seconde, et la rend toujours très-bénigne, et parfaitement semblable à la petite-vérole inoculée; car la plupart des boutons avortent; les autres suppurent à la vérité, mais ne durent que 6 jours, n'ont point d'odeur, et ne sont accompagnés d'aucune fièvre secondaire. Nous avons eu 7 à 8 cas de cette espèce, dans lesquels les boutons ne sont survenus qu'après la formation de l'aréole, autour de l'incision; et dans tous ces cas la maladie a été aussi heureuse qu'elle l'est dans les cas ordinaires de petite-vérole inoculée.
Dans 5 ou 6 autres cas, nous avons vu après le développement de la vaccine, se manifester\setcounter{page}{86} sur tout le corps, des boutons semblables à ceux de la petite-vérole volante, ou plutôt à cette variété de la petite-vérole volante dans laquelle les boutons ne durent à la vérité que trois jours, mais se succèdent les uns aux autres, de manière à prolonger la maladie de plusieurs jours. Ces boutons étoient vésiculaires, remplis d'un fluide limpide comme de l'eau, et entourés à leur base d'une petite aréole. Peut-on aussi attribuer cette espèce d'éruption à l'épidémie régnante qui produit fréquemment la variole et la varicelle simultanément. Ou plutôt, doit-on les considérer comme de véritables boutons de vaccine semblables à celui qui se forme à l'incision. Je penche pour cette dernière opinion, parce que les enfans qui ont été inoculés avec le fluide limpide contenu dans ces boutons à une grande distance de l'incision, ont eu la vaccine comme s'ils avoient été inoculés avec le fluide formé à l'incision même. Mais pourquoi cette éruption générale est-elle si rare qu'on ne la voit pas deux fois sur 100. Je l'ignore: ce qu'il y a de certain, c'est qu'elle n'aggrave point la maladie.
\subsection{Certitude du préservatif.}
6. Nous avons acquis de deux manières\setcounter{page}{87} la certitude que la vaccine inoculée garantit bien surement de la petite-vérole.
1. Par la communication directe ou indirecte que tous nos inoculés vaccins ont eue nécessairement avec une grande multitude d'enfants atteints de la petite-vérole dans tous les quartiers de la ville. On sait que la petite-vérole est encore contagieuse longtemps après que les malades sont en état de sortir. Van-Swieten estime qu'elle l'est encore au bout de 60 jours après son invasion ; or, depuis le 20e jour la plupart des malades sortent, vont et viennent, se répandent librement dans les rues, dans les places publiques, dans les promenades, dans les écoles, dans les temples, etc. Il est impossible, que près de 400 enfants, auxquels on a inoculé la vaccine, depuis 4 mois, eussent tous échappés, s'ils en étoient susceptibles, à une épidémie aussi générale que celle qui règne actuellement ici, et qui a déjà fait périr dans nos murs près de 150 enfants. C'est pourtant ce qui est arrivé ; aucun d'eux n'a pris la petite-vérole, à l'exception de ceux dont j'ai parlé plus haut, et qui en avoient certainement le germe avant leur inoculation.
2. Nous avons de plus, inoculé la petite-vérole de bras à bras, et avec toutes les précautions propres à assurer le succès de\setcounter{page}{88} cette opération à 10 ou 12 de nos inoculés vaccins, et cela plusieurs semaines après la chûte des croûtes de vaccine. Aucun d'eux n'a présenté le moindre indice d'infection générale. L'incision s'est légèrement enflammée; mais elle a séché promptement, sans aréole, et sans aucun symptôme de fièvre.
\subsection{Caractère non-contagieux de la vaccine.}
7. Nous avons acquis à plusieurs reprises la preuve complète que la vaccine n'est point une maladie contagieuse. Dans plusieurs familles, nous avons inoculé deux, trois, ou quatre enfans, les uns après les autres. Ceux qui avoient la maladie ont couché avec ceux auxquels on ne l'avoit pas encore inoculée, et ceux-ci ne l'ont jamais prise que lorsqu'on la leur a inoculée à leur tour. Nous n'avons d'ailleurs vu aucun exemple quelconque de contagion.
\subsection{La vaccine n'excite aucune maladie.}
8. J'ajouterai, enfin, qu'il ne nous a pas paru qu'en aucun cas la vaccine inoculée fût suivie d'aucune autre maladie; ni clous, ni furoncles, ni maux d'yeux, ni maux d'oreilles, ni aucun dépôt, comme on en voit souvent à la suite de la petite-vérole, tant inoculée, que naturelle. Au contraire, nous avons\setcounter{page}{89} inoculé plusieurs enfans très-délicats, dont il semble que la santé ait été, jusqu'à un certain point améliorée par cette opération. Tel est le résultat sommaire de nos observations. Elles s'accordent parfaitement avec celles des Anglais, sur lesquelles je renvoie à la Bibliothèque Britannique, (Sciences et Arts, vol. 9, et suivans.) J'ai rendu dans ce Journal un compte détaillé de tout ce qu'ont publié, à cet égard, les Médecins et Chirurgiens de cette nation. Ce que nous avons vu et ce que nous voyons encore tous les jours ne nous permet pas de douter, que l'inoculation de la vaccine ne soit, et comme préservatif de la petite-vérole, et comme moyen de la détruire à la longue, une des plus belles et des plus importantes découvertes qu'on ait faites, depuis long-temps. Puissent tous les Gouvernemens s'accorder à la répandre, à la faire connoître, à l'encourager, par tous les moyens compatibles avec la liberté! C'est, peut-être, le plus grand service qu'on puisse rendre à l'humanité.
Genève, ce 4e. jour complémentaire an 8.
ODIER, Dr. et Prof.
en Médecine.
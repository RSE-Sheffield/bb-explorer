\setcounter{page}{352}
\chapter{Medecine}
\section{I. OBSERVATIONS ON THE COW-POX, etc. Observations sur la vaccine, par le Dr. W. WOODVILLE, Médecin des Hôpitaux de petite-vérole et d’inoculation, à Londres. 1800. in-8°. pp. 43.}
Nos lecteurs se rappellent sans doute que sur les 600 premières vaccinations que le Dr. Woodville avait faites dans son hôpital, il s'en était trouvé plus de la moitié dans lesquelles on avait observé une éruption générale, que le Dr. n'hésitait pas d'attribuer à la vaccine. (Voy. Bibl. Brit. Sc. et Arts. vol. XII. p. 290.) Cette opinion contredisait trop formellement les observations du Dr. Jenner pour ne pas exciter l'attention de celui-ci. Dans une 3me. brochure, intitulée : Nouveaux faits et nouvelles observations relatives à la vaccine, (Continuation of facts and observations relating to the variolæ vaccinæ) brochure dont il ne nous est parvenu que quelques fragmens par la voie des Journaux, il avait fortement insisté sur la\setcounter{page}{353} propriété qu'a le vrai virus vaccin de ne produire aucune éruption générale, permanente, ou semblable à celle de la petite-vérole. Il citoit en preuve de cette assertion tous les cas de vaccine accidentelle qu'il avoit vus dans le Gloucester-shire; toutes ses vaccinations de 1797, 1798 et 1799; toutes celles d'un grand nombre de praticiens en différens endroits de l'Angleterre, qui jamais, non plus que lui, n'avoient observé de boutons ailleurs qu'à l'endroit de l'incision. De tous ces faits il concluoit que dans les cas où il s'en étoit manifesté, il falloit que le virus employé eût été altéré ou mêlé de manière ou d'autre avec le virus variolique.
Cette conséquence, quoiqu'assez naturelle, paroît cependant erronée. Nous avons vu le Dr. Pearson entrer en lice pour la combattre. Quant au Dr. Woodville, il s'en tient personnellement offensé, elle l'affecte, elle le blesse, elle lui semble l'accuser d'incurie ou de négligence dans ses vaccinations, et le signaler à ce titre comme méritant peu la confiance du public. Il s'en plaint amèrement. Il la réfute avec une sorte d'âpreté qui fait d'autant plus de peine que dans l'exposé des faits sur lesquels il fonde sa réfutation, il y a d'ailleurs un ton de candeur et de vérité, une logique saine et vigoureuse,\setcounter{page}{354} bien suffisante pour entraîner une conviction dont on se sent presque repoussé par l'aigreur du style.
Nous n'épouserons point ces querelles. Nous nous bornerons à citer les faits. Ils sont en eux-mêmes curieux et intéressans. Mais comme cette discussion, qui tient la moitié de la brochure, est susceptible d'être fort abrégée, et qu'elle est d'ailleurs entremêlée et suivie de diverses observations qu'il importe de recueillir sur la marche de la vaccine inoculée, et sur les irrégularités dont elle est susceptible, nous prendrons la liberté, comme nous l'avons fait précédemment, de les classer sous différens chefs.
\subsection{I. Manière d'inoculer.}
L'inoculation de la vaccine manque plus fréquemment que celle de la petite-vérole, sur-tout si le virus est sec. Il ne paroît pas que ce soit là, comme on l'a cru, une preuve de sa plus grande volatilité. Car il peut conserver assez long-temps son activité quoiqu'il se sèche très-rapidement, au point de devenir solide et cassant, comme du verre ou du vernis. Mais dans cet état, il est beaucoup plus indissoluble que le pus variolique, et c'est à cette circonstance que l'auteur attribue\setcounter{page}{355} son manque fréquent de succès. C'est pourquoi il recommande de le délayer très-long-temps, et avec beaucoup de précautions pour le dissoudre complétement, au moment où l'on veut s'en servir. Il préfère l'incision à de simples piqûres \footnote{Le Dr. Aubert qui a pratiqué la vaccination dans l'hôpital même de Woodville, et qui vient de publier sur ce sujet à Paris un Rapport très-intéressant, préfère au contraire la piqûre à l'incision, parce qu'elle donne, dit-il, "une tumeur parfaitement régulière, et qu'on se ménage ainsi un diagnostic aisé." (Rapport sur la vaccine, page 60). Cependant j'ai vu fréquemment l'incision avoir le même effet, le bouton ne se manifestant pas dans toute sa longueur, mais à l'une de ses extrémités seulement. (O)} , parce que le virus la pénètre mieux, surtout si l'on a soin de tenir la lancette avec laquelle on la fait, bien perpendiculaire à la peau, et si au lieu de faire cette incision tout d'un coup, on la fait par reprises, par de simples égratignures, répétées jusqu'à ce qu'il paroisse tant soit peu de sang.
Mais, ajoute-t-il, de quelque manière qu'on s'y prenne, le succès de la vaccination est toujours jusqu'à un certain point précaire; elle manque à peu près une fois sur 60, parce qu'il y a des gens qui ne sont susceptibles\setcounter{page}{356} de prendre ni la petite-vérole, ni la vaccine\footnote{Je doute beaucoup qu'il y ait une aussi grande proportion d'individus qui ne soient pas susceptibles de prendre la petite-vérole. Mais d'un autre côté ce n'est certainement ni la seule cause qui fasse échouer la vaccination, ni même la plus ordinaire. Dans le nombre de mes vaccinés, il s'en est trouvé au moins un sur 4 chez qui il n'y a eu qu'une des deux incisions qui ait pris. Il s'en est encore trouvé à peu près 14 sur 100 que j'ai été obligé de vacciner une seconde fois, mais il ne m'est point encore arrivé d'échouer à une seconde vaccination, si ce n'est dans trois personnes auxquelles j'avois précédemment inoculé la petite-vérole à plusieurs reprises sans qu'elles pussent la prendre. J'ignore à quoi tient ce manqué de succès si fréquent dans une première vaccination. Ce n'est certainement pas à l'indissolubilité du virus, puisque j'ai toujours vacciné de bras à bras et sans délayer le virus. Au surplus je n'ai jamais inoculé que par une incision très-superficielle, sur laquelle j'ai toujours eu soin de passer à plusieurs reprises ma lancette bien imprégnée. (O)}.
\subsection{2. Développement. Fièvre. Efflorescence.}
Le temps auquel la vaccine se développe après l'inoculation, a paru à notre auteur correspondre à celui de la petite-vérole inoculée\footnote{Il y a cependant cette différence, c'est que le virus vaccin se développe beaucoup plus uniformément au 4e. ou 5e. jour. Je ne l'ai vu qu'une seule fois se développer plus tard. C'est ce qui arrive au contraire très fréquemment dans la petite vérole inoculée. (O)}.
\setcounter{page}{357}
Depuis son premier Rapport, il n'a pas vu un seul vacciné avoir aucun symptôme alarmant. Il n'est pas rare que les malades n'éprouvent aucune indisposition\footnote{Cela est vrai, cependant, il m'a paru très rare qu'ils n'aient pas au moins pendant quelques heures un peu de fièvre, c'est-à-dire, une accélération de 30 ou 40 pulsations par minute dans la fréquence du pouls. Mais souvent on ne s'en douteroit pas, cette fièvre n'étant accompagnée ni d'accablement, ni de malaise, ni d'aucun symptôme d'indisposition. Il semble même quelquefois qu'elle ne fait qu'augmenter la gaîté et l'activité de l'enfant. (O)}; mais l'efflorescence qui entoure l'incision et qui survient rarement avant le 8e. jour ou après le 11e. doit être regardée comme un symptôme d'affection générale\footnote{Je pense qu'il en est de même de la petite vérole inoculée. Il y a longtemps que j'ai énoncé mon opinion là-dessus. Cependant un auteur qui m'a probablement mal lu vient de me citer dans le Moniteur comme ayant assimilé la petite vérole inoculée, lorsqu'elle est exempte de fièvre et d'éruption, mais qu'elle produit à l'incision un bouton régulier et une efflorescence bien complète, à ces petites-véroles locales, irrégulières et toujours dénuées d’une aréole qu’on voit quelquefois résulter de l’inoculation de ceux qui ont déjà eu cette maladie ou la vaccine. Qu’il relise donc ce que j’en ai publié dans ce Journal (Sc. et Arts. vol. XI. p. 257.)}. Si le malade n'a point encore eu d'indisposition à cette époque, il peut être assuré qu'il n'en aura point.
Quand l'efflorescence ne se manifeste pas\setcounter{page}{358} avant le 11e. jour, l’indisposition est presque toujours plus forte que lorsqu’elle survient au 8e. ou au 9e. jour \footnote{Dans le cas. dont j’ai parlé plus haut, l’efflorescence fut si tardive qu’elle ne se manifesta qu’au 14e. jour, et cependant la malade âgée de 32 ans n’eut que peu de fièvre et presque aucun symptôme d’indisposition. Ce cas fut d’ailleurs remarquable par l’extrême pâleur de la vésicule qui étoit cependant très-bien formée. Mais ses bords ne rougirent point et il n’y eut aucune dureté dans le tissu cellulaire. L’efflorescence quoique pâle et passagère fut néanmoins très-grande et bien circonscrite. Elle avoit presque trois pouces de diamètre. (O)}.
Cette efflorescence est plus grande et survient plus promptement aux très-petits enfants qu’à ceux qui ont déjà trois ou quatre ans; et la maladie est chez eux beaucoup plus bénigne, au point que la plupart ne manifestent aucun signe d’indisposition et n’ont besoin d’aucun remède \footnote{C’est pourquoi les Médecins et les Chirurgiens de Genève ont cru devoir essayer de généraliser la vaccine en l’inoculant aux petits enfants peu de jours après leur naissance; ce qui dans une Commune de la grandeur de la nôtre, où le nombre des enfants qui viennent au monde est toujours au moins de 15 à 20 par décade, auroient un double avantage; l'un de se ménager une succession de vaccinés, au moyen desquels on pourroit toujours inoculer de bras à bras; l'autre de préserver d'aussi bonne heure que possible la génération future de la chance de prendre la petite-vérole en nourrice. Voici l'avis que les Ministres du culte ont bien voulu, à notre réquisition, se charger de distribuer aux parens de tous les enfans qu'on leur présentera à baptiser. Cet avis a déjà eu d'heureux effets, et les enfans qui ont été vaccinés dans un âge aussi tendre, ont eu la maladie la plus bénigne et la plus régulière. (O)}. Les adutles ont\setcounter{page}{359} souvent des maux de tête, des douleurs dans les membres, et autres symptômes de fièvre, qui durent deux ou trois jours, mais qui pourtant n'ont jamais aucun caractère de gravité qui puisse donner la moindre inquiétude. Une purgation un peu brusque soulage beaucoup en pareil cas les malades.
\comment{This footnote is not attached to the body text}
\footnote{
AVIS AUX PÈRES ET AUX MÈRES.
L'Enfant que vous présentez est exposé à un très grand danger, celui de prendre la petite-vérole, maladie qui, apportée en Europe dans le huitième siècle, s'y est tellement répandue qu'on ne peut plus l'en expulser, et qu'il est moralement impossible d'en préserver les enfans autrement que par l'inoculation.
Heureusement, par un grand bienfait de la Providence, on a découvert depuis quelque temps une manière de la pratiquer, que plusieurs milliers d'expériences ont démontrée être aussi sûre qu'efficace, et qui peut, sans aucun inconvénient, être mise en usage dans toutes les saisons et pour les enfans les plus foibles, les plus petits, et les plus délicats. Elle n'est presque jamais accompagnée ni suivie d'aucun accident; et quand il en survient, ce qui est fort rare, ces accidens ne sont jamais graves.
La maladie qu'elle produit, porte le nom de Vaccine. Elle est toujours extrêmement bénigne, et presque toujours régulière. Elle a un autre avantage très-précieux; c'est qu'elle n'est jamais contagieuse; en sorte qu'en l'inoculant chez soi, on ne risque point de la donner à ses voisins. Et cependant on peut être assuré que les enfans auxquels on l'a communiquée par l'inoculation, sont par-là toujours à l'abri du malheur de prendre la petite-vérole.
Si donc vous voulez conserver votre enfant, nous vous exhortons à le faire incessamment inoculer. Hâtez-vous de le préserver ainsi d'une maladie terrible, qui fait journellement autour de vous les plus cruels ravages, et qu'il peut prendre au moment où vous vous y attendrez le moins. Ne le mettez point en nourrice sans l'avoir soustrait à ce danger.
Ce sont des Médecins et des Chirurgiens, instruits par beaucoup d'études et d'expériences, et qui n'ont aucun intérêt de vous tromper, qui vous donnent ce conseil. Eux aussi, sont pères, et ils n'ont pas balancé à inoculer de très-bonne heure leurs enfans.
Au nom de l'humanité, et par tout ce que vous avez de plus cher, suivez leur exemple. Sinon, craignez d'avoir un jour à pleurer amèrement sur la mort de votre enfant, pour avoir repoussé le moyen de salut qu'on vous propose. Il est simple, et si facile, que vous ne sauriez avoir aucun prétexte pour vous y refuser, ni même pour différer d'y avoir recours.
Les Médecins et Chirurgiens soussignés se feront un devoir et un plaisir d'inoculer tous les enfans qu'on leur présentera, et le public sait que jamais ils n'ont exigé aucune rétribution de ceux qui ne sont pas en état de les satisfaire.
Signé, VIEUSSEUX, ODIER, VIGNIER, MANGET, VEILLARD, COINDET, DE LA RIVE, PESCHIER, Drs. Médecins. JURINE, FINE, MAUNOIR, Chirurgiens.
}
\setcounter{page}{360}
En géneral, la maladie a paru si bénigne que cette consideération a engagé un grand nombre de personnes qui ne croyoient point d'ailleurs á sa proprieété de preéserver de la petite-vérole, à faire vaacciner leurs enfans, dans l'idée que, sans leur faire courir aucun rique, cette petite opération pourroit les préparer avantageusement à l'inoculation de la petite-vérole. Or, comme en aucun cas, on n'a pu réussir á donner celle-ci à ces petits vaccinés, c'est ce qui a le plus contribué\setcounter{page}{361} à rendre la vaccination générale à Londres; d'autant plus qu'il a été bien démontré que la vaccine n'est jamais contagieuse. L'auteur cite un enfant dont le père, la mère et le frère avoient tous trois été vaccinés en même temps, et qui pendant tout le temps de la maladie n'a cessé de coucher avec eux sans la prendre, quoiqu'il en fût bien susceptible, puisqu'ayant ensuite été vacciné lui-même, il l'eut très-régulière et très-complète.
\setcounter{page}{362}
\subsection{3. Irrégularités}
\subsubsection{A. Vaccine incomplète ou bâtarde.}
Lorsqu'il survient dès le second ou le troisième jour après l'inoculation de la vaccine beaucoup de rougeur et une tumeur considérable à l'incision, le manque de succès de la vaccination est aussi certain que lorsqu'il n'en survient point du tout à l'époque ordinaire.
Elle manque aussi quoique la rougeur et la tumeur ne soient point prématurées, lorsqu'il ne s'y forme, ni bouton ni vésicule, mais qu'au 6e. ou 7e. jour, la tumeur suppure irrégulièrement et ne produit qu'une croûte ulcérée. Il ne faut pas confondre ce cas avec celui où un bouton ou une vésicule bien formée, s'ulcère et se convertit en une croûte semblable. Dans ce dernier cas, le bouton vésiculaire ne dura-t-il qu'un jour ou deux, pourvu qu'il soit bien formé, le succès de la vaccination est tout aussi assuré que lorsque la marche de la maladie est parfaitement régulière \footnote{J'avais vacciné un enfant de 7 à 8 ans. Le développement de la vaccine se fit d'abord par un bouton assez rouge et élevé, mais sec. Il n'y eut une apparence vésiculaire qu'au 7e. ou 8e. jour. Cette vésicule se rompit promptement et dégénéra dans l'espace d'un jour ou deux en une croûte ulcérée, ce qui n'empêcha pas la manifestation d'une belle efflorescence au 9e. ou 10e. jour. J'eus de l'inquiétude sur cette vaccination. Je la répétai pour plus de sécurité. Mais elle n'excita aucun signe d'infection. Ce cas confirme l'observation du Dr. W.; mais il tend à prouver aussi qu'on ne peut pas avoir la vaccine deux fois; et c'est ce que j'ai encore vu dans un autre enfant, dont la vésicule se rompit et dégénéra de même en un ulcère, mais sans aucune efflorescence. Je vaccinai cet enfant une seconde fois, mais beaucoup plus long-temps après la première que dans le cas précédent. Cette seconde vaccination n'eut aucune suite. (O)}.
\setcounter{page}{363}
Cette distinction est le résultat d'une grande expérience. Pendant long-temps ces irrégularités ont inquiété l'auteur. Il a souvent réinoculé deux, trois, et jusqu'à quatre fois de suite ceux de ses vaccinés chez lesquels il voyoit le bouton dégénérer ainsi en une croûte ulcérée; mais enfin, de nombreuses observations lui ont appris que le succès de la vaccination dépend entièrement de l'apparence vésiculaire du bouton, bien prononcée, quoique fugitive; et que d'un autre côté, lorsque cette apparence manque, il faut recommencer l'opération.
Il avoit vacciné deux enfans, l'un de quatre ans; l'autre de 18 mois; celui-ci eut une\setcounter{page}{364} vaccine régulière; l'autre eut au 4\textsuperscript{e}. jour, une tumeur et une rougeur de quatre lignes de diamètre autour de l'incision, mais sans apparence vésiculaire. Au 6\textsuperscript{e}. jour la tumeur et la rougeur augmentèrent beaucoup, et se convertirent en un ulcère superficiel qui dura deux ou trois jours. Alors la dureté et l'inflammation cessèrent. L'auteur annonça qu'il falloit réinoculer cet enfant, à la grande surprise de ses parens qui l'avoient trouvé beaucoup plus malade que son frère. Cette seconde inoculation produisit une vaccine très-régulière.
On a plusieurs exemples de semblables irrégularités dans la petite-vérole inoculée, dans laquelle on a souvent été trompé par de fausses apparences\footnote{C'est ainsi que le célèbre Prof. Tissot, qui avoit cru trop légèrement sur l'autorité de Dimsdale et d'autres inocu'ateurs, que le moindre signe d'infection par l'inoculation suffit pour garantir de la petite-vérole, fut cruellement trompé. Il avoit inoculé un enfant qu'il chérissoit. L'inoculation n'eut qu'un demi succès. Malheureusement il s'en contenta. L'enfant prit quelque temps après la petite-vérole et en mourut. J'ai ouï dire que ce triste événement avoit répandu la plus grande amertume sur les dernières années de Mr. Tissot, et qu'il n'avoit jamais pu s'en consoler. (O)}. L'auteur en cite un. C'étoit un enfant de cinq ans qui mourut de\setcounter{page}{365} la petite-vérole confluente, quoiqu'il eût été inoculé trois ans auparavant. Mais l'inoculateur n'avait pas été satisfait, et voulait réinoculer l'enfant. Le père s'y opposa.
\subsubsection{B. Eruptions.}
Lorsque le bouton vésiculaire se convertit en croûte, il survient quelquefois des boutons épars sur tout le corps, mais qui ne durent pas au-delà d'un ou deux jours, et suppurent rarement. Lorsqu'ils suppurent, ils se prolongent comme ceux de la petite-vérole, auxquels ils ressemblent parfaitement.
Ces boutons, que le Dr. W. avait observés si fréquemment dans ses premières vaccinations, doit-on les attribuer au virus employé dans l'inoculation qui les produit? Ce virus, n'est-il pas à juste titre, suspect d'altération ou de mélange avec le virus variolique, et ne peut-on pas regarder l'éruption comme le résultat d'une maladie hybride produite par un virus mélangé?
L'auteur repousse fortement cette opinion par les raisons suivantes:
1°. L'expérience lui a appris que si l'on inocule en même temps la vaccine et la petite-vérole, à un enfant, en faisant même les deux incisions à un pouce de distance l'une de l'autre,\setcounter{page}{366} de manière à produire au 9e. jour, une efflorescence commune à toutes deux, le virus pris sur le bouton vaccin ne donnera jamais que la vaccine, et aussi heureuse, aussi exempte de boutons que si on n'avoit inoculé que la vaccine et qu'on eût pris le virus sur une vache, tandis que le virus pris sur le bouton variolique donnera une petite vérole exempte de toute apparence de vaccine, et bien caractérisée. Le Dr. rappelle ici les expériences par lesquelles il a prouvé que le virus vaccin mêlé avec le variolique ne produit point une maladie hybride, (voy. la Bibl. Brit. Sc. et Art. vol. XII. p. 158.) et il ajoute que le Dr. Marshall \footnote{J'apprends par une lettre adressée au Dr. J. F. Baumgartner, en date du 8 novembre, que ce Médecin qui donnoit avec succès des leçons de médecine à Londres, et qui avoit d'ailleurs une pratique particulière assez étendue, est parti pour Naples, où il se propose d'introduire la vaccination; nouvelle raison d'espérer qu'elle deviendra bientôt générale. (O)}, ayant par inadvertance vacciné cinq individus avec une lancette imprégnée de virus variolique, il n'avoit obtenu qu'une vaccine très-heureuse et sans boutons.
2°. Parce que le virus le plus pur et le mieux choisi, soit par le Dr. Jenner luimême,\setcounter{page}{367} soit par d'autres praticiens de la campagne, pris directement et à plusieurs reprises sur le pis d'une vache, soit dans les environs de Londres, soit dans le Gloucestershire, a produit tout aussi fréquemment des boutons et en grand nombre, dans l'hôpital, que celui qu'on prenoit sur le bras d'un vacciné couvert lui-même de boutons, tandis que ce dernier virus employé à la campagne n'en a produit que bien rarement. En 1799, le Dr. Jenner manquoit de virus. Il en demanda au Dr. qui lui en envoya; ce virus fut pris sur un vacciné de l'hôpital qui avoit 310 boutons. Le D. Jenner fut très-content des effets qu'il produisit. Aucun des enfans à la vaccination desquels il fut employé n'eut l'apparence d'aucune éruption. — Le Dr. Marshall en prit sur quelques-uns de ces enfans, et s'en servit pour vacciner successivement pendant cinq semaines 107 individus, qui tous eurent une vaccine heureuse et sans boutons. — Le Rev. Mr. Holt de Finmere a aussi vacciné, avec un virus pris à l'hôpital, plus de 300 personnes, qui toutes ont eu la maladie régulière et sans boutons, excepté deux seulement qui en eurent chacun une centaine. — Le Rev. Mr. Finch à qui Mr. Holt avoit fourni de ce même virus, en a vacciné 714 individus avec le plus grand succès\setcounter{page}{368} et sans qu'aucun d'eux ait eu la moindre apparence d'éruption\footnote{J'aime à voir les Ministres du culte faire servir ainsi l'ascendant que leur donne leur ministère à dissiper les préjugés, à répandre les vérités utiles, et à rendre à leurs paroissiens un service aussi éclatant que celui de mettre leurs enfants à l'abri d'une contagion également meurtrière et inévitable. C'est ainsi que dans les Indes les Bramines descendent tous les printemps des montagnes pour inoculer, au nom du Dieu de miséricorde qu'ils adorent, ceux des fidèles qui n'ont pas eu la petite-vérole. C'est ainsi que les Jésuites missionnaires dans le Brésil et dans le Paraguay arrachèrent par l'inoculation un immense nombre de victimes à cette maladie que les Européens leur avaient apportée et qui désolait ces malheureux pays. Il serait bien à désirer que tous les ecclésiastiques eussent le bon esprit d'encourager les pères et les mères à profiter de bonne heure des préservatifs que leur offre la Providence contre cet épouvantable fléau. Je ne sais pourquoi la plupart des Curés catholiques de nos environs n'ont cessé depuis longtemps de prévenir leurs paroissiens contre l'inoculation comme si c'était offenser la Divinité que de repousser par tous les moyens possibles une maladie aussi désastreuse que la petite-vérole. S'offense-t-elle de ce qu'on prend du kina pour se guérir de la fièvre? Sait-elle gré aux Turcs de l'insouciance avec laquelle la funeste doctrine du Fatalisme leur fait repousser tous les moyens de se débarrasser de la peste? (O)}.
3°. Le virus employé par le Dr. Jenner lui-même, a donné 12 à 20 boutons permanens,\setcounter{page}{369} parfaitement semblables à ceux de la petite-vérole, à un malade qu'il avoit vacciné à la campagne, et qui vint ensuite se faire soigner à la ville par Mr. Cotton, de qui l'auteur tient cette anecdote. — Mr. Ring lui a dit aussi, que sur 30 individus vaccinés par lui, avec du virus que lui avoit donné Mr. Paytherus, et que celui-ci tenoit du Dr. Jenner, il y en eut un qui eut 150 boutons purulens, opaques, permanens et parfaitement semblables à ceux de la petite-vérole. Le virus le plus pur peut donc en produire, et le virus le plus suspect n'en produire aucun.
4º. Enfin, le pus pris sur ces boutons communique par l'inoculation, non la petite-vérole, mais la vaccine\footnote{Le Dr. Aubert nous apprend que cela n'est point général. L'on a produit dans l'hôpital tantôt la petite-vérole et tantôt la vaccine, en inoculant avec le pus pris sur les boutons qui surviennent aux vaccinés qui ont été exposés aux émanations varioliques (Rapport sur la Vaccine, p. 33.) Il en conclut que quoiqu'il n'y ait aucune différence apparente quelconque entre les uns et les autres, les uns appartiennent cependant à la petite-vérole communiquée par contagion à l'individu vacciné et les autres à la vaccine même. Il me semble qu'on peut concevoir la chose autrement. Car puisqu'il est prouvé par les expériences du Dr. que le pus variolique mêlé avec le virus vaccin, ne donne point une maladie hybride, mais tantôt la petite-vérole et tantôt la vaccine, on peut regarder le pus contenu dans tous ces boutons comme étant vraiment du pus variolique mêlé avec le virus vaccin en supposant que ce dernier s'absorbe toujours et circule avec la masse du sang, mais n'est susceptible de se déposer et de se reproduire que là où il rencontré sur la peau l'espèce d'inflammation nécessaire à son assimilation. C'est pourquoi dans les cas ordinaires de vaccine sans mélange, il ne produit point de boutons. Mais lorsque la contagion variolique en a déjà produit, il s'y accumule et s'y multiplie, d'où résulte un pus mélangé qui donne tantôt l'une et tantôt l'autre des deux maladies. (O)}. Ce n'est donc pas un pus variolique.
\setcounter{page}{370}
Cependant, le D. W. convient, qu'on ne peut attribuer ces boutons qu'aux émanations varioliques, parce qu'on ne les observe presque jamais que dans les endroits où l'on peut supposer l'air imprégné de ces émanations. C'est ainsi que dans son hôpital où les vaccinés respiroient le même air qu'une multitude d'inoculés de la petite-vérole qui y séjournoient en même temps qu'eux, ces boutons se sont manifestés d'abord très-fréquemment. Mais ensuite quand la vaccination y est devenue plus générale, et que l'on y a pris plus de précautions pour écarter les émanations varioliques, les boutons purulens y ont été beaucoup plus rares. Sur 310 personnes\setcounter{page}{371} vaccinées dans cet hôpital pendant le mois qui suivit la publication de son premier Rapport, notre auteur n'en a vu que 39 qui eussent des boutons; savoir 19 sur les 100 premières, 13 sur les 100 suivantes, et 7 seulement sur les 110 restantes. Dès-lors il y en a vacciné plus de 2000, qui toutes ont eu la maladie très-heureuse et sans éruption, à l'exception de 3 ou 4 sur 100 seulement. Dans sa pratique particulière il n'a pas vu un seul de ses vaccinés qui eût de semblables boutons.
Mais on en voit assez fréquemment dans les villes ou les villages où la petite-vérole est épidémique. Dans un village qui se trouvoit dans ce cas là à 8 milles de Londres, le Dr. vaccina plus de 100 personnes. Il y en eut plus de 20 qui eurent une éruption générale. D'autres praticiens ont fait la même observation; et Mr. Evans, chirurgien de Ketley en Shropshire, ayant vacciné 68 personnes dans une maison d'inoculation où il y avoit en même temps un grand nombre d'inoculés de petite-vérole, vit plus de la moitié de ses vaccinés avoir des boutons suppurans par tout le corps. Comment les émanations varioliques les produisent-elles? C'est ce que l'auteur ne prétend point déterminer.
Au surplus, on peut souvent produire à\setcounter{page}{372} volonté de pareils boutons par des piqûres faites avec du virus vaccin, ou du virus variolique frais, aux endroits où l'on veut en faire venir. C'est ce que l'auteur a fait quelquefois pour complaire aux parens qui désiroient en voir à leurs enfans: pour que les piqûres produisent cet effet, il faut les faire au 8e. ou 9e. jour, lorsque l'efflorescence commence.
\subsection{CONCLUSION.}
Lorsqu'on étoit encore partagé, il y a environ 80 ans, sur les avantages de l'inoculation de la petite-vérole, le Dr. Jurin réduisoit la solution du problème aux deux questions suivantes: 1. La petite-vérole inoculée est-elle beaucoup plus bénigne que la petite-vérole naturelle? 2. Préserve-t-elle efficacement de la petite-vérole?
En changeant les termes de la première de ces deux questions, c'est-à-dire, en y substituant la vaccine à la petite-vérole inoculée, et celle-ci à la petite-vérole naturelle, le Dr. qui d'abord avoit hésité sur cette première question, ne balance plus à répondre affirmativement à l'une et à l'autre, en faveur de la vaccine, comparativement à la petite-vérole inoculée. Car depuis son premier\setcounter{page}{373} Rapport, aucun de ses vaccinés n'a éprouvé le moindre accident; et plus de mille d'entr'eux ont subi l'épreuve de l'inoculation variolique, sans en être plus affectés que s'ils avoient eu la petite-vérole.

\section{2. SOME OBSERVATIONS ON VACCINATION, &c. Observation sur la Vaccination, où inoculation de la Vacciné; par RICHARD DUNNING, Chirurgien et Membre de la Société Médicale de Plymouth. Londres 1808. in - 8o. pp. 122.}
LES avantages que promet la pratique de la vaccination, surtout lorsqu'elle sera devenue générale, sont si frappans, ils intéressent si fort l'humanité que pour peu qu'on y réfléchisse, on a peine à contenir un mouvement d'indignation contre ceux qui affectent sur cette grande question, de l'indifférence ou de l'incrédulité, sans daigner se donner la peine d'examiner avec candeur et impartialité jusqu'à quel point elle est décidée par les faits."
"Mais combien ne sont pas plus coupables encore ceux qui saisissent toutes les occasions de décréditer la nouvelle inoculation\setcounter{page}{374} par des sarcasmes, par des insinuations mensongères, par des calomnies, ou même par des voies de fait encore plus criminelles? Car ce n'est pas en France ou en Allemagne seulement, c'est en Angleterre, dans le pays même où le Dr. Jenner annonçoit sa découverte avec tant de réserve et de modestie, qu'il s'est trouvé des hommes assez inconsidérés, disons mieux, assez ennemis du genre humain pour se charger d'un rôle aussi méprisable."
C'est contre leurs odieuses menées que la colère d'un honnête chirurgien de Plymouth, qui a vieilli dans son métier, lui met pour la première fois de sa vie la plume à la main, afin de les pulvériser par le récit des succès qu'a obtenus la vaccination dans la ville où il exerce ses talens, et où elle a été introduite, tant par ses soins que par ceux du Dr. Remmet, dont il parle avec une grande vénération.
"La conduite de nos antagonistes seroit, dit-il, inexcusable, quels que fussent leur rang et leurs occupations dans la société. Mais quand on considère que ce sont, pour la plupart, des gens qui ont reçu une éducation libérale; qui professent un art le plus incompatible de tous avec les suggestions de l'indolence ou d'un sordide intérêt; des gens, en un mot, que\setcounter{page}{375} leur station dans le monde rend les dépositaires de la santé publique, il est difficile de concevoir une manière d'agir aussi coupable, ou pour mieux dire aussi insensée. Car enfin, il est évident que bientôt la vaccination sera si généralement admise et pratiquée, que les clabauderies de quelques gens de l'art ne sauraient prévaloir contre elle, et tourneront infailliblement à leur honte."
Tel est, à peu près, le début de notre auteur. Nous ne le suivrons ni dans ses déclamations, pour la plupart exagérées, ni dans ses raisonnemens souvent trop captieux, ni dans l'exposé des faits généraux qu'il rapporte et qui ne sont qu'une répétition de ce qu'avoient publié avec plus de calme les auteurs dont nous avons déjà analysé les écrits. Il suffira de faire connoître à nos lecteurs, le petit nombre de faits nouveaux et intéressans qu'il a lui-même observés.
\subsection{I. Vaccine naturelle.}
. La vaccine naturelle est une maladie inconnue dans le Devonshire. Cependant, il paroît par une lettre du Rev. Smith à l'auteur, qu'elle s'y est manifestée une fois. Anne Stuttaford, actuellement femme du clerc de sa paroisse, se trouvant en service, il y a 30 ans, chez un fermier nommé Mr.\setcounter{page}{376} James Callard, à trois milles de Plymouth, les vaches de la ferme, au nombre de 16, prirent toutes une maladie qu'on ne connoissoit point dans le pays; cette maladie se manifestoit par des pustules sur leur pis, qui les rendoient très-difficiles à aborder; et elle dura environ dix jours. Pendant ce temps-là, Anne S. fut constamment occupée à les traire avec deux autres femmes qui avoient eu la petite-vérole, et qui ne furent point affectées de la maladie des vaches. Mais elle qui ne l'a jamais eue, ne tarda pas à avoir un gros bouton sur la main, semblable à ceux des vaches, et qu'elle prit pour un clou. Elle y mit un onguent qui ne lui fit aucun bien, et n'empêcha ni le bouton de grossir et de s'enflammer, ni la malade d'avoir de la fièvre et un grand mal de tête pendant quelques jours. Mais dès lors elle s'est toujours bien portée, et a vu et soigné un grand nombre de malades atteints de la petite-vérole, sans la prendre.
C'est là le seul exemple de vaccine naturelle, connu dans le Devonshire
\section{2. Introduction, et succès de la vaccination à Plimouth.}
Mais, dès que le Dr. Jenner eut annoncé sa découverte, notre auteur s'empressa de lui\setcounter{page}{377} écrire. Il en reçut des réponses fort obligeantes qui dissipèrent tous ses doutes. Dès lors le Dr. Remmet et lui, travaillèrent à introduire la vaccination dans le Devonshire. Un respectable Ecclésiastique, Mr. Hitchings, voulut bien lui confier son enfant, âgé de trois mois, pour le faire vacciner. L'heureuse issue de cette première opération frappa le public; et l'exemple d'un homme aussi considéré fut d'un grand poids pour accréditer la nouvelle inoculation. A peine quelques mois se furent écoulés que plusieurs centaines d'individus y furent soumis dans Plymouth avec le plus grand succès. Tous eurent la maladie la plus bénigne. L'auteur n'en vit aucun dans sa pratique qui eût ni plus de 24 heures d'indisposition, ni ulcères rongeans, après la chûte de la croûte, ni éruption générale.
Il décrit la marche de la maladie dans toutes ses périodes, comparativement avec celle de la petite-vérole inoculée, qui est souvent une maladie grave, comme il l'a éprouvé lui-même dans sa famille, et qui exige d'ailleurs, de l'avis des principaux inoculateurs, quelque préparation et des remèdes\footnote{J'en employois autrefois. Mais il y a bien des années que j'en ai reconnu l'inutilité. Je tiens du fameux Sutton qu'il ne les considéroit lui-même que comme un moyen de capter la confiance des parens. Depuis long-temps j'y ai complètement renoncé. Je me contente de faire prendre à mes inoculés un léger purgatif le lendemain de l'inoculation et de les priver de viande et de vin. Quant à mes vaccinés, je ne leur ai jamais rien prescrit de semblable. (O)}. Il entre dans\setcounter{page}{378} une longue discussion sur cet objet, rejette les préparations trop compliquées, mais avoue cependant, qu'il donne toujours lui-même quelques grains de mercure combiné avec des absorbans\footnote{Hydrargyrus cum cretâ. C'est une préparation de la pharmacopée de Londres. Elle consiste uniquement à bien triturer trois parties de mercure avec cinq parties de craie, jusqu'à ce que le mercure soit complètement éteint.  (O)}, deux fois par jour, tant avant qu'après l'inoculation, et un ou deux purgatifs avant la fièvre éruptive ; au lieu que la vaccination n'exige aucun régime préparatoire, ni aucun remède, si ce n'est un léger purgatif, à l'époque de la fièvre, lorsqu'elle est considérable\footnote{Je n'ai point eu occasion d'observer l'utilité des purgatifs dans la fièvre de la vaccine, n'ayant jamais vu aucun vacciné qui fût assez indisposé pour exiger aucun remède à cette époque.  (O)}. Il recommande encore des compresses trempées dans de l'eau et du vinaigre, comme ne lui ayant jamais manqué pour arrêter l'inflammation érysipélateuse qui survient autour de l'incision, lorsqu'elle est\setcounter{page}{379} trop étendue. Nous ne le suivrons pas dans tous ces détails, parce qu'ils ne nous ont paru contenir aucune observation nouvelle.
\subsection{3. Éruption générale.}
Nous préférons arrêter un instant l'attention de nos lecteurs sur l'histoire de deux enfants que l'auteur fut appelé à voir, après leur vaccination par un autre chirurgien. Ces enfants faisaient partie d'une nombreuse famille, dont un autre enfant avoit, depuis la veille, une éruption abondante de petite-vérole naturelle; et c'est ce qui avoit engagé à vacciner, sur le champ ses deux frères, pour les mettre à l'abri de la contagion, car il étoit impossible de les séparer, et ils demeuroient tous dans la même chambre. Or, des deux vaccinés, l'incision avoit dans l'un tous les caractères de la petite-vérole, et dans l'autre ceux de la vaccine. C'est, dit l'auteur, que le premier avoit déjà le germe de la petite-vérole dans le sang; ensorte que la vaccination n'agit sur lui que comme une simple égratignure qui prit l'apparence variolique \footnote{Le Dr. Woodville a vu de même quelquefois dans ses premières vaccinations le bouton vaccin prendre complétement l'apparence variolique; et dans ces cas là, la maladie a été beaucoup plus grave qu'à l'ordinaire. Ici, je n'ai rien vu de semblable; mais le Dr. Veillard, qui a vacciné un grand nombre d'individus fort exposés à la contagion variolique, m'a dit en avoir vu quelques-uns dont l'efflorescence étoit recouverte d'une multitude de boutons confluens comme cela se voit souvent dans la petite-vérole inoculée. Dans ces cas là, il y a eu pour l'ordinaire une éruption variolique abondante. Il me paroît évident que cette apparence n'est jamais que le résultat d'une complication des deux maladies. (O)}, de la même manière que dans la\setcounter{page}{380} petite-vérole naturelle ou inoculée. Les égratignures accidentelles s'enflamment pour l'ordinaire davantage, et s'entourent d'un plus grand nombre de boutons que partout ailleurs\footnote{C'est ce que j'ai vu fréquemment; mais d'une autre côté, j'ai vu une fois un effet contraire. L'avois inoculé un enfant, qui depuis sa naissance avoit sur l'un des hypochondres une de ces taches rouges et relevées qu'on appelle communément des envies. Cette tache ayant une apparence vésiculaire avoit été ouverte et s'étoit trouvée remplie d'une espèce de gelée rouge qui s'étoit promptement régénérée et recouverte d'une pellicule légère. L'enfant eut par l'inoculation une petite-vérole fort abondante. Mais on observa qu'à trois ou quatre pouces de circonférence autour de la tache, il n'y avoit pas un seul bouton. Il sembloit qu'elle les eût repoussés. (O)}, au lieu que le cadet ayant échappé à la contagion de la petite-vérole, avoit subi le cours ordinaire de la vaccine inoculée, sans mélange.\setcounter{page}{381}
Quelquefois encore, on voit les vaccinés avoir des boutons remplis d'une sérosité limpide. Ces boutons peuvent résulter d'une complication de la vaccine avec la petite-vérole volante. L'auteur en cite un exemple frappant, qui lui a été communiqué par Mr. Little. Celui-ci avoit vacciné au même moment et avec le même virus deux enfans, l'un âgé de cinq semaines, et l'autre de trois ou quatre ans. Le plus jeune prit d'abord l'infection. Il fallut revacciner l'autre, de son frère. Au 8e. jour il survint de la fièvre. Les incisions s'enflammèrent comme à l'ordinaire, avec le caractère de la vaccine. Mais au 10e. jour, il survint par tout le corps de gros boutons remplis d'une sérosité limpide; c'étoit bien la petite-vérole volante; car le plus jeune la prit dès le lendemain. La vaccine est donc susceptible de se compliquer avec d'autres maladies éruptives, et cela peut faire illusion.
L'auteur est cependant porté à croire que quelquefois la seule irritation de la vaccine peut produire sur tout le corps une sorte d'éruption fugitive, semblable à celle que produisent certains poisons par leur action sur l'estomac. Quoiqu'il n'en ait jamais vu lui-même dans sa pratique, il en cite deux cas, l'un du Dr. Remmet, et l'autre de Mr. Little. Mais ces boutons, loin de pouvoir être considéres\setcounter{page}{382} comme spécifiques, ou de suivre le cours de ceux de petite vérole, avortent pour la plupart, très promptement; ou s'ils sup purent, ils parcourent si rapidement tous leurs périodes qu'ils se dissipent en moins de trois jours.
\subsection{4. Effets de la vaccination sur la santé.}
Les enfans les plus foibles et les plus va lětudinaires peuvent être vaccinés sans dan ger. On en voit même dont la santé sem ble se fortifier par cette opération. C'est ce que prouve l'histoire d'une jeune fille qui venoit de perdre son père et son oncle de la phthi sie pulmonaire. Elle même étoit extrême ment délicate, sujette à de fréquens vomis semens, ayant habituellement beaucoup d'op pression, le teint très pale et cadavéreux, le visage parsemé de taches livides. Personne as surément n'auroit osé lui inoculer la petite vérole dans cet état de santé. Mais la mère, que la perte de son mari et d'un autre en fant venoit de plonger dans le désespoir, voulant à tout prix sauver sa fille unique du danger de l'épidémie qui entouroit sa mai son, et avoit déjà fait périr un grand nom bre d'individus, le Dr. Remmet qui la soi gnoit fit vacciner cet enfant, avec un tel succès\setcounter{page}{383} que non-seulement la vaccine fut très-heureuse et très-bénigne, mais encore que dès-lors l'enfant recouvra graduellement en peu de mois la meilleure santé possible.
Notre auteur n'ose pas affirmer que cette guérison doit être attribuée à la vaccination; mais il le soupçonne, parce qu'il a vu un autre enfant de deux ans, toujours très-délicat, et qui convalescent d'une inflammation de poitrine, étoit encore très-foible, pâle et oppressé, recouvrer très-promptement, après avoir été vacciné, ses forces, son embonpoint, un bon teint, une respiration libre et facile, en un mot, un état de santé beaucoup meilleur qu'auparavant\footnote{Un enfant vacciné par M. Maunoir avoit le bras couvert de petites taches dartreuses. Chacune de ces taches s'enflamma et produisit un bouton vaccin, après la dessication duquel les taches disparurent. Au reste, ce n'est pas seulement la vraie vaccine qui améliore la santé des enfans foibles et délicats. J'ai vu la vaccine bâtarde produire le même effet sur un enfant qui étoit depuis long-temps sujet à l'insomnie et à la diarrhée. L'incision s'enflamma dès le jour même de l'inoculation. Le lendemain il avoit beaucoup de fièvre et une grande efflorescence autour de l'incision. Mais jamais il n'avoit été plus gai et plus actif que ce jour là; la fièvre et l'inflammation durèrent deux jours et eurent un effet très-marqué sur sa santé, qui fut dès-lors beaucoup meilleure. Je lui ai depuis inoculé la petite-vérole qu'il a supportée fort heureuse. (O)}.
\setcounter{page}{384}
L'auteur n'a d'ailleurs jamais vu la vaccination avoir aucune suite fâcheuse, comme cela n'est que trop commun après la petite-vérole, même inoculée.
\subsection{5. Certitude du préservatif.}
Quant à la question de savoir si la vaccine préserve bien sûrement de la petite-vérole, il a inoculé celle-ci à plusieurs de ses vaccinés, sans avoir jamais produit le plus léger signe d'infection. Il cite à cette occasion un cas frappant. Il avoit inoculé la petite-vérole à un fils du capitaine Birkhead, âgé de 4 mois, et en même temps à l'enfant de sa nourrice, âgé de 2 ans. Le premier, ne manifestant au bout de cinq à six jours aucun signe d'infection, fut transporté à l'autre extrémité de la ville pour le mettre à l'abri de la contagion, et vacciné. Quand la vaccine fut bien développée, il fut de nouveau ramené vers sa nourrice, dont l'enfant avoit alors une centaine de beaux boutons varioliques, en pleine suppuration; ces deux enfans couchèrent dès ce moment dans le même berceau, et ni l'un ni l'autre ne prit la maladie de son compagnon.
Ici l'auteur se fait la question que tant de gens se plaisent à répéter aujourd'hui pour\setcounter{page}{385} inspirer des préventions contre la vaccine. Qui sait, disent-ils, si la propriété préservatrice de cette maladie contre la petite-vérole n'est pas limitée à un petit nombre d'années? On ne l'a inoculée que depuis trois ou quatre ans. Comment peut-on être assuré que dans 5 ou 6 ans les individus vaccinés ne deviendront pas de nouveau susceptibles de prendre la petite-vérole? Attendons d'avoir 50 ou 60 ans par devers nous pour décider ce problème. Après avoir réfuté victorieusement cette objection, et par les nombreux exemples de vaccine naturelle très-ancienne dont la propriété préservatrice contre la petite-vérole s'est constamment manifestée pendant toute la vie des individus qui en avoient été atteints, même 40 ou 50 auparavant, et par l'analogie frappante qu'on observe entre la vaccine et la petite-vérole inoculée, lorsque celle-ci est aussi heureuse qu'il est possible, analogie qui doit porter à croire que puisque celle-ci a incontestablement, malgré sa bénignité, la propriété de mettre pour toujours à l'abri de la petite-vérole, celle-là doit jouir aussi de la même propriété : mais qu'importe, ajoute notre auteur, profitons de ces cinq ou six ans pour rendre la vaccination générale. Il n'en faut pas davantage pour exclure totalement la petite-vérole de la Grande-Bretagne.
\setcounter{page}{386}
Car enfin, la vaccine n'est jamais contagieuse. Elle ne laisse après elle aucun foyer susceptible de perpétuer la maladie. Cela seul devroit la faire adopter généralement, comme un moyen assuré de parvenir à éteindre ce fléau, lors même qu'elle ne surpasseroit pas de beaucoup la petite-vérole inoculée en bénignité. Mais des preuves irrésistibles démontrent qu'elle est si peu dangereuse, que sur plusieurs milliers d'individus vaccinés, il n'en est mort jusqu'à présent qu'un seul dans le cours ordinaire de la vaccination; ensorte que leur probabilité de vie a paru infiniment supérieure à celle qu'ils auroient eue si on ne les avoit pas vaccinés.
"Qui pourroit réfléchir," s'écrie notre auteur, "à la grandeur d'un pareil bien-fait, sans s'étonner que tandis qu'on voit tous les jours la Nation consacrer par des témoignages publics d'admiration et de gratitude le souvenir de nos brillantes victoires et la mémoire des hommes illustres, qui dans ces grandes occasions ont si habilement et si vaillamment dirigé nos flottes et nos armées, on n'ait jusqu'à présent rien proposé de ce genre pour célébrer la découverte de la vaccination, et honorer son trop heureux auteur? Jamais homme a-t-il rendu un service plus signalé à sa patrie et à la postérité? La découverte même de\setcounter{page}{387} la circulation du sang par notre célèbre Harvey a beaucoup moins contribué que la sienne aux progrès de la médecine; et si l'on consulte les annales de l'histoire, on peut affirmer, qu'à l'exception du Christianisme, jamais elles n'ont fait mention d'une victoire plus grande, plus utile et plus importante au genre humain. Il n'en est aucune qui mérite mieux d'être rappelée d'âge en âge par un monument; il seroit digne des illustres médecins qui sont à la tête de notre Faculté d'en poser la première pierre. La beauté et son aimable cortège, l'honoreroit d'un sourire d'approbation; la bienveillance l'éclaireroit de ses plus doux rayons; l'humanité applaudiroit avec énergie; et rien ne contribueroit plus efficacement à réduire au silence les antagonistes de cette importante découverte qu'un pareil témoignage de la reconnoissance publique en faveur du Dr. Jenner."
\section{3. FAMILIAR OBSERVATIONS, etc. Observations familières sur l'inoculation de la vaccine, par ALEXANDRE HERMAN MAC-DONALD. D. M. Hambourg. 6 octobre 1800. 4to. pp. 48.}
Nous ne faisons qu'annoncer cette brochure, sans en présenter l'extrait, parce que quoiqu'assez\setcounter{page}{388} bien faite et très récente, elle ne contient presque rien que nos lecteurs ne connoissent déjà. C'est un simple résumé des observations de Jenner, Pearson, Woodville et de quelques médecins Allemands dont nous avons déjà parlé. L'auteur y cite plusieurs exemples d'individus morts de l'inoculation de la petite-vérole; d'autres qui sont morts de la même maladie prise naturellement, parce qu'on n'avoit pas osé la leur inoculer, soit à cause qu'ils étoient d'une santé foible et délicate, soit parce qu'on ne pouvoit les séparer d'autres personnes de la famille atteintes d'infirmités qui faisoient redouter plus particulièrement pour elles la contagion; d'autres enfin chez lesquels l'inoculation variolique a eu des suites formidables. Il convient de la rareté de ces accidens; mais c'est pourtant à eux qu'il attribue l'effroi qu'inspire encore ce préservatif à une multitude de gens. Il fait voir qu'on n'a rien de semblable à craindre de la vaccination, et il en presse avec chaleur les avantages pour l'expulsion totale de la petite-vérole.
Desine quapropter novitate exterritus ipsâ,
Expuere ex animo rationem, sed magis acri.
Judicio
\setcounter{page}{389}
Judicio perpende ; et si tibi vera videtur, Dede manus \footnote{Cessez donc de vous 'laisser épouvanter par "la nouveauté de la chose. Ne repoussez pas le " moyen qui vous est offert ; mais plutôt examinez- "le avec d'autant plus d'attention et d'impartialité ; " et s'il vous paroît bien fondé , rendez-vous. "}.
C'est l'épigraphe qu'il a choisie.

\setcounter{page}{176}
\chapter{MÉLANGES.}
\section{RÉSULTATS SINGULIERS D'UNE EXPÉRIENCE sur la végétation dans l'eau, communiqués à la Société de Physique et d'Histoire naturelle de Genève, par le Prof. M. A. PICTET, l'un de ses Membres.}
Dans le but d'observer la marche de l'absorption de l'eau par un rameau que l'on forceroit à végéter dans ce liquide, comme aussi pour reconnoître bien exactement l'augmentation de poids que recevtoit le rameau, de l'eau que je supposois qu'il auroit gardée, dans le développement de sa végétation, je procédai de la manière suivante.
J'employai un tube de verre recourbé en forme de syphon renversé; l'une des branches de ce tube demeurant verticale, l'autre étoit légèrement inclinée et se terminoit un peu au-dessus de la première. Je lutai bien exactement, au haut de la branche verticale, avec un lut gras recouvert de plusieurs tours de boyau humecté, l'extrémité inférieure d'un\setcounter{page}{177} rameau de lilas fraîchement coupé, long de 246 millimètres; son diamètre, qui était aussi le diamètre intérieur du tube, était de 6 ⅓ millimètres; et son poids, de 0,394 d'once. Le rameau portait sept boutons, opposés deux à deux, et alternans, comme ils le sont dans cet arbrisseau; ces boutons étaient seulement verdoyans, mais point encore ouverts. Le tube avait été rempli d'eau; ce liquide était en contact avec la section du rameau, au haut de la branche verticale du syphon, et il s'élevait, dans la branche oblique, jusques à son niveau, qui répondoit environ à un pouce au-dessous de l'extrémité ouverte du syphon. Là, j'avais fait une marque, pour remplir le tube chaque jour à la même hauteur, à mesure que la végétation, qui absorbait l'eau dans l'autre branche, la faisait descendre dans celle-ci. Le tube et le rameau étaient attachés à un support léger; et le tout ensemble était mis chaque jour sur le bassin de l'une de ces admirables romaines de Paul, si commodes pour les expériences délicates, et qui, chargée de l'appareil, accusait fort bien les millièmes d'once. Je l'établis sur la cheminée de mon cabinet, dont la température, pendant la durée de l'expérience, ne fut pas Agricult. Vol. 19. N°. 5. Mai 1814.\setcounter{page}{178} au-dessous de 13° ni au-dessous de 15° de la division octogésimale. Le rameau fut en expérience le 1er avril, à neuf heures du matin; et chaque jour, à la même heure, je mesurois sur la branche oblique du tube la quantité dont le niveau de l'eau s'étoit abaissé en vingt-quatre heures; je remplaçois la portion absorbée; je pesois bien exactement l'appareil; et je prenois note de ces résultats, comme aussi de la marche de la végétation dans le déploîement des boutons.
Dans les premières vingt-quatre heures, l'absorption fut de 75 millimètres, mesurés sur le tube; on commença à apercevoir des feuilles dans le développement des boutons. Le second jour, l'absorption fut de 107 millimètres; le rameau végétoit avec vigueur; les feuilles du bouton terminal supérieur avoient 38 millimètres de long, à partir de la base du bouton. Le troisième jour il y eut 148 et demi millimètres d'eau absorbée, et je commençai seulement ce jour là à peser l'appareil, dans le but d'observer l'augmentation journalière de son poids, correspondante à une portion de l'eau qu'il auroit absorbée. Je remarquai à cette occasion, qu'il ne s'en falloit que de 4 p. g. environ, que les millimètres de longueur de la colonne d'eau dans le tube, ne répondissent exactement à des millièmes d'once.
\setcounter{page}{179} L'absorption, dans le quatrième jour, fut de 161,5 millim., pesant 0,167 d'once. A ma bien grande surprise, lorsque je pesai l'appareil, après avoir remplacé dans le tube la quantité absorbée, je la trouvai précisément du même poids que la veille. Il y avoit eu cependant un développement considérable dans la végétation. J'attribuai ce résultat à quelque erreur inaperçue, je suspends toute conclusion à cet égard.
Le lendemain, cinquième jour de l'expérience, les feuilles s'étant développées jusques à la longueur de 47 ½ millimètres dans le bouton supérieur, l'absorption d'eau ayant été précisément la même que la veille (0,167 d'once), je trouvai, après le remplissage ordinaire, le poids total de l'appareil, non pas augmenté mais diminué de 0,008 d'once.
Le jour suivant, absorption, 0,147 d'once; les feuilles supérieures ont 50 millim. de long, et sont portées par une pousse ou branche qui se développe. -- Poids total diminué de 0,006 d'once. Le rameau sortant du premier bouton en descendant, étoit au moins aussi développé que celui du bouton supérieur.
Le lendemain, les feuilles de ce rameau inférieur dépassent les autres; elles ont 57 ½ millim. de long. L'absorption est de 161. Le\setcounter{page}{180} poids de l'appareil se trouve cette fois augmenté de la même quantité précisément dont je l'avois trouvé diminué la veille, c'est-à-dire, de 0,006 d'once.
Le jour suivant absorption 0,173 d'once ( la plus considérable que j'aie observée dans toute la durée de l'expérience ) diminution de poids, précisément égale à l'augmentation de la veille.
Voici quelques observations générales faites pendant le cours de l'expérience.
La végétation du rameau entier; quoique passablement vigoureuse ne marchoit pas du même pas que celle de l'arbrisseau duquel on l'avoit coupé. Le vert des feuilles étoit plus pâle, le 11˚ le bord de quelques-unes annonçoit une maladie, un commencement de dessication ; le 15 toutes les feuilles paroissoient souffrir; les inférieures étoient un peu recoquillées, et desséchées au bout; les rameaux cessèrent de s'allonger.
Le 17 les pousses inférieures étoient en état de dessication, la supérieure soit terminale seule, verdoyoit encore; la pointe de sa plus longue feuille étoit à 71 millim. de sa base.
Le 19 toutes les pousses inférieures sont mortes. La supérieure vit toujours.
\setcounter{page}{181}
Le 21 les feuilles inférieures de celle-ci commencent à se dessécher.
Le 25 elles sont aux $\frac{2}{3}$ desséchées; je termine l'expérience. En voici le résultat.
Après avoir sommé d'une part les augmentations de poids, de l'autre les diminutions, c'est-à-dire, les $+$ et les $-$ dans les différences observées et enregistrées chaque jour, je trouvai que la somme des acquisitions du rameau ne s'élevoit qu'à 0,013 d'once, et celle de ses pertes, à 0,157. Différence ou diminution totale, en vingt-un jours = 0,144 d'once.
Cependant, la somme des absorptions journalières d'eau s'étoit élevée à 2,440 millimètres = 2,537 onces; c'est-à-dire, plus de deux onces et demie. Le rameau avoit produit des pousses longues de plus de 60 millimètres, sur lesquelles des feuilles s'étoient déployées, et néanmoins pendant cette période de végétation et d'absorption du liquide, le poids du rameau, loin de s'accroître, a diminué en tout, de 0,144 d'once, c'est-à-dire, de plus d'un tiers de son poids primitif; diminution qui a eu sur-tout lieu dans les dernières périodes de l'expérience, dans lesquelles la végétation étoit fort affoiblie. Mais à l'époque où elle étoit la plus active, et lorsque le rameau absorboit jusqu'à 0,170 d'eau en vingt\setcounter{page}{182} quatre, on l'a vu diminuer de poids de 9 millièmes d'once dans ce même intervalle.
Il paroîtrait donc que, pendant la durée de cette végétation, l'eau ne laissant rien de sa propre substance dans les parties qui se développaient; que son rôle se bornait à servir de véhicule aux solides déjà existants dans le rameau, et qui, rendus mobiles, abandonnaient la tige pour venir se déployer à mesure, et de jour à jour, sous l'apparence de rameaux et de feuilles, selon un type ou moule, propre sans doute à chaque végétal et qui décide les formes et le port de la plante.
L'eau, après cette fonction, s'évaporait en totalité, même en général en quantité plus considérable que celle absorbée.
Ensorte qu'on pourrait considérer le rameau pendant la durée de l'expérience comme soumis à l'action de deux forces, l'une absorbante, l'autre évaporante.
Elles étoient presque en équilibre pour leurs effets, pendant la période d'énergie de la force vitale du rameau; et la dernière prit le dessus à mesure que l'autre s'affaiblissait.
Pour déterminer l'effet absolu de celle-ci, je pesai au bout de vingt-quatre heures le rameau laissé à lui-même et sans contact d'eau.
Il perdit 0,016 d'once de son poids dans cet intervalle.
\setcounter{page}{183} Cependant, depuis la fameuse expérience du saule de Vanhelmont, répétée et variée depuis par plusieurs physiologistes, entr'autres par notre illustre compatriote Bonnet, on sait qu'une plante qui peut vivre dans l'eau seule, y acquiert tous ses développemens, et tout le poids par lequel ils sont représentés.
Mais il est possible qu'un arbrisseau, tel que celui dont j'avais mis un rameau en expérience, ne puisse se nourrir véritablement que de solides empruntés à la terre, ou à lui-même, s'il s'agit d'un rameau. Les doutes à cet égard pourraient être facilement levés par une suite d'expériences faites sur le plan de celle dont je viens de donner les détails.
\setcounter{page}{140}
\chapter{MÉLANGES.}
\section{LETTER FROM A NORTH COUNTRY PLOUGHMAN, etc. Lettre d'un Laboureur du nord de l'Ecosse sur les concours de labourage. ( Farmer's Magazine, N°. 55.)}
A l'Editeur du Farmer's Magazine.
MR.
Vous serez surpris sans doute de recevoir une lettre d'un laboureur des contrées septentrionales de l'Ecosse. Et vraiment, si ce n'étoit que je me considère comme un peu intéressé dans l'objet de ma lettre, il est probable que vous n'auriez jamais entendu parler de moi; car, ainsi qu'on l'a sagement observé "l'intérêt est l'aiguillon le plus puissant pour faire agir,"; mais trêve d'apologie; au fait.
On me permet quelquefois, Mr. l'Editeur, de mettre le nez dans votre excellent Recueil, et je vous assure que j'ai beaucoup profité de cette lecture, dont je n'ai pourtant joui qu'à la dérobée. On me prêta, dans une de ces dernières soirées, le dernier numéro, et en l'ouvrant (jugez de mon transport, car je suis un vrai patriote!) j'y trouvai\setcounter{page}{141} le détail des primes distribuées cette année, pour l'encouragement de l'agriculture, par cette institution si animée de l'esprit public, la Société des Highlands! Ce que j'ai donc à dire se rapporte aux primes qu'on donne aux laboureurs, et à l'utilité de ce genre de concours.
Jusques à une époque, qui ne date que de peu d'années, j'étois le champion déterminé de ces tournois; mais je commence à douter qu'ils aient véritablement les bons effets que leur attribuent quelques amateurs qui ne sont pas praticiens; sans nier pourtant tout-à-fait qu'ils n'aient offert quelques avantages. Mais pensons un moment à la variété de terrains qu'on rencontre dans presque tous les districts de ce Comté; sera-t-il raisonnable de supposer que tous ces terrains devront être labourés de la même manière? Tout le monde conviendra que non, et sans hésiter.
Voilà déjà un puissant argument contre ces concours de charrues. Supposons-en, par exemple (ce qui n'est point rare) une trentaine, prêtes à entrer en lice dans un même champ. Y en aura-t-il seulement une moitié qui soient construites dans les mêmes principes, indépendamment des différences dans la force des chevaux et l'adresse des laboureurs? Une même charrue est-elle également\setcounter{page}{142} susceptible de s'adapter aux divers sols; et la différence, s'il en existe, dépend-t-elle de la manière dont elle est guidée? Peut-être faudroit-il que les concurrens connussent bien d'avance la manière de labourer toutes les variétés de sols du pays qu'ils habitent, pour éviter les plaintes qu'on les entend articuler dans certaines circonstances. Rien de tout cela n'a lieu.
Dès que le jour du concours est fixé, on avise ceux d'entre les laboureurs de deux ou trois paroisses contigues, qui sont réputés les plus habiles, qu'ils aient à se tenir prêts à montrer leur savoir-faire et celui de leurs bêtes.
Malheureusement c'est un sol où probablement la plupart d'entr'eux, sans oublier votre très-humble serviteur, n'ont jamais eu l'occasion de s'exercer et dont ils n'ont pas labouré un acre dans leur vie. Ce n'est pas tout; à peine sommes-nous sur place que des légions de tisserands, de cordonniers, de tailleurs, etc. chacun dans son costume, accourent de toutes parts, au grand effroi de nos chevaux et presque de nous-mêmes. La supériorité de l'ouvrage est ici bien plus à l'honneur de ceux d'entre ces pauvres animaux, qui ne se laissent pas troubler par ce spectacle inusité, qu'au talent de leurs conducteurs; la plupart des chevaux effarouchés, marchent en zig-zag, au désespoir du laboureur qui n'en peut mais.
\setcounter{page}{143} Si j'osois donner mon avis, voici le plan qui me sembleroit le meilleur pour inspirer l'émulation à mes confrères, et favoriser les progrès de l'art du labourage.
Je proposerois que, dans tout district où ces progrès ont besoin d'être encouragés, et où il existe des corps patriotiques disposés à exercer de cette manière leur influence bienfaisante, on nommât une espèce de Commission, de deux ou trois habiles fermiers, qui seroient chargés de parcourir certains districts immédiatement avant les sémailles, ou avant les cultures à la houe; ou à ces deux époques, d'examiner en détail les labourages de chaque ferme, et d'en faire le rapport par écrit; en se bornant, si l'on veut, à celles d'entre ces fermes où le labourage leur paroîtroit avoir été exécuté d'une manière très-supérieure. Ces rapports seroient remis aux Juges chargés d'adjuger les primes d'encouragement.
Je ne vois qu'une objection; et elle me suggère l'idée que je vais mettre en avant. Si l'on distribue ainsi les primes, comment graduer le mérite individuel du laboureur, dans le cas où plusieurs auroient travaillé dans le même champ? Pour remédier à cet inconvénient, il faudroit que tous les fermiers qui ont plus d'un laboureur, assignassent à chacun d'eux un terrain particulier, qui seroit\setcounter{page}{144} soumis à l'inspection des commissaires. Cette précaution, loin d'être une fatigue de plus pour le fermier seroit au contraire un amusement, qui lui procureroit un avantage réel par l'émulation que cette division du travail établinoit entre ses laboureurs : Ce plan auroit encore l'avantage de contribuer à faire maintenir les charrues, leur train, et les attelages constamment en bon état. Au lieu que par suite de l'usage des concours, j'ai connu de mes confrères qui, après avoir gagné un prix, prenoient une si haute idée de leur mérite, qu'à peine daignoient-ils écouter leurs maîtres lorsqu'ils leur faisoient quelque observation, alléguant que puisqu'ils avoient remporté une récompense ils étoient au moins aussi bons juges qu'eux. Le succès dans un concours est d'ailleurs bien souvent dû au hasard d'une circonstance particulière; et tel laboureur qui échoue ce jour-là, l'auroit emporté toute l'année sur ses compétiteurs.
Je dis donc en somme ; que dans les concours, on prétend récompenser la capacité, et on ne récompense ni l'attention, ni la persévérance, ni la sobriété ; qu'au contraire, en suivant le plan dont j'ai tracé l'esquisse, on récompenseroit à-la-fois la capacité, et les autres qualités qui la garantissent toujours, ou qui du moins en assurent la constance.
Je termine mon épître en affirmant que nous autres laboureurs formons un corps très-utile et bien respectable : et que nous méritons qu'on nous parle notre propre langage.
Un laboureur du Nord.
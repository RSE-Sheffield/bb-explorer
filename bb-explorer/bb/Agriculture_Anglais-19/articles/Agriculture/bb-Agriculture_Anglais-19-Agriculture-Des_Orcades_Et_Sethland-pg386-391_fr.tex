\setcounter{page}{386}
\chapter{Agriculture}
\section{A GENERAL VIEW OF THE AGRICULTURE OF THE ORKNEY AND SHETLAND ISLANDS. \large{Tableau général de l'agriculture des Isles Orcades et Shetland par JOHN SHIRREST (Edimbourg 1814) (Extrait du Farmer's Magazine, août 1814).}}
Ce RAPPORT, quoique l'un des derniers que l'Ecosse ait fourni au Bureau d'agriculture, n'est pas un des moins dignes de l'attention du public; car outre les détails propres à piquer la curiosité des agriculteurs de profession, il en renferme d'autres, qui offrent à tous les lecteurs le même genre d'intérêt que les voyages lointains. On ne connoît guère mieux, même en Angleterre, cette partie presque polaire des domaines britanniques, qu'on ne connoît les isles de la mer du Sud; et l'état de la société, les mœurs et les habitudes y sont très-différens de ce qu'on observe à ces divers égards dans les provinces de l'Angleterre; qui est comme un continent pour ces\setcounter{page}{387} insulaires arctiques. "Dans l'Amérique septentrionale, (a dit un célèbre écrivain de nos jours) un voyageur qui part d'une grande ville où la civilisation est fort avancée , passe par tous ses degrés descendans et par tous ceux d'une industrie décroissante, jusqu'à-ce qu'en un voyage de quelques jours seulement, il arrive à la hutte grossièrement formée d'arbres qu'on vient
d'abattre. Les observations de ce voyageur offrent une sorte d'analyse pratique, à l'aide de laquelle on peut remonter à l'origine des nations et à la constitution des États.
On part de la combinaison la plus compliquée pour arriver aux plus simples élémens; on parcourt en reculant, l'histoire des progrès de l'esprit humain; et on trouve, dans un espace circonscrit, des résultats qu'on ne pouvoit guère attendre que du cours des siècles."
Ces remarques trouvent leur application, même dans les limites des iles britanniques. Les mœurs et la civilisation des habitans des Orcades et des iles Shetland en sont actuellement à-peu-près au terme où se trouvoient, il y a quelques siècles, les provinces du sud de l'Angleterre, et que presque toutes ont laissé bien loin en arrière. Il peut être intéressant aussi pour les agriculteurs en particulier,\setcounter{page}{388} de retrouver, comme dans un tableau vivant, la condition de leurs ancêtres, qui ressembloit à beaucoup d'égards à celle des cultivateurs actuels dans ces isles. Enfin il paroît, d'après le Rapport, qu'on peut puiser, même dans les pratiques grossières de leurs habitans, des procédés utiles, et dont les fermiers peuvent tirer avantage dans les provinces les mieux cultivées de l'Angleterre.
L'auteur des Extraits dont nous tirons la substance des nôtres a classé sous trois chefs les matières qui en font les objets. 1°. Le climat et les circonstances naturelles du pays. 2°. L'état actuel de son agriculture et de son économie rurale. 3°. Les perfectionnemens dont elle est susceptible et les moyens de les obtenir.
Les isles Orkney, connues des anciens sous le nom d'Orcades, sont situées au nord de l'extrémité septentrionale et orientale du comté de Caithness, dont elles sont séparées par le détroit de Pentland (Pentland Frith) large d'environ douze milles. Vingtneuf de ces isles sont habitées; et trente-huit autres isles ou ilots n'offrent que des pâturages, qui nourrissent quelquefois des bestiaux. Ces isles sont en général de formes très irrégulières et dentelées, ce qui procure\setcounter{page}{389} cure de bonnes rades, et d'excellens abris pour les vaissaux. Aucune de leurs terres élevées ne mérite le nom de montagne: on y trouve quelques collines qui s'élèvent environ à mille deux cents pieds au-dessus de la mer. On porte la surface totale de ces isles à trois cent quatre-vingt-quatre mille acres; dont environ quatre-vingt-quatre mille sont cultivés, et les trois cent mille restans, en friche. Le climat est humide et variable, les étés sont courts et peu chauds; les hivers longs mais pas sévères. Le printemps ne commence guère qu'en juin. Une circonstance prouve la douceur du climat en hiver, c'est que les turneps n'y sont point détruits par la gelée; ils ne se pourrissent pas, même à moitié mangés par les brebis ou les lapins; la plaie se referme, et le reste de la racine demeure aussi sain que si elle n'eût pas été attaquée. La partie la plus considérable du sol de ces isles est en bruyères, et en sol tourbeux; on y trouve aussi d'assez bonnes terres; et environ un tiers des terres arables est sabloneux et mêlé en beaucoup d'endroits, de coquillages marins en décomposition. On rencontre assez souvent de la pierre calcaire, mais rarement en quantité considérable dans un même lieu.\setcounter{page}{390} et elle est mêlée de grès. Pour en faire de la chaux, on la casse en morceaux de la grosseur d'un œuf, et on la calcine au feu de tourbe. On emploie rarement, ou pour mieux dire, jamais, la chaux comme engrais; son unique usage est pour la composition du mortier à bâtir. On trouve de la marne en abondance, mais l'exploitation est fort négligée.
La population, en 1811 était de vingt-trois mille deux cent trente-huit, moindre de mille deux cent sept qu'en 1802. Le nombre moyen des habitants par mille carré dans toutes les Orcades est de 38 ¼.
Les îles de Shetland, au nombre de cent, dont plus de trente sont habitées, sont situées environ soixante-milles au nord des Orcades.\footnote{Elles sont toutes situées au-delà du 66° degré de latitude.}. Le sol y est plus inégal et plus montueux que dans les Orcades; et leurs côtes sont plus abruptes. On y compte vingtcinq mille acres de terres arables, et plus de quatre cent mille de terres incultes. Le climat et le sol ne diffèrent pas essentiellement de ceux des Orcades. La pierre calcaire qu'on y trouve presque partout en abondance, est de bonne qualité; mais la chaux et la marne\setcounter{page}{391} n'y ont aucune valeur comme engrais. La population est de vingt-trois mille ames, qui répandues exclusivement, sur le sol cultivé, ne donnent pas moins de quatre cent soixante personnes par mille carré.
(La suite dans un prochain Cahier.)

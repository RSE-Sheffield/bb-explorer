\setcounter{page}{331}
\chapter{Agriculture}
\section{HINTS REGARDING THE CORN LAWS. Quelques idées sur la législation des grains. Par Sir JOHN SINCLAIR ( Farmer's Magazine ).}
APRÈS tout ce qui a été écrit sur la législation des grains, il semble que la question doit être bien éclaircie.
L'essentiel est que nous ayons des grains pour être indépendans des nations ennemies, ou qui peuvent le devenir. Nous avons eu à souffrir beaucoup d'un système différent. Depuis vingt ans, il nous en a coûté à-peu-près soixante millions sterling en achat de grains; c'est-à-dire que nous avons épuisé notre numéraire, encouragé l'agriculture de nos ennemis, et augmenté leurs ressources financières à nos dépens; si notre agriculture avoit été meilleure; si nos communaux avoient été défrichés, nous aurions conservé chez nous une partie de cet immense trésor, mais le fait est que nous ne pouvons ni améliorer notre agriculture actuelle, ni fertiliser nos communaux sans des dépenses très-considérables, en sorte que les prix\setcounter{page}{332} des grains sont nécessairement élevés. Les productions de nos terrains fertiles, lesquels peuvent être cultivés à peu de frais, ne suffiroient pas à nourrir la moitié de la population de la Grande-Bretagne. Nos terres stériles ne peuvent pas être cultivées à si peu de frais, et ce seroit faire un faux calcul que de les mettre en concurrence quant à leur produit, avec les pays étrangers qui nous fournissent des grains.
Rien ne peut être plus fâcheux pour les moyennes et les basses classes de la société que cette fluctuation dans le prix des blés. Elles ne peuvent point proportionner leur dépense à leurs revenus. Si le blé est à très-bon marché, les ouvriers manufacturiers, n’étant plus obligés à un travail assidu pour gagner leur vie, se livrent ou à la paresse ou à des fantaisies extravagantes, au préjudice de ceux qui les emploient et au leur propre. Si au contraire, le blé est à un prix très-élevé, ils tombent dans une profonde misère. Mais si par un système bien entendu on rendoit le prix du blé moins variable, on éviteroit pour ces classes les extrémités dans lesquelles les jette cette impossibilité de calculer, avec quelque certitude, l’avenir.
\setcounter{page}{333} Il faudroit donner la préférence à l'industrie nationale sur les produits étrangers. Mais l'intérêt des négocians et des manufacturiers s'oppose avec force à ce système. Qu'on prohîbe les laines étrangères, qu'on mette des droits considérables sur les étoffes de soie, de coton, de fil importées, qu'on assure à nos fabricans la préférence de leurs étoffes dans les marchés, ils trouveront cela très-politique; mais tandis qu'ils jouissent de ces prérogatives, on les trouvera très-opposés à une mesure qui, en excluant les blés étrangers, encourageroit l'agriculture de l'Angleterre. Assurément ce n'est pas équitable. Si on veut donner quelque encouragement aux classes agricoles et manufacturières, la première doit certainement avoir la préférence, puisqu'elle est, en quelque sorte, attachée au sol, tandis que l'autre abandonnera pour un sol étranger dès que son intérêt s'y trouvera. Il seroit à désirer, au moins, qu'on les fît jouir des mêmes avantages, des mêmes privilèges, et qu'on ne les sacrifiât pas l'une à l'autre.
La classe agricole auroit droit à être indemnisée pour les pertes qu'elle a éprouvées, en supportant ces classes commerciales et manufacturieres. Le lourd fardeau de la dette\setcounter{page}{334} nationale, et les fortes taxes qui en dérivent, sont en grande partie, la suite des efforts qu'on a faits pour soutenir le commerce, et pour procurer des débouchés à nos produits manufacturés. Combien de millions n'ont pas coûté à l'agriculture ces colonies, ces flottes, ces armées destinées à la défense du commerce? Mais quoique l'industrie agricole ait beaucoup souffert par ces causes, c'est un fait bien reconnu qu'elle n'a point obtenu de dédommagement pour ces pertes. Ses revenus demeureront stationnaires, tant qu'elle sera chargée des frais de la guerre et du commerce.
Dans un état qui prospère, la valeur de l'argent, comme moyen d'échange, perd de sa valeur, en même temps que le travail et tous les objets de consommation haussent dans la même proportion. Dans cet état de choses, on voudroit que les produits agricoles ne suivissent pas la même progression. Cela n'est nullement raisonnable.
Tout bien considéré, l'intérêt des propriétaires et des fermiers, enfin, l'intérêt public demandent que dans la législation des grains, on ait égard aux principes suivans. 1°. Il faut une source de subsistances qui soit indépendante des étrangers. 2°. Des prix fixes.\setcounter{page}{335} 3°. Donner toujours la préférence à l'industrie nationale sur l'industrie étrangère. 4°. Des indemnités aux fabriques et au commerce. 5°. Une augmentation de droits sur l'entrée des grains, proportionnée à la hausse de prix des autres articles d'importation. Si l'on agit d'après ces principes, le pays prospérera. Les produits des manufactures trouveront des marchés dans l'intérieur, et elles deviendront plus indépendantes des étrangers. Rien ne paroît plus absurde que d'employer nos capitaux à faire fabriquer pour des nations étrangères, qui ne peuvent nous rembourser qu'en nous envoyant des blés, au détriment de notre agriculture; car l'entrée de ces grains étrangers, nuit essentiellement à l'industrie agricole de notre pays.
On ne peut donc, sans injustice, empêcher le fermier, non-seulement de diminuer les gages et le prix du travail, mais encore d'exporter ses laines, ses grains et toutes les productions de son sol. Il faut de plus, que toutes les marchandises importées, comme les soies, les laines, les cotons, les toiles, le fer, les porcelaines, soient libres de tout impôt. Un tel monopole entraîne une injustice, à moins qu'il n'y ait réciprocité. Ce n'est qu'en adoptant un tel système, que l'on convaincra\setcounter{page}{336} les négocians et les fabriquans de l'injustice de ces mesures, qui sont d'ailleurs impolitiques. Ils sentiroient bientôt que toutes les classes qui composent une grande nation doivent se soutenir les unes les autres, et que cette mesure, quoiqu'elle ne semble avantageuse qu'à une seule, deviendroit par la suite utile à toutes. Surtout, il est du devoir de ceux qui importent, ou qui manufacturent, de ne pas nuire aux intérêts de l'agriculture; car que deviendroient-ils, si les fermiers, leurs plus grands consommateurs, étoient réduits à un état de pauvreté et de misère.
\setcounter{page}{229}
\chapter{AGRICULTURE}
\section{ELEMENTS OF AGRICULTURAL CHEMISTRY, etc. Élémens de chimie-agricole en un Cours de leçons pour le Département d'Agriculture ; par Sir HUMPHRY DAVY. Londres, 1813. \large{( Dixième extrait. Voy. p. 195 ).}}
L'USAGE de la chaux comme mortier n'est pas un sujet convenable de discussion étendu dans un cours de chimie-agricole ; mais comme la théorie de l'opération de la chaux n'est complétement développée dans aucun ouvrage élémentaire à moi connu ; je dirai quelques mots des applications.
La chaux agit comme ciment ; soit par sa combinaison avec l'eau, soit par sa combinaison avec l'acide carbonique.
J'ai déjà parlé de l'hydrate de chaux. Lorsqu'on fait de la pâte avec de la chaux vive et de l'eau, la chaux se durcit promptement en une masse solide et cohérente, laquelle contient dix-sept parties d'eau, et cinquante-cinq de chaux. Si l'on mêle à cet hydrate\setcounter{page}{230} de chaux; avant qu'il soit bien consolidé, de l'oxide rouge de fer, de l'alumine ou de la silice, le mélange acquiert plus de dureté et de cohérence, que si l'on employoit la chaux seule: il paroît que cela est dû à un certain degré d'attraction chimique entre l'hydrate de chaux et ces corps: car l'hydrate est ensuite moins susceptible de décomposition par l'action de l'acide carbonique, et moins soluble dans l'eau.
La base de tous les cimens qu'on emploie dans les ouvrages qui doivent être couverts d'eau, est de l'hydrate de chaux: les pierres à chaux impures sont d'un usage très-convenable pour cela. La pouzolane est principalement composée de silice, d'alumine, et d'oxide de fer. On la mélange avec la chaux pour faire des cimens à employer sous l'eau. Dans la construction du fanal d'Edystone, Mr. Smeaton a employé un ciment composé de parties égales en poids de chaux éteinte, et de pouzolane. La pouzolane est une lave décomposée. Le tarras, que l'on importoit autrefois de Hollande en quantité considérable, n'est qu'un basalte décomposé: deux parties de chaux éteinte, et une partie de tarras forment principalement le mortier employé dans les digues de Hollande. On trouve abondamment dans les Isles Britanniques,\setcounter{page}{231} des substances qui feroient le même effet. La chaussée des Géans, dans le nord de l'Irlande, peut fournir le tarras rouge, en quantité indéfinie : on trouve en Ecosse, et dans le nord de l'Angleterre, où le charbon abonde, beaucoup de basaltes en décomposition. Le ciment de Parker, et les autres cimens de même espèce que l'on fait aux mines d'Ahun de lord Dundas, et de lord Mulgrave, sont des mélanges de pierres ferrugineuses calcinées, avec de l'hydrate de chaux. Les cimens qui agissent en se combinant avec l'acide carbonique, c'est-à-dire, les mortiers ordinaires, se font en mêlant de la chaux éteinte avec du sable. Ces mortiers se durcissent d'abord comme des hydrates, et sont ensuite convertis peu-à-peu en carbonate de chaux, par l'action de l'acide carbonique de l'air. Mr. Tennant a trouvé qu'un mortier de cette espèce, exposé trente-neuf mois à l'air, avoit regagné les soixante-trois pour cent de gaz acide carbonique qui constituent sa proportion dans le carbonate de chaux. Les débris de mortier des maisons abattues doivent principalement leur influence bienfaisante sur la végétation, au carbonate de chaux qu'ils contiennent, et au sable qui s'y trouve. La dureté du mortier des bâtimens trèsvieux\setcounter{page}{232} dépend de sa parfaite conversion en carbonate de chaux. Les pierres à chaux les plus pures sont les meilleures pour cette espèce de mortier. Les pierres à chaux magnésiennes font d'excellent ciment pour l'eau, mais elles agissent avec trop peu d'énergie sur le gaz acide carbonique, pour faire de bon mortier ordinaire.
Pline dit que les Romains faisoient leur meilleur mortier un an avant que de l'employer; ensorte qu'il étoit déjà partiellement combiné avec le gaz acide carbonique, avant qu'on en fit usage.
Dans la calcination de la pierre à chaux, il y a certaines précautions à prendre, lesquelles different selon les pierres. En général, une mesure de charbon de pierre suffit à faire quatre à cinq mesures de chaux. La pierre à chaux magnésienne demande moins de combustible que la pierre à chaux ordinaire. Si l'on calcine une pierre qui contienne beaucoup d'alumine ou de silice, il faut bien prendre garde que le feu ne soit pas trop fort, car cette chaux se vitrifie aisément, à cause de son affinité avec la silice et l'alumine. Dans les endroits où toutes les pierres à chaux sont mélangées, il faut y faire beaucoup d'attention. On peut faire d'assez bonne chaux à un feu à peine\setcounter{page}{233} rouge; et la même pierre se fondroit à un feu blanc.
En général, la pureté des pierres à chaux se reconnoît par la perte de leur poids dans le feu; plus elles perdent, plus elles contiennent de matière calcaire. Les pierres à chaux magnésiennes contiennent plus d'acide carbonique que les pierres à chaux ordinaires; et j'ai toujours trouvé qu'elles perdoient plus de la moitié de leur poids par la calcination.
Les matières calcaires sont employées à l'agriculture dans diverses combinaisons. Une d'entr'elles est le gypse ou sulfate de chaux. Cette substance est composée d'acide sulfurique et de chaux, savoir, cinquante-cinq parties de celle-ci, et soixante et quinze d'acide sulfurique. La sélénite, ou gypse ordinaire, contient aussi beaucoup d'eau.
Il est facile de démontrer la nature du gypse. Si l'on mêle de la chaux vive avec l'acide sulfurique étendu d'eau, il y a production de chaleur violente. Si l'on expose le mélange au feu, l'eau s'évapore et l'on a du gypse, si l'acide étoit en quantité suffisante, ou du gypse mêlé de chaux vive, s'il n'y avoit pas assez d'acide. On trouve quelquefois dans la terre, le gypse absolument dépourvu d'eau.\setcounter{page}{234} Lorsqu'on a privé le gypse d'eau, par le moyen du feu, et qu'après l'avoir réduit en poudre on en fait une pâte avec de l'eau, il devient promptement solide, en se combinant avec elle. La propriété du plâtre comme ciment, et pour modeler, dépend de celle qu'a le gypse, de rendre solide une certaine quantité d'eau. Le gypse est soluble dans environ cinq cents fois son poids d'eau froide; et il est plus soluble dans l'eau chaude; de manière que lorsqu'on a fait bouillir du gypse dans l'eau, il se forme des cristaux quand l'eau se refroidit. On découvre aisément le gypse dans les liquides où il est dissous, par sa propriété de donner des précipités par les solutions d'oxalate, et de sels barytiques. Les agriculteurs ont été partagés sur l'emploi du gypse. On s'en est servi avec avantage en Kent, et Mr. Smith a rendu compte au Département d'agriculture, du succès qu'il avoit éprouvé dans l'emploi de cette substance en Amérique. Mais ce succès a manqué en Angleterre, quoique le gypse ait été employé de diverses manières et sur différentes récoltes. On a mis en avant des opinions très-différentes pour expliquer la manière dont le gypse agit dans la végétation. Quelques personnes ont supposé qu'il agissoit en attirant\setcounter{page}{235} L'humidité de l'air ; mais cette action doit être de peu d'effet. Lorsqu'il est réellement combiné avec l'eau, il la retient trop fortement pour la lâcher aux racines des plantes ; et quant à l'attraction d'adhésion qu'il a pour l'eau, elle a peu de force : d'ailleurs on l'applique en quantité si peu considérable, que cette explication ne paroît pas admissible.
On a dit que le gypse aidoit à la putréfaction des substances animales, et à la décomposition des fumiers. J'ai essayé quelques expériences à ce sujet ; et leur résultat est contraire à cette supposition.
J'ai mêlé des tranches de veau avec un centième de leur poids de gypse. Je les ai exposées comparativement avec d'autres tranches de veau, non saupoudrées de gypse, mais d'ailleurs dans les mêmes circonstances. Il n'y a eu aucune différence dans l'espace de temps nécessaire pour amener la putréfaction, et celle-ci a paru plus rapide là où il n'y avoit pas de gypse. J'ai fait d'autres mélanges semblables, en y employant le gypse en quantité plus ou moins grande ; j'ai essayé de remplacer la viande par du fumier de pigeon, et je n'ai jamais observé que le gypse ait accéléré la putréfaction.
Il y a long-temps qu'on fait en Angleterre\setcounter{page}{236} une suite d'expériences sur l'action du gypse comme engrais, mais la chose n'est pas généralement connue. Les cendres de tourbe de Berkshire et de Wiltshire contiennent beaucoup de gypse: j'en ai trouvé d'un quart à un tiers dans les cendres de tourbe de Newburn, et une plus grande quantité dans les cendres de tourbe du voisinage de Stockbridge. Les parties constituantes de ces cendres sont de la chaux, de l'alumine, et de la silice, avec diverses quantités de sulfate de potasse, un peu de sel commun, et quelquefois de l'oxide de fer. Les cendres rouges sont celles qui contiennent le plus de cette dernière substance. Les cendres de tourbe s'emploient pour saupoudrer les prés artificiels, et particulièrement le sainfoin et le trèfle. J'ai examiné les cendres du sainfoin, du trèfle et du ray-grass, et j'y ai trouvé beaucoup de gypse. Cette substance est probablement une partie nécessaire de leur fibre ligneuse, et y est intimément combinée. Dans cette supposition, l'on expliqueroit aisément l'action du gypse, lorsqu'il est employé en très-petite quantité: j'ai estimé que la récolte d'un acre de trèfle ou de sainfoin, si elle étoit brûlée, donneroit environ trois à quatre bushels de gypse.
\setcounter{page}{237} J'ai examiné le sol d'une pièce près de Newbury. L'échantillon de ce sol avoit été pris dessous un sentier, auprès d'une porte, c'est-à-dire, dans un endroit où le gypse ne pouvoit pas avoir été artificiellement introduit. Je ne pus pas en découvrir la moindre trace; et dans le même moment les cendres de tourbe étoient employées sur le trèfle de cette pièce. Il est probable que la raison qui empêche que le gypse ne soit généralement efficace, c'est que la plupart des sols cultivés contiennent le gypse en quantité suffisante pour les foins artificiels. Dans le cours ordinaire de l'agriculture, on fournit du gypse à la terre avec les engrais, car il est contenu dans le fumier d'étable, et dans celui de tous les bestiaux nourris d'herbe. Cependant les récoltes de céréales ne l'enlève point à la terre, et les turneps lui en enlèvent fort peu. Dans les terrains destinés exclusivement aux pâturages, ou aux récoltes de foin, il est continuellement consommé ou enlevé à la terre. J'ai examiné quatre différens sols soumis à des assolemens ordinaires, et j'y ai cherché le gypse. Un de ces sols étoit un sable léger de Norfolk; un autre, une glaise de Middlesex, donnant de bon froment; le troisième, un sable de Sussex, et le quatrième, une glaise d'Essex.\setcounter{page}{238} Je trouvai le gypse dans tous ces échantillons de terre. Lord Dundas m'apprend qu'ayant essayé l'usage du gypse, et sans succès, sur deux de ses terres, dans le Yorkshire, il eut l'idée de faire examiner la nature de ces deux sols, selon la méthode que j'ai indiquée ci-devant; et il se trouve du gypse dans l'un et dans l'autre.
Si ces faits sont confirmés par des recherches subséquentes, on pourra en tirer des directions utiles pour la pratique. Il est possible que les terres qui ont cessé de donner de belles récoltes de trèfle ou d'autres foins artificiels, puissent être rétablies par le mélange du gypse. J'ai dit que cette substance se trouve en Oxfordshire : elle est aussi abondante dans le Glocestershire, le Sommerséstshire, le Derbyshire, le Yorkshire, etc. et ne demande pour sa préparation que d'être réduite en poudre.
Le Dr. Pearson a présenté au Département d'agriculture, des documens très-intéressans sur l'usage du sulfate de fer, ou vitriol vert, qui est un sel produit avec la tourbe du Bedfordshire; et j'ai vu les effets fertilisans d'une eau ferrugineuse pour arroser un pré du duc de Manchester, près de Woburn : le détail du produit de ce pré a été publié par le Département : je ne doute pas que\setcounter{page}{239} l'eau vitriolique et le sel de sa tourbe n'agissent principalement en produisant du gypse.
Les sols sur lesquels le sulfate de fer et l'eau vitriolique agissent, sont calcaires, et le sulfate de fer y est décomposé par le carbonate de chaux. Le sulfate de fer est composé d'acide sulfurique et d'oxide de fer: c'est un sel acide très-soluble. Lorsqu'une solution de ce sel se trouve mêlée à du carbonate de chaux, l'acide sulfurique quitte l'oxide de fer pour s'unir à la chaux: les composés qui en résultent sont insipides, et comparativement insolubles.
Je recueillis un peu du dépôt de l'eau ferrugineuse sur le sol du pré en question: j'y trouvai du gypse, du carbonate de fer, et du sulfate de fer insoluble. Les principales herbes de ce pré étoient, le vulpin des prés, le pied de poule, la festuque des prés, le fiorin, et la flouve odorante.
J'ai examiné les cendres de trois de ces plantes, savoir, le vulpin des prés, le pied de poule et le fiorin, et j'ai trouvé qu'elles contentoient une forte proportion de gypse. J'ai parlé d'un sol du Lincolnshire, qui est imprégné de vitriol, et qui manque de matière calcaire. Ce sol est stérile; mais c'est en conséquence de l'excès du fer. L'oxide de\setcounter{page}{240} fer entre avantageusement dans la constitution d'un sol, et on le trouve dans les cendres des plantes : il n'est probablement nuisible que dans ses combinaisons acides.
J'ai parlé de certaines tourbes dont les cendres contiennent du gypse ; mais j'ai examiné diverses cendres de tourbe de l'Écosse, de l'Irlande, du pays de Galles, du nord et de l'occident de l'Angleterre, lesquelles n'en contentoient pas une quantité qui pût être utile : ces mêmes cendres abondoient en terre siliceuse, en alumine, et en oxide.
Lord Charleville a trouvé dans quelques cendres des tourbes de l'Irlande le sulfate de potasse.
Il se forme ordinairement des matières vitrioliques, et si le sol inférieur est calcaire, le résultat définitif est la formation du gypse.
En général, lorsqu'une cendre de tourbe qui vient d'être faite, donne une odeur semblable à celle des œufs pourris, si l'on verse du vinaigre sur cette cendre, elle contient du gypse.
Le phosphate de chaux est une combinaison d'acide phosphorique et de chaux, savoir, une proportion de chaux. C'est un composé insoluble dans l'eau pure ; mais soluble dans l'eau mêlée d'un acide quelconque. Il constitue la plus grande partie des os\setcounter{page}{241} calcinés. Il se trouve dans la plupart des substances excrémentielles, et existe dans la paille et le grain de l'orge, du blé, de l'avoine et du seigle, ainsi que dans les fèves, les pois et les vesces. On le trouve natif dans quelques endroits de la Grande-Bretagne, mais seulement en petites quantités. On porte ordinairement le phosphate de chaux, sur les terres avec le fumier. Et il est vraisemblablement nécessaire à toutes les récoltes de grains blancs.
Les cendres des os sont probablement utiles dans les terres arables qui contiennent beaucoup de matières végétales; et elles pourraient rendre les tourbes molles capables de produire du froment; mais les os broyés, sans être calcinés doivent être préférés partout où l'on peut se les procurer dans cet état.
Les composés salins de magnésie ne demandent pas beaucoup de discussion quant à leur emploi comme engrais. Nous avons déjà examiné ce sujet relativement à l'emploi de la pierre à chaux magnésienne. L'acide sulfurique uni à la magnésie forme un sel soluble. Cette substance a été jugée utile comme engrais; mais on ne la trouve pas en suffisante abondance, et on ne peut pas la fabriquer à assez bas prix pour que l'usage en soit économique.
\setcounter{page}{242} Les cendres de bois sont principalement composées d'alkali végétal uni à l'acide carbonique; et comme cet alkali se trouve dans presque toutes les plantes, il est facile de comprendre qu'il puisse faire une partie essentielle de leurs organes. La tendance générale des alkalis est de donner la solubilité aux matières végétales ; et de cette manière ils peuvent rendre les substances charbonneuses capables d'être absorbées par les tubes des fibres radicules des végétaux. L'alkali végétal a aussi une forte attraction pour l'eau ; et même en petites quantités, il peut tendre à donner un degré convenable d'humidité aux engrais ; mais cette influence ne peut guère être secondaire, vu la foible proportion qui existe dans le sol, ou que l'on peut y employer comme engrais.
L'alkali minéral ou la soude se trouve dans les cendres des plantes marines, et peut aussi être tiré du sel commun. Le sel commun est composé du métal nommé sodium combiné avec le chlore ; et la soude pure est le même métal uni à l'oxigène. La présence de l'eau qui fournit l'oxigène fait obtenir la soude, du sel commun, par divers procédés.
Il est probable que quand le sel commun agit comme engrais, c'est en entrant dans sa\setcounter{page}{243} composition des plantes de la même manière que le gypse, le phosphate de chaux, et les alkalis. Le chevalier Pringle a vérifié que le sel commun, employé en petite quantité, facilite la décomposition des matières animales et végétales. Cette circonstance peut le rendre utile dans certains sols. Il est encore avantageux en nuisant aux insectes. Je crois bien démontré, qu'en petite dose, il est quelquefois d'un bon effet comme engrais; et il est probable que son efficacité dépend de la combinaison de plusieurs causes.
On a blâmé l'emploi du sel, parce qu'appliqué en fortes doses, il rend le sol stérile, ou ne fait point de bien; mais c'est une mauvaise manière de raisonner. Nous savons, par de très-anciennes autorités, que le sel appliqué au sol, en grande quantité, le rend stérile; mais nous avons sous les yeux des faits qui prouvent que son usage modéré peut être utile. Le sel de rebut, en Cornouailles, est un admirable engrais; et les fermiers de Cheshire se disputent l'acquisition de la même substance, pour leurs champs.
Il paroît vraisemblable que les mêmes causes qui modifient les effets du gypse, influent sur ceux du sel. Il est probable que la plupart des terrains de l'Angleterre con\setcounter{page}{244} tiennent assez de sel pour les conditions de la végétation ; et que l'addition de cette substance peut, dans ce cas, être non-seulement inutile, mais nuisible. Dans de fortes tempêtes, l'écume de la mer a été transportée jusqu'à cinquante milles du rivage ; le sel doit ainsi être fréquemment fourni, à la terre. J'ai trouvé du sel dans tous les grès que j'ai examinés, et il doit se trouver dans les sols formés du détritus de ces grès. Il est aussi une des parties constituantes de presque tous les engrais animaux et végétaux.
Outre ces composés des terres alcalines et des alkalis, on en a recommandé d'autres, comme propres à animer la végétation ; le nitre est de ce nombre. Ce sel est formé de l'acide nitreux et de la potasse, c'est-à-dire, d'une proportion d'azote, six d'oxigène, et une de potassium. Il n'est pas invraisemblable que le nitre fournit de l'azote pour former de l'albumen ou du gluten dans les plantes qui les contiennent ; mais les sels nitreux sont trop chers pour qu'on puisse les employer comme engrais.
Le Dr. Home dit que le sulfate de potasse que l'on trouve dans certaines cendres de tourbes, est un engrais utile ; mais Mr. Naismith met les résultats en doute : il cite des expé\setcounter{page}{245} expériences contraires à l’opinion que les substances salines puissent être avantageuses à employer comme engrais.
Si l’on ne s’entend pas sur l’effet des substances salines, c’est, je crois, beaucoup parce qu’on les a employées en doses différentes, et en général trop fortes.
Je fis un grand nombre d’expériences en mai et en juin 1807, concernant les effets de diverses substances salines, sur de l’orge et de l’herbe, dans un sol léger et sablonneux, dont la composition était soixante de sable siliceux, vingt-quatre de matière divisée, laquelle contenait sept de carbonate de chaux, douze d’alumine et silice, et un peu moins d’une partie de matière saline, ( surtout du sel commun, avec un peu de gypse et de sulfate de magnésie ) les seize parties restantes pour compléter les cent, étaient de la matière végétale.
J’employois deux fois la semaine les solutions des substances salines, à la quantité de deux onces, sur des espaces en herbe et en orge, séparés par un intervalle suffisant pour éviter l’influence du voisinage. Les substances que j’essayai étaient le surcarbonate, le sulfate, l’acétate, le nitrate, et le muriate de potasse; le sulfate de soude; le sulfate\setcounter{page}{246} fate, le nitrate, le muriate, et le carbonate d'ammoniaque. Je trouvai que dans tous les cas, lorsque la quantité du sel en solution était égale à un trentième du poids de l'eau, l'effet de l'arrosemen était nuisible. Mais cet effet était moins fâcheux quand j'employois le carbonate, le sulfate, et le muriate d'ammoniaque.
Quand les quantités des sels s'élevaient à un trentième de la solution, les effets étaient différens. Les plantes arrosées des solutions de sulfate végétaient de la même manière que les mêmes plantes arrosées d'eau de pluie. Celles que j'arrosais avec les solutions de nitrate, d'acétate, et surcarbonate de potasse, ainsi que de muriate d'ammoniaque avaient une végétation un peu plus active. Les plantes traitées avec la solution de carbonate d'ammoniaque végétaient le plus fortement de toutes; et ce dernier résultat était naturel à attendre, car le carbonate d'ammoniaque est composé de carbone, d'hydrogène, d'azote, et d'oxygène. Il y eut cependant un autre résultat que je n'avais pas prévu: des solutions de nitrate ne faisaient pas un meilleur effet que l'eau de pluie. La solution rougissait la teinture végétale bleue; et probablement l'acide libre faisait un mauvais effet, qui nuisoit au résultat.
\setcounter{page}{247} La suie doit sans doute une partie de ses bons effets au sel ammoniac qu'elle contient. La liqueur produite par la distillation du charbon contient de l'acétate et du carbonate d'ammoniaque, et on dit que cette liqueur est un fort bon engrais.
En 1808, j'éprouvai un bon effet pour la végétation d'un champ de blé, de l'aspersion d'une très-faible solution d'acétate d'ammoniaque.
Les cendres de savoniers ont été recommandées comme engrais; et l'on a cru que leur efficace dépendoit des différentes matières salines qu'elles contiennent; mais la quantité de celles-ci est très-faible, et leurs principaux ingrédients sont la chaux éteinte et la chaux-vive. Dans les bonnes manufactures de savon, il ne reste aux cendres, presque aucune trace d'alkali. La chaux mouillée d'eau de mer en donne davantage, et paroît avoir été employée quelquefois avec plus d'effet que la chaux ordinaire.
Il est inutile de discuter plus au long l'effet des substances salines sur la végétation: si l'on excepte celles que nous avons indiquées ci-dessus, aucune ne peut donner dans sa décomposition, les principes des végétaux, le carbone, l'hydrogène et l'oxygène.
\setcounter{page}{248} Les sulfates alcalins et les muriates terreux ne se trouvent guère dans les plantes, ou quand ils y sont, c’est en si petite quantité qu’il ne peut jamais être utile d’en répandre sur le sol. Nous avons vu que les substances alcalines et terreuses ne sont jamais formées dans la végétation. Et il y a toutes sortes de raisons de croire qu’elles ne sont jamais décomposées, car après avoir été absorbées, elles se retrouvent dans les cendres.
Les bases métalliques de ces substances ne peuvent pas exister en contact avec les fluides aqueux; et ces bases métalliques n’ont jamais pu prendre une autre forme par des procédés artificiels, elles se combinent aisément avec d’autres élémens, mais elles demeurent indestructibles, et ne perdent rien en poids dans leurs diverses combinaisons.
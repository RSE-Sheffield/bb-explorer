\setcounter{page}{145}
\chapter{AGRICULTURE.}
\section{ÉLÉMENS OF AGRICULTURAL CHEMISTRY; etc. Élémens de chimie agricole en un Cours de leçons pour le Département d'Agriculture; par Sir HUMPHRY DAVY. Londres, 1813. \large{( Huitième extrait. Voy. p. 122).}}
Des engrais d'origine animale et végétale. De la manière dont ils deviennent la nourriture des plantes; de la fermentation et de la putréfaction; des différentes espèces d'engrais végétaux; des différentes espèces d'engrais animaux; des engrais mélangés; des principes généraux; relativement à l'application de ces engrais.
C'EST un fait connu de tous les temps, que l'application au sol de certaines substances animales et végétales; accélére la végétation et augmente les produits; mais on discute encore sur la manière dont ils agissent, sur la meilleure méthode de leur emploi, et sur la durée de leur action. Nous\setcounter{page}{146} essayerons de poser sur ce sujet, quelques principes que les découvertes récentes en chimie peuvent éclaircir beaucoup; et je n'ai pas besoin de m'arrêter à prouver combien il est important pour les agriculteurs. Les pores des radicules des plantes sont si petits, qu'on les découvre à peine avec le secours d'une forte loupe. Il n'est donc pas probable que des substances solides puissent passer du sol dans ces pores. J'ai fait une expérience là-dessus. Je mis du charbon pulvérisé très-fin, que j'avois obtenu en lavant de la poudre à canon, dans une phiole d'eau pure, où croissoit une plante de menthe. Cette expérience fut faite au commencement de mai 1805. La plante végéta fortement pendant quinze jours. Je la tirai ensuite de la phiole. Je coupai les radicules en différens endroits; mais je n'y découvris point de matière charbonneuse. Les fibriles les plus délicates n'étoient point noircies par le charbon; ce qui auroit eu lieu si celui-ci avoit été absorbé sous sa forme solide. Il n'y a pas de substance plus nécessaire aux plantes, que la matière charbonneuse; et si elle ne peut être introduite dans les organes des végétaux d'aucune autre manière que dans l'état de solution, il est bien probable que d'autres substances moins essentielles sont dans le même cas.
\setcounter{page}{147} Des expériences que je fis en 1804 me prouvèrent que les plantes mourraient dans les solutions récentes et fortes de sucre, de mucilage, de tannin, de gélatine, et de quelques autres substances. J'éprouvai aussi que les mêmes plantes pouvaient vivre dans ces solutions, lorsqu'elles avaient fermenté. Je supposai alors que la fermentation était nécessaire pour préparer l'aliment des plantes; mais j'ai trouvé depuis, que l'effet délétère des solutions végétales récentes était dû à leur trop grande concentration: il résultait probablement de celle-ci, que les organes des végétaux étaient obstrués par la matière solide, et que la transpiration des feuilles ne pouvait avoir lieu. L'année suivante, au commencement de juin, je répétai ces expériences avec les mêmes substances; mais les solutions étaient plus délayées: elles ne contenaient qu'environ ¹⁄₁₀₀ de parties solides. Les plantes de moutarde croissaient vigoureusement dans ces solutions: moins cependant, dans celle de substances astringentes.
J'essayai d'arroser quelques espaces de gazon dans un jardin, avec des solutions différentes, et un autre espace avec de l'eau commune. Les places arrosées avec des solutions de gelée, de sucre, et de mucilage, eurent la végétation\setcounter{page}{148} la plus forte; et l'espace arrosé avec une solution de tannin végétal plus rapidement que celui que j'avais arrosé d'eau commune.
J'essayai de m'assurer si les substances végétales solubles passoient dans les racines des plantes, sans changer d'état. Pour cela, je comparai les produits de l'analyse des racines de quelques plantes de moutarde, qui avoient végétè dans l'eau commune et dans une solution de sucre. Cent vingt grains des racines qui avoient végétè dans la solution de sucre, donnèrent cinq grains d'un extrait d'un vert pâle, d'un goût douceâtre, et que l'action de l'alcool coaguloit légèrement. Cent vingt grains des racines de moutarde, qui avoient végétè dans l'eau commune, donnèrent trois grains et demi d'un extrait couleur d'olive foncé. Son goût étoit douceâtre, mais plus astringent que celui de l'autre extrait: il se coaguloit plus abondamment par l'alcool.
Quoique ces résultats ne soient pas tout-à-fait décisifs, ils favorisent l'opinion que les matières solubles passent, sans altération, dans les organes des plantes; et cette idée est confirmée par ce fait ci, c'est que les fibriles radicales des plantes que l'on fait végétè dans des infusions de garance, sont teintes\setcounter{page}{149} de ronge. Il est presque prouvé, que les substances les plus nuisibles aux végétaux sont absorbées par eux. J'essayai d'introduire les racines d'une prime-rose, dans une foible solution d'oxide de fer et de vinaigre. Je l'y laissai jusqu'à-ce que les feuilles jaunissent. Je lavai ensuite soigneusement les racines dans de l'eau distillée; je les écrasai, et j'en fis bouillir une petite quantité dans cette même eau distillée. Je soumis ensuite la décoction, passée au filtre, à une infusion de noix de galles. Cette décoction prit une forte teinte de pourpre, ce qui prouve que la solution de fer avoit été absorbée par les vaisseaux, ou les pores des racines. Les substances animales et végétales sont consommées dans l'acte de la végétation; et elles ne peuvent nourrir les plantes, qu'en leur donnant les matières solides qui peuvent se dissoudre dans l'eau, ou des substances gazeuses qui peuvent être absorbées par les fluides contenus dans les feuilles; mais les parties qui passent à l'état gazeux, et se mêlent ainsi à l'atmosphère, ne peuvent produire que peu d'effet, parce que les gaz se mêlent bientôt à la masse de l'air ambiant. Le grand objet, qu'on doit avoir en vue, dans l'application des engrais, est de leur faire:\setcounter{page}{150} donner aux racines des plantes autant de matières solubles qu'il est possible, et cela d'une manière lente et graduée, afin que l'engrais soit employé tout entier à former la sève ou la matière organisée des plantes.
Les fluides mucilagineux, gélatineux, saccharins, huileux et extractifs, ainsi que la solution d'acide carbonique dans l'eau, sont les substances qui contiennent presque tous les principes nécessaires à la vie des plantes; mais il y a peu de cas dans lesquels ces substances puissent être appliquées comme engrais, dans leur forme pure: les engrais végétaux contiennent en général beaucoup de matière fibreuse insoluble, qui doit premièrement éprouver des altérations chimiques, avant de devenir l'aliment des plantes.
Il est convenable de prendre connaissance de la nature de ces altérations, des causes qui les occasionnent, les accélèrent ou les retardent; ainsi que des produits qui en résultent.
Lorsqu'on mouille et expose à l'air, à une température de 55 à 80° F., une matière végétale fraîche, contenant du sucre, du mucilage, de l'amidon, ou quelqu'autre composé végétal soluble dans l'eau, l'oxigène est bientôt absorbé, et il se forme de l'acide carbonique: il se produit de la chaleur, et\setcounter{page}{151} il se dégage des gaz, sur-tout le gaz acide carbonique, l'oxide gazeux de carbone, et l'hydro-carbonate. Il se manifeste un fluide brun, d'un goût légèrement acide ou amer; et si le procédé se prolonge un temps suffisant, il ne reste rien de solide, qu'un peu de matière terreuse et saline, teinte de noir par le charbon.
Le fluide brun formé par la fermentation contient toujours de l'acide acétique, et s'il y a de l'albumine ou du gluten dans la substance végétale, il se dégage aussi de l'alkali volatil.
La marche de la fermentation est d'autant plus rapide, qu'il existe plus de gluten, d'albumine, ou de matières solubles à l'eau, dans la substance; et cela, toutes circonstances d'ailleurs égales. La fibre ligneuse pure ne subit la fermentation que lentement; mais sa texture se détruit, et elle se résout ensuite aisément en de nouveaux élémens lorsqu'on la mélange à des substances plus susceptibles d'altération, et qui contiennent plus d'oxigène et d'hydrogène. Les huiles volatiles et fixes, les résines et la cire, sont plus susceptibles de changement que la fibre ligneuse, par l'action de l'air et de l'eau; mais elles le sont moins que les autres composés végétaux. Les substances\setcounter{page}{152} les plus inflammables deviennent graduellement solubles à l'eau par l'absorption de l'oxigène.
En général, les substances animales sont plus sujettes à la décomposition que les substances végétales. Il s'absorbe de l'oxigène, et il se forme de l'acide carbonique et de l'ammoniaque pendant le procédé de leur putréfaction; ces substances produisent des fluides élastiques composés et fétides, ainsi que de l'azote; elles donnent encore un acide brunâtre, des fluides huileux, et laissent un résidu de sels et de terres, mélangé de matière charbonneuse. Les substances principales qui constituent les différentes parties des animaux, ou qui se trouvent dans leur sang, leurs sécrétions ou leurs excrémens, sont la gélatine, la fibrine, le mucus, la matière grasse ou huileuse, l'albumine, l'urée, l'acide urique, et diverses autres substances acides, salines et terreuses.
La gélatine, combinée avec l'eau, forme la gelée. Elle est fort sujette à la putréfaction : selon MM. Gay-Lussac et Thénard, la gélatine est composée de
47,88 de carbone.
27,207 oxigène.
7,914 hydrogène.
16,998 azote.
\setcounter{page}{153} Ces proportions ne peuvent pas être considérées comme définies; car elles ne sont point entr'elles dans les rapports des simples multiples des nombres qui représentent leurs élémens. Il paroît qu'il en est de même des autres composés animaux, et même des substances végétales. Dans ces corps très-composés, les proportions ne sont pas si faciles à déterminer que dans les corps susceptibles d'être formés artificiellement et qu'on peut appeler binaires, tels que les acides, les alkalis, les oxides, et les sels.
La fibrine est la base des fibres musculaires des animaux; et on peut obtenir la même substance du sang frais et fluide. Si on le remue avec un bâton, la fibrine s'y attache. Elle n'est pas soluble à l'eau; mais Mr. Hatchett a montré, que par l'action des acides, elle devient soluble, et analogue à la gélatine. Elle est moins sujette à se putréfier que celle-ci. Selon Gay-Lussac et Thénard, la fibrine contient:
53,360 de carbone.
19,685 — oxigène.
7,021 — hydrogène.
19,934 — azote.
Le mucus est très-analogue à la gomme, dans ses caractères distinctifs. Le Dr. Bostock\setcounter{page}{154} a prouvé qu'on peut l'obtenir en faisant évaporer la salive. On n'a pas fait d'expériences sur son analyse, mais il est probablement ressemblable à la gomme dans sa composition. Il est susceptible de putréfaction, mais moins rapidement que la fibrine.
Les graisses et huiles animales n'ont pas été analysées exactement; mais il est probable que leur composition est analogue à celle des huiles végétales.
L'albumine a déjà été examinée, et nous avons rendu compte de son analyse.
L'urée s'obtient par l'évaporation de l'urine humaine jusqu'à consistance de sirop, et par l'action de l'alcohol sur la substance cristalline qui s'est formée. On a ainsi une solution de l'urée dans l'alcohol, et on sépare celui-ci par la chaleur. L'urée est très-soluble dans l'eau: on l'en précipite par l'acide nitrique étendu d'eau, et on l'obtient alors sous forme de cristaux brillans et couleur de perles. Cette propriété distingue l'urée de toutes les autres substances animales.
Selon Fourcroy et Vauquelin, 100 parties d'urée distillé, donnent
92,927 parties de carbonate de magnésie.
4,608 de gaz hydrogène carburé.
2,225 de charbon.
\setcounter{page}{155} L’urée subit facilement la putréfaction, surtout par le mélange avec l’albumine et la gélatine.
L’acide urique peut s’obtenir, ainsi que le Dr. Egan l’a prouvé, en versant un acide dans l’urine humaine; et il se forme souvent dans l’urine en cristaux de couleur de brique. Il est composé de carbone, d’hydrogène, d’oxigène, et d’azote; mais leurs proportions n’ont pas été encore déterminées. L’acide urique est une des substances animales les moins sujettes à la putréfaction.
Selon que les proportions des composés animaux sont différentes, les altérations qu’ils éprouvent sont également différentes. Lorsqu’il y a beaucoup de matières salines ou terreuses, le progrès de leur décomposition est moins rapide que lorsqu’ils sont principalement composés de fibrine, d’albumine, de gélatine ou d’urée. L’ammoniaque qui se dégage des composés animaux en putréfaction, peut être considéré comme se formant pendant la durée du procédé même, et cela par la combinaison de l’hydrogène et de l’azote. Si l’on excepte cette production de l’ammoniaque, les phénomènes sont les mêmes pour la décomposition des substances végétales.
Toutes les fois que les engrais consistent principalement en une substance soluble à\setcounter{page}{156} l'eau, il est évident que leur fermentation ou putréfaction doit être prévenue autant qu'il est possible; et le seul cas dans lequel cette putréfaction puisse être utile, est celui où l'engrais est principalement formé de fibre animale ou végétale. Une température au-dessus du point de congélation, la présence de l'eau et de l'oxygène, sont au moins nécessaires dans le commencement du procédé.
Pour empêcher les fumiers de se décomposer, il faut les conserver secs, les défendre du contact de l'air, et les maintenir au frais autant qu'il est possible.
Il paroît que la propriété du sel et de l'alcool, de préserver de la putréfaction, les substances animales et végétales, est due à leur attraction de l'eau, ce qui prévient l'action décomposante de celle-ci; ils agissent aussi en excluant l'air. L'effet de l'application de la glace pour la conservation des substances animales, est dû à la basse température qu'elle maintient. La méthode de Mr. Appert, qui a été publiée récemment, pour la conservation des substances animales et végétales, est uniquement fondée sur l'exclusion de l'air. Il remplit un vase d'étain ou de verre, de viande, ou de végétaux. Il cimente le couvercle de manière à fermer\setcounter{page}{157} tout accès à l'air. Il plonge ensuite à moitié ce vase dans l'eau bouillante, jusqu'à ce que la substance soit cuite au point convenable. Il est probable que dans ce dernier procédé, la petite quantité d'oxygène qui restoit dans le vase clos, est absorbée; car après avoir ouvert un vaisseau de fer étamé qui, la veille, avoit été rempli de viande de bœuf crue; puis exposé à l'action de l'eau chaude, j'ai trouvé que la petite quantité de fluide élastique contenue dans ce vaisseau, étoit un mélange de gaz acide carbonique et d'azote. Lorsqu'il s'agiroit de conserver de la viande, ou des substances végétales en grande masse, pour la marine ou pour l'armée, je pense qu'on réussiroit en forçant le gaz acide carbonique, l'hydrogène, ou l'azote, dans les vaisseaux clos, au moyen d'une pompe de compression semblable à celle qu'on emploie pour les eaux de seltzer factices. Il ne resteroit, dans ce cas, aucune place pour les fluides élastiques qui tendroient à se produire par la décomposition des viandes. La putréfaction ou fermentation ne sauroit avoir lieu sans qu'il s'engendre des fluides élastiques; et la pression agiroit probablement aussi efficacement que le froid, pour la conservation des vivres. Comme les différens engrais contiennent\setcounter{page}{158} différentes proportions des élémens nécessaires à la végétation; ils demandent des traitemens différens pour que leur plein effet soit produit. Je vais donner quelques idées générales sur la manière de conserver et employer les engrais, après avoir décrit leurs propriétés et leur nature.
Les plantes succulentes fraîches contiennent beaucoup de matière sucrée ou mucilagineuse outre la fibre ligneuse. Elles fermentent aisément; et par cette raison, on ne sauroit les employer trop tôt, lorsqu'elles ont cessé de végéter, si l'on veut les faire servir d'engrais.
Lorsqu'on emploie les récoltes vertes pour amender le sol, il faut les enterrer dans le moment de la pleine fleur, ou lorsque la fleur commence à se former : c'est l'époque où elles donnent la plus grande quantité de matière aisément soluble, et où leurs feuilles travaillent à former la plus grande quantité de substance nutritive. Les récoltes vertes, les plantes d'étangs, les raclures des fossés, et toutes les matières végétales fraîches, doivent être employées immédiatement, et sans préparation, lorsqu'on veut les faire servir d'engrais. La décomposition se fait alors peu à-peu dans le sol; les matières solubles se dissolvent par degrés; et cette légère fermentation,\setcounter{page}{159} que modère le défaut de communication avec l'air libre, tend à rendre soluble la fibre ligneuse, sans occasionner la dissipation rapide des gaz;
Lorsqu'on rompt des prés pour semer des graminées, le sol se trouve enrichi non-seulement par la décomposition lente des plantes mortes depuis long-temps, mais les feuilles et les racines des herbes en végétation; et que l'on enterre, donnent des matières sucrées, mucilagineuses et extractives, lesquelles deviennent immédiatement l'aliment de la récolte. La décomposition graduelle fournit ensuite de la nourriture aux récoltes qui succèdent.
Les gâteaux de navette ou de colza contiennent beaucoup de muoilage, d'albumine, et un peu d'huile. Cet engrais, qui est très-utile, sur-tout pour les récoltes de turneps, doit être employé promptement, et maintenu aussi sec qu'il est possible. La meilleure manière de l'appliquer est de la mettre en terre en même temps que la semence. Ceux qui veulent connaître cette méthode dans sa perfection doivent assister à la fête annuelle de la tonte chez Mr. Coke à Holkham.
La poussière de drêche est composée des radicules séparées des grains d'orge, dans le procédé de la fabrication de la bière. Je n'ai\setcounter{page}{160} jamais fait d'expérience sur cet engrais, mais ce qui paroît expliquer son grand effet, c'est la quantité de matière sucrée que cette poussière de drêche contient. Il faut aussi maintenir cet engrais aussi sec qu'il est possible, et prévenir sa fermentation.
Les gâteaux de lin sont une trop bonne nourriture du bétail, pour qu'il convienne de les employer comme engrais. L'eau qui a servi au rouissage du chanvre et du lin a une grande faculté fertilisante. Elle paroît contenir beaucoup de matière végétale extractive, et une substance analogue à l'albumine. Elle se putréfie aisément. Il faut un premier degré de fermentation pour obtenir le chanvre et le lin dans l'état convenable; il faut donc employer l'eau qui a servi au rouissage, aussitôt que cette opération est achevée.
Les varecs sont beaucoup employés comme engrais sur les côtes d'Angleterre et d'Irlande. En faisant digérer dans l'eau bouillante le fucus commun, qui compose la plus grande partie des varecs dont on se sert comme engrais, j'en ai obtenu un huitième de substance gélatineuse, semblable au mucilage. A la distillation, cette plante donne les quatre cinquièmes de son poids en eau, mais point d'ammoniaque. L'eau avoit un goût d'empyreume,\setcounter{page}{161} et un peu aigre. Les cendres contentoient du sel marin, du carbonate de soude, et de la matière charboneuse. Je n'obtins que peu de gaz; mais sur-tout du gaz acide carbonique et de l'oxide gazeux de carbone, avec un peu d'hydro-carbone. L'effet de cet engrais est peu durable, et n'est sensible que sur une récolte. Cela s'explique par la grande quantité d'eau ou des élémens de l'eau, que les varecs contiennent. Exposés à l'air libre, ils se dissolvent et se dissipent; sans produire de chaleur ou fermentation sensible. J'ai vu un gros tas de varecs réduit, en moins de deux ans, à une petite quantité de matière fibreuse noire.
J'ai essayé d'enfermer dans une jarre remplie d'air atmosphérique, les parties les plus solides d'un fucus, pendant quinze jours. Cette plante se flétrit beaucoup; et les parois de la jarre se tapissèrent intérieurement d'eau. J'examinai l'air de la jarre: il avoit perdu de l'oxigène; et contenoit du gaz acide carbonique.
On fait quelquefois fermenter les varecs avant de les employer comme engrais; mais cela ne paroît point nécessaire; car il ne résulte pas de ce procédé que des parties\setcounter{page}{162} fibreuses deviennent solubles; et cependant une portion de l'engrais est perdue.
Les meilleurs agriculteurs emploient les varecs aussi frais qu'ils peuvent se les procurer, et le résultat est d'accord avec la théorie. L'acide carbonique formé dans cette fermentation commençante, doit se dissoudre partiellement dans l'eau, et être absorbé par les racines des plantes.
Les effets des plantes marines comme engrais doivent principalement dépendre de l'acide carbonique et du mucilage soluble que contiennent ces plantes. J'ai trouvé que des fucus qui avoient fermenté de manière à perdre environ une moitié de leur poids, donnoient moins d'un douzième de matière mucilagineuse, d'où l'on peut légitimement conclure qu'une partie de cette substance se détruit pendant la fermentation.
La paille sèche de froment, d'avoine, et d'orge, ainsi que les tiges de fèves et de pois, et enfin le foin qui s'est gâté, peuvent être employés utilement comme engrais. En général l'usage est de faire fermenter ces matières avant de les appliquer; mais on peut douter si cela est convenable dans tous les cas. J'ai obtenu de quatre cents grains de paille d'orge sèche, environ huit grains d'une matière\setcounter{page}{163} brune, d'un goût semblable à celui du mucilage, et soluble dans l'eau. De quatre cents grains de paille de froment, j'ai obtenu cinq grains seulement d'une substance semblable. On ne sauroit douter que la paille des différentes récoltes, si on l'enterroit immédiatement, ne fournit un aliment aux plantes ; mais on objecte contre cette méthode, la difficulté d'enterrer la paille longue, et la mauvaise apparence que cela donne aux champs.
Lorsqu'on a commencé par faire fermenter la paille, son emploi devient plus facile ; mais il y a également sur le tout, une grande perte de matières nutritives. On obtient peut-être plus d'engrais, pour la première récolte, mais la terre est moins améliorée qu'elle ne le seroit si toute la substance végétale étoit extrêmement divisée et mêlée avec le sol.
Il est d'usage de mettre pourrir et d'abandonner à la décomposition sur les tas de fumier, la paille qui ne peut pas être employée à autre chose ; mais il vaudroit la peine d'essayer si l'on n'en tireroit pas un meilleur parti en hâchant cette paille, &c. en l'enterrant sèche, après la charrue. Si elle faisoit moins d'effet d'abord, cet effet seroit beaucoup plus durable.\setcounter{page}{164} La fibre ligneuse pure paroît être la seule substance végétale qui ait besoin d'être préalablement soumise à la fermentation pour devenir un aliment des végétaux. Le tan est une substance de ce genre. Mr. Young dans son excellent Traité des Engrais, qui lui a valu la médaille du duc de Bedford, à la Société d'Agriculture de Bath, affirme que le tan fait plutôt du mal que du bien aux végétaux, et il l'attribue à la matière astringente que cette substance contient; mais dans le fait, le tan a été privé de toute substance soluble par l'opération qu'il a subie dans la fosse, par l'action de l'eau; et si, réellement il nuit à la végétation, c'est probablement par son effet sur l'eau ou par son action mécanique. C'est une substance qui absorbe et retient l'humidité, et qui n'est probablement pas pénétrable par les racines des plantes.
La matière tourbeuse est une substance du même genre, et qui peut demeurer exposée pendant plusieurs années à l'action de l'air et de l'eau, sans éprouver aucune altération. Dans cet état, elle fournit peu ou point de nourriture aux plantes.
La fibre ligneuse ne fermente que dans le cas où elle est mélangée avec des substances\setcounter{page}{165} qui jouent le même rôle que le mucilage, le sucre, l'extrait, ou l'albumine, matières avec lesquelles cette fibre est ordinairement associée dans les végétaux frais. Lord Meadowbank a judicieusement recommandé un mélange de fumier de basse-cour pour faire fermenter la tourbe. Toute substance fermentescible peut remplir cet objet; et plus elle est disposée à fermenter, plus elle sera utile dans ce cas.
Lord Meadowbank établit, qu'une partie de fumier suffit pour faire fermenter quatre parties de tourbe; mais cela doit varier selon la nature du fumier et de la tourbe: lorsqu'il y a dans celle-ci des végétaux encore frais, la fermentation doit être plus prompte.
Le tan, la sciure de bois, et les copeaux doivent demander autant de fumier pour être mis en fermentation que la plus mauvaise espèce de tourbe.
La fibre ligneuse peut être convertie en un engrais par l'action de la chaux: nous nous en occuperons en traitant de celle-ci.
L'analyse de la fibre ligneuse faite par Gay-Lussac et Thénard, montre qu'elle est composée principalement des élémens de l'eau et du carbone, et que celui-ci est en plus grande quantité que dans les autres.\setcounter{page}{166} composés végétaux. On voit que tout procédé qui tend à en ôter la matière carbonneuse, rapproche cette fibre des principes solubles à l'eau: or cela arrive dans la fermentation par l'absorption de l'oxygène, et la production de l'acide carbonique. La chaux produit un effet semblable.
Les cendres de bois contenant beaucoup de charbon, ont été employées avec succès comme engrais. Une partie de leur effet peut être dû à la décomposition lente et graduelle du charbon, lequel paroît susceptible d'absorber l'oxygène et de se changer ainsi en acide carbonique sans qu'il y ait combustion.
En avril 1803, je renfermai dans un tube à moitié plein d'eau pure, et moitié d'air commun, un peu de charbon bien charbonné. Je fermai le tube hermétiquement.
Au printems de 1804, j'ouvris le tube dans l'eau pure; en choisissant un moment où la pression atmosphérique, ainsi que la température, étoient à-peu-près les mêmes qu'au début de l'expérience.
Il entra de l'eau dans le tube, et lorsque j'analysai l'air qui y restoit contenu, je trouvai qu'il ne contenoit que sept pour cent d'oxygène. L'eau du tube étant mélangée avec\setcounter{page}{167} de l'eau de chaux, donna un précipité abondant, en sorte qu'il s'étoit évidemment opéré une formation d'acide carbonique, lequel s'étoit dissous dans l'eau.
Les engrais de substances animales ne demandent pas une préparation chimique pour devenir efficaces. Le grand objet du cultivateur est de les associer au sol dans un état de division suffisante, et de prévenir leur trop rapide décomposition.
Les muscles des animaux terrestres ne s'emploient pas ordinairement comme engrais, quoiqu'il y ait des cas où l'on pourroit le faire avec avantage. Les chevaux, les chiens, les moutons morts d'accident, restent souvent exposés à l'air ou dans l'eau, jusqu'à ce que des animaux carnassiers, ou la putréfaction, s'en emparent. Il y a alors une perte pour le sol, et des exhalaisons putrides nuisibles pour le canton où ces corps se trouvent déposés.
Si on recouvroit ces animaux morts d'une quantité de terre égale à cinq ou six fois leur volume, avec une partie de chaux, pendant quelques mois, ce compost se trouveroit pénétré de matières solubles, et seroit un excellent engrais. On pourroit y ajouter un peu de chaux-vive au moment de l'employer.\setcounter{page}{168} Les poissons font un admirable engrais de quelque manière qu'on les emploie; mais on ne sauroit les employer trop frais. Il faut avoir soin de n'en pas mettre une trop grande quantité. Mr. Young rappelle une expérience dans laquelle des harengs enterrés en semant du blé produisirent une récolte si abondante qu'elle fut versée complètement, avant la moisson.
Il y a dans les comtés de Lincoln, Cambridge, et Norfolk, des petits poissons que l'on prend dans les marais, en si prodigieuse quantité, qu'ils sont un article important d'engrais dans les cantons voisins.
Il est facile d'expliquer comment les poissons agissent comme engrais. Leur peau contient principalement de la gélatine, qui se dissout très-facilement dans l'eau à cause de son peu de cohérence. La graisse, ou l'huile des poissons, est toujours ou sous la peau, ou dans le ventre. Les matières fibreuses contiennent tous les élémens essentiels des substances végétales.
Le carbone et l'hydrogène, qui abondent dans les substances animales, expliquent leurs effets; et la durée de cet effet dépend de la manière dont l'action de l'air et de l'eau les modifie.\setcounter{page}{169} Les os s'emploient beaucoup pour engrais dans les environs de Londres. Après qu'on les a brisés et soumis à l'ébullition, pour en tirer la graisse, on les vend aux agriculteurs. Plus ils sont divisés, plus leur effet est sensible. La dépense pour les faire moudre se retrouverait probablement sur leur plus grande force fertilisante; et dans l'état pulvérulent, ils pourraient être employés avec le semoir, comme on fait les gâteaux de colza.
La poudre d'os, les raclures, ou débris des fabriques où l'on emploie les os, peuvent être appliqués avantageusement comme engrais.
Des sels terreux forment la base des os, principalement le phosphate de chaux, le carbonate de chaux et le phosphate de magnésie. Les substances aisément décomposables dans les os, sont: la graisse, la gélatine et les cartilages, qui semblent être de la même nature que l'albumine coagulé.
Suivant l'analyse de Fourcroy et de Vauquelin, les os sont composés de:
\comment{table}
Matière animale décomposable . . . . 51
Phosphate de chaux . . . . . . . 37 . 7
Carbonate de chaux . . . . . . . 10
Phosphate de magnésie . . . . . . . 1 . 3
100
\setcounter{page}{170}
 Mr. Merat Gnillot a donné l'estimation suivant de la composition des os de différens animaux.
\comment{table}
____....Phosphate de chaux......Carbonate de chaux.
Os de veau.....54.....____
De cheval.....67.5.....1,25
De mouton.....70......5
D'élan.......90........1
De cochon.......52........1
De lièvre.......85........1
De poulet.......72........1,5
De brochet.......64........1
De carpe.......45........5
Dents de chevaux.......85. 5.......2,5
Ivoire.......64........1
Corne de cerf.......27.......1

Ce qui fait le complément du nombre 100 doit être regardé comme de la substance animale décomposable.
La corne est encore un engrais plus actif que les os, à raison de ce qu'elle contient plus de matière animale décomposable. Mr. Hatchett a obtenu de 500 grains de corne de bœuf 1,5 grains de résidu terreux, dont presque la moitié étoit du phosphate de chaux. Les rapures de corne sont un excellent engrais. La matière animale qu'elles contiennent\setcounter{page}{171} paroissent être de la nature de l'albumine coagulé, et deviennent lentement solubles par l'action de l'eau. La matière terreuse dans les cornés, et sur-tout dans les os, prévient la trop rapide décomposition de la substance animale, et rend l'engrais plus durable.
Les cheveux, les plumes, et la laine ont de l'analogie dans leur composition, c'est-à-dire, qu'ils sont formés d'une substance semblable à l'albumine, unie à la gélatine. Les recherches ingénieuses de Mr. Hatchett l'ont prouvé. Sa théorie de leur manière d'agir est la même que pour les os et les cornes.
Les débris des manufactures de cuir et de peaux donnent de très-bons engrais. La gélatine qui se trouve dans toutes les peaux d'animaux, y est dans un état qui favorise sa décomposition graduelle: une fois enterrées, elles fournissent pendant long-temps un aliment aux plantes qui en sont voisines.
Le sang contient tous les principes que l'on trouve dans les autres substances animales; et il est par conséquent, un excellent engrais. Nous avons vu qu'il contenoit de la fibrine: il contient aussi l'albumine. Les globules rouges que quelques chimistes ont supposé être colorés par le fer dans un état particulier\setcounter{page}{172} de combinaison avec l'oxigène, et une matière acide, sont considérés par Mr. Brande comme formés par une substance animale particulière, laquelle contient très-peu de fer.
On emploie pour engrais l'écume du sang de bœuf des raffineries de sucre.
Les coraux et les éponges sont des substances d'origine animale. D'après l'analyse de Mr. Hatchett, il paroît que ces substances contiennent abondamment une matière analogue à l'albumine coagulée. Les éponges fournissent aussi de la gélatine.
Selon Merat Guillot, le corail blanc contient en quantités égales la substance animale et le carbonate de chaux; le corail rouge 46,5 de matière animale et 53,5 de carbonate de chaux; la coralline articulée, 51 de matière animale et 49 de carbonate de chaux.
Je ne crois pas que ces substances soient jamais employées comme engrais en Angleterre, à moins qu'elles ne se trouvent accidentellement mélangées aux plantes marines. Il est probable que les corallines pourraient être appliquées avec avantage comme engrais, car on les trouve en très-grande quantité sur les rochers, dans le fond des étangs rocailleux, et sur les bords de la mer. On pourrait les détacher à la houe, et les\setcounter{page}{173} recueillir sans beaucoup de main-d'œuvre.
Parmi les excrémens des animaux, l'urine est la mieux connue, par les nombreuses expériences chimiques dont elle a été l'objet. L'urine de vache, d'après les expériences de Mr. Brande, contient:
\comment{table}
Eau . . . . . . . . . . . . . . . 65
Phosphate de chaux . . . . . . . . 3
Muriates de potasse et d'ammoniaque . 15
Sulfate de potasse . . . . . . . . . 6
Carbonates de potasse et d'ammoniaque . 4
Urée . . . . . . . . . . . . . . . . 4
Selon Fourcroy et Vauquelin, l'urine de cheval contient:
\comment{table}
Carbonate de chaux . . . . . . . . 11
Carbonate de soude . . . . . . . . 9
Benzoate de soude . . . . . . . . 24
Muriate de potasse . . . . . . . . 9
Urée . . . . . . . . . . . . . . . . 7
Eau et mucilage . . . . . . . . . . 940
En addition à ces substances, Mr. Brande y a trouvé le phosphate de chaux.
L'urine du chameau, de l'âne, du lapin, ont été soumises à diverses expériences, et leur composition a été trouvée la même. Vauquelin\setcounter{page}{174} a découvert de plus, de la gélatine dans l'urine de lapin, et de l'acide urique dans l'urine de la poule.
L'urine humaine contient une plus grande variété de composans que toute autre urine; c'est-à-dire, qu'on y trouve l'urée, l'acide urique, l'acide rosacique, l'acide acétique, l'albumine, la gélatine, une matière résineuse, et plusieurs sels. Elle diffère en composition selon l'état du corps, et la nature des alimens. Dans certaines maladies, il y a beaucoup plus de gélatine et d'albumine, que de coutume; et dans le diabète, l'urine contient du sucre.
Il est probable que l'urine du même animal doit aussi différer selon les alimens dont cet animal fait usage; et cela explique les différences que l'on observe dans les résultats des diverses recherches.
L'urine se décompose promptement; surtout celle des animaux carnivores; plus elle contient de gélatine et d'albumine, plus sa putréfaction est prompte.
Toutes les urines contiennent les élémens essentiels des végétaux, dans un état de solution; et celle qui contient le plus d'albumine, de gélatine, et d'urée, est la plus utile à employer comme engrais.
\setcounter{page}{175} Pendant la putréfaction, la plus grande partie de la matière animale soluble contenue dans l’urine, se détruit et se dissipe : il faut donc l’employer aussi fraîche qu’il est possible ; mais si elle n’est pas mélangée à des matières solides, il faut l’étendre d’eau, car dans son état de pureté elle contient trop de matières animales pour que ce fluide soit propre à être absorbé par les racines des plantes.
L’urine pourrie abonde en sel ammoniac, et quoiqu’elle soit moins active que l’urine fraîche, elle donne un puissant engrais.
Selon l’analyse récente de Berzelius, 1000 parties d’urine contiennent:
\comment{table}
Eau . . . . . . . . . . . . 933.
Urée . . . . . . . . . . . 30.
Acide urique . . . . . . . . . 1.
Muriate d’ammoniaque, acide lactique libre, lactate d’ammoniaque, et matière animale . . . . . 17,14
Le reste est composé de divers sels, phosphates, sulfates, et muriates.
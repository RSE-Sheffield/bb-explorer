\setcounter{page}{66}
\chapter{Agriculture}
\section{AN ACCOUNT OF THE GRASSES, etc. Détails sur l'herbage et le produit de la prairie d'Orcheston, en Wiltshire, par Mr. TANNER. (Farmer's Magazine).}
Un de mes amis me demanda, il y a quelques mois, la description de notre prairie d'Orcheston, des plantes qu'elle donne, et de leur produit en argent. On a fait beaucoup\setcounter{page}{67} de contes extraordinaires sur l'herbe d'Orcheston, et sur la fertilité prodigieuse du terrain où elle croît. Les uns ont affirmé que c'étoit une herbe particulière à ce canton et inconnue ailleurs; d'autres ont dit que c'étoit une variété d'une espèce bien connue. Les botanistes ont visité cette prairie en différentes saisons de l'année, et ont parlé des herbes qui y dominoient au moment de leur observation: cette dernière circonstance plus qu'aucune autre, a donné lieu à la variété des rapports.
Selon la saison de l'année, ou selon que l'été ou l'hiver précédent a été sec ou humide, on voit dominer telle espèce d'herbe ou telle autre. J'ai vu des prés qui une certaine année donnoient la meilleure herbe possible; et dans l'année suivante, en conséquence d'un trop long séjour des eaux, ces mêmes prés donnoient des herbes de mauvaise qualité. La famille des agrostis, et le chiendent y avoient remplacé les pâturins et les autres bonnes herbes. Il arrive le contraire lorsqu'on dessèche un pré marécageux. Celui qui n'a visité qu'une fois la fameuse prairie d'Orcheston ne sauroit donc en faire une description satisfaisante. Nous avons cependant vu un ecclésiastique prétendre que\setcounter{page}{68} son cher fiorin donnoit l'abondance à cette prairie, qu'il n'avoit pas seulement vue: l'esprit de système fait faire d'étranges méprises! La prairie qui a excité tant de curiosité est environ à onze lieues de Salisbury, à Orcheston Ste. Marie: la pièce est de deux acres et demi. Son produit varie selon l'abondance des sources qui découlent des hauteurs voisines. Le sol est un gravier siliceux, et une terre noire et légère. Si l'hiver est humide, les sources sont abondantes au printems: elles inondent le pré, et le produit est très-considérable. L'automne de 1811, et le printems suivant, furent très-favorables à la croissance de l'herbe: aussi les deux récoltes de 1812 ont-elles été fort belles. La première se fit à la fin de mai, et produisit deux tons ou charretées de foin par acre. La seconde récolte se fit en juillet, et donna une charretée et demie. Il n'arrive pas toutes les années, que l'on fauche deux fois; car si le printems est sec, les sources tarissent, bientôt le pré jaunit, et ne donne plus qu'un médiocre pâturage. L'on a beaucoup parlé de l'excellente qualité de ce pré, lorsqu'il est bien récolté. On a dit, par exemple, qu'il engraissoit\setcounter{page}{69} les cochons, et qu'il abondoit en matières sacharines plus qu'aucune autre herbe. Si l'on veut prendre pour juge le propriétaire même du pré et ses voisins, le produit par acre, de cette pièce, pris sur une moyenne de sept ans, ne l'emporte point sur le produit d'un bon sainfoin ou de quelqu'autre pré artificiel. On court de plus grands risques sur les récoltes de foins naturels que sur celles de foins artificiels, parce que les pluies gâtent plus promptement le produit des naturels s'il est très-abondant. Je visitai ce pré pour la première fois, en mai 1811, et je trouvai que la plus grande partie de l'herbe étoit composée de pâturins des prés, avec un peu de vulpin des prés, et de chiendent. Au printems 1812, j'examinai de nouveau la pièce très-attentivement. J'y trouvai les mêmes herbes que l'année précédente, mais leurs proportions étoient un peu différentes: il y avoit une plus grande quantité de chiendent. Au mois d'août, je répétai mon examen, pour voir si le fiorin prenoit le dessus, comme je m'y attendois; mais je ne sus point découvrir d'agrostis stolonifère. Les épis ou panicules de cette seconde coupe me parurent annoncer les mêmes herbes que l'on trouve communément\setcounter{page}{70} dans les pairies : seulement , le chiendent ou triticumrepens y dominoit davantage. Lorsque les prés gazons sont soumis au pâturage, le chiendent y périt bientôt, parce qu'il ne peut pas supporter le piétinement constant du bétail : toutes les fois qu'on trouve cette herbe dans les pâturages, c'est une preuve que le sol n'a pas été soumis depuis long-temps au parcours, ou qu'il est extrêmement fertile. Mais, me dira-t-on, qu'est-ce que cette longue herbe dont on a tant parlé? Il me reste à expliquer une circonstance particulière aux graminées de ce pré et de quelques autres du voisinage, laquelle je ne me rappelle pas d'avoir observé au même degré, et qui probablement, est due à la chaleur et à la richesse de ce sol. Lorsque l'eau commence à abandonner la pièce au printems, l'herbe versé par trop d'abondance, et cela arrive sur-tout aux pâturins. Cette herbe forme une natte sur le sol, et il pousse des racines à toutes les articulations. Avant que les panicules paroissent, on prendroit le pâturin et le vulpin, pour l'agrostis stolonifere. J'ai démêlé les tiges de ces deux plantes jusqu'à leurs racines parmi les autres plantes, dans une longueur de quatorze à\setcounter{page}{71} quinze pieds. A chaque articulation, il y avoit des radicules. Le dernier jet, qui étoit vertical, s'élevoit à deux pieds seulement, et la faux ne coupoit que cela : l'instrument glisse sur la natte qui couvre le sol, et n'y touche point.
L'Agrostis stolonifère est une des plantes les plus tardives de l'Angleterre : même cultivée, elle ne donne jamais une récolte avant l'automne. On se récriera en observant que, dans son état naturel, elle croît si abondamment qu'elle fait une grande partie de la récolte de foin coupée à la fin de mai. Je réponds que c'est une erreur, et que l'on prend pour l'agrostis stolonifère, d'autres graminées qui poussent aussi des racines à toutes leurs articulations.
\setcounter{page}{301}
\chapter{Agriculture}
\section{ESTIMATE OF THE EXPENSE ; etc. Estimation des frais de la culture des grains en Ecosse ( Farmer's Magazine ).}
Dans un temps où la discussion sur la législation des grains, à l'occasion des changemens proposés, occupe les esprits, il auroit été à désirer qu'on rassemblât les faits relativement au prix auquel le grain revient aux fermiers dans les divers sols et dans les trois royaumes. On auroit ainsi appris, non-seulement la moyenne du prix des grains, mais la proportion des élémens qui composent ces prix, savoir, les gages des ouvriers, le fermage et le profit. On auroit su avec une assez grande exactitude, ce que les grains coûtent aux fermiers; et les législateurs, en fixant les droits sur l'importation, auroient eu des données précises. Outre cela, le public auroit été en état de juger pourquoi les grains doivent être à un prix plus haut en Angleterre que dans les contrées qui en exportent. Le public auroit\setcounter{page}{302} su quelle est, par exemple, la portion d'un bushel de blé qui revient au manouvrier, au fermier, et au propriétaire; et si l'un de ces trois avoit été trop payé, on auroit pu examiner s'il convenoit de chercher à remédier à ce mal, ou s'il convenoit mieux d'en abandonner le remède à l'action puissante de l'intérêt des individus. Quelque parti qu'on eût pris, il y auroit toujours eu un bon effet de la chose: le public n'auroit plus été trompé par ceux qui ont intérêt à présenter la question sous un certain point de vue, et si le blâme eût dû tomber sur quelqu'un, il auroit été appliqué avec justice.
Une recherche de ce genre embrasse nécessairement deux questions distinctes: 1°. à combien revient le blé qu'on fait croître maintenant dans les Royaumes-Unis? 2°. Quels sont les élémens de ce prix, et par quelles causes pourra-t-il hausser ou baisser à l'avenir? A ce dernier égard, il seroit fort important de statuer de quelle manière les grandes importations qui ont eu lieu dans les cinquante, et sur-tout dans les vingt dernières années, ont affecté soit le prix auquel le grain revient au fermier, soit celui que paie le consommateur. Le prix des blés auroit-il été plus haut ou plus bas depuis\setcounter{page}{303} 1765, si l'on n'en avoit pas importé du tout, excepté dans les temps de véritable disette? Si l'on encourage également l'importation pour les cinquante années à venir, à des prix au-dessous de celui oú le blé revient au fermier, quelle en sera la conséquence?
Je n'ai pas l'intention d'entrer dans cette recherche. Lorsque les planteurs des isles se plaignirent que le prix du sucre étoit au-dessous de celui auquel cette substance revenoit aux planteurs mêmes par les dépenses inévitables de la culture, ils furent admis à soutenir leur plainte par le tableau détaillé de ces dépenses. Mais lorsque les comités des comtés se réunirent à Edimbourg pour prendre en considération les lois sur les grains, on assure que ces messieurs, non-seulement ne prirent pas connoissance des frais occasionnés par la culture du blé; mais que même ils déclarèrent qu'ils n'auroient aucun égard aux renseignemens positifs de ce genre, qu'on pourroit leur présenter. On résolut sagement de s'en remettre à la décision du Parlement. Mais si l'on vouloit connoître les faits; si l'on demandoit aux pétitionnaires ce qu'ils desirent, et sur quel fondement ils demandent que les lois sur les grains soient modifiées, je ne sais trop ce qu'on leur répondroit : je suppose\setcounter{page}{304} qu'on auroit recours à la mesure par laquelle on auroit dû commencer.
Je ne prétends parler que de ce qui regarde l'Écosse. Et d'abord, il est bien évident qu'on ne peut rien apprendre sur ce que coûtent réellement les grains, par la seule énumération des dépenses de labourage, semaille, hersage, etc. On ne peut recueillir plusieurs années successivement du grain, sans épuiser la terre, même la meilleure. Il faut des jachères mortes, ou des récoltes-jachères, selon le climat et le terrain; et une partie d'une ferme bien cultivée doit toujours être en herbages pour un an ou pour plus long-temps. Il est très-rarement convenable de prendre deux récoltes de grains successivement. Il faut donc connoître toute la dépense du cours de récoltes, en déduire la valeur moyenne des récoltes vertes, et appliquer les dépenses qui restent aux diverses récoltes de grains de l'assolement.
Il y a deux cours de récoltes généralement approuvés en Écosse par les meilleurs agriculteurs: l'un de six ans pour les terres argileuses, l'autre de quatre pour les terres légères et friables, qu'on nomme terres à turneps. On suit aussi d'autres assolemens, et il y a des terrains également propres à l'un et à l'autre, soit aux productions de\setcounter{page}{305} tous deux. On juge néanmoins suffisant, pour les calculs dont il s'agit, de s'en tenir aux deux cours principaux.
Il n'y a pas, en Ecosse, beaucoup de terrains qui puissent être cultivés longtemps, avec profit, sans une addition de fumier à celui qui résulte des récoltes mêmes. Il est donc très-ordinaire de conserver une partie d'une ferme en pâturage pendant deux ans et plus, ce qui enrichit le terrain de manière à ce qu'il exige moins de fumier quand on le sème en grains. Dans les calculs suivants, on ne suppose l'achat d'aucun fumier de basse-cour, et les produits sont estimés comme d'un sol d'une fertilité au-dessus de la moyenne. Il faut aussi expliquer que non seulement le terrain est de bonne qualité, mais qu'on peut y suivre une culture régulière, et qu'on n'a besoin d'aucune dépense extraordinaire pour y construire des bâtimens, des clôtures, des encaissemens, des épierremens, des desséchemens, etc. tout cela a été fait.
C'est un fait bien connu, que les fermiers Ecossais, lors même que leurs baux ne sont que de dix-neuf ans, font souvent de grands frais pour ces objets; et qu'il y a peu de fermes, qui dans les quatre ou six premières années, paient aux fermiers de telles avances.\setcounter{page}{306} Outre les dépenses énumérées, il faudrait donc mettre en ligne de compte le déficit dans les produits, ou les frais extraordinaires, afin de les compenser avec l'intérêt, avant l'expiration du bail. Cela est difficile à estimer, en général. Je n'ai parlé que du chaudage des terres; parce qu'il est d'un usage universel, au moins une fois dans un bail de dix-neuf ou vingt-un ans, à moins que la ferme ne soit placée pour se procurer d'autres engrais, lesquels ne sont pas moins coûteux. Si le terrain est naturellement assez fertile pour produire les récoltes que nous supposons, et cela sans fumier quelconque, la rente de la terre ou le prix de fermage doit s'élever dans la proportion de la somme que le fermier épargne en engrais.
Il y a une circonstance qui rend difficile d'estimer avec une certaine précision les frais de culture, même sur une ferme donnée; c'est la consommation de grains de la ferme elle-même. Les prix de ces grains varient selon les années et les situations, et comme l'estime doit s'en faire sans les conduire au marché, elle est toujours un peu vague.
Les dépenses de la culture varient selon l'étendue des exploitations. Dans les petites fermes, elles doivent être plus élevées que\setcounter{page}{307} dans celles qui admettent un arrangement systématique, et en quelque sorte une division du travail. Le petit fermier ne peut pas employer les instrumens les plus efficaces, mais qui sont aussi les plus chers. Le moulin à battre le blé donne, en particulier, un avantage bien décidé au grand fermier, sur le petit. Il faut donc prendre pour exemple un fermier qui emploie plusieurs charrues, et qui comporte l'usage des instrumens de l'agriculture perfectionnée : une étendue de trois cents acres tient probablement le milieu convenable entre les plus grands et les plus petits, pour les approximations dont il s'agit. Le capital d'un cultivateur se divise nécessairement en deux parts, qu'il faut considérer séparément. La première est appliquée en instrumens et animaux qui se détériorent plus ou moins, et en frais dans le sol. Pour balancer cette dépense, le fermier a droit à un certain intérêt qui lui remplace ces dépenses avant la fin de son bail, que j'ai supposé de 19 ans, comme ils le sont, en général, en Écosse. L'autre part du capital est employée pendant l'année, aux dépenses courantes, et doit rentrer en totalité dans le cours de l'année par le produit des\setcounter{page}{308} terres. Ces deux branches de dépenses sont les premières et les plus considérables.
La troisième dépense est le prix de fermage. Il y a eu à ce sujet quelques confusions d'idées : on a supposé que le fermage n'était pas une dépense nécessaire, mais qu'elle était en partie volontaire de la part du cultivateur. Cela n'est d'accord ni avec les principes généraux, ni avec la pratique. Le cultivateur doit payer au propriétaire de la terre, tout comme au manouvrier et à tous ceux qu'il emploie, la valeur de toutes leurs denrées au prix du marché; la présente recherche a en partie pour but de savoir de quoi dépend ce prix du marché.
Les impôts directs et indirects pèsent sur la culture, et quelques-uns comme l'impôt sur les propriétés et celui sur les chevaux de fermes, sont considérés comme nuisibles à l'agriculture. Les autres impôts doivent tomber sur le propriétaire, et il est seulement nécessaire de les comprendre dans l'estimation du fermage. Quant aux impôts indirects, il est bien difficile d'estimer au juste leur effet sur les dépenses de la culture et sur le prix de ses produits.
Le prix des denrées, après avoir remboursé toutes ces dépenses, doit laisser au\setcounter{page}{309} fermier un profit égal à ceux du commerce, c'est-à-dire, lui donner l'intérêt de son argent, couvrir ses pertes, et lui assurer un profit proportionné à sa peine dans la surveillance.
Comme le travail des chevaux forme une grande partie des charges de la culture, il semble que la marche la plus simple et la plus naturelle seroit de déterminer; 1°. ce que coûte un homme et une paire de chevaux pendant l'année, puis de débiter chaque ouvrage à tant par jour, selon l'application du travail de l'attelage. Le résultat donneroit, sans doute, une idée plus juste de la dépense de chaque opération particulière, qu'on ne l'obtient par la méthode ordinaire, qui est de diviser la totalité des dépenses par le nombre d'acres de la ferme; mais cette marche, extrêmement minutieuse, ne donneroit pas un résultat plus juste, relativement à la moyenne des terres arables. Le nombre d'acres auquel une paire de chevaux peut suffire varie, selon la situation, la nature du sol, le cours de récoltes, l'éloignement du marché, et les facilités pour se procurer les engrais, et le combustible. Mais d'après les données qu'ont fourni divers districts, il n'est pas difficile d'obtenir\setcounter{page}{310} une approximation assez juste sur la proportion de l'étendue des terres argileuses et des terres légères. Sur les faits recueillis, je crois qu'on peut fixer 50 acres de terre argileuse, et soixante de terres à turneps, comme l'étendue à laquelle deux chevaux peuvent suffire dans un cours de six et de quatre ans; en supposant que ces chevaux fassent tous les charriages, compris ceux de la chaux.
Au moyen de ces explications, je crois pouvoir présenter le tableau suivant des dépenses sous divers chefs déjà indiqués.
La table suivante est destinée à montrer les dépenses et les produits des terres arables argileuses ; le résultat donnera le prix d'un bushel de grain quelconque, ( blé, avoine et fèves ) selon les lois actuelles sur les grains.
Estimation des dépenses de la culture des grains en Ecosse, pour les terres argileuses, dans une rotation de six années, et avec un bail de dix-neuf ans.
Fonds capital pour le cheptel d'une terre de 300 acres anglais.
\setcounter{page}{311}
\comment{table}
Valeur du chédal....Epoque du renouvellement....Dépenses annuelles
Douze chevaux, à 50 livr. st.
et un cheval de selle, à 40. L. 640. —
Harnais, 5 livr. chaque. 65. —
Six charrues, à 4 livres sterl.
et deux cultivateurs, à 2 liv. 29. —
Douze herses ordin., à L. 1. 5
et quatre dites de près, à 1. 5
Deux plus fortes, à 2.12 25. 5
Deux rouleaux. 10. 10
Deux semoirs de fèves L. 2. 2.
Six grandes charettes, à 15
et si petites, à 4 et 4. 115. 4
Instrumens pour le service des écuries, brouettes, pelles, etc. 30.
Instrumens pour le service des granges, poids et mesures, sacs, etc.
Faux, rateaux, hoyaux, cordes, bèches, fourches, échelles.
Moulin à battre à six chevaux avec la machine à vaner. 170.
Chaudage 48 bolls (288 bushels) à 6 liv. sterl. par acre. 1800.
L. st. 2887. 1
Epoque du renouvellement. 8 ans.
6 ans.
19 ans.
Dépenses annuelles. L. 80 - 46 3 6
8 18 11
94 14 8
L. st. 229 17 3
\subsection{Déboursés Annuels}
\comment{table}
Un inspecteur 40 liv. st. Six laboureurs, à 35 liv. Deux manouvriers, à 30 liv. st. 310 — —
Journées pour répandre le fumier, sarcler, etc. 60 — —
Charron, maréchal, meunier, sellier. 50 — —
Réparation du moulin à battre le grain. 8 10 —
L. st. 428 10 —
\setcounter{page}{312}
D'autre part . . . . . . L.st. 428 10 Pour les chevaux 192 quarters d'avoine, à 25 sh. le quarter ..........237 10
Fourrage vert, vesces, etc. ......... 91 - Foin 130 stones, de 22 livres, à 10 pences 54 3 4
Pommes de terre ou rutabagas......... 19 10
Semences pour 200 acres dont 100 en blé, 50 en fèves et 50 en avoine, à 25 shel. par acre (y compris les vesces) ........ 250
Semences pour 50 acres en trèfle et ray-grass 50
Fauchage, charriage et réduction en meules de 25 acres de trèfle et ray-grass ( les 25 autres acres étant destinés à être coupés et mangés en vert ) à 15 shel. l'acre....... 18 15 
Récolte de 200 acres, à 12 shel. Dépenses incidentes 2s. 6 d. ........... 145
Dépenses de charroi au marché et pour ramener le charbon et la chaux, etc........ 20
Assurance des bâtimens de ferme et du bétail, réparations ............ 50
Prix de fermage, à 3 liv. st. l'acre, et impôts directs, à 4 sh. l'acre ......... 930
Intérêt annuel du chédal et fonds capital de ferme, savoir, liv. st. 2887. 1, à 10 pr. % 288 14
Intérêt des avances annuelles d'exploitation et des troupeaux achetés pour revendre, montant annuellement à L.st.1614. 8s. 4.d. à 7 1/2 pr. % .......... 121 1
Intérêts d'avances pour le fermier de la première année 800 yards cubes, à 3 s. 12 - st. 2976 1 -
\setcounter{page}{313} 
\subsection{Produit de la ferme de 300 acres.}
\comment{table}
50 Acres en jachère morte. bushels. Liv. sto
50 dits en blé après jachère, 30 bushels par acre ................ 1500.
50 dits en trêfle et ray-grass, à 6 liv. st. par acre ........................ 300
50 dits en avoine, après trêfle et ray-grass, à 48 bushels ................... 2400.
50 dits en fèves après avoine à 27 bushels 1350.
50 dits en blé après fèves à 26 bushels . 1300.
Total bushels. .......... 6550.
Le bushel à 8. 17. 2676
Total en argent. ...... st. 2976

D'après les droits fixés par les lois sur les grains, un bushel de blé doit être égal à 1 ½ de fèves et à 3 d'avoine. On peut donc déterminer de la manière suivante ce que revient chacun de ces grains. 4500 bushels de blé ont coûté à faire croître, sterl. 2676; c'est donc à 11 shillings et 897 millièmes le bushel. Les fèves reviennent à 7 shillings 929 millièmes, et l'avoine 3 shel. 964 millièmes, soit en nombres ronds 12 shillings, pour le blé, 8 pour les fèves et 4 pour l'avoine.
Je sens très-bien que de tels calculs sont sujets à être contestés; et j'ajouterai quelques observations sur les articles des dépenses.\setcounter{page}{314} Ce ne sera pas avec l'espoir de convaincre ceux qui ne parcourront cet article que dans le dessein de le critiquer; mais j'ai en vue de répondre à quelques objections importantes qui peuvent se présenter aux lecteurs de bonne foi.
Il y aura peu de différence d'opinion relativement aux dépenses foncières, et au chédal permanent: il n'y a que la chaux qui puisse donner lieu à quelque discussion. Le cheval de selle est nécessaire à celui qui va et vient pour les achats et les ventes, ainsi que pour remplacer occasionnellement un cheval de travail, s'il lui arrive quelque accident: on ne peut pas le regarder comme une dépense de luxe pour le fermier. Au lieu d'augmenter considérablement le nombre d'instrumens d'agriculture, j'en ai omis plusieurs, parce que l'usage n'en est pas général. Dans le nombre de ceux qui pourroient être utiles, je mentionerai le coupe-paille, une meule tournante; les bases de meules de grains, et les appareils à vapeur. Mais ces objets varient beaucoup dans leurs prix, soit à cause des différentes constructions, soit parce qu'ils sont faits de matériaux plus ou moins parfaits.
Si l'on prend en considération les prix d'Edimbourg, on trouvera mes évaluations\setcounter{page}{315} FRAIS DE LA CULT. DES GRAINS EN ÉCOSSE.
très-modérées. Le prix d'un rouleau de fer fondu, par exemple, est de dix-huit livres sterling et huit shillings à Édimbourg. Une simple brouette coûte de 3 sterl. à 3 et 10. Une charette, de 16 à 18 sterl., et un semoir de fèves, de 25 à 28 shel. On niera pas que le plus souvent, une machine à battre exigeant six chevaux ne coûte plus que je ne l'ai supposé. Quelle soit mise en mouvement par des chevaux ou de l'eau, par le vent ou la vapeur, les frais indispensables qu'elle exige dépassent toujours le prix de la machine elle-même. J'ai vu des fermiers qui avoient dépensé trois ou quatre fois cette somme pour des moulins à vent ou à eau. Il est vrai qu'en faisant usage de ces derniers moyens, on épargne le travail des chevaux, et par conséquent on en peut diminuer le nombre. J'ai pensé qu'il falloit m'en tenir à l'indispensable, quelque fût le moteur employé. Une machine à battre, bien construite, dans les dimensions les plus avantageuses pour une ferme de 300 acres, et une machine à vanner dans les mêmes proportions ne pourroient s'acquérir à moins de 170 sterl.
Il est bien vrai que la somme fixée pour le chaulage ne seroit pas toujours nécessaire. Dans beaucoup de cas, on ne chaude\setcounter{page}{316} qu'une partie de la ferme pendant le cours du bail, quelquefois même point du tout; si les terres l'ont été précédemment. Les fermes qui ont à leur portée et en abondance des plantes marines, le fumier des villes, ou de la marne coquillère, peuvent obtenir les récoltes supposées dans le tableau, sans faire de si grands frais pour le chaudage, et peut-être s'en abstenir complétement. Mais alors dans une situation aussi favorable, le fermage ne s'élèveroit-il pas au-dessus de 3 sterl. par acre? C'est une chose reconnue, que le fermage des terres meilleures que celles que j'ai supposées, est au-dessus de 3 sterl. l'acre. La somme fixée pour le chaudage doit servir pour tous les engrais quelconque pris hors de la ferme. Si le sol est assez riche pour qu'on n'ait pas besoin d'acheter des engrais, il est clair que le fermage augmentera en raison de cette fertilité. Vos lecteurs trouveront, dans l'Agriculture de l'Ecosse de Sir John Sinclair, et dans Les Rapports des Comtés; de nombreux exemples d'une plus forte somme allouée au chaudage d'un acre de terre dans un bail de dix-neuf ans, dans la plupart des comtés d'Ecosse. Je crois que loin de m'accuser d'exagération, ceux qui savent qu'on transporte souvent la chaux à vingt ou trente milles de distance, trouveront\setcounter{page}{317} trouveront qu'il auroit fallu y ajouter au moins la moitié du chariage.
Je suppose que les chevaux sont renouvelés au bout de huit ans; et en conséquence, je porte 12 1/2 pour cent en frais annuels. Un cheval acheté à l'âge de cinq ans peut servir, il est vrai, au-delà de huit ans; mais il est probable qu'à cette époque il ne pourra plus faire tout l'ouvrage qui lui est destiné dans la ferme que je suppose. D'ailleurs, il faut bien prendre un peu de marge pour les accidens possibles et les maladies.
Dans le nombre des articles qui doivent se renouveler au bout de six ans, quelques-uns dureront, sans doute, davantage: les rouleaux, par exemple, et peut-être les roues de charettes; mais la plus grande partie finira avant ce terme: il seroit absurde d'exiger une plus grande précision sur ces objets. Le moulin à battre est calculé de voir durer dix-neuf ans, c'est-à-dire, autant que le bail; et sa valeur au bout de cet espace de temps est réduite à peu de chose.
Le prix de la chaux doit évidemment rentrer au fermier avant l'expiration du terme. Peu importe l'époque où cet argent a été dépensé; que ce soit dans la première\setcounter{page}{318} ou dans la huitième année, la totalité doit être attribuée également à toutes les années du bail.
En explication de ce qui précède, je citerai le Dr. Smith, dans son ouvrage sur la Richesse des Nations.
"Les fermiers ordinaires, dit cet écrivain célèbre, emploient rarement un inspecteur, pour diriger les grandes opérations de leurs fermes. Souvent ils travaillent eux-mêmes, pour labourer, herser, etc. Après que le prix de la ferme est prélevé, ce qui reste du produit des récoltes ne doit pas seulement leur rembourser les frais de culture, et leur donner un profit ordinaire, mais leur valoir les gages qui leur sont dus comme laboureurs, et comme inspecteurs. On appelle ordinairement profit net, tout ce qui reste après avoir prélevé les déboursés et payé le prix de la ferme, cependant il est clair que les gages y sont compris. Si le fermier en se surchargeant de travail s'affranchit de la dépense d'un maître-valet ou d'un autre aide, il est évident qu'il doit gagner lui-même le gage qu'il leur donneroit : c'est pour cela que dans ce cas ces gages sont confondus avec le profit." Le Dr. Smith dans le\setcounter{page}{319} même chapitre distingué expressément le travail de l'inspection et de la direction des terres."
Mais je ne veux pas pousser les choses trop loin. Il est clair qu'il faut distinguer le fermier qui dirige sa ferme du maître-valet qui suit les opérations annuelles. Je ne puis pas croire que sur une ferme de 300 acres, qui emploie six charrues, et souvent un grand nombre d'hommes, de femmes et d'enfants, on puisse faire une telle économie sans perdre beaucoup plus que la valeur des gages. C'est un fait reconnu, qu'ordinairement l'intendant est celui qui est le plus occupé dans la ferme, le premier levé et le dernier couché; il est réellement un domestique pour tous les ouvrages: tantôt il tient la place d'un laboureur absent; tantôt il sème, il construit des meules de foin, ou il sert le moulin à battre. Si un fermier veut faire lui-même toutes ces choses, quoiqu'il possède un capital de plus de 4,000 st., il ne seroit pas juste de lui refuser la douceur de s'en décharger par un gage de 40 sterl. On donne dix fois autant aux commis d'un marchand, et personne n'imagine qu'on ne doive déduire ces sommes des profits mercantiles.
\setcounter{page}{320}Quelques-uns de vos lecteurs trouveront peut-être l'article de l'avoine attribuée aux chevaux exagéré. On peut voir dans le County Reports des données sur ce sujet, de même que dans l'Agriculture de l'Écosse de Sir John Sinclair. Il semblerait même par la correspondance de ce dernier qu'il voudroit en bannir l'usage entièrement; et qu'il suppose les frais de nourriture d'un cheval de travail au-dessous de l'entretien d'une vache à lait. J'imagine cependant que les fermiers qui espèrent faire labourer un acre chaque jour à une paire de chevaux pendant huit mois de l'année, ou quelqu'autre travail dans cette même proportion, trouveront nécessaire de leur donner trois fois par jour, de l'avoine, des fèves ou de l'orge, temps pendant lequel Sir John les suppose uniquement nourris de foin et de paille, et surtout de cette dernière; et que pendant les quatre autre mois, ils leur continueront la même nourriture deux fois le jour, lorsqu'ils les mettent au régime du foin et des vesces.
Je puis vous affirmer que mes chevaux et ceux de plusieurs de mes voisins mangent chaque jour, pendant six mois de l'année, un demi-bushel d'avoine, et qu'on leur en\setcounter{page}{321} donne trois fois le jour, jusqu'à-ce que la semaille des turneps soit achevée, quoiqu'ils aient été nourris avec abondance de trèfle et de ray-grass un mois ou six semaines avant le commencement des semailles. Cependant, malgré tout cela, ils sont sensiblement amaigris vers la fin du travail.
Il faut observer encore que cent stones de foin seulement sont attribuées à chaque cheval, pour toute nourriture pendant les trois mois de sécheresse où on ne peut avoir de fourrage en vert. On ne compte rien pour la paille; et cependant les tiges de pois et de fèves sont très-peu inférieures au foin en valeur; mais les résultats ne sont point échangés, puisqu'on compense la paille par le fumier: c'est une manière de simplifier le calcul, à moins qu'on ne soit placé près de quelque grande ville où on vend la paille et où on achète du fumier. Dans ce cas là, la paille et le fumier ont échangé leur valeur convenue. Le blé et la paille doivent être ajoutés au produit de chaque acre de blé; tandis que l'on doit faire entrer dans l'article des dépenses, la paille consommée par les chevaux.
L'estimation de l'assurance des bâtimens, du chédal et des clôtures pourra paroître\setcounter{page}{322} trop forte à quelques praticiens; mais je puis assurer, d'après ma propre expérience, qu'elle est plutôt trop foible.
Quant à l'article des contributions directes, il suffira d'observer, que je comprends sous ce titre tous les impôts publics et locaux qui tombent au fermier, uniquement comme chargé de l'exploitation de la ferme, et qui sont payées non pas volontairement comme les taxes sur les consommations, et prises sur les revenus ou le profit du fermier, mais sur le produit brut; de la même manière que le fermage ou la dixme. D'après ce principe, les taxes sur les loyers de maisons, sur les fenêtres, etc. doivent être mises au même rang que celles sur le vin, le thé et le sucre. Un fermier qui possède un capital de 400 sterl., payera une somme assez considérable sur ces articles. Il en est tout autrement des impôts qu'il doit payer nécessairement comme fermier, tels que ceux sur les chevaux de travail, sur les chevaux de selle, sur l'entretien des routes, le salaire des maîtres d'école, etc.
Comme la force de ces impôts est connue du fermier lorsqu'il contracte avec le propriétaire, il lui est parfaitement égal de les payer séparément, ou de les comprendre\setcounter{page}{323} dans le prix de la ferme; dans les deux cas, c'est toujours le propriétaire qui paye les taxes, à moins qu'on ne mette de nouveaux impôts pendant le cours du bail.
En Ecosse, les impôts ne sont pas très-onéreux, mais comme quelques-uns varient en quotité suivant les Comtés, il n'est pas aisé de décider le montant par acre. En les fixant à quatre shillings, j'y fais entrer une partie de la taxe sur les propriétés; c'est-à-dire, la quantité dont elle dépasse le revenu réel. Comme on ne peut pas mettre cet excédant sur le consommateur, il faut donc qu'il soit pris sur le revenu. Le revenu moyen lui-même, c'est-à-dire ce qui reste lorsqu'on a remboursé tous les frais, et l'intérêt du capital employé, n'est pas facile à estimer avec certitude, soit en denrées, soit en argent. Il n'est pas nécessaire de spéculer sur ce que doit être le revenu net de la terre pour le propriétaire; mais il s'agit de ce qu'il est pour une terre qui rend les récoltes indiquées dans la table: Je ne disputerai point avec ceux des lecteurs qui trouveront le prix de trois livres sterling de fermage trop haut ou trop bas; je les renvoie au dernier rapport publié par le bureau d'Agriculture. Je trouve encore\setcounter{page}{324} moins nécessaire de combattre ceux qui se récrient sur ce que l'avarice des propriétaires et la folie des fermiers ont élevé trop haut le prix des fermages : ce que je puis leur assurer, c'est qu'ils n'en payeront pas le pain un denier plus cher. Rien de plus ridicule que de prétendre que de hauts fermages renchérissent les denrées. Je renverrai, au besoin, à l'auteur de la Richesse des nations, ou à l'Essai sur la population, pour prouver que ce sont les hauts prix, tels qu'ils ont existé depuis neuf ans, (je pourrois même dire depuis vingt), qui ont fait hausser les fermages, bien loin d'être la conséquence du haut prix de ceux-ci. Je pense que quelques personnes s'étonneront de m'entendre affirmer que les grandes importations du blé ont été la cause des hauts prix, et conséquemment des hauts fermages. Le blé a coûté au consommateur 88 sterl. 11 shellings dans les neuf ans de 1804 à 1812 inclusivement ; et si l'on prend la moyenne des quatre dernières années à part, il a coûté 105,5 sterl. le quarter ; pourquoi s'étonneroit-on d'un marché entre le propriétaire et le fermier, qui fait revenir le blé à 96 sterl. le quarter, car il revient cela d'après le fermage moyen des dixneuf\setcounter{page}{325} dernières années. Quand nous examinons ce qui compose le prix du grain, nous verrons comment une diminution du fermage affecteroit ce prix.
On sait que lorsque le fermier et le propriétaire font un contrat de dix-neuf ans, ce n'est pas au prix du blé seul qu'ils ont égard. Si le terrain ne peut supporter le prix de ferme par l'industrie du labourage, il est probable qu'il le supportera par celle du pâturage. Aussi long-temps que les produits des herbages soutiendront leurs prix, ce ne sera pas le prix seul des grains qui déterminera le fermage. Si le terrain peut rendre cent quatre-vingts livres de bœuf ou de mouton par acre, à raison de sept pence la livre, il y aura 5 liv. 5 sols par acre, à partager entre le propriétaire et le fermier. Si nous prenons pour notre guide dans cette répartition, la taxe des propriétés, le propriétaire doit recevoir 3 liv. sterl. et le fermier 1 liv. 15 sols par acre. Sans doute c'est une chose fâcheuse pour la nation, qu'on mette les terres arables en pâturage; mais pour qu'on laboure, il faut que l'intérêt des deux contractans s'y trouve. Je n'ai pas besoin d'ajouter, que si on suivoit pour ces terres un bon système d'assolement, on augmenteroit\setcounter{page}{326} beaucoup la masse des grains produits, sans obtenir moins en viande et en laitage; par conséquent, le consommateur ne payeroit pas ces articles plus cher.
Le premier pas à faire pour obtenir d'abondantes récoltes de blé, c'est d'encourager cette industrie, de manière à ce que le fermier qui sémera du blé puisse rivaliser pour les profits avec l'engraisseur de troupeaux. Si la récolte de blé est très-considérable, il est clair que le prix de celui-ci doit baisser, quoique puissent faire le fermier et le propriétaire. Aussi long-temps que le consommateur consentira à payer huit à dix deniers, et même un shelling la livre de viande, qui dans d'autres pays est considérée comme un luxe, le propriétaire voudra retirer de sa ferme une rente proportionnée à ce haut prix des produits. Il est absurde de vouloir manger du très-beau pain pour deux deniers la livre, sans cependant diminuer le prix des fermes. Il coûte davantage à faire croître au propriétaire qui n'a point de ferme à payer.
Quant aux profits que le fermier peut obtenir, il y a peu de remarques à faire là-dessus. La première somme de dépenses de 2887 liv. sterl. et d'un shelling, et la\setcounter{page}{327} dernière de 120 liv. sterl. composent un capital distinct, duquel le fermier doit retirer la rente modérée qu'il obtiendroit dans un commerce quelconque. La seule objection que j'aperçoive, c'est que s'il achète de la chaux dans la même proportion pendant les six premières années, son bénéfice de dix pour cent ne pourra commencer qu'à la fin de la troisième année. Il sera toujours plus avantageux au fermier qui fait usage de la chaux, dans quelque proportion que ce soit, de l'employer dès le commencement de son bail. Plus il tarde, et plus il diminue ses chances de profit ; et si son bail est très-avancé, il court les risques de ne pas rentrer dans son capital. Je ne donne pas un grand poids à cette objection, mais je dis que si un fermier garde son argent dans sa poche pendant les cinq ou les dix premières années de son bail, le produit de sa ferme doit être moins fort durant cet espace de temps, que je ne l'ai supposé dans mon calcul.
On dira peut-être, pour les deux autres sommes, montant à 1614 liv. sterl. 8 sh. 4 d., qu'on ne peut leur allouer aucun bénéfice, puisqu'elles s'obtiennent en partie du produit annuel de la ferme. Mais on déduit ordinairement les semences du produit. Dans\setcounter{page}{328} la première année et peut-être dans la seconde, la dépense pour les semences de blé doivent entrer en compte comme le prix des grains de trèfle et de ray-grass. Que le fermier fasse croître ces grains dans la ferme, ou qu'il les fasse venir de Hollande, le résultat doit être le même. Le blé qu'il sème en septembre ne lui reviendra qu'au bout d'un an. Durant cet espace de temps, il auroit pu employer le prix des semences à un commerce profitable. On peut en dire autant de l'avoine et des autres articles de consommation; cependant, comme ces déboursés doivent rentrer dans le courant de l'année, je n'ai porté cet article qu'à 7 1 pour cent.
Encore une objection qu'on peut faire contre l'article des bénéfices, c'est que le fermier a une maison exempte de tout impôt, et que tout son bois de chauffage est transporté par les chevaux de labour. Cela est vrai; mais il faut se souvenir que chaque acre de sa ferme ne produit pas du grain ou d'autres récoltes. Tout l'espace de terrain occupé par les bâtimens, les clôtures, les chemins, les meules de foin et de paille, qui est compris dans les trois cents acres, auroit dû être déduit avant de faire la division\setcounter{page}{329} de ces trois cents acres en six portions, s'il avoit été possible de l'évaluation exactement. Supposons que ce soit le dix pour cent du sol, et dès lors trente acres ne rapportent rien, et cinq acres doivent être déduits de chacune de ces divisions. Si l'on suppose cinq pour cent seulement, alors deux acres et demi d'avoine; deux et demi de fèves, et cinq de blé, ont été comptés à tort dans les produits de la ferme, et ne peuvent être compensés par le loyer de la maison et le chariage de chauffage.
Selon le tableau, la dépense totale de la culture revient à 9 liv. sterl. 18 sh. 5 d. par acre. Si l'on en ôte le fermage et les impôts, il reste 6 liv. 14 s. 5 den. Si l'on en déduit encore les profits du fermier, il reste 5 liv. 6 sh. 3 den. C'est cette dernière somme qu'on peut regarder comme la dépense nécessaire de la culture de l'acre, selon les procédés les plus avantageux pour le terrain dont il est question : c'est ce qui doit être remplacé avant tout.
Je dois observer que sur six acres il y en a un en jachère morte, et que le produit de cet acre doit être défrayé par celui des cinq autres.....
\setcounter{page}{330}Pour éviter les fractions, on peut compter que le bushel de blé revient à douze shellings, lesquels se composent comme suit :
\comment{table}
... Lst. sh. den.
Dépenses de culture, par bush. o 5 10 1/2
Profits du fermier. . . . . . o 1 10 1/4
Fermage et impôts . . . . . . o 4 3 1/4
----Shellings — 12 0.
Si ces résultats se rapprochent beaucoup de la vérité, il est difficile de n'en pas conclure que la législation des grains doit éprouver des changemens. La seule déduction signifiante qu'on puisse faire sur le prix auquel le grain revient au fermier lui-même, ne peut porter que sur le fermage. Supposons le réduit de 25 pour cent, le blé reviendroit encore à 10 shellings 10 deniers le bushel; du moins pendant plusieurs années, et jusqu'à-ce que cette réduction eût fait son effet sur les articles énumérés parmi les dépenses de la culture.
\setcounter{page}{337}
\chapter{AGRICULTURE.}
\section{A GENERAL VIEW OF THE AGRICULTURE, etc. Tableau général de l'agriculture du Comté d'Aberdeen, rédigé sous la direction du Bureau d'Agriculture; par George Skene Keith, ministre de Keith-hall et Kinnel. (I. Vol. 8°. avec fig. ). \large{( Extrait. )}}
On sait que l'un des objets dont le Bureau d'Agriculture s'est principalement occupé, a été, et est encore, de se procurer sur l'état de la culture de tous les Comtés de l'Angleterre, des informations exactes et complètes, fournies par des agriculteurs intelligens et instruits dans chacune de ces provinces; de publier ces Rapports et d'en favoriser la circulation par tous les moyens possibles, et sur-tout dans les contrées même qui en étoient l'objet; ce qui a procuré une grande masse de renseignemens additionnels, qu'on a fondus ensuite dans une réimpression de ces mêmes Rapports, faite à mesure que leurs matériaux ont été complètement rassemblés.\setcounter{page}{338} Lorsque cette grande collection sera terminée, le Bureau se propose d'en publier un extrait, en deux ou trois volumes quarto, qui sera soumis à S. M., et aux deux chambres du Parlement ; et auquel succédera un Tableau général et systématique de l'état de l'Angleterre, sous tous les points de vue qui ont un rapport plus ou moins direct avec l'agriculture ; ainsi, à une époque qui n'est pas très-éloignée, chaque individu pourra se procurer :
1°. Un rapport détaillé de la culture de la contrée qu'il habite ; ou,
2°. Un tableau général de l'état de l'agriculture dans le royaume, divisé en Comtés, Districts, etc. ; ou enfin,
3°. Un système régulier d'instruction agricole, tant particulière à l'Angleterre, que renfermant des connoissances générales plus ou moins applicables à tout les pays.
Le volume de 700 pages, d'un petit caractère, que nous avons sous les yeux, est la réimpression du Rapport particulier de la culture du Comté d'Aberdeen, situé au nord de l'Ecosse, borné à l'est et au nord par l'océan. Une situation géographique aussi particulière ne laisse pas espérer qu'on trouve dans ce volume beaucoup d'informations ou d'instructions applicables au climat de la\setcounter{page}{339} France; aussi n'est-ce pas ce que nous y cherchons aujourd'hui ; nous nous bornerons a en extraire le modèle uniforme d'après lequel tous les Tableaux ou Rapports officiels de chaque Comté ('et celui-ci en particulier) sont ordonnés. Cet arrangement a été le résultat de la méditation et de l'expérience ; c'est le cadre raisonné de la partie agricole de la Statistique ; et sous ce point de vue, il doit intéresser non-seulement les agriculteurs mais les personnes qui aiment à s'occuper de vues générales, et à classer leurs idées et leurs connoissances dans un ordre régulier; disposition qui favorise beaucoup les recherches faites, et à faire. Voici le plan d'après lequel chaque Rapport a été arrangé dans la réimpression dont nous venons de parler.
\subsection{OBSERVATIONS PRÉLIMINAIRES.}
\comment{list}
\subsubsection{CH. I. Situation géographique et circonstances générales.}
Sect. 1. Situation et étendue.
2. Divisions.
3. Climat.
4. Sol et surface.
5. Minéraux.
6. Eaux.
\subsubsection{II. Etat de la propriété.}
Sect. 1. Terres et leur distribution.
2. Mode de possession.
\setcounter{page}{340}
\subsubsection{CH. III. Bâtimens.}
Sect. 1. Maisons des propriétaires.
2. Fermes et offices; réparations.
3. Chaumières.
\subsubsection{IV. Mode d'occupation.}
Sect. 1. Étendue des fermes. — Caractère des fermiers.
2. Revenu — en argent — en nature — en services.
3. Dîmes.
4. Taxe des pauvres.
5. Baux à ferme.
6. Dépense et profit.
\subsubsection{V. Moyens mécaniques de culture.}
\subsubsection{VI. Moyens d'enclorre. — Hayes. — Portes.}
\subsubsection{VII. Sol arable.}
Sect. 1. Labours.
2. Jachères.
3. Rotation des cultures, ou assolemens.
4. Cultures préférées.
5. Cultures peu ordinaires.
\subsubsection{VIII. Prairies.}
Sect. 1. Prairies naturelles et pâturages.
2. Prairies artificielles.
3. Récolte des foins.
4. Nourriture des bestiaux.
\subsubsection{IX. Jardins et vergers.}
\setcounter{page}{341}
\section{CH. X. Bois et plantations.}
\subsubsection{XI. Terres en friche.}
\subsubsection{XII. Améliorations.}
SECT. 1. Desséchemens.
2. Ecobuage.
3. Engrais.
4. Nettoyage.
5. Irrigation.
\subsubsection{XIII. Animaux de ferme.}
SECT. 1. Bêtes à cornes.
2. Brebis.
3. Chevaux.
4. Porcs.
5. Lapins.
6. Volaille.
7. Pigeons.
8. Abeilles.
\subsubsection{XIV. Économie rurale.}
SECT. 1. Travail -- domestiques, manœuvre
-- heures de travail.
2. Provisions.
3. Combustibles.
4. Dépense et profit.
\subsubsection{XV. Économie politique.}
SECT. 1. Routes.
2. Canaux.
3. Foires.
4. Marchés hebdomadaires.
\setcounter{page}{342}
5. Commerce.
6. Manufactures.
7. Pêche.
8. Pauvres.
9. Population.
\subsubsection{CH. XVI. Obstacles aux perfectionnemens, et observations générales sur la législation rurale, et la police.}
\subsubsection{XVII. Observations diverses.}
SECT. 1. Sociétés d'agriculture.
2. Poids et mesures.
\subsubsection{CONCLUSION. Objets et moyens d'améliorations, et mesures dirigées vers ce but.}

Nous extrairons des Observations préliminaires quelques conséquences générales que l'auteur du Rapport signale aux agriculteurs du Comté d'Aberdeen, comme étant le résultat des pratiques admises dans d'autres provinces et garanties par le succès. Ces leçons nous paroissent susceptibles d'une application plus étendue.
"Un grand défaut, dit-il, dans l'économie rurale du Comté d'Aberdeen, est le peu d'attention des fermiers à recueillir des engrais, et à préparer les fumiers ou les compost. Comme l'engrais est la force motrice\setcounter{page}{343} de toute la mécanique agricole, sa production est un objet de première importance. Il se passera sans doute beaucoup de temps avant que les fermiers de l'Aberdeenshire puissent atteindre le degré d'habileté de ceux de Norfolk dans l'art de ramasser et de préparer les divers engrais. Dans ce Comté, douze tombereaux de fumier suffisent à un acre anglais de turneps, ou quinze pour l'acre d'Ecosse; dans le comté d'Aberdeen, on en emploie trois fois davantage, faute de savoir mettre la même attention à le préparer. Peu de fermiers y ont des cours d'étables garnies de litière; et on y voit souvent le fumier de cheval en petits monceaux çà et là, mangés du soleil. Trop souvent les cendres de tourbes, le fumier des cochons, sont abandonnés à la pluie et au vent; et d'autrefois des tourbes entières sont mêlées au fumier sans division ou putréfaction préalable, c'est-à-dire, comme des corps étrangers sans utilité. Tout homme qui aura remarqué avec quel soin on prépare les engrais dans les parties méridionales de l'Ecosse, sera frappé de la supériorité de ces contrées sur celles du nord, sous ce rapport.
Un autre défaut dans l'économie rustique du comté d'Aberdeen, est de trop négliger l'emploi des femmes pour la petite culture.\setcounter{page}{344} et le nettoyage des champs. La nécessité, mère de l'industrie introduira probablement bientôt cette pratique avantageuse. On trouve peu d'hommes qui se vouent à l'état de journaliers, et les femmes ont peu à gagner dans le seul ouvrage auquel elles s'employent, c'est-à-dire, le tricotage des bas, qui ne leur rapporte que quatre pence (huit sols de Fr.) par jour; et puisqu'on ne trouve pas, au plus haut prix, des journaliers, il est humain et politique à-la-fois d'employer des femmes pour tous les ouvrages qui exigent peu de force; comme ramasser les pierres, recueillir les foins, et cultiver à la houe les pommes de terre et les turneps. Il y a eu des temps, et il existe encore des localités dans lesquelles les femmes étoient chargées des parties du travail les plus pénibles; leurs seigneurs et maîtres s'occupoient à remplir de fumier les kessies (paniers faits de paille) que les femmes transportoient sur leur dos jusqu'aux champs. Il y a lieu d'espérer que ces pratiques ne s'introduiront jamais dans le comté d'Aberdeen. On ne doit point charger un être humain du travail qu'une bête de somme peut faire. Mais, tandis que les hommes labourent à la bêche, ou conduisent la charrue, nos femmes pourroient fort bien se charger du\setcounter{page}{345} reste de la culture \footnote{Dans plusieurs parties de la Savoie, et dans la vallée de Chamonix en particulier, si supérieurement cultivée, les femmes font les plus gros ouvrages, elles fossoyent, fauchent, moissonnent, etc. (R)}. Il est agréable de voir dans un champ, des femmes occupées à cultiver des turneps, ou à arracher les mauvaises herbes; on a souvent ce plaisir dans les contrées méridionales de l'Angleterre, mais bien rarement dans l'Aberdeenshire. Cependant le sol de ce Comté est particulièrement adopté aux cultures légères, telles que celle des turneps; on pourrait doubler l'étendue actuelle de culture, si on savait préparer les engrais et employer utilement les femmes. Ainsi il serait de l'intérêt des propriétaires et des fermiers, d'encourager par des primes et par des journées bien payées, les femmes à aider les domestiques et les journaliers dans la culture de récoltes, qui sont très profitables, lorsqu'elles sont soignées. Dans les parties méridionales de l'Écosse, on donne aux femmes qui viennent travailler avec les domestiques, un tiers en sus du gage ordinaire; et il n'est pas douteux que le même encouragement ne produisit les mêmes effets dans le nord.
Une troisième erreur assez générale dans\setcounter{page}{346} le comté d'Aberdeen, est l'usage des propriétaires de prescrire à leurs fermiers ou tenans, de ne pas dépasser un certain nombre de sous-tenans, sans limiter d'ailleurs la quantité de terrain qu'ils pourront attribuer à chacun. L'objet principal de la limitation était la rareté de la tourbe, à laquelle, comme principal combustible, le propriétaire attachoit beaucoup de valeur. Tout sous-fermier qui a un char et un cheval peut amener chez lui une quantité de tourbe qui suffirait à trois journaliers; et ceux-ci, qui n'ont point de véhicule, sont obligés d'en louer à prix d'argent pour ce transport, et par conséquent à l'économiser. Il vaut mieux que le fermier seul ait le privilége des attelages; les sous-tenans doivent être, ou des domestiques mariés, attachés à la ferme, et journaliers; ou des artisans que le fermier peut avoir occasion d'employer, tels que charrons, forgerons, etc. et encore, ceux-ci ne doivent-ils avoir que peu de terrain à cultiver, afin de ne pas négliger leur métier principal. Ainsi, dans le midi de l'Ecosse, le fermier s'entoure de journaliers qui habitent des chaumières aux environs, y cultivent un peu de terrain, et sont employés comme journaliers ordinaires et affidés, pour battre le blé, creuser les fossés, et faire tous les\setcounter{page}{347} ouvrages courants qu'exige la culture. Un fermier prudent et habile ne sauroit trop multiplier autour de lui cette classe de travailleurs, ni trop réduire celle des sous-traitans qui ont un attelage à eux, et sous-louent une portion un peu considérable du sol de la ferme.
Si le comté d'Aberdeen peut prendre des leçons des autres, il a pû leur en donner une, qu'il a reçue lui-même de la nature et de la nécessité? Sa principale récolte a été, pendant nombre d'années sur une partie considérable du sol, l'exploitation des granites roulés dont il étoit presqu'entièrement couvert, et qu'on a employés au pavé de Londres et à divers édifices; le sol ainsi travaillé, rendoit de 30 à 50 L. st. par acre. Il est demeuré semé de gros blocs qui ont forcé à adopter deux genres différens de culture, celle au hoyau, et celle à la charrue, selon les localités; cette double faculté et ce mélange, ont eu les plus heureux résultats, et méritent d'être imités même là où les circonstances n'y forcent pas, comme elles le font dans le Comté d'Aberdeen.
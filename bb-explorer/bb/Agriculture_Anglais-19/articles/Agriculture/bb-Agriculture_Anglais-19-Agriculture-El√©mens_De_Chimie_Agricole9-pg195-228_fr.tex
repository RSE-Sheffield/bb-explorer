\setcounter{page}{195}
\chapter{Agriculture}
\section{ELEMENTS OF AGRICULTURAL CHEMISTRY, etc. Élémens de chimie-agricole en un Cours de leçons pour le Département d’Agriculture ; par Sir HUMPHRY DAVY. Londres, 1813. \large{(Neuvième extrait. Voy. p. 145.)}}}
De tous les engrais excrémentiels, le fumier des oiseaux qui se nourrissent de viandes, est le plus énergique dans ses effets. Les oiseaux de mer sont principalement remarquables sous le rapport. Le guano, dont on fait un grand usage dans l’Amérique méridionale, est l’engrais qui fertilise les plaines stériles du Pérou. Mr. Humboldt nous apprend qu’il est très-abondant dans les isles de Chinché, d’Ilo, d’Iza, et d’Arica, dans la mer Pacifique. On en charge toutes les années, à Chinché, cinquante vaisseaux, dont chacun porte de 1500 à 2000 pieds cubes de cette matière. On l’applique en très-petites quantités, et principalement à la culture du maïs. Je fis quelques expériences en\setcounter{page}{196} 1805, sur des échantillons de cette substance, qui avoient été envoyés au Département d'Agriculture. C'étoit une poudre brune très-fine. Elle noircissoit par l'action de la chaleur, et donnoit une forte fumée ammoniacale. Traitée à l'acide nitrique, cette substance donna de l'acide urique. En 1806, MM. Fourcroy et Vauquelin publièrent une analyse soignée du guano. Ils établissent qu'il contient le quart de son poids d'acide urique, saturé en partie d'ammoniaque, en partie de potasse; et en outre, de l'acide phosphorique combiné avec les mêmes bases, ainsi qu'avec de la chaux. On y trouve encore du muriate et du sulfate de potasse, un peu de matière graisseuse, et de sable quartzeux.
Ses qualités fertilisantes sont faciles à expliquer; et d'après ses composants cette substance doit être un puissant engrais. Elle a besoin d'eau pour dissoudre ses matières solubles.
Le fumier des oiseaux de mer n'a jamais été employé en Angleterre; mais il est probable que le sol même des iles de la côte que ces oiseaux fréquentent beaucoup, seroit un bon engrais. A mon invitation, Sir Robert Vaughan essaya du fumier d'oiseaux de mer apporté de Mérionetshire, l'effet en\setcounter{page}{197} fut grand, mais passager, sur une prairie.
Dans le climat de l'Angleterre les pluies doivent promptement altérer cet engrais, lorsqu'il y est exposé; mais on le trouvoit, avec toute sa force, dans les grottes et les abris de rochers fréquentés par les cormorans et autres oiseaux de mer. J'ai examiné du fumier de cormorant trouvé sur un rocher, près du cap Lézard. Il ne ressembloit point au guano: il étoit blanchâtre, avoit une odeur fétide, comme de matière animale en putréfaction. Traité à la chaux, il donnoit beaucoup d'ammoniaque, et avec l'acide nitrique, il produisoit de l'acide urique.
Le fumier humain est, comme on sait, un très-puissant engrais, et très-disposé à la décomposition. Quoiqu'il diffère dans les proportions de ses composans, il a toujours beaucoup de carbone, d'hydrogène, d'azote, et d'oxigène. Il paroît, d'après l'analyse de Berzolius, qu'il est toujours partiellement soluble à l'eau; et, soit avant soit après la fermentation, il donne un aliment abondant aux plantes. L'odeur désagréable de ce fumier peut être prévenue par le mélange de la chaux-vive. Si l'on étend cette matière mélangée de chaux, en couches minces, par un temps sec, elle se pulvérise aisément,\setcounter{page}{198} et peut s'employer dans la culture au sémoir, comme les gâteaux de colza.
Les Chinois, qui ont plus de connoissances pratiques sur l'usage des engrais qu'aucun autre peuple, mêlent celui-là avec une marne grasse, pour en faire des gâteaux, que l'on sèche au soleil. Les missionnaires Français nous apprennent que ces gâteaux, n'ont aucune mauvaise odeur, et sont un article de commerce. Il est probable que la terre, par sa faculté absorbante, prévient l'effet décomposant de l'eau, et défend cet engrais de l'action de l'air.
Après le fumier humain, le fumier de pigeon occupe le premier rang pour la force fertilisante. En faisant bouillir pendant quelques heures, dans l'eau, cent grains de fumier de pigeon frais, j'ai obtenu vingt-trois grains de matière soluble qui, à la distillation, a produit beaucoup de carbonate d'ammoniaque, en laissant pour résidu une matière charboneuse, du sel commun, et du carbonate de chaux. Le fumier de pigeon fermente aisément quand il est humide, et contient ensuite moins de matière soluble qu'auparavant. De cent parties de ce fumier fermenté, je n'obtins que huit parties solubles, lesquelles donnèrent proportionnément moins de carbonate d'ammoniaque à\setcounter{page}{199} la distillation, que le même fumier frais.
Il est évident que ce fumier doit être appliqué le plus promptement possible; et lorsqu'il est sec, il faut le réduire en poudre pour l'employer dans le sillon avec la semence, quand on sème celle-ci au semoir.
Dans les bois où les pigeons sauvages couchent en bandes nombreuses, le sol est souvent imprégné de leur fiente, et il feroit certainement un bon engrais. En distillant ce sol avec la chaux, j'ai obtenu de l'ammoniaque.
La fiente de volaille s'emploie, ainsi que celle de pigeons, par les tannéurs, pour donner un premier degré de fermentation aux peaux destinées à devenir très-souples. Pour cela, on fond le fumier dans l'eau, et on le laisse subir la putréfaction, laquelle se communique à la peau. Les excrémens de chien s'emploient, par les tannéurs, pour un but semblable. Le résidu des fossés où cette opération se fait, est, dans tous les cas, un excellent engrais.
Le fumier de lapins n'a jamais été analysé. Mr. Fane entretient des lapins tout exprès pour leur fumier, et y trouve son compte. Il l'emploie aussi frais qu'il est possible, et il éprouve que moins il a fermenté, plus il a d'effet.\setcounter{page}{200} MM. Einhof et Thaer ont examiné chimiquement le fumier de bœuf et de vache. Ils ont trouvé qu'il contenoit des matières solubles à l'eau, et que dans la fermentation il donnoit à-peu-près les mêmes produits que les substances végétales, absorbant l'oxygène, et produisant le gaz acide carbonique.
Le fumier de mouton et de daim, lorsqu'il est frais, donne par une longue ébullition, des matières solubles qui vont de deux à trois pour cent du poids total. J'ai examiné ces substances solubles procurées par solution et évaporation: elles contenoient une petite quantité de matière analogue au mucus, et étoient principalement composées d'un extrait amer, soluble dans l'eau et par l'alcool. Elles donnent de l'alkali volatil, par la distillation. J'arrosai pendant plusieurs jours successifs quelques plantes de graminées avec une solution de ces extraits: elles devinrent évidemment d'un vert plus vigoureux que celles qui étoient d'ailleurs dans les mêmes circonstances.
La partie du fumier de vaches, de moutons, ou de daims, qui n'est pas soluble à l'eau, paroît être de la fibre ligneuse pure, et est précisément analogue au résidu des\setcounter{page}{201} végétaux qui servent à leur nourriture, lors que ces végétaux ont été privés de tous leurs matériaux solubles.
Le fumier de cheval donne un fluide brun, lequel fournit, par l'évaporation, un extrait amer; et celui-ci donne de l'alkali volatil en plus grande abondance que le fumier de bœuf.
Si le fumier des bêtes à cornes est employé seul, il n'y a pas de raison pour le faire fermenter avant que de l'enterrer : il faut du moins que cette fermentation soit légère, si elle a lieu. L'herbe qui se trouve dans le voisinage immédiat du fumier de vaches frais, a toujours une végétation trop forte, et qui rend ces plantes de mauvaise qualité. Quelques personnes ont attribué cela à une influence nuisible du fumier frais : il faut uniquement l'attribuer à un excès de nourriture fournie à la plante.
La question de la manière la plus convenable d'employer le fumier de chevaux et de bétail appartient au chapitre des fumiers composés, parce qu'on le mêle ordinairement avec la paille dont on fait la litière: il contient lui-même une grande portion de matières fibreuses végétales.
Une légère fermentation commençante dans le tas est sans doute utile ; car ce commencement\setcounter{page}{202} cement de fermentation dispose la fibre ligneuse à se dissoudre lorsque le fumier est mis en terre; or la fibre ligneuse est toujours très-abondante dans les rebuts que l'on jette au fumier d'une ferme.
Une fermentation trop prolongée est néanmoins très-préjudiciable aux fumiers composés: il vaut mieux qu'il n'y ait aucune fermentation que d'en avoir trop. Nous avons déjà établi ce principe dans les premières leçons. L'excès de la fermentation tend à dissiper les substances les plus utiles à la végétation; et le résultat final de ce procédé est le même que celui de la combustion.
Il est d'usage parmi les agriculteurs de laisser fermenter le fumier jusqu'à-ce que la matière fibreuse des végétaux soit rompue, et que le tas soit devenu tout-à-fait froid. Dans cet état, il se coupe aisément à la bêche.
Indépendamment des raisonnemens qu'indique la théorie contre cet usage, il y a à présenter un grand nombre de faits qui le condamnent.
Pendant la fermentation violente qui est indispensable pour réduire le fumier d'étable à l'état dont nous venons de parler, il se perd, non-seulement beaucoup de fluides, mais aussi beaucoup de gaz, ensorte\setcounter{page}{203} que le fumier est réduit d'une moitié jusqu'à deux tiers en poids. Les principales matières dégagées sont de l'acide carbonique et de l'ammoniaque : or l'un et l'autre peuvent devenir la nourriture des plantes, ainsi que nous l'avons vu ci-devant.
En octobre 1808, je remplis une grande cornue qui pouvoit contenir trois pintes, avec du fumier d'étable en fermentation. J'appliquai un récipient à la cornue, et plaçai le tout sur un appareil pneumatique au mercure, de manière à recueillir les liquides et les fluides élastiques qui s'échappèrent du fumier. Le récipient fut bientôt garni de gouttes de liquides qui coulèrent ensuite dans les côtés. Il se dégagea également des fluides élastiques. Dans trois jours, j'en recueillis trente-cinq pouces cubes. L'analyse me montra que ce fluide élastique étoit composé de vingt-un pouces cubes d'acide carbonique, et que le reste étoit de l'hydrocarbonale avec un peu d'azote, mais probablement pas plus qu'il n'en existoit dans l'air commun du récipient. La quantité du fluide recueilli en même temps montoît à près d'une demi once : il avoit un goût salin, une odeur désagréable, et il contenoit de l'acétate, et du carbonate d'ammoniaque,\setcounter{page}{204} Lorsque j'eus recueilli de tels produits de fumier en fermentation, j'essayai d'introduire le bec d'une autre corhue également pleine d'un fumier de même qualité et en fermentation, de l'introduire, dis-je, dans la terre, sous les racines d'un gazon, dans un jardin. Au bout de huit jours, l'effet devint extrêmement sensible sur l'herbe soumise à l'influence des gaz : elle végéta avec beaucoup plus de vigueur que le reste de ce même gazon.
Outre la déperdition des gaz quand la fermentation est poussée à l'extrême, il y a aussi un désavantage dans la perte de la chaleur. Celle-ci est utile pour hâter la germination des graines, pour aider la jeune plante dans son premier développement, pour conserver une température douce jusques dans l'arrière-automne, et pendant l'hiver.
D'ailleurs, il est démontré en chimie, que les substances se combinent beaucoup plus aisément dans l'instant où elles se dégagent, que lorsqu'elles ont été complètement formées. Dans la fermentation sous terre, la matière produite est appliquée à l'instant, et encore chaude, aux organes des plantes; et elle agit conséquemment avec plus d'efficace que par le fumier qui a déjà fermenté,\setcounter{page}{205} et dont les principes ont passé à des combinaisons nouvelles.
Dans les ouvrages des agriculteurs, on peut trouver une grande masse de faits en faveur de l'application du fumier frais. Mr. Young, dans son Essai sur les engrais, donne un grand nombre d'autorités pour appuyer cette pratique. Bien des gens qui doutaient ont été récemment convaincus; et il n'y a peut-être aucun objet de recherches dans lequel il y ait tant de coïncidence entre les résultats de la théorie et ceux de la pratique. Depuis dix ans, j'ai observé moi-même un grand nombre de faits qui ne laissaient aucun doute. Je citerai seulement celui de tous les agriculteurs dont l'autorité sera, j'en suis sûr, du plus grand poids pour le public, Mr. Coke. Depuis sept ans, il a abandonné le système qu'il suivait autrefois, de n'employer que du fumier fermenté. Il m'apprend que ses récoltes sont tout aussi belles, et qu'il a, en volume, presque le double de fumier à répandre.
On objecte contre l'emploi du fumier frais, que les mauvaises herbes en sont plus abondantes. Si l'on apporte les mauvaises semences avec le fumier, elles germent assurément; mais il est rare que cela puisse aller très loin; et si le terrain n'est pas nettoyé de\setcounter{page}{206} mauvaises herbes, le fumier, quel qu'il soit, fermenté ou non, encourage leur végétation. Lorsqu'on emploie le fumier frais sur les prés, il faut avoir soin de herser et rateler la paille qui reste à la surface quand l'herbe commence à végéter. Cette paille est replacée sur le tas de fumier. De cette manière, rien n'est perdu, et la culture est économique, en même temps qu'efficace.
Lorsqu'on ne peut pas employer immédiatement les fumiers aux récoltes, il importe d'en prévenir la fermentation autant qu'il est possible. Nous avons déjà vu sur quels principes on doit y travailler.
La surface du tas doit être défendue avec soin de l'oxigène de l'atmosphère. Une marne ou glaise compacte est la substance la plus efficace pour envelopper et défendre le tas de l'influence de l'air; et avant de le sceller ainsi, il faut avoir soin que le fumier soit bien sec. Si l'on s'aperçoit qu'il s'échauffe, il faut le retourner pour le refroidir par le contact de l'air.
On a quelquefois recommandé d'arroser les fumiers pour arrêter le progrès de la fermentation. Cela n'est pas conséquent avec les principes de la chimie. En mouillant le tas, on le refroidit pour un peu de temps; mais l'humidité est un agent principal dans\setcounter{page}{207} le procédé de la fermentation putride. La matière fibreuse sèche ne fermente jamais: l'eau est aussi nécessaire que l'air à ce phénomène de la décomposition ; et lorsqu'on en donne au fumier, on lui fournit un agent qui accélère le changement d'état.
Il y a des signes certains et simples, auxquels on peut juger de la rapidité de la décomposition d'un fumier qui fermente, et par conséquent, du mal qui se fait.
Si le thermomètre de Fahrenheit, plongé dans le tas de fumier, n'indique pas une chaleur au-dessus de 100°, il y a peu de danger d'une déperdition considérable de matières aériformes. Si la température est plus élevée, il faut conduire et répandre le fumier immédiatement.
Si l'on tient suspendu au-dessus d'un tas de fumier qui fume, un morceau de papier trempé dans l'acide muriatique, et que ce papier donne une vapeur épaisse, c'est une preuve qu'il se dégage de l'alkali volatil, et que la fermentation va trop loin.
Lorsqu'on veut conserver le fumier longtemps, le local où on le place est d'un choix important. Il faut, si l'on peut, le défendre du soleil. Il seroit fort utile de le tenir sous des hangars, et du côté du nord. Le fond sur lequel repose le tas de fumier doit être\setcounter{page}{208} incliné vers le centre, et pavé en pierres plates. Des rigoles doivent se diriger vers ce centre pour y rassembler l'eau de fumier, qu'une pompe enlève à volonté, et qu'on distribue ensuite sur les terres. On laisse trop souvent perdre un fluide extractif épais, qui se dessèche peu-à-peu et sans aucun usage.
Les raclures des rues, des chemins, les balayures des maisons, sont des fumiers composés: leurs qualités sont diverses, comme leurs composans sont variés. Ordinairement ces engrais sont appliqués convenablement, c'est-à-dire, sans fermentation préalable.
La suie qui est formée par la combustion du charbon de terre, contient aussi des substances qui proviennent des animaux. C'est un puissant engrais. Elle donne à la distillation, du sel ammoniac, et fournit, par l'eau chaude, un extrait brun d'un goût amer. Elle contient une huile empyreumatique. Sa base est du charbon, dans un état qui le rend soluble par l'action de l'oxigène et de l'eau.
Cet engrais est très-propre à être employé sec et sans préparation: il faut le mettre dans le sol avec la semence.
La doctrine de la meilleure méthode pour l'application des fumiers qui proviennent des substances organisées, offre un exemple important\setcounter{page}{209} de la grande économie de la nature et de l’ordre dans lequel tout a été disposé. La mort et la décomposition des animaux tendent à résoudre ces corps organisés en leurs élémens chimiques. Les émanations pernicieuses qui se dégagent pendant la fermentation, semblent indiquer la convenance de les mettre en terre, où elles deviennent la nourriture des plantes. La fermentation et la putréfaction des substances organisées, sont nuisibles lorsqu’elles se passent à l’air libre: ce sont, au contraire, des opérations utiles, lorsqu’elles ont lieu dans la terre. La nourriture des plantes est alors préparée là où elle peut être employée. Ce qui seroit un objet de dégoût et une cause de maladies par l’exposition au grand air, devient, par le mélange avec le sol, la cause de la création de nouvelles substances utiles, et de formes gracieuses. Le gaz fétide produit les parfums de la fleur; et ce qui auroit été un poison de l’air, est converti en alimens sains, pour les animaux et pour l’homme.\setcounter{page}{210} Des engrais minéraux ou fossiles; de leur préparation; de la manière dont ils agissent. De la chaux dans ses divers états; action de la chaux comme engrais et comme ciment. Combinaisons différentes de la chaux. Du gypse; de son usage, et d'autres sels neutres employés comme engrais. Des alkalis, et des sels alkalis; du sel commun.
La conversion de la matière organisée, en nouvelles formes organisées, est un procédé que l'on peut aisément concevoir: il est plus difficile de comprendre les opérations par lesquelles les matières salines et terreuses s'insinuent et deviennent solides dans les fibres des plantes, et comment elles sont utiles aux fonctions de la vie végétative.
Quelques philosophes adoptant le système des anciens, que l'essence de la matière est la même, et que ce que les chimistes appellent des élémens divers, n'est qu'un arrangement des mêmes particules indestructibles ont cherché à prouver que tous les principes contenus dans les plantes peuvent se former de l'atmosphère, et que la végétation est un procédé dans lequel les corps qu'on ne peut ni changer ni former, sont continuellement composés et décomposés. Cette opinion n'a pas été avancée seulement comme une hypothèse:\setcounter{page}{211} pothèse: on a essayé de la soutenir par des expériences. MM. Schrader et Braconnot sont arrivés aux mêmes conclusions, par deux suites d'expériences distinctes l'une de l'autre. Ils établissent que diverses semences mises dans le sable, le soufre, les oxides métalliques, et pourvus d'air et d'eau, ont donné des plantes bien constituées, lesquelles ont fourni, à l'analyse, diverses matières salines et terreuses, qui n'étoient contenues ni dans les semences ni dans les substances où l'on avoit déposé celles-ci, ou du moins ne s'étoient trouvées qu'en quantité beaucoup moindre dans les semences. Ces savans ont conclu de là, que les matières salines et terreuses s'étoient formées de l'air et de l'eau, par l'action des organes des plantes. Les recherches de ces deux philosophes ont été conduites avec beaucoup de génie et d'adresse ; mais les résultats de leurs expériences ont été influencés par des circonstances qui ne pouvoient pas leur être connues dans le temps où ils ont travaillé, parce qu'on n'avoit pas fait alors les découvertes qui ont eu lieu depuis. J'ai éprouvé que l'eau distillée étoit loin d'être exempte de sels en solution. En analysant cette eau, par l'appareil voltaïque,\setcounter{page}{212} j'y ai trouvé des alkalis et des terres; et plusieurs des combinaisons des métaux avec le chlore sont extrêmement volatiles. Si l'on fournit aux plantes une grande abondance d'eau distillée, celle-ci leur donne diverses substances; qui, bien qu'en quantité à peine sensible dans l'eau, peuvent s'accumuler dans la plante, laquelle ne perd probablement par la transpiration que de l'eau pure.
En 1801, je fis une expérience sur la végétation de l'avoine, en lui fournissant une quantité limitée d'eau distillée, et dans un sol composé de carbonate de chaux pur. L'eau et le sol furent placés dans un vase de fer renfermé dans une grande jarre, laquelle communiquoit avec l'atmosphère par un tube recourbé, de manière à empêcher l'entrée de la poussière, et de tout liquide ou corps solide, dans la jarre. J'avois pour objet de m'assurer s'il ne se formoit point de silice dans le procédé de la végétation. L'avoine végéta faiblement, et jaunit avant que les fleurs fussent formées. Je brûlai les plantes entières, et je comparais la cendre avec le même poids de cendres de grains d'avoine. La cendre des plantes donna moins de silice que la cendre des grains; mais la première donna beaucoup plus de carbonate de chaux.\setcounter{page}{213} J'attribue le fait de la moindre quantité de silice à ce que l'enveloppe du grain se détache dans la germination, et que c'est la partie qui abonde le plus en silice. Des plantes vigoureuses d'un champ d'avoine dont le sol étoit un sable fin, donnèrent beaucoup plus que des plantes obtenues par artifice. Les résultats généraux de cette expérience sont opposés à l'idée que les terres puissent se former dans les plantes, par les élémens que leur fournisent l'air et l'eau. D'autres faits combattent cette idée. Jaquin établit que la soude, lorsqu'on la cultive dans des situations éloignées de la mer, donne de l'alkali végétal; et que lorsqu'elle végète au bord de la mer, où abondent les composés qui fournissent l'alkali minéral ou marin, elle donne cet alkali. Duhamel trouva que les plantes qui d'ordinaire croissoient au bord de la mer, faisoient peu de progrès quand on les transportoit dans les terres où il y avoit peu de sel marin. Le tournesol, lorsqu'il croît dans des terrains qui ne contiennent pas du nitre, ne donne pas cette substance: s'il est arrosé avec une solution de nitre, il le donne ensuite abondamment. Les tables de De Saussure montrent que les\setcounter{page}{214} cendres des végétaux sont semblables, dans leur composition, au sol dans lequel elles ont végété. Il fit croître des plantes dans diverses solutions de sels, et il s'assura que toujours une partie du sel étoit absorbée par les plantes, et se retrouvoit sans altération dans leurs organes.
Les animaux ne paroissent pas non plus avoir la faculté de former en eux les substances alkalines et terreuses. Le Dr. Fordice a observé que lorsque les canaris n'ont point d'accès au carbonate de potasse dans le temps où les femelles font leurs œufs, ceux-ci ont une enveloppe tendre, au lieu de coquilles; et s'il y a un procédé dans lequel la nature puisse fournir des ressources de ce genre, c'est probablement celui de la reproduction de l'espèce.
Au point actuel de nos connoissances, il paroît très probable que les diverses terres et les sels contenus dans les organes des plantes sont sortis du sol où ces plantes ont végété, et n'ont pas été formés par de nouveaux arrangemens des composans de l'air et de l'eau. Il est impossible de déterminer à présent quelles seront nos découvertes futures en chimie, et jusqu'à quel point nous simplifierons nos idées sur les principes élémentaires.\setcounter{page}{215} Nous devons raisonner d'après les faits. Nous ne pouvons imiter les facultés de composition qui appartiennent aux végétaux, mais nous pouvons les comprendre, et, autant que nos recherches peuvent nous l'apprendre, nous voyons que les formes composées résultent des formes simples. Nous voyons que les élémens du sol, l'atmosphère, et la terre sont absorbés par les plantes, et deviennent partie de leurs structures belles et variées.
Ceci nous conduit à prendre des idées plus nettes de la manière d'agir des engrais qui ne sont pas nécessairement le résultat de corps organisés, et ne sont pas composés de carbone, hydrogène, oxigène, et azote, en diverses proportions. Ces engrais doivent produire leur effet, soit en devenant partie constituante de la plante, soit en agissant sur un aliment plus nécessaire à celle-ci, de manière à le rendre propre aux usages de la vie végétale.
Les seules substances qu'on puisse appeler proprement engrais fossiles, et qui ne sont point mélangés de débris de corps organisés, sont des terres alcalines, des alkalis, ou leurs composés.
Les deux seules terres alcalines jusqu'ici\setcounter{page}{216} employées à cet usage sont la chaux et la magnésie. La potasse et la soude, les deux alkalis fixes, sont employés dans quelques-uns de leurs composés chimiques. J'établirai les faits venus à ma connoissance sur l'emploi de ces corps comme engrais; mais je m'étendrai davantage sur la chaux. Si on trouve que je donne trop de détails, je réponds que le sujet est d'une grande importance, et que des découvertes récentes y ont jeté beaucoup de jour.
La forme sous laquelle la chaux se montre le plus communément à la surface de la terre, est celle d'une combinaison avec l'acide carbonique, ou air fixe, ou acide aérien. Si l'on jette dans un acide liquide un morceau de pierre à chaux ou de craie, il se fait une effervescence. Elle est due au dégagement du gaz acide carbonique. La chaux se dissout dans la liqueur.
Si l'on chauffe fortement la pierre à chaux, le gaz acide carbonique est chassé, et il ne reste que la terre alkaline pure. Il y a perte de poids, qui va jusqu'à près de la moitié du poids total, lorsque le feu a été violent. Mais si la pierre à chaux étoit bien sèche avant l'opération, elle ne perd que trente-cinq à quarante pour cent de son poids par la calcination.\setcounter{page}{217} Nous avons vu ci-devant que l'air contient toujours du gaz acide carbonique; et que celui-ci précipite la chaux dans la solution de cette dernière par l'eau. Lorsque la chaux pure est exposée à l'air, au bout d'un certain temps elle perd sa causticité, et redevient du carbonate de chaux, c'est-à-dire, la même substance qui est précipitée de l'eau de chaux par le gaz acide carbonique. La chaux pure, c'est-à-dire, telle qu'elle est immédiatement après la calcination, a un goût caustique, et fait sur la langue un effet brûlant. Elle change en vert les couleurs végétales bleues, et elle est soluble à l'eau. Toutes ces propriétés disparoissent par l'union à l'acide carbonique: la chaux alors n'a plus de goût et n'est plus soluble à l'eau. Elle reprend sa faculté effervescente : elle est enfin la même substance chimique que la pierre à chaux et la craie.
Il y a fort peu de pierres à chaux et de craies qui soient uniquement composées de chaux et d'acide carbonique.
Le marbre des statuaires, et certains spaths rhomboïdaux, sont presque seuls dans ce cas, et les diverses propriétés des pierres à chaux, soit comme engrais, soit comme ciment, dépendent de la nature des ingrédients de cette pierre, car le vrai carbonate\setcounter{page}{218} de chaux est toujours le même dans sa composition, ses propriétés et ses effets : il est composé de quarante-un, quatre d'acide carbonique, et de cinquante-cinq de chaux. Si une pierre à chaux fait peu d'effervescence dans les acides, et qu'elle soit assez dure pour entamer le verre, elle contient de la terre siliceuse, et, probablement, de l'alumine. Si elle est d'un brun ou rouge foncé, ou fortement colorée de jaune, cette pierre contient de l'oxide de fer. Si elle n'est pas assez dure pour rayer le verre, si elle fait une effervescence lente, et blanchit l'acide auquel on la soumet, cette pierre à chaux contient de la magnésie. Enfin, si elle est noire et donne par le frottement une odeur fétide, elle contient une substance charbonneuse ou bitumineuse. L'analyse des pierres à chaux n'est point difficile. Nous avons vu sur quels principes on peut déterminer les proportions de leurs parties constituantes autant que cela est nécessaire aux opérations de l'agriculture. Avant de se former une opinion sur la manière dont les propriétés des pierres à chaux sont modifiées par les différens ingrédiens qui y entrent, il est nécessaire de considérer l'action de la chaux pure comme engrais et comme ciment.\setcounter{page}{219} Dans sa pureté, la chaux en poudre ou dissoute à l'eau, nuit à la végétation: J'ai souvent tué des graminées en les arrosant avec de l'eau de chaux. Mais le carbonate de chaux peut être employé utilement comme engrais. On trouve la terre calcaire dans les cendres de la plupart des plantes. Exposée à l'air, la chaux ne demeure pas longtemps caustique: elle s'unit promptement à l'acide carbonique.
Si l'on expose à l'air la chaux récemment calcinée, elle tombe bientôt en poudre, et on l'appelle alors chaux éteinte. On l'éteint aussi en jettant de l'eau dessus, ce qui produit momentanément une grande chaleur: l'eau disparoît.
La chaux éteinte n'est autre chose qu'une combinaison de chaux avec de l'eau. Cinquante-cinq parties de chaux en absorbent dix-sept d'eau. C'est ce que les chimistes appellent hydrate de chaux; et lorsque par une longue exposition à l'air, l'hydrate de chaux devient du carbonate de chaux, l'eau a été chassée, et le gaz acide carbonique a pris sa place.
Lorsque la chaux, soit récemment calcinée, soit éteinte, est mélangée avec une matière fibreuse végétale, humide, ces deux substances\setcounter{page}{220} agissent l'une sur l'autre avec violence, et elles forment ensemble un composé qui d'ordinaire est partiellement soluble à l'eau. Par cette opération, la chaux rend nutritive une substance qui auparavant ne l'étoit pas; et comme le charbon et l'oxigène abondent dans toutes les substances végétales, cette matière fibreuse fournit à la chaux le moyen de former du carbonate de chaux. La chaux éteinte, la pierre à chaux pulvérisée, les marnes et les craies n'ont aucune action de cette espèce sur les substances végétales. Elles préviennent une décomposition trop rapide des substances déjà dissoutes, mais elles n'ont aucune tendance à former les matières solubles. Il est donc évident que la manière d'agir de la chaux-vive dépend de principes tout-à-fait différens de ceux de l'action de la marne ou de la craie. La chaux-vive rend les matières végétales plus rapidement décomposables et solubles, de manière à ce qu'elles puissent servir d'aliment aux plantes: la craie et la marne améliorent le sol sous le rapport de sa consistance et de sa qualité absorbante: ce sont simplement des ingrédiens terreux. La chaux-vive, en perdant sa causticité, opère à la manière de la\setcounter{page}{221} eraie; mais pendant cette transformation, elle rend solubles des matières qui ne l'étoient point.
C'est des circonstances ci-dessus que dépend l'action de la chaux sur les récoltes de froment. Son efficace pour fertiliser les tourbes, et pour mettre en état de donner de bonnes récoltes les terrains qui abondent en racines, en fibres végétales sèches, et en matières inertes, en dépend également.
Pour déterminer s'il convient d'appliquer de la chaux-vive à un sol quelconque, il faut savoir s'il contient beaucoup de matière végétale inerte. Pour déterminer s'il convient d'appliquer de la marne, de la chaux éteinte, ou de la pierre à chaux pulvérisée, il faut savoir si le sol contient déjà beaucoup de matières calcaires. Tous les terrains gagnent par la chaux éteinte; et les sables gagnent plus que les glaîses.
Lorsqu'un sol, qui manque de matière calcaire, contient beaucoup d'engrais végétal soluble, l'application de la chaux-vive ne lui convient pas; car celle-ci tend, soit à décomposer les matières solubles en s'unissant à leur carbone et à leur oxygène, de manière à former du carbonate de chaux, soit à se combiner avec les matières solubles,\setcounter{page}{222} et à former des composés qui ont moins d'attraction pour l'eau que les substances végétales pures. C'est la même chose pour la plupart des engrais animaux; mais l'action de la chaux varie selon la nature des matières animales. La chaux forme avec les huiles, une espèce de savon insoluble, et les décompose ensuite graduellement, en en séparant l'oxygène et le carbone. Elle se combine aussi avec les acides animaux, et aide probablement à leur décomposition, en leur enlevant le carbone combiné avec l'oxygène, les rendant ainsi moins nutritifs pour les végétaux. La chaux tend, par les mêmes causes, à diminuer la faculté nutritive de l'albumen; et elle affaiblit toujours l'efficace des fumiers animaux soit en se combinant avec quelques-uns de leurs élémens, soit en formant avec eux des arrangemens nouveaux. La chaux ne doit jamais être appliquée avec les engrais animaux, à moins qu'ils ne soient trop riches; ou encore à moins qu'il ne s'agisse de prévenir des émanations dangereuses. Elle nuit encore par son mélange avec le fumier ordinaire, parce qu'elle tend à rendre insoluble la matière extractive. J'ai fait une expérience sur ce sujet. Je\setcounter{page}{223} mêlai une certaine quantité d'extrait soluble brun, provenant de crotins de brebis, avec cinq fois le même poids de chaux vive. Je mouillai le mélange avec de l'eau. Il se développa beaucoup de chaleur. Je le laissai reposer vingt-quatre heures. J'y joignis ensuite six ou sept fois autant d'eau en volume. L'eau, après avoir été passée au filtre, fut évaporée à siccité. La matière solide que j'obtins étoit à peine colorée : c'étoit de la chaux, mêlée d'un peu de substance saline.
Dans les cas où la fermentation est utile pour tirer la nourriture d'une substance végétale, la chaux est toujours efficace. J'essayai de mêler du tan avec un cinquième de son poids de chaux vive, et laissai ces deux matières ensemble dans un vaisseau clos pendant trois mois. La chaux s'étoit colorée, et étoit effervescente. Si on faisoit bouillir ce mélange dans l'eau, celle-ci prenoit une teinte rousse, et donnoit par l'évaporation, une poudre de même couleur, qui étoit, sans doute, de la chaux unie à la matière végétale, car cette poudre brûloit en laissant pour résidu de la chaux éteinte.
Les pierres à chaux qui contiennent de l'alumine et de la silice sont moins propres à l'engrais que les pierres à chaux pures;\setcounter{page}{224} mais la chaux qu'on en fait n'a aucune qualité nuisible. Ces pierres sont moins efficaces, seulement par la raison qu'elles contiennent moins de chaux.
J'ai parlé de pierres à chaux bitumineuses. Il y a rarement une proportion un peu considérable de matière charbonneuse dans ces pierres : elle ne va jamais à cinq pour cent du poids total. Ces pierres font d'excellente chaux. La matière charbonneuse peut, dans certains cas, devenir l'aliment des plantes.
L'application des pierres à chaux magnésiennes est d'un grand intérêt. On a reconnu depuis long-temps, dans le voisinage de Doncaster, que la chaux faite d'une certaine pierre, nuisoit à la fertilité. Mr. Tennant a trouvé que cette pierre contenoit de la magnésie. Il essaya de mêler de la magnésie calcinée, dans un sol où il sema diverses plantes. Toutes languirent ou moururent. Il en conclut que les mauvais effets de cette pierre à chaux étoient dus à la présence de la magnésie.
En faisant des expériences sur ce sujet, j'ai vérifié que, certaines pierres à chaux magnésiennes s'employoient comme engrais, avec avantage. On trouvera l'explication de ce fait, en remontant aux principes.
\setcounter{page}{225} La magnésie calcinée attire l'acide carbonique beaucoup moins fortement que la chaux : elle demeure exposée à l'air pendant plusieurs mois, sans perdre sa causticité ; et aussi long-temps qu'il reste de la chaux caustique dans la pierre à chaux magnésienne calcinée, la magnésie ne peut pas se combiner avec l'acide carbonique, parce que la chaux enlève cet acide à la magnésie.
Lorsqu'on calcine une pierre à chaux magnésienne, la magnésie perd son acide carbonique beaucoup plus vite que la chaux ; et s'il n'y a pas, dans le sol, beaucoup de substances animales, pour donner de l'acide carbonique par leur décomposition, la magnésie demeure long-temps caustique, c'est-à-dire, un véritable poison pour certains végétaux. Lorsqu'on peut employer avec avantage la pierre à chaux magnésienne, cela est dû, probablement, à ce que la décomposition du fumiet donne l'acide carbonique.
La magnésie non caustique, c'est-à-dire , saturée de cet acide, est toujours un des utiles constituans du sol. J'ai essayé de jeter du carbonate de magnésie sur de l'herbe, et sur des fromens et des orges en végétation, de manière à blanchir les plantes. Elles n'en souffrirent pas le moins du monde ; et l'un\setcounter{page}{226} des cantons de Cornouailles les plus fertiles (le Lizard) est remarquable par la quantité de carbonate de magnésie que contient son sol. Les pâturages de Lizard donnent une herbe abondante qui nourrit d'excellents moutons. Les parties soumises à la charrue passent pour les meilleures terres à blé du pays.
Ce qui montre que la théorie que j'ai essayé de donner sur la manière d'agir de la pierre à chaux magnésienne, n'est pas sans fondement, c'est le résultat d'une expérience que j'ai faite dans le but de déterminer précisément de quelle manière cette substance agit. Je pris quatre portions distinctes d'un même sol. Je mêlai au n°. 1 la vingtième partie de son poids de magnésie calcinée ; je mêlai au n°. 2, la même dose de même magnésie, et de plus une quantité de tourbe grasse en décomposition, égale au quart du poids du sol. Je mêlai le n°. 3 avec de la tourbe, sans y mettre de magnésie. Enfin je laissai le n°. 4 sans addition d'aucune substance. Ces mélanges furent faits en décembre 1806 ; et en avril 1807, je semai de l'orge dans ces espaces préparés. Les plantes prospérèrent dans le sol sans mélange ; elles furent plus vigoureuses dans le mélange\setcounter{page}{227} mélange de magnésie et de tourbe; un peu moins vigoureuses dans le sol mêlé de tourbe seulement, et très-misérables dans le numéro où j'avois mis la magnésie caustique.
En 1808, j'obtins le même résultat de la même expérience, que je répétai. Je trouvai que la magnésie qui avoit été mélangée à la tourbe étoit devenue fortement effervescente, tandis que la portion dans le sol sans mélange de tourbe ne donnoit l'acide carbonique qu'en petite quantité. Dans un des cas, la magnésie avoit aidé à former un engrais; dans l'autre cas, elle avoit agi comme un poison.
Il est évident, d'après cela, que la chaux des pierres à chaux magnésiennes peut être appliquée en grande quantité à la tourbe; et que quand on a nui à un terrain par l'abus de la pierre à chaux magnésienne, on peut réparer le mal par l'application de la tourbe.
J'ai dit que les pierres à chaux magnésiennes faisoient peu d'effervescence dans les acides. Il y a une épreuve facile pour s'assurer s'il y a de la magnésie dans les pierres à chaux, c'est d'essayer si elle blanchit l'acide nitrique étendu d'eau.
D'après l'analyse de Mr. Tennant, il paroît\setcounter{page}{228} que les pierres à chaux magnésiennes contiennent,
\comment{table}
de 20,3 à 22,5 magnésie.
29,5 à 31,7 chaux.
47,2 acide carbonique.
,8 glaise et oxide de fer.
Les pierres à chaux magnésiennes sont ordinairement colorées de brun ou de jaune pâle. On les trouve dans les comtés de Somerset, Leicester, Derby, Shrop, Durham, et York. Elles abondent dans quelques parties de l'Irlande, sur-tout à Belfort.
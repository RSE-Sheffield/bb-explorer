\setcounter{page}{373} 
\chapter{AGRICULTURE}
\section{On the Rutabaga, etc. Sur le Rutabaga ou le Turnep de Suède: tiré du Coup d'œil général sur l'agriculture du Comté d'Aberdeen, par P. Skene Keith. \large{(Extrait.)}}
Cette plante est une acquisition importante pour les fermes où la culture des turneps est adoptée; et quoique son introduction soit récente, elle ne tardera pas à se multiplier beaucoup. Elle a plusieurs avantages particuliers, que voici:
1°. Lorsque les turneps ont été détruits par la mouche ou par quelqu'autre cause, on peut remplir les vides dans le champ en y transplantant des plantes de rutabaga, si ceux-ci on été semés à la fin d'avril ou au commencement de mai.
2°. Lorsque les autres variétés de turneps ont été détruites par les gelées en hiver, le rutabaga n'est attaqué que par les lièvres; et même lorsqu'ils mangent une partie de Agricult. Vol. 19. N°. 11. Nov. 1814.\setcounter{page}{374} la bulbe, le reste résiste aux plus fortes gelées.
3°. Au printems, lorsque les autres espèces de turneps sont ou épuisées ou détériorées, non-seulement le rutabaga les remplace, mais il présente un aliment qui perfectionne la chair des animaux qu'on engraisse, bien mieux que ne le font les turneps ordinaires.
4°. Lorsque le rutabaga a porté graine, sa bulbe contient encore beaucoup de matières nutritives, car il n'a perdu qu'environ un dixième de sa pesanteur spécifique.
5°. Au commencement du printems, lorsque les feuilles des autres végétaux sont détruites ou gâtées par l'hiver, celles du rutabaga font un excellent légume pour la table.
6°. Ses bulbes, conservées dans un creux en terre, se conservent jusqu'au commencement, et quelquefois jusqu'au milieu de juin; ce qui les rend précieuses ou pour la table, ou pour les chevaux, ou pour les vaches qui allaitent.
7°. En suppléant à l'avoine par le rutabaga le fermier peut faire une grande économie sur la nourriture de ses chevaux.
La première des propriétés avantageuses du rutabaga, qui vient d'être indiquée, celle de remplir les vides des turneps qui ont mal réussi, est si importante, que tous\setcounter{page}{375} les fermiers devraient en semer une certaine quantité en avril, dans un sol bien préparé et fumé. Il peut être convenable d'ajouter que lorsqu'on arrache les jeunes plantes pour les transplanter, l'auteur de cet article a trouvé utile de plonger leurs racines dans un mélange de cendres de tourbe, et d'eau, dans le but d'assurer et d'accélérer leur reprise.
Dans les cas où le rutabaga a paru attaqué par les gelées, on a toujours observé que cet effet provenait, ou d'un mauvais choix de graines, ou de ce que le sol qui les renfermait n'avait pas été convenablement desséché. Aucune plante ne peut résister à-la-fois à la gelée et à l'humidité, qui donne lieu à la formation de la glace dans le tissu végétal.
La troisième des qualités énumérées tout à l'heure est d'une importance telle, que tout fermier qui élève beaucoup de bestiaux devrait toujours cultiver le rutabaga en proportion considérable.
Lorsqu'un bœuf commence à s'engraisser il mange avec voracité, et dans cette période les turneps conviennent fort. Mais quand il est à peu près gras, son estomac affadi exige un aliment plus substantiel.
La pesanteur spécifique du rutabaga est\setcounter{page}{376} -décembre étant 1,035, elle n'étoit plus que 0,940 au mois de juin suivant, c'est-à-dire égale à celle du turnep jaune d'Ecosse, et supérieure à celle de tous les autres turneps dans leur meilleure condition. C'est au mois d'avril qu'une récolte de rutabaga est la plus dense, et que sa valeur est la plus grande.
Lorsqu'on enterre le rutabaga dans des creux, il se conserve frais et mangeable jusqu'au commencement de juin; mais il faut laisser pendant quelques jours des issues à l'air dans la terre dont on les a recouverts.
Il faut ajouter à tous ces avantages du rutabaga celui de réussir dans des terres bien plus fortes que celles qui conviennent exclusivement aux turneps.
L'auteur plaidant à charge et à décharge, passe aux objections.
On dit d'abord qu'il est très-difficile d'obtenir de la graine sûre. Ensuite, que cette plante exige un très-bon terrain fort amendé; enfin que la récolte est plus légère que celle de toute autre espèce de turneps.
L'auteur avoue que la première objection est fondée; et il attribue la difficulté de se procurer de la bonne graine à l'influence qu'ont toutes les plantes de la famille des brassica, à l'époque de la floraison, sur\setcounter{page}{377} L'imprégnation du rutabaga. On ne peut s'y soustraire qu'en cultivant bien à part, et à distance de toute autre plante de cette famille, le rutabaga qu'on destine à porter graine.
La seconde objection n'est pas moins juste, mais elle s'applique à bien d'autres cultures.
Il est vrai qu'on ne recueille pas sur une surface donnée, un aussi grand nombre de tonnes de rutabaga que de turneps; mais, dit l'auteur, une tonne de rutabaga contient autant de matière nutritive que deux des turneps communs blancs, verts, ou rouges; et au moins autant qu'une tonne et demie des meilleurs turneps jaunes.
L'auteur du Rapport a distillé le rutabaga pour en tirer de l'eau-de-vie. Il a été satisfait de la quantité; mais le goût n'en étoit pas agréable. Cependant au bout d'une année de garde, ce qu'il y avoit de désagréable dans la saveur se dissipe.
Voici une dernière qualité du rutabaga. On sait que lorsque les turneps grossissent trop, ils deviennent mous et spongieux, et qu'alors les gelées leur sont très-préjudiciables, mais le rutabaga, même lorsqu'il va jusqu'à huit ou dix livres, poids qu'il ne surpasse guère, demeure assez compacte.
G g 3\setcounter{page}{378} pour résister aux plus fortes gelées.
Le sol qui leur convient le mieux est certainement une terre légère; mais comme on l'a dit, ils peuvent réussir dans une terre forte, bien égouttée et cultivée à la houe à cheval.
Les labours doivent être aussi complets qu'il est possible. On rompt avant l'hiver; on relaboure au printems, de bonne heure; on donne une troisième façon, avant d'étendre le fumier; c'est alors qu'on forme les billons; et enfin une quatrième, immédiatement avant de semer en ligne, lorsqu'on a intention de cultiver à la houe à cheval.
Lorsqu'on n'a besoin que d'un carré de rutabaga destiné à remplir les vides d'une culture de turneps, on se borne à le semier à la volée.
On doit fumer largement. Au moins vingt-quatre tonnes par acre. Il y a cette différence entre le rutabaga et le turnep, que celui-ci peut encore réussir passablement sur une bonne terre quoique non-fumée; le rutabaga n'est point dans ce cas, on n'en obtient rien de bon sans fumier.
Le meilleur moment pour semer cette plante, dans le climat d'Aberdeen, est la fin d'avril, ou le commencement de mai. Si l'on renvoie au mois de juin, époque ordinaire\setcounter{page}{379} de la semaille des turneps, on n'aura qu'une demi récolte. Ici l'auteur cite une anecdote qui prouve ce résultat par sa propre expérience, et contre l'opinion de sir Thomas Beavor qui, dans ses Communications à la société de Bath, dit avoir semé avec succès des rutabagas en juin. Il est sur-tout convenable de les semer de bonne heure lorsqu'ils doivent être transplantés, opération qui retarde toujours jusqu'à un certain point le développement.
On reconnaît deux variétés de rutabaga; le blanc et le jaune. Un individu de chacune de ces variétés ayant été éprouvé quant à la pesanteur spécifique, elle fut trouvée de 1,022 pour le blanc, et de 1,035 pour le jaune. Mais il n'est pas bien certain que cet avantage de densité du jaune sur le blanc ne fût pas plutôt une différence entre deux individus, que le caractère d'une variété comparée à une autre.
On ne doit transplanter le rutabaga que lorsqu'on a des vides à remplir dans un champ de turneps; la récolte qu'on obtient sans transplantation est toujours plus abondante que celle qu'on a transplantée.
Quant au mode de culture, celle en ligne avec la houe à cheval ( le cultivateur ) est certainement la meilleure, et celle qui déa\setcounter{page}{380} barrasse le mieux le terrain des plantes étrangères. Le fermier ne doit jamais semer le rutabaga à la volée, sauf dans le cas particulier mentionné tout-à-l'heure.
On peut aussi sarcler à la main entre les plantes qui sont en ligne; et comme on sème le rutabaga plus tôt que les turneps, on a le temps de donner trois façons; on en retrouve largement les frais dans la récolte.
En semant le rutabaga plus tôt que les turneps, on a plus de chance d'éviter les attaques de la mouche. Mais, comme cette plante doit être très-fumée, et que c'est dans un sol ainsi préparé que cet insecte fait ordinairement le plus de ravage, dès qu'on l'aperçoit, ou qu'on a lieu de soupçonner qu'il existe, il ne faut pas hésiter à faire passer le rouleau de minuit; opération qui ne fait point de tort à la plante lorsqu'elle est exécutée convenablement et à temps.
Quant aux divers emplois dont cette racine est susceptible, on l'a employée avec succès pour nourrir les chevaux, dans le voisinage d'Aberdeen; on la leur coupe par tranche. On la donne aussi avec avantage aux vaches à lait et aux cochons, qui se jettent sur le rutabaga de préférence à tous les autres turneps. Les grands propriétaires\setcounter{page}{381} en font grand cas pour les vaches à lait au printems; mais c'est sur-tout pour les chevaux et pour finir l'engraissage des bêtes à cornes que cette racine est profitable.
Pour établir la quantité absolue de produit de rutabaga par acre, l'auteur du rapport a fait trois essais en grand.
\comment{table}
Tonnes. Quint.
Dans le premier, sol fertile et bien fumé, la récolte d'un acre écossais a pesé. . . . . 38 10 ½
Dans le second. . . . . . . . . . . . 37 16 ¼
Dans le troisième bon terrain mais semé un peu plus tard. 27 12
L'auteur ne croit pas que le produit moyen en le prenant sur tout le Comté, dépasse 20 tonnes (400 quintaux) par acre.
Ici l'auteur cite une récolte extraordinaire de ce genre, obtenue par Mr. James Gordon dans une bonne terre bien fumée et bien nettoyée. Il ne l'a ni vue ni mesurée. Mais elle a été l'objet de la curiosité d'un nombre d'amateurs qui sont venus de loin pour l'admirer. Il n'élève aucun doute sur la véracité du propriétaire, dont voici les expressions:
" Ma récolte de rutabaga a été plus considérable cette année (1798) que dans aucune des précédentes. J'avais semé le 26\setcounter{page}{382} mai; et le 15 mai suivant, le produit d'un billon pesa sept cent quatre-vingt onze livres quatorze onces avoir du poids; ce qui donnoit par acre d'Ecosse cinquante-six tonnes onze quintaux et un quart. Quelques-uns de ces turneps pesoient dix-sept livres avec leurs feuilles; et un grand nombre, seize livres. On n'en fit usage que bien avant dans le printems, et malgré la sévérité extraordinaire de la saison (1799) pas une seule plante de rutabaga n'en souffrit. Le sol qui produisit cette récolte étoit une terre assez forte et plutôt humide; ce qui montre la grande valeur de cette plante. "
Il faut remarquer encore, que le poids mentionné ( cinquante-six tonnes onze quintaux et un quart par acre d'Ecosse ) auroit donné quarante-une tonnes, dix quintaux, vingt-cinq livres, par acre anglais; et qu'au 25 mai 1799, la récolte alors chargée de feuilles, pesoit plusieurs quintaux de plus qu'en janvier avant qu'elles eussent commencé à pousser. Cette récolte surpasse d'un tiers celle du duc de Bedford à Woburn en 1798, savoir, trente-trois tonnes, dix quintaux et demi par acre.
La valeur du rutabaga est plus relative qu'absolue. Par exemple, lorsqu'on l'emploie à nourrir les chevaux, elle vaut ce qu'elle\setcounter{page}{383} épargne d’avoine ou d’autre aliment qu’auraient consommé ces animaux. Ainsi, en supposant qu’un boisseau d’avoine pèse trente-huit livres et demie, et qu’elle coûte 33 shillings le quarter de huit bushels; et admettant, d’autre part, qu’un cheval est mis à un travail assez fort pour exiger quatorze livres d’avoine par jour, mais qu’on ne lui en donne réellement que sept, en substituant aux autres sept livres, vingt-sept livres de rutabaga par jour, on épargnera exactement un quarter en quarante-quatre jours, c’est-à-dire, une valeur de 33 shillings; soit 9 pence par jour, qui établiront la valeur de vingt-sept livres de rutabaga, soit un penny les trois livres; et quel que soit leur prix nominal, dix-huit livres vaudront 6 pence. Et si dix-huit tonnes par acre bien cultivé sont le produit moyen en rutabaga, la valeur de cette récolte, telle qu’elle résulte de l’économie faite sur l’avoine, est de 56 liv. sterl. par acre. Si d’autre part, on emploie le rutabaga à achever l’engraissage pour le boucher, il reçoit sa valeur de deux élémens, le poids de viande et de suif qu’il ajoute à l’animal, et l’amélioration qu’il produit dans la qualité de cette viande. Enfin, si on le donne aux vaches à lait, sa valeur résulte de la quantité additionnelle et de la meilleure\setcounter{page}{384} qualité du lait et de la nourriture plus substantielle fournie à la vache. Ces deux derniers élémens ne sont pas susceptibles d'appréciation mathématique ; mais un boucher sait qu'il y a plus de profit à achever d'engraisser un animal commencé, qu'à le nourrir pendant le même temps avec un aliment dont il est dégoûté. Et les fermiers connoissent aussi très-bien ce qu'il y a à gagner en quantité et en qualité de lait, à employer le rutabaga de préférence aux turneps et à toute la famille des brassica.
Serrant de plus près la comparaison avec le turnep, l'auteur dit que c'est prodiguer inconsidérément le rutabaga, que le donner d'entrée à un bœuf affamé et vorace, lorsqu'on commence à l'engraisser ; les turneps ordinaires font merveilles à cette période du procédé ; on peut donner ensuite les meilleurs turneps jaunes ; et lorsqu'on les a donnés jusqu'à l'époque où ils commencent à pousser, chaque livre de rutabaga, dans les mois de printems, en vaut trois de turneps ordinaires. Un acre des premiers vaudra sur le marché, au moins le double d'un acre des derniers.
Il n'y a pas actuellement dans le comté d'Aberdeen plus de cent acres de cette précieuse racine en culture. C'est l'intime per-\setcounter{page}{385} suasion de sa valeur, qui a engagé le Rapporteur à en parler avec autant d'étendue. Chaque fermier devroit au moins en cultiver un acre, pour six, et même quatre, de turneps. Si, comme il n'y a guères lieu d'en douter, la pomme de terre est la racine la plus précieuse ; la première après, est le rutabaga ; viennent ensuite les carottes ; puis les différentes variétés de turneps. Voici leurs pesanteurs spécifiques relatives.
\comment{table}
Pes. spéc.
Pommes de terre . . . . 1,091
Rutabaga . . . . . . . . 1,032
Carottes . . . . . . . . 1,018
Turneps . . . . . . . . de 0,840 à 0,940
Ce rapport des pesanteurs spécifiques indique à peu près l'ordre des valeurs comparatives, mais non pas leur rapport exact ; car la valeur réelle de ces racines, déduite de la quantité d'aliment profitable qu'elle fournit, ou de la matière saccharine et fermentescible contenue dans un poids donné de chacune de ces racines, ne suit pas le rapport des pesanteurs spécifiques. Ces qualités croissent plutôt comme les cubes ou les troisièmes puissances de ces pesanteurs, à peu près, dans l'opinion du Rapporteur ; opinion, qu'il avance toutefois modestement et en avouant qu'elle n'est pas susceptible\setcounter{page}{386} de démonstration mathématique, quoique l'expression en soit empruntée aux sciences exactes.

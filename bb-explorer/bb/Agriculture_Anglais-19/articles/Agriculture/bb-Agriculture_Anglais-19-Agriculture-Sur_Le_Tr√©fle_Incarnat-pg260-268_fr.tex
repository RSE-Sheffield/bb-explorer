\setcounter{page}{260}
\chapter{Agriculture}
\section{SUR LE TRÈFLE INCARNAT, par CH. PICTET.}
LE 27 juillet 1813, je fis commencer la semaille d'une pièce de dix-huit poses (de vingt-cinq mille six cent pieds de France) en trèfle farouch ou incarnat, mélangé de millet. Je devois le conseil de ce mélange à Mr. de Loïs, qui depuis quelques années s'en trouve très-bien.
La pièce avoit fourni trois récoltes depuis la fumure. Elle venoit de donner du froment, lequel succédoit à un trèfle. Ma graine de trèfle incarnat avoit été tirée de deux endroits différens. La première semée étoit bonne, et la dernière très-médiocre. J'ai soupçonné que celle-ci avoit été échauffée\setcounter{page}{261} au four pour la dégager des gousses. La terre étoit encore fraîche quand la semaille commença; mais celle-ci prit huit jours, et comme le terrain étoit léger, et la chaleur très-forte, la levée fut de plus en plus inégale et foible, à mesure que l'on avança. La pluie tarda long-temps, et il ne tomba que dix lignes d'eau jusqu'au 2 septembre. Dans la portion où la levée du trèfle et du millet avoit été tardive, les plantes prirent peu de développement. J'y mis les moutons au mois d'octobre, et ne fis faucher que les six poses qui avoient été les premières semées: elles donnèrent environ deux milliers pesant, de millet séché pour fourrage, et la valeur de quatre milliers secs, en fourrage vert, donné aux bêtes en octobre et novembre. La faulx coupa deux ou trois pouces du trèfle: il repoussa en novembre, mais je n'essayai pas d'y mettre les moutons, de peur de lui nuire.
L'hiver a été rude; et le printems surtout a été fatal aux luzernes et aux trèfles ordinaires. Le trèfle incarnat n'en a pas souffert. Au mois d'avril, je n'étois pas présent. Mon maître-valet, après avoir fait pâturer les moutons dans toute la partie foible du champ, le fit rompre pour y mettre des pommes de terre. On ne conserva pour graine\setcounter{page}{262} qu'un espace de quatre-vingt sept mille pieds (ou trois poses et demie) où le trèfle paroissoit bien garni: il y avoit cependant de mauvaises places. Le 3 juillet, je l'ai fait couper. J'avois laissé un peu trop mûrir la graine; car elle s'égrenoit prodigieusement à chaque coup de faulx\footnote{Les enveloppes tombent avec les semences qu'elles contiennent.}. Le 6 juillet, je serrai cinq voitures d'environ vingt-cinq quintaux de trèfle sec, et je le fis battre dans les jours suivans, en le faisant passer sous la meule tournante, entre le battage et le vanage. J'ai obtenu douze cent soixante-une livres, poids de marc, de graine bien sèche. Voyant le champ couvert de graine, j'eus l'idée de l'enterrer à la charrue, en y ajoutant du blé sarrasin, pour protéger la jeune plante, et m'assurer une récolte dérobée. Aujourd'hui (30 juillet) le blé sarrasin a deux pouces de haut, et le trèfle couvre la terre. Il y en a au moins quatre fois ce qui auroit été nécessaire, et il périra une grande partie des plantes.
Outre cela, les chariots qui portoient la récolte ont semé des deux côtés du chemin, dans un espace de dix minutes de marche, qui est la distance du champ à la maison,\setcounter{page}{263} une si prodigieuse quantité de graine, que les deux côtés sont maintenant garnis d'un trèfle épais, qui a pris, même dans le gazon sous les haies, parce que sa levée a été favorisée par les pluies.
Enfin le battage de ce trèfle étant difficile, même après l'action de la meule tournante, les ouvriers se sont lassés de faire repasser les mêmes gousses sous le fléau. Ils ont laissé beaucoup de graine dans la poussière. J'ai fait semer celle-ci sur mes prés, et le trèfle est maintenant levé fort épais dans l'herbe. Mes gens ont estimé qu'il étoit resté au moins un quintal de semence dans la poussière. Il s'en est égrené peut-être deux quintaux sur le champ, et autant le long du chemin.
Le trèfle farouch, quoiqu'ayant porté sa graine à maturité, conserve ses feuilles et donne encore un bon fourrage. Mes moutons le mangent avec avidité, et n'en laissent absolument point. J'ai essayé d'en donner aux chevaux, qui l'ont très-bien mangé, quoiqu'ils fussent à la luzerne en vert.
Mr. de Père a beaucoup vanté ce trèfle incarnat, qu'il a tiré des Pyrénées; et il assure, dans son Manuel, que cette plante donne autant dans sa seule coupe (elle n'en donne qu'une) que le trèfle ordinaire dans les deux récoltes. Je ne puis pas en juger,\setcounter{page}{264} parce que je ne l'ai pas vu dans sa pleine fleur, qui est le moment de sa plus grande production; mais mes gens m'ont dit qu'il auroit pu se faucher huit jours avant la luzerne. Il avoit été plâtré, ce qui paroît lui être aussi profitable qu'au trèfle ordinaire.
L'avantage de pouvoir être intercalé entre une récolte de froment, et une de pommes de terre, est très-grand; celui de s'associer très-bien à une autre récolte dérobée, comme blé-sarrasin, millet-fourrage, maïs-fourrage, ou avoine en vert, doit aussi être apprécié. C'est une manière de s'assurer du fourrage pour l'automne, en quantité aussi considérable que le terrain dont on dispose puisse le permettre. On couvre ainsi les frais de labourage, plâtrage, et récolte du trèfle incarnat, ensorte que la production de celui-ci est tout gain. Essayons le calcul des frais et du produit de l'espace de trois poses et demie dont il s'agit.
\comment{table}
Labourage, à 9 fr. par pose . . . . . . . Fr. 31. 50
Hersage et roulage, à 1 fr. 50 . . . . . . 5. 25
Semence, à 12 livres de marc par pose, et
20 s. la livre . . . . . . . . . . . . . . . 42. —
Semaille . . . . . . . . . . . . . . . . . . . 3. —
Semence du millet, à 24 livres par pose . . 21. —
Plâtrage . . . . . . . . . . . . . . . . . . . 9. —
Frais de récolte . . . . . . . . . . . . . . . 30. —
Frais de battage, 21 journées à 2 francs . . 42. —
Total des frais Fr. 183. 75
\setcounter{page}{265}
\subsection{Produit.}
Six milliers pesant de fourrage de millet à 3 fr. le quintal . . . . . . . . . Fr. 180. —
125 quintaux de fourrage de trèfle, qui a donné sa graine, à 2 fr. . . . . . . . . . 250. —
13 quintaux de graine de trèfle, à 75 fr. 975.\footnote{Le prix de 75 francs les 60 kilogrammes est celui que Mr. Villemorin a payé cette graine en 1813. Ce prix qu'on peut regarder maintenant comme le prix du commerce, baisseroit sans doute si on en surchargeait le marché ; mais comme ce trèfle est encore fort peu connu, il est probable que la demande de la graine ira long-temps en croissant, et que pendant plusieurs années, ce sera une très-bonne spéculation agricole, que de le laisser mûrir. Cependant son produit ne doit être calculé que sur le fourrage coupé en pleine fleur.} —
Total du produit brut. Fr. 1405. —
A déduire les frais ci-dessus. 183. 75
Produit net. Fr. 1221. 25
Soit par pose fr. 349.
J'avois déjà essayé deux fois la culture du trèfle incarnat sans réussir. La levée avoit manqué, soit parce que la graine étoit trop vieille, soit peut-être parce qu'elle avoit été échauffée au four. Les marchands grénetiers doivent exiger de ceux qui la fournissent une grande exactitude sur ce point ; et la\setcounter{page}{266} difficulté d'avoir une garantie sûre tend à décourager de la culture de cette plante. Il paroît aussi qu'il importe de semer ce trèfle en temps humide et à raies fraîches, c'est-à-dire, à mesure que la charrue fait le travail.
J'en fis un autre essai l'année dernière, sur deux poses, dans une terre argileuse, et avec du millet. Cet essai a complétement manqué. La levée des deux graines fut foible et lente. Le millet donna très-peu de pâturage dans l'arrière-automne, et le peu de trèfle qui restoit a été tué par l'hiver. Il paroîtroit de là, 1° que les terres légères lui sont plus favorables que les terres argileuses, 2° que le choix de la graine est très-important, 3° qu'il faut des circonstances favorables pour que la levée soit belle et que les premiers progrès soient prompts; 4° que par un temps pluvieux, ce trèfle semé dans les prés, peut y prendre pied, ce qui paroît pouvoir être avantageux pour augmenter la force de la première coupe.
J'ai actuellement dix-sept poses ensemencées en trèfle incarnat, avec du millet, de l'avoine, du maïs, ou des vesces. Sur cette étendue, il n'y a que trois poses en terre argileuse. Comme ce sont les dernières semées, la graine n'a pas levé encore; mais\setcounter{page}{267} dans tout le reste la levée est très-belle, et par conséquent avancée d'environ trois semaines comparativement à l'année dernière. Sur 13 ½ poses, le trèfle succède au froment non fumé ; et sur cet espace dix poses sont à la deuxième année de mon assolement, c'est-à-dire, qu'elles ont donné deux récoltes consécutives de froment\footnote{Voyez le tableau de l'assolement de douze ans, dans le 15ᵉ. volume Agriculture, année 1810.}. Mon intention est d'essayer l'effet du gypse sur quelques parties, en le répandant sur la plante encore très-foible. M. de Loys s'en est bien trouvé pour le trèfle ordinaire.
Je me propose d'essayer de semer le trèfle incarnat sur le froment, au printems, et sur-tout sur le seigle. Une fois que cette plante a fleuri, elle ne repousse plus ; mais je pense que semée en avril, elle ne sera pas encore en fleur au moment de la moisson, et qu'alors elle repoussera vigoureusement, avec l'aide du plâtrage, pour donner une pleine récolte en septembre.
De quelle manière qu'on emploie ce trèfle, il promet des avantages importans ; et on doit s'étonner qu'après ce que Mr. de Père en a dit dans son Manuel, il soit encore si peu répandu en France. La connoissance de\setcounter{page}{268} La pratique acquise par Mr. de Loys, sur cette plante, seroit très-utile au public. Il est probable que quelques-uns de ceux qui ont essayé ce trèfle, ont éprouvé le même mécompte que moi, par la qualité de la graine, ou peut-être par les circonstances de la semaille. Chacun peut recueillir chez soi la quantité de semence nécessaire à son exploitation. On voit que la production de la quantité de graine est prodigieuse, et que par conséquent, il suffit d'un petit espace pour en avoir beaucoup.
Le millet a le même avantage, de donner beaucoup de graine, laquelle étend fort loin, quand on la sème, parce qu'elle est très-menue. C'est, sous ce rapport, la plante la plus économique pour fourrage vert. Elle en donne abondamment, et de bonne qualité, en même temps qu'elle protège efficacement le jeune trèfle. Enfin, par l'association de ces deux plantes, on obtient deux récoltes dérobées par un seul labour, et sans engrais. L'ombre épaisse du trèfle tue les mauvaises herbes; et on est à temps, au printems pour une récolte sarclée, lorsqu'on a coupé le trèfle pour fourrage sec.

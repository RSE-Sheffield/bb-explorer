\setcounter{page}{37}
\chapter{AGRICULTURE.}
\section{PRINCIPES RAISONNÉS D'AGRICULTURE. Traduit de l'allemand d'A. THAER, par E. V. B. CRUD. Tom. Ier. in-4°, 372 p. Genève, 1811, chez J. J. Paschoud, Imprimeur-Libraire; et à Paris chez le même rue Mazarine, n°. 22. \large{(Vingt-deuxième extrait. Voy. p. 296 du vol. préc.)}}
C'EST un usage vicieux que de laisser le fumier long-temps sur le sol, en petits tas tels qu'on les fait en déchargeant les charriots. S'il n'a pas encore subi la fermentation, il se décompose avec grande perte, parce que le vent entraîne les substances volatiles qui s'évaporent; d'ailleurs, cette décomposition se fait d'une manière fort inégale: au centre du tas elle est très-forte, et sur les bords, presque nulle. Les sucs les plus actifs sont entraînés par l'humidité dans le sol, au-dessous du tas, tandis que la partie du fumier qui est moins riche ou moins décomposée, demeure sur place. De cette manière, lors même qu'on donne ensuite les\setcounter{page}{38} plus grands soins à bien répandre la partie qui reste sur le sol, les places où les tas ont été déposés, demeurent quelquefois pendant plusieurs années trop grasses, ensorte que les plantes versent, quoique tout ce qui les environne ait la plus chétive apparence. Il faut donc avoir pour règle invariable de répandre le fumier immédiatement après qu'il a été déposé sur le champ.
L'époque où les fumiers doivent être charriés varie beaucoup, elle dépend des circonstances particulières, et de l'ordonnance de l'économie. Dans les exploitations soumises à l'assolement triennal avec jachère, ainsi que dans celles qui suivent des assolemens avec pâturage, les transports se font ordinairement entre les semailles de printems et la moisson. Il faut avoir soin d'employer le fumier frais dans les champs froids et humides, et le fumier consommé, dans les terrains les plus secs et les plus chauds. Au reste, il n'est pas toujours facile d'arranger les choses de manière que cette disposition puisse avoir lieu. Les exploitations rurales qui peuvent consacrer leur fumier à des produits variés, jouissent de cet avantage, si leurs places à fumier sont bien disposées ; elles peuvent charrier et employer leurs engrais dans l'état qui est le plus avantageux\setcounter{page}{39} au sol, relativement au genre de produit qu'on en exige. Le fumier du commencement ou de la fin de l'hiver, doit être consacré principalement aux récoltes sarclées. Le fumier récent et pailleux convient surtout aux pommes de terre, plantées en terrain glaiseux, non-seulement parce qu'il diminue la ténacité des sols de ce genre, laquelle peut facilement nuire à cette plante lors de sa germination, mais encore parce qu'il met la pomme de terre plantée en communication avec l'atmosphère. Les autres récoltes-racines, et les choux en particulier, se trouvent mieux du fumier consommé; et sur un terrain léger, cet état du fumier est une condition nécessaire de leur réussite. Ensuite on charie le fumier pour les pois et les vesces, et on l'enterre, ou bien l'on se borne à le répandre par dessus la terre. Celui qui a été fait le plus récemment se décompose sans peine, parce que la température est alors plus élevée; on le consacre aux récoltes sarclées plus tardives, surtout au colza. Une partie du fumier recueilli depuis le milieu de l'été, peut encore être consacré aux céréales d'automne. Quoique dans ce genre de culture, on ne fume jamais complétement les terres; quelquefois cependant, on juge convenable de lui donner\setcounter{page}{40} un supplément d’engrais. D’autres fois, on transporte cette partie des fumiers sur le chaume des champs destinés aux récoltes jachères et aux légumes du printemps suivant.
De cette manière, les chariots destinés au transport des engrais ne sont jamais sans emploi, parce qu’il y a toujours des fumiers prêts à être chariés; et comme les travaux d’attelage sont répartis d’une manière égale sur les diverses saisons, on a toujours le temps nécessaire pour faire ces transports.
Les opinions ne sont pas d’accord sur celui des labours après lequel on doit charier les fumiers, lorsqu’on emploie ceux-ci sur la jachère. Le plus grand nombre des cultivateurs arrangent les choses de manière que le fumier soit enterré à l’avant dernier labour, tandis que d’autres craindroient, en faisant ainsi, qu’une partie de ce fumier ne fût ramenée à la superficie par la dernière culture. Quoiqu’on ne doive nullement redouter que ce fumier soit exposé à l’air, et qu’il s’évapore, il est sûrement plus avantageux qu’il reçoive trois labours avant les semailles. Ainsi donc, toutes les fois que cela est possible, il faut le charier de manière à l’enterrer déjà par le premier labour. Si ce labour n’a lieu que vers le milieu de\setcounter{page}{41} l'été, la méthode de l'enterrer au dernier labour devient décidément mauvaise, et c'est peut-être dans ce cas, une des principales causes du non-succès des céréales: en effet, de cette manière, le fumier ne peut jamais être mêlé avec le sol; il demeure en gros morceaux; il s'échauffe trop dans quelques places, tandis que dans d'autres, il ne peut pas se décomposer, en sorte que plusieurs années après, on le retrouve en forme de tourbe, et presque tel qu'on l'avoit enterré. De là vient, qu'il y a une grande inégalité dans la force des plantes, et que quelques-unes forment de grosses touffes, dans lesquelles il se niche une quantité d'insectes et de souris, tandis que d'autres, au contraire, sont d'une foiblesse extrême. Alors les plantes qui avoient poussé trop vigoureusement périssent pendant l'hiver. Il résulte souvent les plus mauvais effets, d'avoir enterré du fumier récent et non-décomposé au dernier des labours destinés aux semailles d'automne. Il se fait alors des vides dans le terrain. Si la température devient humide et chaude, que la semaille ait été faite de bonne heure, et que le fumier ait été ainsi mis en fermentation, les plantes peuvent souvent prendre une croissance trop rapide, mais elles sont d'une complexion foible: probablement elles\setcounter{page}{42} sont trop chargées d'hydrogène. Dans cet état elles ne supportent pas l'hiver, elles pourrissent et meurent. Si ce fumier récent et non divisé, n'entre pas en fermentation, il arrive souvent qu'au printemps, lorsque la sécheresse et la chaleur surviennent, il occasionne la destruction des plantes: elles commencent par perdre leur couleur, et ne tardent pas à périr. Ces faits sont constatés par l'expérience, et si quelquefois, lorsque les circonstances sont très-favorables, ces inconvéniens ne se font pas sentir, ce sont des exceptions assez rares.
Il y a des agriculteurs qui sont prévenus contre la méthode d'enterrer le fumier avant le pénultième labour, croyant qu'il perd ainsi ses sucs, et favorise la végétation des mauvaises herbes. Mais cette abondante germination de mauvaises herbes, loin d'être nuisible, est, au contraire, très-avantageuse, parce que les semences et les racines sont d'autant mieux détruites par ce moyen, et que les nouvelles plantes enterrées avec la charrue, augmentent la fécondité du fumier et du sol. Il suffit de bien observer ce fait, pour abandonner un préjugé que les cultivateurs se sont communiqué les uns aux autres, et qui a été reçu sans examen.
La bonne distribution des engrais est d'une\setcounter{page}{43} si grande importance, qu'elle demande l'attention la plus soutenue. Il faut se garder de fumer trop abondamment, soit sous raies, soit sur le sol, après la semaille; l'excès peut nuire aux récoltes, sur-tout aux céréales, en les faisant verser. Il est souvent arrivé qu'en voulant se procurer une riche moisson, on n'obtenoit qu'une chétive récolte. Il est un maximum d'amendement, duquel on doit approcher pour obtenir de bons produits, mais que l'on ne doit pas excéder pour ne pas s'exposer à une grande perte. On ne peut pas en fixer très-positivement le degré, il varie selon la nature du sol. Un terrain argileux et humide supporte et demande un amendement plus fort qu'un terrain sablonneux, calcaire et chaud. Mais les différences dans la température ont, sur les succès des récoltes, une influence telle, que lorsque cette température est très-favorable à la végétation, la quantité de fumier, qui, en temps ordinaire, eût été la plus convenable, pourroit rendre les céréales trop abondantes, et détériorer beaucoup la récolte. Aussi remarque-t-on que dans les années où la moisson a été fort abondante, la récolte des domaines dont le sol est très-riche, diffère moins de celui dont le sol est appauvri, que cela n'a lieu dans les années ordinaires\setcounter{page}{44} ou mauvaises. Lors donc qu'on fume immédiatement pour les grains, il convient de retrancher quelque chose du maximum d'engrais qu'on croiroit pouvoir donner au sol.
Dans les exploitations rurales qui ont à leur disposition une abondante quantité d'engrais, le meilleur moyen d'échapper au danger d'un amendement trop riche, c'est de ne pas fumer immédiatement pour le blé; mais plutôt pour les récoltes auxquelles une végétation très-forte ne peut jamais être nuisible. Les choux, la plupart des récoltes racines (les pommes de terre exceptées) les fèves en lignes, le maïs, le colza, les vesces à faucher en vert, ne peuvent jamais recevoir trop d'engrais. Ces produits absorbent une assez grande partie des sucs du fumier, pour que les grains qui leur succèdent ne soient pas en souffrance. Le fumier se trouve alors avoir perdu, non-seulement une partie de sa chaleur et de son activité, mais encore son excès d'hydrogène et d'azote, quoiqu'il conserve à-peu-près tout son carbone.
Il arrive bien plus souvent qu'on doit porter son attention sur le défaut contraire, c'est-à-dire, veiller à ce que les champs reçoivent la quantité de fumier dont ils ont besoin. Si l'on ne peut pourvoir à la totalité des terres à fumer, on doit donner d'abord l'engrais\setcounter{page}{45} nécessaire aux champs sur lesquels on peut le mieux compter pour la récolte de grains et de paille, lors même que les terrains moins essentiels devroient demeurer sans engrais, mais il faut se défendre de porter ce principe à l'extrême, comme on le fait fréquemment, en consacrant aux champs les plus fertiles une surabondance de fumier, et en refusant tout aux autres. Dans ce cas, la privation absolue d'engrais rend les terres stériles tellement mauvaises, que la détérioration occasionnée par l'emploi du fumier sur de bons terrains ne sauroit être compensée par l'excédent de produit que donne ce dernier. Ainsi donc, le cultivateur qui desire conserver la totalité de son fonds dans un état prospère, et qui ne borne pas ses vues au présent, se gardera bien de n'égliger ses mauvais terrains pour s'occuper exclusivement des meilleurs.
Lorsqu'on voudra rétablir en fonds ruiné, on sera peut-être obligé, au premier abord, de retrancher aux champs les moins épuisés un peu du fumier, qui sans cela leur eût été consacré, et de le réserver aux terrains de ces premiers champs; par conséquent, le produit total des récoltes sera diminué. L'on doit se préparer à cet inconvénient, çar il arrive souvent que le premier amendement\setcounter{page}{46} donné à un terrain épuisé, ne produit que peu d'effet. Dans la disposition de ces choses de ce genre, il faut autant de réflexion que de capacité, pour garder un juste milieu, et ne pas perdre de vue les moyens d'avoir la paille dont on a besoin, lors même qu'on prendroit son parti de n'obtenir qu'une moindre quantité de grains.
En supposant même qu'on ait une quantité de fumier suffisante pour en donner à chaque champ la juste mesure dont il a besoin, il faut, dans la distribution de ce fumier, faire attention à la nature particulière de chaque terrain. Si l'on veut entretenir dans un état égal de prospérité les sols argileux et tenaces, et ceux qui sont légers, sablonneux et calcaires, il faut donner aux premiers, une plus grand quantité d'engrais à-la-fois, parce qu'ils peuvent la supporter sans que les récoltes risquent de verser, et qu'une petite quantité, au lieu de produire quelqu'effet sur ces sortes de terres, seroit retardée dans sa fermentation, et resteroit dans le même état où elle auroit été enterrée; en revanche, lorsque ces terrains ont été fumés abondamment, ils peuvent rapporter un nombre de récoltes double, sans être épuisés. Dans un terrain léger et chaud, le fumier est promptement décomposé;\setcounter{page}{47} un amendement très-abondant peut y avoir des suites fâcheuses, parce que suivant l'état de la température, il peut ou faire verser les céréales ou les brûler: le fumier y est plus promptement consumé, c'est pourquoi il faut y en mettre plus souvent et en plus petite quantité. Plus un terrain est léger est sablonneux et plus cette attention est nécessaire. En général cependant, il paroît avéré que dans un même nombre d'années, les deux espèces de terrains opposées, demandent autant d'engrais l'une que l'autre.
La quantité de fumier est ordinairement calculée par charges ou chariots à deux ou quatre chevaux; quelquefois aussi par charretées d'un cheval. L'on a dit ailleurs que les chevaux traînent une plus forte charge lorsqu'ils sont employés séparément; c'est aussi le cas pour le charroi des fumiers, toutes choses étant d'ailleurs égales. Un chariot attelé de quatre chevaux, ne contiendra pas une quantité double de celle d'un chariot à deux chevaux: au reste, rien de plus indéterminé que la quantité de fumier qu'on charge sur un chariot; cela ne dépend pas seulement de la force des animaux de trait, mais aussi de l'habitude, du soin qu'on met à charger, de la saison, de l'état des chemins,\setcounter{page}{48} et de l'éloignement. Le poids du même fumier varie aussi suivant qu'il est sec ou humide. Lors donc qu'on veut calculer la quantité de fumier d'après son poids et rédiger le tableau des engrais et de leur distribution, il faut peser une charge, telle qu'on les fait ordinairement, et répéter cela de temps en temps, afin de s'accoutumer à apprécier à l'œil la quantité des fumiers qu'on charie. Cette opération se fait très-facilement au moyen d'une grande romaine; cet instrument présente d'ailleurs des avantages évidents pour une exploitation rurale. Deux mille livres sont une charge médiocre pour un chariot à quatre chevaux, et l'on se rapprochera de la vérité, autant que cela est possible dans une estimation de ce genre, en prenant ce poids pour la moyenne d'un chariot. Cependant, avec de forts attelages, sur de bons chemins, et à de petites distances, on charge souvent jusqu'à 3000 livres. Il y a encore plus d'incertitude sur l'estimation du fumier d'après son volume, parce que ce volume dépend du degré de décomposition auquel la paille est parvenue, et la proportion qui existe entre la quantité de celle-ci, donnée en litière, et les excrémens qui y sont mélangés. Un pied cube de fumier très-pailleux, ne pèse souvent pas audelà\setcounter{page}{49} de 44 livres, tandis qu'un autre, d'un fumier où la paille est réduite en filamens, pèse, sans être comprimé, de 56 à 58 livres. Aussi la quantité de sucs contenue dans le fumier est elle proportionnée plutôt à la pesanteur de celui-ci qu'à son volume.
On consacre à un journal de terre, cinq, huit et jusqu'à dix chariots de fumier, de 2000 livr. chacun. Suivant que la quantité employée est plus ou moins forte, l'on dit qu'on a donné au sol un faible ou léger amendement, qu'on lui a donné un amendement complet ou qu'on l'a fumé fortement.
\comment{table}
Si l'on met sur un journal 5 chariots ou 10,000 livres, il y aura par perche . . . . . . . . . . . 55
Si l'on en met . . . . 6 . . . . 12,000. 66
. . . . . . . . . . . 7 . . . . 14,000. 77
. . . . . . . . . . . 8 . . . . 16,000. 88
. . . . . . . . . . . 9 . . . . 18,000. 100
. . . . . . . . . . 10 . . . . 20,000. 111
Lorsqu'on fume le plus abondamment, il tombe sur un pied carré environ sept dixièmes de livre pesant.
Le transport des fumiers sur les terres est une des opérations agricoles les plus importantes. Pour être faite avec soin et avec ordre, elle demande une attention particulière de la part de l'inspecteur des travaux. Il convient d'y employer autant d'attelages\setcounter{page}{50} que cela est possible, avec un nombre de manouvriers proportionné. Suivant que le terrain sur lequel le fumier doit être conduit est plus rapproché ou plus éloigné, il faut avoir une voiture de rechange pour deux ou trois attelages; afin qu'il y ait toujours un chariot auprès du tas pour occuper les chargeurs. Il faut avoir soin d'établir et de maintenir l'ordre nécessaire dans la distribution du temps; de sorte que, par exemple, de trois attelages il y en ait un en chemin pour revenir, tandis que l'autre décharge au champ, et que le troisième s'y rend; qu'ainsi aucun d'eux ne soit arrêté plus long-temps que cela n'est nécessaire pour atteler les chevaux au chariot qui vient d'être chargé. Il faut donc calculer le temps qui est nécessaire à chaque attelage, pour aller et revenir du champ; on consacre alors à cet ouvrage un nombre de chargeurs tel qu'ils soient toujours occupés, mais sur-tout que les attelages ne soient jamais obligés d'attendre le chariot qu'ils doivent conduire. Comme ce travail est plus ou moins pénible et long, suivant l'état dans lequel le fumier se trouve, on ne peut pas fixer d'une manière générale, le nombre de personnes qu'il est nécessaire d'y employer. On compte ordinairement un homme et\setcounter{page}{51} dèmi où un homme et une femme par attelage ; mais si le travail va fort vite, et si le fumier est très serré, ce nombre suffira à peine. La plus ou moins grande quantité de fumier qu'on veut consacrer à un champ, peut en général, mieux être calculée d'après l'éloignement des petits tas que l'on forme en déchargeant, qu'elle ne peut l'être d'après leur volume. Ordinairement les chariots de deux mille livres, ou même un peu plus forts, produisent environ neuf petits tas, de sorte que chacun de ceux-ci peut contenir environ deux cent vingt-deux livres de fumier. Suivant la quantité d'engrais qu'on veut donner au terrain, on peut facilement calculer l'espace sur lequel une charretée doit s'étendre dans une ligne, et l'éloignement qu'il doit y avoir entre les lignes elles-mêmes. Le meilleur moyen de déterminer la distance des tas, c'est de prendre pour mesure la longueur de l'attelage dès les chevaux de devant ou ceux de derrière, jusqu'à l'arrière-train du chariot ; mesure qu'on peut doubler au besoin : quant à la distance qui doit séparer les lignes, il faut la mesurer au pas, et il convient que cela soit fait par l'inspecteur des travaux lui-même. Quelquefois on juge convenable de fumer une\setcounter{page}{52} partie d'un champ plus fortement que l'autre: les hauteurs peuvent supporter un amendement plus riche, tandis qu'une moindre quantité de fumier suffit aux terrains bas, surtout à ceux qui sont placés immédiatement au-dessous des côteaux, parce que les sucs fertilisans y coulent d'eux-mêmes.
Lorsqu'indépendamment des gens employés à charger, on a encore d'autres ouvriers en suffisance, le mieux est de faire répandre le fumier à mesure qu'on le charrie, afin que l'inspecteur puisse en même temps surveiller ces deux opérations. D'ailleurs, le fumier se divise d'autant plus facilement qu'il est moins serré dans les petits monceaux; nouveau motif de ne pas laisser le fumier ainsi sans l'étendre.
Il est très-essentiel de bien diviser et éparpiller le fumier. Il ne faut donc pas épargner sur le nombre d'ouvriers employés à ce travail. On doit aussi faire marcher à la suite des ouvriers, un homme intelligent, qu'on rend responsable de la bonne répartition du fumier, et qui oblige les autres à faire leur travail avec soin. Il faut également veiller à ce que le fumier soit enterré aussi bien que cela est possible, surtout celui qui est pailleux; et même il convient de faire suivre la charrue par des gens munis de\setcounter{page}{53} fourches et de rateaux, afin de le répar- tir d'une manière plus égale dans le sillon.
Dans plusieurs pays, on mêle ordinaire- ment le fumier pailleux avec toutes sortes de substances végétales, ou simplement avec de la terre, et on le laisse se décomposer entièrement. Au moyen de ce procédé, les parties volatiles et liquides du fumier sont mieux conservées, et si l'addition qui a été faite, est composée de gazons, ses parties peuvent se combiner plus intimément avec la terre et exercer leur action sur elle; alors, et sur-tout si on y a joint un peu de chaux vive, il s'opère diverses compositions et di- verses combinaisons, que sans cela on n'eût obtenues que beaucoup plus tard. Il n'est pas sans vraisemblance, que l'eau ne soit elle-même décomposée en partie, et qu'elle ne passe à l'état solide en entrant dans des combinaisons.
Quelquefois on transporte sur la place à fumier les diverses matières dont le compost doit être formé, et là, on les réunit en tas; d'autres fois on leur destine des places voi- sines des bâtimens, ou mieux encore, dans les champs mêmes où le compost doit être employé; cette dernière méthode épargne\setcounter{page}{54} un double transport des substances qu'on associe au fumier. Il y a deux manières de former les composts. Quelquefois on dispose en couches horizontales les diverses matières dont ils sont composés; ces couches sont placées les unes sur les autres. Au bas du tas, on forme d'abord un lit de terre ou de gazon, auquel on donne de chaque côté en étendue, cinq à six pieds de plus que le tas ne doit avoir. Alors on y met une couche d'environ un pied d'épaisseur, de fumier aussi frais que cela est possible. Au-dessus de cette couche, on met derechef un lit de terre ou de gazon. Si l'on a d'autres matières qui soient susceptibles de putréfaction, on les place sur ce lit, qu'on recouvre d'une autre couche de fumier, et ainsi de suite, jusqu'à ce que le tas ait environ six pieds d'élévation en talus; alors on le couvre de nouveau d'une couche de terre. Souvent on mêle dans ce compost de la chaux vive; mais il ne faut pas que cette chaux soit en contact immédiat avec le fumier, parce qu'elle accélérerait sa décomposition, et donneroit trop d'intensité à celle-ci; on place donc la chaux entre deux couches de terre, ou entre la terre et d'autres substances\setcounter{page}{55} d'une putréfaction difficile, comme des feuilles d'arbres ou des choses semblables. Lorsque la partie qui déborde le tas a été imprégnée d'eau de fumier, on la remue et on l'étend sur le tas.
Le tas de compost s'échauffe et entre en fermentation; on le laisse dans cet état jusqu'à-ce que cette fermentation soit complètement achevée. Lorsqu'on ne sent plus aucune chaleur dans l'intérieur du tas, on le brasse et on l'arrange de manière que ce qui était dessus se trouve dessous, et que ce qui était dehors et non consommé, se trouve dans l'intérieur pour y subir sa fermentation. Souvent alors on place une seconde couche de terre sous le tas.
On donne au tas ainsi remué, une forme longue et étroite, semblable à celle d'un toit, afin qu'il soit plus exposé au contact de l'air. On croit que celui-ci fait augmenter en poids le fumier et le fait gagner en qualité. En effet, au moyen de cette disposition, il s'y forme une grande quantité de nitre; aussi les agriculteurs qui donnent de l'importance aux composts, les remuent-ils fréquemment, afin d'exposer à l'air une couche toujours nouvelle.
La seconde manière de faire le compost,\setcounter{page}{56} c'est de faire amener les différentes matières qu'on y destine tout autour de la place où le tas doit être formé. On étend au milieu la couche de terre sur laquelle on veut placer le tas, et on met auprès des divers monceaux qui sont à l'entour, des ouvriers avec des pelles. Ceux-ci prennent chacun à un de ces monceaux et jettent tous ensemble dans le grand tas. Les différentes substances se trouvent ainsi parfaitement mêlées. On associe de cette manière de la marne, du terreau, de la tourbe brisée et pulvérisée, de la mousse, des feuilles d'arbres, et de celles de pins en particulier, de la sciure, des dépouilles de végétaux et d'animaux, etc. On y joint encore le plus souvent de la chaux, des cendres, de la suie, et parmi toutes ces substances, on mêle du fumier récent, ou bien on arrose ce mélange avec de l'urine et de l'eau de fumier. On y met une plus ou moins grande quantité de chaux, suivant que les substances dont le compost est formé, sont d'une putréfaction plus ou moins difficile; on en met davantage, lorsqu'il y a des matières dans lesquelles l'acide domine, et qui, par conséquent, n'ont pas de la disposition à se décomposer. Plus il y a de substances animales, plus on peut\setcounter{page}{57} épargner la chaux. On doit également laisser reposer ces tas, jusqu'à-ce que leur fermentation soit terminée, alors il convient de les remuer plusieurs fois.
Les agriculteurs qui désapprouvent l'emploi du fumier d'étable dans ces composts, envisagent ce mélange comme une augmentation inutile de travail. Ce fumier, disent-ils, seroit suffisamment mêlé et incorporé à la terre par les labours, et cela auroit lieu d'une manière tout à la fois plus facile et plus convenable que par le moyen des composts. Outre cela, ils allèguent que la fermentation putride du fumier dans la terre est très-avantageuse au sol, et ils ont en effet raison s'ils parlent des terrains argileux et froids.
Mais ce qui plaide d'une manière encore plus forte contre l'emploi universel de ces composts, et qui le rend très-difficile, c'est que par ce moyen, le fumier d'étable ne peut être mis en usage, et n'acquiert son activité qu'une année plus tard, ce qui est d'une très-grande importance dans une exploitation qui n'a pas encore des fumiers en surabondance.
Indépendamment des produits effectifs en denrées, qu'on retire du fumier employé\setcounter{page}{58} immédiatement, on peut également en avoir obtenu de nouveaux engrais, avant l'époque où le compost seroit prêt à être répandu sur le sol.
On ne doit donc penser à établir des composts, que lorsqu'on possède des engrais en surabondance. Mais si l'on est effectivement dans ce cas, il y a de l'avantage à faire les avances qu'ils exigent, surtout si l'on a une grande quantité de ces substances qui ne sont pas d'une décomposition facile. Par là on peut se procurer et tenir en réserve une sorte de trésor, et s'assurer un riche produit des semailles même les plus misérables, en les fortifiant au moyen de cette espèce d'engrais.
Des expériences sans nombre ont confirmé que le meilleur moyen d'employer le compost, consiste à répandre cette espèce d'engrais sur le sol, sans l'enterrer. On charie le compost après le labour de semaille, et on l'éparpille tout de suite sur le champ. On sème immédiatement après, et on enterre avec la herse, ou en donnant un labour très superficiel. On peut aussi employer le compost à fumer dessus les semailles, on l'étend sur les céréales d'automne, quelquefois au printems seulement,\setcounter{page}{59} lorsque leur végétation a recommencé. Cette manière de fumer produit des effets étonnans lors même qu'on n'y emploie qu'une petite quantité de compost; c'est ce que prouve le témoignage des personnes qui l'ont elles-mêmes essayée, et la faveur dont elle jouit dans des contrées entières. Dans un district considérable de l'Angleterre, le comté d'Hereford, elle a été introduite dès les temps les plus reculés; l'on n'y emploie aucun fumier que sous cette forme. ( Les Anglais désignent cette manière de fumer, par le mot top dressing ) sans que la culture soit d'ailleurs très-distinguée dans cette province, les récoltes y sont remarquablement belles, et les cultivateurs de ce pays-là assurent qu'elles ne manquent jamais. Ils attribuent une influence presque magique au compost, répandu sur les céréales pendant leur végétation; selon eux, lorsque le froment a été en grande partie détruit par l'hiver, ou que l'orge, endommagée par la gelée, la sécheresse ou l'humidité, est languissante et végète à peine, le compost y produit des effets surprenans et très-prompts; on voit les plantes reverdir et renaître, aussitôt que le compost est répandu. Cet effet est confirmé d'une manière non équivoque\setcounter{page}{60} par tous les auteurs Anglais. Il y a donc un bien grand avantage à se procurer une provision d'engrais actifs de cette nature, sans cependant se priver de ce qui est nécessaire à l'année actuelle.
On trouve dans divers écrits des recettes qui donnent la quantité de chaque mélange par poids ou mesure. La manière la plus simple, c'est de mêler ensemble toutes les substances animales et végétales dont on dispose, de même que les minérales, si elles sont propres à cet usage, y joindre un peu de chaux vive, ajouter de la terre autant qu'il en faut pour absorber les gaz qui se développent, faire entrer le tout en fermentation, et brasser ensuite plusieurs fois, jusqu'à-ce que le mélange soit transformé en une matière homogène.
Lorsqu'on manque de paille, on se sert de plusieurs autres substances végétales, tant pour absorber les excrémens du bétail, et donner à celui-ci une litière sèche, que pour augmenter la quantité des engrais, parce que les végétaux, consacrés à cet usage sont plus promptement mis en putréfaction et transformés en terreau, lorsqu'ils sont unis à la fiente que lorsqu'on les laisse se décomposer d'eux-mêmes. Le choix de ces substances\setcounter{page}{61} doit donc se faire d'après leurs qualités comme litière, et leur plus ou moins de disposition à se décomposer promptement.
Les feuilles de sapin sont très-bonnes pour litière. Les forêts des contrées qui manquent de paille sont peuplées de cette espèce d'arbres. Lorsque les feuilles sont mêlées avec des excrémens d'animaux, elles se décomposent plus facilement que lorsqu'elles sont seules. Elles retardent cependant la fermentation du fumier, qui doit alors demeurer en tas plus long-temps. Lorsque la décomposition a en lieu, ce fumier, loin d'être inférieur à celui de paille, semble avoir plutôt l'avantage, parce que les feuilles de pin contiennent beaucoup plus de sucs nutritifs que la paille.
La feuille de chêne est d'une décomposition difficile, et contient une substance astringente, qui n'est pas favorable à la végétation, tant que la décomposition complète de cette feuille n'a pas eu lieu. Aussi faut-il que le fumier dont elle fait partie, demeure long-temps en tas, si l'on veut en tirer de véritables avantages. Si on l'emploie trop tôt, les feuilles restent un temps considérable avant de se putréfier, et peuvent\setcounter{page}{62} alors être plus nuisibles qu'avantageuses sur-tout dans les terrains légers.
Les feuilles des hêtres, des noyers, des châtaigniers, lorsqu'elles sont encore vertes, paroissent, il est vrai, plus contraires à la végétation que celle des chênes, puisqu'il ne croît presque pas d'herbe sous ces arbres; mais elles perdent bientôt leurs qualités nuisibles dans le fumier, et se décomposent beaucoup plus promptement. On a éprouvé de beaucoup meilleurs effets des fumiers faits avec ces espèces de feuilles que de ceux faits avec la feuille de chêne.
Les feuilles de l'aune, du saule et du peuplier, paroissent également avoir de la facilité à se décomposer; mais elles ont peu de consistance, et n'augmentent pas considérablement le volume du fumier qu'elles recueillent.
Dans les exploitations où toute la récolte de paille est employée à la nourriture des bestiaux, on compte essentiellement sur les feuilles pour litière, en remplacement de la paille; mais c'est un fait reconnu que c'est toujours au détriment des forêts, et que le désavantage qui en résulte pour celles-ci, dépasse les profits qu'en retire cette triste agriculture. L'obligation de fournir à cet usage\setcounter{page}{63} est devenu une servitude très-onéreuse pour le propriétaire des forêts, et son abolition a trouvé de grandes difficultés dans le système de culture établi. A la vérité, lorsque c'est le propriétaire de la forêt lui-même qui en fait usage, et qu'il a du discernement et de la modération, il peut quelquefois en tirer des avantages réels; mais ceux qui jouissent de ce droit sur la possession d'autrui n'usent ordinairement pas de semblables ménagemens.
Dans les pays qui produisent de la bruyère, c'est de cette plante qu'on fait le plus souvent usage pour litière, lorsqu'on a épuisé les ressources dont nous venons de parler. Quelquefois on fauche la bruyère, d'autrefois on écroute le sol où elle végète, avec une houe convenable à cette opération, et l'on transporte ainsi la plante elle-même, et la partie du sol qui a été enlevée par la houe. Quoique la bruyère ne se putréfie que difficilement dans le cours d'une année, les excrémens d'animaux auxquels elle est mêlée, la rendent si molle, et la privent tellement de la faculté astringente, que lorsqu'on la transporte sur les terres, elle ne tarde pas à être complétement décomposée et divisée. Dans une partie de la principauté de Lunnébourg,\setcounter{page}{64} de l'évêché de Brême, et de la Poméranie, beaucoup de gens considèrent la bruyère comme une chose tellement indispensable à l'économie rurale, qu'ils ne veulent pas mettre les landes en culture, quoiqu'ils en reconnaissent la possibilité, parce qu'ils ne croient pas pouvoir se passer de la bruyère pour faire des engrais.
En effet, cela demeurera vrai aussi longtemps qu'ils ne changeront pas la disposition de leur économie rurale. Au moyen d'un droit de recueillir de la bruyère sur le terrain d'autrui, plusieurs cultivateurs qui faisoient un usage rigoureux de ce droit, ont pu maintenir dans un état de fécondité frappant, des champs qui étoient d'ailleurs de mauvaise qualité. Mais comme la bruyère ne croît que lentement, sur-tout lorsqu'on a enlevé avec elle la superficie du sol, il faut peut-être cent journaux de bruyère, pour maintenir un seul journal de terre arable dans un état de prospérité; ainsi donc, cette opération ne peut être continuée que dans de petits domaines entourés de vastes étendues de terrains incultes. Si l'on recueille la bruyère à des distances considérables, les ouvriers et les attelages y sont employés une grande partie de l'année. Souvent il est plus\setcounter{page}{65} difficile de se procurer la quantité de bruyère qui doit fournir à l'amendement d'un journal de terrain; que de couvrir la même étendue avec de la marne ou du terreau. Cependant personne n'est arrêté par les frais qu'occasionne ce premier travail, tandis qu'on est effrayé par la perspective du dernier, tant l'habitude a de puissance et de force.
On ne se borne pas à étendre de la bruyère sous le bétail, on mêle encore avec le fumier les gazons enlevés à la houe en écroulant le sol, et l'on fait du tout, dans les champs, des tas qu'on laisse subsister jusqu'à l'entière décomposition de ces matières.
Lorsque ce fumier ainsi mêlé d'une petite quantité d'excrémens d'animaux, est bien consommé, et qu'on l'étend sur les champs en quantité suffisante, il produit souvent de très-belles récoltes de seigle et sur-tout de blé noir. Comme il n'y pousse que très-peu de mauvaises herbes, le terrain n'a pas besoin de jachère, et il rapporte consécutivement six ou sept récoltes, qui, du reste, vont déclinant successivement. Les personnes qui ne savent pas à combien de difficultés cette acquisition d'engrais est liée, sont disposées à envisager cette opération.\setcounter{page}{66} comme très-recommandable, et les terrains à bruyère comme d'une grande utilité. Le célèbre De Luc, entr'autres, dans son voyage à travers ces contrées, trouva dans cette circonstance des motifs pour se prononcer contre le partage des communes. Il est sans contredit des cas où, sans faire un mauvais calcul, le cultivateur peut avoir recours à ce moyen, et où il peut avec avantage employer de la bruyère pour la litière des bestiaux. Cela peut sur-tout convenir dans les bergeries, parce que cette plante est plus facilement décomposée par l'action du fumier de mouton.

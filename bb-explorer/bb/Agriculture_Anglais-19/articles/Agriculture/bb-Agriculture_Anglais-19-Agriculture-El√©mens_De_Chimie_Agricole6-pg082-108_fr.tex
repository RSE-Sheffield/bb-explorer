\setcounter{page}{82}
\chapter{Agriculture}
\section{ELEMENTS OF AGRICULTURAL CHEMISTRY, etc. Élémens de chimie-agricole en un cours de leçons pour le Département d'Agriculture; par Sir HUMPHRY DAVY. Londres, 1813. \large{(Sixième extrait. Voy. p. 3).}}
LES produits végétaux les plus importans et les plus communs, le mucilage, la fécule, le sucre, et la fibre ligneuse, sont composés d'eau, ou des élémens de l'eau dans leur proportion convenable; et de charbon. Ces élémens, en totalité ou en partie, existent dans toutes les plantes; et la décomposition de l'acide carbonique, et la combinaison de l'eau dans la charpente du végétal, sont des procédés qui doivent avoir lieu presque universellement. Lorsque le gluten et l'albumine se trouvent dans les plantes, on peut soupçonner que l'azote qu'elles contiennent provient de l'atmosphère; mais ce n'est pas une vérité d'expérience. On pourroit tenter celle-ci sur les mousserons et les champignons.\setcounter{page}{83} ÉLÉMENS DE CHIMIE-AGRICOLE. Dans la formation des bourgeons, et celle des rejetons des racines, l'oxygène paroît être uniformément absorbé, comme dans la germination des graines. J'ai exposé une petite pomme de terre humectée d'eau commune, à vingt-quatre pouces cubes d'air atmosphérique, à la température de 59° F. On vit paroître une pousse dès le troisième jour : j'examinai l'air lorsqu'elle eut atteint la longueur d'un demi pouce ; il y avoit eu à-peu-près un pouce cube d'oxygène absorbé, et le même volume d'acide carbonique formé. La saveur de la pousse étoit sucrée; et il y avoit probablement quelque liaison entre l'absorption de l'oxygène et la production de l'acide carbonique d'une part, et d'autre part, la conversion d'une partie de la fécule en sucre. Lorsqu'on fait dégeler les pommes de terre gelées, on leur trouve une saveur sucrée. Probablement il y a absorption d'oxygène dans ce procédé; et s'il en est ainsi, on pourroit prévenir ce changement en faisant dégeler hors du contact de l'eau, par exemple, sous de l'eau récemment soumise à l'ébullition. Lorsque le blé talle, c'est-à-dire, lorsque des pousses latérales se développent autour de la tige principale, il y a tout lieu de croire qu'il se fait une absorption d'oxygène\setcounter{page}{84} car la tige, à l'endroit du déploiement, contient toujours du sucre, et les pousses sortent d'une partie privée de l'action de la lumière. La culture à la houe favorise ce procédé, car la terre qu'on élève contre la tige la met à l'abri de la lumière, sans la priver d'oxigène. J'ai compté de quarante à cent vingt pousses produites par un seul grain de blé, dans une récolte assez ordinaire de froment en culture; et Sir Kenelm Digby nous apprend, qu'en 1660 les Pères de la doctrine chrétienne à Paris conservoient et montroient comme curiosité une plante d'orge, qui avoit deux cent quarante-neuf tiges sortant d'une seule racine, ou d'un seul grain; ils y comptèrent plus de dix-huit mille grains.
La grande multiplication qui résulte de la transplantation du blé, dépend de cette circonstance, savoir, que chaque rejeton du blé qui talle, peut être séparé de la tige et traité comme une plante distincte. On trouve dans les Trans. Phil. Vol. LVIII, p. 203, le fait suivant: Mr. C. Miller, de Cambridge, planta quelques grains de blé le 2 juin 1766; le 8 août il en arracha un plant, qu'il partagea en dix-huit rejetons; ceux-ci furent replantés et séparés de nouveau, et plantés séparément dans les mois de septembre et d'octobre, pour passer l'hiver: cette division pro\setcounter{page}{85} duisit soixante-sept plants. On les reprit en mars et avril; et ils produisirent cinq cents plants; on obtint ainsi, d'un seul grain de froment, vingt-un mille cent neuf épis, qui donnèrent quarante-sept livres sept onces de froment, dont le nombre de grains fut estimé à cinq cent septante-six mille huit cent quarante.
Il est évident, d'après ce qui précède, que le changement qui a lieu dans les sucs de la feuille par l'action de la lumière solaire, doit tendre à augmenter la proportion de matière inflammable dans leurs autres parties constituantes. On remarque que les feuilles des plantes qui croissent dans les ténèbres ou à l'ombre, sont toujours pâles; leurs sucs sont aqueux ou saccharins, et elles ne donnent ni huiles ni résines. Je vais raconter une expérience sur ce sujet.
Je pris deux poids égaux, chacun de quatre cents grains, des feuilles de deux plantes d'endive; l'une d'un vert brillant; elle avoit toujours été exposée à la lumière; l'autre, presque blanche, celle-ci avoit crû à l'ombre sous une boîte. On les fit bouillir l'une et l'autre dans l'eau, jusques à l'état de pulpe; on fit sécher ensuite la partie non dissoute, et on l'exposa à l'action de l'alcool chaud. La matière tirée des feuilles vertes\setcounter{page}{86} lui donna une couleur d'olive; celle des feuilles pâles ne changea pas sa couleur. L'alcohol, dans lequel les feuilles pâles avoient été mises en digestion, laissa à peine un résidu sensible après qu'on l'eut fait évaporer; tandis que ce résidu fut très- considérable pour les feuilles vertes, traitées de même. On prit cinq grains de ce résidu pour l'allumer; il brûla avec flamme résineuse. Les feuilles vertes donnèrent cinquante-trois grains de fibres ligneuses; les pâles, seulement trente-un.
On a dit, dans la troisième leçon, que la sève, dans les cas ordinaires, descend probablement des feuilles dans l'écorce. Le tissu de celle-ci est ordinairement si lâche, que l'atmosphère pourroit bien exercer quelque influence sur son intérieur au travers des couches corticales. Mais les changemens qui ont lieu dans les feuilles paroissent suffisans pour expliquer la différence qui existe entre les produits qu'on obtient de l'écorce, et ceux de l'aubier; les premiers contiennent plus de matière charbonneuse que les derniers.
Lorsque l'on considère la ressemblance des élémens des divers produits végétaux, sous le point de vue indiqué dans la troisième leçon, il est facile de concevoir comment\setcounter{page}{87} les diverses parties organiques peuvent être formées de la même sève, par l’influence variée de la chaleur, de la lumière, et de l’air. Les liquides saccharins et mucilagineux peuvent se convertir par l’abstraction de l’oxygène, en produits inflammables, fixes et volatils, tels que les résines, les huiles, le camphre, la fibre ligneuse, etc.; et par la soustraction du carbone et de l’hydrogène, des matières éminemment combustibles, et insolubles dans l’eau peuvent être converties en fécule, en sucre, en acide végétal, et autres substances semblables. Les huiles volatiles limpides elles-mêmes, qui contiennent le principe odorant de la fleur, sont composées des mêmes élémens essentiels que la fibre ligneuse, mais en des proportions différentes. Les unes et les autres sont formées par des changemens opérés dans les mêmes organes, sur les mêmes matériaux, et dans le même temps.
Mr. Vauquelin a tenté récemment d’estimer les changemens chimiques qui ont lieu dans la végétation, en analysant quelquesunes des parties organisées du marronnier d’Inde dans les différentes périodes de son accroissement. Il trouva, dans les boutons recueillis au mois de mars 1812, le tannin, et la matière albumineuse, susceptibles d’être\setcounter{page}{88} obtenus séparément; mais se combinant ensemble après leur extraction. Dans les écailles qui enveloppent le bouton, il trouva le principe tannant, un peu de matière sucrée, de la résine, et une huile fixe. Il obtint des feuilles développées, les mêmes principes que ceux extraits des boutons, et de plus, une matière verte résineuse, particulière. Les pétales des fleurs donnèrent une résine jaunâtre, de la matière sucrée, de l'albumine, et un peu de cire. Les étamines fournirent du sucre, de la résine, et du tannin.
Les marrons d'Inde, examinés d'abord après leur formation, donnèrent une grande quantité d'une matière qui parut être une combinaison d'albumine et de tannin. Toutes les parties de la plante contenoient des combinaisons salines des acides acétique, et phosphorique.
Mr. Vauquelin ne put pas obtenir une quantité suffisante de sève du marronier pour la soumettre à l'analyse ; circonstance fort à regretter; et il n'a pas donné les quantités relatives des diverses substances qu'il a distinguées dans les boutons, les feuilles, les fleurs, et les graines. Il est cependant probable, d'après son travail non terminé, que la quantité de matière résineuse s'augmente dans la feuille; et que la pulpe fibreuse\setcounter{page}{89} breuse blanche du marron d'Inde est formée par l'action réciproque de la matière albumineuse et astringente fournie probablement par les différentes cellules ou par des vaisseaux. J'ai déjà dit, que le cambium \footnote{On a vu ailleurs, que c'est le liquide mucilagineux qui se trouve entre le bois et l'écorce.} qui paroît fournir à la formation des parties nouvelles dans le tronc et les branches, doit probablement sa solidification au mélange de deux variétés de sève; l'une, qui monte des racines, et l'autre, qui probablement descend des feuilles. J'essayai, au mois de mai 1804, à l'époque où le cambium se formoit dans le chêne, d'établir la nature de l'action de la sève de l'aubier sur les liquides de l'écorce. En perçant l'aubier dans un jeune chêne, et appliquant une petite pompe aspirante à l'ouverture, je me procurai facilement une petite quantité de sève; mais je ne pus point en tirer de l'écorce par ce procédé. Je fus forcé de recourir à une solution aqueuse de ses principes, en mettant infuser dans l'eau chaude une petite quantité d'écorce fraîche. Le liquide que j'obtins ainsi étoit fortement coloré, et astringent; il produisoit immédiatement\setcounter{page}{90} dans la sève de l'aubier, un précipité, non coloré, et de saveur douceâtre et légèrement astringente.
L'accroissement des arbres et des plantes, doit dépendre de la quantité de sève qui passe dans leurs organes; de la qualité de cette sève, et des modifications qu'elle éprouve par les influences atmosphériques. L'eau, comme véhicule des principes nutritifs des plantes, est l'excrétion particulière des feuilles. Le Dr. Hales a trouvé qu'un girasol transpiroit par ses feuilles, dans une journée de douze heures, une livre quatorze onces d'eau; quantité qui toute entière avoit été aspirée par les racines.
J'ai légèrement touché dans la seconde et troisième leçon, aux causes de l'ascension de la sève. C'est par suite de l'attraction capillaire que les racines prennent les liquides du sol; mais cette propriété seule n'explique pas l'élévation rapide de la sève jusques dans les feuilles. Ceci est prouvé sans replique par le fait suivant, que rapporte Hales (Stat. Veget T. I. p. 114.) On coupa en entier un sep. de vigne de quatre à cinq ans; et on lui adapta avec soin un tube de verre, recourbé en syphon, et rempli de mercure; de manière que la force de la sève ascendante pouvoit se mesurer\setcounter{page}{91} par son effet, pour faire monter ce métal liquide. On trouva, au bout de peu de jours, que la sève avait été poussée en avant assez fort pour élever le mercure de trente-huit pouces; effet qui surpasse de beaucoup celui de la pression atmosphérique ordinaire. L'attraction capillaire ne s'exerce que par les surfaces des petits vaisseaux, et ne peut jamais élever les liquides dans des tubes au-dessus des vaisseaux eux-mêmes.
J'ai cité au commencement de la troisième leçon, l'opinion de Mr. Knight, qui attribue principalement aux contractions et expansions alternatives du grain argenté de l'aubier l'ascension des liquides qu'il renferme. Les conjectures de cet excellent physiologiste sont rendues très-probables par les faits dont il les appuie. Il a trouvé qu'une légère augmentation de température suffisoit pour faire séparer les unes des autres les fibres du grain argenté, et qu'une foible diminution de chaleur les faisoit se rapprocher. C'est au printemps et en automne que l'ascension de la sève est la plus rapide, et c'est aussi l'époque de l'année où la température est la plus variable; or si l'on suppose que dans leurs dilatations et contractions alternatives les fibres élastiques du grain argenté exercent une pression sur les\setcounter{page}{92} cellules et les vaisseaux qui contiennent le fluide absorbé par l'attraction capillaire des racines, ce liquide doit toujours se mouvoir de bas en haut vers les points où sa quantité est moindre.
Les expériences de Montgolfier, le célèbre inventeur des ballons aérostatiques, ont montré que l'eau peut être soulevée jusques à une hauteur presqu'indéfinie par une très petite force, pourvu que sa pression soit diminuée par des divisions continuelles dans la colonne du liquide. On a de fortes raisons de supposer que ce principe d'action doit aider à l'ascension de la sève dans les cellules et les vaisseaux qui n'ont aucun moyen rectiligne de communication, et qui présentent dans tout leur tissu des obstacles à la pression verticale de la sève.
Les changements qui ont lieu dans les feuilles et les boutons, et la faculté de transpiration de ces organes sont intimément liés avec le mouvement ascendant de la sève. Plusieurs expériences du Dr. Hales prouvent cette dépendance.
Il sépara une branche de pommier, et la plongea dans l'eau, après lui avoir appliqué un tube indicateur, au mercure. Pendant que ce rameau posséda ses feuilles, l'action ascendante éleva le mercure de quatre\setcounter{page}{93} pouces; lorsqu'on les lui eut enlevées, elle le souleva à peine d'un quart de pouce.
On remarque, aussi, que les arbres dont les feuilles sont molles, d'un tissu spongieux, et poreuses, à leur surface supérieure, font monter la sève d'une manière beaucoup plus énergique.
Le même physicien exact que je viens de citer a trouvé que le poirier, le coignassier, le cerisier, le noyer, le pêcher, le groseillier, l'aune, et le sycomore, qui tous ont des feuilles molles et sans vernis font monter le mercure (dans des circonstances favorables) de trois jusques à six pouces; tandis que l'ormeau, le chêne, le châtaigner, le coudrier, le saule, et le frêne, qui ont des feuilles plus fermes et plus lisses, n'élèvent le mercure que de un ou deux pouces, dans les expériences de succion dont on a parlé.
Les arbres toujours verts, et ceux qui portent des feuilles vernies, l'élèvent à peine; sur-tout le laurier, et le laurustin.
Il convient de citer les faits qui montrent que, dans plusieurs cas les fluides descendent par l'écorce. Ces résultats sont de nature moins équivoque que ceux qui prouvent l'ascension de la sève par l'aubier; quoiqu'il que plusieurs de ces derniers soient satisfaisans.\setcounter{page}{94} Mr. Baisto mit tremper des branches de divers arbres dans une infusion de garance, et les y maintint pendant long-temps. Il trouva, dans tous les cas, que le bois devenoit rouge avant l'écorce, et que celle-ci ne commençoit à prendre de la couleur que lorsque le bois étoit totalement coloré et jusqu'à ce que les feuilles fussent affectées; et que la matière colorante commençoit à se manifester en haut, dans l'écorce en contact immédiat avec les feuilles.
Mr. Bonnet a fait des expériences semblables, qui lui ont fourni des résultats analogues quoique moins distincts que ceux de Mr. Baisto.
Du Hamel a trouvé que, dans différentes espèces de pin, et dans d'autres arbres, lorsqu'on enlevoit des bandes d'écorce, la partie supérieure seule de la plaie donnoit du liquide et que l'inférieure demeuroit sèche.
On peut faire la même remarque en été sur les arbres à fruit, lorsque l'écorce seule est enlevée sans qu'on ait touché à l'aubier.
J'ai dit, dans la troisième leçon que, lorsqu'il se forme de l'écorce nouvelle à la place d'un anneau de l'ancienne qu'on a enlevé, il commence à paroître au bord supérieur de la plaie et se déploie lentement de haut en bas. Si l'expérience est bien faite, on ne\setcounter{page}{95} voit point de matière corticale nouvelle s'élever de bas en haut dans cet anneau. Je suppose l'expérience bien faite; car si on laisse une couche corticale intérieure qui communique avec le bord supérieur, il se formera au-dessous, de l'écorce nouvelle qui se couvrira d'épiderme et qui paroîtra comme si elle procédoit de l'aubier mis à nud, et comme si elle eût été formée dans la plaie. Cette circonstance pourroit induire en erreur.
Dans l'été de 1804 j'examinai quelques ormeaux à Kensington. L'écorce de plusieurs d'entre'eux avoit été fort attaquée, et dans quelques endroits, enlevée en lambeaux de plus d'un pied quarré. Dans la plupart de ces plaies la nouvelle écorce se formoit en descendant, et en se déployant par degrés autour de l'ouverture. Mais on voyoit dans deux cas, et très-distinctement, l'écorce se former vers le bord inférieur. Je fus très-surpris d'abord, de cette apparence, si contraire à l'opinion reçue; mais en passant la pointe d'un canif le long de la surface de l'aubier, de bas en haut, je trouvai qu'une portion de la couche corticale, qui étoit de la couleur de l'aubier, étoit demeurée en communication avec le bord supérieur de la plaie, et que cette couche avoit contribué\setcounter{page}{96} à la formation de la nouvelle écorce. Je n'ai pas eu d'occasion récente de revoir ces arbres, mais je ne doute pas qu'on ne puisse encore observer ce phénomène, car l'écorce ne peut être rétablie qu'au bout de plusieurs années. Pour expliquer le résultat obtenu par Mr. Palisot de Beauvois, dont j'ai parlé dans la troisième leçon, on peut supposer que le liquide cortical coulant le long de l'aubier sur l'écorce isolée, produisit son accroissement ; ou bien, on pourroit dire qu'à l'époque où l'écorce fut séparée de celle qui l'entouroit, elle contenoit assez de liquide cortical pour former de nouvelles parties solides par son action sur la sève de l'aubier. Le mouvement de la sève dans l'écorce paroît sur-tout être influencé par la gravitation. Lorsque les parties aqueuses ont été en grande partie dissipées par la transpiration des feuilles, et que les ingrédiens mucilagineux, inflammables, et astringens ont été augmentés par l'action de la chaleur, de la lumière, et de l'air, l'impulsion continuelle qui a lieu de bas en haut dans l'aubier chasse le résidu épaissi, dans les vaisseaux corticaux, qui ne reçoivent pas d'autre nourriture. Arrivé là, sa pesanteur le pousse en bas, avec une vitesse qui doit dépendre\setcounter{page}{97} de la consommation générale qui se fait des liquides de l'écorce, dans le procédé de la végétation; car il y a tout lieu de croire qu'aucun fluide ne retourne au sol par les racines; et il est impossible de concevoir une libre communication latérale entre les vaisseaux absorbans de l'aubier, dans les racines, et ceux qui fournissent les liquides à l'écorce; car s'il en existoit une pareille, il n'y auroit pas de raison pour que la sève ne montât aussi bien par l'écorce que par l'aubier; puisque ces deux organes seroient soumis à la même force physique.
Quelques auteurs ont supposé que la sève monte dans l'aubier, et descend dans l'écorce, par l'effet d'une force semblable à celle qui produit la circulation du sang, et analogue à la force musculaire, appartenant aux parois des vaisseaux.
Le Dr Thomson, a établi dans son Systême de chimie un fait qu'il regarde comme démontrant l'existence de l'irritabilité dans les organes végétaux vivans. Lorsqu'on sépare dans une tige d'Euphorbe (euphorbia peplis) les feuilles, des racines, par deux incisions parallèles, le jus laiteux sort des deux sections. "Or, dit l'ingénieux auteur, cet effet ne peut avoir lieu sans une action propre aux vaisseaux, car ils ne pouvoient être plus que\setcounter{page}{98} pleins lorsqu'on les a coupés ; et leur diamètre est si petit, que s'il n'éprouvoit aucun changement ni influence particulière, l'attraction capillaire seroit plus que suffisante pour soutenir les liquides dans leur intérieur, et empêcher qu'une seule goutte s'écoulât au-dehors. Puis donc qu'il s'échappe, il faut qu'il soit chassé par quelque force particulière.
On peut répondre à ce raisonnement, que les parois de tous les vaisseaux sont flexibles et capables de céder à la gravitation, comme le font les veines dans le système animal long-temps après qu'elles ont perdu toute leur vitalité. Cet effet est absolument différent de l'irritabilité, et on peut comparer le fait cité à ce qui arriveroit à un tube de résine élastique plein d'un liquide, et à qui l'on feroit deux piqûres, l'une au-dessus de l'autre ; le liquide s'écouleroit par les deux, mais plus abondamment par l'inférieure : et c'est ce qui a lieu dans l'euphorbe.
Le Dr. Barton a trouvé que les plantes végétoient mieux dans l'eau où l'on avoit mis un peu de camphre que dans l'eau pure. On a cité ce fait en faveur du système de l'irritabilité vasculaire dans les végétaux, qu'on suppose dans ce cas exaltée par ce stimulant particulier. Cette théorie me semble peu satisfaisante. Nous connoissons au\setcounter{page}{99} camphre une saveur piquante désagréable et une odeur très-forte; mais quant à son influence sur le corps humain les médecins ne sont pas d'accord; les uns le regardent comme stimulant, les autres, comme sédatif. Et lors même que l'irritabilité végétale seroit prouvée, on ne pourroit pas en conclure que parce que le camphre aide à l'accroissement des plantes, il le fait en vertu d'une action sur leur force vitale; on ne peut pas conclure de l'influence de qualités non prouvées à l'existence d'une propriété également incertaine.
Il est facile de concevoir que le camphre peut favoriser la végétation, tout comme le font les matières sucrées, et mucilagineuses, et sur-tout les huiles, auxquelles il ressemble par sa nature chimique, et qui fournissent aux plantes un aliment et non un stimulant. Ce sont-là des matériaux d'assimilation, bien plutôt que d'irritation organique.
Ainsi donc, les argumens cités en faveur d'une contraction des vaisseaux, analogue à l'action musculaire, n'ont guères de poids; il existe d'ailleurs des faits qui la rendent fort improbable.
Lorsqu'en hiver on introduit un rameau de vigne ou de quelqu'autre arbre dans l'intérieur d'une serre chaude, en laissant tout\setcounter{page}{100} le reste de la plante exposé à l'air froid, on ne tarde pas à voir la sève se mettre en mouvement vers les boutons du rameau réchauffé, ces boutons se développent peu-à-peu, et commencent à transpirer: enfin les feuilles se montrent. Or, s'il falloit, pour que la sève montât dans les vaisseaux, ou les cellules, qu'ils éprouvassent quelques contractions particulières, il ne seroit pas possible que l'application de la chaleur à un seul rameau, vers son extrémité, suscitât l'irritabilité dans un tronc qui en est distant de plusieurs pieds, ou dans des racines que renferme un sol refroidi. Mais, si l'on accorde, que l'action de la chaleur fait monter le liquide simplement en "l'allégeant", en facilitant l'action capillaire, et en dilatant les fibres du grain argenté, le phénomène se trouvera en parfait accord avec les principes énoncés dans cette leçon même.
L'ilex, ou chêne vert, conserve ses feuilles en hiver, lors même qu'il est enté sur le chêne commun; et par suite de l'action des feuilles il y a un certain mouvement de la sève vers l'ilex qui, comme dans le cas précédent, ne peut pas s'accorder avec la théorie de l'irritabilité.
Il est impossible de lire avec attention la *Statique des végétaux* de Hales, sans se pénétrer\setcounter{page}{101} nétrer du sentiment des rapports intimes qui existent entre le mouvement de la sève et l'action des causes physiques connues. Cet habile physicien a observé dans un même arbre, que dans une matinée froide et obscure, dans laquelle il ne montoit point de sève, un rayon de soleil agissant pendant une demi-heure suffisoit pour mettre ce liquide en plein mouvement. L'effet étoit subitement arrêté, par le changement du vent, en passant du sud au nord; dans une soirée froide succédant à un jour chaud, la sève commençoit à redescendre; une pluie chaude, et une giboulée de grésil produisoient des effets contraires.
On voit aussi, d'après un nombre de ses observations, que les différentes forces qui agissent sur l'arbre adulte produisent des effets divers, selon les saisons.
Ainsi, aux premiers jours du printems, avant que les boutons se déployent, les variations de la température, et les changemens hygrométriques de l'air exercent leur plus grande influence sur la dilatation et la contraction des vaisseaux; c'est alors ce que les jardiniers appellent la saison des pleurs de l'arbre. Lorsque les feuilles sont tout-à-fait déployées, c'est vers ces organes que la sève se porte principalement. C'est pourquoi un\setcounter{page}{102} arbre qui donne beaucoup de sève par une entaille faite au tronc, à l'époque où les boutons vont s'ouvrir, n'en fournit plus en été lorsque les feuilles sont parfaites. Mais, dans les temps variables, vers la fin de l'automne lorsque les feuilles tombent, il peut donner encore un peu de sève dans les jours les plus chauds, mais non en d'autres circonstances. Dans tous ces cas il ne se passe rien d'analogue aux fonctions qui dépendent de l'irritabilité dans le système animal.
Dans celui-ci, le cœur et les artères sont en pulsation constante; leurs fonctions ne cessent pas un instant, dans tous les climats et en toute saison, vers les neiges du pôle comme sous le soleil du tropique; elles ne cessent point pendant le sommeil nocturne commun à la plupart des animaux, ni dans ce long sommeil d'hiver particulier à certaines espèces. Cette faculté est en rapport avec l'organisation animale et n'appartient qu'aux êtres doués de la faculté locomotive volontaire; elle naît avec la première apparence de la vitalité; elle ne disparaît qu'avec la dernière étincelle de la vie.
On peut dire avec justesse que les végétaux sont des systèmes vivans, dans ce sens, savoir, qu'ils sont doués des moyens de convertir les élémens communs de la matière,\setcounter{page}{103} en structure organisée, tant par assimilation que par reproduction. Mais il ne faut pas se laisser induire en erreur par une application trop étendue du mot vie, comme s'il existoit dans celle des plantes une force quelconque semblable à celle qui produit la vie dans les animaux. Pour mettre en action les fonctions végétales, il suffit des agens physiques ordinaires; mais dans les systèmes animaux, ces agens sont subordonnés à un principe, d'un ordre supérieur. Et pour exprimer plus simplement ma pensée, je dirai, qu'il est sans doute peu de physiciens disposés à attribuer à l'économie animale quelque chose d'étranger à la matière commune, quelque principe immatériel; il faut laisser pareille doctrine aux poètes; l'imagination peut aisément créer des Dryades pour nos arbres, et attacher des Sylphes à nos fleurs; mais ces dryades et ces sylphes ne peuvent figurer dans la physiologie végétale; et c'est pour des raisons à-peu-près aussi fortes, qu'il faut en exclure l'irritabilité, et l'animalité.
Lorsque l'opération des divers agens physiques sur les vaisseaux à sève des plantes vient à cesser, et que ce liquide est en repos, les matières qui y étoient dissoutes par\setcounter{page}{104} La chaleur se déposent contre les parois des tubes, dont le diamètre est déjà fort diminué. En conséquence de ce dépôt, une matière nutritive se trouve préparée pour les premiers besoins de la plante au retour du printemps, pour aider à l'ouverture des boutons et à leur expansion, à l'époque où par suite du défaut de feuilles, le mouvement est encore foible.
C'est le Dr. Darwin, qui le premier a signalé ce principe important de l'économie végétale; et Mr. Knight l'a développé et appuyé d'un nombre d'expériences.
Il a fait plusieurs incisions dans l'aubier du sycomore et du bouleau, à diverses hauteurs; et en examinant la sève qui en découlait, il la trouva d'autant plus douce et mucilagineuse, que l'ouverture d'où elle sortoit étoit plus élevée. Il ne pouvoit attribuer cet effet à aucune autre cause qu'à ce que ce liquide avoit dissous du sucre et du mucilage, qui avoient été comme emmagasinés pendant l'hiver.
Il examina l'aubier de divers plantons de chêne crus dans la même forêt; dont quelques-uns avoient été coupés en hiver, et d'autres, en été: il trouva toujours plus de matière soluble dans le bois coupé en hiver:\setcounter{page}{105} et sa pesanteur spécifique étoit aussi plus grande.
Cette différence a lieu dans tous les arbres, comme aussi dans les gramens et les arbrisseaux. Les nœuds des gramens perennes contiennent plus de matière sucrée et mucilagineuse en hiver que dans toute autre saison; et c'est pour cela que le fiorin (agrostis alba) qui abonde en nœuds de ce genre, fournit une nourriture si bonne pour l'hiver.
C'est au fort de cette saison, que les racines des arbrisseaux contiennent la plus grande quantité de matière nutritive; et le bulbe, dans toutes les plantes qui ont cet organe, est le receptacle dans lequel cette matière est en magasin pendant l'hiver.
Dans les plantes annuelles, la production des fleurs et des graines semble absorber toute la matière nutritive; et il n'existe rien qui tende à la conserver.
Lorsque les gramens perennes sont broutés fort près du sol en automne, les fermiers ont souvent observé qu'ils ne poussent jamais vigoureusement au printemps. C'est parce que cette partie de la tige, qui au retour de la végétation auroit fourni de la sève concrète, n'existe plus.
\setcounter{page}{106} Les constructeurs de navires préfèrent les chênes dont l’écorce a été enlevée au printemps, et qui ont été coupés dans l’automne ou l’hiver suivant. La raison de cette supériorité est la dépense qui se fait au printemps, de la sève concrète pour la nourriture de la feuille; et la circulation étant détruite, cette sève ne se reproduit pas. Le bois ayant ses pores vidés de matière sucrée est moins susceptible d’être mis en fermentation par l’action de l’humidité et de l’air.
Dans les arbres perennes, il se produit chaque année un nouvel aubier, et par conséquent, un nouveau système de vaisseaux, dans lesquels la nourriture pour l’année suivante est déposée; de manière que les nouveaux boutons, sont munis, comme la plumule de la graine, d’un réservoir de matière essentielle à leur premier développement.
L’aubier, à mesure qu’il vieillit, se convertit en bois de cœur; et constamment pressé, comme il l’est, par la force expansive des nouvelles fibres, il devient plus dur, plus dense, et à la fin, il perd tout-à-fait sa structure vasculaire. Au bout d’un temps plus ou moins long, il obéit aux lois communes de la matière morte, il se désunit,\setcounter{page}{107} sé décompose, et se convertit en ses élémens aériformes et carboniques ; c'est-à-dire, qu'il se réduit aux principes dont il avoit été composé.
C'est la décomposition naturelle du cœur du bois qui constitue la grande limite de l'âge et de l'accroissement des arbres. Cette décomposition attaque plus fréquemment les jeunes branches des vieux arbres, que les branches de même grosseur des jeunes plantes. Il en est de même des greffes ; celles-ci ne sont nourries que par la sève de l'arbre sur lequel elles sont entées ; les propriétés ne sont point changées par cette circonstance ; leurs feuilles, leurs fleurs et leurs fruits sont les mêmes que si la greffe eût végété sur l'arbre où elle a été prise. Le seul avantage qui résulte de l'ente, est de procurer au rameau greffé une nourriture plus abondante et plus saine que celle qu'il auroit eue dans son état naturel ; il est rendu plus vigoureux et il produit de plus belles fleurs et de meilleurs fruits. D'autre part, il partage les infirmités et la disposition à la décadence, de l'arbre vieux dont il a été tiré.
Ceci paroît résulter évidemment des observations et des expériences de Mr. Knight. Il a souvent essayé d'enter de jeunes rejetons\setcounter{page}{108} et des pousses vigoureuses provenant de vieux arbres à bon fruit, sur de jeunes sauvageons. Les entes végétèrent bien pendant deux ou trois ans; mais elles cessèrent de prospérer, et montrèrent les mêmes signes de vieillesse et de décadence qu'on marquoit dans les arbres d'où elles avoient été extraites.
C'est par cette raison, qu'un si grand nombre des variétés de pommes qu'on vantoit autrefois pour leur saveur et leur usage dans la confection du cidre se sont détériorées peu-à-peu, et menacent de disparoître. Le golden pippin, la calvine rouge, et le moil, si parfaites au commencement du dernier siècle, sont arrivées actuellement au dernier terme de leur détérioration; on a beau chercher à les maintenir par des greffes choisies, on ne fait que multiplier une variété maladive et épuisée.
(La suite au Cahier prochain.)
\setcounter{page}{269}
\chapter{Agriculture}
\section{ELEMENTS OF AGRICULTURAL CHEMISTRY, etc. Élémens de chimie-agricole en un Cours de leçons pour le Département d'Agriculture; par Sir HUMPHRY DAVY. Londres, 1813. \large{( Dernier extrait. Voy. p. 229 ).}}
De l'amélioration des terrains par l'écobuage; principes chimiques de cette opération. De l'arrosement et de ses effets. De la jachère; de ses avantages et de ses inconvéniens. Des rotations de récoltes. Du pâturage. De divers objets agricoles dans leurs rapports avec la chimie.
L'AMÉLIORATION des terrains stériles, par le moyen de l'écobuage, étoit connue des Romains. Virgile en fait mention dans son premier livre des Georgiques. Cet usage est encore fort suivi dans plusieurs parties de l'Angleterre. La théorie de l'écobuage a donné lieu à beaucoup de discussions parmi les savans et les agriculteurs. Elle repose uniquement\setcounter{page}{270} sur la chimie, et je crois pouvoir donner à cet égard des éclaircissemens satisfaisans.
Ainsi que nous l'avons vu ci-devant, tous les sols ont pour base un mélange des terres primitives et d'oxide de fer. Ces terres ont un certain degré d'attraction les unes avec les autres. Pour considérer cette attraction sous son vrai point de vue, il ne faut que faire attention à la composition d'une pierre siliceuse quelconque. Par exemple, le feldspath contient de la silice, de l'alumine, de la chaux, de l'alkali fixe et de l'oxide de fer, réunis par leur attraction chimique. Broyez le feldspath, et vous avez une poudre impalpable comme la glaise. Si vous chauffez cette poudre fortement elle se fond, et donne, en se refroidissant, une pierre semblable à ce qu'elle étoit précédemment. Les parties séparées par une action mécanique, se réunissent de nouveau par une attraction chimique. Si la poudre, au lieu d'être chauffée assez fort pour être fondue, n'éprouve qu'une chaleur moindre, ses parties ne se combinent que superficiellement ensemble, sous forme de grès; et celui-ci étant brisé, ressemble à du sable.
Si l'on compare la faculté du feldspath en poudre d'absorber l'humidité de l'atmosphère,\setcounter{page}{271} avec cette même faculté du feldspath, après l’action de la chaleur, on trouve que la dernière est beaucoup moindre. On observe le même effet, si l’on fait la même épreuve sur les autres pierres siliceuses ou alumineuses.
J’ai trouvé que deux portions de basaltes, l’une et l’autre broyées en poudre impalpable, dont l’une avoit été exposée à un feu violent, et l’autre à la température de l’eau bouillante seulement, gagnoient trèsdifféremment en poids par l’exposition à l’air libre: la première acquit deux grains seulement, tandis que la seconde gagna sept grains.
Lorsqu’on brûle un sol argileux ou terreux, l’effet est semblable: on le rapproche de l’état de sable. Dans la fabrication des briques, on voit un exemple de la chose: si on applique à la langue un morceau de terre à brique sèche, elle y adhère fortement, à cause de sa capacité d’absorber l’eau; mais les briques cuites, n’adhèrent presque pas à la langue.
L’écobuage rend le sol moins compacte, moins tenace, moins rétentif de l’humidité; et quand ce procédé est convenablement appliqué, il peut convertir un sol roide, humide, et par conséquent froid, en un\setcounter{page}{272} sol pulvérulent, sec, et chaud, par conséquent beaucoup plus propre à la végétation.
Les chimistes spéculatifs objectent surtout à l’écobuage, qu’il détruit la matière animale et végétale, c’est-à-dire, l’engrais du sol; mais dans les cas où la contexture des ingrédients terreux de ce même sol est améliorée d’une manière permanente; ce désavantage temporaire est plus que compensé. Lorsque la matière végétale inerte se trouve en excès dans le sol, cette destruction partielle doit être avantageuse, et la matière charbonneuse qui reste dans les cendres, doit être plus utile à la récolte, que la fibre végétale dont elle provient.
J’ai examiné par l’analyse chimique, trois échantillons de cendres de divers sols écobués. L’un étoit envoyé par Mr. Boys de Kent, qui a fait un traité de l’écobuage. Le sol dont cette cendre provenoit étoit crayeux, elle contenoit,
\comment{list}
80 Carbonate de chaux.
11 Gypse.
9 Charbon.
15 Oxyde de fer.
3 Matière saline ( sulfate de potasse, et muriate de magnésie, avec un peu d’alcali végétal ).
\setcounter{page}{273} Le reste étoit de l'alumine et de la silice. Mr. Boys estime qu'un acre rend 2660 bushels de cendres. Leur poids seroit 1729 quintaux, savoir:
\comment{list}
691,60 Carbonate de chaux.
95, 9 Gypse.
129,67 Oxyde de fer.
25,93 Matière saline.
77,80 Charbon.
Dans ce cas, il y a évidemment en production considérable de matière capable de devenir de l'engrais, par cette opération de l'écobuage. Le charbon étoit fort divisé, et en étant exposé sur une grande surface, devoit se convertir peu-à-peu en acide carbonique; et ainsi que je l'ai remarqué, le gypse et l'oxide de fer, paroissent faire un très-bon effet sur les terres lorsqu'ils contiennent un excès de carbonate de chaux.
Le second échantillon provenoit d'un sol voisin de Coleorton en Leicestershire, lequel contenoit seulement quatre pour cent de carbonate de chaux, trois quarts de sable siliceux léger, et environ un quart de glaise. La pièce étoit en gazon avant l'écobuage, et
100 parties des cendres donnoient,
\comment{list}
6 Charbon.
\setcounter{page}{274}3 Muriate de soude, sulfate de potasse et et un peu d'alkali végétal.
9 Oxide de fer.
Le reste étoit les terres.
Le charbon étoit très-divisé, et la présence de l'alkali devoit augmenter sa solubilité.
Le troisième échantillon étoit d'une glaise tenace de Cornouailles. Le sol avoit été mis en culture par l'incinération de la bruyère, dix ans auparavant ; mais comme il avoit été négligé, la fougère y avoit poussé dans beaucoup d'endroits, ce qui le fit écobuer pour la seconde fois. Cent parties de cendres contenoient,
\comment{list}
8 Parties charbon.
2 Matières salines, sur-tout sel commun et un peu d'alkali.
7 Oxide de fer.
2 Carbonate de chaux.
Le reste, de l'alumine et de la silice.
La quantité du charbon étoit plus forte que dans le second échantillon. Je soupçonnois que le sel provenoit de la mer, distante de deux milles seulement. Il y avoit certainement, dans ce sol là, un excès de fibre végétale inerte, et de matière végétale vivante\setcounter{page}{275} inutile. J'ai su que l'amélioration avait été grande. On a rapporté l'effet de l'écobuage à beaucoup de causes obscures; mais je crois qu'on doit l'attribuer entièrement à la diminution de la cohérence des glaises, à la destruction de la matière végétale inutile, et à sa conversion en engrais. Mr. le Dr. Darwin a supposé que la glaise en se torréfiant pouvait absorber de l'atmosphère quelques principes nutritifs, qu'elle donnait ensuite aux plantes; mais les terres sont des oxydes métalliques purs, saturés d'oxygène. La tendance de l'action du feu est de chasser les principes volatils qu'une substance tient en combinaison. Si l'oxyde de fer qui se trouve dans un terrain n'est pas saturé d'oxygène, sa torréfaction tend à l'unir plus fortement avec ce principe; et de là vient que l'action du feu sur les briques leur donne une couleur rouge. L'oxyde de fer étant saturé d'oxygène, a moins d'attraction pour les acides que les autres oxydes, et il est moins facilement dissous dans le sol par les fluides acides: dans cet état, il paroît agir de la même manière que les terres. Un auteur ingénieux, que j'ai cité, suppose que l'oxyde de fer est un poison pour les plantes, lorsqu'il est combiné avec l'acide carbonique,\setcounter{page}{276} et que la torréfaction est utile, en chassant cet acide carbonique ; mais le carbonate de fer n'est point soluble à l'eau ; et j'ai fait croître très-vigoureusement du cresson dans un sol composé de quatre cinquièmes de carbonate de chaux, et un cinquième de carbonate de fer. Celui-ci abonde dans quelques-uns des sols les plus fertiles de l'Angleterre, sur-tout le sol rougeâtre à houblon. Il n'y a aucune raison théorique de supposer que l'acide carbonique, qui est une nourriture essentielle des végétaux, pût dans aucune de ces combinaisons, leur devenir nuisible. On sait aussi que la chaux et la magnésie sont nuisibles aux plantes, lorsque ces substances ne sont pas unies à l'acide carbonique.
Tous les sols qui contiennent trop de fibres végétales mortes, et qui par cette raison perdent entre un tiers et une moitié de leur poids par l'incinération, et tous les terrains dont les terres constituantes sont dans un état de pulvérulence impalpable, comme les glaises et les marnes, tous ces terrains, dis-je, sont améliorés par l'écobuage ; mais dans les sables grossiers, et dans les terres riches qui sont composées de justes proportions ; enfin, dans tous les cas où la texture du sol est suffisamment légère, et la matière\setcounter{page}{277} organisable suffisamment soluble, le procédé de la torréfaction ne sauroit être utile.
Tous les sables siliceux stériles doivent y perdre ; et ici la pratique s'accorde avec la théorie : Mr. Young dans son essai sur les engrais dit que l'écobuage a été reconnu nuissible pour les terrains sablonneux ; et les bons agriculteurs ne l'emploient jamais pour les sables siliceux. Un agriculteur de Mount's-Bay m'a dit avoir écobué, il y a quelques années, un petit champ que dès lors il n'avoit pas pu remettre en bon état. J'examinai cette pièce, l'herbe y étoit foible et rare, et le sol étoit un sable siliceux aride.
L'arrossement est au premier coup-d'œil une opération inverse de la torréfaction; et en général, dans la nature, l'action de l'eau tend à diviser beaucoup les substances ; mais dans l'arrossement des prairies, l'effet dépend de plusieurs causes, les unes mécaniques et les autres chimiques.
L'eau est absolument essentielle à la végétation. Lorsque la terre a été couverte d'eau pendant l'hiver, ou au printems, et que l'humidité a pénétré profondément dans le sol, elle devient une source de nourriture pour les plantes pendant l'été, et prévient les mauvais effets de la sécheresse.
Lorsque les eaux employées à l'irrigation\setcounter{page}{278} ont coulé sur une contrée calcaire, elles sont ordinairement imprégnées de carbonate de chaux, et elles tendent par là à améliorer le sol.
L'eau de rivière contient en général une certaine quantité de matières organiques : cette quantité est beaucoup plus grande après les pluies; elle l'est sur-tout quand la rivière a inondé un pays cultivé.
Même dans le cas où l'on arrose avec de l'eau pure, cette eau produit une diffusion plus égale de la matière nutritive existante dans le sol; et elle contribue à préserver les racines de l'herbe des effets de la gelée.
La gravité spécifique de l'eau à 42° F. est plus considérable qu'elle ne l'est à 32° F., soit au point de congélation : il en résulte que lorsqu'on arrose un pré en hiver, l'eau qui est immédiatement en contact avec l'herbe, est rarement au-dessous de 40°; or cette température n'est point préjudiciable aux organes des plantes.
En mars 1804, j'examinai la température de l'eau d'une prairie arrosée dans le Berkshire. A sept heures du matin, le thermomètre étoit à 29° F. à l'air libre. L'eau étoit gelée au-dessus de l'herbe; et cependant la température du sol au fond de l'eau étoit à 43°.\setcounter{page}{279} Les eaux des étangs qui nourrissent le meilleur poisson sont les plus efficaces pour l’arrose-ment des prés; mais la plupart des avan-tages de l’irrigation peuvent s’obtenir de tou-tes les eaux. Il est reconnu cependant en principe que les eaux qui tiennent du fer en dissolution et qui ont des effets fertili-sans lorsqu’on les applique aux terres cal-caires , sont nuisibles aux terrains qui ne font point d’effervescence avec les acides, et que les eaux calcaires , c’est-à-dire, celles qui déposent par l’ébullition sont celles qui ont le plus d’effet sur les terrains siliceux , et sur ceux qui ne contiennent pas beaucoup de carbonate de chaux.
Les procédés les plus importans pour l’a-mélioration du terrain, sont ceux que nous avons déjà indiqués ; et dans lesquels on modifie les parties constituantes du sol; mais il y a une opération dont la pratique est très-ancienne , et dans laquelle on ne fait qu’exposer à l’air les parties du sol par le procédé du labourage.
Les avantages des labours de jachère ont été fort exagérés. Une jachère d’été peut être quelquefois nécessaire dans des ter-rains surchargés de mauvaises herbes , sur-tout s’ils sont sablonneux et ne peuvent être soumis avantageusement à l’écobuage ; mais\setcounter{page}{280} c'est certainement une opération vicieuse si on la considère comme partie nécessaire d'un système agricole.
Quelques auteurs ont supposé que certains principes fertilisans étoient tirés de l'atmosphère; que ces principes s'épuisoient dans une succession de récoltes, et que le sol , pulvérisé par les labours et laissé en repos attiroit de nouveau ces principes fertilisans pour se les approprier; mais les faits sont contraires à cette supposition. Les terres qui composent un sol quelconque, ne peuvent pas se combiner avec plus d'oxygiène qu'elles n'en ont déjà; aucune de ces terres ne peut s'unir à l'azote ; et celles qui peuvent attirer l'acide carbonique en sont toujours saturées dans les sols où la pratique de la jachère est établie. L'ancienne et vague opinion de l'usage du nitre, et des sels nitreux dans la végétation, a été souvent mise en avant pour justifier la pratique des jachères. Il se produit des sels nitreux pendant l'exposition à l'air des terrains qui contiennent des débris végétaux et animaux; et cette production est sur-tout considérable pendant les chaleurs, mais c'est probablement par la combinaison de l'azote avec l'oxygiène de l'atmosphère que cette opération a lieu, c'est-à-dire, par l'emploi d'un élément qui sans cela auroit formé\setcounter{page}{281} de l'ammoniaque : or, nous avons vu que les composés d'ammoniaque sont beaucoup plus efficaces que les composés nitreux, pour encourager la végétation.
Lorsqu'on enterre de mauvaises herbes dans le sol, elles fournissent par leur décomposition graduelle, une certaine quantité de matières solubles; mais on peut douter si après les procédés d'une jachère complette, il existe dans le sol autant d'engrais utiles qu'il en existoit immédiatement après le premier labour.
Pendant tout le temps que durent les opérations de la jachère, il se forme de l'acide carbonique, par l'action de la matière végétale sur l'oxigène de l'air; et la plus grande partie de cet acide carbonique est perdue pour le sol dans lequel il s'est formé, et se dissipe dans l'atmosphère.
L'action du soleil sur la surface du sol tend à dégager les matières fluides volatiles et gazeuses qui y sont contenues. La chaleur augmente la rapidité de la fermentation; et pendant la jachère d'été, il se produit beaucoup de nourriture pour les plantes, tandis qu'il n'y a point de végétaux utiles pour l'absorber.
Lorsque le terrain n'est pas employé à préparer de la nourriture pour les animaux, il\setcounter{page}{282} devroit l'être à préparer de l'engrais pour les plantes : or ceci est effectué par les récoltes vertes, lesquelles absorbent le carbone de l'acide carbonique de l'atmosphère. Dans la jachère d'été il y a toujours un espace de temps perdu, pendant lequel on pourroit faire croître des plantes, soit pour la nourriture des animaux, soit pour servir d'aliment à la récolte suivante : d'ailleurs la texture du sol n'est pas autant améliorée par l'exposition à l'air que dans la jachère d'hiver, où la force expansive de la glace, la dissolution graduelle de la neige, et les alternatives d'humidité et de sécheresse, tendent à pulvériser le sol et à mélanger ses diverses parties.
Dans la culture en lignes, on conserve la netteté du sol par l'extirpation plus facile des mauvaises herbes. L'engrais est fourni, soit par la récolte verte elle-même, soit par le fumier des bestiaux qu'elle nourrit; et les plantes à larges feuilles sont cultivées en alternant avec celles qui donnent les grains. C'est un grand avantage de la disposition d'un assolement que de pouvoir y employer la totalité des engrais, et que la partie de ceux-ci qui n'est pas propre à une récolte, demeure en provision dans le sol, pour une autre. Ainsi dans l'assolement de Mr. Coke, les\setcounter{page}{283} turneps viennent les premiers et sont fumés avec du fumier frais, lequel fournit immédiatement de la matière soluble en suffisance pour nourrir la récolte, en même temps que la chaleur produite par la fermentation fait germer plus promptement la graine, et croître plus vite la jeune plante. Après les turneps, il sème de l'orge et une graine de pré artificiel. Le terrain étant peu épuisé par les turneps, la récolte de grain profite de la matière soluble du fumier à mesure qu'il se décompose. Le trèfle, le ray-grass et les autres graminées succèdent. Ils ne tirent du sol qu'une petite partie de leur matière organisée, mais probablement ils s'approprient le gypse du fumier, lequel gypse aurait été inutile aux autres productions. Ces plantes absorbent aussi beaucoup de nourriture de l'atmosphère, au moyen de leurs feuilles; et lorsqu'à la fin de la seconde année on rompt le gazon, le sol est enrichi des débris des feuilles et des racines, dont le froissement qui succède profite. À cette époque la partie ligneuse du fumier répandu quatre ans auparavant est en état de dissolution, et donne le phosphate de chaux et les autres parties d'une solution difficile. Dès que la récolte épuisante est faite, on fume de nouveau avec du fumier frais.
\setcounter{page}{284} Mr. Gregg, dont le Département d'Agriculture a publié le beau systême agricole, et qui a eu le mérite d'adopter le premier une méthode semblable à celle de Mr. Coke, en l'appliquant aux terres argileuses, laisse le sol en pré pendant deux ans après l'orge, sème des pois et des fèves sur le gazon rompu, et fait succéder le froment. Quelquefois après le froment, il sème des vesces et de l'orge d'hiver, qui se mangent sur place au printemps, avant de revenir aux turneps. Il paroît que les pois et les fèves sont très-propres à préparer le sol au froment. On les cultive alternativement avec celui-ci, pendant plusieurs années consécutives, dans les sols d'alluvion de Parret, et auprès des dunes de Sussex. Il paroît, d'après l'analyse, que les pois et fèves contiennent un peu d'une matière analogue à l'albumen, mais que l'azote, qui est une partie constituante de cette substance, est fourni par l'atmosphère. La feuille sèche de la fève, donne dans sa combustion, une odeur qui ressemble à celle de la matière animale en décomposition: et en se décomposant dans le sol, cette feuille peut fournir des principes capables d'entrer dans le gluten du froment. Quoique les diverses plantes aient beaucoup d'analogie dans leur composition, cependant\setcounter{page}{285} les différences spécifiques dans les produits de plusieurs d'entr'elles, et les faits que nous avons constatés, prouvent qu'elles tirent du sol des matériaux différens selon les plantes; et bien que les végétaux qui ont peu de feuilles épuisent proportionnellement plus le sol de la matière nutritive commune à toutes les plantes, cependant certaines plantes demandent que le sol soit fourni de certains principes, pour pouvoir réussir. Les fraises et les pommes de terre produisent d'abord abondamment dans les prés rompus; mais en peu d'années leur produit diminue, et elles exigent un sol nouveau. L'organisation de ces plantes est telle qu'elles tendent sans cesse à s'étendre et à s'éloigner de la plante-mère, pour chercher un sol différent. On remarque que le sol se lasse du trèfle, par exemple (c'est l'expression populaire) et une des raisons probables de ce fait, l'épuisement du gypse dans le sol a été indiquée ci-dessus.
Certains champignons fournissent un exemple remarquable de la faculté des végétaux d'épuiser le sol de quelques principes nécessaires à leur croissance. On dit que les mousserons ne viennent jamais à la même place deux années de suite; et le Dr. Wollaston a prouvé qu'ils s'étendaient en cercle d'où ils avaient\setcounter{page}{286} Laston a attribué le phénomène des cercles magiques, dans les prairies, à la présence d'un certain fungus, qui épuise chaque année le sol sur lequel il végète, de l'aliment particulier qui lui est propre. La plante provenant de la semence des plantes de l'année précédente, ne peut végéter sur le même terrain, et la partie qui est en dedans du cercle étant déjà épuisée, les champignons ne peuvent prospérer qu'en dehors de ce cercle, lequel s'étend, par conséquent toutes les années, tandis que la décomposition des champignons qui ont péri, fournit de la nourriture aux graminées, ensorte que celles-ci croissent avec vigueur, et sont d'un vert foncé.
Lorsqu'on nourrit du bétail sur un terrain qui ne profite pas de son fumier, le terrain s'épuise toujours : cela est sur-tout vrai des chevaux de charroi : ils pâturent pendant la nuit et perdent la plus grande partie de leur fumier durant leur travail de la journée.
L'exportation des grains d'un pays, à moins qu'elle ne soit compensée par l'introduction des matières qui peuvent se convertir en engrais, doit finalement tendre à épuiser le sol. Quelques-unes des parties aujourd'hui stériles de l'Afrique septentrionale et de l'Asie mimeure\setcounter{page}{287} étoient autrefois fertiles. La Sicile étoit le grenier de l'Italie; et la quantité de grains que les Romains en ont tiré jadis, est probablement la cause principale de sa stérilité actuelle. Le système présent du commerce de l'Angleterre avec cette isle, doit avoir pour effet de lui fournir des substances, qui, par leur usage et leur décomposition, rendront de la fertilité aux terres. Les grains, le sucre, les suifs, les huiles, les peaux, les fourrures, les vins, les soies, le coton, sont importés en Sicile, tandis que la mer lui fournit du poisson. Parmi les nombreux objets d'exportation de l'Angleterre, les laines, les toiles, et les cuirs, sont presque les seules substances qui contiennent des matières nutritives extraites du sol.
Dans tous les assolemens, il convient que chaque partie du sol soit rendue aussi utile qu'il se peut aux diverses plantes; mais la profondeur à laquelle on laboure doit dépendre de la nature du sol inférieur et de la couche supérieure. Dans les bonnes terres argileuses, il est difficile de labourer trop profond; et dans les sols sablonneux, à moins que la couche inférieure ne contienne quelque principe nuisible à la végétation, il est également avantageux de labourer profond.\setcounter{page}{288} Lorsque les racines sont profondément enterrées, elles sont moins sujettes à souffrir de la sécheresse et de l'humidité excessives. Les radicules s'étendent plus librement dans toutes les directions; et l'espace qui fournit à la nourriture est plus considérable que lorsque les plantes sont peu enterrées.
On a beaucoup discuté les avantages des prés durables: les circonstances de situation et de climat, doivent décider la question. Si l'on peut arroser, les prés rendent beaucoup avec peu de travail; et là où il pleut souvent, on trouve un avantage semblable. Dans le voisinage d'une grande ville où le foin est toujours demandé, et où on peut se procurer aisément des engrais, on retrouve en augmentation de la récolte les frais de l'application du fumier; mais en général, ce n'est pas un système recommandable, que celui de fumer les prés. Le Dr Coventry observe avec raison que, par ce procédé, il y a une plus grande déperdition de l'engrais, que lorsqu'il est enterré pour les récoltes semées. La perte qui a lieu, dans ce cas, par l'exposition au soleil et par les vents, fournit un nouveau motif pour l'emploi des fumiers frais sur les prés, de préférence aux fumiers pourris.
\setcounter{page}{289} On a donné peu d'attention au choix des herbes les plus propres à faire des prés naturels permanens. La circonstance principale qui détermine la valeur d'une plante de prés, est la quantité de matière nutritive que fournit la récolte entière. Mais le temps de la récolte et la durée de cette plante, sont des circonstances très-importantes aussi. Une plante qui fournit de la nourriture verte toute l'année, peut être plus précieuse que celle qui ne donneroit son fourrage qu'en été, lors même que la quantité de matière nutritive fournie par celle-ci seroit beaucoup plus grande.
Les plantes stolonifères, comme les agrostis, fournissent du pâturage toute l'année; et la sève concrète qui se trouve en provision dans leurs nœuds, en fait une bonne nourriture d'hiver. J'ai vu couper quatre yards carrés de fiorin, à la fin de janvier, dans un pré de cette plante, dont le sol est une glaise froide, et appartient à la comtesse de Hardwicke. Cet espace fournit vingt-huit livres de fourrage \footnote{Les trente-six pieds anglais de superficie répondent à-peu-près à trente-un pieds de France. Les vingt-huit livres anglaises en font à-peu-près vingt-six, poids de marc. Ce seroit donc sur la pose (de 25600 pieds de France) deux cent quatorze quintaux de fourrage vert, lesquels supposés réduits par la dessiccation selon la formule ordinaire, donneroient cinquante-trois quintaux et demi de foin sec, par pose, soit le double d'une belle récolte de luzerne. Cependant la réduction supposée des trois quarts par dessiccation, est probablement beaucoup trop forte. Les rapports curieux que nous avons présentés depuis quatre ans dans la Bibl. Brit. sur cette plante, ne sont donc pas exagérés. Il paroît par l'article inséré au n°. de juillet 1814, que les produits du fiorin sont encore plus grands en Irlande qu'en Angleterre. (R)}, duquel mille parties.\setcounter{page}{290} en donnèrent soixante-quatre de matière nutritive, composée d'environ un sixième de sucre et cinq sixièmes de mucilage, avec un peu de matière extractive. Dans une autre expérience, quatre yards carrés ont donné vingt-sept livres de fourrage vert. Ce fiorin est inférieur en qualité à celui dont j'ai parlé ci-devant, et qui avait été coupé en décembre, dans un terrain plus riche, en Middlesex, chez Sir Joseph Banks.
Pour être dans sa perfection, le fiorin demande un terrain et un climat humides; il donne d'abondants produits dans les terres glaises froides où les autres herbes ne réussissent pas. Dans les sables légers et dans les situations sèches, son produit est très-inférieur en quantité et qualité.
\setcounter{page}{291} Les graminées qui fournissent le plus de matières nutritives dès le printems, sont le pâturin des prés, et le vulpin des prés; mais leur produit au moment de la fleur et de la maturité de la graine, est inférieur à celui de beaucoup d'autres graminées : leur repousse est néanmoins abondante. D'après les expériences du duc de Bedford, la grande festuque des prés est celle de toutes les graminées qui donne la plus grande quantité de substance nutritive lorsqu'on la coupe au moment de la fleur; et le fléau est la plante qui en donne le plus si on la coupe quand la graine est mûre. L'herbe qui produit le plus de regain, de toutes celles que le duc de Bedford a éprouvée, est le pâturin maritime ( sea-meadow-grass ). La nature a mélangé, dans les pâturages ou prés naturels, diverses herbes dont les produits sont différens, selon les saisons. Il faut imiter ce mélange lorsqu'on fait des prés de gazon ; et en ayant égard à la nature du sol, on feroit peut-être de meilleurs prés que ceux qui se sont formés naturellement : il faudroit faire attention à la faculté de chaque herbe de donner abondamment au printems, en été, en automne, ou en hiver.
\setcounter{page}{292} Toutes les mauvaises herbes doivent être extirpées avant que la graine soit mûre, et cela dans les champs comme dans les prés. Si on les souffre dans les haies, il faut les couper avant la maturité des semences, et les mettre en tas pour les convertir en engrais. De cette manière, elles fourniront de la matière nutritive par leur décomposition, et la dispersion de leurs semences ne les multipliera pas. L'agriculteur qui permet aux mauvaises herbes de rester sur plante jusqu'à la maturité de leur graine, nuit, non seulement à lui-même, mais à ses voisins, à cause de la dispersion de la graine par les vents: il suffit de quelques chardons pour empoisonner toute une ferme, et le duvet qui sert d'ailes à leurs graines les transporte à de grandes distances. La nature a pris tant de précautions pour perpétuer l'espèce de chaque plante, qu'il faut de grands soins pour se garantir de celles qui nuisent à l'agriculteur. Les semences qui ne sont pas en contact avec l'air, peuvent rester pendant plusieurs années enfouies dans le sol, puis germer quand les circonstances leur deviennent favorables \footnote{L'apparition des plantes dans les endroits, où il n'y en avoit pas eu auparavant de même espèce, peut s'expliquer aisément par cette circonstance. D'autres encore peuvent produire le même phénomène. Beaucoup de graines sont entraînées d'une île à l'autre par les courants de la mer, et sont défendues par leurs dures enveloppes, de l'action immédiate de l'eau. On trouve souvent sur les côtes d'Angleterre, des graines des îles d'Amérique, lesquelles germent sans difficulté, d'autres graines sont transportées dans l'estomac des oiseaux, sans germer: le germe ne se développe qu'après qu'elles sont déposées dans les excréments. Les semences légères des lichens volent probablement dans toutes les parties de l'atmosphère, et abondent sur la surface de la mer. (A)}, et les semences ailées peuvent\setcounter{page}{293} être transportées par les vents à d'immenses distances. Le fleabane du Canada a été trouvé en Europe, il n'y a pas un grand nombre d'années, et Linné suppose que sa graine ailée peut avoir été apportée dans les airs.
Il y a divers avantages à nourrir les animaux dans les écuries avec du fourrage vert. Les plantes sont moins tourmentées si on les coupe à la faux, que lorsqu'on les fait rompre par la dent des bestiaux. Ceux-ci en écrasent d'ailleurs beaucoup en pâturant. Au ratelier, ils mangent toutes les plantes pêle-mêle sans les choisir, et tout se consomme. La préférence ou la répugnance des animaux pour une plante quelconque, ne prouve rien relativement à sa faculté nutritive:\setcounter{page}{294} le bétail commence par refuser les gâteaux de graine de lin, l'une des substances les plus nutritives qu'on puisse lui donner\footnote{Les observations suivantes sur diverses graminées sont de Mr. George Sinclair, jardinier du duc de Bedford. Le Lolium perenne (ray-grass), l'ivraye vivace. Les moutons préfèrent cette herbe à toute autre, dans les premiers temps de sa pousse; mais lorsque la semence approche de la maturité, ils ne touchent plus à cette plante. Une partie du parc de Woburn fut mise en pré, savoir, la moitié de la pièce en ray-grass et trèfle-blanc, et l'autre moitié en pied-de-poule, et trèfle ordinaire. Depuis le printems jusqu'au solstice, les moutons se tenoient presque constamment sur le ray-grass; mais après cela ils l'abandonnoient, et pâturoient avec une égale constance la partie en pied-de-poule, pendant le reste de la saison.
Dactylis-glomerata (Cocks-foot) pied-de-poule. Les bœufs, les chevaux, les moutons, mangent cette herbe avidement. Les bœufs continuent à manger les tiges et les fleurs, depuis le moment de la floraison jusqu'à celui de la maturité de la graine. Cela a été démontré avec évidence dans la pièce dont il s'agit. Les bœufs se tenoient, en général, au pied-de-poule et au trèfle ordinaire, et les moutons au ray-grass et trèfle-blanc. On lit dans les Amœnitates academiæ des disciples de Linné, que les bœufs refusent le pied-de-poule; mais le fait cité ci-dessus est en opposition à cette assertion.
Alopecurus pratensis. (Meadow fox tail), vulpin des prés. Les moutons et les chevaux paroissent aimer cette herbe mieux que ne le font les bœufs. Elle se plaît sur un sol qui n'est ni sec ni humide, et rend beaucoup. Dans la prairie arrosée de Priestiley, elle fait une grande partie de l'herbe de cette excellente pièce. Le vulpin y occupe toujours exclusivement le haut des à-dos, et s'étend à environ six pieds des deux côtés des rigoles. L'espace au-dessous est garni de pied-de-poule; de festuca pratensis, de festuca duriuscula, d'agrostis stolonifera, d'agrostis palustris, de flouve odorante et de pâturin à tige rude, avec un mélange de quelques autres herbes.
Phleum-pratense, (meadow cat's tail). Fléau des prés. Cette herbe est mangée avidement par les bœufs et les chevaux. Le Dr. Pulteney dit que les moutons ne l'aiment pas. Mais dans les endroits où le fléau des prés abonde, il ne paroît point que les brebis le rejettent, ou l'évitent. On a remarqué que les moutons négligent le *phleum-nodosum*, le *fleum-alpinum*, la poa-fertilis, et la poa-compressa, pour pâturer le fléau des prés quand ils en ont le choix. Il paroît que pour la perfection de cette plante il faut un terrain riche, argileux et profond.
Agrostis stolonifera, Fiorin. On lit dans les Amanitates Academica que les chevaux, les moutons et les bœufs mangent cette herbe avec empressement. On a essayé à la ferme de Maulden, appartenant au duc de Bedford, de placer devant les chevaux du fiorin akernativement avec de l'autre foin, sans qu'ils manifestassent aucune préférence; mais il paroît pleinement prouvé par les expérience du Dr. Richardson, que les chevaux et les bêtes à cornes préfèrent le fiorin vert au foin. On a également des preuves satisfaisantes des grands produits du fiorin en Angleterre. Lady Hardwicke a rendu compte d'un essai qu'elle a fait avec cette herbe. Elle a nourri pendant quinze jours vingt-trois vaches, un poulain et plusieurs cochons avec le produit d'un seul acre.
Poa-trivialis. Pâturin commun à tiges rudes. Les bœufs, les chevaux et les moutons mangent cette herbe avec avidité. Les lièvres la mangent de même; mais ils donnent une préférence décidée au pâturin des prés qui lui ressemble beaucoup.
Poa pratensis. Pâturin des prés à tige unie. Les bœufs et les chevaux mangent cette herbe comme les autres; mais les moutons lui préfèrent la festuque dure et la festuque des troupeaux, plantes qui se plaisent dans le même sol. Cette espèce de graminée épuise le terrain plus qu'aucune autre peut-être. Les racines étant nombreuses, et fort traçantes, se rendent, en deux ou trois ans complètement maîtresses du sol, et y forment une masse de fibres entrelacées. Le produit diminue alors graduellement. Le pâturin des prés est commun dans quelques prés, sur les chaussées sèches, et croît même sur les murailles.
Cynosurus cristatus, Cynosure à crête. Les brebis de South-down et les daims paroissent aimer beaucoup cette herbe. Elle fait la plus grande partie du gazon dans quelques endroits du parc de Woburn où ces animaux paissent de préférence; tandis qu’ils négligent celles où dominent l’agrostis capillaris, l’agrostis pumilis, la festuca ovina, la festuca duriuscula, et la festuca cambrica. En revanche, la race des moutons de Galles broute ces herbes de préférence et néglige le Cynosurus cristatus, le Lolium perenne et la Poa trivialis. Agrostis vulgaris ou capillaris. C’est une herbe très-commune sur les mauvais terrains sablonneux. Le gros bétail ne l’aime guère, s’il peut manger autre chose; mais comme nous venons de le voir, les moutons de Galles la recherchent. C’est une chose remarquable que ces moutons, quoique nés à Woburn et élevés dans le parc, préfèrent les herbes qui croissent naturellement sur les montagnes de Galles, lors même qu’ils ont à leur portée les meilleures graminées: il sembleroit que cette préférence tient à autre chose qu’à l’habitude. Festuca ovina. Festuque des troupeaux. Tous les bestiaux aiment cette herbe; mais il paroît par l’expérience faite sur les sols argileux, qu’elle n’y tient pas long-temps, et que les graminées plus vigoureuses l’étouf-fent. Sur les terrains secs et peu profonds qui ne peuvent pas produire les espèces les plus fortes, cette festuque des moutons devroit faire l'herbe principale, ou plutôt la seule; car dans son état naturel on la trouve peu mélangée avec d'autres plantes.
Festuca duriuscula. Festuque dure. C'est une des meilleures herbes peu élevées. Tous les bestiaux l'aiment. Les lièvres la broûtent jusqu'à la racine. On la trouve dans la plupart des bons prés et des bons pâturages.
Festuca pratensis. Festuque des prés. Cette plante se trouve presque toujours dans les bonnes prairies. Les bœufs l'aiment singulièrement. Les moutons et les cheveaux la recherchent. Elle paroît réussir le mieux possible lorsqu'elle est associée à la festuque dure et au pâturin commun.
Avena elatior. Grand fromental. Cette graminée est très-productive, et se rencontre fréquemment dans les prés et pâturages; mais le bétail ne l'aime pas, les chevaux sur-tout en font peu de cas. Elle donne peu de substance nutritive. Le sol qui paroît lui convenir le mieux est la glaise tenace.
Avena flavescens. Fromental jaunâtre. Cette plante paroît aimer les terres sèches, et convenir aussi bien aux bœufs et aux moutons que les autres herbes qui croissent avec: l'application d'un engrais calcaire en double le produit, ou peu s'en faut.
Holcus lanatus, (Meadow softgrass). Houlque laineuse.}
C'est une plante très-commune, qui croît sur tous les terrains, riches ou stériles. Elle donne beaucoup de graine que les vents dispersent aisément. Les bestiaux quels qu'ils soient, ne l'aiment guères soit en pâturage soit en foin : celui-ci est velu, et spongieux.
Anthoxantum odoratum. (Sweet scented vernal grass). Flouve odorante. Les chevaux, les bœufs et les moutons mangent cette herbe; mais s'ils ont à choisir le vulpin des prés, le trèfle-blanc, le pied-de-poule, le pâturin des prés, ils n'y touchent pas. Mr. Grand, de Leighton, établit en prairie la moitié d'une pièce avec de la flouve odorante, mêlée de trèfle-blanc, et l'autre moitié avec du vulpin des prés, mêlé de trèfle ordinaire. Les moutons laissoient la première portion, et broûtoient continuellement la dernière. L'auteur a vu les deux parties dans le moment où la récolte étoit sur pied, et rien n'étoit plus frappant. On avoit essayé de semer une même quantité de trèfle-blanc avec chacune des deux graminées; mais comme la flouve s'élève peu, le trèfle qui lui étoit associé fut plus beau que l'autre. (A)}.
\setcounter{page}{295}Si l'on compose artificiellement une nourriture pour le bétail, il faut qu'elle se rapproche, autant qu'il est possible, de l'état de la nourriture naturelle. Ainsi, lorsqu'on\setcounter{page}{296} donne du sucre aux bestiaux, il faut y mêler une matière fibreuse sèche, comme de la paille hâchée ou du foin, pour que les fonctions de l'estomac et des intestins se fassent comme à l'ordinaire.
\setcounter{page}{297}Lorsqu’on lave les brebis, il faut éviter de se servir des eaux qui contiennent du carbonate de chaux; car cette substance décompose le suint, qui est un savon animal, et sert de défense naturelle à la laine. La\setcounter{page}{298} laine lavée fréquemment dans l'eau calcaire, devient rude et cassante. Les laines d'Espagne et de Saxe, qui sont les plus fines, sont les plus abondantes en suint. Vauquelin a analysé divers suints: il a trouvé qu'ils étoient composés principalement d'une matière huileuse
\setcounter{page}{299} et de potasse, soit d'un savon, avec excès d'huile. Il y a aussi trouvé une quantité notable d'acétate de potasse; un peu de carbonate de potasse, de muriate de potasse, et une matière animale odorante.

\setcounter{page}{109}
\chapter{AGRICULTURE}
\section{SKETCH OF THE PRESENT STATE, etc. Esquisse de l'état présent de l'agriculture dans le comté de Berwick, par le Rév. James THOMSON, ministre d'Eccles, dans ce pays. ( Annals of Philosophy, etc. Par Th. THOMSOM, No. IV.) \large{(Second extrait. Voy. p. 73.)}}
JACHÈRES. On traite les jachères dans le comté de Berwick, de la manière suivante. Aussitôt qu'il est possible après la moisson, on donne un premier labour. On a soin, dans les terres fortes, de faire les billons assez élevés, pour que l'eau puisse s'écouler dans les sillons intermédiaires, qu'on ouvre à la charrue et qu'on nettoie à la bêche, afin que le superflu liquide puisse s'écouler sans obstacle dans les fossés qui environnent l'enclos. Ce premier travail est considéré comme très-utile. Il enterre le chaume et le convertit en terreau ; il laisse le champ dans un état de dessication aussi parfaite qu'il est possible\setcounter{page}{110}, et les gelées de l'hiver achèvent d'amenuiser le sol. On fait en général ce labour aussi profond qu'on le peut, parce qu'à raison de l'état du sol, la charrue peut alors le pénétrer aisément jusques à la profondeur requise. Il facilite aussi les opérations suivantes du même genre.
On donne rarement le second labour avant l'époque où l'on sème l'avoine et quelquefois aussi l'orge. Si le sol est garni de mauvaises plantes, telles que le chiendent, et si la saison est sèche, alors on herse soigneusement, on recueille à la main le chiendent, qu'on entasse en monceau. On enlève ensuite ceux-ci, et on les mêle avec de la chaux, ou avec de la chaux et du fumier d'étable, et on en forme ainsi un excellent compost. Il est fort essentiel de ramasser le chiendent le plutôt possible avant qu'il commence à végéter, parce qu'on peut l'arracher plus aisément à l'époque où le sol vient d'être amenuisé par les gelées. Mais si la saison est humide, on ne peut pas herser suffisamment ni enlever le chiendent, jusqu'au troisième labour.
On fait celui-ci lorsqu'on suppose que toutes les mauvaises plantes ont poussé, mais n'ont pas pu encore végéter assez pour appauvrir le sol. Alors on herse bien complété\setcounter{page}{111} tement, et on ramasse de nouveau le chiendent.
Le quatrième et cinquième labour se donnent à des intervalles réglés par l'apparition des mauvaises plantes qui peuvent se montrer encore. On fait succéder la herse à chaque labour, et on ramasse avec soin le chiendent, s'il en reste. Mais si on a apporté aux opérations du second et troisième labour le soin convenable, il doit rester bien peu de plantes nuisibles excepté les annuelles. On donne fréquemment plus de cinq labours si le sol n'a pas été suffisamment nettoyé et pulvérisé.
Il faut remarquer, que lorsque le champ ne se trouve pas encore assez amenuisé, l'usage du rouleau peut devenir nécessaire, parce que la crête des sillons cache quelquefois beaucoup de plantes. C'est toujours avant le dernier labour qu'on répand l'engrais, c'est-à-dire, la chaux, le compost, ou le fumier pur, selon les circonstances; quelquefois on étend ce dernier avant les deux derniers labours. Nous ne disons rien du mode particulier de chaque labour, parce que la pratique n'en est pas uniforme; mais nous pouvons ajouter qu'il est d'usage de labourer en travers des billons, au moins dans deux des labours.\setcounter{page}{112} Sur cette jachère ainsi préparée, on sème le froment. Si la saison est favorable, c'est-à-dire, pas trop humide, et si le sol est très-léger, on préfère semer au semoir, ou avec la machine ordinaire, ou en ouvrant dès sillons avec la charrue sans coutre, et en y jetant à la volée le grain qui se ramasse naturellement dans les sillons, où la herse l'enterre; on le voit ensuite lever en lignes régulières.
\subsection{EPOQUE DES SEMAILLES.}
On sème le froment hiverné depuis le commencement ou le milieu de septembre jusques à la St. Martin, selon la saison, et les convenances du fermier. Après la jachère, la dernière moitié de septembre est considérée comme le meilleur temps des semailles lorsque la saison est favorable, après les pommes de terre, aussitôt qu'elles sont recueillies; après les turneps de Suède, à la St. Martin. En 1812 on a beaucoup semé de froment en décembre. Après les turneps ordinaires; comme le sol est nettoyé par degrés, on sème en janvier, février, et mars, selon que le temps le permet. On l'appelle alors blé de printems; et on a trouvé, d'après l'expérience de fermiers judicieux, que malgré l'identité des deux grains, qui sont l'un et l'autre, le triticum hybernum, celui qu'on a semé au printems,\setcounter{page}{113} lorsqu'on l'emploie ensuite pour semence, mûrit quinze jours plus tôt que le produit du blé semé en automne, ou avant l'hiver.
\subsection{Esquisse de la culture des turneps en Berwickshire.}
L'introduction des turneps est l'un des perfectionnemens les plus importans de l'agriculture moderne. Aucun fermier intelligent ne niera que la culture des turneps ne fût connue et pratiquée dans le comté de Norfolk long-temps avant celui de Berwick. Quelques-uns attribuent l'introduction des turneps dans cette dernière contrée à lord Marchmont; d'autres à Mr J. Hunter, qui a été pendant quelque temps l'homme d'affaires de ce seigneur; d'autres à Mr Pringle; mais on s'accorde à dire que Mr W. Dawson actuellement vivant, a eu le principal mérite d'introduire la culture des turneps au semoir dans notre contrée, et de l'amener au degré de perfection où nous la voyons. Convenir de ce fait n'est pas accorder un foible mérite à Mr Dawson, car entre l'ancienne culture après semis à la volée, et la culture actuelle au semoir, il y a un grand déploiement de génie. Le premier genre de culture ne pouvoit réussir en grand sur un sol argiK 3\setcounter{page}{114} lieux parce qu'il ne permettoit pas l'emploi du procédé le plus simple pour le nettoyer des plantes nuisibles; mais la culture au semoir, tandis qu'elle assure une récolte au moins aussi abondante, donne le moyen de nettoyer et d'amenuiser le sol, d'introduire l'instrument le plus efficace et qui travaille avec le plus d'économie, la charrue; et de donner à celui qui conduit la houe une dextérité, une régularité et une prestesse vraiment surprenantes. On ne s'étonnera donc plus si la culture à la volée est tout-à-fait abandonnée, et si celle au semoir est solidement établie non-seulement dans le Berwickshire, mais dans tous les comtés voisins.
\subsection{PRÉPARATION DU CHAMP}
Le sol qu'on destine aux turneps pour la saison suivante est toujours labouré; s'il est possible, après la moisson, afin que les gelées de l'hiver puissent achever de l'amenuiser. On lui donne le second labour au printemps ou au commencement de l'été, après qu'on a semé l'avoine et l'orge et qu'on a planté les pommes de terre. Alors on le herse et on ramasse les mauvaises herbes. On lui donne un troisième, un quatrième, et un cinquième labour, et on le herse autant de fois, dans les mois de mai et de juin, et on ramasse chaque fois les herbes déracinées; on passe aussi dans\setcounter{page}{115} L'occasion un rouleau pesant, de bois, pour pulvériser davantage le sol : ainsi on le traite précisément comme une jachère.
\subsection{BILLONS}
La manière la plus ordinaire de préparer les billons pour les turneps est de les laisser plats ; mais dans un sol argileux tel que celui de la Merse, les sillons relevés sont préférables.
\subsection{SILLONS}
Le sol étant pulvérisé et débarrassé de toutes les mauvaises herbes, on trace à la charrue sur toute la surface, des sillons distans de vingt-six à trente pouces l'un de l'autre. L'expérience a prouvé que cet intervalle est le plus convenable pour que les plantes trouvent à se nourrir sans se nuire réciproquement, et pour que la charrue passe aisément entre les lignes. Lorsque la surface est applanie, on sillonne le champ obliquement ; c'est-à-dire, que les sillons forment des angles aigus avec les premiers billons. Lorsque ceux-ci sont convexes, on dirige quelquefois les sillons à angles droits avec eux ; mais, plus ordinairement, on trace les sillons dans la même direction longitudinale que les billons.
\subsection{ENGRAIS}
Lorsque les sillons sont ouverts on porte le fumier au champ et on le répand dans ces mêmes sillons. On préfère celui qui est le plus complétement pourri.\setcounter{page}{116} On laboure ensuite de manière à couvrir le fumier, sur lequel on ouvre de suite de nouveaux sillons.
\subsection{SEMAILLE}
On sème les turneps avec un semoir qui distribue la semence en ligne droite sur la crête du sillon. Une petite herse et un petit rouleau suivent le semoir pour recouvrir la graine et presser légèrement la terre sur elle. On sème assez épais, afin d'assurer une sortie suffisante. Quelques personnes emploient des semoirs qui sèment deux, trois, etc. jusqu'à cinq lignes à-la-fois; mais à moins que la surface des sillons ne soit parfaitement nivelée, ces machines ne font pas l'ouvrage aussi correctement que le semoir à un seul sillon.
\subsection{ESPÈCES}
Les variétés de turneps qu'on emploie sont le plus ordinairement la jaune, et la blanche ou globe; la racine de disette ou rutabaga, appelée quelquefois turnep de Suède, plante que les botanistes classent dans un genre différent, ce qui importe peu aux fermiers, pourvu qu'ils puissent l'employer aux mêmes usages.
\subsection{EPOQUE DE LA SEMAILLE}
Le rutabaga étant plus long à mûrir que le turnep ordinaire, est ordinairement semé avant la fin de mai, lorsque la saison est favorable. On sème ensuite le turnep jaune; le meilleur temps pour\setcounter{page}{117} semer le turnep globe est depuis le milieu de juin au milieu de juillet. Mais ces époques peuvent varier un peu selon les circonstances; et les fermiers qui sèment des champs immenses de turneps doivent s'y prendre plus tôt que ceux dont la culture est moins étendue.
\subsection{PREMIÈRE CULTURE À LA HOUE.}
Si le temps est chaud, et le sol ni trop sec ni trop humide, la graine lève en peu de jours. Dès que les plants sont assez forts pour recevoir une première culture, on fait passer entre les sillons la houe pour déraciner les mauvaises plantes, et pulvériser le sol entre les lignes. Ensuite on met à l'ouvrage les bécheurs pour éclaircir les turneps, arracher les herbes restantes, et rechausser les turneps dans l'espace étroit où la houe n'a pas touché. On donne au fer des houes une longueur telle qu'elle puisse servir de mesure pour l'intervalle à laisser entre une plante et l'autre, car il faut isoler chacune d'elles. Cet ouvrage se fait ordinairement par des femmes et des enfants, accompagnés d'un surveillant ou du fermier lui-même pour diriger le travail. On remarque avec surprise, la dextérité, la précision, et la promptitude d'expédition de ces travailleurs. Ils poussent la houe en l'éloignant d'eux en travers du sillon, en détruisant tous\setcounter{page}{118} les plants inutiles que la houe touche: ils la ramènent, en produisant le même effet, en laissant rarement plus d'une plante sur une intervalle de six à huit pouces. Ils vont ainsi, en poussant et ramenant tour-à-tour la houe, avec la plus grande agilité. On croiroit à les voir travailler, qu'ils vont tout détruire; et lorsque vous regardez au sol qu'ils laissent derrière eux, vous y trouvez chaque plante laissée seule à la distance convenable, couchée sur le côté, et attachée au sol quelquefois par une seule fibre. On les croiroit condamnées à périr; mais en peu de jours elles se relèvent et deviennent très-vigoureuses.
\subsection{SECONDE CULTURE.} Après la première culture, les plantes font des progrès rapides. Lorsqu'elles ont atteint un développement suffisant pour recevoir une seconde culture, on recommence. Les travailleurs ont à réparer les défauts de la première opération; à réduire à un seul les plants partout où on les a laissé doubles ou triples; à détruire les mauvaises herbes dans les intervalles, et de chaque côté; ensuite on ramène à la charrue, la terre vers les racines. Les plantes laissées alors à l'influence de la chaleur et de l'humidité déploient bientôt leurs larges feuilles, et montrent à l'œil la plus riche verdure.
\subsection{EMPLOI DE LA RÉCOLTE.} Lorsque le foin\setcounter{page}{119} en vert commence à manquer après la moisson, le fermier a recours à ses turneps semés les premiers. Pour semer le froment avant l'hiver, quelques-uns ramassent le rutabaga avant la St. Martin, coupent les feuilles et les racines, les emportent et les mettent à l'abri pour les garder, jusqu'à ce que la récolte des turneps soit consommée. Quelques-uns aussi, et dans le même but, ramassent une partie de leurs turneps ordinaires. Ceux-ci peuvent se conserver deux mois et le rutabaga jusques à Pentecôte. Ainsi, au moyen de ces deux plantes, le fermier est abondamment fourni de nourriture verte et succulente pour ses bêtes à cornes et ses chevaux, depuis la St. Martin jusqu'à Pentecôte.
On laisse fréquemment manger sur place les turneps aux brebis, à tant par acre. On peut procéder ainsi, sans danger pour ces animaux, dans un sol argileux immédiatement après la moisson, ou au printems; et dans les terrains légers et secs, pendant tout l'hiver. Le prix par acre est de 5 à 8 liv. st. à distance des villes; mais ce prix est plus élevé là où on peut vendre le lait avec avantage. Les consommateurs ont à s'accoutumer au goût que ce genre de nourriture communique au lait; mais quand l'habitude en est\setcounter{page}{120} 
prise, on ne s’en aperçoit plus. Ceux qui ne peuvent pas s’accoutumer le font aisément disparoître en dissolvant un peu de salpêtre dans le lait, tel qu’on vient de le traire, et encore chaud.
\subsection{EXTRACTION DES TURNEPS EN HIVER SUR UN SOL GRAS}
Il y a beaucoup de difficulté à extraire et emporter les turneps sur un sol gras sans le détériorer. Car si on le fait parcourir par les chariots lorsqu’il est humide, sa surface devient comme du mortier; et si on le laisse pendant un certain temps dans cet état, quoiqu’on y sème dans la saison suivante, on n’aura pas une demi-récolte. C’est cette difficulté qui établit la distinction entre un sol à turneps, et un sol argileux. Sur le premier on peut transporter les turneps tout l’hiver, sans difficulté, et sans lui nuire.
Il n’est pas douteux cependant, que les turneps qui ont crû dans un sol argileux sont supérieurs en qualité et dans l’usage à ceux que produit un sol graveleux. Ils sont moins spongieux, plus denses, et plus gros en général, que ces derniers. Les bouchers préfèrent aussi les bêtes nourries des turneps qui ont crû sur un sol argileux; il seroit donc à desirer qu’on trouvât un moyen mé\setcounter{page}{121} canique, d'éviter l'inconvénient qui vient d'être signalé. On a proposé plusieurs procédés. Le plus simple, et celui qui se présente le premier, est de faire des billons élevés. D'autres non-seulement les font tels, mais ils tracent les sillons à angles droits de la direction des billons, afin que la pluie s'écoule bien facilement dans les profondes raies qui séparent les billons. D'autres enfin, lorsqu'il ne gèle pas, ramassent autant de turneps qu'il leur en faudra, calcul fait, pour quelques semaines ; ils coupent les feuilles, et mettent les turneps en petits tas, qu'ils recouvrent des feuilles pour les mettre à l'abri de la gelée. Ils attendent ainsi l'époque où cette même gelée consolidera assez le terrain pour que le char ne le tourmente pas. On peut conserver ainsi les turneps pendant deux mois. Il est fort important de labourer aussitôt qu'il est possible après l'enlèvement des turneps, si le sol le permet, c'est-à-dire, s'il n'est pas trop humide. C'est ce qui a lieu souvent en hiver, lorsque les billons ont été tenus assez élevés pour que l'eau ne séjournât nulle part.
\setcounter{page}{249}
\section{On The Varieties Of Fiorin. Des Variétés du Fiorin.}
L'EXTRAIT suivant d'une lettre du Dr. Richardson est tiré du n°. de mai 1814 de Farmer's Magazine.
"Les stolons que vous m'avez envoyés à examiner ne sont pas autre chose que le véritable fiorin. Comme ceux de l'Aïra aquatica, sont fort semblables, ce pourroit être cette plante; mais comme ils sont le résultat de la culture, ce doit être le fiorin. Les intervalles entre les articulations ne prouvent rien parce qu'ils varient beaucoup. Vous ne m'avez pas envoyé les panicules. Le diamètre du chaume de l'aïra est plus grand, et le foin devient par conséquent plus flasque en séchant; mais cette plante est décidément aquatique; et je n'ai jamais pu la faire réussir par culture.
" La question des variétés du fiorin est curieuse en elle-même; mais ses difficultés ne frappent que ceux qui ont la manie Linnéenne, et qui ne veulent pas voir l'inutilité de cette classification malheureuse et forcée."
\setcounter{page}{250}
"Mes disciples sont maintenant nombreux. Ils réussissent, et ils ignorent ce que c'est qu'une panicule. Ils n'ont pas la plus légère curiosité de savoir comment la nature reproduit les mêmes plantes, et cependant ils ne se trompent jamais; ils ne se plaignent jamais d'avoir fait choix d'une mauvaise variété de fiorin. Il semble qu'ils aient pour maxime, *Grede quod habes*, et *habes*: ils vont en avant avec confiance, et réussissent toujours.
Lorsque j'ai vu qu'en Angleterre on s'occupoit si vivement de la question des variétés du fiorin, et qu'on y attachoit tant d'importance, j'ai cru convenable d'y donner quelqu'attention, et de jeter, s'il étoit possible, quelque jour sur cette matière, dont, au reste, je m'étois déjà occupé auparavant."
"Pendant l'hiver et au commencement du printems j'examinai les racines de quelques herbes stolonifères et je ne découvris que deux variétés: l'une avec de petites racines fibreuses, et tenant fort peu en terre, l'autre avec quelques racines semblables à celles du chiendent, mais pénétrant plutôt en terre obliquement que dans une direction horizontale. Les radicules fibreuses sortoient des racines en anneaux circulaires, comme au chiendent. "
"Je fis, dans mon jardin, deux essais rapprochés\setcounter{page}{251} de transplantation de ces deux variétés. Malheureusement celle dont la racine ressemble à la racine du chiendent, ne donna pas une seule panicule : l'autre variété en donna beaucoup et de la véritable sorte.
"Je fus très-surpris de trouver peu de différence dans la force de végétation des plantes; et si je n'y avois pas porté mon attention, je n'y aurois remarqué aucune différence. Je conserve ces deux espèces pour l'année prochaine; j'espère qu'elles donneront des panicules."
"Dans le courant de juin, les panicules se montrèrent dans mes prés, et je cherchai avec beaucoup d'attention les différences. Celles du véritable fiorin sont d'abord formées, en fer de lance, d'un brun pourpre. La plupart ont leurs épilets divergens; et elles brunissent: celles dont les épilets ne divergent pas ressemblent à un holcus brun compacte."
"Dans mes prés, et parmi les plantes dont les panicules divergeoient, j'ai découvert nettement deux variétés. Les panicules de l'une et de l'autre étoient coniques; mais dans l'une le cône étoit prolongé et l'angle du sommet aigu. Dans l'autre ce sommet étoit arrondi. Ces deux variétés étoient tellement confondues dans les masses des tiges, qu'on\setcounter{page}{252} ne pouvoit discerner de supériorité de l’une sur l’autre."
" Je cherchai dans les champs ces deux agrostis, et je les trouvai en abondance ; mais celle dont la panicule étoit arrondie au sommet se trouvoit seulement dans les terrains secs, au lieu que l’autre se voyoit dans toutes les terres, depuis les plus sèches jusqu’aux plus humides. "
" Au commencement d’octobre j’examinai une prairie très-abondante de fiorin, appartenant à l’évêque de Derry. Dans une partie, les faucheurs coupoient une récolte superbe ; et dans tout ce qui restoit à couper, il n’y avoit pas une panicule étrangère, excepté quelques houlques. Les panicules d’agrostis pointues abondoient ; mais il n’y en avoit point d’arrondies. La terre étoit une glaise froide, et n’avoit pas été suffisamment égouttée. Les mauvaises herbes avoient été arrachées avec soin. "
" J’examinai ensuite une autre récolte du même propriétaire ; la plus belle que je me rappelle avoir vue. Le terrain étoit riche et humide, arrosé par l’eau des égouts de Derry ; et à mon avis, il y avoit trop d’eau. "
" J’examinai soigneusement les panicules. Il y avoit aussi quelques épis de holcus, mais un très-grand nombre de panicules\setcounter{page}{253} pointues d'agrostis, c'est-à-dire, de vrai fiorin. Les autres panicules que je remarquai étoient celles du foin aquatique, de la festuque flottante, et du vulpin articulé. Les deux premières plantes sont décidément aquatiques : la dernière est amphibie. Je demandai à l'évêque où il avoit pris les stolons pour faire sa prairie, il me répondit que c'étoit sur le bord septentrional d'un fossé humide, dans son domaine." Ce qui se passe des deux côtés de la mer ne se ressemble point. En Irlande, nous ne pensons jamais aux variétés. Nous tombons toujours sur le vrai fiorin; nous en obtenons des récoltes immenses, et ces exemples se multiplient tous les jours. En Angleterre, les agriculteurs sont en défiance. Ils ont voulu être sûrs de leur fait, avant que de planter des prairies en fiorin. Ils ont consulté les laborieux élèves de Linné; ils ont découvert quarante-trois variétés d'agrostis; et sur toutes ces variétés, ils n'ont pas su en trouver une seule qui fût applicable à l'usage." Ne permettez pas qu'ils infestent de leur prudence et de leurs doutes le midi de l'Irlande. Je commence à avoir peur pour vous, quand je vous entends dire : "Nous avons certainement le fiorin dans le pays : si nous pouvons en être bien sûrs et le reconnoît re\setcounter{page}{254} avec certitude, vous entendriez parler de fortes récoltes dans le midi de l'Irlande. Je vous réponds à cela, que vous aurez ces belles récoltes de fiorin, dès que vous cesserez d'être si prudens. Laissez-là vos disciples de Linné, et leurs fatiguantes distinctions ; allez dans vos marais, dans vos fossés, sur-tout dans vos montagnes. Choisissez des plantes vigoureuses, dans les endroits humides. Transplantez-les dans un terrain sec; ou desséché, et de bonne qualité. Faites cette opération comme je vous l'ai souvent recommandé ; et vous ne pourrez pas vous tromper beaucoup."
La festuque flottante ne peut pas vous induire en erreur ; le foin aquatique (aira aquatica) ressemble beaucoup plus au fiorin ; mais l'une et l'autre ne vivent que dans l'eau. Elles disparoîtront bientôt de votre nouveau pré ; et le fiorin ( qui toujours se trouve mélangé avec ces deux plantes ) les remplacera, et couvrira la terre.
Le fait est que, de toutes les variétés d'agrostis, ou plantes stolonifères, le fiorin est la seule qui puisse supporter également une situation humide et une terre sèche.
Sept années d'observation attentive à l'histoire naturelle et aux habitudes du fiorin, et des succès dans la culture de cette\setcounter{page}{255} plante, qui ont surpassé toutes mes espérances, (ce dont je puis citer plusieurs de vos compatriotes pour témoins), m'ont donné une pleine confiance. Ainsi, je vous le répète: allez chercher des stolons dans vos fossés humides et sur vos montagnes; et laissez les panicules aux élèves de Linné, pour qu'ils s'en amusent."
\subsection{Observations d'un Correspondant sur la lettre du Dr. RICHARDSON.}
"Le docteur nous dit : "la question des variétés du fiorin est curieuse en elle-même, mais ses difficultés ne frappent que ceux qui ont la manie linnéenne, et qui ne veulent pas voir l'inutilité de cette classification malheureuse et forcée."
Comme le Dr. Richardson connoît sans doute très-bien la classification linnéenne, je n'ai pas besoin de lui rappeler que l'organisation de laquelle dépend la classification des plantes, n'a aucun rapport aux distinctions qui marquent leurs variétés. Il est donc difficile de comprendre comment l'intelligence du système de Linné et le respect pour ce philosophe, doivent rendre un homme moins capable de reconnaître des variétés. \setcounter{page}{256} Les variétés peuvent être infinies parmi les plantes comme elles le sont parmi les animaux. Les distinctions ne peuvent donc être déterminées qu'à mesure que ces variétés se font connoître. Les variétés mêmes offrent des différences selon les individus, ou des sous-variétés. Il y a des agrostis stolonifères qui prennent des apparences diverses et nouvelles ; cela doit provenir de la graine. Une certaine variété peut être propagée par les marcottes ou les stolons, pendant un certain temps seulement ; et lorsque la plante qui a fourni ces stolons meurt, ceux-ci s'affaiblissent et meurent aussi. Une seule plante d'agrostis provenue de graine, peut donner des marcottes ou stolons en nombre suffisant pour couvrir un vaste espace, et former un pré ; et c'est probablement une variété très-productive, provenue de graine, qui a fourni les stolons d'une longueur prodigieuse dans le pré de Ste. Marie d'Orcheston, en Wiltshire, dont on a fait mention autrefois, et qui aujourd'hui paroît anéanti. On ne peut pas mieux espérer la même variété d'une plante quelconque, par la graine, qu'on ne peut espérer le même individu des animaux.
Le docteur dit que les panicules du fiorin se développent dans tous les prés au mois\setcounter{page}{257} de juin. Je ne crois pas qu'aucun agrostis stolonifere, indigene dans la Grande-Bretagne, développe ses panicules dans le mois de juin. Il me paroît donc probable qu'il y a à présent en Irlande des variétés différentes de celles d'Angleterre, et qui sont beaucoup plus productives. C'est cette circonstance ou telle autre non encore expliquée, qui fait apparemment que tandis que le docteur recueille de neuf à dix charretées de deux milliers pesant de fiorin sec dans un acre anglais, nos récoltes de cette plante en Angleterre, ne passent pas trois charretées de même poids par acre.
Le docteur plaide courageusement la cause de l'ignorance contre celle de la science. "Mes disciples," dit-il, "sont nombreux maintenant. Ils réussissent, et ils ignorent ce que" c'est qu'une panicule. Ils n'ont pas la plus" légère curiosité de savoir comment la nature reproduit les mêmes plantes, et cependant ils ne se trompent jamais : ils ne" se plaignent jamais d'avoir fait choix d'une" mauvaise variété de fiorin. Il semble qu'ils" aient pour maxime : crede quod habes et" habes. Ils vont en avant avec confiance," et ils réussissent toujours." Cette maxime des élèves du docteur ressemble un peu à l'histoire du bonnet de Fortunatus, dans les contes de ma mere l'oie.\setcounter{page}{258} Quoique les élèves du docteur ne sachent ce que c'est qu'une panicule, et ne se trompent jamais, il paroît que lui-même a jugé convenable d'avoir recours à l'observation des panicules pour distinguer entr'elles les deux seules variétés d'agrostis stolonifères, qu'il ait jugées digne d'être remarquées pendant sept ans de pratique sur l'histoire naturelle, d'observation sur les habitudes du fiorin, et de culture de cette plante. Le Dr Richardson pense-t-il en effet, qu'il n'y ait pas le moindre avantage d'une variété sur une autre, parmi les agrostis stolonifères, et que ce choix puisse être absolument confié au hasard, quel que soit le sol dont il s'agit? Dans le jardin botanique d'Édimbourg, il y a six différens compartimens destinés à des agrostis stolonifères. Chaque espace a sa teinte de vert; chaque variété offre des différences dans le nombre, la longueur, et l'épaisseur de ses feuilles : toutes cependant ont été plantées, en 1813, le même jour, sur le même sol, et dans la même situation.
Je demande si des plantes prises dans les compartimens qui offrent la plus belle végétation ne vaudroient pas mieux pour en faire un pré, que les plantes des compartimens où cette végétation est la plus foible.
On pourroit faire les mêmes questions sur\setcounter{page}{259} la qualité des semences des grains, et sur le choix des greffes pour les arbres fruitiers.
Le fiorin est assurément une plante précieuse dans certaines circonstances et dans certaines situations; mais je demande d’être dispensé de croire que toutes les variétés sont également précieuses, quels que soient le sol et la situation, tandis que cela n’est pas vrai pour les autres plantes de prés. Il est permis de conjecturer, que certaines variétés sont propres à certaines terres et à certaines situations; et d’autres à d’autres; qu’en conséquence, il seroit utile de s’assurer quelles sont les variétés qui conviennent le mieux dans telles circonstances données.
Le Dr. Richardson a recommandé d’employer les stolons de préférence à la graine, pour former des prés; et il ne parle pas de l’âge des plantes. Il croit peut-être qu’elles sont immortelles. En dépit des miracles que la foi opère parmi ses disciples, il trouvera, je pense, qu’un pré de son herbe favorite a besoin d’être renouvelé par la semence, tout comme un bois de chênes ou un verger; qu’un sol ou un climat quelconque ne peut prolonger la vie d’une plante au-delà du maximum marqué par la nature. Quelques agriculteurs se sont plaints de ce que leurs plantes de fiorin mouroient dans les\setcounter{page}{260} prés qu'ils avoient établis : cela pouvoit venir de ce qu'ils avoient tiré leurs stolons, de plantes déjà vieilles.
La sollicitude du Dr. Richardson, pour exalter le mérite du fiorin, pourroit bien faire l'effet opposé, et nuire pour un temps à la diffusion de la culture de cette plante là où elle peut être utile.

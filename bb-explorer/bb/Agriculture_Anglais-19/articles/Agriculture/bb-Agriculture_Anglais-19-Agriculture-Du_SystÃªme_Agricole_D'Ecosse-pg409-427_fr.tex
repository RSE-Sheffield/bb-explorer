\setcounter{page}{409}
\chapter{AGRICULTURE.}
\section{An ACCOUNT OF THE SYSTEMS, etc. Exposé des systèmes de culture adoptés dans les districts d'Ecosse les mieux soignés; avec quelques observations sur les perfectionnemens dont ils sont susceptibles, etc. Par le R. H. Sir John SINCLAIR, Président du Bureau d'Agriculture. 8°. pp. 660 ( Extrait tiré du Monthly Review, sept. 1814 ). \large{( Traduction )}}
Les combinaisons de l'expérience sur des matières qui appartiennent aux sciences pratiques ressemblent aux rayons de lumière qui convergent vers un même foyer. Dans l'importante affaire de l'agriculture, on peut tirer de grands avantages de la comparaison de différens systèmes; parce que l'émulation est excitée par l'opposition des usages et des modes de culture adoptés dans certains districts, avec ceux qu'on suit dans d'autres contrées. Il est bon toutefois de remarquer que les contrastes de cette espèce donnent
Agricult. Vol. 19. N°. 12. Déc. 1814.\setcounter{page}{410} une leçon mortifiante aux cultivateurs des régions les plus favorisées. Lorsqu'il s'agit de surmonter des difficultés, il faut déployer beaucoup d'habileté et une industrie très-persévérante; et souvent par ces moyens, la nature est forcée à devenir propice au-delà de ce qu'on auroit pu espérer d'elle; tandis que ceux que rien n'engage à de semblables efforts demeurent étrangers à l'emploi des moyens que la Providence avoit mis dans leurs mains.
Ces considérations expliqueront jusqu'à un certain point les perfectionnemens introduits dans les systèmes agricoles du nord de notre isle, et qui sont tels, que le fermier Anglais est souvent forcé de reconnoître, à sa honte, que la pratique d'Écosse est en plusieurs points supérieure à la sienne.
Dans l'Introduction de l'ouvrage utile que nous avons sous les yeux, Sir J. Sinclair indique diverses circonstances qui ont contribué à l'excellence de l'agriculture écossaise; et elles mériteroient d'être connues dans toutes les parties de l'Angleterre. Quoique certainement le climat d'Écosse soit peu favorable à la végétation, ce désavantage est souvent plus que compensé par l'assiduité avec laquelle le fermier Écossais sait se prévaloir de tous les moyens et les secours qui sont à sa portée. Parmi les causes morales qui ont con-\setcounter{page}{411} tribué à l'amélioration des contrées au nord de la Tweed, on compte avec raison l'établissement des écoles de paroisse, qui ont considérablement amélioré l'éducation des fermiers ; la classe inférieure du peuple est élevée dans des principes de vertu, seules bases des bonnes habitudes. Voici les remarques de l'auteur sur cet objet important.
"Les frais de culture en Écosse ne different pas essentiellement de ce qu'on obtient à cet égard en Angleterre dans plusieurs districts étendus ; et les journaliers Anglais font certainement autant d'ouvrage dans le même temps que ceux d'Écosse. Mais les domestiques employés à la culture en Écosse, sont en général plus traitables, plus faciles à nourrir, moins intéressés, moins accoutumés à perdre leur temps et à dissiper leurs gages; ils sont plus réguliers et plus assidus aux heures de travail ; ils ont ordinairement reçu une meilleure éducation; enfin, leur conduite en tous points est beaucoup meilleure que celle des domestiques Anglais."
C'est cette éducation, commencée de bonne heure, qui a contribué à former le caractère des Écossais, qu'on sait être en général industrieux, économes, intelligens, et doués de persévérance. Ces qualités les rendent propres à arriver à un degré assez éminent\setcounter{page}{412} dans les arts ou les sciences lorsque les circonstances les appellent à s'en occuper.
On peut douter que l'établissement des banques provinciales, et l'extension du papier-monnoie aient été plus utiles au fermier Ecossais qu'à l'Anglais; mais ce qui suit montrera que le premier possède des avantages dont le second est privé.
"En Ecosse ( dit Sir Sinclair ) les lois sont favorables à la culture et aux perfectionnemens ; on doit attribuer à cette circonstance l'excellence comparative de l'agriculture écossaise."
Il y a des réglemens qui facilitent la division des Communes ; on élève des haies entre des champs limitrophes, aux frais communs des propriétaires. La valeur des dîmes à lever sur une propriété peut être fixée à la décision de la Cour suprême, et lorsqu'elle a été une fois établie par évaluation légale, et convertie en ce qu'on pourroit appeler rente en grains, elle ne peut plus être sujette à augmentation. Le possesseur d'un fonds de terre peut en général libérer sa propriété de toutes demandes indéfinies sur elle, qui seroient de nature féodale ; les baux sont interprétés par les Cours de judicature d'une manière favorable au fermier ; et les propriétaires d'un sol en général ne sont pas soumis à des\setcounter{page}{413} charges arbitraires sous la dénomination de Taxe des pauvres.
Indépendamment des causes qui, en théorie, peuvent avoir opéré en faveur de l'Écosse, c'est un fait notoire, que son système de culture est arrivé à un degré de perfection tel, qu'on le cite comme modèle pour tous les districts dont la situation est analogue; dans ce qui concerne les terres arables.
Le but de l'auteur est de placer ce modèle à portée et en vue de tous les agriculteurs. Son ouvrage est divisé en deux parties. Voici le tableau qu'il en présente lui-même. La première partie est purement pratique. On commence par y indiquer en détail les procédés considérés comme les meilleurs par les fermiers Écossais les plus distingués relativement aux points qui exigent une attention préalable avant de commencer la culture des terres arables. 2°. On entre dans les détails qui sont liés avec la culture actuelle d'une ferme et les moyens les plus profitables d'entretenir son fonds de bestiaux. Cette partie de l'ouvrage se termine, 1°. par un tableau général des systèmes de culture adoptés en Écosse; 2°. par un exposé des perfectionnements dont quelques-uns de ces systèmes sont susceptibles; 3°. par quelques\setcounter{page}{414} servations sur les moyens par lesquels les pratiques utiles des meilleurs fermiers Anglais peuvent être le plus avantageusement répandues dans les districts les moins avancés en Angleterre et en Ecosse. On y joint un aperçu général des avantages publics et particuliers qu'on peut retirer de leur adoption plus générale."
" Il y a pourtant certaines questions, liées à l'agriculture d'Ecosse, et aux genres d'améliorations dont elle est susceptible, qui sont assez compliquées, et dont la solution exige des recherches approfondies, et beaucoup de méditation. Comme, par exemple, l'étendue des fermes, et l'établissement d'un système libéral de liaisons entre le propriétaire et le fermier. Ces points, en y comprenant aussi le caractère moral des journaliers qu'on emploie aux travaux de la terre dans les districts de l'Ecosse les mieux cultivés, ne sont traités que dans une division à part, et discutés dans autant de dissertations séparées."
" Un appendix énoncera quelques particularités qui sont en rapport avec le défrichement des terres incultes ou abandonnées en Ecosse, sujet sur lequel l'attention publique ne sauroit être trop fréquemment ramenée. Il y est question aussi de plusieurs autres points sur lesquels l'auteur se flattoit\setcounter{page}{415} que l'information qu'il alloit communiquer seroit acceptable au public.
On ne peut indiquer ici que d'une manière très-abrégée, un plan aussi vaste que celui qui vient d'être annoncé; et cependant la notice que nous publions montrera aux agriculteurs que l'ensemble du système est très-complet.
Dans son premier chapitre, Sir John porte l'attention sur les points à examiner avant de commencer les labourages quelconques. C'est, la position la plus convenable, et la meilleure construction à donner aux bâtimens des fermes et aux offices.-- L'étendue et la forme des champs -- la manière de les enclore-celle de les dessécher -- les chemins de dépouille sur la ferme et son voisinage -- tout l'attirail des instrumens de culture. -- Les animaux à entretenir sur la ferme -- le sol, le climat, la hauteur au-dessus du niveau de la mer, l'exposition; enfin, la position de la ferme relativement aux marchés voisins.
L'auteur pose comme axiome en agriculture, que le bâtiment principal de la ferme et ses dépendances doivent être placés aussi près qu'il est possible du centre de la propriété. Dans la section qui traite de l'étendue et de la forme à donner aux champs, Sir réprouve avec raison pour ceux qu'on\setcounter{page}{416} destinée principalement à la culture du grain, les petites divisions, de figure irrégulière, environnées de grands arbres ou de fortes haies. Il donne à cette occasion les directions judicieuses que voici : "Dans les petites fermes situées près des villes, de six à douze acres peuvent suffire pour un champ. Mais là où les fermes sont d'une étendue convenable, on recommande des champs, de vingt jusques à cinquante acres, et quelquefois jusques à soixante. Dans aucun cas il ne faut les mettre au-dessous de vingt acres d'Ecosse ou vingt-cinq, d'Angleterre. Mr. Brown, de Markle, dont personne n'ignore les connoissances sur toutes les branches de l'agriculture, considère un champ de trente acres d'Ecosse ou trente huit d'Angleterre, comme l'étendue moyenne la plus convenable à la culture dans les grandes fermes, lorsque les circonstances permettent d'aller jusques-là." Parmi les instrumens de l'exploitation, Sir John non-seulement recommande fortement la machine à battre les grains, mais il cite en preuve du crédit qu'elle a acquis en Ecosse l'observation suivante. "Dans le seul district de Carse de Gowrie, qui a environ quatorze milles de long sur quatre de large, il n'y a pas moins de cent\setcounter{page}{417} vingt machines à battre le blé, mises en mouvement par des chevaux; et dix par l'eau. Dans d'autres parties de l'Ecosse, ces machines sont devenues d'un usage si général, qu'on ne trouve plus guère de journaliers qui veuillent battre au fléau. La construction de ces machines est devenue une vocation ou un métier particulier, distinct de celui du charpentier, constructeur des moulins ordinaires.
Comme preuve de ce qu'il avance en faveur de ces machines, l'auteur cite le calcul suivant, fait par Mr. Brown, des profits qu'il y auroit à faire pour le public dans l'adoption générale de ces machines.
\comment{table}
1. Le nombre d'acres qui produisent du grain en Angleterre est de ...................... 8 millions
2. Le produit moyen de cette surface arable, à 3 quarters par acre ......... quarters 24 millions
3. Excès du produit du battage à la machine sur celui au fléau, estimé à du produit; soit, quarters ................................. 1,200,000
4. Valeur de cette augmentation de produit, à 40 sh. le quarter ............... L. st. 2,400,000
5. Economie sur les frais d'exploitation, à 1 sh. par quarter ............................... 1,200,000
6. Profit total annuel possible ..................... 3,600,000
7. Profit actuel, d'après la supposition que seulement la moitié du grain produit soit battu par année ..................... 1,800,000
Y a-t-il donc lieu de s'étonner, si l'on\setcounter{page}{418} entend Mr. Brown affirmer, que la machine à battre le blé est l'instrument le plus utile que puisse posséder un fermier: il soutient que cette machine ajoute plus au produit ordinaire de la culture qu'aucune invention moderne ; et qu'on doit considérer son introduction dans l'économie agricole comme le plus grand perfectionnement qu'ait reçu celle-ci dans tout le cours du siècle."
L'auteur ajoute, à la fin de cette section, qu'il n'est aucun pays de l'Europe, parmi ceux qu'on peut citer en agriculture, où les instrumens de culture soient en si petit nombre, si simples, et à aussi bon marché qu'ils le sont en Écosse ; circonstance de grande importance pour le fermier industrieux. Il dit encore, que la machine à battre le blé, quoiqu'elle coûte beaucoup d'argent, est encore la plus économique de toutes , si l'on considère l'épargne de main-d'œuvre qu'elle procure.
Dans le chapitre qui traite des animaux de ferme , l'auteur entre dans de grands détails sur la comparaison du système des pâturages avec celui de la nourriture à l'étable; comme aussi sur toute la manipulation de la laiterie moderne. Il considère comme le perfectionnement le plus essentiel récemment introduit dans ces procédés, l'invention des\setcounter{page}{419} bassins à lait de fer fondu, imaginée par Mr. Baird, des fonderies de Shott près Whitburn, dans le comté de West-Lothian. Ces vases de fonte adoucie par la cémentation dans le charbon, sont polis au tour à l'intérieur, puis enduits d'une couche d'étain, destinée à empêcher le contact du lait avec le fer, qu'il feroit rouiller. L'extérieur du vase est recouvert d'une peinture à l'huile, qui le met aussi à l'abri de la rouille. Ces vases ont le grand avantage de conserver ce degré de fraîcheur qui contribue essentiellement à la séparation de la crème; on les nettoie aussi très-facilement avec un peu de craie étendue sur un morceau d'étoffe de laine; enfin, on les étame de nouveau à peu de frais. Ils sont devenus tellement à la mode, que les entrepreneurs peuvent à peine suffire à la demande. Nos fermiers Anglais, qui tirent parti du laitage, se prévaudront sans doute de cette invention; mais il faut pourtant observer que ces vases de fer si recommandés par le Président ne sont comparés par lui qu'aux vases de bois, et non aux grands bassins de faïance, qu'on emploie aux mêmes usages dans la partie méridionale de l'Angleterre.
Après avoir discuté et approfondi les divers objets qu'on vient d'énoncer, Sir John\setcounter{page}{420} termine son chapitre par une courte récapitulation, toute à l'honneur des fermiers Écossais.
"Les fermiers, dit-il, ont été trop souvent tournés en ridicule, et traités de race stupide et ignorante. On peut dire, au contraire, que dans les districts bien cultivés d'Écosse, ils sont tellement instruits de tous les détails de leur profession, qu'il n'y a guères de classe dans la société qui se distingue par des connoissances plus étendues et plus variées. Loin donc de considérer un véritable fermier comme un grossier et ignorant paysan, on devroit le compter parmi les individus les plus précieux à la société, et les plus utilement instruits dans les arts desquels dépendent la prospérité et le bonheur d'un grand pays."
Après avoir aidé le fermier dans la disposition générale du système qu'il doit adopter, Sir John passe dans son second chapitre, aux détails d'expérience, et il met en évidence les traits caractéristiques de la culture écossaise, sous les deux rapports de l'exploitation d'une ferme arable, et de l'entretien de son capital en bestiaux et instrumens de tout genre. Il range toutes les informations que lui a procuré sa correspondance très-étendue, sous les chefs suivans.
\setcounter{page}{421} Préparation préliminaire du sol, et desséchement — engrais — labours — jachères — genre des récoltes à choisir — assolemens à adopter — semailles et hersages — arrachement des plantes parasites.— Moisson.— Préparation du grain pour le marché — nourriture des animaux en vert.— Est-il convenable d'avoir une portion quelconque de la ferme en pâturage permanent? — De la meilleure manière de faire les foins.
Il est bien difficile d'entrer ici dans le détail des faits nombreux rassemblés dans ce chapitre. Il faut se borner à faire un choix. Ainsi, nous apprenons, qu'il n'y a aucun pays en Europe où l'on fasse autant d'usage de la chaux calcinée pour engrais, et où on l'emploie en quantité aussi considérable que dans les parties d'Ecosse où la culture a fait, et fait encore les progrès les plus marqués.— Que la culture du blé a pris un grand accroissement en Ecosse, et qu'elle est introduite dans un nombre de districts où elle étoit jadis inconnue, et même dans des endroits élevés de 500 à 600 pieds an-dessus du niveau de la mer,\footnote{Nous soupçonnons ici une faute d'impression ; car cette élévation est bien peu considérable. Peut-être auroit-il fallu lire yards au lieu de feet ; ce qui auroit triplé la hauteur. (R)}.— Qu'aucun grain ne\setcounter{page}{422} réussit mieux en Ecosse que l'avoine. — Qu'à raison de l'humidité du climat, on fait usage avec succès de cônes construits avec des perches, et autour desquels on forme les meules de grains, qui ont alors dans l'intérieur un espace vide dans lequel l'air peut circuler. — Que la pratique de nourrir les bestiaux en vert à l'étable a prévalu dans presque tous les districts bien cultivés d'Ecosse. — Enfin, que lorsqu'on fait les foins, on se trouve très-bien, sur-tout dans la saison précaire, de la méthode du Lancashire, de disposer le foin en tipples aussitôt qu'il est coupé.
Voici comment on fait ces tipples. Un ouvrier roule de la main droite l'ondin en dedans, et ramasse ainsi une forte poignée; il en fait autant de la gauche, et réunissant les deux poignées il en forme une sorte de faisceau qui peut peser de 8 à 12 liv., et qu'on dresse debout contre les jambes ou entre les pieds. On forme un lien de quelques plantes réunies et tordues ensemble, et on en lie le sommet du faisceau, qu'on dresse et ajuste en cône autant qu'il est possible. Au bout de quelques heures la surface extérieure de ce cône devient si régulière et si glissante que les plus fortes pluies ne pénètrent guères à l'intérieur, et que si le cône se mouille finalement, il est bien plus vite desséché ensuite\setcounter{page}{423} que s'il restoit couché par terre. Quand il est sec on en forme la meule d'été, et même celle d'hiver, s'il est parfaitement sec; mais jamais on n'ouvre ni ne secoue ces cônes pour hâter leur dessication, car cette précaution seroit superflue. Par ce procédé il ne se perd pas une feuille, et le foin demeure aussi vert que s'il avoit été séché dans un livre.
Quelques observations générales sont annexées à ces grands détails, par forme de conclusion. Elles sont suivies d'un plan pour transporter la culture d'Écosse en Angleterre, au moyen duquel les champs anglais seroient dorénavant, comme les jardins l'ont été depuis long-temps, exclusivement cultivés par des Écossais. Voici les principaux traits du projet.
" Si un propriétaire de terres en Angleterre est convaincu que, tant pour son propre avantage que pour l'intérêt public, il convient de changer le système de culture adopté dans ses propriétés, on soumet à ses réflexions le plan qui suit."
"Il pourra être agréable et utile au propriétaire encore jeune et actif, d'examiner sur place l'état véritable de la culture écossaise dans les districts les plus soignés; d'étudier de près la manière dont elle est conduite,\setcounter{page}{424} et les effets qui en sont résultés; enfin de voir jusqu'à quel degré les procédés qui la caractérisent sont applicables à sa propriété territoriale."
" Si cette excursion n'est pas praticable, il est invité à consulter les personnes qui connaissent à fond le système qu'il est question d'adopter; en évitant soigneusement de s'adresser aux individus suspects de donner plus d'attention à leurs intérêts propres, qu'à ceux des personnes qui les consultent ou les emploient."
" Si le propriétaire a une ferme à diriger, il pourroit être à propos de se procurer d'Ecosse un sur-intendant ou baillif, dont l'exemple pourroit contribuer à dissiper les préjugés des fermiers du voisinage contre le nouveau système qu'on se propose d'établir."
" Si quelque ferme de 300 et 500 acres étoit accessible, il pourroit être convenable de la louer à quelque fermier Ecossais industrieux; dans le but d'ouvrir les yeux des autres fermiers sur les avantages du nouveau système."
" Le propriétaire doit se décider à accorder des baux de vingt-un ans aux fermiers nés dans le Comté, et de vingt-cinq ans aux étrangers qui viennent s'y fixer. Sans cette condition, il ne doit pas s'attendre qu'aucun système\setcounter{page}{425} perfectionné puisse s'introduire."
" Les baux doivent être dressés avec des clauses libérales; mais en stipulant un accroissement gradué de revenu, en partie fondé sur le prix des grains, afin de prévenir toute déduction un peu importante dans la rente relative du sol."
" Les avances du propriétaire doivent dépendre de l'habileté avec laquelle il sait placer son argent au profit de sa terre. Ce qui est exécuté par le tenant, l'est en général avec économie et jugement, mais on auroit tort de détourner de leur véritable objet les moyens et les efforts d'un nouveau tenant, en le forçant à consacrer à des perfectionnemens permanens, ce capital qui devroit être employé à l'achat seul du bétail, etc. et à l'exploitation et aux engrais du sol."
" Si ces mesures étoient généralement adoptées; on a tout lieu de croire que le revenu d'une surface égale au moins à dix millions d'acres, en Angleterre, pourroit être doublé, et son produit considérablement augmenté."
" Quelques-uns recommandent la mesure d'envoyer les enfans des fermiers pour une ou plusieurs années, se placer chez des Ecossais, là où la culture est la meilleure,\setcounter{page}{426} pour y apprendre les principes de l'art. C'est un moyen un peu lent, à la vérité, mais en même temps sûr, d'arriver à son but, pourvu que les jeunes gens envoyés soient obligés de s'occuper eux-mêmes des travaux manuels de la ferme, et qu'on ne les laisse point perdre leur temps à des courses dans le pays."
" Mais après tout, l'introduction de nouveaux fermiers, là où on peut s'en procurer est le meilleur plan à adopter."
" Les auteurs de l'Extrait que nous traduisons, se prononcent contre la convenance de cette importation de nouveaux fermiers, qui ne leur paroît fondée sur aucune nécessité. Ils ne veulent pas croire non plus, qu'un agriculteur Ecossais pût parvenir à doubler les récoltes actuelles en Angleterre, nous consentirions volontiers, disent-ils, à prendre leçon de nos compatriotes du nord, mais pourquoi adopterions-nous des plans qui tendroient à envoyer à l'hôpital nos propres fermiers, pour leur substituer un assortiment d'étrangers ?"
Ce chapitre tout long qu'il est, présente encore des additions, sur lesquelles les auteurs de l'Extrait gardent le silence, dans le but d'arriver à la seconde partie, qui renferme des questions liées aux perfectionnemens généraux\setcounter{page}{427} dont l'agriculture est susceptible, surtout en Ecosse. Nous réservons cette seconde partie pour un cahier prochain.
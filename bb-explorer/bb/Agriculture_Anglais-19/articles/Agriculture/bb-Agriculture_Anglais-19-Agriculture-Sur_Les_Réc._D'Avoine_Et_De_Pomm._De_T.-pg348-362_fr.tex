\setcounter{page}{348}
\chapter{Agriculture}
\section{OBSERVATIONS ON THE QUANTITIES, etc. Observations sur la quantité comparative de nourriture, obtenue d'une récolte d'avoine et d'une récolte de pommes de terre. Farmer's Magazine. Août 1814 \footnote{Extrait par le Prof. De La Rive.}.}
DANS le Dictionnaire de Johnson, on trouve à l'article AVOINE. Espèce de grain quel'on donne ordinairement en Angleterre aux chevaux, mais qui en Ecosse sert de nourriture au peuple \footnote{Vid. Johnson's Dict. art. Oats.}. Quel est le sens de cette définition? Johnson veut-il faire entendre par là, que l'Ecosse est moins riche que l'Angleterre, que le peuple y est plus sobre et par conséquent meilleur, que les mœurs y sont pures, plus hospitalières, que le luxe n'ayant pas pénétré dans ce pays, il peut se nourrir sur ses propres productions, et ne dépend pas comme l'Angleterre de l'importation des blés étrangers? Si tel est le sens de la définition, je l'approuve, mais pourquoi lui donner une tournure sarcastique\setcounter{page}{349} et offensante? Pourquoi avancer avec une espèce de dérision, que la nourriture consacrée aux chevaux en Angleterre, est la même que celle du peuple Ecossais? J'avoue, que je ne puis m'empêcher de reconnaître là, les traces de ces préjugés nombreux que les Anglais ont, et que Johnson, surtout, avait contre l'Ecosse. Préjugés bien peu fondés, puisque c'est dans cette nation, que l'on trouve peut-être plus qu'ailleurs, une grande générosité de sentiments, des vertus domestiques, la cordialité et l'hospitalité la plus touchante. Patrie des Robertson, des Hume, des Blair, des Cullen, des Dugald-Stewart, des Gregory, des Playfair, etc. etc. L'Ecosse rivalise, si elle ne surpasse peut-être l'Angleterre, par le nombre des hommes illustres qu'elle a produits. Fière de l'incorporation de l'Ecosse, l'Angleterre devrait proscrire d'un de ses livres classiques, le plus répandu peut-être, ce trait de satyre dénué de sel, et qui fait peu d'honneur à celui qui l'a écrit. Mais venons au mémoire dont j'ai l'intention de donner un extrait.
La nourriture ordinaire aux classes laborieuses, composant la majorité de la population de chaque pays, est un sujet de recherches qui mérite la plus grande attention. Elle ne se règle pas seulement sur les moyens\setcounter{page}{350} de subsistance seuls, mais aussi sur l'espèce de nourriture que la majorité de la nation consomme. La même quantité de terrain de la même espèce, fournira des alimens à un plus ou moins grand nombre, suivant que ces alimens consisteront, en un mélange de substances animales et végétales, ou seront composés uniquement des dernières. Si la manière de se nourrir en Angleterre, parmi les basses classes de la société, étoit la même qu'elle l'est en Irlande, ou même en Écosse, il n'est pas douteux que le terrain anglais cultivé, seroit suffisant pour alimenter une population double de celle qui existe, et si l'artisan Irlandais consommoit autant de froment et de viande que l'anglais, il n'est pas probable que malgré le perfectionnement de son agriculture, l'Irlande pût fournir à ses voisins autant de blé qu'elle le fait. L'augmentation de la population, ne peut en aucune manière expliquer seul, pourquoi la Grande-Bretagne, qui pendant long-temps exportoit des substances nutritives, est obligée maintenant d'en tirer des autres pays, quoique depuis cinquante ans son sol, par les améliorations de culture, ait été rendu beaucoup plus productif.
On dit en général que la population est nécessairement limitée par les moyens de\setcounter{page}{351} subsistance; mais c'est en connoissant quelle est la nourriture ordinaire des basses classes de la société, que nous aurons seulement un moyen juste de nous former une idée correcte, soit de leur aisance individuelle, soit de l'état progressif, stationnaire, ou rétrograde, de la société en général. Nous pouvons hardiment avancer, d'après l'espèce de nourriture du peuple anglais, qu'il a des moyens d'obtenir les choses nécessaires et mêmes agréables à la vie, moyens qui ne sont à la portée, ni des Écossais ni des Irlandais.
Si la nourriture commune du peuple est un moyen de juger la prospérité publique, plus que sur le prix du travail, il doit être de quelqu'importance de marquer les différens changemens qu'elle a subis de temps à autres. En comparant la manière de vivre actuelle avec l'ancienne, nous pouvons nous former une idée des progrès de chaque pays. Il est vrai que ces progrès peuvent se montrer sous d'autres formes; comme dans l'amélioration des habits, des logemens; plus encore peut-être dans celle de l'éducation des basses classes; mais il y a toute espèce de raison de croire, que ces améliorations accompagnent toujours l'usage général d'une meilleure nourriture.
\setcounter{page}{352} J'ai été conduit à faire ces réflexions\footnote{C'est l'auteur du mémoire qui parle.}, par la lecture de quelques écrits, qui ont pour objet de montrer, combien il faudroit comparativement peu de terrain, pour fournir toute la subsistance à notre population actuelle, si l'avoine ou les pommes de terre, étoient la nourriture ordinaire du peuple de toute la Grande-Bretagne, comme ces substances le sont, l'une de l'Ecosse et l'autre de l'Irlande. Je vais donner en peu de mots les observations comparatives que j'ai faites sur la consommation de ces deux espèces de nourriture, et je laisserai aux lecteurs le soin de tirer les conclusions. Relativement à l'avoine, mes informations sont exactes et complètes; mais comme il y a peu de personnes en Ecosse, qui vivent entièrement, ou même une partie de l'année, sur des pommes de terre, la consommation de cette racine n'a pas été examinée avec autant de soin. Farine d'avoine. — Les différens points que nous avons à considérer en traitant ce sujet sont : 1°. la quantité d'avoine produite dans un espace donné de bon terrain, un acre par exemple; 2°. le poids de la farine obtenue de cette avoine; 3°. la forme sous laquelle on emploie cette farine comme nourriture; 4°. le poids de la farine d'avoine requis\setcounter{page}{353} pour un repas, ou pour la nourriture d'une semaine d'un ouvrier, et cela quand elle est préparée de la manière la plus ordinaire. Nous allons examiner ces différens points.
\section{1°. Relativement au produit de l'avoine,} on sait que la moyenne est de soixante boisseaux par acre écossais (quatre acres écossais sont égaux à cinq anglais), tel est le produit moyen des districts les mieux cultivés. Mais il est de peu d'importance de fixer cette estimation, parce que lorsque nous comparerons le produit de l'avoine avec celui des pommes de terre, nous supposerons que le terrain est de la même qualité.
\section{2°. Les avoines de différentes espèces,} quoique produites par le même terrain, donnent différentes quantités de farine. L'avoine de la meilleure qualité, donnera environ pour six boisseaux, neuf stones écossaises (la stone étant de dix-sept livres), et cela sans allouer aucune déduction pour les meuniers, qui sont quelquefois payés en nature, d'où nous conclurons, que soixante boisseaux produisent 1475 livres de farine.
\section{3°. La farine d'avoine est préparée de différentes manières,} mais en Ecosse elle est plus généralement employée en guise de pain, sous la forme de gâteaux minces, ou en guise de soupes préparées avec du lait.\setcounter{page}{354} et de la bière. Nous ne parlerons maintenant que de sa préparation sous forme de soupe. Son augmentation de poids, suivant les différentes manières de la préparer, a été examinée par Mr. Robertson. "D'après des expériences faites avec soin" dit Mr. Robertson "j'ai trouvé que douze livres de farine d'avoine produisent en gâteaux durs. Liv. 12
Les mêmes préparées en soupes épaisses, ainsi que le peuple les fait ordinairement. 27
Les mêmes préparées sous la forme d'une soupe claire, mais ensuite rendues plus épaisses en les faisant bouillir long-temps. 39
Les mêmes sous la forme de sowens et de flummery\footnote{Ces deux mots qui n'en ont point de correspondance en français, sont presque synonymes et expriment une sorte de gelée faite avec de la farine de froment ou d'avoine.}. 56"
Cette dernière forme est recommandée par Mr. Robertson, dans les temps de disette, et il assure qu'elle est également saine et nutritive.
4°. Le poids de la farine, lorsqu'elle est préparée sous la forme de soupe, nécessaire à la nourriture d'un homme ordinaire, a été fixé de différentes manières; mais je m'en tiendrois à ce qui a été dit là-dessus par deux personnes intelligentes, qui paroissent avoir donné une attention particulière à ce\setcounter{page}{355} SUR LES REC. D'AVOINE ET DE POMM. DE T. (355) sujet. Dans différens endroits dix-sept livres et demie de farine, sont la ration ordinaire pour la semaine d'un homme; cette quantité est plus considérable, que celle qui est consommée en réalité, elle renferme un surplus, qui sert de payement pour la cuisson et pour le sel. "Quand l'homme fait usage d'un peu de lait, dit le Dr. Skene Keith, il lui faut quatorze livres par semaine, soit deux livres par jour; quand il a une quantité suffisante de lait, il ne consomme que douze livres de farine d'avoine; et s'il a du lait en abondance, et s'il vit sur-tout sur de la soupe claire, huit livres trois quarts de cette même farine sont suffisantes. " Suivant le même auteur, lorsqu'il s'agit même d'un ouvrage très-pénible, dix livres et demie de farine, réduites en soupe avec du lait, en supposant que l'on donne un gallon de lait par jour, nourrissent aisément un ouvrier pendant une semaine. Mr. Roberston observe, que pendant la moisson, lorsque les fermiers nourrissent leurs ouvriers, huit livres trois quarts de farine, font de la soupe pour douze personnes; ce qui fait onze onces deux tiers pour chaque repas, soit trente-cinq onces par jour, ou quinze livres et cinq onces par semaine, pour chaque personne. Mais, ajoute-t-il, il est\setcounter{page}{356} reconnu que cette quantité de farine est très-considérable.Dans le Kincardineshire en particulier, où les ouvriers de terre non-mariés, ont coutume de manger ensemble, chacun consomme quatorze livres de farine par semaine, ce qui fait deux livres par jour, ou dix onces deux tiers pour chaque repas ; et dans ce comté, il y a plusieurs personnes qui ne prennent que de la farine d'avoine, sans aucune autre nourriture solide, trois fois par jour, pendant toute l'année. Le Dr. Skene Keith s'accorde avec Mr. Robertson, et croit comme lui que deux livres de farine d'avoine, peuvent être regardées, comme la consommation journalière d'un ouvrier.
Suivant ce compte, mille cinq cent soixante-quinze livres de cette farine, qui sont le produit d'un acre de terrain, réduites en soupes et mangées avec un peu de lait, nourriront un homme pendant sept cent quatre-vingt-sept jours et demi, soit cent douze semaines et demie, ou deux ans huit semaines et trois jours et demi.
Il faut remarquer, avant d'abandonner ce sujet, que l'avoine fournit encore quelques produits, soit pour les hommes, soit pour les bêtes. Les sowens et le flummery, dont nous avons déjà parlé, s'obtiennent en faisant tremper des semences ou les bâles que l'on sépare de la farine par le crible, outre cela\setcounter{page}{357} les meuniers gardent ordinairement quelques rebuts de farine, dont ils se servent pour leurs cochons; il y a aussi une certaine quantité d'un grain inférieur, qui est séparée par le crible, puis enfin la paille. Tous ces rebuts ne peuvent pas valoir moins de 40 shellings par acre, soit qu'ils soient consommés par le bétail, soit qu'une partie ne serve qu'à faire du fumier.
Pommes de terre. Relativement à cet objet, il suffira de connoître, 1°. la quantité, ou le poids du produit par acre. 2°. La consommation journalière d'un ouvrier, qui vit sur cette substance. Afin de rendre notre comparaison plus exacte, nous y ajouterons la petite quantité de lait, dont il a été question lorsque nous avons parlé de la farine d'avoine.
1°. Le produit des pommes de terre prises sur le terrain, ne peut pas s'estimer à moins de huit tonnes par acre, en supposant que le terrain soit le même que celui qui a produit de l'avoine, et fumé à la manière ordinaire. Le poids de cette récolte, ainsi que celui des turneps et des autres plantes bulbeuses, dépend beaucoup de la quantité de fumier employé. Souvent à la vérité on obtient un plus grand produit que celui de huit tonnes, quand on peut se procurer du fumier\setcounter{page}{358} ad libitum. Mais à moins qu'ils ne soient dans le voisinage des grandes villes, les fermiers n'ont pas d'autres ressources que leurs propres fumiers, lesquels peuvent rarement en fournir plus de douze à quinze tonnes par acre, sans nuire aux autres récoltes. Un fermier sur lequel je puis compter et qui habite l'East Lothian, m'a assuré qu'il a eu souvent quatre-vingt-dix boisseaux d'avoine sur un acre écossais, mais qu'il n'a jamais tiré de ce même terrain, plus de huit tonnes de pommes de terre.
Avant de faire des recherches sur la consommation des pommes de terre, il seroit à souhaiter que l'on pût connoître la perte qui a lieu dans leur poids, depuis le moment où l'on les enlève de dessus le terrain, jusqu'à celui où on les mange. Je ne connois aucune expérience exacte à cet égard. Mais il est certain que cette perte doit être considérable. On sait que les pommes de terre perdent une partie de leur poids par la fermentation, après avoir été entassées, soit dans les magasins, soit sur le champ; les lavages diminuent aussi leur poids, elles en perdent environ un seizième par l'ébullition, et à moins qu'elles ne soient d'une très-belle qualité, elles en perdent encore autant, lorsqu'on les dépouille de leur peau. Dans chaque récolte aussi, il y a toujours une certaine quantité\setcounter{page}{359} de petites pommes de terre, qui ne valent pas la peine d'être préparées pour servir de nourriture, et dans plusieurs cas on en rejette une petite quantité, à cause de leur mauvaise qualité. Il est donc probable, que huit tonnes de pommes de terre, peuvent être réduites à sept tonnes, avant qu'elles puissent être prêtes à servir de nourriture pour les hommes, et que le quart ou la cinquième partie d'une tonne de rebut, devient la part des cochons ou de la volaille.
2°. Le poids des pommes de terre consommées ordinairement par un ouvrier dans chaque repas, a été fixé par un auteur à quatre livres, et par un autre à trois livres. Mais c'est de l'Irlande, que nous viennent les informations sur ce sujet, car dans ce pays les pommes de terre forment la seule nourriture de la grande masse du peuple, pendant neuf mois de l'année. Suivant le rapport du comité du Bureau d'agriculture, sur la culture et l'usage des pommes de terre, imprimé en 1795, il paroît qu'un homme, sa femme, et quatre enfans, mangent trois cent trente-six livres de pommes de terre par semaine, ce qui fait huit livres par jour pour chacun; dans un autre endroit du rapport, il est dit: que deux cent cinquante-deux livres par semaine suffisent, soit six livres pour chacun\setcounter{page}{360} par jour ; en général, sur une moyenne, deux cent soixante-six livres de pommes de terre par semaine, sont considérées comme une nourriture suffisante pour cette famille, ce qui fait environ six livres et demie par jour pour chacun. A cette dernière quantité de pommes de terre, on ajoute ordinairement un peu de lait de beurre pour le déjeûner et pour le souper. Dans le même rapport, il est dit, que deux cent quatre-vingts livres seront consommées par six personnes en six jours, ce qui fait environ huit livres par jour pour chacun. Maintenant, si nous supposons que la consommation des quatre enfans soit égale à celle de leur père et de leur mère, alors, quatre grandes personnes mangeront $8 \times 6 = 48$ livres par jour, ce qui fait douze livres pour chacun. Il paroit donc en général, que l'on peut regarder la quantité de douze livres de pommes de terre, comme la consommation journalière d'un homme. D'après ce que nous avons déjà dit relativement à la farine d'avoine, on sait que deux livres par jour sont suffisantes pour un ouvrier en Ecosse ; six livres de pommes de terre sont donc égales à une livre de farine d'avoine ; et cela s'accorde très-bien avec le rapport du Bureau, où il est dit, que quarante livres de farine équivaloient pour la nourriture de cette même\setcounter{page}{361} famille, à deux cent cinquante-deux livres de pommes de terre.
Ce poids est celui des pommes de terre crues ; dont le produit va, ainsi que nous l'avons vu, à huit tonnes par acre, soit 17920 livres ; cette somme, divisée par douze livres, qui expriment la consommation journalière, donne 1493 1/3, produit qui est le nombre de jours pendant lesquels un acre de terre pourra nourrir un ouvrier ; c'est-à-dire, quatre ans et trente-trois jours et demi. Un acre de pommes de terre
donne donc de la nourriture pour . . . 1493,333 jours.
Un acre d'avoine . . . . . . 787, 5
En faveur des pommes de terre 705,833
D'après ces résultats, il paroît que la quantité de nourriture produite par les pommes de terre dans un espace donné de terrain, a été extrêmement exagérée ; et dans le fait, une récolte de pommes de terre comparée avec une récolte d'avoine, fournit de la nourriture pour l'homme, dans la proportion de 2 à 1 à-peu-près. Mais il est évident, que pour que les résultats de cette comparaison soient de quelqu'utilité, il faudroit avoir égard à d'autres circonstances. L'avoine semée, et les pommes de terre plantées, devroient être déduites ; et ce qui est bien plus important, on devroit\setcounter{page}{362} avoir égard à la nourriture consommée par les ouvriers et les chevaux, pendant la culture de ces deux récoltes. Même après cette opération, la comparaison seroit bornée à une seule vue du sujet, savoir, à la quantité de nourriture; il faudroit en outre examiner, si une culture plus étendue des pommes de terre conviendroit aux intérêts des cultivateurs.
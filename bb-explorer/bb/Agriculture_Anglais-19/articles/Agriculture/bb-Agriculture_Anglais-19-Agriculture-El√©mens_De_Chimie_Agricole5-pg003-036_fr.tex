\setcounter{page}{3}
\chapter{AGRICULTURE.}
\section{Elements of Agricultural Chemistry, etc. Élémens de chimie-agricole en un cours de leçons pour le Département d'Agriculture; par Sir HUMPHRY DAVY. Londres, 1813, \large{(Cinquième extrait. Voy. p. 462 du vol. préc.)}}
On a déjà indiqué en général dans les leçons qui précèdent, quelle étoit la constitution de l'atmosphère: on a désigné l'eau, le gaz acide carbonique, l'oxygène et l'azote, comme ses principaux ingrédients; mais pour se former une idée juste du rôle que jouent ces fluides dans la végétation, il faut examiner de plus près leur nature particulière et leur mode d'action. C'est ce que je vais entreprendre: cette recherche fournira, je l'espère, des applications à la pratique de l'agriculture; elle présentera aussi quelques éclaircissemens sur la manière dont les plantes se nourrissent, dont leurs organes se déploient, et dont leurs fonctions se développent.\setcounter{page}{4} Si, après avoir fait chauffer jusqu'au rouge une dose du sel appelé muriate de chaux, on l'expose à l'air, même dans le temps le plus sec et le plus froid, son poids s'augmente, et il devient humide; même liquide, au bout d'un certain temps. Si on le met alors distiller dans une cornue, il ne rend que de l'eau pure; il reprend graduellement son premier état; et si on le chauffe jusqu'au rouge, on retrouve son premier poids. Il est donc évident que l'eau qu'il a acquise avoit été puisée dans l'air. Et on peut prouver que cette eau existoit dans l'air sous forme invisible et élastique; parce qu'en soumettant à l'action absorbante de ce sel un volume d'air donné, on trouve en employant un appareil convenable, que le volume et le poids de cet air sont moindres qu'auparavant.
La quantité absolue d'eau, contenue dans l'air sous forme de vapeur invisible, varie avec la température. A 50°. F. l'air contient environ $\frac{1}{20}$ de son volume de vapeur aqueuse; et, comme la pesanteur spécifique de cette vapeur est à celle de l'air, à-peu-près comme 10 à 15, cette proportion répond à environ $\frac{1}{75}$ de son poids.
A 100°. F. si l'air communique librement avec l'eau, il en prend, sous forme de vapeur,\setcounter{page}{5} environ $\frac{1}{14}$ de son volume, ou $\frac{1}{31}$ de son poids. C'est la condensation de cette vapeur, par la diminution de la température de l'atmosphère, qui est, selon toute apparence, la cause principale de la formation des nuages, et de la chute de la rosée, du brouillard, de la neige, ou de la grêle.
On a exposé dans la leçon précédente la faculté que possèdent diverses substances d'absorber, par attraction d'adhésion, la vapeur aqueuse de l'atmosphère. Les feuilles des plantes vivantes paroissent aussi agir sur cette vapeur élastique, et l'absorber. Quelques plantes augmentent en poids par cette cause, lorsqu'on les tient suspendues dans l'air, sans communication avec le sol: telles sont le sempervivum tectorum et diverses espèces d'aloës. Dans les températures très-élevées et lorsque le sol est desséché, la vie des plantes paroît être entretenue par la faculté absorbante de leurs feuilles et c'est une belle remarque à faire dans l'économie de la nature, que de voir la vapeur aqueuse devenir d'autant plus abondante dans l'atmosphère qu'elle est plus nécessaire à l'entretien de la vie; et cette ressource se multiplie lorsque toutes les autres tarissent.
Nous avons annoncé que l'eau n'étoit point\setcounter{page}{6} une substance simple. Il convient d'indiquer ici comment l'expérience prouve qu'elle peut se décomposer en deux ingrédiens; l'oxigène et l'hydrogène, et se reformer par la réunion de ces deux substances.
Si l'on met en contact dans un tube de verre, le métal appelé potassium; avec une petite quantité d'eau, il agit sur elle avec grande violence; il se dégage un fluide élastique, qui se trouve être l'hydrogène; et le potassium est modifié comme il le seroit s'il avoit absorbé une petite quantité d'oxigène. Le poids de l'hydrogène dégagé est à celui de l'oxigène uni au potassium, comme 1 à 15. Or, si l'on introduit dans un vase fermé, deux parties, en volume, d'hydrogène, et une en volume, d'oxigène, quantité dont les poids sont respectivement comme 2 à 15, et qu'on allume ce mélange par l'étincelle électrique, il s'enflamme et se condense en 17 parties d'eau pure.
Il est évident, d'après les faits établis dans notre troisième leçon, que l'eau forme la plus grande partie de la sève des plantes; et que cette substance, ou ses élémens, entrent pour beaucoup dans la constitution de leurs organes et de leurs produits solides.
L'eau, dans son état élastique, et dans l'état liquide, est absolument nécessaire à\setcounter{page}{7} l'économie de la végétation; et même elle n'est pas sans utilité sous sa forme solide; la glace, et sur-tout la neige sont de mauvais conducteurs de chaleur; et lorsque la terre est couverte de neige, et l'eau ou la surface du sol d'une croûte de glace, les racines, ou les bulbes des plantes au-dessous sont mises à l'abri par cette barrière de l'influence de l'atmosphère, dont la température dans les hivers du nord est à l'ordinaire fort au-dessous du terme de la congélation. Cette eau devient la première nourriture de la plante à l'arrivée du printems. L'expansion de l'eau pendant qu'elle se congèle, conversion dans laquelle son volume s'accroît de ¹/₁₃, et sa contraction, au dégel, tendent à pulvériser le sol, et à faciliter l'accès de l'air dans son intérieur. Si l'on expose à l'air une solution transparente de chaux dans l'eau, on voit bientôt se former à sa surface une pellicule, et tomber peu-à-peu au fond du liquide une matière solide; enfin l'eau perd au bout de quelque temps sa saveur caustique. Cette suite d'effets est due à la combinaison de la chaux dissoute dans l'eau, avec le gaz acide carbonique suspendu dans l'atmosphère. On peut le prouver, en recueillant la pellicule et le précipité dont on vient de parler.\setcounter{page}{8} et en les faisant chauffer fortement dans un tube de platine, ou de fer. Il en sort du gaz acide carbonique; et la matière solide redevient de la chaux-vive, qui, replacée dans cette même eau, s'y dissout de nouveau, et la reconstitue eau de chaux.
La quantité relative de gaz acide carbonique qui existe dans l'atmosphère est très-peu considérable. Il n'est pas facile de la déterminer avec précision : et elle varie selon les situations. Mais là où l'air circule librement, les limites de cette aliquote sont entre $\frac{1}{300}$ et $\frac{1}{800}$ du volume de l'air. Le gaz acide carbonique est plus pesant d'environ $\frac{1}{2}$ que les autres ingrédients élastiques, de l'atmosphère. On pourroit en inférer, au premier aperçu, qu'il devroit être plus abondant dans les couches basses de l'air; cependant il n'en est pas ainsi, sauf les cas où il a été immédiatement produit à la surface du sol par quelque procédé chimique. Les fluides élastiques, malgré les différences dans leurs pesanteurs spécifiques, ont une tendance à se mêler uniformément par une sorte d'attraction; et les vents, et d'autres causes agitent et délayent continuellement ensemble les différentes parties de l'atmosphère. De Saussure éprouva que l'eau de chaux se précipitoit, sur le Mont-Blanc le\setcounter{page}{9} point le plus élevé de l'Europe ; et on a toujours trouvé le gaz acide carbonique, en proportion ordinaire, dans tous les échantillons d'air atmosphérique que les aéronautes ont rapporté des hautes régions de l'atmosphère.
On peut montrer par expérience, d'une manière très-simple, la composition du gaz acide carbonique. Si l'on allume au verre ardent un poids de treize grains de charbon bien préparé, sous une cloche de verre qui contienne 100 pouces cubes de gaz oxigène, le charbon disparaîtra dans sa totalité ; et si l'expérience est bien conduite, tout l'oxigène, à l'exception peut-être de quelques pouces cubes, se trouvera converti en acide carbonique. Et, ce qui est très-remarquable, le volume du gaz demeure sensiblement le même. Cette dernière circonstance permet d'estimer d'une manière bien simple les quantités respectives de charbon pur (carbone) et d'oxigène, qui constituent le gaz acide carbonique ; le poids de 100 pouces cubes de gaz acide carbonique est à celui de 100 pouces cubes de gaz oxigène, comme 47 à 34 ; de manière que 47 parties, en poids, de gaz acide carbonique doivent être composées de 34 parties d'oxigène et de 13 de carbone. Ce sont précisément.\setcounter{page}{10} les nombres indiqués dans notre seconde leçon.
On décompose facilement l'acide carbonique en l'exposant à l'action du potassium, aidée de la chaleur. Le métal se combine avec l'oxygène, et le carbone se dépose, sous l'apparence d'une poudre noire.
La principale consommation de l'acide carbonique dans l'atmosphère paroît avoir pour objet la nourriture des plantes; quelques-unes paroissent tirer principalement de cette source le carbone dont elles sont en partie composées.
Le gaz acide carbonique se forme pendant la fermentation, la combustion, la putréfaction, la respiration, et un nombre d'opérations qui ont lieu à la surface du globe. On ne connoît pas d'autre procédé naturel, que la végétation, qui tende à le décomposer.
Après qu'une portion donnée d'air commun a été privée de la vapeur aqueuse et du gaz acide carbonique, qu'elle renferme ordinairement, elle ne paroît que bien peu altérée dans ses propriétés; elle entretient, comme auparavant, la combustion et la vie animale. On a plusieurs moyens de séparer l'un de l'autre ses deux ingrédients, l'oxygène et l'azote. Un des plus simples est de\setcounter{page}{11} brûler du phosphore dans un volume d'air donné et renfermé; ce combustible absorbe l'oxigène, et laisse l'azote. Cent parties de cet air, dans lequel le phosphore a brûlé jusqu'à extinction, donnent en résidu 79 parties d'azote; et en mêlant cet azote avec 21 parties de gaz oxigène préparé artificiellement, on produit un fluide élastique qui possède toutes les propriétés caractéristiques de l'air commun. Pour se procurer du gaz oxigène extrait de l'air ordinaire, il faut tenir dans cet air pendant un temps un peu long une certaine quantité de mercure à la température d'environ 600° F.; il y passe à l'état d'une poudre rouge; si on élève ensuite cette même poudre à la température voisine de l'ignition, elle repasse à l'état de mercure, et abandonne l'oxigène, qui prend l'état de gaz.
. L'oxigène est essentiel à quelques-unes des fonctions des végétaux; mais c'est surtout dans ses rapports avec l'économie animale, qu'il joue un rôle important dans la nature. Sa présence est absolument nécessaire à la très-grande pluralité des animaux. L'air atmosphérique, introduit par la respiration dans les poumons des animaux, ou passant, au moyen de la dissolution dans l'eau, dans les ouïes des poissons, perd de\setcounter{page}{12} l’oxigène ; et le volume de ce gaz, qui disparaît, est remplacé par un volume égal de gaz acide carbonique.
On ne connoît pas bien le rôle que joue l’azote dans la végétation ; comme on le trouve dans l’analyse de quelques-uns des produits végétaux, il peut être absorbé de l’atmosphère par certaines plantes ; il empêche que l’action de l’oxigène ne soit trop énergique, et sert comme d’intermède ou de milieu pour l’action des parties plus essentielles de l’air. Cette circonstance est dans les analogies de la nature ; car on peut remarquer que les élémens les plus abondans à la surface solide du globe, ne sont pas précisément les plus essentiels à l’existence des êtres organisés.
L’action de l’atmosphère sur les plantes n’est pas la même dans les diverses périodes de leur accroissement ; et elle varie avec les degrés de développement ou de décadence de leurs organes. On peut déjà s’être formé une idée générale de son influence, d’après les circonstances mentionnées ; je vais entrer dans quelques détails sur cet objet, en tâchant de le lier à des considérations générales sur la marche de la végétation.
Si après avoir humecté une graine, on l’expose à l’air dans une température qui\setcounter{page}{13} ne soit pas au-dessous de 45° F., elle ne tarde pas à germer. Il en sort une plumule, qui s'élève, et une radicule, qui descend.
Si cette opération s'exécute dans un air renfermé, on trouve que, dans le procédé de la germination l'oxigène est absorbé, en tout, ou en partie. L'azote demeure intact; et loin qu'il disparoisse de l'acide carbonique, on en trouve dans cet air un peu plus qu'il n'y en avoit auparavant.
Les graines ne peuvent germer sans la présence de l'oxigène. Lorsqu'on les place humectées, dans le vide pneumatique, dans l'azote, ou dans l'acide carbonique pur, elles se gonflent, mais elles ne végètent pas; et si on les tient quelque temps dans ces gaz, elles perdent leur force vitale et subissent la putréfaction.
Si l'on examine une graine avant sa germination, on la trouvera plus ou moins insipide, et tout au moins, sans saveur sucrée ; elle la prend d'abord après la germination. Son mucilage coagulé, ou sa fécule, se convertit en sucre par ce procédé naturel; ainsi, une substance de solution difficile se change en une autre facilement soluble; et le sucre, charrié dans les cellules, ou les petits vaisseaux des cotylédons, devient l'aliment de la jeune plante. Les faits\setcounter{page}{14} développés dans notre troisième leçon peuvent faire comprendre aisément la nature de cette métamorphose ; et la production de l'acide carbonique donne de la probabilité à l'idée que la principale différence chimique entre le sucre et le mucilage dépend d'une différence assez légère dans les proportions de leur carbone.
On a comparé l'absorption de l'oxigène par la graine dans la germination, à celle qui a lieu dans le développement de la vie du fœtus dans l'œuf. Mais ce n'est là qu'une analogie éloignée. Tous les animaux, depuis les classes inférieures jusques aux plus parfaites, ont besoin du secours de l'oxigène\footnote{Les œufs imprégnés des insectes, et même ceux des poissons, ne sont pas productifs s'ils ne sont en contact avec l'air, c'est-à-dire, si les fœtus ne peuvent pas respirer. J'ai trouvé que les œufs des phalènes ne produisoient pas de larves lorsqu'on les renfermoit dans l'acide carbonique pur ; et lorsqu'on les plaçoit dans l'air commun, une portion de l'oxigène disparoissait et il se formoit de l'acide carbonique. Le poisson dans son œuf prend l'oxigène de l'air qui est dissous dans l'eau ; et les poissons qui fraient au printems et en été dans l'eau tranquille, tels que le brochet, la carpe, la perche, etc. déposent leurs œufs sur des plantes qui vivent sous l'eau, et dont les feuilles, dans l'exercice de leurs fonctions vitales, fournissent l'oxigène à l'eau. Les poissons qui fraient en hiver, tels que le saumon et}.\setcounter{page}{15} Depuis l'instant où le cœur commence à battre, jusques à celui où ses pulsations cessent, le sang est constamment imprégné par l'air; et la fonction de la respiration est invariable \footnote
(a) Excepté toutefois pendant la gestation. (R){la truite, cherchent les lieux où il y a une affluence constante d'eau vive, aussi près des sources qu'il est possible, et dans les courans les plus rapides, loin de toute stagnation, et là où l'eau est saturée de l'air auquel elle a été exposée pendant qu'elle se séparoit de lui dans les nuages. C'est l'instinct, qui conduit ces poissons à chercher l'air pour leurs œufs, qui les fait quitter les mers ou les lacs pour remonter dans les rivières contre les courans les plus rapides, et même contre les chutes naturelles ou artificielles de l'eau jusques à des hauteurs surprenantes. (A)}. Il se produit de l'acide carbonique dans le procédé; mais le changement chimique qui a lieu sur le sang est inconnu; et il n'y a pas lieu de supposer qu'il s'y forme aucune substance analogue au sucre. Dans la production d'une plante provenant d'une semence, il faut quelque réservoir alimentaire avant que la sève puisse monter par la racine. Ce réservoir est le cotylédon, dans lequel l'aliment est emmagasiné, sous forme insoluble, et protégé, s'il est nécessaire, pendant l'hiver; puis rendu soluble par des agens, qui sont constamment à sa surface. Le changement de la fécule\setcounter{page}{16} en sucre, lié avec l'absorption de l'oxygène, peut plutôt être comparé à un procédé de fermentation qu'à celui de la respiration. C'est un changement produit sur la matière non organisée et qu'on peut imiter artificiellement; et dans la plupart des changemens chimiques qui ont lieu lorsque les composés végétaux sont exposés à l'air, l'oxygène est absorbé, et l'acide carbonique formé, ou développé.
Il est évident, que dans tous les cas où une graine est mise en terre pour y germer, il faut la semer de manière qu'elle y éprouve pleinement l'influence de l'air; et l'une des causes qui rendent les sols glaiseux si peu fertiles est l'obstacle qu'oppose la force adhésive de la glaise, à l'action de l'air sur la semence, qui en est revêtue de toutes parts.
Dans les terres sableuses, le sol est toujours facile à pénétrer par les influences atmosphériques; mais dans les argileuses, on ne sauroit, pour ainsi dire, trop chercher à les diviser par des moyens mécaniques dans les travaux de la culture. Toute graine qui ne reçoit pas sa dose d'air suffisante, produit toujours une plante foible et maladive.
Le procédé du brasseur, dont on a déjà parlé,\setcounter{page}{17} n'est qu'une germination hâtée et artificielle, dans laquelle la fécule du cotylédon est changée en sucre, lequel, à son tour, devient de l'esprit ardent par la fermentation.
Il est très-évident, d'après les principes chimiques de la fermentation, que le procédé préliminaire du brasseur ne doit pas aller plus loin que ce qui est nécessaire pour faire pousser la radicule, et qu'il faut l'arrêter dès qu'elle se montre distinctement. Si on le pousse jusques au développement parfait de la radicule et de la plumule, ce développement n'aura lieu qu'aux dépens d'une quantité considérable de matière sucrée, et on aura d'autant moins d'esprit ardent, dans la fermentation.
Comme cette circonstance est de quelque importance, je fis au mois d'octobre 1806 une expérience sur cet objet. Je déterminai par l'action de l'alcohol, les quantités relatives de matière sucrée dans deux quantités égales de la même orge; dans l'une d'elles, la germination s'étoit avancée jusqu'à produire dans la plupart des grains, une radicule d'environ un quart de pouce; dans l'autre, on l'avoit arrêtée avant qu'elle eût une ligne de long. La quantité de sucre fournie par cette dernière, comparée à celle que\setcounter{page}{18} donne la première, fut dans le rapport approché de six à cinq.
La matière sucrée contenue dans les cotylédons à l'époque de leur changement en feuilles séminales, les expose beaucoup aux attaques des insectes, car ce principe est nutritif pour les animaux comme pour les plantes; et on voit souvent les insectes causer de grands ravages dans les semis à cette première époque de la végétation.
La mouche du turnep, insecte du genre des coeoptères, se fixe sur les feuilles séminales de cette plante à l'époque où elles commencent leurs fonctions; mais lorsque les feuilles grossières de la plumule sont développées, l'insecte ne peut plus nuire essentiellement à la plante.
On a proposé divers moyens pour détruire cette mouche, ou pour en préserver les jeunes plantes; on imaginoit de semer de la graine de radis en même temps que celle de turnep, dans l'idée que l'insecte attaqueroit de préférence la première de ces deux plantes, et laisseroit l'autre. On dit que ce procédé n'a pas eu de succès, et que le moucheron attaquoit indifféremment l'une et l'autre.
Il existe plusieurs dissolvans chimiques, qui peuvent accélérer beaucoup le procédé de la germination lorsqu'on en mouille préalablement\setcounter{page}{19} lablement les graines. Comme, dans ces cas, les feuilles séminales paroissent plus promptement et commencent plus tôt leurs fonctions, j'ai proposé comme sujet d'expérience de tenter si l'on ne pourroit pas se prévaloir de l'action de ces dissolvans pour amener plus rapidement la jeune plante à l'état dans lequel elle est à l'abri de l'insecte. Mais le résultat n'a pas répondu à mon espérance; les graines traitées de cette manière germoient bien plus vite, il est vrai, mais elles ne produisoient pas des plantes vigoureuses, et la plupart mouroient après leur premier jet. Au mois de septembre 1807, je mis tremper pendant douze heures des doses égales de graine de radis, dans des solutions de chlorine; d'acide nitrique très-étendu; d'acide sulfurique également délayé; d'une solution foible d'oxi-sulfate de fer; enfin, d'eau commune. Les graines plongées dans le chlorine et l'oxi-sulfate poussèrent leur germe en deux jours; dans l'acide nitrique, en trois jours; dans l'acide sulfurique, en cinq; et dans l'eau, en sept jours. Mais, dans les cas de germination prématurée, quoique la plantule parût très-vigoureuse pendant un temps court, au bout de quinze jours elle devint foible, et moins vigoureuse à cette époque, que les pousses qui s'étoient développées\setcounter{page}{20} naturellement ; de manière que je doute de l’utilité probable de ce procédé. Il semble que, dans les corps organisés, la décadence prématurée soit une conséquence nécessaire de l’accroissement trop rapide ; il ne faut donc chercher les perfectionnemens que par une marche assortie à celle de la nature elle-même, qui prend son temps pour chaque chose.
Entre les substances chimiques il y en a qui attaquent, même mortellement, les insectes, sans nuire aux plantes, et même quelquefois en favorisant leur végétation. On a essayé plusieurs de ces mélanges avec des succès divers ; un mélange de soufré et de chaux, qui tue les limaçons ne prévient pas les ravages de l’insecte des turneps ; le duc de Bedford eut la bonté de faire en grand dans ses fermes de Woburn, l’essai de ce procédé ; les insectes attaquèrent la portion saupoudrée du mélange, à peu près comme celle qui ne l’avoit pas été.
Des mélanges de suie et de chaux vive, ou d’urine et de chaux, seroient probablement plus efficaces. L’alkali volatil qui se dégage de ces mélanges est nuisible aux insectes, et contribue à nourrir les plantes. Mr. T. A. Knight \footnote{Voici la note que Mr. Knight a eu la bonté de m’adresser à ce sujet.
"L’expérience que j’ai tentée l’année dernière et la précédente, pour mettre les turneps à l'abri de l'insecte qui les attaque, n'a pas été assez répétée pour me permettre de prononcer définitivement. L'an passé tous mes turneps ont parfaitement réussi. D'après l'idée dont vous m'aviez fait part il y a quelques années dans notre entrevue à Holkham, que la chaux éteinte dans l'urine pourroit peut-être tuer ou repousser les insectes en question, j'essayai cette préparation, mêlée avec trois parties de suie; je mis le tout dans un barril percé de trous au forêt, par lesquels une certaine quantité de cette composition (environ 4 bushels par acre) pouvoit sortir et tomber dans les sillons avec la graine de turneps. Je ne sais si je dois attribuer l'effet à la force éminemment stimulante que ce mélange procura à la végétation, ou à une saveur qui déplut aux insectes; mais en 1811, les sillons voisins de ceux ainsi préparés furent mangés, et les autres, à peine touchés. Je me propose à l'avenir de semer d'abord une première dose de graine dans le sillon, en arrosant de la composition l'arrête de ce même sillon; ensuite, de semer à la volée au moins une livre de graine sur tout le terrain. Ce procédé ne sera pas couteux (2 sh. par acre) et la houe à cheval (le cultivateur) me débarrassera de toutes les plantes surnuméraires entre les lignes, si ces plantes échappent aux mouches qui les auront attaquées de préférence; car j'ai toujours remarqué que ces insectes aiment mieux les turneps qui croissent dans un sol maigre, que ceux dont la végétation est plus vigoureuse. Un des avantages de cette pratique paroît être l'accélération procurée à l'accroissement de la plante, par l'effet très-stimulant de la nourriture qu'elle reçoit dès que son germe s'est développé, et long-temps avant que les radicules aient atteint le fumier du sol. Les directions ci-dessus ne s'appliquent qu'aux turneps semés sur la crête des sillons, avec l'engrais immédiatement au-dessous; et je suis tout-à-fait persuadé que dans tout terrain le turnep devrait être cultivé de cette manière. La grande proximité entre la plante et son engrais, et le temps très-court requis pour charrier l'aliment dans la feuille, et ramener la matière organisable aux racines sont, dans mon hypothèse, des objets d'une grande importance : et les résultats que donne la pratique s'accordent avec cette théorie.} m’apprend qu’il a essayé\setcounter{page}{21} avec succès l'emploi des vapeurs ammoniacales; mais il faut un plus grand nombre\setcounter{page}{22} d'essais pour établir bien décidément l'efficacité du procédé. Au demeurant, on ne court aucun risque en l'employant, car s'il ne réussit pas à préserver la plante de l'insecte, il agit utilement comme engrais.
Lorsque les racines et les feuilles de la jeune plante sont formées, les cellules et les tubes qui composent son intérieur se remplissent d'un liquide, ordinairement fourni par le sol; et la fonction de la nutrition a lieu par l'action des organes de la plante sur les élémens extérieurs. Les parties constituantes de l'air concourent dans ce procédé; mais, comme on peut s'y attendre, elles agissent différemment dans différentes circonstances.
Lorsqu'une plante qui végète et dont les racines reçoivent la nourriture convenable,\setcounter{page}{23} est exposée en présence de la lumière solaire, à une quantité donnée d'air atmosphérique, qui contient sa proportion ordinaire d'acide carbonique, cet acide disparoit au bout d'un certain temps, et il est remplacé par une dose d'oxygène; à mesure qu'on introduit du nouvel acide carbonique, on voit se manifester la même absorption, de manière qu'on peut en conclure qu'une plante qui végète sous l'influence des rayons solaires, gagne du carbone, et donne à la place de l'oxygène à l'air environnant.
Ce fait est prouvé par un nombre d'expériences faites par MM. Priestley, Ingénhousz, Woodhouse, et Th. De Saussure\footnote{On doit aussi beaucoup d'expériences de ce genre à Mr. Senebier. Il y a lieu de s'étonner que l'auteur n'en fasse pas mention ici, puisqu'il le cite peu après.}.
J'en ai répété beaucoup, avec des résultats semblables. C'est la feuille particulièrement, qui absorbe l'acide carbonique et qui rend l'oxygène; et lors même qu'elles sont séparées de la plante, si on les place dans l'air qui contient l'acide carbonique, ou dans de l'eau qui en soit imprégnée, elles continuent à produire cet échange.
L'acide carbonique est probablement absorbé par les liquides contenus dans les\setcounter{page}{24} cellules de la partie verte ou parenchymateuse de la feuille; et c'est aussi de cette partie que sort l'oxygène pendant l'action de la lumière solaire. Mr. Senebier a trouvé que la feuille dont on a enlevé l'épiderme, continuoit à produire l'oxygène lorsqu'on la plaçoit dans de l'eau imprégnée de gaz acide carbonique, et on voyoit sortir des globules d'air du parenchyme mis à nud. Les expériences de ce physicien et celles de Mr. Woodhouse, montrent que les feuilles dans lesquelles cette partie est la plus abondante, sont celles qui donnent le plus d'oxygène dans l'eau imprégnée d'acide carbonique. Quelques plantes, en petit nombre \footnote{J'ai trouvé que l'Arenaria tenuifolia produisoit l'oxygène dans l'acide carbonique, qui étoit à-peu-près pur: (A)} peuvent végéter dans une atmosphère artificielle, composée principalement d'acide carbonique; et plusieurs peuvent vivre pendant quelque temps dans un air qui en contient depuis un tiers jusques à la moitié de son volume; mais elles ne végètent pas aussi bien que si la dose étoit moindre. On a trouvé que les plantes exposées à la lumière produisoient le gaz oxygène dans un milieu élastique et dans l'eau qui ne contenoit point de gaz acide carbonique; mais\setcounter{page}{25} en quantité beaucoup moindre que lorsque le gaz acide carbonique était présent.
Dans l'obscurité, les plantes ne produisent point de gaz oxygène, quel que soit le milieu élastique dans lequel elles végètent; et elles n'absorbent point de gaz acide carbonique. Au contraire, et pour l'ordinaire, l'oxygène (s'il est présent) est absorbé, et il se produit du gaz acide carbonique.
Dans les changemens qu'éprouve la composition des parties organiques, il est probable que les composés saccharins se forment principalement pendant l'absence de la lumière; les gommes, les fibres ligneuses, les huiles, et les résines, au contraire, sous son influence; et il est possible que le dégagement du gaz acide carbonique, ou sa formation pendant la nuit, soit nécessaire pour donner une plus grande solubilité à certains composés dans la plante. Je soupçonne une fois que tout le gaz acide carbonique, produit par les plantes dans l'obscurité, ou à l'ombre, pourrait bien provenir de la décomposition de quelque partie de la feuille ou de l'épiderme; mais les expériences récentes de Mr. D. Ellis sont opposées à cette idée; et j'ai trouvé qu'une plante de céleri parfaitement bien portante, placée pendant peu d'heures seulement, dans une portion d'air donnée,\setcounter{page}{26} y occasionna une production de gaz acide carbonique, et une absorption d'oxygène. Quelques personnes ont supposé que les plantes exposées, à l'air libre, aux vicissitudes du soleil et de l'ombre, de la lumière et des ténèbres, consument plus d'oxygène qu'elles n'en produisent, et que leur action permanente sur l'air est semblable à celle des animaux. L'auteur que je viens de citer adopte cette opinion en traitant ce sujet dans ses ingénieuses recherches sur la végétation. Mais, toutes les expériences par lesquelles on a tenté d'appuyer cette idée, et particulièrement les siennes, ont été faites dans des circonstances peu favorables à l'exactitude des résultats. Les plantes ont été renfermées et nourries d'une manière étrangère à leur nature; et l'influence de la lumière sur elles a été fort diminuée par la présence des milieux au travers desquels on l'a forcée de passer. Les plantes confinées dans des portions limitées d'air atmosphérique, sont bientôt en état de souffrance; leurs feuilles se fanent, se décomposent, et alors détruisent rapidement l'oxygène de l'air. Dans quelques-unes des premières expériences du Dr. Priestley, avant qu'il connût l'action de la lumière sur les feuilles, il remarqua que l'air qui avait supporté la combustion et la respiration, étoit purifié par la végétation des\setcounter{page}{27} plantes qu'on y renfermoit pendant quelques jours et quelques nuits de suite; et ses expériences méritent d'autant plus de confiance que les plantes dans plusieurs d'entr'elles croissoient dans leur état naturel; et qu'on se bornoit à en introduire des pousses, ou des branches au travers de l'eau dans l'air renfermé.
J'ai fait quelques recherches sur cet objet, et je vais en donner les détails. Le 12 juillet 1800 je plaçai un gazon de quatre pouces en quarré garni d'herbe, et sur-tout de queue de renard, (alopecurus pratensis) et de trèfle-blanc, dans un plat de porcelaine, posé sur un plus grand, peu profond, et rempli d'eau. Je couvris le tout d'un récipient de flint-glass, qui contenoit 380 p. c. d'air commun, dans son état naturel. L'appareil étoit exposé dans un jardin, c'est-à-dire, soumis aux mêmes influences de la lumière que s'il eût été à l'air libre. Le 20 juillet, on examina les résultats. Le volume du gaz s'étoit augmenté de 15 pouces cubes; mais la température s'étoit élevée de 64 à 71, et la pression de l'atmosphère qui, le 12, soutenoit 30,1 pouces de mercure, en soutenoit 30,2 le 20\footnote{L'auteur ne donne ici que les élémens de la réduction à faire du gaz au même volume dans les deux époques comparées. Nous trouvons que la différence de température de la première à la seconde l'augmentoit de 0,016, et que la différence de pression le diminuoit de 0,005; ces aliquotes, appliquées avec leurs signes respectifs au volume absolu (380 p. c.) deviennent - 6,080, et + 1,140; différence - 4,940 pouc. cub. ou, en nombres ronds, 5 pouces cubes à soustraire de l'augmentation apparente de 15 p. c.; ce qui la réduit à 10 soit 1/3 du volume primitif. (R)}. Quelques-unes\setcounter{page}{28} des feuilles du trèfle-blanc et de la queue de renard avoient jauni; et le gazon entier paroissoit moins végétant qu'il ne l'étoit à l'époque de son introduction. Un pouce cube du gaz, agité dans l'eau de chaux, la troubla légèrement; et l'absorption fut d'un peu moins de 1/8 du volume. 100 parties du gaz résidu, exposées à une solution de sulfate vert de fer, imprégnée de gaz nitreux (substance qui absorbe rapidement l'oxigène de l'air) furent réduites à 80 parties; 100 parties de l'air du jardin soumises à la même épreuve, furent réduites à 79. Si l'on part du résultat de cette expérience, il sembleroit que l'air avoit été légèrement détérioré par l'action des gazons. Mais il faut remarquer que le temps avoit été beaucoup plus couvert qu'à l'ordinaire, pendant ces huit jours; les plantes n'avoient pas été fournies de gaz acide carbonique d'une manière naturelle; et la quantité qui\setcounter{page}{29} s'en formoit pendant la nuit et par l'action des feuilles flétries, dut avoir été en partie dissoute par l'eau. Je prouvai que la conjecture étoit fondée en versant dans cette eau de l'eau de chaux, qui produisit de suite un précipité. Je serois disposé à attribuer l'augmentation d'azote à l'air commun dégagé de l'eau.
Je regarde l'expérience suivante comme faite dans des circonstances plus analogues à celles qui existent dans la nature. Je mis dans un bassin de porcelaine qui flottoit sur de l'eau imprégnée de gaz acide carbonique, un gazon de quatre pouces en quarré, pris dans un pré arrosé, garni de meadow-grass (variété de poa) d'alopecurus, et de vernal meadow-grass (anthoxanthum) on couvrit le bassin d'un récipient de flint-glass, mince, terminé en haut par un entonnoir à robinet ; et on plaça l'appareil en plein air. On arrosoit le gazon tous les jours, au moyen du robinet. De même tous les jours on enlevoit avec un syphon une certaine quantité de l'eau qu'on remplacoit par de l'eau saturée de gaz acide carbonique; de manière qu'on peut croire qu'une certaine quantité de cet acide étoit toujours présente dans l'intérieur du récipient.
Le 7 juillet 1807, premier jour de l'expérience, le temps fut couvert le matin, mais clair l'après-midi. Le therm. à 67½ F. barom.\setcounter{page}{30} 30, 2. On s'aperçut vers le soir, d'un léger accroissement dans le volume du gaz; les trois jours suivants furent sereins, mais le matin du 11, le ciel se couvrit. On observa alors une augmentation considérable dans le volume du gaz; le 12 fut couvert, avec quelques éclaircis; il y eut aussi accroissement du gaz, mais moindre que dans les jours sereins. Vers neuf heures du soir le 14, le récipient étoit tout plein; et en comparant ce volume à celui que renfermoit primitivement l'appareil, il devoit avoir acquis au moins 30 p. c. de fluide élastique. Il s'échappa quelques globules de gaz dans le cours de la journée. A dix heures du matin, le 15, j'examinai une portion du gaz; il contenoit moins de ⅛ de gaz acide carbonique; 100 parties du gaz de la cloche exposées à la solution de sulfate imprégnée, se réduisirent à 75; ensorte que cet air étoit de 4 p. °/° plus pur que celui de l'atmosphère. Je détaillerai encore une autre expérience, dont les résultats furent également décisifs. On courba une branche de vigne, qui avoit trois feuilles en forte végétation, et qui demeuroit attachée à la souche, de manière à la faire passer sous le récipient employé dans la dernière expérience. On maintint de même l'eau qui contenoit cet air, imprégnée de gaz acide carbonique. L'expérience\setcounter{page}{31} dura du 6 août au 14 (1807). Pendant ce temps, quoique le ciel eût été en général couvert et qu'il fût même tombé de la pluie, le volume du fluide élastique ne cessa point de s'augmenter. On examina sa qualité dans la matinée du 15; il contenoit ¹⁄₄₃ de gaz acide carbonique; et 100 parties de ce gaz en contenaient 23,5 de gaz oxigène.
Ces faits confirment l'opinion, déjà populaire, que, lorsque les feuilles des plantes sont en pleine végétation, elles tendent à purifier l'atmosphère dans les variations ordinaires du temps et ses alternatives de lumière et d'obscurité.
Dans la germination, et dans la décadence des feuilles, il doit y avoir absorption d'oxigène; mais, si l'on considère la grande étendue de la portion du globe couverte de plantes perennes, et que la moitié de sa surface est toujours exposée aux rayons solaires, il paroîtra bien probable qu'il se produit plus d'oxigène qu'il ne s'en consomme, dans l'ensemble des procédés de la végétation; et que c'est à cette circonstance qu'il faut principalement attribuer l'uniformité qui existe dans la constitution chimique de l'atmosphère.
Les animaux, loin de produire de l'oxigène dans aucune de leurs fonctions vitales, en consomment continuellement, plus\setcounter{page}{32} ou moins. Mais l'étendue du règne animal est très-petite, comparée à celle du règne végétal; et la quantité de gaz acide carbonique produite dans la respiration et dans les diverses combustions et fermentations qui s'opèrent à la surface du globe, est une très-faible aliquote du volume total de l'atmosphère; et si chaque plante pendant la durée de sa vie, fait une petite addition à l'oxygène de l'air, et consomme un peu d'acide carbonique, cette compensation peut suffire au maintien de l'équilibre dans la constitution atmosphérique.
On pourroit objecter à ces vues, que, si les feuilles des plantes purifient l'atmosphère, la fin de l'automne, l'hiver, et le commencement du printems devroient, dans nos climats, être des saisons peu salubres, à raison de la diminution d'oxygène et de l'augmentation d'acide carbonique dans l'air, qui devroient résulter de l'absence des feuilles en végétation: mais, on peut résoudre cette objection d'une manière satisfaisante. Les vents, dont quelques-uns se meuvent avec une vitesse de 60 à 100 milles à l'heure, agitent continuellement l'atmosphère, et en mêlent ensemble toutes les parties. Dans notre hiver, les vents violens du SO amènent en Europe l'air purifié par les vastes forêts de l'Amérique\setcounter{page}{33} méridionale; c'est par ces grands mouvements que l'équilibre se conserve dans les qualités salubres de l'air, et qu'il demeure toujours approprié à l'entretien de la vie. Ces événements, qu'une ignorance superstitieuse attribuait jadis à la colère céleste ou à l'influence des esprits malfaisans, et dans lesquels elle n'apercevait que désordre et confusion, la science nous les montre aujourd'hui comme les dispensations utiles d'une intelligence suprême, et comme intimément associés à l'ordre et, à l'harmonie de notre système.
J'ai réclamé, dans la première partie de cette leçon, contre un rapprochement forcé que quelques personnes ont voulu faire entre l'absorption de l'oxigène et la formation du gaz acide carbonique dans la germination, et la respiration du fœtus. Je puis avancer des arguments semblables contre l'existence d'une analogie qu'on prétendrait exister entre les fonctions des feuilles de la plante adulte, et celle des poumons de l'animal respirant. Les plantes ne végètent bien qu'en présence de la lumière; la plupart périssent lorsqu'on les en prive. On ne peut supposer que la production de l'oxigène par le travail de la feuille, production qu'on sait être liée avec la conservation qu'on sait être liée avec la conservation\setcounter{page}{34} de la couleur naturelle de cet organe, soit le résultat d'une fonction qui supposerait un état de maladie, ou que la plante peut acquérir du carbone en plein jour, lorsqu'elle est dans toute sa vigueur, lorsque la sève monte, et lorsqu'elle déploie toutes ses facultés de nutrition, uniquement dans le but de le rendre la nuit, lorsque ses feuilles sont fermées, que le mouvement de la sève est imparfait, et que la plante est dans une sorte de sommeil. Un nombre de plantes qui croissent sur les rochers, ou dans les terrains qui ne contiennent point de carbone, ne peuvent puiser ce principe que dans l'atmosphère; et la feuille peut être considérée, en même temps comme un organe d'absorption, et aussi comme un organe dans lequel la sève peut éprouver divers changemens chimiques.
Lorsque les racines des plantes n'absorbent que de l'eau pure, ce fluide, en arrivant aux feuilles, y acquiert probablement une faculté plus particulière d'absorber l'acide carbonique de l'atmosphère. Lorsque l'eau en est saturée, les feuilles peuvent en laisser échapper une certaine quantité, même sous l'influence de la lumière solaire; mais une partie aussi y est décomposée; comme l'ont prouvé les expériences de Mr Senebier.\setcounter{page}{35} Lorsque le liquide absorbé par les racines des plantes contient beaucoup de matière carbonacée, il est probable que les feuilles lâchent de l'acide carbonique, même au soleil. En un mot, la fonction de la feuille doit varier selon la composition de la sève qui l'abreuve, et selon la nature des produits qu'elle forme. Lorsque ce produit doit être du sucre, ainsi que cela arrive aux premiers jours du printemps, à l'époque du développement des boutons et des fleurs, il est probable qu'il sort moins d'oxigène de la plante qu'à l'époque où la plante mûrit, et où la fécule, la gomme, les huiles, se forment; et c'est dans la saison du soleil le plus intense, que la maturité des graines a ordinairement lieu. Lorsque, dans le procédé de la végétation, les sucs acides des fruits passent à l'état sucré, il y a lieu de croire que la plante doit émettre plus d'oxigène, ou en faire plus entrer dans de nouvelles combinaisons, que dans d'autres périodes de son existence. Car, ainsi qu'on l'a montré dans la troisième leçon, tous les acides végétaux contiennent plus d'oxigène que le sucre. Il paroît probable que, dans quelques cas dans lesquels les corps huileux et résineux se forment dans l'acte de la végétation, l'eau peut être décomposée, son\setcounter{page}{36} oxygène être mis en liberté, et son hydrogène absorbé.
J'ai déjà dit que quelques plantes produisent de l'oxygène dans l'eau pure. Le Dr. Ingenhousz a trouvé cette faculté dans plusieurs espèces de confers; j'ai essayé les feuilles de beaucoup de plantes, surtout de celles qui produisent les huiles volatiles. Lorsque ces feuilles sont exposées dans l'eau saturée de gaz oxygène, celui-ci se dégage sous l'influence des rayons solaires; mais sa quantité est peu considérable, et toujours limitée; et je n'ai jamais pu découvrir avec certitude si l'effet était dû exclusivement aux forces de végétation de la plante; quoique cela paroisse probable. J'obtins, il y a quinze ans, une quantité considérable de gaz oxygène par des feuilles de vigne exposées sous l'eau pure aux rayons solaires; mais en répétant souvent cet essai depuis, les quantités de gaz dégagé ont toujours été beaucoup moindres. Je ne sais si la différence est due à l'état particulier des feuilles, ou à quelques confers, qui se seraient trouvées adhérentes au vase; ou à quelqu'autre source d'anomalie ou d'erreur.
( Le suite au prochain Cahier.)
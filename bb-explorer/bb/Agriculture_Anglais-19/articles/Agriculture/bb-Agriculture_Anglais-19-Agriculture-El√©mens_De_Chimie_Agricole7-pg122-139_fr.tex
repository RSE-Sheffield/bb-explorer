\setcounter{page}{122}
\chapter{Agriculture}
\section{ELEMENTS OF AGRICULTURAL CHEMISTRY, etc. Élémens de chimie-agricole en un Cours de leçons pour le Département d'Agriculture; par Sir HUMPHRY DAVY. Londres, 1813. \large{(Septième extrait. Voy. p. 82.)}}
Les arbres qui ont le cœur de bois le plus ferme et le moins poreux sont ceux qui durent le plus long-temps.
En général, les quantités respectives de charbon que fournissent les divers bois, offrent un indice assez exact de leur longévité. Ceux dans lesquels la matière charbonneuse et la terre abondent sont les plus permanens; et ceux où l'on trouve la grande proportion d'élémens gazeux. sont les plus destructibles.
Parmi nos arbres indigènes, le châtaigner et le chêne durent le plus long-temps; et le châtaigner donne un peu plus de charbon que le chêne.
On a quelquefois pris l'un de ces bois pour l'autre dans de vieux bâtimens gothiques.\setcounter{page}{123} Mais on peut aisément les distinguer en ceci, savoir, que les pores de l'aubier, et du chêne, sont beaucoup plus grands et plus nombreux, tandis qu'on ne peut découvrir qu'à la loupe ceux du châtaignier. Par suite de la lenteur avec laquelle le bois de cœur se décompose dans le chêne et le châtaignier, ces arbres peuvent, dans des circonstances favorables, arriver à un âge qui ne doit pas être fort au-dessous de mille ans.
Le hêtre, le frêne, et le sycomore n'ont probablement pas la vie de moitié aussi longue. Le pommier ne dure guère au-delà de deux cents ans; mais le poirier (selon Mr. Knight) peut vivre jusqu'au double de cette période. On dit que la plupart de nos meilleurs pommiers ont été introduits en Angleterre par un jardinier de Henri VIII, et ils sont actuellement en état de vieillesse.
Le chêne et le châtaignier se détériorent beaucoup plus tôt dans un sol humide qu'ailleurs, et leur bois y devient moins dur. Dans ce cas, les vaisseaux séveux sont plus dilatés, quoiqu'il y entre moins de matière nutritive, le tissu général du bois est plus lâche; il se fend plus aisément; et il est plus sujet à être influencé par les variations de l'atmosphère.\setcounter{page}{124} Les arbres de même espèce vivent en général beaucoup plus longtemps dans les climats septentrionaux que dans le midi. Cela est probablement dû à ce que le froid prévient toute espèce de fermentation et de décomposition; les matières animales et végétales résistent à la putréfaction dans les températures très-basses; et dans les hivers du nord, non-seulement la vie des végétaux, mais leur décadence même, doit être suspendue.
La qualité antiputride des climats froids est mise particulièrement en évidence par l'exemple des rhinocéros et des mammouth, qu'on a trouvés récemment en Sibérie, conservés dans leur intégrité sous le sol gelé, dans lequel ils étoient probablement ensevelis depuis le déluge. J'ai examiné une portion de la peau d'un mammouth, envoyée en Angleterre, et qui portoit encore un poil grossier; elle avoit tous les caractères chimiques d'une peau récemment desséchée.
Les arbres qui croissent dans des situations fort exposées aux vents, ont le bois plus dur et plus ferme que ceux qui sont fort abrités. La sève dense est déterminée par l'agitation des branches plus petites, à se porter vers le tronc et les grosses branches,\setcounter{page}{125} où le nouvel aubier en devient plus épais et plus ferme. C'est dans les arbres ainsi exposés qu'on trouve le plus fréquemment ces branches coudées en équerre, si précieuses pour les usages de la marine. Les vents qui travaillent les arbres dans les situations élevées leur donnent peu-à-peu la forme la mieux calculée pour résister à leurs efforts. Le chêne de montagne devient très-robuste; il est fortement attaché au sol, et brave les tempêtes.
La dégénération des meilleures variétés d'arbres fruitiers, multipliés dans le pays par la greffe, est une circonstance qui mérite beaucoup d'attention. Il n'y a aucun moyen de les conserver, et il n'existe d'autre ressource que l'espérance de produire de nouvelles variétés par les semences.
Lorsqu'une espèce a été améliorée par la culture, les graines qu'elle fournit, produisent, toutes choses égales d'ailleurs, des plantes plus vigoureuses et plus parfaites. C'est ainsi surtout qu'on a perfectionné notre horticulture, et les produits principaux de nos campagnes.
Le froment, dans son état primitif, et comme production naturelle du sol, paraît avoir été une très-petite graminée; et le\setcounter{page}{126} perfectionnement est beaucoup plus remarquable dans la pomme et la prune. La pomme sauvage paroît avoir été la mère de toutes les autres. Et on ne peut guère imaginer de fruits plus différens en couleur, grosseur, et apparence, que la prune sauvage, et le riche magnum bonum.
Les semences des plantes, exaltées par la culture, fournissent toujours des variétés plus grosses et plus parfaites. Mais la saveur, et même la couleur du fruit paroissent être de pur accident. Ainsi, une centaine de pepins du gold-pippin, produiront tous de beaux pommiers à larges feuilles, qui porteront de fort gros fruits : mais les saveurs et couleurs des pommes qui proviendront de chacun seront différentes, et aucune ne ressemblera à celle de la pomme primitive elle-même. Les unes seront douces ; d'autres acides, d'autres amères, d'autres enfin aromatiques ; celles-ci seront jaunes ; celles-là vertes ; d'autres, rouges ; d'autres, seront rayées. Mais, toutes ces pommes seront plus parfaites que celles qu'auroit produit la graine de la pomme sauvage, qui fait naître toujours des pommes de la même espèce, c'est-à-dire, petites et acides.
Les moyens de l'horticulteur ne s'étendent que jusques à la multiplication des meiljeures\setcounter{page}{127} variétés par la greffe. On ne peut point les rendre permanentes; et les bons fruits qui naissent actuellement dans nos jardins, sont les produits de quelques sauvajeons choisis probablement sur des centaines de milliers; les résultats de beaucoup de travail et d'industrie, et d'expériences très multipliées.
Plus les feuilles d'un sauvajeon sont larges et épaisses, et ses boutons développés, et plus il est à présumer qu'il produira une bonne qualité de fruit. Il ne faut jamais les choisir à courtes feuilles, car ils se rapprochent alors du sauvajeon primitif; tandis que les qualités opposées annoncent déjà l'influence de la culture.
En général, dans le choix des semences, il semblerait que celles qui proviennent des variétés les plus cultivées doivent donner les produits les plus vigoureux; mais il faut de temps en temps les changer et comme croiser les races.
On peut aisément faire naître une nouvelle variété en secouant la poussière des étamines d'une espèce sur les pistils d'une autre. Les expériences de Mr. Knight semblent garantir que ce procédé a de grands avantages.
Les gros pois que Mr. Knight a obtenus en croisant deux variétés, sont devenus célèbres,\setcounter{page}{128} chez les horticulteurs; et j'espère qu'ils seront bientôt l'objet de la culture des fermiers.
J'ai vu plusieurs de ses pommiers de race croisée, qui promettent de rivaliser avec les meilleurs de ceux que la vieillesse altère dans les contrées à cidre.
Et ses expériences sur le croisement du blé, très-facile à opérer puisqu'il suffit de semer ensemble plusieurs variétés, conduisent à un résultat très-important. Il nous apprend (Trans. Phil. 1799) que, dans les années 1795 et 1796, où la récolte presque entière ne donna que du grain sans farine, les variétés seules, obtenues par le croisement, échappèrent à ce fléau; quoique semées dans des terrains très-divers et des expositions fort différentes.
Les procédés du jardinier pour augmenter le nombre des branches à fruit et pour perfectionner le fruit sur des branches particulières, sont tous éclaircis par les principes avancés dans cette leçon.
Lorsqu'on dispose les arbres en espaliers, la force de pesanteur se trouve particulièrement dirigée vers la partie latérale des branches, ce qui détermine une plus grande quantité de sève à se porter aux boutons à\setcounter{page}{129} fruit ; et qui contribue à rendre le produit plus abondant \footnote{Si ce genre d'influence existe, il devroit y avoir une différence marquée dans les produits de la région inférieure des branches horizontales de l'espalier, comparés à ceux de la région supérieure de ces mêmes branches. Nous ignorons si cette différence a été remarquée. (R)}.
On a souvent recommandé comme procédé fructifère, la ligature d’une branche par un fil de fer, ou un lien végétal. Dans ce cas la descente de la sève dans l’écorce doit être retenue par la ligature, ce qui conserve au-dessus de celle-ci plus de matière nutritive applicable aux développemens des fruits.
Lorsqu’on greffe, les vaisseaux de l’écorce du sauvageon et de la greffe ne peuvent pas arriver aussi parfaitement en contact que ceux de l’aubier, qui sont bien plus nombreux et plus également distribués. Delà probablement, la circulation de haut en bas est obstruée, et la tendance du rameau greffé à développer ses boutons à fruit en est proportionnellement accrue.
L’effet de la taille tend à faire arriver plus de nourriture aux parties restantes ; car la sève coule latéralement comme verticalement.
Les mêmes considérations expliquent l’accroissement\setcounter{page}{130} des fruits restans sur un arbre lorsqu'on en a diminué le nombre.
De même que les plantes sont susceptibles d'être améliorées et rendues plus vivaces par des méthodes de culture particulières, ainsi, conformément à la loi générale des changemens, elles souffrent lorsqu'on les place dans des circonstances défavorables, et leur vieillesse en est avancée d'autant.
Les plantes des climats chauds, transportées dans les pays froids, (et vice versâ) lorsqu'elles peuvent survivre à ce changement, en souffrent toutes plus ou moins.
On sait qu'il n'y a qu'un très-petit nombre des plantes des tropiques qui puissent vivre en Angleterre, excepté dans nos serres chaudes. On peut dire que la vigne est chez nous dans une sorte d'état de maladie pendant tout l'été ; et son fruit, sauf certains cas extraordinaires, contient toujours un excès d'acide. Le pin gigantesque du nord, transporté dans les régions équatoriales, y devient un nain dégénéré. On pourroit citer un grand nombre d'exemples analogues.
On a beaucoup écrit, et les physiologistes ont fait plusieurs remarques ingénieuses sur ce qu'on a nommé les habitudes des plantes. Ainsi, lorsqu'on transplante un arbre, il meurt, ou paroît souffrir si on n'a pas eu\setcounter{page}{131} soin de l'orienter comme il l'étoit avant ce changement. Les graines apportées des climats chauds germent ici beaucoup plus tôt dans la saison que ne le font les mêmes espèces venant des climats froids. Le pommier de Sibérie, où un court été de trois mois succède immédiatement à un très-long hiver, lorsqu'on le transporte en Angleterre, fleurit de si bonne heure dans la première année de sa transplantation, qu'il est fort sujet à souffrir des dernières gelées du printemps.
Il n'est pas difficile de concevoir des effets aussi intimément liés avec l'état de santé ou de maladie des plantes. L'organisation du germe, dans les graines ou dans le bouton, doit être différente selon qu'il a éprouvé plus de chaleur, ou de plus grandes alternatives de température pendant sa formation; et ses développemens sont en rapport intime avec cette organisation. Dans un climat très-variable, les formations ont dû être interrompues, et s'opérer comme par couches successives. Dans une température égale, l'accroissement a dû être uniforme; et pour cette dernière agrégation, l'influence des changemens brusques doit être plus ou moins nuisible.
Cependant on peut dans beaucoup de cas modifier par degrés la disposition naturelle\setcounter{page}{132} des arbres, et les acclimater peu-à-peu. Le myrte, qui naît au midi de l'Europe, succombe toujours aux froids de l'hiver dans notre climat, si on l'y expose jeune; mais si on l'y accoutume peu-à-peu, en le mettant à l'abri du froid pendant ses premières années, il en vient à supporter même un froid très-sévère. Dans le midi et l'ouest de l'Angleterre, le myrte fleurit et porte des graines comme tout autre arbre en plein vent; et on remarque que les rejettons de ces arbres ainsi acclimatés sont beaucoup plus robustes que ceux des myrtes qu'on a particulièrement abrités.
L'arbutus (sans doute d'après un principe analogue de culture) est devenu l'ornement principal des lacs du midi de l'Irlande; il réussit même dans les situations froides et montueuses; et on ne peut guères douter que les rejettons de cet arbre, accoutumés à un climat tempéré, ne pussent être aisément multipliés en Angleterre.
Les mêmes principes qui s'appliquent aux changemens de température peuvent expliquer l'influence de l'humidité et de la sécheresse. Les rejettons d'un arbre accoutumé à un sol humide, périront dans une terre sèche, lors même que ce terrain seroit en général plus favorable qu'un autre à la végétation\setcounter{page}{133} de cette espèce. Et, ainsi que nous l'avons dit, plus haut, les arbres qui ont vécu dans le centre d'une forêt succombent tôt ou tard lorsqu'on les expose, adultes, aux vents et aux intempéries en coupant ceux qui les avoient mis jusqu'alors à l'abri.
Dans tous les cas où les arbres ont vécu dans des situations élevées, exposés à l'action du soleil, des vents et de la pluie, ils grandissent peu, mais ils deviennent robustes, leurs branches prennent de fortes courbures, et leurs troncs, des formes peu gracieuses. En revanche, les arbres et arbrisseaux trop abrités, s'élèvent beaucoup, mais leurs branches sont foibles et maigres, leurs feuilles pâles et en état de maladie, et dans les cas extrêmes de ce genre, ils ne portent point de fruit. L'exclusion de la lumière suffit seule pour les rendre ainsi malades, comme le prouvent les expériences de Bonnet. Cet ingénieux physiologiste sema trois graines de pois dans la même variété du sol. L'un des metura exposé à l'air libre; le second fut renfermé dans un tube de verre; le troisième dans un tuyau de bois. La plante se développa dans le tube de verre à-peu-près comme celle à l'air libre; mais celle qui croissoit dans le tuyau de bois devint blanche, étiolée, et\setcounter{page}{134} s'allongea beaucoup plus que les autres.
Les plantes qui croissent dans un sol qui ne peut leur fournir en suffisance un aliment provenant de matière jadis organisée, s'élèvent peu en général. Leurs feuilles sont brunes ou vert foncé, et leurs fibres ligneuses abonent en matière terreuse ; celles qui végètent dans des sols tourbeux ou dans des terrains trop abondamment fournis de matière animale ou végétale, se développent rapidement, elles produisent des grandes feuilles, d'un vert brillant, elles abondent en sève, et sont en général printannières.
Lorsqu'un sol se trouve trop riche pour le froment, on est assez dans l'usage d'éclaircir les plants, pour diminuer leur surabondance et les rendre moins sujets à verser avant la maturité.
L'excès de pauvreté ou de richesse du terrain est presque également nuisible au fermier; et la meilleure constitution d'un sol arable est celle dans laquelle la partie terreuse, aqueuse, et l'engrais sont convenablement associés; et où la matière animale ou végétale décomposable n'excède pas le quart du poids des ingrédients terreux.
La pauvreté du sol produit souvent sur l'écorce des arbres une espèce d'érosion cancereuse, qui accompagne presque toujours\setcounter{page}{135} leur vieillesse. La cause de cette maladie paroît être un excès de matière alkaline et terreuse dans la sève descendante. J'ai souvent trouvé le carbonate de chaux sur les bords de la playe cancéreuse des pommiers, et l'almine, qui contient l'alkali fixe est abondante dans le chancre de l'ormeau. La vieillesse de l'arbre n'est pas sans analogie avec celle de l'animal chez lequel, comme on sait, la matière solide osseuse est en excès, dans la dernière période de sa vie.
On essaie ordinairement de guérir cette maladie du chancre en taillant les bords de la playe et en liant dessus de l'écorce fraîche, ou en y appliquant un emplâtre de terre. Mais ces procédés, quoique vantés, contribuent probablement très-peu à la régénération de l'écorce. Peut-être l'application d'un acide foible sur la partie attaquée auroit-elle du succès; et dans les cas où l'arbre malade seroit d'une grande valeur, on pourroit l'arroser de temps en temps avec l'eau acidulée. Ce procédé est naturellement suggéré par la qualité terreuse et alkaline qui domine dans la sécrétion morbide; mais des circonstances imprévues peuvent se présenter en obstacle.
Indépendamment des maladies qui ont leur\setcounter{page}{136} source dans la constitution de la plante, ou dans l'action défavorable des élémens extérieurs, il y en a beaucoup d'autres, peut-être plus dangereuses, qui proviennent de l'influence d'êtres vivans, étrangers à la plante. Celles-là sont les plus difficiles à traiter, et les plus nuisibles aux travaux du fermier, et à ses espérances.
On voit naître dans tous les climats, des plantes parasites, de différentes espèces, qui s'attachent aux arbres et aux arbrisseaux, qui se nourrissent de la sève de ceux-ci, détruisent leur santé, attaquent jusqu'à leur vie, et sont peut-être les ennemis les plus formidables des espèces végétales supérieures et cultivées.
La rouille, qui a souvent fait de grands ravages dans nos récoltes de froment, sur-tout en 1804, est une espèce de fungus, si petit, qu'on ne le distingue qu'à la loupe, et qui se propage rapidement par sa graine. Plusieurs botanistes ont établi ce fait; et la question a été complétement éclaircie par les recherches approfondies du Président de la Société Royale.
Ce champignon se développe rapidement de plante en plante, il se fixe dans les cellules qui communiquent avec les tubes communs,\setcounter{page}{137} et s'approprie la nourriture destinée au grain.
On n'a point encore trouvé de remède à cette maladie. On la préviendrait peut-être en partie, si l'on n'employait jamais en fumier la paille ainsi attaquée, et si l'on arrachait soigneusement les plantes de blé, comme de mauvaises herbes, à mesure qu'elles donnent des signes de cette contagion.
La notion populaire que le voisinage du berberis ( épine-vinette ) continue à la provoquer, mérite d'être examinée. Cet arbrisseau porte souvent un fungus qui pourrait dégénérer en celui du blé. Si cela étoit, l'influence serait bien naturellement expliquée.
Il y a tout lieu de croire, d'après les recherches de Sir Joseph Banks, que le charbon du froment est occasionné par un très-petit champignon qui se fixe dans le grain. Ses produits par l'analyse ressemblent beaucoup à ceux que donne la vesse de loup, et un changement aussi complet dans la constitution du grain ne pourrait guère se concevoir sans admettre une action organique particulière.
Le gui et le lierre, les mousses et les lichens, en se fixant sur les arbres nuisent\setcounter{page}{138} uniformément à leur végétation, quoique dans des degrés différens. Ces plantes sont nourries par les vaisseaux à sève latéraux, et elles privent de nourriture la partie des branches qui se trouve au-dessus.
Les insectes ne sont guères moins nuisibles à la végétation que ne le sont les plantes parasites.
L'énumération des animaux destructeurs et tyrans du règne végétal seroit presque le catalogue des classes de la zoologie. Chaque espèce de plante, est, pour ainsi dire, la demeure ou le domaine de quelque tribu d'insectes, et depuis la sauterelle, la chenille et le limaçon, jusqu'à l'aphis imperceptible, une variété surprenante d'insectes inférieurs se nourrit des ravages qu'elle exerce dans le règne végétal.
J'ai déjà parlé de l'insecte qui se nourrit sur les feuilles séminales du turnep.
La mouche hessoise, encore plus dangereuse pour le froment, a plus d'une fois menacé de famine les États-Unis; et le Gouvernement français s'occupe actuellement des moyens de détruire la larve des sauterelles \footnote{Janvier 1813.}.\setcounter{page}{139} En général, les temps humides sont les plus favorables à la propagation de la rouille et des petites plantes parasites. Le temps sec favorise, au contraire, le développement des insectes. La nature tend toujours vers la multiplication des êtres vivans, et dans la sage et grande économie du système organisé, ces agens même, qui semblent nuire aux espérances et détruire les jouissances de l'homme, sont finalement en rapport avec le déploiement de ses facultés et leur perfectionnement. Ainsi son industrie s'éveille, son activité se ranime, par les défauts mêmes du climat et de la saison. Par les accidens qui dérangent ses travaux, il est conduit à développer ses talens et ses ressources, à s'occuper de l'avenir, et à considérer le règne végétal, non comme un héritage assuré et inaltérable, qui fournira spontanément à tous ses besoins; mais comme une possession douteuse et précaire, que rien que le travail ne peut lui assurer, et qui ne peut être améliorée que par le génie.
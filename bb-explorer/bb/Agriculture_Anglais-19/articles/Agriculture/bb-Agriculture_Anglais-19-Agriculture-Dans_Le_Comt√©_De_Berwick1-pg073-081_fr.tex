\setcounter{page}{73}
\chapter{Agriculture}
\section{SKETCH OF THE PRESENT STATE, etc. Esquisse de l'état présent de l'agriculture dans le comté de Berwick, par le Rév. James THOMSON, ministre d'Eccles, dans ce pays. ( Annals of Philosophy , etc. Par Th. THOMSON, N°. IV. ) \large{( Extrait ).}}
Le comté de Berwick est situé dans la partie méridionale et orientale de l'Ecosse, et il est bordé à l'est par la mer. Sa longueur moyenne est de 26 milles et ¾ et sa largeur de 17. Sa surface est de 285440 acres.
On peut le diviser en deux grands districts ; le Merse, et le Lammermuir ; et deux plus petits ; Lauderdale, et les terres arables vers la côte. Le Merse comprend la plus grande partie des terres basses ; Lammermuir est une rangée de collines au nord de la Merse. On se bornera à parler de cette dernière, et des deux petits districts; l'agriculture...
Agricult. Vol. 19. N°. 3. Mars 1814.\setcounter{page}{74} ture y a fait des progrès étonnants depuis un demi siècle. On ne décrira pas les pratiques de chaque fermier, mais on donnera un précis de la manière de procéder de ceux que l'on regarde comme les plus habiles.
Il serait à désirer qu'on pût donner une idée du climat de ce Comté par des tableaux météorologiques; mais les fermiers ne s'en occupent guères. Nous dirons seulement qu'on a remarqué que les fièvres intermittentes, jadis très-fréquentes dans le pays, ont presque entièrement disparu. On attribue cette amélioration aux desséchemens nombreux qu'on a faits.
SOL. Le sol des parties inférieures du Berwickshire est divisé, par les fermiers, en deux variétés, désignées par les épithètes de fond fermé et fond ouvert (close bottom and open bottom). Le premier est en général glaiseux; et sa seconde couche est si dense et adhésive, que la pluie ne peut pas la pénétrer; ce qui occasionne des inondations fréquentes dans les lieux bas qui ne sont pas desséchés par des aqueducs. L'autre est une terre arable, dont le dessous, perméable à l'eau, agit comme un crible et n'en laisse jamais séjourner de superflue. Ces deux qualités générales de sol offrent plusieurs variétés, qui résultent de mélanges, naturels ou artificiels.\setcounter{page}{75}
\section{OBJET DES FERMIERS.}
Le but du fermier du Berwickshire est de faire produire au sol le plus qu'il est possible de l'espèce de grain la plus profitable, et qui convient le mieux à la terre qu'il porte. Mais pour obtenir de bonnes récoltes de grains, il faut des engrais; et le meilleur fumier est celui des bestiaux. En conséquence, le fermier en entretient autant qu'il en peut largement nourrir. De plus, l'expérience lui apprend, qu'en étendant ses vues aux bestiaux aussi bien qu'aux grains, il assure bien mieux ses profits; car il arrive souvent que lorsque les grains sont bon marché les bestiaux sont chers; et vice versâ. Il se porte vers l'un ou l'autre de ces genres d'industrie, selon que les circonstances la permettent; en maintenant toutefois une sorte d'équilibre entre l'un et l'autre.
\subsection{ETENDUE DES ENCLOS.}
Toutes les fermes dans la Mersé sont renfermées, et subdivisées en champs, qui ont, en général, une certaine proportion avec l'étendue de la ferme. Quelques-uns préfèrent les grands enclos afin de réserver ainsi aux grains une portion du sol, qui serait inutilement employée aux haies. D'autres, qui s'occupent particulièrement de l'industrie des moutons, préférant les petits enclos. L'étendue des enclos varie entre cinq acres et cinquante; on en\setcounter{page}{76} voit même de cent ; mais ils ne sont pas communs.
\subsection{ETENDUE DES FERMES}
Relativement à l'étendue des fermes, on peut considérer deux sortes de moyennes. Ou, compter le nombre des acres attachés à la maison du fermier et aux dépendances, ou à celle-ci seulement ; ou bien, considérer le nombre d'acres affermées par un seul individu, soit que les terres se touchent ou qu'elles soient séparées. Dans le premier cas, les extrêmes sont de deux cents à quatre cents acres. Dans le second, on en trouve de douze cents jusques à deux mille acres.
\subsection{RENTE}
La rente du sol, d'après les baux des sept dernières années, varie entre 30 sh. et 4 liv. sterl. l'acre. Mais il faut se rappeler que la rente ne dépend pas uniquement de la qualité du sol ; elle est modifiée par d'autres circonstances, telles que la distance du marché, et celle des fours à chaux et de la houille. Le voisinage du port de Berwick élève au moins de 10 shillings par acre la rente annuelle des terres, comparativement à celles qui en sont à quinze milles ou plus loin.
\subsection{BAUX}
Toutes les fermes sont données à bail, dont la durée est, en général, de dix-neuf à vingt-un ans. Ces périodes per\setcounter{page}{77} mettent au fermier de consacrer un capital considérable en améliorations; avec la certitude de retrouver ses avances, avec intérêt, avant l'expiration de son bail. Les propriétaires judicieux n'imposent aucunes conditions ou restrictions aux bons fermiers, sur le mode de culture, sauf pendant les quatre ou cinq dernières années de leur bail.
\subsection{CAPITAL NÉCESSAIRE} — On estime que le fermier doit posséder un capital de 500 liv. sterl. par chaque centaine d'acres qu'il prend à bail, pour l'achat des divers objets et des grains nécessaires à son entreprise, et pour demeurer solvable, dans les cas de mauvaise récolte. D'autres croient que ce capital n'est pas suffisant. Si le fermier en étoit absolument dénué, et ne pouvoit faire d'avances, dans beaucoup de cas les produits de la ferme ne suffiroient pas à acquitter la rente.
\subsection{ENCLOS} Les fermes sont presque partout revêtues et subdivisées par des haies d'épines, et par des fossés de cinq pieds de large en haut, et de trois de profondeur. On nettoie ceux-ci ordinairement tous les quatre ans, ou dans l'année de jachère du champ adjacent. Ces fossés servent aussi à l'écoulement des eaux.
\subsection{DESSÉCHEMENS} — A l'ouverture d'un nouveau bail, l'une des premières opérations du\setcounter{page}{78} fermier judicieux est le desséchement du sol; on la considère comme la plus importante, et comme la base de toutes les autres améliorations; car un sol imprégné d'eau est de bien peu d'usage. Les fossés les plus ordinaires sont destinés à emmener l'eau qui s'arrêteroit à la surface. Partout où le champ présente une dépression qui retiendroit l'eau, on ouvre un fossé pour lui procurer de l'écoulement. On leur donne ordinairement trois pieds de profondeur là où la pente est suffisante pour emmener l'eau. Quelquefois il faut descendre à quatre, et même à cinq pieds. Là où le sol est très-plat, il est quelquefois impossible de les creuser au-delà de deux pieds et demi; mais ces fossés ont peu d'effet, à moins qu'on ne les remplisse de pierres jusqu'à un pied de la surface; cette pratique a déjà rendu les pierres très-rares dans les champs; lorsqu'on ne peut s'en procurer, on leur substitue des fagots d'épines, coupés à la longueur d'un pied ou un pied et demi seulement, et qu'on place dans le fossé, non pas tout-à-fait verticalement, mais appuyés les uns contre les autres. Là où on ne peut se procurer ni cailloux ni fagots, on remplit le fond des aqueducs de branchages de sapin ou d'autres arbustes; dans tous les cas, on recouvre les fagots ou\setcounter{page}{79} les branches d’une couche de paille ou de toute autre matière qui retienne la terre; la construction de ces aqueducs est une opération délicate et qui demande d’être surveillée. Le prix ordinaire d’un fossé de trois pieds de profond, est de 6 à 8 den. st. pour une longueur de dix-huit pieds; 2 den. st. pour la façon d’une même étendue de fagots; et 2 den. sterl. pour les arranger dans le fossé et les couvrir de terre.
\subsection{BILLONS} — On a beaucoup disputé sur la largeur et la longueur la plus convenable à donner aux billons dans le labourage. La longueur dépend évidemment de celle du champ, dont la forme dépend elle-même d’un nombre de considérations accessoires; même dans une terre meuble, et dont le fond est perméable, les billons ne sont pas de rigueur; seulement ils sont utiles au fermier pour lui subdiviser à l’œil son champ, de manière à faciliter ses calculs sur les proportions de l’engrais et de la semence. Mais dans les terres dont le fond est imperméable à l’eau, les billons sont nécessaires; il faut une double pente, et un sillon ouvert de part et d’autre pour l’écoulement des eaux; la saillie de la crête du billon doit varier selon que la terre est plus ou moins pesante, ou adhésive. L’essentiel est que dans toute\setcounter{page}{80} l'étendue du champ l'eau se loge dans les sillons intermédiaires, et qu'elle y trouve de l'écoulement. Ainsi, des billons de trente à quarante pieds seraient trop larges, à moins qu'on ne leur donnât à la crête une hauteur tout-à-fait inconvenante. La plupart des fermiers donnent à leurs billons dans un sol glaiseux, de douze à dix-huit pieds, avec une pente double, dont la section présente un segment de cercle. Ces règles s'appliquent également à tous les genres de récoltes, grains, turneps, pommes de terre, etc.
\subsection{RÈGLES GÉNÉRALES DE CULTURE} Le propriétaire peut être considéré comme un négociant, qui fournit au fermier la matière première. Le fermier est à son tour une espèce de manufacturier, qui exploite cette matière, à son plus grand avantage; il doit viser à produire les récoltes les plus lucratives, aussi abondantes qu'il est possible, et avec la plus grande économie de main-d'œuvre. L'expérience a appris que les moyens d'arriver à ces résultats sont; 1º. le dessèchement du sol, s'il est humide; 2º. la destruction de toutes les plantes parasites et étrangères à celle qu'on veut cultiver; 3º. l'application d'un engrais approprié, et en quantité suffisante; 4º. le sol doit être rendu meuble, et perméable à l'engrais, aux racines,\setcounter{page}{81} et aux variations de l'humidité et de la température ; 5°. il faut lui donner l'assolement le plus convenable. Les fermiers du Berwickshire portent leur attention sur tous ces points.
Pour détruire les mauvaises herbes, ils emploient la jachère, ou une récolte de turneps en culture; ce dernier procédé suffit lorsque les plantes parasites sont annuelles. Mais dans les terres fortes, infestées de chiendent, il faut absolument la jachère, avec plusieurs labours, à intervalles bien choisis; indépendamment de l'avantage particulier de ce procédé pour la destruction des plantes nuisibles, il contribue essentiellement à rendre la terre meuble, et accessible à toutes les influences de l'air et du soleil.
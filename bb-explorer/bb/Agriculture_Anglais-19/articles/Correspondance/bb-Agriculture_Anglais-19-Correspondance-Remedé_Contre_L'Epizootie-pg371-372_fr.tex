\setcounter{page}{371}
\chapter{Correspondance}
\section{LETTRE DE MR. CHANCEY, DE LYON, A MR. CH. PICTET, sur un remède nouveau et très-simple, employé avec succès en Hongrie, dans les cas d'épizootie.}
Belleville, 10 octobre 1814.
MR.
Je crains bien que les armées Autrichiennes n'ayent porté chez vous l'épizootie qui ravage nos campagnes, épizootie qui règne constamment en Hongrie, tantôt dans un canton tantôt dans un autre, et où elle exerce les plus grands ravages.
Mr. le vicomte de Bussy, général au service de l'Autriche, qui réside une partie de l'année en Hongrie, arrivé de ce pays-là depuis très-peu de temps, eut avant-hier la complaisance d'écrire en ma présence le remède nouvellement découvert en Hongrie, qui guérit\setcounter{page}{372} promptement le bétail attaqué de cette maladie. Je m'empresse, monsieur, de vous le transmettre; Mr. le vicomte de Bussy m'y a autorisé; désirant que ce remède ait la plus grande publicité, sa publication dans votre journal en instruira l'Europe entière.
\subsection{Remède contre l'épizootie, actuellement regnante, épreuve en Hongrie et connu seulement depuis quelques mois.}
"Mr. le vicomte de Bussy se trouvoit alors dans le pays, au moment où la maladie faisoit le plus grand ravage ; il eut la satisfaction d'être témoin que l'expérience ayant été faite sur la bête la plus malade, même désespérée, dès le lendemain elle avoit repris sa gaieté et son appétit. Dès ce moment toutes les bêtes malades furent guéries radicalement, et il n'en est plus mort une seule.
Ce remède a été découvert par Mr. Cohr, médecin-vétérinaire de Stuhl-Weissenbourg en Hongrie, autrement Abbe Royale, la première expérience a été faite chez Mr. le baron de Bereny.
Prendre du levain de bière, le délayer avec de la bière faite jusqu'à-ce qu'on puisse le prendre par cuillerées.
En prendre six cuillerées à bouche qu'on verse dans une chopine de bière, et la faire avaler à la bête malade; répéter ce procédé trois fois par jour. Si la bête étoit encore aussi mal le lendemain, répéter encore ce qu'on a fait la veille. La maladie cède dès le premier jour; alors on ne fait avaler que deux chopines, et ensuite une.
Mr. le vicomte de Bussy assure que dans la Hongrie il n'a pas vu une seule fois faire usage de ce remède sans que dès le premier jour la cure ne se soit manifestée avantageusement."
J'ai l'honneur d'être, etc.
CHANCEY.
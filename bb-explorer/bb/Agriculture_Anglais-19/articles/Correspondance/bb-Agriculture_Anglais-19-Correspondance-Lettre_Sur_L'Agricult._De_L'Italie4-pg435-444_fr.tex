\setcounter{page}{435}
\chapter{CORRESPONDANCE}
\section{QUATRIÈME LETTRE à Mr. CHARLES PICTET, l'un des Rédacteurs de la Bibliothèque Britannique, sur l'agriculture d'Italie.}
Parme le 10 sept. 1812.
PLUS on avance vers l'orient, en suivant le cours du Pô, plus aussi, Monsieur, la couche de terre végétale devient profonde et fertile; mais aussi les rivières, dont le lit est encore profond et contenu, aux pieds des Alpes, coulent à fleur de terre en approchant de l'Adriatique; le sol est par conséquent plus arrosé et plus humide. Aussi les cultures céréales diminuent, et celle des prairies s'étend sur de plus vastes espaces.
Ce changement devient sensible depuis les environs de Plaisance. La subdivision des fermes et le système de leur administration sont les mêmes que dans le Piémont; mais l'assolement et les revenus agricoles varient.
Ce sont moins les grains que les bestiaux, qui font la richesse de toute cette portion de la Lombardie. Elle en devient plus belle et plus animée aux yeux du voyageur.
Les bêtes à cornes y sont aussi nombreuses qu'en Suisse, et nulle part il n'existe une aussi belle race de porcs. Ces deux espèces d'animaux y sont seules en usage, il n'y a ni chevaux,\setcounter{page}{436} ni bêtes à laine. Chaque métairie, et elles sont innombrables, possède une famille de cochons, qui fait presque partie de celle du métayer: car c'est à eux qu'il prodigue son intérêt et ses premiers soins. Aussi les seuls états de Parme et de Modène vendent annuellement soixante mille porcs aux Génois et aux Toscans. Leur race est la plus féconde et la mieux conformée que j'aie vu nulle part; la qualité de leur viande est supérieure, et elle m'a paru réunir des caractères assez précieux pour qu'il valût la peine de s'en procurer des individus. Leur nourriture consiste en trèfle, en fèves et en glands. Toute cette partie de la Lombardie est plantée de superbes chênes, dont la tige élevée supporte un branchage majestueux, qui donne à toute cette campagne une fraîcheur et une verdure, qu'on ne s'attend point à trouver en Italie. Ces chênes procurent une récolte de glands, qui est respectée dans le pays comme un produit important. Ce que j'ai remarqué avec étonnement, c'est qu'ils nuisent à peine aux récoltes qui croissent à leur ombre: ce qui ne peut être attribué qu'au triple effet de la fertilité du sol, de son arroseinent, et du soleil de l'Italie.\setcounter{page}{437} On sait que les vacheries des plaines qui avoisinent le cours du Pô, produisent les fromages parmesans, dont la consommation est si prodigieuse dans toute l'Italie. Ces prairies sont les plus fertiles de la terre ; constamment arrosées, elles produisent trois, et quelquefois quatre coupes de fourrage. Mais subdivisées en une infinité de parcelles qui dépendent d'une multitude de métairies, il y en a peu qui puissent à elles seules alimenter une fromagerie, qui exige la totalité du lait fourni par la réunion de cinquante vaches au moins. Pour obtenir ce résultat, les Lombards ont dès long-temps imaginé de former des sociétés de voisinage pour fabriquer en commun leur fromage. Deux fois par jour on apporte le lait des cinquante ou soixante vaches sociétaires au manoir commun, où le fromager tient compte à chaque intéressé de sa portion de lait. Il établit ainsi à chacun un compte courant, qui se solde tous les six mois, et s'acquitte par une quantité proportionnée de fromages. Cette méthode ingénieuse a passé en Suisse, elle est décrite en détail dans un excellent ouvrage publié à Genève, par Mr. Charles Lullin, et il seroit à desirer qu'elle fût répandue à-peu-près partout: car je ne connois\setcounter{page}{438} guères de localités, où elle ne fut d'un grand avantage.
La race des bêtes à cornes change aussi dans les environs de Plaisance. On cesse de voir ces grands boeufs au poil fauve et aux petites cornes, du Piémont; mais les campagnes sont couvertes de belles vaches gris ardoise, à jambes fines, à corsage cylindrique, à l'oeil vif, et à cornes longues et régulièrement contournées. Cette race est évidemment le produit d'un croisement continuel entre la race hongraise et celle des petits Cantons de la Suisse.
Cette superbe race hongraise, qui subsiste sans mélange dans l'Italie méridionale, fournit les plus beaux et les meilleurs boeufs qui existent; mais les vaches en sont mauvaises laitières, et les Lombards ont senti, depuis long-temps, qu'il falloit la croiser pour y remédier et tirer de leurs prairies tout le produit dont elles étoient susceptibles. Ainsi, dès une époque dont la date est inconnue, deux mille vaches passent annuellement le St. Gothard et viennent se répandre dans la Lombardie, où elles apportent un principe de régénération d'espèce, qui seul conserve aux races d'Italie des qualités qui les rendent précieuses.
\setcounter{page}{439} Ces vaches suisses ne sont pas elles-mêmes de la race Bernoise connue en France, et que ses couleurs vives et ses belles formes distinguent si fort. Celle des petits Cantons me paroît être elle-même par ses couleurs ternes, ses cornes longues et minces et ses formes déliées, un produit de race hongraise, très-amélioré par la nourriture, le climat et les soins. Elle s'assortit ainsi complètement avec la race italienne dont l'origine est commune.
L'administration des fermes est comme en Piémont, un bail à moitié fruit; mais l'assolement adopté dans ces métairies est un peu différent. Les prairies en occupent un plus grand espace, et le maïs cède une grande portion du sol à la culture du chanvre et et des fèves d'hiver. L'assolement est assez généralement celui-ci:
\comment{table}
Première année. . . Maïs et chanvre fumé.
Seconde . . . . Blé.
Troisième . . . Fèves d'hiver.
Quatrième . . . Blé fumé.
Cinquième . . . Trèfle, retourné après la première coupe.
Sixième . . . Blé.
Dans les environs de Parme, on a commencé\setcounter{page}{440} à cultiver le tabac avec un grand succès, et il remplace alors la première année le maïs et le chanvre.
Cet assolement est plus productif encore que celui du Piémont; mais il appartient à un sol très-riche et à la grande abondance d’engrais que procurent les vacheries; puisqu’elle permet de fumer tous les trois ans; en Piémont, on ne peut y parvenir que tous les quatre ans.
Je ne m’étendrai pas, monsieur, sur cette belle succession de cultures, qui fournit en six ans, quatre récoltes céréales, une de chanvre, et une destinée aux animaux. Cette succession rapide est, comme vous le remarquerez, si habilement entremêlée, que la fertilité de la terre n’en est nullement épuisée, en même temps qu’elle permet de donner au sol toutes les préparations nécessaires et de le nettoyer par des cultures sarclées à intervalles égaux.
Celle des fèves d’hiver est la seule qui me paroît avoir une importance sur laquelle il convient d’insister.
Vous savez, monsieur, que depuis quelques années nous l’avons transportée avec un grand succès dans les environs de Genève : c’est-à-dire, dans l’un des climats où l’hiver est\setcounter{page}{441} le plus rude. Cette plante les supporte donc sans inconvénient et peut être introduite dans les régions septentrionales, et jouer un grand rôle dans leur agriculture: car elle entre admirablement dans tous les assolemens, dont elle comble les vides. La fève d'hiver ressemble à celle de printems par sa plante, ses fleurs et sa graine; elle se sème au commencement de septembre, et il faut qu'elle devienne forte dans l'automne pour supporter mieux les intempéries de l'hiver. Sa tige se fane et périt dans les gelées et sous les longues neiges; mais dès les premiers jours du printems elle repousse du collet deux ou trois nouvelles tiges, qui se chargent de fleurs au mois de mai, et mûrissent à la fin de juillet. Sa culture est extrêmement simple; après la récolte du blé fumé, on retourne la terre par un seul labour, et on la laisse émietter par l'influence de la saison. Aux premiers jours de septembre on sème les fèves, soit en les enterrant à la charrue, soit en les recouvrant à la herse, soit enfin avec le semoir, qui les place par rangées, de manière à pouvoir au printems les sarcler avec la houe à cheval. Si on ne suit pas cette dernière méthode, il faut les sarcler à la main dans le courant d'avril.
\setcounter{page}{442} La récolte étant faite dès le mois de juillet, le cultivateur a tout le temps de préparer sa terre, afin de recevoir de nouveau la semence du blé qui lui succède et qui presque toujours réussit bien.
Cette culture appropriée aux terres franches et argileuses où celle des racines réussit moins bien, s'associe heureusement avec les différentes époques de culture et de semailles, et maintient la fertilité du sol. Elle réunit donc toutes les qualités désirables, et je ne doute pas qu'elle ne s'étende avec rapidité.
Tel est, monsieur, le tableau raccourci que j'ai cru devoir vous tracer, de la culture et des assolemens de la Lombardie, c'est-à-dire, de la première région agricole de l'Italie, que j'ai eu l'honneur de vous signaler dans ma première lettre. Vous voyez que ces assolemens sont presque tous dirigés vers les cultures nutritives, et que hors le chanvre, il n'y en a aucune d'industrielle. Le résultat de ces abondans moyens d'alimens est une immense population, dont aucune branche n'est manufacturière, parce qu'elle n'a à sa portée aucune matière première.
Cette population est, d'après cela, divisée en quatre classes seulement; celle des fonctionnaires publics et des militaires; celle\setcounter{page}{443} des propriétaires de toute la surface du sol, qui vivent de la rente des métairies; celle des marchands et des artisans; et enfin celle des cultivateurs-métayers, non-propriétaires du sol, et qui ne vivent que de l'industrie rurale.
Cette dernière classe réside uniquement dans les métairies éparses qui couvrent toute la surface de la Lombardie; tandis que les trois autres habitent dans des villes, ou de gros bourgs. C'est pourquoi on ne voit point de hameaux, point de réunions de paysans propriétaires, si communes en France, dans toute cette contrée. En revanche, la totalité des terres étant entre les mains des capitalistes propriétaires, cette classe de rentiers est plus nombreuse ici que nulle part et a produit l'accumulation de ces villes, qui présentent un agréable aspect d'aisance.
Cet ordre de choses, qui multiplie à l'oeil l'opulence publique, a cependant les graves inconvéniens de retenir toute la classe aisée des propriétaires dans une sécurité qui tend à lui donner, faute d'intérêts sérieux, cette oisiveté et cette paralysie morale tant reprochée aux Italiens, en même temps elle jette toute la classe des cultivateurs usufruitiers dans un trop grand désintéressement de la chose publique, à laquelle la propriété ne\setcounter{page}{444} la lie jamais; toujours sûre d'employer ses bras, qui constituent son seul capital, elle ne peut jamais s'inquiéter d'évènemens qui ne peuvent l'atteindre; toujours privée de capital, elle ne peut jamais sortir de son état, et il en résulte une langueur morale que le besoin de vivre peut seule vaincre.
La masse des marchands et des artisans, bornée dans ses entreprises par la mesure immédiate de la consommation locale, a également peu de changemens à attendre dans son avenir, et par conséquent, peu de stimulans à son activité. L'ordre social présente dès long-temps, dans toutes ces contrées, quelque chose d'assez bon pour qu'il ne vaille pas la peine de le changer; et une sorte de sécurité dans l'existence, qui garantit l'avenir comme le présent, et fait respecter l'un et l'autre.
La guerre l'avoit altérée momentanément; la paix l'a ramenée, parce qu'elle a ses racines dans les dispositions locales du sol, ainsi que dans les divisions et l'emploi de toute la population.
J'ai l'honneur d'être, etc.
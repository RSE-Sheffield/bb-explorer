\setcounter{page}{427}
\chapter{CORRESPONDANCE}
\section{TROISIÈME LETTRE à Mr. CHARLES PICTET, l'un des Rédacteurs de la Bibliothèque Britannique, sur l'agriculture d'Italie.}
à la Mandria de Chivas, 20 juillet 1812.
BIEN que vous ayez donné, Monsieur, il y a quelques années, une excellente description de la Mandria, je ne puis cependant quitter le Piémont sans vous parler de cet établissement, le plus beau peut-être qui existe en Europe, et vous instruire de la suite de son exploitation.
Vous connoissez, Monsieur, la disposition et l'étendue de ce vaste domaine, qui renferme dans un carré long parfait deux mille six cents arpens, arrosés par un canal et divisés par des chemins de dépouille en cent vingt-six carrés égaux, dont un tiers est en prairie et le reste en terres arables.
\setcounter{page}{428} Vous savez que le but de la Société pastorale, qui s'est chargée de l'administration de ce superbe domaine, étoit l'entretien et l'amélioration d'un troupeau de six mille mérinos.
Une telle entreprise étoit une innovation dans l'agriculture du Piémont, qui n'ayant que des prés arrosés, point de jachères ou de terres vaines n'est pas un pays de moutons. Ils sont en effet un hors-d'œuvre dans ces exploitations, et ne pourroient pas même s'y maintenir si le voisinage des Alpes ne permettoit pas d'envoyer les bêtes à laine passer les cinq mois d'été sur les montagnes. A leur retour, les vastes terres de la Mandria leur fournissent encore six semaines de parcours; le reste de l'année se passe à la crèche. Vous voyez par là, Monsieur, que les bêtes à laine ne sont point une partie constituante de l'agriculture de la Mandria, dont elles consomment pendant l'hiver les fourrages surabondans; mais qu'elles pourroient être sans inconvénient remplacées par un autre bétail, et que même elles ne peuvent s'y soutenir que par le voisinage des Alpes.
Mais la richesse de leurs pâturages, l'abondance et la qualité des fourrages d'hiver\setcounter{page}{429} et les soins continuels du comte Lodi ont exercé sur cette race une grande influence. Elle a acquis un développement et des formes qui la distinguent de toutes les autres. Plus élancée que celle de Rambouillet, elle a autant de poids, et des formes aussi belles et aussi arrondies. Les béliers sont peu chargés de cornes, ils ont l'aspect moins farouche, ils dépouillent d'énormes toisons dont l'échantillon légèrement lustré me paraît se rapprocher de la laine électorale de Saxe. Ce beau troupeau, qui a néanmoins des émules en Piémont chez Mrs. de Laval et de Collegno, a joui du plus heureux succès jusqu'en 1814; mais à cette époque son mouvement a été paralysé, les laines se sont avilies, et le défaut de toute vente a obligé de livrer à la boucherie tout ce qui étoit médiocre, ainsi que tous les agneaux qui ne provenoient pas du troupeau d'élite. Cette destruction, malheureuse pour la société, a eu l'avantage d'embellir et de perfectionner leur type par l'écartement forcé de tout animal inférieur.
Il y a dans l'économie et l'administration de la Mandria un trait de génie qui m'a singulièrement frappé, et qui, je crois, devroit être profondément étudié afin de servir\setcounter{page}{430} d'exemple dans les pays de grande exploitation. J'ai dit, Monsieur, dans ma précédente lettre, que le Piémont étoit un pays de petite culture et de fermes divisées; mais la Mandria, ancien haras du Roi, présentoit une immense surface plane, régulière et contigue, de deux mille six cents arpens, n'ayant qu'un manoir au centre. Elle s'offrait ainsi avec tous les caractères qui entraînent et nécessitent l'application de la grande culture, qui y étoit aussi précédemment en usage. Mais le comte Lodi, comprenant tous les avantages de la petite culture du Piémont, a entrepris de la transporter dans l'immense cadre de la Mandria; et c'est à ce tour de force qu'il est parvenu. Les moyens qu'il a employés sont aussi ingénieux qu'ils paroissent simples. C'est la subdivision de la propriété, et l'ordre merveilleux dans l'exécution des travaux.
Le sol de la Mandria étant homogène, étoit susceptible d'être soumis au même assolement; le comte Lodi n'a pas cherché à changer celui qui est pratiqué dans le Piémont; il y a invariablement soumis toute la Mandria, ainsi son assolement est
1°. Année — Maïs fumé....
2°. Année — Blé......
\setcounter{page}{431}
3º. Année — Trèfle suivi de jachère...
4º. Année — Blé....
Sur la sôle du maïs il réserve seulement vingt arpens de pommes de terre, destinées aux moutons, c'est la seule innovation qu'il ait eu besoin d'adopter.
Pour maintenir cet ordre régulier et systématique, au lieu de profiter de son vaste espace suivant la bévue ordinaire, pour agrandir ses champs, il a au contraire encadré d'une haie d'aulnes chaque parcelle égale et régulière de vingt arpens. Une allée qui sépare chaque deux rangées de ces cadres sert à la dépouille de chacune de ces parcelles.
Du moment que cette division a été opérée, le domaine ne s'est plus présenté à l'imagination dans son immensité; mais seulement comme une nombreuse réunion de petites fermes. C'est aussi sous ce rapport que le comte Lodi l'a considéré. Déterminé sur l'assolement qu'il vouloit y adapter, il n'a point fait le calcul ordinaire des grandes fermes: c'est-à-dire, l'économie des ateliers, et la négligence qui en résulte dans toutes les parties médiocres ou éloignées de la ferme. Il s'est assuré de la somme du travail nécessaire à la stricte exécution de son assolement\setcounter{page}{432} dans chaque parcelle du domaine, et additionnant cette somme, il a monté ses atteliers sur cette base. Tout jusqu'ici se borne à un calcul simple; mais la grande difficulté était de mettre en mouvement cette machine, qui sous un cadre immense représente l'action multipliée de vingt exploitations ordinaires. Il y est parvenu en imprimant à tout son système une monture militaire; et en établissant ainsi une hiérarchie, une responsabilité, et une fixité invariables dans ses atteliers.
Ils sont composés de domestiques à l'année et de journaliers à la semaine. Tous s'obligent, en se présentant à suivre l'ordre établi, et cette obligation n'a donné quelque peine à fixer que dans les commencemens; l'habitude est dès long-temps si bien prise qu'elle n'offre plus aucune résistance.
L'administration n'est chargée d'aucune nourriture; domestiques et ouvriers s'arrangent entr'eux pour former des sociétés de gamelles; ils sont payés de tout en argent; les premiers seulement ont des jardins, dont l'étendue est en raison de leur grade et pour le travail desquels il leur est accordé un temps convenu.
Les domestiques sont divisés en autant de\setcounter{page}{433} compagnies qu'il y a d'espèces d'ateliers; à la tête de chacune de ces compagnies est un chef ou capitaine, chargé de la responsabilité de l'ordre et du travail; il prend les ordres du chef suprême et les distribue dans les escouades; sous lui sont des lieutenans et des caporaux. Ainsi, les bergers de moutons forment une compagnie, de même que les bouviers, les charretiers, et l'atelier des ouvriers de terre. Les journaliers se placent dans chaque escouade en proportion du besoin, et sont alors sous les ordres des officiers et des caporaux. Tous les travaux se commencent et s'achèvent au son régulier de la cloche, et les caporaux toujours présens surveillent à la fois leur exécution et leur durée.
Pour pouvoir maintenir cette fixité dans l'ordre du travail, le comte Lodi a établi le principe de ne jamais séparer les ateliers, sous quelque prétexte que ce puisse être. Ses champs étant tous égaux, il y porte à la fois tous ses ouvriers, et le travail doit être fini dans un temps donné. On y parvient en faisant travailler les ouvriers, de même que les charrues, en alignement. Jamais je n'ai vu de plus belle scène champêtre que celle que m'ont offert vingt charrues également espacées sur le même champ, marchant à hauteur et\setcounter{page}{434} dans un alignement parfait, se retournant toutes à la fois à la voix du caporal et recommençant dans le même ordre leur marche grave, qui avoit je ne sais quoi de silencieux et de solemnel. C'était aussi une belle scène que celle de cent cinquante faucheurs rangés sur une ligne oblique, abattant en mesure une herbe abondante, et suivis d'une égale ligne de faneuses formant en arrière une parallèle exacte et épanchant les ondins à mesure que la rosée s'évaporeit.
C'est ainsi que par un ordre merveilleux le comte Lodi est parvenu à maintenir une exécution invariable dans ses travaux; par ce moyen il a pu transporter les soins, l'exactitude et les détails de la petite culture sur l'espace immense de deux mille six cents arpens; sur toute cette étendue il n'y a pas un pouce de terre qui reste en arrière, la totalité de la ferme est entrée dans le cadre qui lui a été tracé; tous reçoivent également leur portion de culture et d'engrais, et tous répondent à ces soins par des récoltes qu'on n'attendroit pas d'un sol médiocre et d'une aussi vaste manutention. -- Mais rien n'est si puissant que la volonté de l'homme, quand elle est forte et durable.
J'ai l'honneur d'être, etc., etc.
\setcounter{page}{362}
\chapter{CORRESPONDANCE.}
\section{LETTRE à Mr. CHARLES PICTET, l'un des Rédacteurs de la Bibliothèque Britannique, sur l'agriculture d'Italie.}
Genève ce 10 octobre 1814.
MR.
J'ai parcouru l'Italie pendant deux ans, pour en étudier l'agriculture; vous avez cru qu'il y auroit quelque intérêt à publier les remarques que j'ai été appelé à faire sur cette belle contrée, et j'ai sur-tout à cœur de mettre son agriculture en comparaison avec celle de l'Angleterre, que vous avez si bien décrite. Je m'empresse donc, monsieur, à vous envoyer les lettres que j'ai écrites pendant\setcounter{page}{363} mon séjour en Italie et qui présentent un aperçu assez exact de sa culture. Recevez, monsieur, etc.
\subsection{PREMIÈRE LETTRE}
Turin ce 12 mai 1812
J'AI été il y a vingt ans en Italie; cette intéressante contrée apparoissait alors à mon imagination comme un pays enchanté, séparé du reste de la terre par des abîmes à peine accessibles. Il me sembloit que je trouverois au-delà des Alpes, des nations, des mœurs et une nature, dont la physionomie auroit un caractère tout particulier; et je me séparai en partant, de tout ce qui m'étoit cher, avec le même sentiment que j'aurois éprouvé si j'avois dû visiter des peuples inconnus.
Je viens de traverser les Alpes; mais je n'ai plus ressenti l'émotion qu'elles m'avoient inspirée jadis. Ces routes majestueuses, qui traversent les montagnes comme un fleuve, sont sans doute le plus beau monument de notre civilisation; mais en réunissant les peuples elles effacent leur physionomie, elles leur donnent des besoins et des habitudes semblables, ainsi que des mœurs pareilles. L'instinct de nationalité se dissipe peu-à-peu, et\setcounter{page}{364} l'on traversera bientôt toute l'Europe en croyant rester toujours au milieu du même peuple.
Ce sentiment m'a frappé dès mon arrivée à Turin ; je me suis cru dans une grande ville de France, tellement tout ce qui frappait mes yeux étoit semblable. Costumes, habitudes, spectacles, boutiques, et jusqu'aux placards qui tapissent les rues, tout est jeté dans le même moule. Il en est de même des intérêts qui agitent la vie; les lois qui en disposent, les institutions qui la dirigent, les espérances, les craintes, tout vient de Paris, ainsi que les Restaurateurs et le Journal des modes.
Quel sera le résultat de cette fusion de tous les peuples, de cette uniformité de mœurs, de cette culture commune, qui donne à tous les intérêts une même tendance? Les physionomies nationales, en s'effaçant sur tout le continent, éteindront-elles en même temps les rivalités des peuples, et ce sentiment intime par lequel chaque individu, chaque nation se désigne à elle-même, et qui lui est propre comme sa respiration? J'ai de la peine à le croire; quelque chose me blesse dans cette image; et lorsque je parcours les rivages du Tibre, j'ai beau lire les arrêtés du Conseil d'état affichés sur les poteaux, un cri de ma conscience me révèle que je suis chez les antiques Romains et non sur les\setcounter{page}{365} bords de l'Ardèche ou de l'Aveyron. Les peuples sans histoire et sans traditions peuvent se fondre les uns dans les autres ; mais ceux qui doivent aux siècles une longue renommée leur conservent involontairement un culte, qui devient l'ame de leur nationalité; et je doute que des décrets puissent détruire ce culte et anéantir l'histoire. Toutefois, la ruine presqu'universelle de tout le système féodal, de toutes les anciennes corporations, de toutes les institutions qui en provenoient, doivent, en proclamant l'égalité des droits, être l'action des forces individuelles pour les placer dans la loi, c'est-à-dire, dans la force universelle. Dès lors les distinctions ne pouvant plus appartenir au rang, seront l'apanage exclusif des places et de la fortune, et il n'y aura plus de jouissances prisées que celles qu'on obtiendra par ces deux moyens. Dès lors tous les intérêts, toute la tendance de la civilisation iront à se les procurer; et comme les moyens pour y parvenir sont les mêmes partout, toute la population européenne sera inspirée par le même génie, et c'est ainsi que le règne mercantile s'établira sur les débris de la féodalité et des privilèges. Règne inconnu pour nous, et dont les résultats se voilent dans l'avenir. Ce qu'il y a de singulier dans l'histoire\setcounter{page}{366} de cette année, c'est la lutte que Napoléon fait à cette tendance du siècle. Il veut pour ses intérêts politiques, briser l'influence du règne et des mœurs mercantiles; et il ne peut y réussir. Personne n'eût imaginé jadis de mettre le monde en armes pour des denrées coloniales; et si même on l'eût essayé, cela eût été facile: car les mœurs publiques n'y auroient point opposé de résistance; elles y attachoient peu d'intérêt. Aujourd'hui, c'est la moitié de la vie, c'en est les deux tiers; et, les peuples appellent indépendance et liberté, le droit de recevoir du sucre de qui bon leur semble.
Cette lutte aussi finira, et l'ordre politique qui lui succédera sera forcé de sanctionner les nouvelles mœurs, qui sont à la fois la cause et le résultat de la marche séculaire de la civilisation. Ce n'est qu'alors qu'il sera possible d'en apprécier toutes les conséquences; mais il est curieux de signaler dans la route du temps, les faits par lesquels ces grands changemens s'annoncent.
Je ne sais comment je me trouve, monsieur, si loin de mon sujet; cette marche générale des choses ne lui est cependant pas tout-à-fait étrangère: car les relations de la culture et du commerce sont intimes, et l'Italie est l'un des pays où l'on peut le mieux remarquer cette marche. La destruction\setcounter{page}{367} de tous les vestiges de la féodalité, et le nouvel esprit qui la remplace ne s'y sont pas opérés, comme en France, par la plus épouvantable violence; l'exemple, la force des choses, et le droit de conquête y ont importé ce mouvement, et on peut facilement en observer les oscillations. C'est un point de vue sous lequel l'Italie offre aujourd'hui un grand intérêt.
Ses vieilles aristocraties sont tombées au prémier souffle, ses petites souverainetés ont disparu ; de plus grandes divisions territoriales ont donné un moment aux Italiens l'espérance de se réunir après vingt siècles en un seul corps de nation ; ils ont reçu une législation commune avec une impulsion semblable ; enfin, on les a tout à la fois débarrassés de tout ce qui étoit privilége, et on les a jetés dans les cadres de la plus redoutable des armées. Etonnés d'y figurer, ils se sont faits soldats parce qu'ils n'avoient pas d'autre parti à prendre ; et là, ils ont combattu comme anonymes, ou sous le nom de Français ; mais avec une bravoure, qui leur a appris qu'ils n'étoient pas plus étrangers aux périls de la guerre qu'aux délices du repos.
De si grands changemens dans la destinée de ce peuple en ont apporté dans leurs habitudes, et dans leurs intérêts. Ils ont voyagé\setcounter{page}{368} en masse, déportés qu'ils étoient par la conscription; et ils ont ainsi été chercher dans l'étranger de nouvelles séries de mœurs et de connoissances, tandis que ces mêmes étrangers apportoient chez eux de nouvelles institutions et des formes réglementaires inconnues. Un gouvernement rapide et extrême dans ses volontés a établi violemment chez eux une police et une exactitude qu'ils ne connoissoient pas, pendant que la conscription enlevoit cette classe oisive de la société qui encombroit les rues et donnoit aux Italiens cette réputation de fainéantise, dont les étrangers la gratifioient; les classes inférieures étoient ainsi déportées ou refoulées dans les rangs des hommes laborieux, dans le même temps où les hautes classes dépouillées de tous leurs privilèges, et appauvries par les vexations de la guerre, ont été obligées de prendre un esprit d'intérieur et des goûts domestiques qui manquoient totalement à leur moralité. La suppression des couvens, en laissant aux mères l'éducation de leurs enfans leur a inspiré ce besoin des convenances, instinct de l'amour maternel, et qui a peu-à-peu banni cette tolérance de mœurs, dont l'habitude seule balançoit l'immoralité; par-là l'esprit de famille achèvera peut-être de s'établir en Italię.
Tout\setcounter{page}{369} cela, monsieur, est encore en ébauche; et il faut beaucoup de temps pour satisfaire un changement complet dans la moralité d'un peuple; mais on peut le prévoir parce que les circonstances l'appellent: ce qu'on aperçoit distinctement aujourd'hui c'est une activité beaucoup plus grande; un amour d'ordre, d'arrangement dans les affaires, un desir d'améliorer sa situation par l'intelligence et l'économie; un desir enfin de s'employer à quelque chose, penchant qui autrefois ne se portoit qu'à n'être bon à rien. Enfin on y sent répandu partout le principe moteur de cette action; la conscience de ses moyens et une certaine estime de soi, dont les Italiens avoient jadis consenti à se priver, avec la sanction universelle, et qu'ils regagnent aujourd'hui à la pointe de l'épée.
Le mouvement vers l'amélioration et l'Ordre s'est porté immédiatement vers l'agriculture, puisque l'Italie n'a point de fabriques, et que ses ports étoient à peu-près fermés au commerce d'échange. Cette amélioration n'avoit pas un champ très-vaste à parcourir, parce que l'Italie, cette vieille patrie de la civilisation, avoit déjà éprouvé toutes les phases de la prospérité publique, et que sa culture n'est pas susceptible de beaucoup\setcounter{page}{370} d’améliorations. D’ailleurs, elles ont un but et un terme, qui est celui de la consommation. Or, l’Italie abonde en excellentes denrées; elles y sont généralement à bas prix : on ne peut donc guère se flatter d’obtenir de grands bénéfices en rompant l’équilibre de la culture actuelle pour introduire de nouvelles séries de productions. Cette innovation est toutefois possible, et je vous indiquerai, monsieur, à mesure, les changemens qui se sont essayés et ceux qu’on pourroit tenter encore.
Jusqu’à présent, l’Italie a cherché à perfectionner la culture des grains et des chanvres, afin de remplacer, pour la consommation de toute la côte de Gênes et de Provence, les blés et les chanvres que la mer, devenue stérile, n’apportoit plus dans les ports de cette côte. C’est à Gênes, Savone et Nice, que se rendoit journellement une masse énorme d’approvisionnemens que fournissoit la Lombardie.
Mais avant d’entrer dans les détails de la culture italienne, je dois chercher, monsieur, à en établir les grandes bases. Pour y parvenir, je diviserai l’Italie en trois régions agricoles, dont l’exploitation et les produits n’ont aucuns rapports. La première, que je désignerai sous le nom de région de Maïs, comprend toute la plaine de Lombardie du pied des Alpes jusqu’aux versants septentrionaux de l’Appennin, qui s’étendent de Coni jusqu’à Bologne. La seconde, appelée\setcounter{page}{371} région des Oliviers, s'étend sur tous les versants méridionaux de l'Appennin, de Nice jusqu'à Florence; et tournant au midi avec la chaîne des monts, elle se prolonge avec elle jusqu'au fond de l'Italie. Parallèlement à cette chaîne, qui divise l'Italie, la troisième région désignée sous le nom de Pays de mauvais air, s'étend le long du rivage de la Méditerranée, de Pise jusqu'à Terracine.
Nous étudierons successivement la culture de chacune de ces trois régions. J'aurai peu de chose à vous dire de la seconde, si bien décrite par Mr. Sismondi; je voudrois pouvoir réussir aussi bien dans la description des deux autres régions; et je l'essayerai dans la première lettre que j'aurai l'honneur de vous adresser. Recevez, monsieur, etc.

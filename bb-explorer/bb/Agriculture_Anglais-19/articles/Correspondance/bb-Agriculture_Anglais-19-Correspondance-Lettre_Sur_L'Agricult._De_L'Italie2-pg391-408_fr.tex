\setcounter{page}{391}
\chapter{CORRESPONDANCE.}
\section{SECONDE LETTRE à Mr. CHARLES PICTET, l'un des Rédacteurs de la Bibliothèque Britannique, sur l'agriculture d'Italie.}
Asti 10 juillet 1814.
LA première région agricole de l'Italie que j'ai eu l'honneur de vous indiquer, Monsieur, dans une précédente lettre, s'étend du pied des Alpes à ceux de l'Apennin, sur cette immense plaine délaissée par les eaux, qui commencée au passage de Suze, et ne finit qu'aux bornes orientales de l'Italie; divisée par le cours du Pô en deux parties presqu'égales, cette vaste contrée peut être appelée le jardin de l'Europe, et en est sans contredit la partie que la nature a le plus favorisée.\setcounter{page}{392} Le sol déposé par les eaux est presque partout aligné sous un niveau parfait et aussi riche que profond; on ne trouve des bancs de galets qu'en se rapprochant des montagnes; toute la plaine est un terreau noir, d'une grande fertilité. La hauteur des montagnes qui dominent toute la Lombardie y verse une abondance prodigieuse de courants d'eaux, que l'art n'est pas encore parvenu à maîtriser; mais dont il a su diviser à l'infini le mouvement en multipliant partout les canaux d'irrigation; en sorte qu'il n'y a à presque pas de fermés ni de prairies qui n'aient à leur sortie un canal et une écluse.
Ce luxe d'irrigation se déploie sous le plus beau climat, et s'unit à l'action d'un soleil méridional, pour produire tous les phénomènes de la plus vigoureuse végétation.
De si grands avantages naturels ont accumulé dès long-temps une immense population dans cette heureuse Lombardie; et avec elle toutes ses conséquences; le rapprochement des villes et par conséquent des marchés, la beauté des chemins qui y conduisent de tous les points de la campagne, les campagnes elles-mêmes, subdivisées en un nombre infini de petites propriétés, au centre desquels on a bâti la ferme qui les exploite; le sol de ces champs profite avec\setcounter{page}{393} un grand art et de manière à ne laisser perdre pour l'agriculture ni espace ni temps, la succession des récoltes y étant aussi précipitée qu'habilement entremêlée. Les récoltes sont encadrées par des plantations d'arbres à fruits de toutes espèces, mélangés avec des mariers, des peupliers et des chênes; et pour que ces derniers même ne se bornent pas à donner oisivement leur ombrage, ils servent de soutiens aux seps de la vigne, dont les pampres s'élançant de toutes parts les recouvrent comme un dôme, et retombent en festons.
Le luxe des plantations est tel dans presque toute la Lombardie que l'œil du passager n'en peut percer l'épaisseur; il y voyage dans un horizon toujours voilé et qu'il ne découvre qu'à mesure qu'il avance. Cette succession de tableaux, qui prépare toujours à l'imagination quelque chose d'inattendu, cette fraîcheur de verdure, ces habitations sans nombre qui réunissent à la commodité une sorte d'élégance; ces campagnes dont l'ombrage a je ne sais quoi d'agreste, tandis que leur culture annonce la plus riche économie, offrent un contraste et une harmonie que nulle autre contrée ne présente au même degré. On n'y trouve pas cette végétation large et monotone des\setcounter{page}{394} terres de l'Inde, ni des vastes entourées qui s'étendent sur les plaines uniformes du nord; on n'y trouve pas non plus ces sites sauvages dans lesquels les vallons de la Suisse déploient leur fraîcheur; mais on parcourt une nature où ces divers horizons semblent s'unir ensemble pour les rappeler tous à-la-fois. Telles sont les campagnes que la Lombardie livre avec tant de complaisance à l'art de la culture. Cet art y est simple parce qu'il y est très-perfectionné et que les bonnes pratiques rurales y sont devenues usuelles et comme une habitude chez les cultivateurs. L'abondance de la population et la variété des récoltes a dû nécessairement établir la division des fermes, ainsi qu'elle a lieu dans tous les pays d'une haute fertilité et dont la culture exige beaucoup de ces petits soins journaliers que l'intérêt de famille peut seul perfectionner: aussi y a-t-il très-peu de fermes dans toute la Lombardie qui aient plus de soixante arpens de quarante-huit mille pieds carrés, comme aussi il y en a peu qui en ayent moins de dix. L'abondance des capitaux a mis dès longtemps toutes les terres dans les mains des hautes classes de la nation et des habitans des villes, où il n'y a presqu'aucun paysan\setcounter{page}{395} propriétaire, tous les cultivateurs sont métayers, c'est-à-dire exploitants à moitié fruit. Cet usage est universel; on y connaît à peine les baux à rente fixe.
Il résulte de cet ordre de choses que bien que la nature eût semblé destiner la Lombardie à figurer entre les pays de grande culture; elle est devenue par l'état des institutions sociales, un véritable pays de petite culture.
Je vais, Monsieur, pour rendre plus sensible les divers procédés de culture, vous décrire la disposition et l'histoire d'une propriété située à quelques lieues de Turin, où j'ai passé quelques jours et étudié les pratiques rurales. Elles sont tellement les mêmes dans tout le Piémont, que cet échantillon donnera une idée complète de l'ensemble.
Cette terre contient près de trois cents arpens, divisés en quatre exploitations. Ils se prolongent le long d'un canal, sur une médiocre largeur. Il est ombragé par un alignement à perte de vue d'aulnes, de peupliers et d'arbres de toutes espèces, dont l'immense élévation abrite tout le domaine contre les ouragans.
Le château est situé à l'une de ses extrémités, et devant lui se déploie une riche\setcounter{page}{396} prairie que le canal arrose et qui est parsemée de groupes d'arbres et d'arbustes. Cette prairie forme la réserve du maître; et à son extrémité opposée se trouve la première métairie, dont les toits se découvrent au milieu du beau verger qui encadre la prairie et la rend en même temps un but de promenade, comme un intérêt dans le paysage.
Cette ferme, comme toutes celles de la Lombardie, offre une largeur et une richesse de bâtimens inconnue dans presque tous les autres pays de l'Europe. Construite en briques rouges, réunissant la solidité à l'élégance, elle présente dans sa forme régulière quelque chose de rustique qui lui conserve son aspect champêtre.
Les constructions s'étendent sur les quatre côtés égaux d'une vaste cour; au milieu de l'un de ces côtés s'élève un pavillon à deux étages dont l'effet agréable consiste dans la justesse de ses proportions. Le plain-pied sert au logement du métayer et à serrer ses provisions; l'étage supérieur est destiné aux magasins de grains. Deux corps de logis s'allongent des deux côtés du pavillon et terminent cette face de la cour. Ils ne s'élèvent qu'à la hauteur du premier étage et contiennent l'un l'écurie des bœufs, l'autre celle des vaches; on communique à l'une et\setcounter{page}{397} l'autre par l'intérieur. Ces étables, qui ont douze pieds d'élévation, sont voûtées et blanchies de manière que la poussière ne tombe jamais sur les bestiaux; et rien n'est si propre et si soigné que ces étables où les animaux plongés dans une abondante litière indiquent par leur douceur combien ils ont à se louer des soins qu'on leur donne.
Sur les trois autres côtés de la cour règne un portique de vingt à vingt-quatre pieds de largeur et de quinze à seize d'élévation, dont la toiture repose du côté extérieur sur un rang de colonnes aussi espacées entr'elles qu'éloignées du mur, ensorte que le portique se trouve divisé à chaque colonnement en autant de carrés parfaits.
C'est sous le vaste espace contenu dans le vide de ces portiques que se déposent au niveau du sol les fourrages, les pailles, tous les produits de la ferme, ainsi que les chars et les instrumens aratoires. Une moitié de la cour est pavée, l'autre offre une aire pour le foulement des grains. Le dépôt des fumiers est hors de la cour, ensorte que rien ne la salit, et qu'elle présente au milieu de ses colonnes symétriques un ensemble si régulier et si commode qu'on y éprouve un sentiment d'ordre et de soins dont nos fermes sales et désordonnées ne donnent aucune idée.
\setcounter{page}{398}Tel est le modèle uniforme de toutes les fermes de la Lombardie, aux dimensions près, et qui devroit être celui de toute l'Europe; car c'est le modèle qui présente le plus de vide avec le moins de construction, celui qui assure la plus parfaite conservation des denrées avec le plus de facilité d'en disposer; c'est à-la-fois le plus économique et le moins exposé au feu. Il exige sans doute pour sa construction une masse énorme de briques, mais chaque propriétaire les fait fabriquer lui-même, il prépare à l'avance tous les objets nécessaires à cette fabrication; et des ouvriers du métier les façonnent, à forfait. Cette opération n'est ni aussi embarrassante ni aussi coûteuse qu'on pourroit le croire et ne se représente pas souvent.
Les murs extérieurs de la ferme, au lieu de pierres étoient partout tapissés de vigne dont les gros raisins donnent un mauvais vin; mais que le metayer consomme et que l'habitude lui fait trouver passable. Une porte extérieure s'ouvre du pavillon sur le jardin, il est séparé par une haie, des terres arables, et orné de quelques figuiers, d'arbustes et de fleurs. Sous les portiques sont ménagées de grandes portes auxquelles aboutissent les chemins de transport qui desservent les diverses parties de la ferme et en forment les divisions.
\setcounter{page}{399}La partie du domaine qui avoisine le canal est destinée à une prairie à demeure, qui s'arrose par inondation et dont la végétation toujours active permet de la faucher trois fois ; les herbes qui y croissent sont l'avena elatior, le paturin, le ray-grass, le plantain lancéolé et les différens trèfles. La prairie occupe assez généralement un quart de la ferme, les trois autres sont réservés aux terres arables. Celles-ci sont divisées par des rangées d'arbres, le plus souvent de mûriers, quelquefois aussi d'érables et de cerisiers qui portent de la vigne et multiplient ainsi les récoltes sans occuper d'espace.
La totalité de cette métairie renferme à peu-près soixante arpens ; et lorsqu'on les a traversé, on arrive par des chemins ombragés de mûriers, à la seconde ferme qui est en tout semblable à la première, et de celle-ci à la troisième ; cet ensemble, avec la réserve du maître et quelques bois, forme une des belles terres du Piémont.
Dans chacune de ces fermes vit une famille de métayers ; ils habitent souvent la même ferme, de père en fils, elle est pour eux comme une antique patrie et ils ne songent guère à en renouveler le bail, qui se poursuit d'âge en âge sous les mêmes conditions sans écriture et sans enregistrement.\setcounter{page}{400} Par ces conditions le cheptel appartient au maître; mais le métayer jouit de son revenu, moyennant une rente fixe qu'il est tenu d'acquitter en argent et qu'on évalue à la moitié du revenu net de la prairie: c'est-à-dire à 40 fr. l'arpent; mais il jouit des récoltes de trèfle sans aucune redevance. Toutes les autres récoltes sont partagées en nature, en présence de l'agent du maître, blés, maïs, vins, chanvres, soies, etc.
Cette manière de contracter est singulièrement avantageuse au propriétaire, lequel, sans autre avance que celle des impôts, reçoit une rente fixe de sa prairie, et une moitié franche de tous les produits bruts de sa terre; valeur sur laquelle il peut facilement spéculer pour ne la vendre qu'au moment le plus favorable: car n'ayant point de débours à faire pour l'exploitation, il est pour ses denrées dans la position d'un négociant, et il est rare qu'il n'en profite pas. Mais cette économie ne peut avoir lieu que dans une contrée où le peu d'étendue des fermes et la réunion des terres, permet de les cultiver avec les bras de la famille; où le travail s'opère avec des bœufs dont l'élève et l'engrais sont un produit pour le métayer, au lieu d'exiger une perte annuelle comme les chevaux, et où le climat\setcounter{page}{401} et la fertilité du sol favorisent un emploi continuel du terrain, une grande variété de récoltes et un produit un peu plus élevé des grains : le métayer dans ce cas ne payant pas de rente fixe, et travaillant avec sa famille, n'est pas appelé à des avances d'argent. Il entretient son ménage sur les récoltes de menus grains et se procure suffisamment de numéraire par le produit de la basse-cour et la vente de sa part des blés.
Ce genre d'économie est en même temps celui qui fait abonder sur les marchés la plus grande quantité de denrées. J'avance ceci contre l'opinion d'Arthur Young, qui attribue exclusivement cet avantage aux grandes fermes. Mais en étudiant l'histoire de l'exploitation que je viens de décrire, on sentira en premier lieu, que la multiplication des fermes, multiplie en même temps les plantations, les jardins, les basse-cours, et obtient ainsi du sol une abondance de petites productions qui est perdue par les grands fermages. En second lieu, le métayer forcé de vivre d'économie, profite avec soin de toutes ses menues denrées afin de pouvoir conduire au marché sa denrée vendable; c'est-à-dire son blé, quantité qu'on peut évaluer à un quart de la production totale de la ferme; la portion du maître s'y présente en totalité; ensorte que dans ce systême\setcounter{page}{402} le système les trois quarts du produit brut de la ferme s'offrent en vente, il présente ainsi accroissement dans la masse de production et ménagement dans la consommation intérieure. Je crois qu'aucun pays ne peut mettre en vente une aussi grande proportion de son produit; elle ne doit aller en France qu'au tiers à en juger par la proportion qui y existe entre les habitans des villes et des campagnes; en Angleterre elle s'élève peut-être à la moitié; en Suisse elle est presque nulle; et c'est pourquoi la vie animale y est aussi chère.
- L'accumulation des villes est énorme dans le Piémont; et ce pays dont l'étendue bornée est disputée par un grand espace de montagnes, alimente encore en grains et en bestiaux la rivière de Gènes, Nice et jusqu'au port de Toulon. Sans pouvoir en faire un calcul exact, on sent d'après cela qu'il y a une surabondance de denrées dans ce pays, qui doit-être attribuée à son économie générale, plus encore qu'à sa fertilité absolue: car le blé ne rend pas tout-à-fait le six pour un dans le Piémont.
- Mais il faut convenir que cette économie n'est convenable que dans les contrées où les avances de capitaux l'ont mis dès long-temps l'agriculture au point d'un maximum de production, où l'expérience a déterminé\setcounter{page}{403} en excellent ordre d'assolemens et où la division convenable des propriétés est fixée : dans tout pays d'amélioration et qui par conséquent demande des avances de capitaux, il n'y a que les rentes fixes à long terme qui puissent les faire mettre en dehors, et préparer par leur moyen leur prospérité future.
Mais il est temps, Monsieur, que je vous entretienne de la culture pratiquée dans la ferme que j'ai entreprise de vous décrire.
Elle a soixante arpents, dont quinze en prairies; le reste en terres arables, pour la plupart plantées; sur ces dernières, à peu près dix sont semées en trèfle. Cette dernière récolte, jointe au produit du foin, entretient huit bœufs et treize vaches ou élèves, dont deux jeunes bœufs, et un méchant cheval dont le seul emploi est d'aller au marché et de fouler les grains; en tout vingt-deux têtes, ou environ une par arpent de terre à fourrage. Les bêtes à cornes sont de la race du Querci, répandue dans tout le midi de la France, en Dauphiné et en Savoye. Elles sont seulement plus élancées et ont les cornes plus petites; mais elles ont les mêmes caractères, le même poil fauve clair, la même différence de taille entre le mâle et la femelle, en sorte que la vache reste petite et de vilaine forme, tandis que le bœuf devient très-grand et très-musculeux,\setcounter{page}{404} mais sans acquérir cependant de belles formes. Quoiqu'il y ait prodigieusement de bétail en Piémont, les cultivateurs n'ont pas appris à tirer, à l'exemple du Milanois, un grand parti du laitage; leurs vaches sont peu laitières, aussi l'élève et l'engrais des bœufs est beaucoup plus prisé. Ainsi, dans cette métairie, on élève chaque année une paire de bœufs; à la troisième on commence à les atteler pour les petits travaux de la ferme; dans la quatrième et la cinquième ils font le gros travail; à cinq ans on les engraisse, ils atteignent souvent la valeur de 1000 à 1100 francs; c'est un des meilleurs revenus du métayer; ils sont toujours engraissés de pouture et on les finit avec la farine de maïs. Enfin, pour l'exploitation de quarante-cinq arpens on emploie deux paires de bœufs de quatre et de cinq ans, qui conduisent deux charrues, une paire de trois ans chargés de tous les petits travaux, et deux paires de jeunes élèves, avec le roussin, qui foule le blé et va au marché. Chaque charrue a ainsi trente-deux arpens à labourer dans la saison. Vous avez si bien décrit, monsieur, il y a quelques années, la belle charrue du Piémont, ainsi que l'art avec lequel ces habiles laboureurs savent la manier, que je crois superflu de le répéter\setcounter{page}{405} ici. Je ne puis cependant m'empêcher de vous parler de la manière dont ils sont parvenus à exécuter avec leur seule charrue tous les travaux de culture sur récolte, et de binage ; pour lesquels on a inventé une foule d'instruments en Angleterre. Rien n'est plus net et plus exact que les binages donnés au maïs en pleine végétation avec une charrue à deux bœufs, sans qu'une seule plante soit offensée, et en détruisant complètement toutes les plantes parasites. Je puis de même vous assurer que les pommes de terre que j'ai admirées à Hofwyl n'étoient pas mieux traitées qu'un champ de vingt arpens que j'ai examiné à la Mandria et dont toutes les cultures avoient été faites avec la seule charrue.
L'assolement généralement suivi est de quatre ans, savoir :
Première année: Maïs fumé, Haricots id., Chanvre id.
Seconde année: Blé.
Troisième année: Trèfle labouré après la première coupe, suivi d'une jachère.
Quatrième année: Blé.
Cet assolement peut être rangé parmi les plus productifs ; et le maintien de la fertilité du sol prouve que, malgré la répétition des céréales, il peut se poursuivre indéfiniment. A la vérité il faut attribuer\setcounter{page}{406} résultat, à l'abondance des engrais fournis par une prairie fauchée trois fois et qui se reversent en entier sur les terres arables.
La culture du maïs est regardée comme préparatoire dans cet ordre : on lui réserve tous les engrais ; les sarclages, les buttages maintiennent le terrain dans une propreté complète ; rien aussi n'est si beau que la récolte qui en provient et celle qui la suit.
Ces plantes, rangées dans un ordre parfait, élevant majestueusement leurs fleurs jaunissantes, donnent, je ne sais quel air de pompe aux campagnes d'Italie, qui ajoute à leur beauté.
Le produit de maïs est assez considérable; mais il a sur-tout l'avantage de nourrir presqu'uniquement toute la population champêtre du Piémont, qui en mange le grain sous toutes les formes. On entremêle cette culture d'une quantité de haricots de diverses espèces et de beaucoup de chanvre.
La récolte du maïs est terminée en septembre, et on prépare sur le champ la terre pour la semaille des blés. On les sème sur des billons très-étroits, sous raie et enterrés à la charrue, sur un sol très-net et qui a été au printems abondamment fumé. Nul autre soin n'est donné au blé jusqu'à sa récolté, qui a lieu dès le commencement de juillet.
Aussitôt que le blé s'est desséché dans les\setcounter{page}{407} tas placés sous les portiques de la cour et dans les jours chauds du mois d'août, on le foule sur l'aire préparée au fond de la cour; au lieu de le faire dépiquer par un immense troupeau de haridelles, suivant la stupide coutume de Provence, ou de le laisser dévorer par les souris pendant un an, d'après la méchante habitude de Paris, on le foule avec un cylindre traîné par un cheval qu'un petit garçon dirige, pendant que les ouvriers de la ferme retournent les pailles avec des fourches. Cette opération dure à-peu-près deux semaines, elle est aussi économique que prompte; et elle dépouille complétement le grain.
Le trèfle a été semé au printems sur les blés fumés; la végétation active de l'Italie le fait monter en fleurs dès la première automne, et donne en octobre une bonne coupe, après laquelle il sort avec la prairie au parcours d'automne. Au printems, il se revêt d'une nouvelle verdure, fleurit, se fauche encore une fois, mais les grandes chaleurs ne permettant plus d'espérer une seconde coupe, on se hâte de le retourner; et le sol reçoit une jachère de trois cultures avant la semaille du blé.
Ainsi, dans cet assolement de quatre ans, on trouve trois récoltes destinées à la nourriture de l'homme, une jachère et deux récoltes\setcounter{page}{408} pour les animaux. A ces produits il faut joindre celui du chanvre qui est quelquefois considérable, celui des soies, du vin, des légumes, fruits, basse-cours, et enfin celui de l'élève, du laitage et de l'engrais des bestiaux.
D'après ces détails, vous voyez, monsieur, qu'une ferme de soixante arpens alimente une famille, composée de huit ou neuf individus; qu'on y entretient vingt-deux têtes de gros bétail, dont deux bœufs et une vache sont engraissés chaque année, ainsi qu'un ou deux cochons; qu'on y récolte pour vingt-cinq louis de soie au moins; qu'elle fournit plus de vin que la consommation n'en exige; que la récolte préparatoire du maïs et des haricots alimente presqu'uniquement les métayers; et que presque tout le blé peut se livrer au commerce, ainsi qu'une foule de menues denrées, il vous sera, d'après cela, facile de concevoir comment le Piémont est peut-être de tous les pays du monde celui où l'économie et l'administration des terres est la mieux entendue et le phénomène de sa grande population et de son immense exportation de denrées vous sera expliqué.
Recevez, monsieur, etc.
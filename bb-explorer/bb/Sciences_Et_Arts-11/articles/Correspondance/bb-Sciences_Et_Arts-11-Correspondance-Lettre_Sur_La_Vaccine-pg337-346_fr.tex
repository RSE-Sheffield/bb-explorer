\setcounter{page}{337}
\chapter{Correspondance}
\section{LETTRE AUX RÉDACTEURS DE LA BIBLIOTHÈQUE BRITANNIQUE.}
Vienne le 27 Juillet 1799.
AYANT eu l'occasion de vérifier les expériences du Dr. Jenner sur la vaccine, je m'empresse de vous en faire connaître les premiers essais. Quoiqu'en fort petit nombre, leur coïncidence avec les observations & la description de Jenner est infiniment satisfaisante ; l'on peut du moins en conclure que rien ne sera plus facile que l'adoption de cette nouvelle méthode, dans les pays mêmes où les vaches ne seraient pas sujettes à cette maladie. Le Dr. Pearson de Londres envoya, ce printemps, à notre compatriote, le Dr. Peschier, des fils imprégnés de matiere vaccine variolique, fixés sur le papier d'une lettre au moyen d'oublies à cacheter. Un médecin de cette ville se détermina sur le champ à inoculer avec ces fils trois de ses enfans, dont l'un avoit eu la petite-vérole, mais non les deux autres. L'inoculation\setcounter{page}{338} se fit par une légère incision de l'épiderme, dans laquelle on inséra à chaque bras un fil imprégné, de cette longueur-, & qui fut fixé à l'incision par le moyen d'un emplâtre adhésif. L'inoculation ne produisit aucun effet sur l'enfant qui avoit déjà eu la petite-vérole.
Dans l'un des deux autres enfans l'inoculation ne réussit point, c'est-à-dire, l'incision se referma sans inflammation, ni éruption. Dans le troisième elle prit aux deux bras & produisit sur chacun une pustule absolument semblable à celles qui sont représentées dans la gravure seconde & troisième de l'ouvrage du Dr. Jenner. L'enfant eut le 8me. jour une fièvre légère qui dura près de trois jours, mais avec des remissions considérables. Je n'eus pas l'occasion de faire un Journal exact du cours de cette maladie ; je vis cependant l'enfant très-souvent & j'ai remarqué qu'elle avoit été absolument semblable aux deux cas dont je vais parler, à quelque petite différence près dans le degré de la fièvre qui l'accompagnoit. J'observerai avant d'aller plus loin, qu'une précaution recommandée par le Dr. Pearson, fut négligée dans l'inoculation du 2d. enfant ( c'est-à-dire de celui des deux qui n'avoit jamais eu la petite-vérole, & sur lequel l'inoculation de la vaccine ne laissa pas que de manquer) savoir, celle de tremper les fils dans de l'eau chaude, & qu'au moment\setcounter{page}{339} où l'emplâtre fut appliqué, les personnes présentes eurent des doutes qu'ils eussent été bien inférés dans l'incision.
Encouragé par le succès de l'inoculation du 3me. enfant, & trouvant tant de raisons de croire à la véracité du Dr. Jenner, en voyant tout ce qu'il avoit dit & décrit se vérifier si exactement; je me déterminai sans peine à inoculer de la maniere ordinaire, mon fils aîné avec la matiere du bras de l'autre enfant. Voici le Journal exact de la maladie:
Charles D., âgé de 3 ans, jouissant d'une fanté parfaite, à l'exception d'une légere rougeur accidentelle de l'œil, fut inoculé le 10 du mois de mai, sans avoir pris aucun remede préparatoire. Une seule piqûre fut faite à chaque bras.
13. Mai, la piqûre des deux bras commence à devenir rouge & à s'enflammer. D'ailleurs l'enfant est fort bien.
14. La pustule commence à se former, est dure au toucher & le cercle rouge s'étend. D'ailleurs comme hier.
15. Le cercle comme hier; les pustules s'enflent & il s'y forme une matiere claire. L'enfant se porte bien.
16. Les pustules augmentent, la matiere est claire; vers le soir il étoit plus pâle qu'à l'ordinaire & a eu quelques frissons.
17. De la fievre pendant la nuit; il s'est plaint.\setcounter{page}{340} de soif & de chaleur; le matin encore quelques légers frissons, mais peu-à-près la chaleur du corps est devenue naturelle & l'enfant aussi gai qu'à l'ordinaire. Les pustules augmentent en grosseur & le cercle rouge en largeur. La matiere est toujours limpide.
18. Point de fièvre pendant la nuit. La circonférence des pustules & des aréoles augmente. La matiere est claire mais plus abondante. Il n'y a pas une seule pustule, sur tout le corps.
19. Pendant la nuit il s'est plaint quelque fois de froid. Pustules plus pleines & aréoles plus larges. Pendant le jour, aussi gai que de coutume. Pendant tout le cours de cette petite maladie ses fonctions ont été les mêmes qu'en parfaite santé.
20. Une bonne nuit; il se forme une croûte au milieu de la pustule.
21. La pustule se sèche, la croûte se forme du centre à la circonférence & les aréoles rouges disparaissent. La matiere a toujours été claire & limpide. La croûte s'est séchée graduellement & bénignement.
15. Juin. Les croûtes sont tombées & la peau a conservé la fossette ordinaire.
15. Juillet. Après avoir pris un purgatif de manne & de jalap, l'enfant a été inoculé avec de la matiere de petite-vérole. Sa santé est parfaite.
17. Les piqûres sont un peu rouges.
\setcounter{page}{341}
18. Il a ouvert les piqûres en se grattant.
19. Il se forme une croûte molle ; mais il n'y a point d'inflammation, & l'enfant se porte fort bien.
21. Aucun symptôme de fièvre. Une des croûtes est sèche, l'autre est encore molle.
23. Les deux croûtes sèches. Point de fièvre.
24, 25 & 26 point de fièvre.
\subsection{Observation I I}
Pierre D. âgé de 18 mois fut inoculé le 20 mai, avec la matière prise du bras de son frère. C'est un enfant fort & vigoureux, qui a toujours joui de la meilleure santé ; mais est sujet à des croûtes assez épaisses qui paroissent de temps en temps sur diverses parties du corps. Au moment de l'inoculation, il en avoit beaucoup sur les jambes ; 8 jours avant & le jour même de l'inoculation, il a pris un purgatif de manne. L'inoculation s'est faite par trois piqûres aux bras.
21. Les piqûres du bras droit deviennent rouges.
23. Il se forme une vésicule sur chaque piqûre.
25. Elles s'augmentent & se remplissent d'une matière claire. Une d'elles a disparu.
27. Il a été inquiet pendant la nuit ; il a eu de la chaleur aux mains & dans tout le corps.\setcounter{page}{342} & beaucoup de soif. Le matin ces symptômes avoient disparu. Les deux pustules croissent lentement.
28. De 9 heures jusqu'à minuit les mêmes légers symptômes de fièvre. Il a enlevé avec les ongles les pustules. D'ailleurs il est fort bien.
29. Il se forme une nouvelle croûte. Dans tout le cours de sa maladie il n'a paru aucune éruption. La matière des pustules a toujours été limpide & n'est jamais parvenue à la consistance de pus.
5. Juillet. Les croûtes sont tombées. Elles ont été souvent emportées par les ongles de l'enfant pendant son sommeil; mais cependant les nouvelles ont toujours été de la nature la plus bénigne, & n'ont jamais rongé profondément. Enfin pour empêcher l'enfant de se gratter, les bras ont été panés avec de l'emplâtre de céruse & l'on lui a mis un bandage; au-bout de quelques jours ils étoient parfaitement cicatrisés.
15. Juillet. Après avoir pris un purgatif il a été inoculé avec de la matière de petite-vérole. Il a encore beaucoup de croûtes sur une jambe & se porte très-bien.
17. La piqûre du bras gauche est un peu enflammée & a un cercle rouge fort étroit. Celle du bras droit est presque sèche.
19. Les deux piqûres sont tout-à-fait sèches\setcounter{page}{343} & le cercle rouge a disparu. L'enfant se porte fort bien.
21. Aucun symptôme de fièvre.
23. Idem.
24, 25 & 26. Idem.
Ces trois observations font certainement remarquables par leur ressemblance à celles des médecins Anglais. Elles nous prouvent que la matiere vaccine variolique ne perd point de sa force par un long voyage; que la maladie est absolument la même quoique son venin ait passé par plusieurs individus; que la maladie est plus douce que la petite-vérole inoculée la plus bénigne; que rien ne met obstacle à cette maniere d'inoculer dans les pays même où la vaccine n'est pas connue.
La facilité qu'il y a à répandre cette maladie par le moyen des fils, d'un pays ou d'une ville à l'autre, refute les doutes que vous avez témoigné ( Sc. & Arts. Vol. IX. p. 370, 371. à la note ) sur la possibilité de pratiquer cette inoculation hors de l'Angleterre.
La méthode de l'introduire par le moyen des fils imprégnés est certainement préférable à celle de la communication du javart du cheval à la vache; puisque ce ne seroit qu'introduire sans nécessité une maladie de plus chez les vaches.
Quant au point important de cette maladie, que les médecins Anglais nous assurent n'être jamais contagieuse, il n'est pas indifférent d'observer\setcounter{page}{344} que mon fils cadet a constamment couché dans la même chambre que son frere depuis le 10 jusqu'au 20 de mai ; qu'il n'a cessé d'être exposé à la contagion, & que néanmoins il ne s'en est manifeste aucun signe ; que l'inflammation des yeux de l'ainé s'est guérie pendant le cours même de la maladie, & qu'aucune éruption n'a paru sur le corps du cadet, malgré que sa peau dut en être plus susceptible, à cause des croûtes qu'il avoit aux jambes & aux cuisses, qui n'en ont point été augmentées. En cas que vos amis de Londres ne vous aient pas déjà envoyé des fils imprégnés de matiere vaccine, je vous envoye ci-inclus un morceau de linge qui l'est, au moyen duquel vous pouvez mettre les médecins de Genève à portée de répandre dans notre Patrie les avantages de cette intéressante découverte. Priez-les de vouloir bien essayer si l'enlèvement de l'épiderme par un morceau d'emplâtre vesicatoire & l'application d'un fil retenu sur la partie par un emplâtre adhesif, ne réuffiroit pas aussi bien qu'un fil inséré dans l'incision. Cette méthode feroit infiniment préférable, vu qu'il feroit beaucoup plus facile de retenir le fil sur une surface que dans une insertion aussi étroite que celle qui se fait avec la pointe d'une lancette. Je me propose d'en faire moi-même l'essai incessamment ; je vous prie de me communiquer le résultat de ceux des médecins Genevois.
\setcounter{page}{345}
Comme il est à désirer pour le bien de l'humanité que cette méthode devienne générale, veuillez bien donner dans votre Recueil une place à ces premiers essais, qui me paraissent fort encourageants. D'après ce que je puis entrevoir, cette méthode prendra faveur dans ce pays ; j'aurai l'occasion de vous mander le résultat de plusieurs inoculations semblables dont je vais être chargé & que le succès que j'ai eu avec mes enfants a décidées. Je n'ai pas encore eu l'occasion de faire sur cette maladie des vaches des recherches exactes ; cependant d'après une foule d'informations prises de tout côté, il paraît qu'elle n'est point connue des médecins vétérinaires de ce pays, ni des particuliers qui possèdent des laiteries considérables dans plusieurs endroits de la Monarchie Autrichienne.
Quant au javart, soit les médecins vétérinaires, soit les amateurs de chevaux, ne me semblent pas avoir une idée bien distincte de cette maladie ; ils n'ont du moins aucune connaissance de sa propagation chez les maréchaux & palfreniers. Au reste, il me paraît que dans ce pays ce sont les femmes seules qui traient les vaches.
Un gentilhomme Anglais qui a quitté depuis peu l'Angleterre, m'a dit il y a quelques jours, que l'inoculation de la vaccine est très-généralement en vogue en Angleterre & que dans\setcounter{page}{346} le courant de ce printemps, S. A. R. le Duc d'York, Commandant en chef des troupes de S. M. B. a ordonné, sans en forcer aucun, que tous les soldats des régimens des gardes du Roi, qui n'avoient jamais eu la petite-vérole, fussent inoculés avec la vaccine; tous l'ont eue heureusement, excepté quatre chez qui elle n'a pas pris; tous ont été inoculé avec de la petite-vérole ordinaire & aucun n'en a eu le moindre symptôme.
J'ai l'honneur d'être, &c.
J. DE CARRO, D. M.
P. S. Depuis que j'ai commencé cette lettre, j'ai inoculé il y a 3 jours, deux enfans avec la vaccine; 4 autres le feront auffitôt que la matiere des puftules fera affez avancée.\footnote{J'ai profité de l'envoi fait par le Dr. De Carro pour inoculer la vaccine à un enfant de 8 ans, d'après lequel je compte en inoculer d'autres; mais je n'ai employé ni le fil, ni le vesicatoire. Ces moyens étoient autrefois en ufage à Genève. Nous y avons entièrement renoncé depuis long-temps, parce qu'ils compliquent l'opération, exigent tous les jours un petit pansement qu'on a quelquefois de la peine à obtenir d'un enfant un peu sauvage, & ne permettent pas d'ailleurs de suivre bien les progrès de l'infection. J'ai donc trempé rapidement dans de l'eau chaude un petit morceau de la toile imprégnée. Je l'ai mise sur un verre concave, où je l'ai bien exprimée à plusieurs reprises avec le dos de ma lancette, jusqu'a-ce que l'humidité qui en fortoir étant un peu laiteufse, j'ai po croice le vrius bien délayé dans cette goutte d'eau, avec laquelle j'ai ensuite innoculé à la maniere du Dr. Dimsdale, c'est-à-dire par une simple incision à chaque bras sans aucun pansement. Je rendrai compte dans ce Journal du résultant de l'inoculation que je viens de faire, & de celles qui pourront la fuivre. (O)}
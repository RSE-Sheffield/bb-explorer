\setcounter{page}{242}
\chapter{MEDECINE}
\section{AN INQUIRY CONCERNING THE HISTORY, &c. Recherches sur l'histoire de la petite-vérole des vaches; dirigées dans le but de la substituer à la petite-vérole ordinaire & d'anéantir celle-ci: par G. PEARSON, médecin de l'hôpital de St. Georges, in-8°. p. 116. 2 shell. 6 d. Johnson 1798.}
IL était facile de prévoir que les gens de l'art & les amis de l'humanité chercheroient à constater la singulière découverte du Dr. Jenner sur la petite-vérole des vaches\footnote{Voyez notre extrait de l'ouvrage du Dr. Jenner T. IX. Sc. & Arts p. 258. C'est une maladie analogue à la petite-vérole, qui se communique de même par l'inoculation naturelle ou artificielle & qui, infiniment moins dangereuse que la petite-vérole ordinaire, paroît être un préservatif assuré contre celle-ci. (R)} & entreprendroient à cet égard une suite d'expériences raisonnées; le Dr. Pearson vient déjà de faire paroître sur ce sujet un ouvrage qu'on peut considérer comme un excellent développement du texte fourni par le Dr. Jenner.
Après quelques remarques qui servent d'introduction, le Dr. Pearson examine jusqu'à quel\setcounter{page}{243} point on peut regarder comme authentiques & bien prouvés, les principaux faits avancés par le Dr. Jenner sur la petite-vérole des vaches; il y ajoute ensuite ce que ses propres essais, & les informations sûres qu'il s'est procurées lui ont appris sur ce même objet. Nous allons le suivre dans cet examen, & l'on pourra juger des progrès qui ont été faits jusqu'à ce moment dans cette intéressante & utile recherche.
Le premier fait à établir est énoncé de la manière suivante.
1°. Les personnes qui ont éprouvé la FIEVRE SPÉCIFIQUE, & L'ÉRUPTION LOCALE occasionnées par le virus de la petite-vérole des vaches communiqué accidentellement, & qui n'avaient jamais eu la petite-vérole ordinaire, perdent la faculté de prendre cette dernière maladie.
Le Dr. Jenner avait déjà prouvé cette affection par un assez grand nombre de faits, & le Dr. Pearson en ajoute d'autres qui vont à l'appui. Nous citerons de préférence ceux qu'il a observés lui-même; en voici quelques-uns.
* Je fus appelé comme médecin, mercredi 14 juin (1798) conjointement avec Mr. Lucas. pharmacien, pour visiter un malade dans la ferme de Mr. Willan près de la nouvelle route, dans le quartier de Marybone. On entretient dans cette ferme 800 à 1000 vaches. Je profitai de l'occasion pour y faire quelques questions\setcounter{page}{244} sur la petite-vérole des vaches. On me dit qu'elle se manifestait assez fréquemment dans cette ferme, surtout en hiver. On paraissait l'attribuer au passage subit d'une nourriture maigre à une pâture plus substantielle. Les domestiques de la ferme la connaissaient aussi fort bien, & quelques-uns l'avaient prise en trayant les vaches malades. Je demandai à les voir; on m'amena trois des domestiques mâles, Th. Edinburgh, Th. Grimshaw, & J. Clarke, qui avaient la petite-vérole des vaches, & jamais la petite-vérole ordinaire. Je les engageai à se laisser inoculer celle-ci; & pour m'assurer de l'activité du venin variolique que j'employais, j'inoculai en même temps William Kent & Thomas East, qui n'avaient jamais eu ni l'une ni l'autre des deux maladies." Le samedi 17 juin Mr. Lucas inocula en ma présence & celle du Dr. Woodville trois de ces domestiques, savoir; Edinburgh, East, & Kent; il leur fit à chaque bras une incision plus grande qu'on ne la fait d'ordinaire, & il y mit plus de matière purulente qu'on n'a coutume d'en insérer \footnote{C'était sans doute pour rendre le succès de l'inoculation plus assuré. Cependant l'expérience a démontré que plus l'incision est petite & superficielle, mieux elle réussit. J'ai vu fréquemment de grandes & profondes incisions manquer complètement, surtout s'il en sort beaucoup de sang, soit parce que ce fluide lavant la plaie en emporte le virus qu'on y insere, soit peut-être parce que les vaisseaux lymphatiques qui doivent absorber ce virus rampant ou s'ouvrent à la surface de la peau, immédiatement au-dessous de l'épiderme, & que plus profondément il ne s'en trouve pas un aussi grand nombre, soit enfin parce que l'irritation plus vive qui résulte d'une plaie profonde & étendue empêche le jeu de l'absorption. Ce qu'il y a de certain, c'est que les vaisseaux lymphatiques sont d'autant plus sujets à s'enflammer en conséquence d'une irritation à la surface, que cette irritation est moins profonde. On ne voit presque jamais ces fortes d'accidents à la suite d'une grande plaie. Ils sont assez communs dans les plaies superficielles & qui n'ont qu'effleuré la peau; surtout si elles ont été faites avec quelque substance suspecte de saleté. Il suffit donc dans tous les cas d'ouvrir l'épiderme par une incision longue tout au plus de trois millimètres (& de pouce) comme le conseille le Dr. Dimsdale, & telle qu'il n'en sorte que tant soit peu de sang. C'est encore une erreur de croire que plus on met de matière purulente dans l'incision, & plus le succès en est assuré. L'expérience a prouvé que le plus petit atome d'un pus bien choisi, c'est-à-dire bien liquide, & par conséquent bien pénétrant est tout aussi actif qu'une grande quantité de ce même pus, pourvu qu'on ait bien soin d'essuyer à plusieurs reprises la lancette sur l'incision en en tenant avec le pouce & l'index de la main gauche les bords tendus & écartés. (Q)}. On prit le pus dans\setcounter{page}{245} les boutons d'un jeune garçon qui étoit présent & avoit été inoculé 14 jours auparavant par le Dr. Woodville.
\setcounter{page}{246}
"Th. Edimburgh, âgé de 26 ans, servoit depuis sept ans dans la ferme. Il n'avoit jamais eu la petite-vérole, ni la maladie éruptive appelée chicken-pox\footnote{Petite-vérole des poulets. C'est ce que nous appelons la petite-vérole volante. Les Anglais désignent cette maladie par plusieurs noms différents ; petite-vérole de poulets, petite-vérole de cochons (Swine-pox) &c. J'ignore l'étymologie de ces dénominations. Dans une petite brochure publiée il y a 30 ans par le fameux Holwell sur la manière dont les Bramines traitent la petite-vérole dans les Indes, cet auteur affirme que dans les grandes épidémies de petite-vérole, on voit fréquemment les poulets & autres oiseaux de basse-cour en être atteints. Les cochons y font-ils aussi sujets ? Quoiqu'il en soit, je soupçonne qu'il y a plusieurs espèces différentes de maladies éruptives semblables à la petite-vérole, que nous avons mal-à-propos confondues sous le nom de petite-vérole volante. J'ai vu un enfant avoir la véritable petite-vérole volante bien caractérisée un an après avoir eu une maladie que je n'avois pas vue, mais qu'on m'assura avoir ressemblé à celle que je voyois. S'il n'y avoit aucune différence spécifique entre ces deux maladies, on peut donc avoir deux fois la petite-vérole volante, ce qui ne me paroît pas probable. J'ai vu encore plusieurs fois des maladies très-bénignes ressembler assez à la petite-vérole ordinaire, ou à la petite-vérole volante, sans avoir précisément les caractères ni de l'une ni de l'autre. Les boutons étoient en petits nombre, remplis d'un pus blanc & non limpide; mais ils ne duroient que deux ou trois jours. Il seroit important de bien examiner ces différences pour ne pas accréditer l'erreur qu'on peut avoir deux fois la petite-vérole ordinaire. (O)}, ni aucune éruption de ce genre, mais il avoit eu certainement la petite-vérole\setcounter{page}{247} des vaches six ans auparavant. L'éruption, qui s'étoit faite dans les paumes de ses mains, l'avoit forcé à quitter son travail ordinaire pour aller se faire traiter à l'hôpital; & il certifia que son camarade Grimshaw fut atteint en même temps de la même maladie. On découvroit une cicatrice dans la paume de sa main, & nulle part ailleurs. Il dit, que pendant trois jours, dans le cours de son indisposition, il avoit éprouvé une douleur aux aisselles, avec enflure, & qu'on ne pouvoit l'y toucher sans lui faire mal. D'après son rapport, la maladie fut beaucoup plus longue & plus pénible qu'elle ne l'est d'ordinaire; il est possible que l'épaisseur de sa peau, épaisseur qui frappa lorsqu'on lui fit l'incision pour l'inoculer, eût contribué à aggraver les symptômes. C'est là un soupçon que l'expérience seule pourra vérifier`\footnote{Ce soupçon ne me paroît fondé ni en théorie, & ni en pratique. J'ai inoculé un très-grand nombre d'enfans. Je n'ai jamais observé que l'épaisseur de la peau influât d'aucune maniere sur la bénignité de la maladie, & j'avoue que je n'en conçois pas la possibilité. Le virus est absorbé, ou il ne l'est pas. S'il ne l'est pas, il n'aura aucune action. S'il l'est, qu'importe que la peau soit mince ou épaisse? Il ne m'a pas même paru que cette circonstance influât jamais sur le succès ou le non succès de l'opération. Elle manque plus fréquemment sur les enfans à la mammelle qui ont la peau extrêmement mince & délicate, que sur ceux d'un âge plus avancé. (O)}.
\setcounter{page}{248}
"Troisieme jour. On apperçoit un peu d'enflure à l'endroit de l'infection. Aucun autre symptôme ne se manifeste, & l'inoculé ne se plaint de rien."
"Cinquieme jour. L'apparence de l'endroit inoculé au bras gauche ressemble à une piqûre de coufin. Mr. Wacksel, apothicaire de l'hôpital des variolés, observa que l'inflammation avoit paru trop promptement pour ressembler à celle qu'occasionne l'infection variolique lorsqu'elle produit finalement la maladie. Il y avoit eu à l'autre bras une petite croûte qu'on avoit enlevée en la frottant & qui avoit laissé une marque rouge à peine visible. Le sujet inoculé continue à se trouver très-bien."
"Huitieme jour. L'inflammation au bras gauche avoit cessé, & s'étoit terminée par une petite croûte. Le bras droit n'offroit rien de nouveau, & l'inoculé se trouvoit d'ailleurs parfaitement bien."
"Je l'envoyai à cette époque avec Mr. Wacksel à l'hôpital des variolés, où on l'inocula une seconde fois avec du pus pris d'un malade présent."
"Quatrieme jour après la seconde inoculation.\setcounter{page}{249} On vit un peu d'inflammation à l'endroit de l'infection, sur l'un des bras; & rien sur l'autre. L'inoculé ne se plaignit dans l'intervalle que d'un peu de mal de tête le lendemain de l'inoculation."
"Huitieme jour après la seconde inoculation. On vit une petite croûte sèche à l'endroit de chacune des infections. D'ailleurs, aucun autre symptôme de maladie n'avoit paru."
\subsection{SECOND CAS}
"Th. Grimshaw étoit âgé d'environ 30 ans. Il n'étoit en service dans cette ferme que depuis sept semaines; mais il y avoit été six ans auparavant, & y avoit eu, à cette époque, la petite vérole des vaches. Il certifia que son camarade Edimburgh l'avoit eue dans le même temps. Il se rappeloit fort bien d'avoir eu des douleurs aux aisselles, qu'on ne pouvoit toucher non plus sans lui faire mal; mais il fut guéri beaucoup plus promptement qu'Edimburgh."
"Le 19 juin, Grimshaw fut inoculé aux deux bras, à l'hôpital des variolés. On prit le pus d'un malade présent.""
Troisieme jour. On apperçut un peu d'inflammation; & vue à la loupe, l'incision parut contenir un peu de matiere fluide, comme si la maladie dût s'ensuivre. Cependant l'inoculé se trouvoit très-bien."
"Sixieme jour. L'inflammation, qui s'étoit un\setcounter{page}{250} peu étendue autour de l'incision, s'étoit dissipée. Il ne restoit plus qu'une croûte sèche. L'inoculé n'avoit rien éprouvé d'ailleurs. On l'inocula ce jour pour la seconde fois, dans le même hôpital, & par le même procédé."
"Quatrieme jour de la feconde inoculation. On n'apperçoit pas le moindre fymptôme d'inflammation autour de la feconde incision, & l'inoculé ne fe plaint d'aucun mal."
"Huitieme jour de la feconde inoculation. On ne voit pas la moindre trace d'inflammation. L'inoculé s'est trouvé parfaitement bien dans tout l'intervalle."
\subsection{TROISIÈME CAS}
"John Clarke, âgé de 26 ans, avoit eu la petite vérole des vaches dix ans auparavant à Abingdon. Il y avoit été traité par un médecin de l'endroit. Il fut inoculé le 19 juin par Mr Wackfel dans le même hôpital que le précédent, & avec le pus d'un malade présent."
"Troisieme jour. On apperçoit de l'inflammation & un fluide sous l'épiderme à l'endroit de l'incision ; mais ces symptômes seroient prématurés s'ils appartenoient à la petite vérole."
"Sixieme jour. Les apparences d'inflammation, & la matiere fluide qu'on apperçoit au bras droit, font douter si l'infection variolique a produit ou non son effet ; mais on ne voit rien de pareil au bras gauche, & l'inflammation y est tout-à-fait dissipée."
\setcounter{page}{251}
"On inocule aujourd'hui Clarke pour la seconde fois, dans le même hôpital, & toujours avec le pus tiré d'un malade présent."
"Huitieme jour après la seconde inoculation. Aucun autre effet que de l'inflammation, & ensuite dessèchement de la seconde incision faite."
"L'inflammation qui eut lieu au bras droit d'après la première inoculation, se dissipa au bout d'un jour ou deux après le dernier rapport. L'inoculé n'éprouva pas d'ailleurs un seul instant de malaise."
Les deux autres sujets, W. Kent & Th. East, qui n'avoient point eu la petite vérole des vaches, eurent la petite-vérole ordinaire en conséquence de l'inoculation."
D'après ces exemples, & beaucoup d'autres, rapportés dans les termes les plus positifs, il est difficile de se refuser à croire que la fièvre spécifique & l'inflammation locale nommée petite-vérole des vaches rend ceux qui ont éprouvé ces symptômes réunis, non susceptibles de prendre la petite-vérole ordinaire. — Mais rappelons-nous que cette assertion ne repose que sur un certain nombre d'expériences ; & qu'une doctrine aussi nouvelle, si foiblement appuyée par l'analogie, si singuliere en un mot, ne peut être solidement établie qu'à la suite d'expériences très-nombreuses, & dont le résultat ait été uniforme ; car une seule exception renverseroit tout.
\setcounter{page}{252}
Voici le fecond principe qu'il s'agit aussi de prouver.
"11°. Les personnes qui ont éprouvé la fievre spécifique & l'inflammation particuliere qu'occasionne L'INOCULATION DU VIRUS DE LA PETITE-VÉROLE DES VACHES, & qui n'ont jamais eu la petite vérole, sont rendues par-là incapables de la prendre. ▸ Le Dr. Jenner avoit déjà fourni des preuves très-frappantes de la vérité de cet aphorisme ; voici les témoignages du Dr. Pearson, en surcroit de preuve. C'est le Dr. Pulteney de Blandford qui fournit le premier."
"1°. Un fermier de cet endroit inocula sa femme & ses enfans avec de la matiere tirée du pis d'une vache atteinte de la petite-vérole. Au bout de huit jours, l'inflammation parut aux bras, & les malades se trouvérent si mal à leur aise que le fermier en fut alarmé, quoique mal-à-propos ; & il étoit prêt à faire appeler le médecin, lorsque tous les inoculés se sentirent beaucoup mieux. On leur inocula ensuite la petite-vérole ordinaire, mais inutilement; aucun d'eux ne la prit. ▸ On ne s'adressa pas à moi dans ce cas particulier, mais je puis d'ailleurs répondre du fait."
"2°. Mr. Downe, de Bridgport, me fournit le fait important qui suit. "R. F. demeurant près de Bridport, étant âgé de vingt ans, se trouva dans une ferme dont les vaches étoient infectées de la petite vérole. On l'invita à se la laisser\setcounter{page}{253} inoculer, dans le but de se mettre à l'abri de la petite vérole ordinaire qu'il n'avoit jamais eue; il y consentit; on lui fit à la main deux ou trois égratignures avec une aiguille trempée dans le pus d'une vache. Il n'en éprouva aucun effet pendant environ huit jours; au bout de ce terme il survint de l'inflammation aux égratignures, la main enfla, il prit mal à la tête & il eut d'autres symptômes d'une fièvre éruptive. On lui recommanda de se tenir beaucoup en plein air, & il suivit ce conseil; au bout de 4 à 5 jours les symptômes de fièvre diminuèrent à mesure que les boutons de la main approcherent de la maturité; enfin, ils se dessécherent & ont laissé des traces visibles. Mon grand-pere l'inocula ensuite deux fois, & long-temps après, mon pere l'inocula deux autres fois, pour chercher à lui donner la petite - vérole naturelle; mais ce fut inutilement. Il n'eut qu'une légère irritation locale, la même qu'on observe dans les sujets qu'on inocule ayant déjà eu la petite vérole. On ne croyoit pas en l'inoculant, qu'on pût lui procurer la petite - vérole; mais on l'essayoit, soit à titre d'expérience, soit à titre de précaution contre la petite vérole épidémique qui étoit alors dans la famille dont il faisoit partie. Elle y a reparu plusieurs fois dès lors, & il n'a jamais rien fait pour l'éviter, persuadé qu'il en étoit absolument à l'abri.
Mr. Downe me communique un autre cas\setcounter{page}{254} qui ne prouve rien sur la petite vérole naturelle, mais qui mérite attention. "J'ai causé dernièrement, dit-il, avec un homme qui, en badinant, avoit été inoculé à la main avec du pus de petite-vérole des vaches. L'incision se ferma d'abord, puis s'enflamma au bout de quelques jours : il eut l'enflure aux aisselles, le mal-aise, & une fièvre légère. Ces symptômes ne furent pas suivis d'éruption, mais il y eut assez de suppuration à l'endroit des incisions, & il y reste des cicatrices très-marquées."
"3°. Mr. Dolling de Blandford me fait part des faits suivans."
Mr. Juftings d'Axminfter inocula sa femme & ses enfans avec du pus tiré des boutons d'une vache qui avoit la petite-vérole : environ huit jours après l'inoculation il parut beaucoup d'inflammation aux bras, & les malades éprouverent tant de mal-aise qu'on fit venir le Dr. Meach, de Cerne, pour les traiter. Ils ne tarderent pas à se trouver mieux, & furent bientôt guéris. Mr. Trobridge leur inocula ensuite la petite-vérole ordinaire, mais sans succès."
Tels sont les témoignages fournis en addition de ceux déjà mis en avant par le Dr. Jenner. Ils sont en petit nombre ; mais si l'on considère que la petite-vérole des vaches est toujours, à ce qu'on affirme, communiquée par inoculation, casuelle ou intentionnelle, & jamais par simple contact, on doit convenir que tous les cas dans\setcounter{page}{255} lefquels cette maladie a prévenu & empêché efficacement l'action future du virus variolique ordinaire, font des argumens très-probans en faveur de la thesse qu'on cherche à prouver.
111° Aphorifme. Le mal produit par l'inoculation du pus de la vache, ne differe pas de celui qui est produit par l'inoculation d'un pus analogue tiré de l'homme; & on n'observe aucune différence dans les effets occasionnés sur le premier individu humain inoculé du pus de la brute, & ceux produits par le pus élaboré successivement dans cinq créatures humaines, après avoir été originairement tiré de la brute.
Le Dr. Pearson n'a ajouté aucune preuve à celles fournies par le Dr. Jenner à cet égard\footnote{Je regrette qu'un observateur aussi instruit & aussi attentif que le Dr. P. n'ait pas examiné d'une manière toute particulière l'origine de la petite-vérole des vaches, & l'élaboration qu'elle subit dans le corps de ces animaux. Car puisqu'on n'observe aucune élaboration semblable de ce pus dans le corps humain, aucune différence marquée dans ses effets, soit qu'on le tire directement de la vache, ou de l'individu humain auquel il s'est communiqué, j'ai peine à comprendre ce qu'affirme le Dr. Jenner de ce même pus, quand on le tire des chevaux attaqués du javart, c'est qu'alors il n'a que foiblement & incompletement la propriété de garantir de la petite-vérole ordinaire. C'est-là, ce me semble, un point essentiel à examiner dans cette doctrine. Car puisque les individus qui ont pris la maladie des chevaux, ne font pas pour cela complètement à l'abri de la petite-vérole ordinaire, & puisque d'un autre côté, après avoir passé par le pis de la vache, le pus des chevaux n'est plus susceptible d'aucune élaboration particulière, il reste toujours à bien constater s'il est vrai que cette élaboration se fasse dans le corps de la vache, & si elle ne pourroit pas se faire dans celui d'un autre animal, dans celui de l'homme, par exemple. Et quoique les cas cités par le D. Jenner (p. 280 & 281 de ce Journal, T. IX) semblent prouver que le corps humain ne le modifie pas au point de lui donner la propriété de préférer entièrement de la petite-vérole ordinaire, il est possible, comme je l'ai observé (page 370) que la modification n'ait lieu qu'après l'action générale du virus. Il feroit donc très-essentiel de faire l'expérience de la manière que je l'ai indiquée. (Ibidem.) Jusqu'à-ce qu'on en connoisse les résultats, & que de fréquens essais de ce genre ayant toujours uniformément bien réussi, il restera toujours quelque doute, au moins dans mon esprit, sur le parti qu'on peut tirer de la découverte; & en tout état de cause, de grandes & presque insurmontables difficultés dans son application. (O)}.\setcounter{page}{256} IVe. Un sujet qui a éprouvé la fièvre spécifique & l'inflammation locale produites par le virus variolique des vaches, est susceptible d'être affecté de la même manière par le même virus, mais il est à l'abri des effets de celui de la petite-vérole ordinaire. Le Dr. Pearson remarque que les gens de l'art ont beaucoup de peine à céder à cette conviction. Le seul témoignage additionnel qu'il cite en faveur de l'aphorisme en question est\setcounter{page}{257} celui de Mr. Woodman d'Aylesbury, qui dit que la petite-vérole des vaches ne s'exclut point elle-même, car les bergers l'ont quelquefois à plusieurs reprises. Les observations du Dr. à cet égard méritent d'être transcrites.
La preuve acquise de ce fait indique seulement, selon moi, d'une manière satisfaisante, que l'affection locale de la petite-vérole des vaches, peut reparoitre plus d'une fois chez le même sujet ; mais il n'est point prouvé à mes yeux que la fièvre particulière qui accompagne cette affection, attaque plus d'une fois la même personne ; on ne peut s'assurer de ce fait que par une suite d'observations, dans lesquelles on l'ait particulièrement pour objet. Ce sont aussi des observations futures qui pourront décider si dans les cas (s'il en existe) où un individu, après avoir eu la petite-vérole des vaches, prendroit la petite-vérole ordinaire, la première avoit produit simplement l'affection locale ou si elle avoit été accompagnée de la fièvre spécifique. Il paroit assez constaté, que le virus variolique ordinaire peut donner la petite-vérole par simple affection locale sans action générale sur le système entier de l'individu ; & dans ce cas, le sujet est encore susceptible de reprendre la petite-vérole \footnote{J'ai cependant vu des inoculés dans lesquels on n'apercevoit aucun signe d'action générale, mais une petite-vérole locale, complette & bien caractérisée non-seulement par l'inflammation fuccessivement augmentée de l'incifion jufqu'à produire un gros bouton, plein de pus, mais encore par l'apparition de cette efflorescence ou aréole érysipelateuse qui l'entoure communément au le. jour. Et dans ces cas-là, j'ai toujours considéré l'action du virus comme fuffifante pour garantir parfaitement de la petite-vérole naturelle; parce qu'il n'eft jamais arrivé qu'en pareil cas une feconde inoculation ait réussi, ni produit rien de semblable. Cette aréole ne se manifeste pas toujours. On ne la voit, par exemple presque jamais dans les inoculations qui produisent une petite-vérole très-abondante, ou confluente. Souvent même dans les petites-véroles bénignes elle manque. Mais lorfqu'elle se manifeste, même sans aucune apparence de fièvre ou de mal-aise général, j'affirme sans craindre d'être démenti par aucun fait, que le fuccès de l'inoculation est complet. Cette aréole seroit-elle donc toujours l'effet d'une action générale, d'un mouvement fébrile quelquefois trop fugitif pour être aperçu? Je fuis porté à le croire, tant parce qu'elle paraît, pour ainsi dire, tout d'un coup, au moment où l'éruption générale, s'il doit y en avoir une, est prête à commencer, que parce qu'à l'instant où elle se manifeste, l'action locale diminue, & ne tarde pas à cesser complètement. Sous ce point de vue, l'affertion de l'auteur subsisteroît dans toute son étendue, sinon jamais formellement. (O)}. Cependant, dans l'une &\setcounter{page}{258} l'autre, c'est-à-dire, dans l'affection locale, & dans l'affection générale, la matière purulente produite eft capable de propager l'infection, & de produire une petite-vérole, ou locale, ou complète. Il paraît probable, d'après ces confidérations,\setcounter{page}{259} que le virus des vaches peut produire de même des effets ou locaux ou généraux, non-seulement dans l'animal humain, mais chez les vaches elles-mêmes. J'avoue cependant que le cas indiqué, p. 51, dans l'ouvrage du Dr. Jenner, tendroit à détruire cette opinion\footnote{Voyez le T. IX. de ce Journal Sc. & Arts page 381.}.
V. Un individu qui a eu la petite vérole ordinaire, n'est point à l'abri de prendre celle des vaches.
Ici le Dr. P. ajoute peu d'informations à celles données par le Dr. Jenner. Mais la preuve positive fournie par ce dernier\footnote{Voyez le 7me cas T. IX. Sc. & Arts p. 274.} du fait en question, fait naître cette distinction assez curieuse ; savoir, que tandis que la petite-vérole des vaches rend les sujets qui l'ont éprouvée incapables de prendre la petite-vérole ordinaire, l'inverse de cette proposition n'est pas vraie, c'est-à-dire, que la petite-vérole ordinaire ne met point à l'abri de celle des vaches.
VI. "La petite-vérole des vaches ne se communique point par des miasmes, des effluves ou par un gaz ; ni par contact du pus sur la peau en très-petite quantité ; il ne paroît pas qu'on puisse l'inoculer autrement que par des divisions de la peau, coupures, déchirures, piqûres, &c."
VII. "L'affection locale de la petite-vérole des vaches produit occasionnellement est généralement\setcounter{page}{260} plus sévère & plus longue que celle qui résulte de cette même affection dans le cas de l'inoculation de la petite-vérole ordinaire. Mais dans aucun cas la fièvre spécifique de la petite-vérole des vaches n'entraîne de danger; & il n'existe aucune observation que cette maladie ait jamais été mortelle."
VIII°."On n'a observé aucune maladie ou affection particulière qu'on pût considérer comme étant la conséquence de la petite-vérole des vaches. Elle n'a jamais paru donner lieu à l'apparition d'une maladie à laquelle l'individu pût être considéré comme déjà disposé; & on n'a pas observé qu'elle produisît une prédisposition à aucune maladie particulière."
A ce sujet, le Dr. Pearfon s'exprime de la manière suivante.
"Quoique la pratique du Dr. Jenner, le témoignage uniforme de mes correspondants, & tout ce que nous connoissons jusqu'à présent de la nature de la maladie en question, aillent à l'appui de ces faits importants; il ne me semble pas cependant que l'ensemble des observations doive faire accorder aux conclusions rien de plus qu'une grande probabilité. Plusieurs centaines d'individus, ont été inoculés du virus variolique ordinaire, sous l'inspection de divers praticiens, sans qu'aucune autre maladie, ni aucune disposition à une autre maladie en ait été la conséquence. Il n'est pas douteux cependant,\setcounter{page}{261} que dans un certain nombre de cas l'inoculation n'ait eu cet effet\footnote{J'ai vu en effet quelques cas, mais en très-petit nombre, dans lesquels la petite-vérole inoculée a paru exciter une action scrofuleuse, & produire des ophthalmies ou des dépôts. Mais outre que ces suites font beaucoup plus fréquentes dans la petite-vérole naturelle, on voit aussi des enfans délicats, infirmes & sujets à toutes sortes de maux, particulièrement à des maladies cutanées ou nerveuses, que la petite-vérole, tant naturelle qu'inoculée, guérit radicalement. Il y a près de 50 ans qu'un célèbre médecin de Breslaw, frappé de cette considération, publia un traité de la petite-vérole ( Variolarum ratio exposita a Dr. Joh. Goth. de Hahn; Wratislavi; 1751 ) dans lequel il représente cette maladie comme si elle ne consistoit que dans un développement particulier du corps humain, analogue à la dentition, plus ou moins nécessaire à tous les hommes, pour la bonne organisation de leur peau, sujet comme la dentition, à être accompagné d'accidens plus ou moins graves, mais suivi pour l'ordinaire de conséquences très-avantageuses à la santé. Il avoit la même idée de la rougeole, sur laquelle il a aussi publié un traité à la suite & en confirmation de sa théorie de la petite-vérole. ( Morbilli variolarum vindices, a D. J. G. de Hahn; Wratisl. 1753.) (O)}.
"Nous sommes donc acheminés à croire qu'il faut attendre des observations plus nombreuses & plus exactes pour autoriser dans toute leur généralité les conclusions énoncées,ection{\S. VIII.
\setcounter{page}{262}
IX.e. Le virus de la petite-vérole des vaches peut produire l’affection locale particulière qui caractérise cette maladie, mais sans entraîner l’affection générale qui l’accompagne souvent; & dans ce cas, l’individu n’est point à l’abri de la petite-vérole ordinaire.
Tels sont les aphorismes ou les conséquences tirées de certaines classes de faits, que le Dr. Fearson a cherché à développer ou à prouver d’une manière plus satisfaisante que n’avoit pû le faire le Dr. Jenner, & il nous paroît y avoir réussi.
Dans la seconde partie de son ouvrage, l’auteur développe les avantages que la pratique de la médecine pourroit retirer de la substitution d’une de ces deux maladies à l’autre. Il n’est pas douteux en effet, que sous le rapport du danger réel, de la nature des symptômes, de leur durée; enfin, des conséquences sur la constitution de l’individu, la petite-vérole des vaches n’ait l’avantage le plus marqué sur celle qui fait quelquefois de si terribles ravages, & laisse souvent des traces bien fâcheuses.
Mais il faut ici des expériences très-multipliées, très-variées, faites sur une très-grande échelle. Il faut une longue fuite de faits, avant qu’on doive songer à abandonner une pratique aussi utile & aussi bien établie que l’est l’inoculation ordinaire. On peut présumer qu’il s’écoulera un long-temps avant qu’on puisse produire\setcounter{page}{263} en faveur de cette nouvelle petite-vérole un témoignage qui l'emporte sur le fait suivant affirmé par le Dr. Woodville. "Depuis le mois de janvier, dit-il, jusqu'au mois d'août inclusivement, sur plus de 1700 sujets inoculés à l'hôpital destiné à cette pratique, & en y comprenant les maladies externes, il n'en est mort que deux, & l'un & l'autre appartenoient à cette dernière classe."
Il n'en est pas moins vrai que les travaux du Dr. Jenner & ceux de notre auteur sur l'origine, la nature & les effets de la petite-vérole des vaches, font très-curieux, très-intéressants, & méritent toutes fortes d'encouragemens. Sans oser nous flatter, comme le Dr. Pearson que cette petite-vérole si bénigne puisse de long-temps remplacer sa cruelle & destructive sœur, on ne doit point douter que l'esprit de recherche porté sur cet objet ne produise des résultats utiles; & quelle qu'en puisse être l'issue finale on ne peut qu'applaudir au zèle & aux intentions des médecins philanthropes qui s'en occupent.
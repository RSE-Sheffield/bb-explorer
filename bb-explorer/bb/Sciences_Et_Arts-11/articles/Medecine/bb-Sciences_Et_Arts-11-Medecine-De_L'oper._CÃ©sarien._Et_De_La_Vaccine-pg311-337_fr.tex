\setcounter{page}{311}
\chapter{Médecine}
\section{REFLEXIONS ON THE PROPRIETY OF PERFORMING THE CÆSAREAN OPERATION, &c. Réflexions sur l'Opération Césarienne, avec quelques observations sur le Cancer & quelques expériences sur la prétendue origine de la Vaccine\footnote{Le nom de petite-vérole des vaches est incommode & difficile à manier dans un écrit, comparativement avec celui de la petite-vérole ordinaire, avec lequel une erreur de plume peut facilement la confondre. C'eft-ce qui nous engage à hafarder une autre nom. En latin on appelleroit cette maladie variola vaccina. Ce nom francifé feroit la variole vaccine. Pour abréger, nous l'appellerons à l'avenir la vaccine. Nous fommes d'autant mieux fondés à adopter cette dénomination, que comme aucun auteur Français n'avoit parlé avant nous de cette maladie, c'eft nous qui avions forgé le nom de petite-vérole des vaches, d'après l'Anglais Cow-Pox. Perfonne ne peut nous contefter le droit de revenir de notre traduction. (R)} ou petite-vérole des vaches; par W. SIMMONS, Membre du corps des Chirurgiens de Londres, & premier Chirurgien de l'Hôpital de Manchester. in-8°. pag. 97. 1798.}
QUELQUES-UNS de nos lecteurs trouveront peut-être que nous les entretenons trop\setcounter{page}{312} fouvent de la vaccine. Mais nous prions de confidérer que fi l'on peut fe flatter que l'inoculation de cette maladie préferve sûrement d'un fleau auffi deftructeur que la petite-vérole, fans être jamais fuivie d'aucun accident mortel, & fans courir le rifque de propager dans les environs, une épidémie dangereufe, c'eft ici une découverte importante, furtout fi, comme le préfume le Dr. Pearfon, cette maladie pouvoit préférver auffi de plufieurs autres poifons, tels que la rougeole, la fiévre rouge, la coqueluche, &c. Un des Journaliftes qui ont rendu compte de fon ouvrage, va jufqu'à demander, fi, ce virus inféré dans les morfures faites par un animal enragé, ne pourroit pas changer celui de la rage, & convertir ces morfures en des ulceres femblables à ceux qu'il produit lui-même, & par conféquent exempts de tout danger; fi le même changement ne pourroit pas avoir lieu en conféquence de fon application fur un ulcere cancereux, &c. Quelque foit le\setcounter{page}{313} peu de probabilité de toutes ces conjectures, qui font certainement trop hasardées ; elles méritent, cependant, ainsi que tout ce qui est relatif à l’origine, à la nature & aux effets d’une maladie aussi singuliere que la vaccine, un examen approfondi.
A tout ce que nous avons rapporté jusqu’à présent du fait principal de l’inoculation de cette maladie, dans le but de préserver de la petite-vérole, nous devons ajouter ( d’après l’*Analytical Review*) que depuis l’ouvrage du Dr. Pearson, dont nous avons donné l’extrait dans notre dernier Numéro, lui & le Dr. Woodville, ont inoculé de cette maniere, plus de 160 personnes de tout âge; que la pluspart ont été beaucoup moins malades qu’on ne l’est ordinairement de la petite-vérole inoculée, qu’on a ensuite inoculé la petite-vérole à 60 d’entr’elles, & qu’aucune d’elles ne l’a prise. Tous ces faits, avec plusieurs autres semblables, communiqués par divers correspondans de différentes parties de la Grande-Bretagne, où l’attention des gens de l’art commence à se porter très-généralement sur cet objet, seront confignés dans un ouvrage que le Dr. Woodville se propose de publier bientôt.
En attendant, nous allons transcrire de la Brochure que nous annonçons, les détails de quelques expériences qui contredisent formellement une partie des faits allégués par le Dr.\setcounter{page}{314} Jenner, relativement à l'origine de la vaccine; Il est bon d'instruire le procès à charge & à décharge sur tous les points. Disons un mot auparavant de l'objet principal de la brochure. Il est d'un genre très différent. Mais c'est la maniere actuelle de la plupart des auteurs Anglais. Ils accumulent dans leurs publications les sujets les plus disparates. Et l'on ne doit pas s'étonner que celui-ci parle de la vaccine à propos de l'opération Césarienne, puisque le Dr. Beddoës en avoit bien parlé à propos de hernies. (Voyez sa traduction du Traité de Gimbernat sur les hernies fémorales, pag. 63.) Nos lecteurs savent sans doute que lorsqu'il y a impossibilité physique & bien démontrée, à ce qu'une femme dans les douleurs de l'enfantement puisse accoucher par les moyens ordinaires, on a proposé de sortir son enfant par une incision faite au ventre. C'est ce qu'on appelle, l'opération Césarienne, du mot latin *cæsus*, qui veut dire *coupé*. Un passage mal entendu, de Pline l'ancien \footnote{Après avoir parlé de ceux qui naissent par les pieds, & avoir remarqué que cette maniere de venir au monde est presque toujours de mauvais augure pour le bonheur futur de l'enfant qui se présente ainsi, Pline ajoute. *Auspicatius enecta parente gignuntur*, sicut Scipio africanus prior natus, primus que Cæsarum, a cæso matris utero dictus, qua de causa & cæsones appellati. Simili modo natus & Manlius, qui Carthagimem cum exercitu intravit. (Plin. Hist. natur. Lib. VII. cap. 9.) Il est évident que les mots primus Cæsarum veulent dire dans ce passage le premier de ceux qui ont porté le nom de César. Ils ne peuvent pas s'appliquer à Jules César, dont le pere s'appeloit L. César, & dont la mere, au rapport de Suétone vivoit encore, lors de son expédition dans la Grande-Bretagne.} a fait croire que la mere\setcounter{page}{315} de Jules César avoit subi cette opération, & que le surnom de cet homme célèbre dérivoit delà. L'opération dont parle Pline ne paroît avoir été faite qu'après la mort de la mere pour sauver l'enfant. Les modernes, plus hardis, l'ont quelquefois entreprise du vivant même de la mere. Mais notre auteur espere, "qu'à l'avenir il n'en fera plus question ; qu'on en bannira jusqu'à la moindre trace de tous les livres qui traitent de son art. Elle ne l'a que trop déshonoré jusqu'à présent," ajoute-t-il. "On ne peut jamais être excusable de l'avoir faite sur une femme vivante."
Quelque sévére que soit ce jugement, nous nous abstiendrons d'examiner la question. Les motifs de l'auteur pour la décider si péremptoirement nous paroissent un peu suspects. Un de ses compétiteurs dans l'art des accouchemens venoit de faire l'opération césarienne à Manchester après une consultation avec quatre Officiers de santé, qui convaincus qu'il n'y avoit aucun autre moyen de sauver la vie de l'enfant,\setcounter{page}{316} & que même la destruction de cet enfant ne sauveroit point fon infortunée mere, l'avoient unanimément confeillée. L'opération avoit mal réuffi. L'enfant s'étoit trouvé mort, & la mere avoit péri quelques heures après. C'eft à cette époque que Mr. S. publie fa brochure. Il y fait un grand étalage d'érudition; cite une multitude d'anciens auteurs qui ont parlé de l'opération, tronque leurs expreffions, ne préfente que celles qui lui font favorables, attefte que jamais elle n'a réuffi dans la Grande-Bretagne, & termine fon amère critique par la phrafe que nous venons de citer. Son adverfaire lui répond par une brochure de 240 pages, intitulée: A Defence of the cesarian opération, &c. Justification de l'opération césarienne, avec des observations nouvelles, & sept gravures, par John Hull, Dr. en médecine, membre du Corps des chirurgiens de Londres, secrétaire de la Société littéraire & philosophique de Manchef ter; adressée à Mr. W. Simmons, auteur des Réflexions, &c. in-8°, 1799. Ces deux Traités ne font donc que des ouvrages polémiques; il ne nous convient point d'entrer dans la difpute, & d'autant moins que depuis quelques années il s'est élevé à Paris & fur le même objet, une difpute femblable, qu'on foutient encore de part & d'autre avec beaucoup de chaleur. Nous dirons seulement que Mr. S. paroît s'être exagéré les objestions qu'on peut\setcounter{page}{317} faire contre l'opération, sans doute elle est délicate, dangereuse, souvent mortelle. Mais elle a été faite plusieurs fois avec succès. Il n'est pas besoin de recourir à des autorités suspectes ou bien anciennes pour justifier cette affertion. Nous lisons dans le Moniteur, n°. 308, an 7, que tout récemment le Président de la Municipalité de la Haye a donné publiquement deux médailles au cit. B. Schuring, chirurgien, pour le service qu'il a rendu à l'humanité en faisant cette opération le 16 floréal dernier à la cit. B. De Voogt. La mere & l'enfant se portent bien.
Ceux de nos lecteurs qui seraient curieux d'approfondir ce sujet, verront dans la Bibliotheque Germanique des cit. Brewer & De La Roche, un excellent Extrait des diverses observations de ce genre, toutes modernes & authentiques que le célèbre Richter a consignées pour & contre, dans sa Bibliotheque de chirurgie\footnote{Voyez la Bibl. German. T. I. p. 121. L'opération paraît avoir eu plus de succès sur le Continent qu'en Angleterre. On y cite cependant trois cas dans lesquels elle a réussi. A la vérité c'étaient des cas de conceptions extra-utérines, dans lesquels par conséquent on ne fut pas dans la nécessité de couper la matrice, puisque l'enfant était placé en-dehors de cet organe, dans les trompes de Fallope, ou dans les ovaires. J'ai vu un cas de cette espèce, dans lesquel il étoit à craindre que l'enfant par ses mouvemens ne rompît son enveloppe, & ne tombât tout-à-coup dans le ventre de sa mere, ce qui l'auroit tuée subitement. Les Rédacteurs de la Biblioth. German. en ont dit un mot, p. 150; mais comme les détails en font assez curieux, & font le plus grand honneur à la sagacité de mon illustre collègue le Professeur Jurine, je suis bien aise de les consigner dans ce Journal. Voici le fait. Je fus appelé le 11e. mars 1796, à 9 heures du matin à voir une jeune femme qui se croyoit enceinte de 4 mois, & qui, à l'occasion d'un mouvement qu'elle avoit fait dans son lit, avoit été tout-à-coup saisie à 8 heures du matin de douleurs atroces dans le bas-ventre, avec de l'oppression & des vomissemens continuels. Il lui étoit difficile de déterminer exactement le siege de ces douleurs. Elle les éprouvoit dans tout le bas-ventre; mais en la palpant, on sentoit une tumeur flottante, assez dure & douloureuse au toucher, placée obliquement à l'aine gauche. Elle se plaignoit en outre d'un grand besoin d'uriner qu'il lui étoit impossible de satisfaire, & pour lequel il fallut la sonder. Son pouls étoit un peu agité, mais d'ailleurs assez bon. On me raconta qu'elle avoit eu depuis deux mois & demi à Chambéry quatre accès de douleurs semblables, toujours subites & provoquées par un mouvement brusque. La violence de ces accès avoit toujours été en augmentant. Le premier n'a-voit duré que 24 heures. Le second & le troisième avoient été plus longs. Le dernier avoit été terrible.
La malade étoit restée long-temps sans pouls & avec les extrémités froides. Les douleurs n'avoient cessé complétement qu'au bout de quinze jours. L'intervalle des quatre premiers avoit été de 13 jours. Le dernier étoit survenu cinq semaines avant celui pour lequel j'étois appelé. Dans celui-ci, la violence des douleurs étoit telle que la malade remplissoit la maison de ses cris, & que ce ne fut qu'avec beaucoup de peine que les saignées, les lavemens & les anodins lui donnèrent enfin quelque soulagement. La manière subite dont ces accidens étoient survenus & la nature des symptômes me firent d'abord soupçonner une retroversion de la matrice. Je fis appeler le Prof. Jurine pour m'en assurer. D'après l'examen qu'il fit de la malade, son avis fut que ce n'étoit point cela, mais une conception extra-utérine, dans laquelle les douleurs étoient produites par les mouvemens convulsifs de l'enfant, mouvemens qui aboutiroient probablement enfin à rompre son enveloppe. Il avoit vu, me dit-il, deux cas semblables, dans lesquels les douleurs avoient, comme ici, une forte de périodicité, & venoient tout aussi subitement. Ces deux cas avoient été mortels, mais dans une période beaucoup plus avancée de la grossesse, & sans qu'il y eût aucun doute sur l'existence de l'enfant. Pour nous en assurer dans notre malade, nous introduisîmes le lendemain une sonde élastique dans l'orifice de la matrice, qui étoit assez ouvert pour être susceptible de cet examen. La sonde pénétra facilement dans la cavité qui se trouva assez dilatée mais vide. Maupoir qui fut aussi appelé à voir la malade la sonda de même, & obtint un résultat semblable. Si donc il y avoit un enfant, il n'étoit pas dans la matrice. Cependant les signes de grossesse n'étoient pas équivoques. Bientôt il survint une autre tumeur à l'aine droite. Les mouvemens de l'enfant se firent appercevoir d'une maniere distincte dans cette tumeur, & on les sentoit sous la main en la palpant, comme si cet enfant n'étoit recouvert que par la peau. Toutes les fois que ces mouvemens étoient brusques & violens, la malade souffroit cruellement. Mais ces douleurs n'avoient plus de périodicité marquée. Elles étoient beaucoup plus fréquentes qu'au commencement, & elles devinrent enfin si intolérables que la malade craignant tous les jours d'y succomber, nous sollicitoit elle-même avec instances de lui faire l'opération. C'étoit bien notre avis; mais nous ne crûmes pas devoir l'entreprendre sans consulter trois autres Officiers de santé qui furent appelés pour en conférer avec nous. Heureusement la malade ne souffroit pas beaucoup dans ce moment. Les médecins & le chirurgien qui la voyoient pour la premiere fois ne purent sentir les mouvemens de l'enfant ni se convaincre par eux-mêmes de son existence. Dans un cas aussi grave, on ne peut pas croire sur parole. On résolut donc d'attendre un autre accès de douleurs pour prendre un parti. Dans l'intervalle, l'enfant mourut. Les mouvemens cessèrent tout-à-coup. Les douleurs changent de nature. Il survint de la fièvre avec une perte assez forte, & pendant quelques semaines, nous sûmes fort incertains sur ce qui en arriveroit. Enfin, tous les symptômes de maladie s'apaisèrent peu-à-peu sans aucune crise, & la malade est aujourd'hui parfaitement bien. Mais quoiqu'il y ait déjà trois ans & demi écoulés depuis la mort de l'enfant, elle le porte encore du côté droit, où on le sent très-distinctement, couché obliquement le long de l'aine. Il est devenu plus petit. La tumeur du côté gauche a totalement disparu depuis long-temps. La région de la matrice est dans un état parfaitement naturel. Et tout annonce que ce sera ici un exemple de ces conceptions extra-utérines, dans lesquelles l'enfant se dessèche sans se corrompre, sans se faire jour au-dehors, & sans produire aucun accident. Mais s'il avoit vécu, n'auroit-il pas fallu à tout prix faire l'opération, & Mr. dira-t-il qu'on n'auroit pas été excusable d'entreprendre? (O)}. Voici comment le Rédacteur de ces\setcounter{page}{318} Extrait le terminé. "Nous croyons pouvoir conclure de l'ensemble des faits que nous venons de rapporter, qu'il y a des cas où l'o \setcounter{page}{319} pération césarienne est absolument nécessaire; que malgré le grand nombre d'exemples de son peu de succès, elle n'est point mortelle\setcounter{page}{320} par elle-même, & que si les gens de l'art donnent enfin à cet objet important toute l'attention qu'il mérite, ils parviendront à la rendre beaucoup moins funeste qu'elle ne l'a été jusqu'à ce jour."
A ses observations sur l'opération césarienne, qui remplissent les deux tiers de son ouvrage, Mr. S. en ajoute quelques-unes sur l'emploi de l'arsenic dans les cancers. Il y a long-temps\setcounter{page}{321} que l'application extérieure de ce remede a été recommandée dans les ulceres cancereux. On fait que la chaux-vive, ainsi que plusieurs sels & oxides métalliques ont la propriété de priver de toute vitalité les parties de la peau & les chairs sur lesquelles on les applique, de manière à les convertir très-promptement en une croûte sèche, autour de laquelle la nature établit ensuite un cercle purulent qui peu-à-peu s'étend\setcounter{page}{322} fous cette croûte, la mine & la fait tomber. La croûte, elle-même s'appelle une escarre, d'un mot grec qui veut dire tout-à-la-fois, un foyer & une croûte. Les remèdes qui la produisent s'appellent des escarotiques. Le feu, produisant aussi le même effet, surtout quand il est appliqué par le moyen d'un fer bien rouge, ces remèdes s'appellent aussi des caustiques, autre mot grec qui veut dire brûlans. On emploie plutôt cette dernière dénomination que la première, pour désigner ces applications, quand elles se font sur la peau dans son état d'intégrité, & la première plutôt que la dernière, quand on applique le remède sur un ulcère. Or, l'effet de ces applications est, pour l'ordinaire, non-seulement de convertir un ulcère en une croûte sèche que la nature sépare ensuite par une bonne suppuration, mais encore de détruire le virus même qui produit & entretient l'ulcère. Cependant, quelque efficace que soient le feu & les autres escarotiques pour détruire les ulcères malins d'un autre genre; l'ulcère cancéreux avoit jusqu'à présent bravé leur action. Or, ce qu'ils n'ont pu produire sur cette horrible maladie, on l'a depuis long-temps espéré de l'arsénic, qui de tous les escarotiques paroît être le plus actif, & qui a de plus cet avantage par dessus les autres, que son action ne s'étend pas latéralement, mais se borne à l'endroit qu'il touche. Il y a plus de 400 ans\setcounter{page}{323} que Guy de Chauliac, (sav ant médecin du 14 me siècle, architecte du Pape Clément VI à Avignon, auteur de l'excellente description de la peste de 1347, que Bocace a enrichie de tous les charmes de sa diction) l'avoit déjà recommandé dans le chapitre de sa grande chirurgie, qui traite des médicamens corrosifs; & la base du remede connu sous le nom de Bernard, ou du frère Cosme, est l'arsénic. (Voyez le Journal de médecine, chirurgie & pharmacie, janvier & juin 1782, Tom. 57.) Malheureusement, ce remède a jusqu'à présent été principalement employé par des empiriques, entre les mains desquels il a eu quelque succès trop souvent balancés par des effets funestes. (Bibl. Germ. T. 2. p. 243.) Pour en revenir à notre auteur, il paroît que non loin du lieu de son établissement, à Witworth en Lancashire, il y avoit aussi des guérisseurs de cette espèce qui en faisoient usage & qui avoient la réputation d'avoir fait plusieurs cures. L'indignation qu'inspire à Mr. S., cette réputation, suivant lui usurpée, allume sa bile. Il cite quelques cas malheureux, où l'arsénic appliqué sur le cancer n'a produit aucun soulagement, d'autres où il a aggravé le mal; il atténue ses succès par la difficulté de distinguer un ulcère cancéreux d'un ulcère qui ne l'est pas, & il décide enfin, affirmativement que l'arsénic, ne peut jamais être employé utilement à l'extérieur\setcounter{page}{324} dans de véritables cas de cancer. Cette affertion ne paraît pas fondée sur des faits bien concluants. On vient de publier dans la Bibliothèque Germanique un grand nombre d'observations authentiques des bons effets de ce remede, tirées pour la plupart du Journal de Loder, savant Professeur de Jena, & rédigées par Theden, premier chirurgien des armées du Roi de Prusse à Berlin, d'après lesquelles il confie que l'arsenic appliqué extérieurement, mais en très-petites doses & mélangé d'autres ingrédients propres à en atténuer l'activité, a souvent opéré la guérison d'ulcères cancéreux, particulièrement au visage. Il ne paraît pas qu'il ait aussi bien réussi dans ceux du sein. Peut-être ceux d'après lesquels notre auteur rejette son application, étaient-ils de cette derniere espèce. Peut-être, aussi avait-on employé imprudemment le remede en trop haute dose, & sans une suffisante quantité de mélange. On lit dans Fernell, fameux médecin Français du 16me. siècle, une observation de ce genre, où l'application extérieure de ce remede sur un sein cancéreux fut suivie des plus horribles accidents, & promptement mortelle. "Quocirca ," ajoute-t-il , "ejus modi remedia prorsus exterminanda sunt."
Mr. S. ne paraît cependant pas éloigné de croire aux bons effets de l'arsenic employé intérieurement dans cette maladie. Il cite une\setcounter{page}{325} observation unique à la vérité, mais frappante de l'avantage qu'on avoit retiré une malade à laquelle il donna pendant cinq à fix semaines la solution minérale du Dr. Fowler\footnote{La solution minérale du Dr. Fowler est composée de 64 grains (poids Anglais) d'oxide blanc d'arsénic mêlé avec une égale quantité de potasse bien pure. On fait dissoudre & combiner ensemble ces deux sels par une légère ébullition dans une demi-pinte d'eau distillée (237 grammes.) On ajoute ensuite une demi-once (15 grammes) d'esprit de lavande composé, & enfin autant d'eau distillée qu'il en faut pour que la totalité du mélange fasse une pinte (474 grammes); ce qui, vu la différence des poids Anglais fait environ une partie d'arsénic sur 115 de la solution, ou plutôt une partie d'arséniate de potasse avec surabondance d'alkali (car c'est ce qui doit résulter du mélange des deux sels) sur 58 de la solution. On en donne trois fois par jour de deux à quinze gouttes. Ce remède est fort employé depuis quelques années en Angleterre dans les cancers, dans les fièvres intermittentes, dans les maux de tête périodiques, dans les maladies cutanées rebelles, & tout récemment il vient d'être recommandé fortement par le Dr. Ferriar de Bath, comme un remède souverain dans la coqueluche. On ne doit pas le considérer comme de l'arsénic pur. Les propriétés des composés ne font point les mêmes que celles de leurs élémens; & il n'y a probablement pas plus d'analogie entre l'arsénic & l'arséniate en question, qu'il n'y en a entre l'acide sulfurique & les sulfates de soude ou de potasse. Mais est-on bien assuré que la combinaison des deux sels se fasse assez intimément par une légère ébullition? Ne vaudroit-il pas mieux employer dans ce but l'arséniate préparé à la maniere de Guiton de Morveau, c'est-à-dire en mêlant parties égales de nitre & d'arsenic dans une retorte qu'on expose d'abord à une douce chaleur, graduellement augmentée jusqu'à rougir le fonds de la retorte? De cette maniere la base alkaline du nitre se combine intimément dans la retorte avec l'acide arsénical & forme un sel neutre susceptible de solution & d'une belle cristallisation en aiguilles prismatiques. J'en ai vu de très-beau chez Colladon. (O)}, à la\setcounter{page}{326} dose de 12 gouttes trois fois par jour. Une plus forte dose, ajoute-t-il aggravoit les symptômes. On se demande ici comment & sous quel point de vue l'arsenic, ce poison terrible, a-t-il pu être employé intérieurement comme remede? Il n'est pas aisé de répondre à cette question.
La manière d'agir des remedes & surtout des remedes internes, nous est en général assez peu connue. Peut-être a-t-on soupçonné que si l'on parvenoit à tempérer l'excessive énergie de l'arsenic, cette énergie même feroit un moyen efficace de rendre aux vaisseaux le ton qui leur manque en certains cas. C'est principalement en effet, dans les cas d'atonie qu'on l'a furtout employé, dans les mêmes circonstances dans lesquelles on employe le kina. C'est ainsi que depuis long-temps on en a fait usage dans les fièvres intermittentes\footnote{Je voyois un malade atteint d'une fièvre quarte affez rebelle. Je le traitois depuis long-temps par le kina et les amers, il s'impatienta. Un de ses amis l'engagea à prendre quelques gouttes d'une eau limpide qu'il lui vanta comme un excellent remède contre la fièvre, et qui parut au malade n'avoir que peu de goût. Il n'en prit qu'une fois; mais la fièvre ne revint plus. C'était probablement le remède fameux qui porte en Irlande le nom de gouttes insipides pour les fièvres d'accès (tasteless ague drops) on fait aujourd'hui que la base de ce remède est l'arsenic. (O)}. Et comme\setcounter{page}{327} le kina est un des meilleurs remèdes connus pour les ulcères de mauvaise nature, on a pu croire qu'un remède beaucoup plus actif encore, réussirait dans ceux dans lesquels le kina a jusqu'à présent échoué. C'est aux gens de l'art à apprécier ce raisonnement qui n'entraînera jamais un médecin sage et prudent, tant qu'il aura d'autres ressources connues, comme on en a presque toujours pour les fièvres, mais qui dans une maladie aussi cruelle et aussi désespérée que le cancer ulcéré, peut facilement séduire le pauvre malade\footnote{J'ai eu connaissance d'un seul cas de cancer au visage, dans lequel un médecin instruit et prudent, employa l'arsenic, tant à l'extérieur qu'à l'intérieur, avec le consentement du malade et de ses parents. Il eut d'abord une grande apparence de succès. Mais cela ne se soutint pas. Le mal revint ensuite avec plus de violence que jamais, et le malade mourut, (O)}.
Venons maintenant aux expériences de notre auteur sur la vaccine. On se rappelle que le\setcounter{page}{328} Dr. Jenner attribuoit l'origine de cette maladie au javart des chevaux; qu'il cite une multitude de faits d'après lesquels il paraissoit évident ou du moins très-probable qu'elle ne se déclare dans une ferme que lorsque l'on fait traire les vaches par des domestiques employés en même temps à panser les ulcères des chevaux atteints du javart, à moins que quelque vache nouvellement introduite dans le troupeau ne la communique aux autres par l'intervention des laitières; que les chevaux atteints du javart communiquent quelquefois directement aux hommes qui les pansent, une maladie éruptive qui a jusques à un certain point la même propriété, mais pas aussi complettement que la vaccine proprement dite, de préserver pour toujours de la petite-vérole; que les boutons qui la constituent renferment un fluide susceptible d'être inoculé; que la maladie qui résulte de cette inoculation présente exactement les mêmes apparences que celle qu'on inocule d'après les vaches; que même les férolités épanchées par quelque affection érysipélateuse sous la peau du cheval, dans tout autre endroit que le talon, & par des causes très-différentes du javart, produisent aussi le même effet que le pus du javart même, &c.— Les conséquences singulieres que le Dr. Jenner tiroit de tous ces faits ont été constatées. Les faits eux-mêmes ont à quelques égards été révoqués\setcounter{page}{329} en doute. "Il paraît," dit le Dr. Pearson, "que la vaccine se manifeste spontanément dans les troupeaux, au printemps depuis le mois de février jusqu'au mois de mai, & quelquefois aussi en automne & en hiver. La plupart des médecins & chirurgiens avec lesquels j'ai entretenu des correspondances sur cet objet la regardent comme indépendante du javart. Mr. Woodman, chirurgien d'Aylesbury, est le seul qui partage sur ce point l'opinion du Dr. Jenner, opinion qui est au reste très-généralement répandue dans le pays, & même parmi les fermiers près de Londres. Mais j'ai trouvé que la vaccine s'est manifestée dans plusieurs fermes, quoiqu'aucune nouvelle bête n'eût été introduite dans le troupeau, quoiqu'aucune des personnes employées à les traire n'approchât des chevaux, quoiqu'il n'y eût aucun cheval atteint du javart, & même quoiqu'on ne tînt point de chevaux dans la ferme."
L'origine de cette maladie est donc encore un problème. Pour le résoudre, Mr. résolut de faire sur cet objet les expériences directes que le Dr. J. regrettait de n'avoir pu faire. Il s'adressa à un chirurgien vétérinaire qui lui fournit des chevaux atteints du javart, de manière à pouvoir prendre le fluide contenu dans les ulcères érysipélateux qui constituent cette maladie, au moment de sa formation, ou à\setcounter{page}{330} toute autre époque qu'il jugeroit plus convenable. Il se procura ensuite un troupeau de 30 vaches entièrement à sa disposition. Mais il se borna aux quatre expériences suivantes, soit parce que leur résultat négatif ne l'encourageoit pas à en faire d'autres, soit parce qu'il ne put se procurer du vrai pus de vaccine. Voici comment il raconte le résultat de ces quatre expériences.
\subsection{Premiere Expérience.
"Le 29 octobre, je pris un peu du fluide contenu dans les ulcères d'un cheval qui avoit les talons fort enflammés. C'étoit quelques heures seulement après que la maladie eut commencé à se manifester, & avant qu'on eût fait aucun pansement. Ce fluide étoit très-liquide, clair & d'une couleur jaunâtre. Je m'en servis pour inoculer trois enfants bien portants & de bonne mine, l'un âgé de trois ans, un autre de six mois, & un autre de trois mois. Je leur fis à chacun quatre piqûres au bras gauche, à une petite distance l'une de l'autre."
"Le 1er novembre, les bords des piqûres étoient un peu rouges, mais pas plus qu'ils ne l'auroient été, si elles avoient été faites avec une lancette propre & sans virus. On prescrivit un régime rafraîchissant, mais sans remede."
"Le 5, toutes les piqûres étoient guéries. La peau avoit recouvré sa couleur & son apparence naturelle."
\setcounter{page}{331}
"Le même jour, on inocula la petite-vérole à ces trois enfants avec du pus très-délayé dans de l'eau chaude, & par une seule piqûre au bras gauche, au centre des quatre précédentes."
"Le 8, il y avoit des signes très-évidents d'infection."
"La maladie fit son cours ordinaire, & fut très-bénigne."
\subsection{Seconde Exp.}
"Trois enfants furent inoculés le 16 novembre, de la même maniere, mais avec le fluide ichoreux & brunâtre qui se trouvoit en grande abondance dans le talon très-enflammé d'un cheval, atteint du javart depuis trente-six heures. La coloration de ce pus datait tout au plus de vingt-quatre heures; & comme le cheval était réservé pour mon usage, on ne lui avoit fait aucun pansement."
"Il ne résulta de cette inoculation aucune inflammation ni aucun symptôme de maladie."
\subsection{Troisième Exp.}
"Avec le même fluide j'inoculai trois vaches, en faisant une piqûre à chaque pis; & comme l'épiderme se trouva plus épais, vu la saison, qu'il ne l'auroit été trois ou quatre mois plutôt, je fis l'insertion du virus avec beaucoup de soin, afin que cette circonstance ne fit pas manquer l'opération."
"Les traces des piqûres demeurèrent visibles pendant plusieurs jours, au bout desquels elles disparurent sans avoir excité le moindre symptôme de maladie."
\setcounter{page}{332}
"Il faut observer que la maladie absolument inconnue dans le Cheshire & dans le Lancashire (quoique dans le premier de ces deux Comtés il y ait de grandes laiteries, où l'on emploie entièrement les hommes à panser les chevaux & à traire les vaches), en sorte qu'on ne peut pas supposer que si les vaches inoculées avec le virus du javart n'ont pas pris la vaccine, c'est parce qu'elles l'avoient eue auparavant.
Quatrième Exp. "J'inoculai en même-temps la petite-vérole à deux autres vaches du même troupeau, par une piqûre à chaque pis. Mais cette inoculation n'eut aucun effet faisible."
Telles sont les expériences rapportées par Mr. Elles paroissent en contradiction directe avec celles du Dr. Jenner. Peut-être cependant leur opposition est-elle plus apparente que réelle. Il ne seroit pas impossible de concilier les faits par des hypothèses. Mais ces hypothèses seroient elles-mêmes fort singulières & fort peu probables. On pourroit supposer, par exemple, que le javart est susceptible de se communiquer par contagion, mais non par inoculation, à moins que d'avoir subi une modification particulière dans le corps d'un homme. On citeroit, à l'appui de cette hypothèse, l'expérience infructueuse qu'a faite le Dr. Jenner pour inoculer directement le javart d'un cheval à une vache, expérience dont il attribue le non-succès\setcounter{page}{333} à ce que le pus qui avoit servi à l'inoculation étoit très-épais. On feroit remarquer qu'il n'a réussi à l'inoculer que d'après un homme. On pourroit objecter encore à la première expérience de Mr. S., que la petite-vérole inoculée dès le septieme jour après le javart, a pû arrêter le cours de cette dernière maladie, & rendre ensuite l'infection impossible. On pourroit dire que, comme l'insinue le Dr. Jenner, les vaches ne sont gueres susceptibles de prendre la vaccine que dans certaines saisons de l'année, & particulièrement au printemps. On pourroit demander encore, avec un journaliste anglais, s'il ne feroit pas possible que les ulcères du javart fussent modifiés par ceux de la vaccine, tellement que lorsqu'après avoir trait une vache atteinte de cette dernière maladie, un domestique auroit pansé un cheval atteint du javart, les ulcères de celui-ci se changeassent en d'autres semblables à ceux de la vaccine, & pussent alors communiquer non le javart, mais la vaccine elle-même, ou une maladie analogue. On pourroit enfin soutenir avec plus de vraisemblance, qu'un petit nombre d'expériences négatives ne peut pas balancer un grand nombre d'expériences positives; qu'il est imprudent de tirer des conséquences précipitées de faits isolés, & peu libéral de s'en prévaloir pour traiter de fabuleux ou mensongers des faits contraires.\setcounter{page}{334} Mais cette logique n'est pas celle de Mr. Voici ses conclusions :
"Si ceux qui ont eu la vaccine font pour toujours à l'abri de la petite-vérole, on peut raisonnablement en conclure que le même poison produit l'une & l'autre maladie. Or c'est un fait bien reconnu que la petite-vérole peut être communiquée tant par le fluide érysipélateux qu'on obtient par l'inoculation, à l'endroit de l'incision & avant la fièvre éruptive, que par le fluide plus épais qui se trouve dans les boutons à l'époque de leur maturité complete, ou même par les croûtes sèches qui leur succedent sur la fin de la maladie. Et comme on suppose que le fluide produit par le javart, auquel on attribue la vaccine, n'est contagieux que dans la premiere époque de la maladie, lorsque ce fluide est encore clair & transparent, cette différence prouve qu'il n'y a pas d'identité entre les deux poisons, & ne permet plus de supposer aucune analogie entre le javart, la vaccine, & la petite-vérole\footnote{Ce raisonnement ne sauroit être bon qu'autant qu'on se croiroit toujours fondé à regarder comme parfaitement identiques deux agens qui produisent le même effet, & à nier qu'il puisse y avoir aucune analogie entr'eux, lorsque leur identité, n'est pas complete. Or cette supposition est très-éloignée de la vérité. On voit à chaque instant des effets semblables tenir à des causes différentes. La soude & la potasse neutralisent les acides, & par-là ces deux fels ont entr'eux une analogie très-marquée. Mais il ne s'ensuit pas qu'ils ne fassent qu'un seul & même sel. La chaleur, la présure & les acides coagulent également un blanc d'œuf. Il y a donc quelque analogie entre ces trois agens. Mais personne ne sauroit tester leur différence. (O)}.\setcounter{page}{335} " Douze piqûres faites aux trois vaches inoculées avec le virus du javart, & vingt-quatre aux six enfants inoculés de même, n'ont produit aucun effet, tandis qu'une seule faite sur chacun de ces enfants avec le virus de la petite-vérole, même très-délayé, a complètement réussi. Donc, la probabilité qui résulte de ces expériences en faveur de la possibilité de communiquer la petite-vérole, en l'inoculant, est comparativement à celle de communiquer le javart par le même moyen, comme 24 à 1, relativement aux enfants, & comme 12 à 1 relativement aux vaches. Et puisqu'on a fait inutilement huit piqûres aux deux vaches inoculées avec le virus variolique, la possibilité de communiquer la petite-vérole aux enfants par l'inoculation est comparativement à celle de la communiquer aux vaches par le même moyen, comme 8 à 1. L'absurdité de ce calcul faute aux yeux\footnote{C'est à-peu-près comme si quelqu'un me disoit pendant que vous vous reposiez dans votre cabinet, j'ai fait 500 pas à la promenade. Donc je marche. 500 fois plus vite que vous. Si au lieu de faire une piqûre avec le pus variolique, l'auteur en avoit fait six, & une seule à chacun des six enfans inoculés avec le javart, se croiroit-il fondé à dire que la probabilité de communiquer la petite-vérole ou le javart par inoculation est égale? Au reste, ceci pourroit donner lieu à une question sur laquelle les avis ont été partagés. Le succès de l'inoculation dépend-il du nombre des piqûres. Quant-à-moi, je ne le crois point, parce que je n'ai apperçu aucune différence entre les cas où je n'en ai fait qu'une & ceux où j'en ai fait quatre. Je n'oserois pas affirmer cependant que cela soit parfaitement indifférent; & le plus prudent est d'en faire au moins une à chaque bras. Il arrive souvent que l'une manque, pendant que l'autre réussit. Mais une seule qui réussit m'a constamment paru produire un effet général tout aussi considérable & aussi marqué que lorsqu'il y en a quatre à-la-fois qui prennent bien. (O)}.\setcounter{page}{336} Il n'en est pas moins vrai que ces expériences font défavorables à l'opinion qui attribue la vaccine au javart. Mais elles n'attaquent point le fait principal de la possibilité de préserver le corps humain de la petite-vérole par l'inoculation de la vaccine, sans courir le risque d'une éruption confluente, ou d'aucun accident mortel, ni celui de répandre la contagion. C'est sur ce fait, qui n'a point encore été contesté, que nous desirerions particulièrement fixer l'attention de nos lecteurs. Il est d'autant plus important que déjà l'on a commencé sur le Continent à en tirer parti. C'est ce qu'on verra par la lettre \setcounter{page}{337}suivante que nous venons de recevoir d'un de nos compatriotes, & que nous empressons de rendre publique. On ne sauroit trop promptement répandre une découverte utile.
